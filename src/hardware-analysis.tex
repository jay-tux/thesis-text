\chapter{The Cold-Start Problem in Hardware}\label{ch:hw-analysis}

\section{Initial profiling}\label{sec:initial-profiling}
Firstly, we need to quantify and detect whether the cold-start problem exists in GPGPU hardware.
To this end, we've analyzed some workloads and benchmarks using the NVIDIA Nsight Compute tool.
This tool allows us to control the caches during the execution of an application.
Additionally, we can gather detailed metrics, like IPC, for each kernel in the workload.

Each workload was profiled twice: once like it would run normally, and once with the caches flushed between kernel invocations.
Nsight Compute supports this option by using the \verb|--cache-control=all| argument.
To keep the runtime in check, we limited the execution to 20\thinspace000 kernels per workload.
These experiments were run on an NVIDIA GeForce RTX 3080\cite{nvidia-wp}.
This GPU has 68 SM cores, 6 MB of L2 cache.

After profiling, we attempted to match each profiled non-flushed kernel with its flushed counterpart.
This proved easier said than done, as the kernel IDs were not consistent between the two profiles.

\begin{figure}
    \centering
    \resizebox{0.45\textwidth}{!}{%% Creator: Matplotlib, PGF backend
%%
%% To include the figure in your LaTeX document, write
%%   \input{<filename>.pgf}
%%
%% Make sure the required packages are loaded in your preamble
%%   \usepackage{pgf}
%%
%% Also ensure that all the required font packages are loaded; for instance,
%% the lmodern package is sometimes necessary when using math font.
%%   \usepackage{lmodern}
%%
%% Figures using additional raster images can only be included by \input if
%% they are in the same directory as the main LaTeX file. For loading figures
%% from other directories you can use the `import` package
%%   \usepackage{import}
%%
%% and then include the figures with
%%   \import{<path to file>}{<filename>.pgf}
%%
%% Matplotlib used the following preamble
%%   \def\mathdefault#1{#1}
%%   \everymath=\expandafter{\the\everymath\displaystyle}
%%   
%%   \usepackage{fontspec}
%%   \setmainfont{DejaVuSerif.ttf}[Path=\detokenize{/usr/lib/python3.11/site-packages/matplotlib/mpl-data/fonts/ttf/}]
%%   \setsansfont{DejaVuSans.ttf}[Path=\detokenize{/usr/lib/python3.11/site-packages/matplotlib/mpl-data/fonts/ttf/}]
%%   \setmonofont{DejaVuSansMono.ttf}[Path=\detokenize{/usr/lib/python3.11/site-packages/matplotlib/mpl-data/fonts/ttf/}]
%%   \makeatletter\@ifpackageloaded{underscore}{}{\usepackage[strings]{underscore}}\makeatother
%%
\begingroup%
\makeatletter%
\begin{pgfpicture}%
\pgfpathrectangle{\pgfpointorigin}{\pgfqpoint{6.400000in}{4.800000in}}%
\pgfusepath{use as bounding box, clip}%
\begin{pgfscope}%
\pgfsetbuttcap%
\pgfsetmiterjoin%
\definecolor{currentfill}{rgb}{1.000000,1.000000,1.000000}%
\pgfsetfillcolor{currentfill}%
\pgfsetlinewidth{0.000000pt}%
\definecolor{currentstroke}{rgb}{1.000000,1.000000,1.000000}%
\pgfsetstrokecolor{currentstroke}%
\pgfsetdash{}{0pt}%
\pgfpathmoveto{\pgfqpoint{0.000000in}{0.000000in}}%
\pgfpathlineto{\pgfqpoint{6.400000in}{0.000000in}}%
\pgfpathlineto{\pgfqpoint{6.400000in}{4.800000in}}%
\pgfpathlineto{\pgfqpoint{0.000000in}{4.800000in}}%
\pgfpathlineto{\pgfqpoint{0.000000in}{0.000000in}}%
\pgfpathclose%
\pgfusepath{fill}%
\end{pgfscope}%
\begin{pgfscope}%
\pgfsetbuttcap%
\pgfsetmiterjoin%
\definecolor{currentfill}{rgb}{1.000000,1.000000,1.000000}%
\pgfsetfillcolor{currentfill}%
\pgfsetlinewidth{0.000000pt}%
\definecolor{currentstroke}{rgb}{0.000000,0.000000,0.000000}%
\pgfsetstrokecolor{currentstroke}%
\pgfsetstrokeopacity{0.000000}%
\pgfsetdash{}{0pt}%
\pgfpathmoveto{\pgfqpoint{0.617715in}{1.333988in}}%
\pgfpathlineto{\pgfqpoint{6.083829in}{1.333988in}}%
\pgfpathlineto{\pgfqpoint{6.083829in}{4.412376in}}%
\pgfpathlineto{\pgfqpoint{0.617715in}{4.412376in}}%
\pgfpathlineto{\pgfqpoint{0.617715in}{1.333988in}}%
\pgfpathclose%
\pgfusepath{fill}%
\end{pgfscope}%
\begin{pgfscope}%
\pgfsetbuttcap%
\pgfsetroundjoin%
\definecolor{currentfill}{rgb}{0.000000,0.000000,0.000000}%
\pgfsetfillcolor{currentfill}%
\pgfsetlinewidth{0.803000pt}%
\definecolor{currentstroke}{rgb}{0.000000,0.000000,0.000000}%
\pgfsetstrokecolor{currentstroke}%
\pgfsetdash{}{0pt}%
\pgfsys@defobject{currentmarker}{\pgfqpoint{0.000000in}{-0.048611in}}{\pgfqpoint{0.000000in}{0.000000in}}{%
\pgfpathmoveto{\pgfqpoint{0.000000in}{0.000000in}}%
\pgfpathlineto{\pgfqpoint{0.000000in}{-0.048611in}}%
\pgfusepath{stroke,fill}%
}%
\begin{pgfscope}%
\pgfsys@transformshift{0.799919in}{1.333988in}%
\pgfsys@useobject{currentmarker}{}%
\end{pgfscope}%
\end{pgfscope}%
\begin{pgfscope}%
\definecolor{textcolor}{rgb}{0.000000,0.000000,0.000000}%
\pgfsetstrokecolor{textcolor}%
\pgfsetfillcolor{textcolor}%
\pgftext[x=0.618700in, y=0.732642in, left, base,rotate=45.000000]{\color{textcolor}{\sffamily\fontsize{11.000000}{13.200000}\selectfont\catcode`\^=\active\def^{\ifmmode\sp\else\^{}\fi}\catcode`\%=\active\def%{\%}3d-unet}}%
\end{pgfscope}%
\begin{pgfscope}%
\pgfsetbuttcap%
\pgfsetroundjoin%
\definecolor{currentfill}{rgb}{0.000000,0.000000,0.000000}%
\pgfsetfillcolor{currentfill}%
\pgfsetlinewidth{0.803000pt}%
\definecolor{currentstroke}{rgb}{0.000000,0.000000,0.000000}%
\pgfsetstrokecolor{currentstroke}%
\pgfsetdash{}{0pt}%
\pgfsys@defobject{currentmarker}{\pgfqpoint{0.000000in}{-0.048611in}}{\pgfqpoint{0.000000in}{0.000000in}}{%
\pgfpathmoveto{\pgfqpoint{0.000000in}{0.000000in}}%
\pgfpathlineto{\pgfqpoint{0.000000in}{-0.048611in}}%
\pgfusepath{stroke,fill}%
}%
\begin{pgfscope}%
\pgfsys@transformshift{1.164327in}{1.333988in}%
\pgfsys@useobject{currentmarker}{}%
\end{pgfscope}%
\end{pgfscope}%
\begin{pgfscope}%
\definecolor{textcolor}{rgb}{0.000000,0.000000,0.000000}%
\pgfsetstrokecolor{textcolor}%
\pgfsetfillcolor{textcolor}%
\pgftext[x=1.083225in, y=0.932878in, left, base,rotate=45.000000]{\color{textcolor}{\sffamily\fontsize{11.000000}{13.200000}\selectfont\catcode`\^=\active\def^{\ifmmode\sp\else\^{}\fi}\catcode`\%=\active\def%{\%}bert}}%
\end{pgfscope}%
\begin{pgfscope}%
\pgfsetbuttcap%
\pgfsetroundjoin%
\definecolor{currentfill}{rgb}{0.000000,0.000000,0.000000}%
\pgfsetfillcolor{currentfill}%
\pgfsetlinewidth{0.803000pt}%
\definecolor{currentstroke}{rgb}{0.000000,0.000000,0.000000}%
\pgfsetstrokecolor{currentstroke}%
\pgfsetdash{}{0pt}%
\pgfsys@defobject{currentmarker}{\pgfqpoint{0.000000in}{-0.048611in}}{\pgfqpoint{0.000000in}{0.000000in}}{%
\pgfpathmoveto{\pgfqpoint{0.000000in}{0.000000in}}%
\pgfpathlineto{\pgfqpoint{0.000000in}{-0.048611in}}%
\pgfusepath{stroke,fill}%
}%
\begin{pgfscope}%
\pgfsys@transformshift{1.528734in}{1.333988in}%
\pgfsys@useobject{currentmarker}{}%
\end{pgfscope}%
\end{pgfscope}%
\begin{pgfscope}%
\definecolor{textcolor}{rgb}{0.000000,0.000000,0.000000}%
\pgfsetstrokecolor{textcolor}%
\pgfsetfillcolor{textcolor}%
\pgftext[x=1.460266in, y=0.958145in, left, base,rotate=45.000000]{\color{textcolor}{\sffamily\fontsize{11.000000}{13.200000}\selectfont\catcode`\^=\active\def^{\ifmmode\sp\else\^{}\fi}\catcode`\%=\active\def%{\%}dcg}}%
\end{pgfscope}%
\begin{pgfscope}%
\pgfsetbuttcap%
\pgfsetroundjoin%
\definecolor{currentfill}{rgb}{0.000000,0.000000,0.000000}%
\pgfsetfillcolor{currentfill}%
\pgfsetlinewidth{0.803000pt}%
\definecolor{currentstroke}{rgb}{0.000000,0.000000,0.000000}%
\pgfsetstrokecolor{currentstroke}%
\pgfsetdash{}{0pt}%
\pgfsys@defobject{currentmarker}{\pgfqpoint{0.000000in}{-0.048611in}}{\pgfqpoint{0.000000in}{0.000000in}}{%
\pgfpathmoveto{\pgfqpoint{0.000000in}{0.000000in}}%
\pgfpathlineto{\pgfqpoint{0.000000in}{-0.048611in}}%
\pgfusepath{stroke,fill}%
}%
\begin{pgfscope}%
\pgfsys@transformshift{1.893142in}{1.333988in}%
\pgfsys@useobject{currentmarker}{}%
\end{pgfscope}%
\end{pgfscope}%
\begin{pgfscope}%
\definecolor{textcolor}{rgb}{0.000000,0.000000,0.000000}%
\pgfsetstrokecolor{textcolor}%
\pgfsetfillcolor{textcolor}%
\pgftext[x=1.807899in, y=0.924596in, left, base,rotate=45.000000]{\color{textcolor}{\sffamily\fontsize{11.000000}{13.200000}\selectfont\catcode`\^=\active\def^{\ifmmode\sp\else\^{}\fi}\catcode`\%=\active\def%{\%}gms}}%
\end{pgfscope}%
\begin{pgfscope}%
\pgfsetbuttcap%
\pgfsetroundjoin%
\definecolor{currentfill}{rgb}{0.000000,0.000000,0.000000}%
\pgfsetfillcolor{currentfill}%
\pgfsetlinewidth{0.803000pt}%
\definecolor{currentstroke}{rgb}{0.000000,0.000000,0.000000}%
\pgfsetstrokecolor{currentstroke}%
\pgfsetdash{}{0pt}%
\pgfsys@defobject{currentmarker}{\pgfqpoint{0.000000in}{-0.048611in}}{\pgfqpoint{0.000000in}{0.000000in}}{%
\pgfpathmoveto{\pgfqpoint{0.000000in}{0.000000in}}%
\pgfpathlineto{\pgfqpoint{0.000000in}{-0.048611in}}%
\pgfusepath{stroke,fill}%
}%
\begin{pgfscope}%
\pgfsys@transformshift{2.257549in}{1.333988in}%
\pgfsys@useobject{currentmarker}{}%
\end{pgfscope}%
\end{pgfscope}%
\begin{pgfscope}%
\definecolor{textcolor}{rgb}{0.000000,0.000000,0.000000}%
\pgfsetstrokecolor{textcolor}%
\pgfsetfillcolor{textcolor}%
\pgftext[x=2.196624in, y=0.973231in, left, base,rotate=45.000000]{\color{textcolor}{\sffamily\fontsize{11.000000}{13.200000}\selectfont\catcode`\^=\active\def^{\ifmmode\sp\else\^{}\fi}\catcode`\%=\active\def%{\%}gru}}%
\end{pgfscope}%
\begin{pgfscope}%
\pgfsetbuttcap%
\pgfsetroundjoin%
\definecolor{currentfill}{rgb}{0.000000,0.000000,0.000000}%
\pgfsetfillcolor{currentfill}%
\pgfsetlinewidth{0.803000pt}%
\definecolor{currentstroke}{rgb}{0.000000,0.000000,0.000000}%
\pgfsetstrokecolor{currentstroke}%
\pgfsetdash{}{0pt}%
\pgfsys@defobject{currentmarker}{\pgfqpoint{0.000000in}{-0.048611in}}{\pgfqpoint{0.000000in}{0.000000in}}{%
\pgfpathmoveto{\pgfqpoint{0.000000in}{0.000000in}}%
\pgfpathlineto{\pgfqpoint{0.000000in}{-0.048611in}}%
\pgfusepath{stroke,fill}%
}%
\begin{pgfscope}%
\pgfsys@transformshift{2.621957in}{1.333988in}%
\pgfsys@useobject{currentmarker}{}%
\end{pgfscope}%
\end{pgfscope}%
\begin{pgfscope}%
\definecolor{textcolor}{rgb}{0.000000,0.000000,0.000000}%
\pgfsetstrokecolor{textcolor}%
\pgfsetfillcolor{textcolor}%
\pgftext[x=2.568153in, y=0.987473in, left, base,rotate=45.000000]{\color{textcolor}{\sffamily\fontsize{11.000000}{13.200000}\selectfont\catcode`\^=\active\def^{\ifmmode\sp\else\^{}\fi}\catcode`\%=\active\def%{\%}gst}}%
\end{pgfscope}%
\begin{pgfscope}%
\pgfsetbuttcap%
\pgfsetroundjoin%
\definecolor{currentfill}{rgb}{0.000000,0.000000,0.000000}%
\pgfsetfillcolor{currentfill}%
\pgfsetlinewidth{0.803000pt}%
\definecolor{currentstroke}{rgb}{0.000000,0.000000,0.000000}%
\pgfsetstrokecolor{currentstroke}%
\pgfsetdash{}{0pt}%
\pgfsys@defobject{currentmarker}{\pgfqpoint{0.000000in}{-0.048611in}}{\pgfqpoint{0.000000in}{0.000000in}}{%
\pgfpathmoveto{\pgfqpoint{0.000000in}{0.000000in}}%
\pgfpathlineto{\pgfqpoint{0.000000in}{-0.048611in}}%
\pgfusepath{stroke,fill}%
}%
\begin{pgfscope}%
\pgfsys@transformshift{2.986364in}{1.333988in}%
\pgfsys@useobject{currentmarker}{}%
\end{pgfscope}%
\end{pgfscope}%
\begin{pgfscope}%
\definecolor{textcolor}{rgb}{0.000000,0.000000,0.000000}%
\pgfsetstrokecolor{textcolor}%
\pgfsetfillcolor{textcolor}%
\pgftext[x=2.945695in, y=1.013742in, left, base,rotate=45.000000]{\color{textcolor}{\sffamily\fontsize{11.000000}{13.200000}\selectfont\catcode`\^=\active\def^{\ifmmode\sp\else\^{}\fi}\catcode`\%=\active\def%{\%}lgt}}%
\end{pgfscope}%
\begin{pgfscope}%
\pgfsetbuttcap%
\pgfsetroundjoin%
\definecolor{currentfill}{rgb}{0.000000,0.000000,0.000000}%
\pgfsetfillcolor{currentfill}%
\pgfsetlinewidth{0.803000pt}%
\definecolor{currentstroke}{rgb}{0.000000,0.000000,0.000000}%
\pgfsetstrokecolor{currentstroke}%
\pgfsetdash{}{0pt}%
\pgfsys@defobject{currentmarker}{\pgfqpoint{0.000000in}{-0.048611in}}{\pgfqpoint{0.000000in}{0.000000in}}{%
\pgfpathmoveto{\pgfqpoint{0.000000in}{0.000000in}}%
\pgfpathlineto{\pgfqpoint{0.000000in}{-0.048611in}}%
\pgfusepath{stroke,fill}%
}%
\begin{pgfscope}%
\pgfsys@transformshift{3.350772in}{1.333988in}%
\pgfsys@useobject{currentmarker}{}%
\end{pgfscope}%
\end{pgfscope}%
\begin{pgfscope}%
\definecolor{textcolor}{rgb}{0.000000,0.000000,0.000000}%
\pgfsetstrokecolor{textcolor}%
\pgfsetfillcolor{textcolor}%
\pgftext[x=3.283253in, y=0.960044in, left, base,rotate=45.000000]{\color{textcolor}{\sffamily\fontsize{11.000000}{13.200000}\selectfont\catcode`\^=\active\def^{\ifmmode\sp\else\^{}\fi}\catcode`\%=\active\def%{\%}lmc}}%
\end{pgfscope}%
\begin{pgfscope}%
\pgfsetbuttcap%
\pgfsetroundjoin%
\definecolor{currentfill}{rgb}{0.000000,0.000000,0.000000}%
\pgfsetfillcolor{currentfill}%
\pgfsetlinewidth{0.803000pt}%
\definecolor{currentstroke}{rgb}{0.000000,0.000000,0.000000}%
\pgfsetstrokecolor{currentstroke}%
\pgfsetdash{}{0pt}%
\pgfsys@defobject{currentmarker}{\pgfqpoint{0.000000in}{-0.048611in}}{\pgfqpoint{0.000000in}{0.000000in}}{%
\pgfpathmoveto{\pgfqpoint{0.000000in}{0.000000in}}%
\pgfpathlineto{\pgfqpoint{0.000000in}{-0.048611in}}%
\pgfusepath{stroke,fill}%
}%
\begin{pgfscope}%
\pgfsys@transformshift{3.715180in}{1.333988in}%
\pgfsys@useobject{currentmarker}{}%
\end{pgfscope}%
\end{pgfscope}%
\begin{pgfscope}%
\definecolor{textcolor}{rgb}{0.000000,0.000000,0.000000}%
\pgfsetstrokecolor{textcolor}%
\pgfsetfillcolor{textcolor}%
\pgftext[x=3.655151in, y=0.975024in, left, base,rotate=45.000000]{\color{textcolor}{\sffamily\fontsize{11.000000}{13.200000}\selectfont\catcode`\^=\active\def^{\ifmmode\sp\else\^{}\fi}\catcode`\%=\active\def%{\%}lmr}}%
\end{pgfscope}%
\begin{pgfscope}%
\pgfsetbuttcap%
\pgfsetroundjoin%
\definecolor{currentfill}{rgb}{0.000000,0.000000,0.000000}%
\pgfsetfillcolor{currentfill}%
\pgfsetlinewidth{0.803000pt}%
\definecolor{currentstroke}{rgb}{0.000000,0.000000,0.000000}%
\pgfsetstrokecolor{currentstroke}%
\pgfsetdash{}{0pt}%
\pgfsys@defobject{currentmarker}{\pgfqpoint{0.000000in}{-0.048611in}}{\pgfqpoint{0.000000in}{0.000000in}}{%
\pgfpathmoveto{\pgfqpoint{0.000000in}{0.000000in}}%
\pgfpathlineto{\pgfqpoint{0.000000in}{-0.048611in}}%
\pgfusepath{stroke,fill}%
}%
\begin{pgfscope}%
\pgfsys@transformshift{4.079587in}{1.333988in}%
\pgfsys@useobject{currentmarker}{}%
\end{pgfscope}%
\end{pgfscope}%
\begin{pgfscope}%
\definecolor{textcolor}{rgb}{0.000000,0.000000,0.000000}%
\pgfsetstrokecolor{textcolor}%
\pgfsetfillcolor{textcolor}%
\pgftext[x=4.025836in, y=0.987579in, left, base,rotate=45.000000]{\color{textcolor}{\sffamily\fontsize{11.000000}{13.200000}\selectfont\catcode`\^=\active\def^{\ifmmode\sp\else\^{}\fi}\catcode`\%=\active\def%{\%}nst}}%
\end{pgfscope}%
\begin{pgfscope}%
\pgfsetbuttcap%
\pgfsetroundjoin%
\definecolor{currentfill}{rgb}{0.000000,0.000000,0.000000}%
\pgfsetfillcolor{currentfill}%
\pgfsetlinewidth{0.803000pt}%
\definecolor{currentstroke}{rgb}{0.000000,0.000000,0.000000}%
\pgfsetstrokecolor{currentstroke}%
\pgfsetdash{}{0pt}%
\pgfsys@defobject{currentmarker}{\pgfqpoint{0.000000in}{-0.048611in}}{\pgfqpoint{0.000000in}{0.000000in}}{%
\pgfpathmoveto{\pgfqpoint{0.000000in}{0.000000in}}%
\pgfpathlineto{\pgfqpoint{0.000000in}{-0.048611in}}%
\pgfusepath{stroke,fill}%
}%
\begin{pgfscope}%
\pgfsys@transformshift{4.443995in}{1.333988in}%
\pgfsys@useobject{currentmarker}{}%
\end{pgfscope}%
\end{pgfscope}%
\begin{pgfscope}%
\definecolor{textcolor}{rgb}{0.000000,0.000000,0.000000}%
\pgfsetstrokecolor{textcolor}%
\pgfsetfillcolor{textcolor}%
\pgftext[x=4.234027in, y=0.675146in, left, base,rotate=45.000000]{\color{textcolor}{\sffamily\fontsize{11.000000}{13.200000}\selectfont\catcode`\^=\active\def^{\ifmmode\sp\else\^{}\fi}\catcode`\%=\active\def%{\%}resnet50}}%
\end{pgfscope}%
\begin{pgfscope}%
\pgfsetbuttcap%
\pgfsetroundjoin%
\definecolor{currentfill}{rgb}{0.000000,0.000000,0.000000}%
\pgfsetfillcolor{currentfill}%
\pgfsetlinewidth{0.803000pt}%
\definecolor{currentstroke}{rgb}{0.000000,0.000000,0.000000}%
\pgfsetstrokecolor{currentstroke}%
\pgfsetdash{}{0pt}%
\pgfsys@defobject{currentmarker}{\pgfqpoint{0.000000in}{-0.048611in}}{\pgfqpoint{0.000000in}{0.000000in}}{%
\pgfpathmoveto{\pgfqpoint{0.000000in}{0.000000in}}%
\pgfpathlineto{\pgfqpoint{0.000000in}{-0.048611in}}%
\pgfusepath{stroke,fill}%
}%
\begin{pgfscope}%
\pgfsys@transformshift{4.808402in}{1.333988in}%
\pgfsys@useobject{currentmarker}{}%
\end{pgfscope}%
\end{pgfscope}%
\begin{pgfscope}%
\definecolor{textcolor}{rgb}{0.000000,0.000000,0.000000}%
\pgfsetstrokecolor{textcolor}%
\pgfsetfillcolor{textcolor}%
\pgftext[x=4.781975in, y=1.042227in, left, base,rotate=45.000000]{\color{textcolor}{\sffamily\fontsize{11.000000}{13.200000}\selectfont\catcode`\^=\active\def^{\ifmmode\sp\else\^{}\fi}\catcode`\%=\active\def%{\%}rfl}}%
\end{pgfscope}%
\begin{pgfscope}%
\pgfsetbuttcap%
\pgfsetroundjoin%
\definecolor{currentfill}{rgb}{0.000000,0.000000,0.000000}%
\pgfsetfillcolor{currentfill}%
\pgfsetlinewidth{0.803000pt}%
\definecolor{currentstroke}{rgb}{0.000000,0.000000,0.000000}%
\pgfsetstrokecolor{currentstroke}%
\pgfsetdash{}{0pt}%
\pgfsys@defobject{currentmarker}{\pgfqpoint{0.000000in}{-0.048611in}}{\pgfqpoint{0.000000in}{0.000000in}}{%
\pgfpathmoveto{\pgfqpoint{0.000000in}{0.000000in}}%
\pgfpathlineto{\pgfqpoint{0.000000in}{-0.048611in}}%
\pgfusepath{stroke,fill}%
}%
\begin{pgfscope}%
\pgfsys@transformshift{5.172810in}{1.333988in}%
\pgfsys@useobject{currentmarker}{}%
\end{pgfscope}%
\end{pgfscope}%
\begin{pgfscope}%
\definecolor{textcolor}{rgb}{0.000000,0.000000,0.000000}%
\pgfsetstrokecolor{textcolor}%
\pgfsetfillcolor{textcolor}%
\pgftext[x=5.119006in, y=0.987473in, left, base,rotate=45.000000]{\color{textcolor}{\sffamily\fontsize{11.000000}{13.200000}\selectfont\catcode`\^=\active\def^{\ifmmode\sp\else\^{}\fi}\catcode`\%=\active\def%{\%}spt}}%
\end{pgfscope}%
\begin{pgfscope}%
\pgfsetbuttcap%
\pgfsetroundjoin%
\definecolor{currentfill}{rgb}{0.000000,0.000000,0.000000}%
\pgfsetfillcolor{currentfill}%
\pgfsetlinewidth{0.803000pt}%
\definecolor{currentstroke}{rgb}{0.000000,0.000000,0.000000}%
\pgfsetstrokecolor{currentstroke}%
\pgfsetdash{}{0pt}%
\pgfsys@defobject{currentmarker}{\pgfqpoint{0.000000in}{-0.048611in}}{\pgfqpoint{0.000000in}{0.000000in}}{%
\pgfpathmoveto{\pgfqpoint{0.000000in}{0.000000in}}%
\pgfpathlineto{\pgfqpoint{0.000000in}{-0.048611in}}%
\pgfusepath{stroke,fill}%
}%
\begin{pgfscope}%
\pgfsys@transformshift{5.537217in}{1.333988in}%
\pgfsys@useobject{currentmarker}{}%
\end{pgfscope}%
\end{pgfscope}%
\begin{pgfscope}%
\definecolor{textcolor}{rgb}{0.000000,0.000000,0.000000}%
\pgfsetstrokecolor{textcolor}%
\pgfsetfillcolor{textcolor}%
\pgftext[x=5.185117in, y=0.390881in, left, base,rotate=45.000000]{\color{textcolor}{\sffamily\fontsize{11.000000}{13.200000}\selectfont\catcode`\^=\active\def^{\ifmmode\sp\else\^{}\fi}\catcode`\%=\active\def%{\%}ssd-mobilenet}}%
\end{pgfscope}%
\begin{pgfscope}%
\pgfsetbuttcap%
\pgfsetroundjoin%
\definecolor{currentfill}{rgb}{0.000000,0.000000,0.000000}%
\pgfsetfillcolor{currentfill}%
\pgfsetlinewidth{0.803000pt}%
\definecolor{currentstroke}{rgb}{0.000000,0.000000,0.000000}%
\pgfsetstrokecolor{currentstroke}%
\pgfsetdash{}{0pt}%
\pgfsys@defobject{currentmarker}{\pgfqpoint{0.000000in}{-0.048611in}}{\pgfqpoint{0.000000in}{0.000000in}}{%
\pgfpathmoveto{\pgfqpoint{0.000000in}{0.000000in}}%
\pgfpathlineto{\pgfqpoint{0.000000in}{-0.048611in}}%
\pgfusepath{stroke,fill}%
}%
\begin{pgfscope}%
\pgfsys@transformshift{5.901625in}{1.333988in}%
\pgfsys@useobject{currentmarker}{}%
\end{pgfscope}%
\end{pgfscope}%
\begin{pgfscope}%
\definecolor{textcolor}{rgb}{0.000000,0.000000,0.000000}%
\pgfsetstrokecolor{textcolor}%
\pgfsetfillcolor{textcolor}%
\pgftext[x=5.650328in, y=0.592488in, left, base,rotate=45.000000]{\color{textcolor}{\sffamily\fontsize{11.000000}{13.200000}\selectfont\catcode`\^=\active\def^{\ifmmode\sp\else\^{}\fi}\catcode`\%=\active\def%{\%}ssd-resnet}}%
\end{pgfscope}%
\begin{pgfscope}%
\definecolor{textcolor}{rgb}{0.000000,0.000000,0.000000}%
\pgfsetstrokecolor{textcolor}%
\pgfsetfillcolor{textcolor}%
\pgftext[x=3.350772in,y=0.312854in,,top]{\color{textcolor}{\sffamily\fontsize{11.000000}{13.200000}\selectfont\catcode`\^=\active\def^{\ifmmode\sp\else\^{}\fi}\catcode`\%=\active\def%{\%}Workload}}%
\end{pgfscope}%
\begin{pgfscope}%
\pgfsetbuttcap%
\pgfsetroundjoin%
\definecolor{currentfill}{rgb}{0.000000,0.000000,0.000000}%
\pgfsetfillcolor{currentfill}%
\pgfsetlinewidth{0.803000pt}%
\definecolor{currentstroke}{rgb}{0.000000,0.000000,0.000000}%
\pgfsetstrokecolor{currentstroke}%
\pgfsetdash{}{0pt}%
\pgfsys@defobject{currentmarker}{\pgfqpoint{-0.048611in}{0.000000in}}{\pgfqpoint{-0.000000in}{0.000000in}}{%
\pgfpathmoveto{\pgfqpoint{-0.000000in}{0.000000in}}%
\pgfpathlineto{\pgfqpoint{-0.048611in}{0.000000in}}%
\pgfusepath{stroke,fill}%
}%
\begin{pgfscope}%
\pgfsys@transformshift{0.617715in}{1.473912in}%
\pgfsys@useobject{currentmarker}{}%
\end{pgfscope}%
\end{pgfscope}%
\begin{pgfscope}%
\definecolor{textcolor}{rgb}{0.000000,0.000000,0.000000}%
\pgfsetstrokecolor{textcolor}%
\pgfsetfillcolor{textcolor}%
\pgftext[x=0.444451in, y=1.415874in, left, base]{\color{textcolor}{\sffamily\fontsize{11.000000}{13.200000}\selectfont\catcode`\^=\active\def^{\ifmmode\sp\else\^{}\fi}\catcode`\%=\active\def%{\%}$\mathdefault{0}$}}%
\end{pgfscope}%
\begin{pgfscope}%
\pgfsetbuttcap%
\pgfsetroundjoin%
\definecolor{currentfill}{rgb}{0.000000,0.000000,0.000000}%
\pgfsetfillcolor{currentfill}%
\pgfsetlinewidth{0.803000pt}%
\definecolor{currentstroke}{rgb}{0.000000,0.000000,0.000000}%
\pgfsetstrokecolor{currentstroke}%
\pgfsetdash{}{0pt}%
\pgfsys@defobject{currentmarker}{\pgfqpoint{-0.048611in}{0.000000in}}{\pgfqpoint{-0.000000in}{0.000000in}}{%
\pgfpathmoveto{\pgfqpoint{-0.000000in}{0.000000in}}%
\pgfpathlineto{\pgfqpoint{-0.048611in}{0.000000in}}%
\pgfusepath{stroke,fill}%
}%
\begin{pgfscope}%
\pgfsys@transformshift{0.617715in}{1.854077in}%
\pgfsys@useobject{currentmarker}{}%
\end{pgfscope}%
\end{pgfscope}%
\begin{pgfscope}%
\definecolor{textcolor}{rgb}{0.000000,0.000000,0.000000}%
\pgfsetstrokecolor{textcolor}%
\pgfsetfillcolor{textcolor}%
\pgftext[x=0.368410in, y=1.796039in, left, base]{\color{textcolor}{\sffamily\fontsize{11.000000}{13.200000}\selectfont\catcode`\^=\active\def^{\ifmmode\sp\else\^{}\fi}\catcode`\%=\active\def%{\%}$\mathdefault{10}$}}%
\end{pgfscope}%
\begin{pgfscope}%
\pgfsetbuttcap%
\pgfsetroundjoin%
\definecolor{currentfill}{rgb}{0.000000,0.000000,0.000000}%
\pgfsetfillcolor{currentfill}%
\pgfsetlinewidth{0.803000pt}%
\definecolor{currentstroke}{rgb}{0.000000,0.000000,0.000000}%
\pgfsetstrokecolor{currentstroke}%
\pgfsetdash{}{0pt}%
\pgfsys@defobject{currentmarker}{\pgfqpoint{-0.048611in}{0.000000in}}{\pgfqpoint{-0.000000in}{0.000000in}}{%
\pgfpathmoveto{\pgfqpoint{-0.000000in}{0.000000in}}%
\pgfpathlineto{\pgfqpoint{-0.048611in}{0.000000in}}%
\pgfusepath{stroke,fill}%
}%
\begin{pgfscope}%
\pgfsys@transformshift{0.617715in}{2.234242in}%
\pgfsys@useobject{currentmarker}{}%
\end{pgfscope}%
\end{pgfscope}%
\begin{pgfscope}%
\definecolor{textcolor}{rgb}{0.000000,0.000000,0.000000}%
\pgfsetstrokecolor{textcolor}%
\pgfsetfillcolor{textcolor}%
\pgftext[x=0.368410in, y=2.176205in, left, base]{\color{textcolor}{\sffamily\fontsize{11.000000}{13.200000}\selectfont\catcode`\^=\active\def^{\ifmmode\sp\else\^{}\fi}\catcode`\%=\active\def%{\%}$\mathdefault{20}$}}%
\end{pgfscope}%
\begin{pgfscope}%
\pgfsetbuttcap%
\pgfsetroundjoin%
\definecolor{currentfill}{rgb}{0.000000,0.000000,0.000000}%
\pgfsetfillcolor{currentfill}%
\pgfsetlinewidth{0.803000pt}%
\definecolor{currentstroke}{rgb}{0.000000,0.000000,0.000000}%
\pgfsetstrokecolor{currentstroke}%
\pgfsetdash{}{0pt}%
\pgfsys@defobject{currentmarker}{\pgfqpoint{-0.048611in}{0.000000in}}{\pgfqpoint{-0.000000in}{0.000000in}}{%
\pgfpathmoveto{\pgfqpoint{-0.000000in}{0.000000in}}%
\pgfpathlineto{\pgfqpoint{-0.048611in}{0.000000in}}%
\pgfusepath{stroke,fill}%
}%
\begin{pgfscope}%
\pgfsys@transformshift{0.617715in}{2.614408in}%
\pgfsys@useobject{currentmarker}{}%
\end{pgfscope}%
\end{pgfscope}%
\begin{pgfscope}%
\definecolor{textcolor}{rgb}{0.000000,0.000000,0.000000}%
\pgfsetstrokecolor{textcolor}%
\pgfsetfillcolor{textcolor}%
\pgftext[x=0.368410in, y=2.556370in, left, base]{\color{textcolor}{\sffamily\fontsize{11.000000}{13.200000}\selectfont\catcode`\^=\active\def^{\ifmmode\sp\else\^{}\fi}\catcode`\%=\active\def%{\%}$\mathdefault{30}$}}%
\end{pgfscope}%
\begin{pgfscope}%
\pgfsetbuttcap%
\pgfsetroundjoin%
\definecolor{currentfill}{rgb}{0.000000,0.000000,0.000000}%
\pgfsetfillcolor{currentfill}%
\pgfsetlinewidth{0.803000pt}%
\definecolor{currentstroke}{rgb}{0.000000,0.000000,0.000000}%
\pgfsetstrokecolor{currentstroke}%
\pgfsetdash{}{0pt}%
\pgfsys@defobject{currentmarker}{\pgfqpoint{-0.048611in}{0.000000in}}{\pgfqpoint{-0.000000in}{0.000000in}}{%
\pgfpathmoveto{\pgfqpoint{-0.000000in}{0.000000in}}%
\pgfpathlineto{\pgfqpoint{-0.048611in}{0.000000in}}%
\pgfusepath{stroke,fill}%
}%
\begin{pgfscope}%
\pgfsys@transformshift{0.617715in}{2.994573in}%
\pgfsys@useobject{currentmarker}{}%
\end{pgfscope}%
\end{pgfscope}%
\begin{pgfscope}%
\definecolor{textcolor}{rgb}{0.000000,0.000000,0.000000}%
\pgfsetstrokecolor{textcolor}%
\pgfsetfillcolor{textcolor}%
\pgftext[x=0.368410in, y=2.936536in, left, base]{\color{textcolor}{\sffamily\fontsize{11.000000}{13.200000}\selectfont\catcode`\^=\active\def^{\ifmmode\sp\else\^{}\fi}\catcode`\%=\active\def%{\%}$\mathdefault{40}$}}%
\end{pgfscope}%
\begin{pgfscope}%
\pgfsetbuttcap%
\pgfsetroundjoin%
\definecolor{currentfill}{rgb}{0.000000,0.000000,0.000000}%
\pgfsetfillcolor{currentfill}%
\pgfsetlinewidth{0.803000pt}%
\definecolor{currentstroke}{rgb}{0.000000,0.000000,0.000000}%
\pgfsetstrokecolor{currentstroke}%
\pgfsetdash{}{0pt}%
\pgfsys@defobject{currentmarker}{\pgfqpoint{-0.048611in}{0.000000in}}{\pgfqpoint{-0.000000in}{0.000000in}}{%
\pgfpathmoveto{\pgfqpoint{-0.000000in}{0.000000in}}%
\pgfpathlineto{\pgfqpoint{-0.048611in}{0.000000in}}%
\pgfusepath{stroke,fill}%
}%
\begin{pgfscope}%
\pgfsys@transformshift{0.617715in}{3.374739in}%
\pgfsys@useobject{currentmarker}{}%
\end{pgfscope}%
\end{pgfscope}%
\begin{pgfscope}%
\definecolor{textcolor}{rgb}{0.000000,0.000000,0.000000}%
\pgfsetstrokecolor{textcolor}%
\pgfsetfillcolor{textcolor}%
\pgftext[x=0.368410in, y=3.316701in, left, base]{\color{textcolor}{\sffamily\fontsize{11.000000}{13.200000}\selectfont\catcode`\^=\active\def^{\ifmmode\sp\else\^{}\fi}\catcode`\%=\active\def%{\%}$\mathdefault{50}$}}%
\end{pgfscope}%
\begin{pgfscope}%
\pgfsetbuttcap%
\pgfsetroundjoin%
\definecolor{currentfill}{rgb}{0.000000,0.000000,0.000000}%
\pgfsetfillcolor{currentfill}%
\pgfsetlinewidth{0.803000pt}%
\definecolor{currentstroke}{rgb}{0.000000,0.000000,0.000000}%
\pgfsetstrokecolor{currentstroke}%
\pgfsetdash{}{0pt}%
\pgfsys@defobject{currentmarker}{\pgfqpoint{-0.048611in}{0.000000in}}{\pgfqpoint{-0.000000in}{0.000000in}}{%
\pgfpathmoveto{\pgfqpoint{-0.000000in}{0.000000in}}%
\pgfpathlineto{\pgfqpoint{-0.048611in}{0.000000in}}%
\pgfusepath{stroke,fill}%
}%
\begin{pgfscope}%
\pgfsys@transformshift{0.617715in}{3.754904in}%
\pgfsys@useobject{currentmarker}{}%
\end{pgfscope}%
\end{pgfscope}%
\begin{pgfscope}%
\definecolor{textcolor}{rgb}{0.000000,0.000000,0.000000}%
\pgfsetstrokecolor{textcolor}%
\pgfsetfillcolor{textcolor}%
\pgftext[x=0.368410in, y=3.696866in, left, base]{\color{textcolor}{\sffamily\fontsize{11.000000}{13.200000}\selectfont\catcode`\^=\active\def^{\ifmmode\sp\else\^{}\fi}\catcode`\%=\active\def%{\%}$\mathdefault{60}$}}%
\end{pgfscope}%
\begin{pgfscope}%
\pgfsetbuttcap%
\pgfsetroundjoin%
\definecolor{currentfill}{rgb}{0.000000,0.000000,0.000000}%
\pgfsetfillcolor{currentfill}%
\pgfsetlinewidth{0.803000pt}%
\definecolor{currentstroke}{rgb}{0.000000,0.000000,0.000000}%
\pgfsetstrokecolor{currentstroke}%
\pgfsetdash{}{0pt}%
\pgfsys@defobject{currentmarker}{\pgfqpoint{-0.048611in}{0.000000in}}{\pgfqpoint{-0.000000in}{0.000000in}}{%
\pgfpathmoveto{\pgfqpoint{-0.000000in}{0.000000in}}%
\pgfpathlineto{\pgfqpoint{-0.048611in}{0.000000in}}%
\pgfusepath{stroke,fill}%
}%
\begin{pgfscope}%
\pgfsys@transformshift{0.617715in}{4.135070in}%
\pgfsys@useobject{currentmarker}{}%
\end{pgfscope}%
\end{pgfscope}%
\begin{pgfscope}%
\definecolor{textcolor}{rgb}{0.000000,0.000000,0.000000}%
\pgfsetstrokecolor{textcolor}%
\pgfsetfillcolor{textcolor}%
\pgftext[x=0.368410in, y=4.077032in, left, base]{\color{textcolor}{\sffamily\fontsize{11.000000}{13.200000}\selectfont\catcode`\^=\active\def^{\ifmmode\sp\else\^{}\fi}\catcode`\%=\active\def%{\%}$\mathdefault{70}$}}%
\end{pgfscope}%
\begin{pgfscope}%
\definecolor{textcolor}{rgb}{0.000000,0.000000,0.000000}%
\pgfsetstrokecolor{textcolor}%
\pgfsetfillcolor{textcolor}%
\pgftext[x=0.312854in,y=2.873182in,,bottom,rotate=90.000000]{\color{textcolor}{\sffamily\fontsize{11.000000}{13.200000}\selectfont\catcode`\^=\active\def^{\ifmmode\sp\else\^{}\fi}\catcode`\%=\active\def%{\%}Relative IPC difference (%)}}%
\end{pgfscope}%
\begin{pgfscope}%
\pgfpathrectangle{\pgfqpoint{0.617715in}{1.333988in}}{\pgfqpoint{5.466114in}{3.078389in}}%
\pgfusepath{clip}%
\pgfsetrectcap%
\pgfsetroundjoin%
\pgfsetlinewidth{1.003750pt}%
\definecolor{currentstroke}{rgb}{0.000000,0.000000,0.000000}%
\pgfsetstrokecolor{currentstroke}%
\pgfsetdash{}{0pt}%
\pgfpathmoveto{\pgfqpoint{0.708817in}{1.483184in}}%
\pgfpathlineto{\pgfqpoint{0.891021in}{1.483184in}}%
\pgfpathlineto{\pgfqpoint{0.891021in}{1.514390in}}%
\pgfpathlineto{\pgfqpoint{0.708817in}{1.514390in}}%
\pgfpathlineto{\pgfqpoint{0.708817in}{1.483184in}}%
\pgfusepath{stroke}%
\end{pgfscope}%
\begin{pgfscope}%
\pgfpathrectangle{\pgfqpoint{0.617715in}{1.333988in}}{\pgfqpoint{5.466114in}{3.078389in}}%
\pgfusepath{clip}%
\pgfsetrectcap%
\pgfsetroundjoin%
\pgfsetlinewidth{1.003750pt}%
\definecolor{currentstroke}{rgb}{0.000000,0.000000,0.000000}%
\pgfsetstrokecolor{currentstroke}%
\pgfsetdash{}{0pt}%
\pgfpathmoveto{\pgfqpoint{0.799919in}{1.483184in}}%
\pgfpathlineto{\pgfqpoint{0.799919in}{1.474006in}}%
\pgfusepath{stroke}%
\end{pgfscope}%
\begin{pgfscope}%
\pgfpathrectangle{\pgfqpoint{0.617715in}{1.333988in}}{\pgfqpoint{5.466114in}{3.078389in}}%
\pgfusepath{clip}%
\pgfsetrectcap%
\pgfsetroundjoin%
\pgfsetlinewidth{1.003750pt}%
\definecolor{currentstroke}{rgb}{0.000000,0.000000,0.000000}%
\pgfsetstrokecolor{currentstroke}%
\pgfsetdash{}{0pt}%
\pgfpathmoveto{\pgfqpoint{0.799919in}{1.514390in}}%
\pgfpathlineto{\pgfqpoint{0.799919in}{1.531241in}}%
\pgfusepath{stroke}%
\end{pgfscope}%
\begin{pgfscope}%
\pgfpathrectangle{\pgfqpoint{0.617715in}{1.333988in}}{\pgfqpoint{5.466114in}{3.078389in}}%
\pgfusepath{clip}%
\pgfsetrectcap%
\pgfsetroundjoin%
\pgfsetlinewidth{1.003750pt}%
\definecolor{currentstroke}{rgb}{0.000000,0.000000,0.000000}%
\pgfsetstrokecolor{currentstroke}%
\pgfsetdash{}{0pt}%
\pgfpathmoveto{\pgfqpoint{0.754368in}{1.474006in}}%
\pgfpathlineto{\pgfqpoint{0.845470in}{1.474006in}}%
\pgfusepath{stroke}%
\end{pgfscope}%
\begin{pgfscope}%
\pgfpathrectangle{\pgfqpoint{0.617715in}{1.333988in}}{\pgfqpoint{5.466114in}{3.078389in}}%
\pgfusepath{clip}%
\pgfsetrectcap%
\pgfsetroundjoin%
\pgfsetlinewidth{1.003750pt}%
\definecolor{currentstroke}{rgb}{0.000000,0.000000,0.000000}%
\pgfsetstrokecolor{currentstroke}%
\pgfsetdash{}{0pt}%
\pgfpathmoveto{\pgfqpoint{0.754368in}{1.531241in}}%
\pgfpathlineto{\pgfqpoint{0.845470in}{1.531241in}}%
\pgfusepath{stroke}%
\end{pgfscope}%
\begin{pgfscope}%
\pgfpathrectangle{\pgfqpoint{0.617715in}{1.333988in}}{\pgfqpoint{5.466114in}{3.078389in}}%
\pgfusepath{clip}%
\pgfsetbuttcap%
\pgfsetroundjoin%
\definecolor{currentfill}{rgb}{0.000000,0.000000,0.000000}%
\pgfsetfillcolor{currentfill}%
\pgfsetfillopacity{0.000000}%
\pgfsetlinewidth{1.003750pt}%
\definecolor{currentstroke}{rgb}{0.000000,0.000000,0.000000}%
\pgfsetstrokecolor{currentstroke}%
\pgfsetdash{}{0pt}%
\pgfsys@defobject{currentmarker}{\pgfqpoint{-0.041667in}{-0.041667in}}{\pgfqpoint{0.041667in}{0.041667in}}{%
\pgfpathmoveto{\pgfqpoint{0.000000in}{-0.041667in}}%
\pgfpathcurveto{\pgfqpoint{0.011050in}{-0.041667in}}{\pgfqpoint{0.021649in}{-0.037276in}}{\pgfqpoint{0.029463in}{-0.029463in}}%
\pgfpathcurveto{\pgfqpoint{0.037276in}{-0.021649in}}{\pgfqpoint{0.041667in}{-0.011050in}}{\pgfqpoint{0.041667in}{0.000000in}}%
\pgfpathcurveto{\pgfqpoint{0.041667in}{0.011050in}}{\pgfqpoint{0.037276in}{0.021649in}}{\pgfqpoint{0.029463in}{0.029463in}}%
\pgfpathcurveto{\pgfqpoint{0.021649in}{0.037276in}}{\pgfqpoint{0.011050in}{0.041667in}}{\pgfqpoint{0.000000in}{0.041667in}}%
\pgfpathcurveto{\pgfqpoint{-0.011050in}{0.041667in}}{\pgfqpoint{-0.021649in}{0.037276in}}{\pgfqpoint{-0.029463in}{0.029463in}}%
\pgfpathcurveto{\pgfqpoint{-0.037276in}{0.021649in}}{\pgfqpoint{-0.041667in}{0.011050in}}{\pgfqpoint{-0.041667in}{0.000000in}}%
\pgfpathcurveto{\pgfqpoint{-0.041667in}{-0.011050in}}{\pgfqpoint{-0.037276in}{-0.021649in}}{\pgfqpoint{-0.029463in}{-0.029463in}}%
\pgfpathcurveto{\pgfqpoint{-0.021649in}{-0.037276in}}{\pgfqpoint{-0.011050in}{-0.041667in}}{\pgfqpoint{0.000000in}{-0.041667in}}%
\pgfpathlineto{\pgfqpoint{0.000000in}{-0.041667in}}%
\pgfpathclose%
\pgfusepath{stroke,fill}%
}%
\begin{pgfscope}%
\pgfsys@transformshift{0.799919in}{1.601915in}%
\pgfsys@useobject{currentmarker}{}%
\end{pgfscope}%
\begin{pgfscope}%
\pgfsys@transformshift{0.799919in}{1.675493in}%
\pgfsys@useobject{currentmarker}{}%
\end{pgfscope}%
\begin{pgfscope}%
\pgfsys@transformshift{0.799919in}{1.568304in}%
\pgfsys@useobject{currentmarker}{}%
\end{pgfscope}%
\begin{pgfscope}%
\pgfsys@transformshift{0.799919in}{1.903316in}%
\pgfsys@useobject{currentmarker}{}%
\end{pgfscope}%
\begin{pgfscope}%
\pgfsys@transformshift{0.799919in}{1.585337in}%
\pgfsys@useobject{currentmarker}{}%
\end{pgfscope}%
\begin{pgfscope}%
\pgfsys@transformshift{0.799919in}{1.672010in}%
\pgfsys@useobject{currentmarker}{}%
\end{pgfscope}%
\begin{pgfscope}%
\pgfsys@transformshift{0.799919in}{1.593271in}%
\pgfsys@useobject{currentmarker}{}%
\end{pgfscope}%
\begin{pgfscope}%
\pgfsys@transformshift{0.799919in}{1.907327in}%
\pgfsys@useobject{currentmarker}{}%
\end{pgfscope}%
\end{pgfscope}%
\begin{pgfscope}%
\pgfpathrectangle{\pgfqpoint{0.617715in}{1.333988in}}{\pgfqpoint{5.466114in}{3.078389in}}%
\pgfusepath{clip}%
\pgfsetrectcap%
\pgfsetroundjoin%
\pgfsetlinewidth{1.003750pt}%
\definecolor{currentstroke}{rgb}{0.000000,0.000000,0.000000}%
\pgfsetstrokecolor{currentstroke}%
\pgfsetdash{}{0pt}%
\pgfpathmoveto{\pgfqpoint{1.073225in}{1.474335in}}%
\pgfpathlineto{\pgfqpoint{1.255429in}{1.474335in}}%
\pgfpathlineto{\pgfqpoint{1.255429in}{1.478752in}}%
\pgfpathlineto{\pgfqpoint{1.073225in}{1.478752in}}%
\pgfpathlineto{\pgfqpoint{1.073225in}{1.474335in}}%
\pgfusepath{stroke}%
\end{pgfscope}%
\begin{pgfscope}%
\pgfpathrectangle{\pgfqpoint{0.617715in}{1.333988in}}{\pgfqpoint{5.466114in}{3.078389in}}%
\pgfusepath{clip}%
\pgfsetrectcap%
\pgfsetroundjoin%
\pgfsetlinewidth{1.003750pt}%
\definecolor{currentstroke}{rgb}{0.000000,0.000000,0.000000}%
\pgfsetstrokecolor{currentstroke}%
\pgfsetdash{}{0pt}%
\pgfpathmoveto{\pgfqpoint{1.164327in}{1.474335in}}%
\pgfpathlineto{\pgfqpoint{1.164327in}{1.473914in}}%
\pgfusepath{stroke}%
\end{pgfscope}%
\begin{pgfscope}%
\pgfpathrectangle{\pgfqpoint{0.617715in}{1.333988in}}{\pgfqpoint{5.466114in}{3.078389in}}%
\pgfusepath{clip}%
\pgfsetrectcap%
\pgfsetroundjoin%
\pgfsetlinewidth{1.003750pt}%
\definecolor{currentstroke}{rgb}{0.000000,0.000000,0.000000}%
\pgfsetstrokecolor{currentstroke}%
\pgfsetdash{}{0pt}%
\pgfpathmoveto{\pgfqpoint{1.164327in}{1.478752in}}%
\pgfpathlineto{\pgfqpoint{1.164327in}{1.485349in}}%
\pgfusepath{stroke}%
\end{pgfscope}%
\begin{pgfscope}%
\pgfpathrectangle{\pgfqpoint{0.617715in}{1.333988in}}{\pgfqpoint{5.466114in}{3.078389in}}%
\pgfusepath{clip}%
\pgfsetrectcap%
\pgfsetroundjoin%
\pgfsetlinewidth{1.003750pt}%
\definecolor{currentstroke}{rgb}{0.000000,0.000000,0.000000}%
\pgfsetstrokecolor{currentstroke}%
\pgfsetdash{}{0pt}%
\pgfpathmoveto{\pgfqpoint{1.118776in}{1.473914in}}%
\pgfpathlineto{\pgfqpoint{1.209878in}{1.473914in}}%
\pgfusepath{stroke}%
\end{pgfscope}%
\begin{pgfscope}%
\pgfpathrectangle{\pgfqpoint{0.617715in}{1.333988in}}{\pgfqpoint{5.466114in}{3.078389in}}%
\pgfusepath{clip}%
\pgfsetrectcap%
\pgfsetroundjoin%
\pgfsetlinewidth{1.003750pt}%
\definecolor{currentstroke}{rgb}{0.000000,0.000000,0.000000}%
\pgfsetstrokecolor{currentstroke}%
\pgfsetdash{}{0pt}%
\pgfpathmoveto{\pgfqpoint{1.118776in}{1.485349in}}%
\pgfpathlineto{\pgfqpoint{1.209878in}{1.485349in}}%
\pgfusepath{stroke}%
\end{pgfscope}%
\begin{pgfscope}%
\pgfpathrectangle{\pgfqpoint{0.617715in}{1.333988in}}{\pgfqpoint{5.466114in}{3.078389in}}%
\pgfusepath{clip}%
\pgfsetbuttcap%
\pgfsetroundjoin%
\definecolor{currentfill}{rgb}{0.000000,0.000000,0.000000}%
\pgfsetfillcolor{currentfill}%
\pgfsetfillopacity{0.000000}%
\pgfsetlinewidth{1.003750pt}%
\definecolor{currentstroke}{rgb}{0.000000,0.000000,0.000000}%
\pgfsetstrokecolor{currentstroke}%
\pgfsetdash{}{0pt}%
\pgfsys@defobject{currentmarker}{\pgfqpoint{-0.041667in}{-0.041667in}}{\pgfqpoint{0.041667in}{0.041667in}}{%
\pgfpathmoveto{\pgfqpoint{0.000000in}{-0.041667in}}%
\pgfpathcurveto{\pgfqpoint{0.011050in}{-0.041667in}}{\pgfqpoint{0.021649in}{-0.037276in}}{\pgfqpoint{0.029463in}{-0.029463in}}%
\pgfpathcurveto{\pgfqpoint{0.037276in}{-0.021649in}}{\pgfqpoint{0.041667in}{-0.011050in}}{\pgfqpoint{0.041667in}{0.000000in}}%
\pgfpathcurveto{\pgfqpoint{0.041667in}{0.011050in}}{\pgfqpoint{0.037276in}{0.021649in}}{\pgfqpoint{0.029463in}{0.029463in}}%
\pgfpathcurveto{\pgfqpoint{0.021649in}{0.037276in}}{\pgfqpoint{0.011050in}{0.041667in}}{\pgfqpoint{0.000000in}{0.041667in}}%
\pgfpathcurveto{\pgfqpoint{-0.011050in}{0.041667in}}{\pgfqpoint{-0.021649in}{0.037276in}}{\pgfqpoint{-0.029463in}{0.029463in}}%
\pgfpathcurveto{\pgfqpoint{-0.037276in}{0.021649in}}{\pgfqpoint{-0.041667in}{0.011050in}}{\pgfqpoint{-0.041667in}{0.000000in}}%
\pgfpathcurveto{\pgfqpoint{-0.041667in}{-0.011050in}}{\pgfqpoint{-0.037276in}{-0.021649in}}{\pgfqpoint{-0.029463in}{-0.029463in}}%
\pgfpathcurveto{\pgfqpoint{-0.021649in}{-0.037276in}}{\pgfqpoint{-0.011050in}{-0.041667in}}{\pgfqpoint{0.000000in}{-0.041667in}}%
\pgfpathlineto{\pgfqpoint{0.000000in}{-0.041667in}}%
\pgfpathclose%
\pgfusepath{stroke,fill}%
}%
\begin{pgfscope}%
\pgfsys@transformshift{1.164327in}{1.488151in}%
\pgfsys@useobject{currentmarker}{}%
\end{pgfscope}%
\begin{pgfscope}%
\pgfsys@transformshift{1.164327in}{1.488103in}%
\pgfsys@useobject{currentmarker}{}%
\end{pgfscope}%
\begin{pgfscope}%
\pgfsys@transformshift{1.164327in}{1.489642in}%
\pgfsys@useobject{currentmarker}{}%
\end{pgfscope}%
\begin{pgfscope}%
\pgfsys@transformshift{1.164327in}{1.487935in}%
\pgfsys@useobject{currentmarker}{}%
\end{pgfscope}%
\begin{pgfscope}%
\pgfsys@transformshift{1.164327in}{1.517876in}%
\pgfsys@useobject{currentmarker}{}%
\end{pgfscope}%
\begin{pgfscope}%
\pgfsys@transformshift{1.164327in}{1.508985in}%
\pgfsys@useobject{currentmarker}{}%
\end{pgfscope}%
\begin{pgfscope}%
\pgfsys@transformshift{1.164327in}{1.495722in}%
\pgfsys@useobject{currentmarker}{}%
\end{pgfscope}%
\begin{pgfscope}%
\pgfsys@transformshift{1.164327in}{1.526075in}%
\pgfsys@useobject{currentmarker}{}%
\end{pgfscope}%
\begin{pgfscope}%
\pgfsys@transformshift{1.164327in}{1.505880in}%
\pgfsys@useobject{currentmarker}{}%
\end{pgfscope}%
\begin{pgfscope}%
\pgfsys@transformshift{1.164327in}{1.489367in}%
\pgfsys@useobject{currentmarker}{}%
\end{pgfscope}%
\begin{pgfscope}%
\pgfsys@transformshift{1.164327in}{1.495651in}%
\pgfsys@useobject{currentmarker}{}%
\end{pgfscope}%
\begin{pgfscope}%
\pgfsys@transformshift{1.164327in}{1.492862in}%
\pgfsys@useobject{currentmarker}{}%
\end{pgfscope}%
\begin{pgfscope}%
\pgfsys@transformshift{1.164327in}{1.495974in}%
\pgfsys@useobject{currentmarker}{}%
\end{pgfscope}%
\begin{pgfscope}%
\pgfsys@transformshift{1.164327in}{1.491183in}%
\pgfsys@useobject{currentmarker}{}%
\end{pgfscope}%
\begin{pgfscope}%
\pgfsys@transformshift{1.164327in}{1.493339in}%
\pgfsys@useobject{currentmarker}{}%
\end{pgfscope}%
\begin{pgfscope}%
\pgfsys@transformshift{1.164327in}{1.490587in}%
\pgfsys@useobject{currentmarker}{}%
\end{pgfscope}%
\begin{pgfscope}%
\pgfsys@transformshift{1.164327in}{1.491894in}%
\pgfsys@useobject{currentmarker}{}%
\end{pgfscope}%
\begin{pgfscope}%
\pgfsys@transformshift{1.164327in}{1.489263in}%
\pgfsys@useobject{currentmarker}{}%
\end{pgfscope}%
\begin{pgfscope}%
\pgfsys@transformshift{1.164327in}{1.485920in}%
\pgfsys@useobject{currentmarker}{}%
\end{pgfscope}%
\begin{pgfscope}%
\pgfsys@transformshift{1.164327in}{1.490852in}%
\pgfsys@useobject{currentmarker}{}%
\end{pgfscope}%
\begin{pgfscope}%
\pgfsys@transformshift{1.164327in}{1.493989in}%
\pgfsys@useobject{currentmarker}{}%
\end{pgfscope}%
\begin{pgfscope}%
\pgfsys@transformshift{1.164327in}{1.491224in}%
\pgfsys@useobject{currentmarker}{}%
\end{pgfscope}%
\begin{pgfscope}%
\pgfsys@transformshift{1.164327in}{1.497957in}%
\pgfsys@useobject{currentmarker}{}%
\end{pgfscope}%
\begin{pgfscope}%
\pgfsys@transformshift{1.164327in}{1.496448in}%
\pgfsys@useobject{currentmarker}{}%
\end{pgfscope}%
\begin{pgfscope}%
\pgfsys@transformshift{1.164327in}{1.486741in}%
\pgfsys@useobject{currentmarker}{}%
\end{pgfscope}%
\begin{pgfscope}%
\pgfsys@transformshift{1.164327in}{1.488987in}%
\pgfsys@useobject{currentmarker}{}%
\end{pgfscope}%
\begin{pgfscope}%
\pgfsys@transformshift{1.164327in}{1.503744in}%
\pgfsys@useobject{currentmarker}{}%
\end{pgfscope}%
\begin{pgfscope}%
\pgfsys@transformshift{1.164327in}{1.490071in}%
\pgfsys@useobject{currentmarker}{}%
\end{pgfscope}%
\begin{pgfscope}%
\pgfsys@transformshift{1.164327in}{1.564234in}%
\pgfsys@useobject{currentmarker}{}%
\end{pgfscope}%
\begin{pgfscope}%
\pgfsys@transformshift{1.164327in}{1.499398in}%
\pgfsys@useobject{currentmarker}{}%
\end{pgfscope}%
\begin{pgfscope}%
\pgfsys@transformshift{1.164327in}{1.497005in}%
\pgfsys@useobject{currentmarker}{}%
\end{pgfscope}%
\begin{pgfscope}%
\pgfsys@transformshift{1.164327in}{1.500488in}%
\pgfsys@useobject{currentmarker}{}%
\end{pgfscope}%
\begin{pgfscope}%
\pgfsys@transformshift{1.164327in}{1.504054in}%
\pgfsys@useobject{currentmarker}{}%
\end{pgfscope}%
\begin{pgfscope}%
\pgfsys@transformshift{1.164327in}{1.520199in}%
\pgfsys@useobject{currentmarker}{}%
\end{pgfscope}%
\begin{pgfscope}%
\pgfsys@transformshift{1.164327in}{1.487151in}%
\pgfsys@useobject{currentmarker}{}%
\end{pgfscope}%
\begin{pgfscope}%
\pgfsys@transformshift{1.164327in}{1.497569in}%
\pgfsys@useobject{currentmarker}{}%
\end{pgfscope}%
\begin{pgfscope}%
\pgfsys@transformshift{1.164327in}{1.487916in}%
\pgfsys@useobject{currentmarker}{}%
\end{pgfscope}%
\begin{pgfscope}%
\pgfsys@transformshift{1.164327in}{1.565022in}%
\pgfsys@useobject{currentmarker}{}%
\end{pgfscope}%
\begin{pgfscope}%
\pgfsys@transformshift{1.164327in}{1.568238in}%
\pgfsys@useobject{currentmarker}{}%
\end{pgfscope}%
\begin{pgfscope}%
\pgfsys@transformshift{1.164327in}{1.497501in}%
\pgfsys@useobject{currentmarker}{}%
\end{pgfscope}%
\begin{pgfscope}%
\pgfsys@transformshift{1.164327in}{1.489684in}%
\pgfsys@useobject{currentmarker}{}%
\end{pgfscope}%
\begin{pgfscope}%
\pgfsys@transformshift{1.164327in}{1.501981in}%
\pgfsys@useobject{currentmarker}{}%
\end{pgfscope}%
\begin{pgfscope}%
\pgfsys@transformshift{1.164327in}{1.489008in}%
\pgfsys@useobject{currentmarker}{}%
\end{pgfscope}%
\begin{pgfscope}%
\pgfsys@transformshift{1.164327in}{1.493567in}%
\pgfsys@useobject{currentmarker}{}%
\end{pgfscope}%
\begin{pgfscope}%
\pgfsys@transformshift{1.164327in}{1.491828in}%
\pgfsys@useobject{currentmarker}{}%
\end{pgfscope}%
\begin{pgfscope}%
\pgfsys@transformshift{1.164327in}{1.497130in}%
\pgfsys@useobject{currentmarker}{}%
\end{pgfscope}%
\begin{pgfscope}%
\pgfsys@transformshift{1.164327in}{1.500563in}%
\pgfsys@useobject{currentmarker}{}%
\end{pgfscope}%
\begin{pgfscope}%
\pgfsys@transformshift{1.164327in}{1.494686in}%
\pgfsys@useobject{currentmarker}{}%
\end{pgfscope}%
\begin{pgfscope}%
\pgfsys@transformshift{1.164327in}{1.548485in}%
\pgfsys@useobject{currentmarker}{}%
\end{pgfscope}%
\begin{pgfscope}%
\pgfsys@transformshift{1.164327in}{1.486048in}%
\pgfsys@useobject{currentmarker}{}%
\end{pgfscope}%
\begin{pgfscope}%
\pgfsys@transformshift{1.164327in}{1.488334in}%
\pgfsys@useobject{currentmarker}{}%
\end{pgfscope}%
\begin{pgfscope}%
\pgfsys@transformshift{1.164327in}{1.498568in}%
\pgfsys@useobject{currentmarker}{}%
\end{pgfscope}%
\begin{pgfscope}%
\pgfsys@transformshift{1.164327in}{1.554637in}%
\pgfsys@useobject{currentmarker}{}%
\end{pgfscope}%
\begin{pgfscope}%
\pgfsys@transformshift{1.164327in}{1.495580in}%
\pgfsys@useobject{currentmarker}{}%
\end{pgfscope}%
\begin{pgfscope}%
\pgfsys@transformshift{1.164327in}{1.492588in}%
\pgfsys@useobject{currentmarker}{}%
\end{pgfscope}%
\begin{pgfscope}%
\pgfsys@transformshift{1.164327in}{1.494480in}%
\pgfsys@useobject{currentmarker}{}%
\end{pgfscope}%
\begin{pgfscope}%
\pgfsys@transformshift{1.164327in}{1.489767in}%
\pgfsys@useobject{currentmarker}{}%
\end{pgfscope}%
\begin{pgfscope}%
\pgfsys@transformshift{1.164327in}{1.488798in}%
\pgfsys@useobject{currentmarker}{}%
\end{pgfscope}%
\begin{pgfscope}%
\pgfsys@transformshift{1.164327in}{1.486318in}%
\pgfsys@useobject{currentmarker}{}%
\end{pgfscope}%
\begin{pgfscope}%
\pgfsys@transformshift{1.164327in}{1.489076in}%
\pgfsys@useobject{currentmarker}{}%
\end{pgfscope}%
\begin{pgfscope}%
\pgfsys@transformshift{1.164327in}{1.489024in}%
\pgfsys@useobject{currentmarker}{}%
\end{pgfscope}%
\begin{pgfscope}%
\pgfsys@transformshift{1.164327in}{1.492790in}%
\pgfsys@useobject{currentmarker}{}%
\end{pgfscope}%
\begin{pgfscope}%
\pgfsys@transformshift{1.164327in}{1.488617in}%
\pgfsys@useobject{currentmarker}{}%
\end{pgfscope}%
\begin{pgfscope}%
\pgfsys@transformshift{1.164327in}{1.497334in}%
\pgfsys@useobject{currentmarker}{}%
\end{pgfscope}%
\begin{pgfscope}%
\pgfsys@transformshift{1.164327in}{1.494497in}%
\pgfsys@useobject{currentmarker}{}%
\end{pgfscope}%
\begin{pgfscope}%
\pgfsys@transformshift{1.164327in}{1.576950in}%
\pgfsys@useobject{currentmarker}{}%
\end{pgfscope}%
\begin{pgfscope}%
\pgfsys@transformshift{1.164327in}{1.546022in}%
\pgfsys@useobject{currentmarker}{}%
\end{pgfscope}%
\begin{pgfscope}%
\pgfsys@transformshift{1.164327in}{1.488445in}%
\pgfsys@useobject{currentmarker}{}%
\end{pgfscope}%
\begin{pgfscope}%
\pgfsys@transformshift{1.164327in}{1.510983in}%
\pgfsys@useobject{currentmarker}{}%
\end{pgfscope}%
\begin{pgfscope}%
\pgfsys@transformshift{1.164327in}{1.498685in}%
\pgfsys@useobject{currentmarker}{}%
\end{pgfscope}%
\begin{pgfscope}%
\pgfsys@transformshift{1.164327in}{1.531788in}%
\pgfsys@useobject{currentmarker}{}%
\end{pgfscope}%
\begin{pgfscope}%
\pgfsys@transformshift{1.164327in}{1.540769in}%
\pgfsys@useobject{currentmarker}{}%
\end{pgfscope}%
\begin{pgfscope}%
\pgfsys@transformshift{1.164327in}{1.520845in}%
\pgfsys@useobject{currentmarker}{}%
\end{pgfscope}%
\begin{pgfscope}%
\pgfsys@transformshift{1.164327in}{1.532470in}%
\pgfsys@useobject{currentmarker}{}%
\end{pgfscope}%
\begin{pgfscope}%
\pgfsys@transformshift{1.164327in}{1.485806in}%
\pgfsys@useobject{currentmarker}{}%
\end{pgfscope}%
\begin{pgfscope}%
\pgfsys@transformshift{1.164327in}{1.499891in}%
\pgfsys@useobject{currentmarker}{}%
\end{pgfscope}%
\begin{pgfscope}%
\pgfsys@transformshift{1.164327in}{1.554734in}%
\pgfsys@useobject{currentmarker}{}%
\end{pgfscope}%
\begin{pgfscope}%
\pgfsys@transformshift{1.164327in}{1.519351in}%
\pgfsys@useobject{currentmarker}{}%
\end{pgfscope}%
\begin{pgfscope}%
\pgfsys@transformshift{1.164327in}{1.567996in}%
\pgfsys@useobject{currentmarker}{}%
\end{pgfscope}%
\begin{pgfscope}%
\pgfsys@transformshift{1.164327in}{1.536132in}%
\pgfsys@useobject{currentmarker}{}%
\end{pgfscope}%
\begin{pgfscope}%
\pgfsys@transformshift{1.164327in}{1.572459in}%
\pgfsys@useobject{currentmarker}{}%
\end{pgfscope}%
\begin{pgfscope}%
\pgfsys@transformshift{1.164327in}{1.581195in}%
\pgfsys@useobject{currentmarker}{}%
\end{pgfscope}%
\begin{pgfscope}%
\pgfsys@transformshift{1.164327in}{1.528505in}%
\pgfsys@useobject{currentmarker}{}%
\end{pgfscope}%
\begin{pgfscope}%
\pgfsys@transformshift{1.164327in}{1.662866in}%
\pgfsys@useobject{currentmarker}{}%
\end{pgfscope}%
\begin{pgfscope}%
\pgfsys@transformshift{1.164327in}{1.522998in}%
\pgfsys@useobject{currentmarker}{}%
\end{pgfscope}%
\begin{pgfscope}%
\pgfsys@transformshift{1.164327in}{1.592855in}%
\pgfsys@useobject{currentmarker}{}%
\end{pgfscope}%
\begin{pgfscope}%
\pgfsys@transformshift{1.164327in}{1.528082in}%
\pgfsys@useobject{currentmarker}{}%
\end{pgfscope}%
\begin{pgfscope}%
\pgfsys@transformshift{1.164327in}{1.571095in}%
\pgfsys@useobject{currentmarker}{}%
\end{pgfscope}%
\begin{pgfscope}%
\pgfsys@transformshift{1.164327in}{1.552828in}%
\pgfsys@useobject{currentmarker}{}%
\end{pgfscope}%
\begin{pgfscope}%
\pgfsys@transformshift{1.164327in}{1.503254in}%
\pgfsys@useobject{currentmarker}{}%
\end{pgfscope}%
\begin{pgfscope}%
\pgfsys@transformshift{1.164327in}{1.494768in}%
\pgfsys@useobject{currentmarker}{}%
\end{pgfscope}%
\begin{pgfscope}%
\pgfsys@transformshift{1.164327in}{1.536177in}%
\pgfsys@useobject{currentmarker}{}%
\end{pgfscope}%
\begin{pgfscope}%
\pgfsys@transformshift{1.164327in}{1.489372in}%
\pgfsys@useobject{currentmarker}{}%
\end{pgfscope}%
\begin{pgfscope}%
\pgfsys@transformshift{1.164327in}{1.521074in}%
\pgfsys@useobject{currentmarker}{}%
\end{pgfscope}%
\begin{pgfscope}%
\pgfsys@transformshift{1.164327in}{1.530579in}%
\pgfsys@useobject{currentmarker}{}%
\end{pgfscope}%
\begin{pgfscope}%
\pgfsys@transformshift{1.164327in}{1.546616in}%
\pgfsys@useobject{currentmarker}{}%
\end{pgfscope}%
\begin{pgfscope}%
\pgfsys@transformshift{1.164327in}{1.489620in}%
\pgfsys@useobject{currentmarker}{}%
\end{pgfscope}%
\begin{pgfscope}%
\pgfsys@transformshift{1.164327in}{1.532325in}%
\pgfsys@useobject{currentmarker}{}%
\end{pgfscope}%
\begin{pgfscope}%
\pgfsys@transformshift{1.164327in}{1.581892in}%
\pgfsys@useobject{currentmarker}{}%
\end{pgfscope}%
\begin{pgfscope}%
\pgfsys@transformshift{1.164327in}{1.565539in}%
\pgfsys@useobject{currentmarker}{}%
\end{pgfscope}%
\begin{pgfscope}%
\pgfsys@transformshift{1.164327in}{1.524634in}%
\pgfsys@useobject{currentmarker}{}%
\end{pgfscope}%
\begin{pgfscope}%
\pgfsys@transformshift{1.164327in}{1.495864in}%
\pgfsys@useobject{currentmarker}{}%
\end{pgfscope}%
\begin{pgfscope}%
\pgfsys@transformshift{1.164327in}{1.495500in}%
\pgfsys@useobject{currentmarker}{}%
\end{pgfscope}%
\begin{pgfscope}%
\pgfsys@transformshift{1.164327in}{1.510137in}%
\pgfsys@useobject{currentmarker}{}%
\end{pgfscope}%
\begin{pgfscope}%
\pgfsys@transformshift{1.164327in}{1.508538in}%
\pgfsys@useobject{currentmarker}{}%
\end{pgfscope}%
\begin{pgfscope}%
\pgfsys@transformshift{1.164327in}{1.597739in}%
\pgfsys@useobject{currentmarker}{}%
\end{pgfscope}%
\begin{pgfscope}%
\pgfsys@transformshift{1.164327in}{1.520236in}%
\pgfsys@useobject{currentmarker}{}%
\end{pgfscope}%
\begin{pgfscope}%
\pgfsys@transformshift{1.164327in}{1.499606in}%
\pgfsys@useobject{currentmarker}{}%
\end{pgfscope}%
\begin{pgfscope}%
\pgfsys@transformshift{1.164327in}{1.512292in}%
\pgfsys@useobject{currentmarker}{}%
\end{pgfscope}%
\begin{pgfscope}%
\pgfsys@transformshift{1.164327in}{1.513841in}%
\pgfsys@useobject{currentmarker}{}%
\end{pgfscope}%
\begin{pgfscope}%
\pgfsys@transformshift{1.164327in}{1.547232in}%
\pgfsys@useobject{currentmarker}{}%
\end{pgfscope}%
\begin{pgfscope}%
\pgfsys@transformshift{1.164327in}{1.632630in}%
\pgfsys@useobject{currentmarker}{}%
\end{pgfscope}%
\begin{pgfscope}%
\pgfsys@transformshift{1.164327in}{1.613927in}%
\pgfsys@useobject{currentmarker}{}%
\end{pgfscope}%
\begin{pgfscope}%
\pgfsys@transformshift{1.164327in}{1.500265in}%
\pgfsys@useobject{currentmarker}{}%
\end{pgfscope}%
\begin{pgfscope}%
\pgfsys@transformshift{1.164327in}{1.491755in}%
\pgfsys@useobject{currentmarker}{}%
\end{pgfscope}%
\begin{pgfscope}%
\pgfsys@transformshift{1.164327in}{1.567327in}%
\pgfsys@useobject{currentmarker}{}%
\end{pgfscope}%
\begin{pgfscope}%
\pgfsys@transformshift{1.164327in}{1.521279in}%
\pgfsys@useobject{currentmarker}{}%
\end{pgfscope}%
\begin{pgfscope}%
\pgfsys@transformshift{1.164327in}{1.522775in}%
\pgfsys@useobject{currentmarker}{}%
\end{pgfscope}%
\begin{pgfscope}%
\pgfsys@transformshift{1.164327in}{1.496814in}%
\pgfsys@useobject{currentmarker}{}%
\end{pgfscope}%
\begin{pgfscope}%
\pgfsys@transformshift{1.164327in}{1.538125in}%
\pgfsys@useobject{currentmarker}{}%
\end{pgfscope}%
\begin{pgfscope}%
\pgfsys@transformshift{1.164327in}{1.487574in}%
\pgfsys@useobject{currentmarker}{}%
\end{pgfscope}%
\begin{pgfscope}%
\pgfsys@transformshift{1.164327in}{1.532706in}%
\pgfsys@useobject{currentmarker}{}%
\end{pgfscope}%
\begin{pgfscope}%
\pgfsys@transformshift{1.164327in}{1.492986in}%
\pgfsys@useobject{currentmarker}{}%
\end{pgfscope}%
\begin{pgfscope}%
\pgfsys@transformshift{1.164327in}{1.500545in}%
\pgfsys@useobject{currentmarker}{}%
\end{pgfscope}%
\begin{pgfscope}%
\pgfsys@transformshift{1.164327in}{1.537477in}%
\pgfsys@useobject{currentmarker}{}%
\end{pgfscope}%
\begin{pgfscope}%
\pgfsys@transformshift{1.164327in}{1.512641in}%
\pgfsys@useobject{currentmarker}{}%
\end{pgfscope}%
\begin{pgfscope}%
\pgfsys@transformshift{1.164327in}{1.605886in}%
\pgfsys@useobject{currentmarker}{}%
\end{pgfscope}%
\begin{pgfscope}%
\pgfsys@transformshift{1.164327in}{1.486823in}%
\pgfsys@useobject{currentmarker}{}%
\end{pgfscope}%
\begin{pgfscope}%
\pgfsys@transformshift{1.164327in}{1.488341in}%
\pgfsys@useobject{currentmarker}{}%
\end{pgfscope}%
\begin{pgfscope}%
\pgfsys@transformshift{1.164327in}{1.552008in}%
\pgfsys@useobject{currentmarker}{}%
\end{pgfscope}%
\begin{pgfscope}%
\pgfsys@transformshift{1.164327in}{1.528314in}%
\pgfsys@useobject{currentmarker}{}%
\end{pgfscope}%
\begin{pgfscope}%
\pgfsys@transformshift{1.164327in}{1.627534in}%
\pgfsys@useobject{currentmarker}{}%
\end{pgfscope}%
\begin{pgfscope}%
\pgfsys@transformshift{1.164327in}{1.504914in}%
\pgfsys@useobject{currentmarker}{}%
\end{pgfscope}%
\begin{pgfscope}%
\pgfsys@transformshift{1.164327in}{1.532956in}%
\pgfsys@useobject{currentmarker}{}%
\end{pgfscope}%
\begin{pgfscope}%
\pgfsys@transformshift{1.164327in}{1.569491in}%
\pgfsys@useobject{currentmarker}{}%
\end{pgfscope}%
\begin{pgfscope}%
\pgfsys@transformshift{1.164327in}{1.692694in}%
\pgfsys@useobject{currentmarker}{}%
\end{pgfscope}%
\begin{pgfscope}%
\pgfsys@transformshift{1.164327in}{1.511691in}%
\pgfsys@useobject{currentmarker}{}%
\end{pgfscope}%
\end{pgfscope}%
\begin{pgfscope}%
\pgfpathrectangle{\pgfqpoint{0.617715in}{1.333988in}}{\pgfqpoint{5.466114in}{3.078389in}}%
\pgfusepath{clip}%
\pgfsetrectcap%
\pgfsetroundjoin%
\pgfsetlinewidth{1.003750pt}%
\definecolor{currentstroke}{rgb}{0.000000,0.000000,0.000000}%
\pgfsetstrokecolor{currentstroke}%
\pgfsetdash{}{0pt}%
\pgfpathmoveto{\pgfqpoint{1.437632in}{1.526351in}}%
\pgfpathlineto{\pgfqpoint{1.619836in}{1.526351in}}%
\pgfpathlineto{\pgfqpoint{1.619836in}{1.701969in}}%
\pgfpathlineto{\pgfqpoint{1.437632in}{1.701969in}}%
\pgfpathlineto{\pgfqpoint{1.437632in}{1.526351in}}%
\pgfusepath{stroke}%
\end{pgfscope}%
\begin{pgfscope}%
\pgfpathrectangle{\pgfqpoint{0.617715in}{1.333988in}}{\pgfqpoint{5.466114in}{3.078389in}}%
\pgfusepath{clip}%
\pgfsetrectcap%
\pgfsetroundjoin%
\pgfsetlinewidth{1.003750pt}%
\definecolor{currentstroke}{rgb}{0.000000,0.000000,0.000000}%
\pgfsetstrokecolor{currentstroke}%
\pgfsetdash{}{0pt}%
\pgfpathmoveto{\pgfqpoint{1.528734in}{1.526351in}}%
\pgfpathlineto{\pgfqpoint{1.528734in}{1.473972in}}%
\pgfusepath{stroke}%
\end{pgfscope}%
\begin{pgfscope}%
\pgfpathrectangle{\pgfqpoint{0.617715in}{1.333988in}}{\pgfqpoint{5.466114in}{3.078389in}}%
\pgfusepath{clip}%
\pgfsetrectcap%
\pgfsetroundjoin%
\pgfsetlinewidth{1.003750pt}%
\definecolor{currentstroke}{rgb}{0.000000,0.000000,0.000000}%
\pgfsetstrokecolor{currentstroke}%
\pgfsetdash{}{0pt}%
\pgfpathmoveto{\pgfqpoint{1.528734in}{1.701969in}}%
\pgfpathlineto{\pgfqpoint{1.528734in}{1.957486in}}%
\pgfusepath{stroke}%
\end{pgfscope}%
\begin{pgfscope}%
\pgfpathrectangle{\pgfqpoint{0.617715in}{1.333988in}}{\pgfqpoint{5.466114in}{3.078389in}}%
\pgfusepath{clip}%
\pgfsetrectcap%
\pgfsetroundjoin%
\pgfsetlinewidth{1.003750pt}%
\definecolor{currentstroke}{rgb}{0.000000,0.000000,0.000000}%
\pgfsetstrokecolor{currentstroke}%
\pgfsetdash{}{0pt}%
\pgfpathmoveto{\pgfqpoint{1.483183in}{1.473972in}}%
\pgfpathlineto{\pgfqpoint{1.574285in}{1.473972in}}%
\pgfusepath{stroke}%
\end{pgfscope}%
\begin{pgfscope}%
\pgfpathrectangle{\pgfqpoint{0.617715in}{1.333988in}}{\pgfqpoint{5.466114in}{3.078389in}}%
\pgfusepath{clip}%
\pgfsetrectcap%
\pgfsetroundjoin%
\pgfsetlinewidth{1.003750pt}%
\definecolor{currentstroke}{rgb}{0.000000,0.000000,0.000000}%
\pgfsetstrokecolor{currentstroke}%
\pgfsetdash{}{0pt}%
\pgfpathmoveto{\pgfqpoint{1.483183in}{1.957486in}}%
\pgfpathlineto{\pgfqpoint{1.574285in}{1.957486in}}%
\pgfusepath{stroke}%
\end{pgfscope}%
\begin{pgfscope}%
\pgfpathrectangle{\pgfqpoint{0.617715in}{1.333988in}}{\pgfqpoint{5.466114in}{3.078389in}}%
\pgfusepath{clip}%
\pgfsetbuttcap%
\pgfsetroundjoin%
\definecolor{currentfill}{rgb}{0.000000,0.000000,0.000000}%
\pgfsetfillcolor{currentfill}%
\pgfsetfillopacity{0.000000}%
\pgfsetlinewidth{1.003750pt}%
\definecolor{currentstroke}{rgb}{0.000000,0.000000,0.000000}%
\pgfsetstrokecolor{currentstroke}%
\pgfsetdash{}{0pt}%
\pgfsys@defobject{currentmarker}{\pgfqpoint{-0.041667in}{-0.041667in}}{\pgfqpoint{0.041667in}{0.041667in}}{%
\pgfpathmoveto{\pgfqpoint{0.000000in}{-0.041667in}}%
\pgfpathcurveto{\pgfqpoint{0.011050in}{-0.041667in}}{\pgfqpoint{0.021649in}{-0.037276in}}{\pgfqpoint{0.029463in}{-0.029463in}}%
\pgfpathcurveto{\pgfqpoint{0.037276in}{-0.021649in}}{\pgfqpoint{0.041667in}{-0.011050in}}{\pgfqpoint{0.041667in}{0.000000in}}%
\pgfpathcurveto{\pgfqpoint{0.041667in}{0.011050in}}{\pgfqpoint{0.037276in}{0.021649in}}{\pgfqpoint{0.029463in}{0.029463in}}%
\pgfpathcurveto{\pgfqpoint{0.021649in}{0.037276in}}{\pgfqpoint{0.011050in}{0.041667in}}{\pgfqpoint{0.000000in}{0.041667in}}%
\pgfpathcurveto{\pgfqpoint{-0.011050in}{0.041667in}}{\pgfqpoint{-0.021649in}{0.037276in}}{\pgfqpoint{-0.029463in}{0.029463in}}%
\pgfpathcurveto{\pgfqpoint{-0.037276in}{0.021649in}}{\pgfqpoint{-0.041667in}{0.011050in}}{\pgfqpoint{-0.041667in}{0.000000in}}%
\pgfpathcurveto{\pgfqpoint{-0.041667in}{-0.011050in}}{\pgfqpoint{-0.037276in}{-0.021649in}}{\pgfqpoint{-0.029463in}{-0.029463in}}%
\pgfpathcurveto{\pgfqpoint{-0.021649in}{-0.037276in}}{\pgfqpoint{-0.011050in}{-0.041667in}}{\pgfqpoint{0.000000in}{-0.041667in}}%
\pgfpathlineto{\pgfqpoint{0.000000in}{-0.041667in}}%
\pgfpathclose%
\pgfusepath{stroke,fill}%
}%
\begin{pgfscope}%
\pgfsys@transformshift{1.528734in}{2.347299in}%
\pgfsys@useobject{currentmarker}{}%
\end{pgfscope}%
\begin{pgfscope}%
\pgfsys@transformshift{1.528734in}{2.144552in}%
\pgfsys@useobject{currentmarker}{}%
\end{pgfscope}%
\begin{pgfscope}%
\pgfsys@transformshift{1.528734in}{2.392307in}%
\pgfsys@useobject{currentmarker}{}%
\end{pgfscope}%
\begin{pgfscope}%
\pgfsys@transformshift{1.528734in}{2.309455in}%
\pgfsys@useobject{currentmarker}{}%
\end{pgfscope}%
\begin{pgfscope}%
\pgfsys@transformshift{1.528734in}{2.924373in}%
\pgfsys@useobject{currentmarker}{}%
\end{pgfscope}%
\begin{pgfscope}%
\pgfsys@transformshift{1.528734in}{1.989169in}%
\pgfsys@useobject{currentmarker}{}%
\end{pgfscope}%
\end{pgfscope}%
\begin{pgfscope}%
\pgfpathrectangle{\pgfqpoint{0.617715in}{1.333988in}}{\pgfqpoint{5.466114in}{3.078389in}}%
\pgfusepath{clip}%
\pgfsetrectcap%
\pgfsetroundjoin%
\pgfsetlinewidth{1.003750pt}%
\definecolor{currentstroke}{rgb}{0.000000,0.000000,0.000000}%
\pgfsetstrokecolor{currentstroke}%
\pgfsetdash{}{0pt}%
\pgfpathmoveto{\pgfqpoint{1.802040in}{1.674076in}}%
\pgfpathlineto{\pgfqpoint{1.984244in}{1.674076in}}%
\pgfpathlineto{\pgfqpoint{1.984244in}{1.674076in}}%
\pgfpathlineto{\pgfqpoint{1.802040in}{1.674076in}}%
\pgfpathlineto{\pgfqpoint{1.802040in}{1.674076in}}%
\pgfusepath{stroke}%
\end{pgfscope}%
\begin{pgfscope}%
\pgfpathrectangle{\pgfqpoint{0.617715in}{1.333988in}}{\pgfqpoint{5.466114in}{3.078389in}}%
\pgfusepath{clip}%
\pgfsetrectcap%
\pgfsetroundjoin%
\pgfsetlinewidth{1.003750pt}%
\definecolor{currentstroke}{rgb}{0.000000,0.000000,0.000000}%
\pgfsetstrokecolor{currentstroke}%
\pgfsetdash{}{0pt}%
\pgfpathmoveto{\pgfqpoint{1.893142in}{1.674076in}}%
\pgfpathlineto{\pgfqpoint{1.893142in}{1.674076in}}%
\pgfusepath{stroke}%
\end{pgfscope}%
\begin{pgfscope}%
\pgfpathrectangle{\pgfqpoint{0.617715in}{1.333988in}}{\pgfqpoint{5.466114in}{3.078389in}}%
\pgfusepath{clip}%
\pgfsetrectcap%
\pgfsetroundjoin%
\pgfsetlinewidth{1.003750pt}%
\definecolor{currentstroke}{rgb}{0.000000,0.000000,0.000000}%
\pgfsetstrokecolor{currentstroke}%
\pgfsetdash{}{0pt}%
\pgfpathmoveto{\pgfqpoint{1.893142in}{1.674076in}}%
\pgfpathlineto{\pgfqpoint{1.893142in}{1.674076in}}%
\pgfusepath{stroke}%
\end{pgfscope}%
\begin{pgfscope}%
\pgfpathrectangle{\pgfqpoint{0.617715in}{1.333988in}}{\pgfqpoint{5.466114in}{3.078389in}}%
\pgfusepath{clip}%
\pgfsetrectcap%
\pgfsetroundjoin%
\pgfsetlinewidth{1.003750pt}%
\definecolor{currentstroke}{rgb}{0.000000,0.000000,0.000000}%
\pgfsetstrokecolor{currentstroke}%
\pgfsetdash{}{0pt}%
\pgfpathmoveto{\pgfqpoint{1.847591in}{1.674076in}}%
\pgfpathlineto{\pgfqpoint{1.938693in}{1.674076in}}%
\pgfusepath{stroke}%
\end{pgfscope}%
\begin{pgfscope}%
\pgfpathrectangle{\pgfqpoint{0.617715in}{1.333988in}}{\pgfqpoint{5.466114in}{3.078389in}}%
\pgfusepath{clip}%
\pgfsetrectcap%
\pgfsetroundjoin%
\pgfsetlinewidth{1.003750pt}%
\definecolor{currentstroke}{rgb}{0.000000,0.000000,0.000000}%
\pgfsetstrokecolor{currentstroke}%
\pgfsetdash{}{0pt}%
\pgfpathmoveto{\pgfqpoint{1.847591in}{1.674076in}}%
\pgfpathlineto{\pgfqpoint{1.938693in}{1.674076in}}%
\pgfusepath{stroke}%
\end{pgfscope}%
\begin{pgfscope}%
\pgfpathrectangle{\pgfqpoint{0.617715in}{1.333988in}}{\pgfqpoint{5.466114in}{3.078389in}}%
\pgfusepath{clip}%
\pgfsetrectcap%
\pgfsetroundjoin%
\pgfsetlinewidth{1.003750pt}%
\definecolor{currentstroke}{rgb}{0.000000,0.000000,0.000000}%
\pgfsetstrokecolor{currentstroke}%
\pgfsetdash{}{0pt}%
\pgfpathmoveto{\pgfqpoint{2.166447in}{1.693582in}}%
\pgfpathlineto{\pgfqpoint{2.348651in}{1.693582in}}%
\pgfpathlineto{\pgfqpoint{2.348651in}{1.993710in}}%
\pgfpathlineto{\pgfqpoint{2.166447in}{1.993710in}}%
\pgfpathlineto{\pgfqpoint{2.166447in}{1.693582in}}%
\pgfusepath{stroke}%
\end{pgfscope}%
\begin{pgfscope}%
\pgfpathrectangle{\pgfqpoint{0.617715in}{1.333988in}}{\pgfqpoint{5.466114in}{3.078389in}}%
\pgfusepath{clip}%
\pgfsetrectcap%
\pgfsetroundjoin%
\pgfsetlinewidth{1.003750pt}%
\definecolor{currentstroke}{rgb}{0.000000,0.000000,0.000000}%
\pgfsetstrokecolor{currentstroke}%
\pgfsetdash{}{0pt}%
\pgfpathmoveto{\pgfqpoint{2.257549in}{1.693582in}}%
\pgfpathlineto{\pgfqpoint{2.257549in}{1.482544in}}%
\pgfusepath{stroke}%
\end{pgfscope}%
\begin{pgfscope}%
\pgfpathrectangle{\pgfqpoint{0.617715in}{1.333988in}}{\pgfqpoint{5.466114in}{3.078389in}}%
\pgfusepath{clip}%
\pgfsetrectcap%
\pgfsetroundjoin%
\pgfsetlinewidth{1.003750pt}%
\definecolor{currentstroke}{rgb}{0.000000,0.000000,0.000000}%
\pgfsetstrokecolor{currentstroke}%
\pgfsetdash{}{0pt}%
\pgfpathmoveto{\pgfqpoint{2.257549in}{1.993710in}}%
\pgfpathlineto{\pgfqpoint{2.257549in}{2.296871in}}%
\pgfusepath{stroke}%
\end{pgfscope}%
\begin{pgfscope}%
\pgfpathrectangle{\pgfqpoint{0.617715in}{1.333988in}}{\pgfqpoint{5.466114in}{3.078389in}}%
\pgfusepath{clip}%
\pgfsetrectcap%
\pgfsetroundjoin%
\pgfsetlinewidth{1.003750pt}%
\definecolor{currentstroke}{rgb}{0.000000,0.000000,0.000000}%
\pgfsetstrokecolor{currentstroke}%
\pgfsetdash{}{0pt}%
\pgfpathmoveto{\pgfqpoint{2.211998in}{1.482544in}}%
\pgfpathlineto{\pgfqpoint{2.303100in}{1.482544in}}%
\pgfusepath{stroke}%
\end{pgfscope}%
\begin{pgfscope}%
\pgfpathrectangle{\pgfqpoint{0.617715in}{1.333988in}}{\pgfqpoint{5.466114in}{3.078389in}}%
\pgfusepath{clip}%
\pgfsetrectcap%
\pgfsetroundjoin%
\pgfsetlinewidth{1.003750pt}%
\definecolor{currentstroke}{rgb}{0.000000,0.000000,0.000000}%
\pgfsetstrokecolor{currentstroke}%
\pgfsetdash{}{0pt}%
\pgfpathmoveto{\pgfqpoint{2.211998in}{2.296871in}}%
\pgfpathlineto{\pgfqpoint{2.303100in}{2.296871in}}%
\pgfusepath{stroke}%
\end{pgfscope}%
\begin{pgfscope}%
\pgfpathrectangle{\pgfqpoint{0.617715in}{1.333988in}}{\pgfqpoint{5.466114in}{3.078389in}}%
\pgfusepath{clip}%
\pgfsetbuttcap%
\pgfsetroundjoin%
\definecolor{currentfill}{rgb}{0.000000,0.000000,0.000000}%
\pgfsetfillcolor{currentfill}%
\pgfsetfillopacity{0.000000}%
\pgfsetlinewidth{1.003750pt}%
\definecolor{currentstroke}{rgb}{0.000000,0.000000,0.000000}%
\pgfsetstrokecolor{currentstroke}%
\pgfsetdash{}{0pt}%
\pgfsys@defobject{currentmarker}{\pgfqpoint{-0.041667in}{-0.041667in}}{\pgfqpoint{0.041667in}{0.041667in}}{%
\pgfpathmoveto{\pgfqpoint{0.000000in}{-0.041667in}}%
\pgfpathcurveto{\pgfqpoint{0.011050in}{-0.041667in}}{\pgfqpoint{0.021649in}{-0.037276in}}{\pgfqpoint{0.029463in}{-0.029463in}}%
\pgfpathcurveto{\pgfqpoint{0.037276in}{-0.021649in}}{\pgfqpoint{0.041667in}{-0.011050in}}{\pgfqpoint{0.041667in}{0.000000in}}%
\pgfpathcurveto{\pgfqpoint{0.041667in}{0.011050in}}{\pgfqpoint{0.037276in}{0.021649in}}{\pgfqpoint{0.029463in}{0.029463in}}%
\pgfpathcurveto{\pgfqpoint{0.021649in}{0.037276in}}{\pgfqpoint{0.011050in}{0.041667in}}{\pgfqpoint{0.000000in}{0.041667in}}%
\pgfpathcurveto{\pgfqpoint{-0.011050in}{0.041667in}}{\pgfqpoint{-0.021649in}{0.037276in}}{\pgfqpoint{-0.029463in}{0.029463in}}%
\pgfpathcurveto{\pgfqpoint{-0.037276in}{0.021649in}}{\pgfqpoint{-0.041667in}{0.011050in}}{\pgfqpoint{-0.041667in}{0.000000in}}%
\pgfpathcurveto{\pgfqpoint{-0.041667in}{-0.011050in}}{\pgfqpoint{-0.037276in}{-0.021649in}}{\pgfqpoint{-0.029463in}{-0.029463in}}%
\pgfpathcurveto{\pgfqpoint{-0.021649in}{-0.037276in}}{\pgfqpoint{-0.011050in}{-0.041667in}}{\pgfqpoint{0.000000in}{-0.041667in}}%
\pgfpathlineto{\pgfqpoint{0.000000in}{-0.041667in}}%
\pgfpathclose%
\pgfusepath{stroke,fill}%
}%
\begin{pgfscope}%
\pgfsys@transformshift{2.257549in}{3.072152in}%
\pgfsys@useobject{currentmarker}{}%
\end{pgfscope}%
\begin{pgfscope}%
\pgfsys@transformshift{2.257549in}{2.805991in}%
\pgfsys@useobject{currentmarker}{}%
\end{pgfscope}%
\begin{pgfscope}%
\pgfsys@transformshift{2.257549in}{3.076014in}%
\pgfsys@useobject{currentmarker}{}%
\end{pgfscope}%
\begin{pgfscope}%
\pgfsys@transformshift{2.257549in}{3.870457in}%
\pgfsys@useobject{currentmarker}{}%
\end{pgfscope}%
\begin{pgfscope}%
\pgfsys@transformshift{2.257549in}{3.084494in}%
\pgfsys@useobject{currentmarker}{}%
\end{pgfscope}%
\begin{pgfscope}%
\pgfsys@transformshift{2.257549in}{3.875493in}%
\pgfsys@useobject{currentmarker}{}%
\end{pgfscope}%
\begin{pgfscope}%
\pgfsys@transformshift{2.257549in}{3.877716in}%
\pgfsys@useobject{currentmarker}{}%
\end{pgfscope}%
\begin{pgfscope}%
\pgfsys@transformshift{2.257549in}{3.092625in}%
\pgfsys@useobject{currentmarker}{}%
\end{pgfscope}%
\begin{pgfscope}%
\pgfsys@transformshift{2.257549in}{2.815009in}%
\pgfsys@useobject{currentmarker}{}%
\end{pgfscope}%
\begin{pgfscope}%
\pgfsys@transformshift{2.257549in}{2.814260in}%
\pgfsys@useobject{currentmarker}{}%
\end{pgfscope}%
\begin{pgfscope}%
\pgfsys@transformshift{2.257549in}{2.814751in}%
\pgfsys@useobject{currentmarker}{}%
\end{pgfscope}%
\begin{pgfscope}%
\pgfsys@transformshift{2.257549in}{4.272450in}%
\pgfsys@useobject{currentmarker}{}%
\end{pgfscope}%
\begin{pgfscope}%
\pgfsys@transformshift{2.257549in}{2.497996in}%
\pgfsys@useobject{currentmarker}{}%
\end{pgfscope}%
\begin{pgfscope}%
\pgfsys@transformshift{2.257549in}{2.601124in}%
\pgfsys@useobject{currentmarker}{}%
\end{pgfscope}%
\end{pgfscope}%
\begin{pgfscope}%
\pgfpathrectangle{\pgfqpoint{0.617715in}{1.333988in}}{\pgfqpoint{5.466114in}{3.078389in}}%
\pgfusepath{clip}%
\pgfsetrectcap%
\pgfsetroundjoin%
\pgfsetlinewidth{1.003750pt}%
\definecolor{currentstroke}{rgb}{0.000000,0.000000,0.000000}%
\pgfsetstrokecolor{currentstroke}%
\pgfsetdash{}{0pt}%
\pgfpathmoveto{\pgfqpoint{2.530855in}{1.494806in}}%
\pgfpathlineto{\pgfqpoint{2.713059in}{1.494806in}}%
\pgfpathlineto{\pgfqpoint{2.713059in}{1.637629in}}%
\pgfpathlineto{\pgfqpoint{2.530855in}{1.637629in}}%
\pgfpathlineto{\pgfqpoint{2.530855in}{1.494806in}}%
\pgfusepath{stroke}%
\end{pgfscope}%
\begin{pgfscope}%
\pgfpathrectangle{\pgfqpoint{0.617715in}{1.333988in}}{\pgfqpoint{5.466114in}{3.078389in}}%
\pgfusepath{clip}%
\pgfsetrectcap%
\pgfsetroundjoin%
\pgfsetlinewidth{1.003750pt}%
\definecolor{currentstroke}{rgb}{0.000000,0.000000,0.000000}%
\pgfsetstrokecolor{currentstroke}%
\pgfsetdash{}{0pt}%
\pgfpathmoveto{\pgfqpoint{2.621957in}{1.494806in}}%
\pgfpathlineto{\pgfqpoint{2.621957in}{1.474121in}}%
\pgfusepath{stroke}%
\end{pgfscope}%
\begin{pgfscope}%
\pgfpathrectangle{\pgfqpoint{0.617715in}{1.333988in}}{\pgfqpoint{5.466114in}{3.078389in}}%
\pgfusepath{clip}%
\pgfsetrectcap%
\pgfsetroundjoin%
\pgfsetlinewidth{1.003750pt}%
\definecolor{currentstroke}{rgb}{0.000000,0.000000,0.000000}%
\pgfsetstrokecolor{currentstroke}%
\pgfsetdash{}{0pt}%
\pgfpathmoveto{\pgfqpoint{2.621957in}{1.637629in}}%
\pgfpathlineto{\pgfqpoint{2.621957in}{1.847127in}}%
\pgfusepath{stroke}%
\end{pgfscope}%
\begin{pgfscope}%
\pgfpathrectangle{\pgfqpoint{0.617715in}{1.333988in}}{\pgfqpoint{5.466114in}{3.078389in}}%
\pgfusepath{clip}%
\pgfsetrectcap%
\pgfsetroundjoin%
\pgfsetlinewidth{1.003750pt}%
\definecolor{currentstroke}{rgb}{0.000000,0.000000,0.000000}%
\pgfsetstrokecolor{currentstroke}%
\pgfsetdash{}{0pt}%
\pgfpathmoveto{\pgfqpoint{2.576406in}{1.474121in}}%
\pgfpathlineto{\pgfqpoint{2.667508in}{1.474121in}}%
\pgfusepath{stroke}%
\end{pgfscope}%
\begin{pgfscope}%
\pgfpathrectangle{\pgfqpoint{0.617715in}{1.333988in}}{\pgfqpoint{5.466114in}{3.078389in}}%
\pgfusepath{clip}%
\pgfsetrectcap%
\pgfsetroundjoin%
\pgfsetlinewidth{1.003750pt}%
\definecolor{currentstroke}{rgb}{0.000000,0.000000,0.000000}%
\pgfsetstrokecolor{currentstroke}%
\pgfsetdash{}{0pt}%
\pgfpathmoveto{\pgfqpoint{2.576406in}{1.847127in}}%
\pgfpathlineto{\pgfqpoint{2.667508in}{1.847127in}}%
\pgfusepath{stroke}%
\end{pgfscope}%
\begin{pgfscope}%
\pgfpathrectangle{\pgfqpoint{0.617715in}{1.333988in}}{\pgfqpoint{5.466114in}{3.078389in}}%
\pgfusepath{clip}%
\pgfsetrectcap%
\pgfsetroundjoin%
\pgfsetlinewidth{1.003750pt}%
\definecolor{currentstroke}{rgb}{0.000000,0.000000,0.000000}%
\pgfsetstrokecolor{currentstroke}%
\pgfsetdash{}{0pt}%
\pgfpathmoveto{\pgfqpoint{2.895263in}{1.475563in}}%
\pgfpathlineto{\pgfqpoint{3.077466in}{1.475563in}}%
\pgfpathlineto{\pgfqpoint{3.077466in}{1.475563in}}%
\pgfpathlineto{\pgfqpoint{2.895263in}{1.475563in}}%
\pgfpathlineto{\pgfqpoint{2.895263in}{1.475563in}}%
\pgfusepath{stroke}%
\end{pgfscope}%
\begin{pgfscope}%
\pgfpathrectangle{\pgfqpoint{0.617715in}{1.333988in}}{\pgfqpoint{5.466114in}{3.078389in}}%
\pgfusepath{clip}%
\pgfsetrectcap%
\pgfsetroundjoin%
\pgfsetlinewidth{1.003750pt}%
\definecolor{currentstroke}{rgb}{0.000000,0.000000,0.000000}%
\pgfsetstrokecolor{currentstroke}%
\pgfsetdash{}{0pt}%
\pgfpathmoveto{\pgfqpoint{2.986364in}{1.475563in}}%
\pgfpathlineto{\pgfqpoint{2.986364in}{1.475563in}}%
\pgfusepath{stroke}%
\end{pgfscope}%
\begin{pgfscope}%
\pgfpathrectangle{\pgfqpoint{0.617715in}{1.333988in}}{\pgfqpoint{5.466114in}{3.078389in}}%
\pgfusepath{clip}%
\pgfsetrectcap%
\pgfsetroundjoin%
\pgfsetlinewidth{1.003750pt}%
\definecolor{currentstroke}{rgb}{0.000000,0.000000,0.000000}%
\pgfsetstrokecolor{currentstroke}%
\pgfsetdash{}{0pt}%
\pgfpathmoveto{\pgfqpoint{2.986364in}{1.475563in}}%
\pgfpathlineto{\pgfqpoint{2.986364in}{1.475563in}}%
\pgfusepath{stroke}%
\end{pgfscope}%
\begin{pgfscope}%
\pgfpathrectangle{\pgfqpoint{0.617715in}{1.333988in}}{\pgfqpoint{5.466114in}{3.078389in}}%
\pgfusepath{clip}%
\pgfsetrectcap%
\pgfsetroundjoin%
\pgfsetlinewidth{1.003750pt}%
\definecolor{currentstroke}{rgb}{0.000000,0.000000,0.000000}%
\pgfsetstrokecolor{currentstroke}%
\pgfsetdash{}{0pt}%
\pgfpathmoveto{\pgfqpoint{2.940814in}{1.475563in}}%
\pgfpathlineto{\pgfqpoint{3.031915in}{1.475563in}}%
\pgfusepath{stroke}%
\end{pgfscope}%
\begin{pgfscope}%
\pgfpathrectangle{\pgfqpoint{0.617715in}{1.333988in}}{\pgfqpoint{5.466114in}{3.078389in}}%
\pgfusepath{clip}%
\pgfsetrectcap%
\pgfsetroundjoin%
\pgfsetlinewidth{1.003750pt}%
\definecolor{currentstroke}{rgb}{0.000000,0.000000,0.000000}%
\pgfsetstrokecolor{currentstroke}%
\pgfsetdash{}{0pt}%
\pgfpathmoveto{\pgfqpoint{2.940814in}{1.475563in}}%
\pgfpathlineto{\pgfqpoint{3.031915in}{1.475563in}}%
\pgfusepath{stroke}%
\end{pgfscope}%
\begin{pgfscope}%
\pgfpathrectangle{\pgfqpoint{0.617715in}{1.333988in}}{\pgfqpoint{5.466114in}{3.078389in}}%
\pgfusepath{clip}%
\pgfsetrectcap%
\pgfsetroundjoin%
\pgfsetlinewidth{1.003750pt}%
\definecolor{currentstroke}{rgb}{0.000000,0.000000,0.000000}%
\pgfsetstrokecolor{currentstroke}%
\pgfsetdash{}{0pt}%
\pgfpathmoveto{\pgfqpoint{3.259670in}{1.475286in}}%
\pgfpathlineto{\pgfqpoint{3.441874in}{1.475286in}}%
\pgfpathlineto{\pgfqpoint{3.441874in}{1.500248in}}%
\pgfpathlineto{\pgfqpoint{3.259670in}{1.500248in}}%
\pgfpathlineto{\pgfqpoint{3.259670in}{1.475286in}}%
\pgfusepath{stroke}%
\end{pgfscope}%
\begin{pgfscope}%
\pgfpathrectangle{\pgfqpoint{0.617715in}{1.333988in}}{\pgfqpoint{5.466114in}{3.078389in}}%
\pgfusepath{clip}%
\pgfsetrectcap%
\pgfsetroundjoin%
\pgfsetlinewidth{1.003750pt}%
\definecolor{currentstroke}{rgb}{0.000000,0.000000,0.000000}%
\pgfsetstrokecolor{currentstroke}%
\pgfsetdash{}{0pt}%
\pgfpathmoveto{\pgfqpoint{3.350772in}{1.475286in}}%
\pgfpathlineto{\pgfqpoint{3.350772in}{1.473960in}}%
\pgfusepath{stroke}%
\end{pgfscope}%
\begin{pgfscope}%
\pgfpathrectangle{\pgfqpoint{0.617715in}{1.333988in}}{\pgfqpoint{5.466114in}{3.078389in}}%
\pgfusepath{clip}%
\pgfsetrectcap%
\pgfsetroundjoin%
\pgfsetlinewidth{1.003750pt}%
\definecolor{currentstroke}{rgb}{0.000000,0.000000,0.000000}%
\pgfsetstrokecolor{currentstroke}%
\pgfsetdash{}{0pt}%
\pgfpathmoveto{\pgfqpoint{3.350772in}{1.500248in}}%
\pgfpathlineto{\pgfqpoint{3.350772in}{1.537608in}}%
\pgfusepath{stroke}%
\end{pgfscope}%
\begin{pgfscope}%
\pgfpathrectangle{\pgfqpoint{0.617715in}{1.333988in}}{\pgfqpoint{5.466114in}{3.078389in}}%
\pgfusepath{clip}%
\pgfsetrectcap%
\pgfsetroundjoin%
\pgfsetlinewidth{1.003750pt}%
\definecolor{currentstroke}{rgb}{0.000000,0.000000,0.000000}%
\pgfsetstrokecolor{currentstroke}%
\pgfsetdash{}{0pt}%
\pgfpathmoveto{\pgfqpoint{3.305221in}{1.473960in}}%
\pgfpathlineto{\pgfqpoint{3.396323in}{1.473960in}}%
\pgfusepath{stroke}%
\end{pgfscope}%
\begin{pgfscope}%
\pgfpathrectangle{\pgfqpoint{0.617715in}{1.333988in}}{\pgfqpoint{5.466114in}{3.078389in}}%
\pgfusepath{clip}%
\pgfsetrectcap%
\pgfsetroundjoin%
\pgfsetlinewidth{1.003750pt}%
\definecolor{currentstroke}{rgb}{0.000000,0.000000,0.000000}%
\pgfsetstrokecolor{currentstroke}%
\pgfsetdash{}{0pt}%
\pgfpathmoveto{\pgfqpoint{3.305221in}{1.537608in}}%
\pgfpathlineto{\pgfqpoint{3.396323in}{1.537608in}}%
\pgfusepath{stroke}%
\end{pgfscope}%
\begin{pgfscope}%
\pgfpathrectangle{\pgfqpoint{0.617715in}{1.333988in}}{\pgfqpoint{5.466114in}{3.078389in}}%
\pgfusepath{clip}%
\pgfsetbuttcap%
\pgfsetroundjoin%
\definecolor{currentfill}{rgb}{0.000000,0.000000,0.000000}%
\pgfsetfillcolor{currentfill}%
\pgfsetfillopacity{0.000000}%
\pgfsetlinewidth{1.003750pt}%
\definecolor{currentstroke}{rgb}{0.000000,0.000000,0.000000}%
\pgfsetstrokecolor{currentstroke}%
\pgfsetdash{}{0pt}%
\pgfsys@defobject{currentmarker}{\pgfqpoint{-0.041667in}{-0.041667in}}{\pgfqpoint{0.041667in}{0.041667in}}{%
\pgfpathmoveto{\pgfqpoint{0.000000in}{-0.041667in}}%
\pgfpathcurveto{\pgfqpoint{0.011050in}{-0.041667in}}{\pgfqpoint{0.021649in}{-0.037276in}}{\pgfqpoint{0.029463in}{-0.029463in}}%
\pgfpathcurveto{\pgfqpoint{0.037276in}{-0.021649in}}{\pgfqpoint{0.041667in}{-0.011050in}}{\pgfqpoint{0.041667in}{0.000000in}}%
\pgfpathcurveto{\pgfqpoint{0.041667in}{0.011050in}}{\pgfqpoint{0.037276in}{0.021649in}}{\pgfqpoint{0.029463in}{0.029463in}}%
\pgfpathcurveto{\pgfqpoint{0.021649in}{0.037276in}}{\pgfqpoint{0.011050in}{0.041667in}}{\pgfqpoint{0.000000in}{0.041667in}}%
\pgfpathcurveto{\pgfqpoint{-0.011050in}{0.041667in}}{\pgfqpoint{-0.021649in}{0.037276in}}{\pgfqpoint{-0.029463in}{0.029463in}}%
\pgfpathcurveto{\pgfqpoint{-0.037276in}{0.021649in}}{\pgfqpoint{-0.041667in}{0.011050in}}{\pgfqpoint{-0.041667in}{0.000000in}}%
\pgfpathcurveto{\pgfqpoint{-0.041667in}{-0.011050in}}{\pgfqpoint{-0.037276in}{-0.021649in}}{\pgfqpoint{-0.029463in}{-0.029463in}}%
\pgfpathcurveto{\pgfqpoint{-0.021649in}{-0.037276in}}{\pgfqpoint{-0.011050in}{-0.041667in}}{\pgfqpoint{0.000000in}{-0.041667in}}%
\pgfpathlineto{\pgfqpoint{0.000000in}{-0.041667in}}%
\pgfpathclose%
\pgfusepath{stroke,fill}%
}%
\begin{pgfscope}%
\pgfsys@transformshift{3.350772in}{1.594749in}%
\pgfsys@useobject{currentmarker}{}%
\end{pgfscope}%
\begin{pgfscope}%
\pgfsys@transformshift{3.350772in}{1.628799in}%
\pgfsys@useobject{currentmarker}{}%
\end{pgfscope}%
\begin{pgfscope}%
\pgfsys@transformshift{3.350772in}{1.625728in}%
\pgfsys@useobject{currentmarker}{}%
\end{pgfscope}%
\begin{pgfscope}%
\pgfsys@transformshift{3.350772in}{1.722361in}%
\pgfsys@useobject{currentmarker}{}%
\end{pgfscope}%
\begin{pgfscope}%
\pgfsys@transformshift{3.350772in}{1.594435in}%
\pgfsys@useobject{currentmarker}{}%
\end{pgfscope}%
\begin{pgfscope}%
\pgfsys@transformshift{3.350772in}{1.567171in}%
\pgfsys@useobject{currentmarker}{}%
\end{pgfscope}%
\begin{pgfscope}%
\pgfsys@transformshift{3.350772in}{1.583334in}%
\pgfsys@useobject{currentmarker}{}%
\end{pgfscope}%
\begin{pgfscope}%
\pgfsys@transformshift{3.350772in}{1.568350in}%
\pgfsys@useobject{currentmarker}{}%
\end{pgfscope}%
\begin{pgfscope}%
\pgfsys@transformshift{3.350772in}{1.693638in}%
\pgfsys@useobject{currentmarker}{}%
\end{pgfscope}%
\begin{pgfscope}%
\pgfsys@transformshift{3.350772in}{1.541673in}%
\pgfsys@useobject{currentmarker}{}%
\end{pgfscope}%
\begin{pgfscope}%
\pgfsys@transformshift{3.350772in}{1.552101in}%
\pgfsys@useobject{currentmarker}{}%
\end{pgfscope}%
\begin{pgfscope}%
\pgfsys@transformshift{3.350772in}{1.542424in}%
\pgfsys@useobject{currentmarker}{}%
\end{pgfscope}%
\begin{pgfscope}%
\pgfsys@transformshift{3.350772in}{1.619913in}%
\pgfsys@useobject{currentmarker}{}%
\end{pgfscope}%
\begin{pgfscope}%
\pgfsys@transformshift{3.350772in}{1.544375in}%
\pgfsys@useobject{currentmarker}{}%
\end{pgfscope}%
\begin{pgfscope}%
\pgfsys@transformshift{3.350772in}{1.560521in}%
\pgfsys@useobject{currentmarker}{}%
\end{pgfscope}%
\begin{pgfscope}%
\pgfsys@transformshift{3.350772in}{1.571079in}%
\pgfsys@useobject{currentmarker}{}%
\end{pgfscope}%
\begin{pgfscope}%
\pgfsys@transformshift{3.350772in}{1.662871in}%
\pgfsys@useobject{currentmarker}{}%
\end{pgfscope}%
\begin{pgfscope}%
\pgfsys@transformshift{3.350772in}{1.722219in}%
\pgfsys@useobject{currentmarker}{}%
\end{pgfscope}%
\begin{pgfscope}%
\pgfsys@transformshift{3.350772in}{1.613831in}%
\pgfsys@useobject{currentmarker}{}%
\end{pgfscope}%
\begin{pgfscope}%
\pgfsys@transformshift{3.350772in}{1.699752in}%
\pgfsys@useobject{currentmarker}{}%
\end{pgfscope}%
\begin{pgfscope}%
\pgfsys@transformshift{3.350772in}{1.636432in}%
\pgfsys@useobject{currentmarker}{}%
\end{pgfscope}%
\begin{pgfscope}%
\pgfsys@transformshift{3.350772in}{1.846006in}%
\pgfsys@useobject{currentmarker}{}%
\end{pgfscope}%
\end{pgfscope}%
\begin{pgfscope}%
\pgfpathrectangle{\pgfqpoint{0.617715in}{1.333988in}}{\pgfqpoint{5.466114in}{3.078389in}}%
\pgfusepath{clip}%
\pgfsetrectcap%
\pgfsetroundjoin%
\pgfsetlinewidth{1.003750pt}%
\definecolor{currentstroke}{rgb}{0.000000,0.000000,0.000000}%
\pgfsetstrokecolor{currentstroke}%
\pgfsetdash{}{0pt}%
\pgfusepath{stroke}%
\end{pgfscope}%
\begin{pgfscope}%
\pgfpathrectangle{\pgfqpoint{0.617715in}{1.333988in}}{\pgfqpoint{5.466114in}{3.078389in}}%
\pgfusepath{clip}%
\pgfsetrectcap%
\pgfsetroundjoin%
\pgfsetlinewidth{1.003750pt}%
\definecolor{currentstroke}{rgb}{0.000000,0.000000,0.000000}%
\pgfsetstrokecolor{currentstroke}%
\pgfsetdash{}{0pt}%
\pgfusepath{stroke}%
\end{pgfscope}%
\begin{pgfscope}%
\pgfpathrectangle{\pgfqpoint{0.617715in}{1.333988in}}{\pgfqpoint{5.466114in}{3.078389in}}%
\pgfusepath{clip}%
\pgfsetrectcap%
\pgfsetroundjoin%
\pgfsetlinewidth{1.003750pt}%
\definecolor{currentstroke}{rgb}{0.000000,0.000000,0.000000}%
\pgfsetstrokecolor{currentstroke}%
\pgfsetdash{}{0pt}%
\pgfusepath{stroke}%
\end{pgfscope}%
\begin{pgfscope}%
\pgfpathrectangle{\pgfqpoint{0.617715in}{1.333988in}}{\pgfqpoint{5.466114in}{3.078389in}}%
\pgfusepath{clip}%
\pgfsetrectcap%
\pgfsetroundjoin%
\pgfsetlinewidth{1.003750pt}%
\definecolor{currentstroke}{rgb}{0.000000,0.000000,0.000000}%
\pgfsetstrokecolor{currentstroke}%
\pgfsetdash{}{0pt}%
\pgfusepath{stroke}%
\end{pgfscope}%
\begin{pgfscope}%
\pgfpathrectangle{\pgfqpoint{0.617715in}{1.333988in}}{\pgfqpoint{5.466114in}{3.078389in}}%
\pgfusepath{clip}%
\pgfsetrectcap%
\pgfsetroundjoin%
\pgfsetlinewidth{1.003750pt}%
\definecolor{currentstroke}{rgb}{0.000000,0.000000,0.000000}%
\pgfsetstrokecolor{currentstroke}%
\pgfsetdash{}{0pt}%
\pgfusepath{stroke}%
\end{pgfscope}%
\begin{pgfscope}%
\pgfpathrectangle{\pgfqpoint{0.617715in}{1.333988in}}{\pgfqpoint{5.466114in}{3.078389in}}%
\pgfusepath{clip}%
\pgfsetrectcap%
\pgfsetroundjoin%
\pgfsetlinewidth{1.003750pt}%
\definecolor{currentstroke}{rgb}{0.000000,0.000000,0.000000}%
\pgfsetstrokecolor{currentstroke}%
\pgfsetdash{}{0pt}%
\pgfpathmoveto{\pgfqpoint{3.988485in}{1.488074in}}%
\pgfpathlineto{\pgfqpoint{4.170689in}{1.488074in}}%
\pgfpathlineto{\pgfqpoint{4.170689in}{1.585852in}}%
\pgfpathlineto{\pgfqpoint{3.988485in}{1.585852in}}%
\pgfpathlineto{\pgfqpoint{3.988485in}{1.488074in}}%
\pgfusepath{stroke}%
\end{pgfscope}%
\begin{pgfscope}%
\pgfpathrectangle{\pgfqpoint{0.617715in}{1.333988in}}{\pgfqpoint{5.466114in}{3.078389in}}%
\pgfusepath{clip}%
\pgfsetrectcap%
\pgfsetroundjoin%
\pgfsetlinewidth{1.003750pt}%
\definecolor{currentstroke}{rgb}{0.000000,0.000000,0.000000}%
\pgfsetstrokecolor{currentstroke}%
\pgfsetdash{}{0pt}%
\pgfpathmoveto{\pgfqpoint{4.079587in}{1.488074in}}%
\pgfpathlineto{\pgfqpoint{4.079587in}{1.474072in}}%
\pgfusepath{stroke}%
\end{pgfscope}%
\begin{pgfscope}%
\pgfpathrectangle{\pgfqpoint{0.617715in}{1.333988in}}{\pgfqpoint{5.466114in}{3.078389in}}%
\pgfusepath{clip}%
\pgfsetrectcap%
\pgfsetroundjoin%
\pgfsetlinewidth{1.003750pt}%
\definecolor{currentstroke}{rgb}{0.000000,0.000000,0.000000}%
\pgfsetstrokecolor{currentstroke}%
\pgfsetdash{}{0pt}%
\pgfpathmoveto{\pgfqpoint{4.079587in}{1.585852in}}%
\pgfpathlineto{\pgfqpoint{4.079587in}{1.729446in}}%
\pgfusepath{stroke}%
\end{pgfscope}%
\begin{pgfscope}%
\pgfpathrectangle{\pgfqpoint{0.617715in}{1.333988in}}{\pgfqpoint{5.466114in}{3.078389in}}%
\pgfusepath{clip}%
\pgfsetrectcap%
\pgfsetroundjoin%
\pgfsetlinewidth{1.003750pt}%
\definecolor{currentstroke}{rgb}{0.000000,0.000000,0.000000}%
\pgfsetstrokecolor{currentstroke}%
\pgfsetdash{}{0pt}%
\pgfpathmoveto{\pgfqpoint{4.034036in}{1.474072in}}%
\pgfpathlineto{\pgfqpoint{4.125138in}{1.474072in}}%
\pgfusepath{stroke}%
\end{pgfscope}%
\begin{pgfscope}%
\pgfpathrectangle{\pgfqpoint{0.617715in}{1.333988in}}{\pgfqpoint{5.466114in}{3.078389in}}%
\pgfusepath{clip}%
\pgfsetrectcap%
\pgfsetroundjoin%
\pgfsetlinewidth{1.003750pt}%
\definecolor{currentstroke}{rgb}{0.000000,0.000000,0.000000}%
\pgfsetstrokecolor{currentstroke}%
\pgfsetdash{}{0pt}%
\pgfpathmoveto{\pgfqpoint{4.034036in}{1.729446in}}%
\pgfpathlineto{\pgfqpoint{4.125138in}{1.729446in}}%
\pgfusepath{stroke}%
\end{pgfscope}%
\begin{pgfscope}%
\pgfpathrectangle{\pgfqpoint{0.617715in}{1.333988in}}{\pgfqpoint{5.466114in}{3.078389in}}%
\pgfusepath{clip}%
\pgfsetbuttcap%
\pgfsetroundjoin%
\definecolor{currentfill}{rgb}{0.000000,0.000000,0.000000}%
\pgfsetfillcolor{currentfill}%
\pgfsetfillopacity{0.000000}%
\pgfsetlinewidth{1.003750pt}%
\definecolor{currentstroke}{rgb}{0.000000,0.000000,0.000000}%
\pgfsetstrokecolor{currentstroke}%
\pgfsetdash{}{0pt}%
\pgfsys@defobject{currentmarker}{\pgfqpoint{-0.041667in}{-0.041667in}}{\pgfqpoint{0.041667in}{0.041667in}}{%
\pgfpathmoveto{\pgfqpoint{0.000000in}{-0.041667in}}%
\pgfpathcurveto{\pgfqpoint{0.011050in}{-0.041667in}}{\pgfqpoint{0.021649in}{-0.037276in}}{\pgfqpoint{0.029463in}{-0.029463in}}%
\pgfpathcurveto{\pgfqpoint{0.037276in}{-0.021649in}}{\pgfqpoint{0.041667in}{-0.011050in}}{\pgfqpoint{0.041667in}{0.000000in}}%
\pgfpathcurveto{\pgfqpoint{0.041667in}{0.011050in}}{\pgfqpoint{0.037276in}{0.021649in}}{\pgfqpoint{0.029463in}{0.029463in}}%
\pgfpathcurveto{\pgfqpoint{0.021649in}{0.037276in}}{\pgfqpoint{0.011050in}{0.041667in}}{\pgfqpoint{0.000000in}{0.041667in}}%
\pgfpathcurveto{\pgfqpoint{-0.011050in}{0.041667in}}{\pgfqpoint{-0.021649in}{0.037276in}}{\pgfqpoint{-0.029463in}{0.029463in}}%
\pgfpathcurveto{\pgfqpoint{-0.037276in}{0.021649in}}{\pgfqpoint{-0.041667in}{0.011050in}}{\pgfqpoint{-0.041667in}{0.000000in}}%
\pgfpathcurveto{\pgfqpoint{-0.041667in}{-0.011050in}}{\pgfqpoint{-0.037276in}{-0.021649in}}{\pgfqpoint{-0.029463in}{-0.029463in}}%
\pgfpathcurveto{\pgfqpoint{-0.021649in}{-0.037276in}}{\pgfqpoint{-0.011050in}{-0.041667in}}{\pgfqpoint{0.000000in}{-0.041667in}}%
\pgfpathlineto{\pgfqpoint{0.000000in}{-0.041667in}}%
\pgfpathclose%
\pgfusepath{stroke,fill}%
}%
\begin{pgfscope}%
\pgfsys@transformshift{4.079587in}{1.748447in}%
\pgfsys@useobject{currentmarker}{}%
\end{pgfscope}%
\begin{pgfscope}%
\pgfsys@transformshift{4.079587in}{1.745064in}%
\pgfsys@useobject{currentmarker}{}%
\end{pgfscope}%
\begin{pgfscope}%
\pgfsys@transformshift{4.079587in}{1.778497in}%
\pgfsys@useobject{currentmarker}{}%
\end{pgfscope}%
\begin{pgfscope}%
\pgfsys@transformshift{4.079587in}{2.207713in}%
\pgfsys@useobject{currentmarker}{}%
\end{pgfscope}%
\begin{pgfscope}%
\pgfsys@transformshift{4.079587in}{1.946021in}%
\pgfsys@useobject{currentmarker}{}%
\end{pgfscope}%
\begin{pgfscope}%
\pgfsys@transformshift{4.079587in}{1.909125in}%
\pgfsys@useobject{currentmarker}{}%
\end{pgfscope}%
\begin{pgfscope}%
\pgfsys@transformshift{4.079587in}{1.892899in}%
\pgfsys@useobject{currentmarker}{}%
\end{pgfscope}%
\begin{pgfscope}%
\pgfsys@transformshift{4.079587in}{2.482806in}%
\pgfsys@useobject{currentmarker}{}%
\end{pgfscope}%
\begin{pgfscope}%
\pgfsys@transformshift{4.079587in}{1.957239in}%
\pgfsys@useobject{currentmarker}{}%
\end{pgfscope}%
\begin{pgfscope}%
\pgfsys@transformshift{4.079587in}{1.849028in}%
\pgfsys@useobject{currentmarker}{}%
\end{pgfscope}%
\begin{pgfscope}%
\pgfsys@transformshift{4.079587in}{1.843391in}%
\pgfsys@useobject{currentmarker}{}%
\end{pgfscope}%
\begin{pgfscope}%
\pgfsys@transformshift{4.079587in}{1.870908in}%
\pgfsys@useobject{currentmarker}{}%
\end{pgfscope}%
\begin{pgfscope}%
\pgfsys@transformshift{4.079587in}{1.876652in}%
\pgfsys@useobject{currentmarker}{}%
\end{pgfscope}%
\end{pgfscope}%
\begin{pgfscope}%
\pgfpathrectangle{\pgfqpoint{0.617715in}{1.333988in}}{\pgfqpoint{5.466114in}{3.078389in}}%
\pgfusepath{clip}%
\pgfsetrectcap%
\pgfsetroundjoin%
\pgfsetlinewidth{1.003750pt}%
\definecolor{currentstroke}{rgb}{0.000000,0.000000,0.000000}%
\pgfsetstrokecolor{currentstroke}%
\pgfsetdash{}{0pt}%
\pgfpathmoveto{\pgfqpoint{4.352893in}{1.474328in}}%
\pgfpathlineto{\pgfqpoint{4.535097in}{1.474328in}}%
\pgfpathlineto{\pgfqpoint{4.535097in}{1.475632in}}%
\pgfpathlineto{\pgfqpoint{4.352893in}{1.475632in}}%
\pgfpathlineto{\pgfqpoint{4.352893in}{1.474328in}}%
\pgfusepath{stroke}%
\end{pgfscope}%
\begin{pgfscope}%
\pgfpathrectangle{\pgfqpoint{0.617715in}{1.333988in}}{\pgfqpoint{5.466114in}{3.078389in}}%
\pgfusepath{clip}%
\pgfsetrectcap%
\pgfsetroundjoin%
\pgfsetlinewidth{1.003750pt}%
\definecolor{currentstroke}{rgb}{0.000000,0.000000,0.000000}%
\pgfsetstrokecolor{currentstroke}%
\pgfsetdash{}{0pt}%
\pgfpathmoveto{\pgfqpoint{4.443995in}{1.474328in}}%
\pgfpathlineto{\pgfqpoint{4.443995in}{1.473974in}}%
\pgfusepath{stroke}%
\end{pgfscope}%
\begin{pgfscope}%
\pgfpathrectangle{\pgfqpoint{0.617715in}{1.333988in}}{\pgfqpoint{5.466114in}{3.078389in}}%
\pgfusepath{clip}%
\pgfsetrectcap%
\pgfsetroundjoin%
\pgfsetlinewidth{1.003750pt}%
\definecolor{currentstroke}{rgb}{0.000000,0.000000,0.000000}%
\pgfsetstrokecolor{currentstroke}%
\pgfsetdash{}{0pt}%
\pgfpathmoveto{\pgfqpoint{4.443995in}{1.475632in}}%
\pgfpathlineto{\pgfqpoint{4.443995in}{1.475927in}}%
\pgfusepath{stroke}%
\end{pgfscope}%
\begin{pgfscope}%
\pgfpathrectangle{\pgfqpoint{0.617715in}{1.333988in}}{\pgfqpoint{5.466114in}{3.078389in}}%
\pgfusepath{clip}%
\pgfsetrectcap%
\pgfsetroundjoin%
\pgfsetlinewidth{1.003750pt}%
\definecolor{currentstroke}{rgb}{0.000000,0.000000,0.000000}%
\pgfsetstrokecolor{currentstroke}%
\pgfsetdash{}{0pt}%
\pgfpathmoveto{\pgfqpoint{4.398444in}{1.473974in}}%
\pgfpathlineto{\pgfqpoint{4.489546in}{1.473974in}}%
\pgfusepath{stroke}%
\end{pgfscope}%
\begin{pgfscope}%
\pgfpathrectangle{\pgfqpoint{0.617715in}{1.333988in}}{\pgfqpoint{5.466114in}{3.078389in}}%
\pgfusepath{clip}%
\pgfsetrectcap%
\pgfsetroundjoin%
\pgfsetlinewidth{1.003750pt}%
\definecolor{currentstroke}{rgb}{0.000000,0.000000,0.000000}%
\pgfsetstrokecolor{currentstroke}%
\pgfsetdash{}{0pt}%
\pgfpathmoveto{\pgfqpoint{4.398444in}{1.475927in}}%
\pgfpathlineto{\pgfqpoint{4.489546in}{1.475927in}}%
\pgfusepath{stroke}%
\end{pgfscope}%
\begin{pgfscope}%
\pgfpathrectangle{\pgfqpoint{0.617715in}{1.333988in}}{\pgfqpoint{5.466114in}{3.078389in}}%
\pgfusepath{clip}%
\pgfsetbuttcap%
\pgfsetroundjoin%
\definecolor{currentfill}{rgb}{0.000000,0.000000,0.000000}%
\pgfsetfillcolor{currentfill}%
\pgfsetfillopacity{0.000000}%
\pgfsetlinewidth{1.003750pt}%
\definecolor{currentstroke}{rgb}{0.000000,0.000000,0.000000}%
\pgfsetstrokecolor{currentstroke}%
\pgfsetdash{}{0pt}%
\pgfsys@defobject{currentmarker}{\pgfqpoint{-0.041667in}{-0.041667in}}{\pgfqpoint{0.041667in}{0.041667in}}{%
\pgfpathmoveto{\pgfqpoint{0.000000in}{-0.041667in}}%
\pgfpathcurveto{\pgfqpoint{0.011050in}{-0.041667in}}{\pgfqpoint{0.021649in}{-0.037276in}}{\pgfqpoint{0.029463in}{-0.029463in}}%
\pgfpathcurveto{\pgfqpoint{0.037276in}{-0.021649in}}{\pgfqpoint{0.041667in}{-0.011050in}}{\pgfqpoint{0.041667in}{0.000000in}}%
\pgfpathcurveto{\pgfqpoint{0.041667in}{0.011050in}}{\pgfqpoint{0.037276in}{0.021649in}}{\pgfqpoint{0.029463in}{0.029463in}}%
\pgfpathcurveto{\pgfqpoint{0.021649in}{0.037276in}}{\pgfqpoint{0.011050in}{0.041667in}}{\pgfqpoint{0.000000in}{0.041667in}}%
\pgfpathcurveto{\pgfqpoint{-0.011050in}{0.041667in}}{\pgfqpoint{-0.021649in}{0.037276in}}{\pgfqpoint{-0.029463in}{0.029463in}}%
\pgfpathcurveto{\pgfqpoint{-0.037276in}{0.021649in}}{\pgfqpoint{-0.041667in}{0.011050in}}{\pgfqpoint{-0.041667in}{0.000000in}}%
\pgfpathcurveto{\pgfqpoint{-0.041667in}{-0.011050in}}{\pgfqpoint{-0.037276in}{-0.021649in}}{\pgfqpoint{-0.029463in}{-0.029463in}}%
\pgfpathcurveto{\pgfqpoint{-0.021649in}{-0.037276in}}{\pgfqpoint{-0.011050in}{-0.041667in}}{\pgfqpoint{0.000000in}{-0.041667in}}%
\pgfpathlineto{\pgfqpoint{0.000000in}{-0.041667in}}%
\pgfpathclose%
\pgfusepath{stroke,fill}%
}%
\begin{pgfscope}%
\pgfsys@transformshift{4.443995in}{1.479425in}%
\pgfsys@useobject{currentmarker}{}%
\end{pgfscope}%
\begin{pgfscope}%
\pgfsys@transformshift{4.443995in}{1.479606in}%
\pgfsys@useobject{currentmarker}{}%
\end{pgfscope}%
\begin{pgfscope}%
\pgfsys@transformshift{4.443995in}{1.481926in}%
\pgfsys@useobject{currentmarker}{}%
\end{pgfscope}%
\begin{pgfscope}%
\pgfsys@transformshift{4.443995in}{1.521716in}%
\pgfsys@useobject{currentmarker}{}%
\end{pgfscope}%
\begin{pgfscope}%
\pgfsys@transformshift{4.443995in}{1.483665in}%
\pgfsys@useobject{currentmarker}{}%
\end{pgfscope}%
\end{pgfscope}%
\begin{pgfscope}%
\pgfpathrectangle{\pgfqpoint{0.617715in}{1.333988in}}{\pgfqpoint{5.466114in}{3.078389in}}%
\pgfusepath{clip}%
\pgfsetrectcap%
\pgfsetroundjoin%
\pgfsetlinewidth{1.003750pt}%
\definecolor{currentstroke}{rgb}{0.000000,0.000000,0.000000}%
\pgfsetstrokecolor{currentstroke}%
\pgfsetdash{}{0pt}%
\pgfpathmoveto{\pgfqpoint{4.717300in}{1.506582in}}%
\pgfpathlineto{\pgfqpoint{4.899504in}{1.506582in}}%
\pgfpathlineto{\pgfqpoint{4.899504in}{1.673625in}}%
\pgfpathlineto{\pgfqpoint{4.717300in}{1.673625in}}%
\pgfpathlineto{\pgfqpoint{4.717300in}{1.506582in}}%
\pgfusepath{stroke}%
\end{pgfscope}%
\begin{pgfscope}%
\pgfpathrectangle{\pgfqpoint{0.617715in}{1.333988in}}{\pgfqpoint{5.466114in}{3.078389in}}%
\pgfusepath{clip}%
\pgfsetrectcap%
\pgfsetroundjoin%
\pgfsetlinewidth{1.003750pt}%
\definecolor{currentstroke}{rgb}{0.000000,0.000000,0.000000}%
\pgfsetstrokecolor{currentstroke}%
\pgfsetdash{}{0pt}%
\pgfpathmoveto{\pgfqpoint{4.808402in}{1.506582in}}%
\pgfpathlineto{\pgfqpoint{4.808402in}{1.474147in}}%
\pgfusepath{stroke}%
\end{pgfscope}%
\begin{pgfscope}%
\pgfpathrectangle{\pgfqpoint{0.617715in}{1.333988in}}{\pgfqpoint{5.466114in}{3.078389in}}%
\pgfusepath{clip}%
\pgfsetrectcap%
\pgfsetroundjoin%
\pgfsetlinewidth{1.003750pt}%
\definecolor{currentstroke}{rgb}{0.000000,0.000000,0.000000}%
\pgfsetstrokecolor{currentstroke}%
\pgfsetdash{}{0pt}%
\pgfpathmoveto{\pgfqpoint{4.808402in}{1.673625in}}%
\pgfpathlineto{\pgfqpoint{4.808402in}{1.892600in}}%
\pgfusepath{stroke}%
\end{pgfscope}%
\begin{pgfscope}%
\pgfpathrectangle{\pgfqpoint{0.617715in}{1.333988in}}{\pgfqpoint{5.466114in}{3.078389in}}%
\pgfusepath{clip}%
\pgfsetrectcap%
\pgfsetroundjoin%
\pgfsetlinewidth{1.003750pt}%
\definecolor{currentstroke}{rgb}{0.000000,0.000000,0.000000}%
\pgfsetstrokecolor{currentstroke}%
\pgfsetdash{}{0pt}%
\pgfpathmoveto{\pgfqpoint{4.762851in}{1.474147in}}%
\pgfpathlineto{\pgfqpoint{4.853953in}{1.474147in}}%
\pgfusepath{stroke}%
\end{pgfscope}%
\begin{pgfscope}%
\pgfpathrectangle{\pgfqpoint{0.617715in}{1.333988in}}{\pgfqpoint{5.466114in}{3.078389in}}%
\pgfusepath{clip}%
\pgfsetrectcap%
\pgfsetroundjoin%
\pgfsetlinewidth{1.003750pt}%
\definecolor{currentstroke}{rgb}{0.000000,0.000000,0.000000}%
\pgfsetstrokecolor{currentstroke}%
\pgfsetdash{}{0pt}%
\pgfpathmoveto{\pgfqpoint{4.762851in}{1.892600in}}%
\pgfpathlineto{\pgfqpoint{4.853953in}{1.892600in}}%
\pgfusepath{stroke}%
\end{pgfscope}%
\begin{pgfscope}%
\pgfpathrectangle{\pgfqpoint{0.617715in}{1.333988in}}{\pgfqpoint{5.466114in}{3.078389in}}%
\pgfusepath{clip}%
\pgfsetbuttcap%
\pgfsetroundjoin%
\definecolor{currentfill}{rgb}{0.000000,0.000000,0.000000}%
\pgfsetfillcolor{currentfill}%
\pgfsetfillopacity{0.000000}%
\pgfsetlinewidth{1.003750pt}%
\definecolor{currentstroke}{rgb}{0.000000,0.000000,0.000000}%
\pgfsetstrokecolor{currentstroke}%
\pgfsetdash{}{0pt}%
\pgfsys@defobject{currentmarker}{\pgfqpoint{-0.041667in}{-0.041667in}}{\pgfqpoint{0.041667in}{0.041667in}}{%
\pgfpathmoveto{\pgfqpoint{0.000000in}{-0.041667in}}%
\pgfpathcurveto{\pgfqpoint{0.011050in}{-0.041667in}}{\pgfqpoint{0.021649in}{-0.037276in}}{\pgfqpoint{0.029463in}{-0.029463in}}%
\pgfpathcurveto{\pgfqpoint{0.037276in}{-0.021649in}}{\pgfqpoint{0.041667in}{-0.011050in}}{\pgfqpoint{0.041667in}{0.000000in}}%
\pgfpathcurveto{\pgfqpoint{0.041667in}{0.011050in}}{\pgfqpoint{0.037276in}{0.021649in}}{\pgfqpoint{0.029463in}{0.029463in}}%
\pgfpathcurveto{\pgfqpoint{0.021649in}{0.037276in}}{\pgfqpoint{0.011050in}{0.041667in}}{\pgfqpoint{0.000000in}{0.041667in}}%
\pgfpathcurveto{\pgfqpoint{-0.011050in}{0.041667in}}{\pgfqpoint{-0.021649in}{0.037276in}}{\pgfqpoint{-0.029463in}{0.029463in}}%
\pgfpathcurveto{\pgfqpoint{-0.037276in}{0.021649in}}{\pgfqpoint{-0.041667in}{0.011050in}}{\pgfqpoint{-0.041667in}{0.000000in}}%
\pgfpathcurveto{\pgfqpoint{-0.041667in}{-0.011050in}}{\pgfqpoint{-0.037276in}{-0.021649in}}{\pgfqpoint{-0.029463in}{-0.029463in}}%
\pgfpathcurveto{\pgfqpoint{-0.021649in}{-0.037276in}}{\pgfqpoint{-0.011050in}{-0.041667in}}{\pgfqpoint{0.000000in}{-0.041667in}}%
\pgfpathlineto{\pgfqpoint{0.000000in}{-0.041667in}}%
\pgfpathclose%
\pgfusepath{stroke,fill}%
}%
\begin{pgfscope}%
\pgfsys@transformshift{4.808402in}{1.964030in}%
\pgfsys@useobject{currentmarker}{}%
\end{pgfscope}%
\begin{pgfscope}%
\pgfsys@transformshift{4.808402in}{1.925227in}%
\pgfsys@useobject{currentmarker}{}%
\end{pgfscope}%
\begin{pgfscope}%
\pgfsys@transformshift{4.808402in}{2.098125in}%
\pgfsys@useobject{currentmarker}{}%
\end{pgfscope}%
\end{pgfscope}%
\begin{pgfscope}%
\pgfpathrectangle{\pgfqpoint{0.617715in}{1.333988in}}{\pgfqpoint{5.466114in}{3.078389in}}%
\pgfusepath{clip}%
\pgfsetrectcap%
\pgfsetroundjoin%
\pgfsetlinewidth{1.003750pt}%
\definecolor{currentstroke}{rgb}{0.000000,0.000000,0.000000}%
\pgfsetstrokecolor{currentstroke}%
\pgfsetdash{}{0pt}%
\pgfpathmoveto{\pgfqpoint{5.081708in}{1.518501in}}%
\pgfpathlineto{\pgfqpoint{5.263912in}{1.518501in}}%
\pgfpathlineto{\pgfqpoint{5.263912in}{1.718350in}}%
\pgfpathlineto{\pgfqpoint{5.081708in}{1.718350in}}%
\pgfpathlineto{\pgfqpoint{5.081708in}{1.518501in}}%
\pgfusepath{stroke}%
\end{pgfscope}%
\begin{pgfscope}%
\pgfpathrectangle{\pgfqpoint{0.617715in}{1.333988in}}{\pgfqpoint{5.466114in}{3.078389in}}%
\pgfusepath{clip}%
\pgfsetrectcap%
\pgfsetroundjoin%
\pgfsetlinewidth{1.003750pt}%
\definecolor{currentstroke}{rgb}{0.000000,0.000000,0.000000}%
\pgfsetstrokecolor{currentstroke}%
\pgfsetdash{}{0pt}%
\pgfpathmoveto{\pgfqpoint{5.172810in}{1.518501in}}%
\pgfpathlineto{\pgfqpoint{5.172810in}{1.476360in}}%
\pgfusepath{stroke}%
\end{pgfscope}%
\begin{pgfscope}%
\pgfpathrectangle{\pgfqpoint{0.617715in}{1.333988in}}{\pgfqpoint{5.466114in}{3.078389in}}%
\pgfusepath{clip}%
\pgfsetrectcap%
\pgfsetroundjoin%
\pgfsetlinewidth{1.003750pt}%
\definecolor{currentstroke}{rgb}{0.000000,0.000000,0.000000}%
\pgfsetstrokecolor{currentstroke}%
\pgfsetdash{}{0pt}%
\pgfpathmoveto{\pgfqpoint{5.172810in}{1.718350in}}%
\pgfpathlineto{\pgfqpoint{5.172810in}{1.979692in}}%
\pgfusepath{stroke}%
\end{pgfscope}%
\begin{pgfscope}%
\pgfpathrectangle{\pgfqpoint{0.617715in}{1.333988in}}{\pgfqpoint{5.466114in}{3.078389in}}%
\pgfusepath{clip}%
\pgfsetrectcap%
\pgfsetroundjoin%
\pgfsetlinewidth{1.003750pt}%
\definecolor{currentstroke}{rgb}{0.000000,0.000000,0.000000}%
\pgfsetstrokecolor{currentstroke}%
\pgfsetdash{}{0pt}%
\pgfpathmoveto{\pgfqpoint{5.127259in}{1.476360in}}%
\pgfpathlineto{\pgfqpoint{5.218361in}{1.476360in}}%
\pgfusepath{stroke}%
\end{pgfscope}%
\begin{pgfscope}%
\pgfpathrectangle{\pgfqpoint{0.617715in}{1.333988in}}{\pgfqpoint{5.466114in}{3.078389in}}%
\pgfusepath{clip}%
\pgfsetrectcap%
\pgfsetroundjoin%
\pgfsetlinewidth{1.003750pt}%
\definecolor{currentstroke}{rgb}{0.000000,0.000000,0.000000}%
\pgfsetstrokecolor{currentstroke}%
\pgfsetdash{}{0pt}%
\pgfpathmoveto{\pgfqpoint{5.127259in}{1.979692in}}%
\pgfpathlineto{\pgfqpoint{5.218361in}{1.979692in}}%
\pgfusepath{stroke}%
\end{pgfscope}%
\begin{pgfscope}%
\pgfpathrectangle{\pgfqpoint{0.617715in}{1.333988in}}{\pgfqpoint{5.466114in}{3.078389in}}%
\pgfusepath{clip}%
\pgfsetbuttcap%
\pgfsetroundjoin%
\definecolor{currentfill}{rgb}{0.000000,0.000000,0.000000}%
\pgfsetfillcolor{currentfill}%
\pgfsetfillopacity{0.000000}%
\pgfsetlinewidth{1.003750pt}%
\definecolor{currentstroke}{rgb}{0.000000,0.000000,0.000000}%
\pgfsetstrokecolor{currentstroke}%
\pgfsetdash{}{0pt}%
\pgfsys@defobject{currentmarker}{\pgfqpoint{-0.041667in}{-0.041667in}}{\pgfqpoint{0.041667in}{0.041667in}}{%
\pgfpathmoveto{\pgfqpoint{0.000000in}{-0.041667in}}%
\pgfpathcurveto{\pgfqpoint{0.011050in}{-0.041667in}}{\pgfqpoint{0.021649in}{-0.037276in}}{\pgfqpoint{0.029463in}{-0.029463in}}%
\pgfpathcurveto{\pgfqpoint{0.037276in}{-0.021649in}}{\pgfqpoint{0.041667in}{-0.011050in}}{\pgfqpoint{0.041667in}{0.000000in}}%
\pgfpathcurveto{\pgfqpoint{0.041667in}{0.011050in}}{\pgfqpoint{0.037276in}{0.021649in}}{\pgfqpoint{0.029463in}{0.029463in}}%
\pgfpathcurveto{\pgfqpoint{0.021649in}{0.037276in}}{\pgfqpoint{0.011050in}{0.041667in}}{\pgfqpoint{0.000000in}{0.041667in}}%
\pgfpathcurveto{\pgfqpoint{-0.011050in}{0.041667in}}{\pgfqpoint{-0.021649in}{0.037276in}}{\pgfqpoint{-0.029463in}{0.029463in}}%
\pgfpathcurveto{\pgfqpoint{-0.037276in}{0.021649in}}{\pgfqpoint{-0.041667in}{0.011050in}}{\pgfqpoint{-0.041667in}{0.000000in}}%
\pgfpathcurveto{\pgfqpoint{-0.041667in}{-0.011050in}}{\pgfqpoint{-0.037276in}{-0.021649in}}{\pgfqpoint{-0.029463in}{-0.029463in}}%
\pgfpathcurveto{\pgfqpoint{-0.021649in}{-0.037276in}}{\pgfqpoint{-0.011050in}{-0.041667in}}{\pgfqpoint{0.000000in}{-0.041667in}}%
\pgfpathlineto{\pgfqpoint{0.000000in}{-0.041667in}}%
\pgfpathclose%
\pgfusepath{stroke,fill}%
}%
\begin{pgfscope}%
\pgfsys@transformshift{5.172810in}{2.046445in}%
\pgfsys@useobject{currentmarker}{}%
\end{pgfscope}%
\end{pgfscope}%
\begin{pgfscope}%
\pgfpathrectangle{\pgfqpoint{0.617715in}{1.333988in}}{\pgfqpoint{5.466114in}{3.078389in}}%
\pgfusepath{clip}%
\pgfsetrectcap%
\pgfsetroundjoin%
\pgfsetlinewidth{1.003750pt}%
\definecolor{currentstroke}{rgb}{0.000000,0.000000,0.000000}%
\pgfsetstrokecolor{currentstroke}%
\pgfsetdash{}{0pt}%
\pgfpathmoveto{\pgfqpoint{5.446116in}{1.474897in}}%
\pgfpathlineto{\pgfqpoint{5.628319in}{1.474897in}}%
\pgfpathlineto{\pgfqpoint{5.628319in}{1.552877in}}%
\pgfpathlineto{\pgfqpoint{5.446116in}{1.552877in}}%
\pgfpathlineto{\pgfqpoint{5.446116in}{1.474897in}}%
\pgfusepath{stroke}%
\end{pgfscope}%
\begin{pgfscope}%
\pgfpathrectangle{\pgfqpoint{0.617715in}{1.333988in}}{\pgfqpoint{5.466114in}{3.078389in}}%
\pgfusepath{clip}%
\pgfsetrectcap%
\pgfsetroundjoin%
\pgfsetlinewidth{1.003750pt}%
\definecolor{currentstroke}{rgb}{0.000000,0.000000,0.000000}%
\pgfsetstrokecolor{currentstroke}%
\pgfsetdash{}{0pt}%
\pgfpathmoveto{\pgfqpoint{5.537217in}{1.474897in}}%
\pgfpathlineto{\pgfqpoint{5.537217in}{1.473922in}}%
\pgfusepath{stroke}%
\end{pgfscope}%
\begin{pgfscope}%
\pgfpathrectangle{\pgfqpoint{0.617715in}{1.333988in}}{\pgfqpoint{5.466114in}{3.078389in}}%
\pgfusepath{clip}%
\pgfsetrectcap%
\pgfsetroundjoin%
\pgfsetlinewidth{1.003750pt}%
\definecolor{currentstroke}{rgb}{0.000000,0.000000,0.000000}%
\pgfsetstrokecolor{currentstroke}%
\pgfsetdash{}{0pt}%
\pgfpathmoveto{\pgfqpoint{5.537217in}{1.552877in}}%
\pgfpathlineto{\pgfqpoint{5.537217in}{1.605896in}}%
\pgfusepath{stroke}%
\end{pgfscope}%
\begin{pgfscope}%
\pgfpathrectangle{\pgfqpoint{0.617715in}{1.333988in}}{\pgfqpoint{5.466114in}{3.078389in}}%
\pgfusepath{clip}%
\pgfsetrectcap%
\pgfsetroundjoin%
\pgfsetlinewidth{1.003750pt}%
\definecolor{currentstroke}{rgb}{0.000000,0.000000,0.000000}%
\pgfsetstrokecolor{currentstroke}%
\pgfsetdash{}{0pt}%
\pgfpathmoveto{\pgfqpoint{5.491667in}{1.473922in}}%
\pgfpathlineto{\pgfqpoint{5.582768in}{1.473922in}}%
\pgfusepath{stroke}%
\end{pgfscope}%
\begin{pgfscope}%
\pgfpathrectangle{\pgfqpoint{0.617715in}{1.333988in}}{\pgfqpoint{5.466114in}{3.078389in}}%
\pgfusepath{clip}%
\pgfsetrectcap%
\pgfsetroundjoin%
\pgfsetlinewidth{1.003750pt}%
\definecolor{currentstroke}{rgb}{0.000000,0.000000,0.000000}%
\pgfsetstrokecolor{currentstroke}%
\pgfsetdash{}{0pt}%
\pgfpathmoveto{\pgfqpoint{5.491667in}{1.605896in}}%
\pgfpathlineto{\pgfqpoint{5.582768in}{1.605896in}}%
\pgfusepath{stroke}%
\end{pgfscope}%
\begin{pgfscope}%
\pgfpathrectangle{\pgfqpoint{0.617715in}{1.333988in}}{\pgfqpoint{5.466114in}{3.078389in}}%
\pgfusepath{clip}%
\pgfsetbuttcap%
\pgfsetroundjoin%
\definecolor{currentfill}{rgb}{0.000000,0.000000,0.000000}%
\pgfsetfillcolor{currentfill}%
\pgfsetfillopacity{0.000000}%
\pgfsetlinewidth{1.003750pt}%
\definecolor{currentstroke}{rgb}{0.000000,0.000000,0.000000}%
\pgfsetstrokecolor{currentstroke}%
\pgfsetdash{}{0pt}%
\pgfsys@defobject{currentmarker}{\pgfqpoint{-0.041667in}{-0.041667in}}{\pgfqpoint{0.041667in}{0.041667in}}{%
\pgfpathmoveto{\pgfqpoint{0.000000in}{-0.041667in}}%
\pgfpathcurveto{\pgfqpoint{0.011050in}{-0.041667in}}{\pgfqpoint{0.021649in}{-0.037276in}}{\pgfqpoint{0.029463in}{-0.029463in}}%
\pgfpathcurveto{\pgfqpoint{0.037276in}{-0.021649in}}{\pgfqpoint{0.041667in}{-0.011050in}}{\pgfqpoint{0.041667in}{0.000000in}}%
\pgfpathcurveto{\pgfqpoint{0.041667in}{0.011050in}}{\pgfqpoint{0.037276in}{0.021649in}}{\pgfqpoint{0.029463in}{0.029463in}}%
\pgfpathcurveto{\pgfqpoint{0.021649in}{0.037276in}}{\pgfqpoint{0.011050in}{0.041667in}}{\pgfqpoint{0.000000in}{0.041667in}}%
\pgfpathcurveto{\pgfqpoint{-0.011050in}{0.041667in}}{\pgfqpoint{-0.021649in}{0.037276in}}{\pgfqpoint{-0.029463in}{0.029463in}}%
\pgfpathcurveto{\pgfqpoint{-0.037276in}{0.021649in}}{\pgfqpoint{-0.041667in}{0.011050in}}{\pgfqpoint{-0.041667in}{0.000000in}}%
\pgfpathcurveto{\pgfqpoint{-0.041667in}{-0.011050in}}{\pgfqpoint{-0.037276in}{-0.021649in}}{\pgfqpoint{-0.029463in}{-0.029463in}}%
\pgfpathcurveto{\pgfqpoint{-0.021649in}{-0.037276in}}{\pgfqpoint{-0.011050in}{-0.041667in}}{\pgfqpoint{0.000000in}{-0.041667in}}%
\pgfpathlineto{\pgfqpoint{0.000000in}{-0.041667in}}%
\pgfpathclose%
\pgfusepath{stroke,fill}%
}%
\begin{pgfscope}%
\pgfsys@transformshift{5.537217in}{1.671538in}%
\pgfsys@useobject{currentmarker}{}%
\end{pgfscope}%
\begin{pgfscope}%
\pgfsys@transformshift{5.537217in}{1.681939in}%
\pgfsys@useobject{currentmarker}{}%
\end{pgfscope}%
\begin{pgfscope}%
\pgfsys@transformshift{5.537217in}{2.201934in}%
\pgfsys@useobject{currentmarker}{}%
\end{pgfscope}%
\begin{pgfscope}%
\pgfsys@transformshift{5.537217in}{1.799390in}%
\pgfsys@useobject{currentmarker}{}%
\end{pgfscope}%
\begin{pgfscope}%
\pgfsys@transformshift{5.537217in}{1.806732in}%
\pgfsys@useobject{currentmarker}{}%
\end{pgfscope}%
\end{pgfscope}%
\begin{pgfscope}%
\pgfpathrectangle{\pgfqpoint{0.617715in}{1.333988in}}{\pgfqpoint{5.466114in}{3.078389in}}%
\pgfusepath{clip}%
\pgfsetrectcap%
\pgfsetroundjoin%
\pgfsetlinewidth{1.003750pt}%
\definecolor{currentstroke}{rgb}{0.000000,0.000000,0.000000}%
\pgfsetstrokecolor{currentstroke}%
\pgfsetdash{}{0pt}%
\pgfpathmoveto{\pgfqpoint{5.810523in}{1.474641in}}%
\pgfpathlineto{\pgfqpoint{5.992727in}{1.474641in}}%
\pgfpathlineto{\pgfqpoint{5.992727in}{1.566668in}}%
\pgfpathlineto{\pgfqpoint{5.810523in}{1.566668in}}%
\pgfpathlineto{\pgfqpoint{5.810523in}{1.474641in}}%
\pgfusepath{stroke}%
\end{pgfscope}%
\begin{pgfscope}%
\pgfpathrectangle{\pgfqpoint{0.617715in}{1.333988in}}{\pgfqpoint{5.466114in}{3.078389in}}%
\pgfusepath{clip}%
\pgfsetrectcap%
\pgfsetroundjoin%
\pgfsetlinewidth{1.003750pt}%
\definecolor{currentstroke}{rgb}{0.000000,0.000000,0.000000}%
\pgfsetstrokecolor{currentstroke}%
\pgfsetdash{}{0pt}%
\pgfpathmoveto{\pgfqpoint{5.901625in}{1.474641in}}%
\pgfpathlineto{\pgfqpoint{5.901625in}{1.473946in}}%
\pgfusepath{stroke}%
\end{pgfscope}%
\begin{pgfscope}%
\pgfpathrectangle{\pgfqpoint{0.617715in}{1.333988in}}{\pgfqpoint{5.466114in}{3.078389in}}%
\pgfusepath{clip}%
\pgfsetrectcap%
\pgfsetroundjoin%
\pgfsetlinewidth{1.003750pt}%
\definecolor{currentstroke}{rgb}{0.000000,0.000000,0.000000}%
\pgfsetstrokecolor{currentstroke}%
\pgfsetdash{}{0pt}%
\pgfpathmoveto{\pgfqpoint{5.901625in}{1.566668in}}%
\pgfpathlineto{\pgfqpoint{5.901625in}{1.660541in}}%
\pgfusepath{stroke}%
\end{pgfscope}%
\begin{pgfscope}%
\pgfpathrectangle{\pgfqpoint{0.617715in}{1.333988in}}{\pgfqpoint{5.466114in}{3.078389in}}%
\pgfusepath{clip}%
\pgfsetrectcap%
\pgfsetroundjoin%
\pgfsetlinewidth{1.003750pt}%
\definecolor{currentstroke}{rgb}{0.000000,0.000000,0.000000}%
\pgfsetstrokecolor{currentstroke}%
\pgfsetdash{}{0pt}%
\pgfpathmoveto{\pgfqpoint{5.856074in}{1.473946in}}%
\pgfpathlineto{\pgfqpoint{5.947176in}{1.473946in}}%
\pgfusepath{stroke}%
\end{pgfscope}%
\begin{pgfscope}%
\pgfpathrectangle{\pgfqpoint{0.617715in}{1.333988in}}{\pgfqpoint{5.466114in}{3.078389in}}%
\pgfusepath{clip}%
\pgfsetrectcap%
\pgfsetroundjoin%
\pgfsetlinewidth{1.003750pt}%
\definecolor{currentstroke}{rgb}{0.000000,0.000000,0.000000}%
\pgfsetstrokecolor{currentstroke}%
\pgfsetdash{}{0pt}%
\pgfpathmoveto{\pgfqpoint{5.856074in}{1.660541in}}%
\pgfpathlineto{\pgfqpoint{5.947176in}{1.660541in}}%
\pgfusepath{stroke}%
\end{pgfscope}%
\begin{pgfscope}%
\pgfpathrectangle{\pgfqpoint{0.617715in}{1.333988in}}{\pgfqpoint{5.466114in}{3.078389in}}%
\pgfusepath{clip}%
\pgfsetbuttcap%
\pgfsetroundjoin%
\definecolor{currentfill}{rgb}{0.000000,0.000000,0.000000}%
\pgfsetfillcolor{currentfill}%
\pgfsetfillopacity{0.000000}%
\pgfsetlinewidth{1.003750pt}%
\definecolor{currentstroke}{rgb}{0.000000,0.000000,0.000000}%
\pgfsetstrokecolor{currentstroke}%
\pgfsetdash{}{0pt}%
\pgfsys@defobject{currentmarker}{\pgfqpoint{-0.041667in}{-0.041667in}}{\pgfqpoint{0.041667in}{0.041667in}}{%
\pgfpathmoveto{\pgfqpoint{0.000000in}{-0.041667in}}%
\pgfpathcurveto{\pgfqpoint{0.011050in}{-0.041667in}}{\pgfqpoint{0.021649in}{-0.037276in}}{\pgfqpoint{0.029463in}{-0.029463in}}%
\pgfpathcurveto{\pgfqpoint{0.037276in}{-0.021649in}}{\pgfqpoint{0.041667in}{-0.011050in}}{\pgfqpoint{0.041667in}{0.000000in}}%
\pgfpathcurveto{\pgfqpoint{0.041667in}{0.011050in}}{\pgfqpoint{0.037276in}{0.021649in}}{\pgfqpoint{0.029463in}{0.029463in}}%
\pgfpathcurveto{\pgfqpoint{0.021649in}{0.037276in}}{\pgfqpoint{0.011050in}{0.041667in}}{\pgfqpoint{0.000000in}{0.041667in}}%
\pgfpathcurveto{\pgfqpoint{-0.011050in}{0.041667in}}{\pgfqpoint{-0.021649in}{0.037276in}}{\pgfqpoint{-0.029463in}{0.029463in}}%
\pgfpathcurveto{\pgfqpoint{-0.037276in}{0.021649in}}{\pgfqpoint{-0.041667in}{0.011050in}}{\pgfqpoint{-0.041667in}{0.000000in}}%
\pgfpathcurveto{\pgfqpoint{-0.041667in}{-0.011050in}}{\pgfqpoint{-0.037276in}{-0.021649in}}{\pgfqpoint{-0.029463in}{-0.029463in}}%
\pgfpathcurveto{\pgfqpoint{-0.021649in}{-0.037276in}}{\pgfqpoint{-0.011050in}{-0.041667in}}{\pgfqpoint{0.000000in}{-0.041667in}}%
\pgfpathlineto{\pgfqpoint{0.000000in}{-0.041667in}}%
\pgfpathclose%
\pgfusepath{stroke,fill}%
}%
\begin{pgfscope}%
\pgfsys@transformshift{5.901625in}{1.933679in}%
\pgfsys@useobject{currentmarker}{}%
\end{pgfscope}%
\begin{pgfscope}%
\pgfsys@transformshift{5.901625in}{2.365638in}%
\pgfsys@useobject{currentmarker}{}%
\end{pgfscope}%
\begin{pgfscope}%
\pgfsys@transformshift{5.901625in}{2.016760in}%
\pgfsys@useobject{currentmarker}{}%
\end{pgfscope}%
\begin{pgfscope}%
\pgfsys@transformshift{5.901625in}{1.766372in}%
\pgfsys@useobject{currentmarker}{}%
\end{pgfscope}%
\begin{pgfscope}%
\pgfsys@transformshift{5.901625in}{1.886076in}%
\pgfsys@useobject{currentmarker}{}%
\end{pgfscope}%
\end{pgfscope}%
\begin{pgfscope}%
\pgfpathrectangle{\pgfqpoint{0.617715in}{1.333988in}}{\pgfqpoint{5.466114in}{3.078389in}}%
\pgfusepath{clip}%
\pgfsetbuttcap%
\pgfsetroundjoin%
\pgfsetlinewidth{1.003750pt}%
\definecolor{currentstroke}{rgb}{1.000000,0.498039,0.054902}%
\pgfsetstrokecolor{currentstroke}%
\pgfsetdash{}{0pt}%
\pgfpathmoveto{\pgfqpoint{0.708817in}{1.494273in}}%
\pgfpathlineto{\pgfqpoint{0.891021in}{1.494273in}}%
\pgfusepath{stroke}%
\end{pgfscope}%
\begin{pgfscope}%
\pgfpathrectangle{\pgfqpoint{0.617715in}{1.333988in}}{\pgfqpoint{5.466114in}{3.078389in}}%
\pgfusepath{clip}%
\pgfsetbuttcap%
\pgfsetroundjoin%
\pgfsetlinewidth{1.003750pt}%
\definecolor{currentstroke}{rgb}{1.000000,0.498039,0.054902}%
\pgfsetstrokecolor{currentstroke}%
\pgfsetdash{}{0pt}%
\pgfpathmoveto{\pgfqpoint{1.073225in}{1.475445in}}%
\pgfpathlineto{\pgfqpoint{1.255429in}{1.475445in}}%
\pgfusepath{stroke}%
\end{pgfscope}%
\begin{pgfscope}%
\pgfpathrectangle{\pgfqpoint{0.617715in}{1.333988in}}{\pgfqpoint{5.466114in}{3.078389in}}%
\pgfusepath{clip}%
\pgfsetbuttcap%
\pgfsetroundjoin%
\pgfsetlinewidth{1.003750pt}%
\definecolor{currentstroke}{rgb}{1.000000,0.498039,0.054902}%
\pgfsetstrokecolor{currentstroke}%
\pgfsetdash{}{0pt}%
\pgfpathmoveto{\pgfqpoint{1.437632in}{1.601323in}}%
\pgfpathlineto{\pgfqpoint{1.619836in}{1.601323in}}%
\pgfusepath{stroke}%
\end{pgfscope}%
\begin{pgfscope}%
\pgfpathrectangle{\pgfqpoint{0.617715in}{1.333988in}}{\pgfqpoint{5.466114in}{3.078389in}}%
\pgfusepath{clip}%
\pgfsetbuttcap%
\pgfsetroundjoin%
\pgfsetlinewidth{1.003750pt}%
\definecolor{currentstroke}{rgb}{1.000000,0.498039,0.054902}%
\pgfsetstrokecolor{currentstroke}%
\pgfsetdash{}{0pt}%
\pgfpathmoveto{\pgfqpoint{1.802040in}{1.674076in}}%
\pgfpathlineto{\pgfqpoint{1.984244in}{1.674076in}}%
\pgfusepath{stroke}%
\end{pgfscope}%
\begin{pgfscope}%
\pgfpathrectangle{\pgfqpoint{0.617715in}{1.333988in}}{\pgfqpoint{5.466114in}{3.078389in}}%
\pgfusepath{clip}%
\pgfsetbuttcap%
\pgfsetroundjoin%
\pgfsetlinewidth{1.003750pt}%
\definecolor{currentstroke}{rgb}{1.000000,0.498039,0.054902}%
\pgfsetstrokecolor{currentstroke}%
\pgfsetdash{}{0pt}%
\pgfpathmoveto{\pgfqpoint{2.166447in}{1.771674in}}%
\pgfpathlineto{\pgfqpoint{2.348651in}{1.771674in}}%
\pgfusepath{stroke}%
\end{pgfscope}%
\begin{pgfscope}%
\pgfpathrectangle{\pgfqpoint{0.617715in}{1.333988in}}{\pgfqpoint{5.466114in}{3.078389in}}%
\pgfusepath{clip}%
\pgfsetbuttcap%
\pgfsetroundjoin%
\pgfsetlinewidth{1.003750pt}%
\definecolor{currentstroke}{rgb}{1.000000,0.498039,0.054902}%
\pgfsetstrokecolor{currentstroke}%
\pgfsetdash{}{0pt}%
\pgfpathmoveto{\pgfqpoint{2.530855in}{1.553996in}}%
\pgfpathlineto{\pgfqpoint{2.713059in}{1.553996in}}%
\pgfusepath{stroke}%
\end{pgfscope}%
\begin{pgfscope}%
\pgfpathrectangle{\pgfqpoint{0.617715in}{1.333988in}}{\pgfqpoint{5.466114in}{3.078389in}}%
\pgfusepath{clip}%
\pgfsetbuttcap%
\pgfsetroundjoin%
\pgfsetlinewidth{1.003750pt}%
\definecolor{currentstroke}{rgb}{1.000000,0.498039,0.054902}%
\pgfsetstrokecolor{currentstroke}%
\pgfsetdash{}{0pt}%
\pgfpathmoveto{\pgfqpoint{2.895263in}{1.475563in}}%
\pgfpathlineto{\pgfqpoint{3.077466in}{1.475563in}}%
\pgfusepath{stroke}%
\end{pgfscope}%
\begin{pgfscope}%
\pgfpathrectangle{\pgfqpoint{0.617715in}{1.333988in}}{\pgfqpoint{5.466114in}{3.078389in}}%
\pgfusepath{clip}%
\pgfsetbuttcap%
\pgfsetroundjoin%
\pgfsetlinewidth{1.003750pt}%
\definecolor{currentstroke}{rgb}{1.000000,0.498039,0.054902}%
\pgfsetstrokecolor{currentstroke}%
\pgfsetdash{}{0pt}%
\pgfpathmoveto{\pgfqpoint{3.259670in}{1.478865in}}%
\pgfpathlineto{\pgfqpoint{3.441874in}{1.478865in}}%
\pgfusepath{stroke}%
\end{pgfscope}%
\begin{pgfscope}%
\pgfpathrectangle{\pgfqpoint{0.617715in}{1.333988in}}{\pgfqpoint{5.466114in}{3.078389in}}%
\pgfusepath{clip}%
\pgfsetbuttcap%
\pgfsetroundjoin%
\pgfsetlinewidth{1.003750pt}%
\definecolor{currentstroke}{rgb}{1.000000,0.498039,0.054902}%
\pgfsetstrokecolor{currentstroke}%
\pgfsetdash{}{0pt}%
\pgfusepath{stroke}%
\end{pgfscope}%
\begin{pgfscope}%
\pgfpathrectangle{\pgfqpoint{0.617715in}{1.333988in}}{\pgfqpoint{5.466114in}{3.078389in}}%
\pgfusepath{clip}%
\pgfsetbuttcap%
\pgfsetroundjoin%
\pgfsetlinewidth{1.003750pt}%
\definecolor{currentstroke}{rgb}{1.000000,0.498039,0.054902}%
\pgfsetstrokecolor{currentstroke}%
\pgfsetdash{}{0pt}%
\pgfpathmoveto{\pgfqpoint{3.988485in}{1.507218in}}%
\pgfpathlineto{\pgfqpoint{4.170689in}{1.507218in}}%
\pgfusepath{stroke}%
\end{pgfscope}%
\begin{pgfscope}%
\pgfpathrectangle{\pgfqpoint{0.617715in}{1.333988in}}{\pgfqpoint{5.466114in}{3.078389in}}%
\pgfusepath{clip}%
\pgfsetbuttcap%
\pgfsetroundjoin%
\pgfsetlinewidth{1.003750pt}%
\definecolor{currentstroke}{rgb}{1.000000,0.498039,0.054902}%
\pgfsetstrokecolor{currentstroke}%
\pgfsetdash{}{0pt}%
\pgfpathmoveto{\pgfqpoint{4.352893in}{1.474772in}}%
\pgfpathlineto{\pgfqpoint{4.535097in}{1.474772in}}%
\pgfusepath{stroke}%
\end{pgfscope}%
\begin{pgfscope}%
\pgfpathrectangle{\pgfqpoint{0.617715in}{1.333988in}}{\pgfqpoint{5.466114in}{3.078389in}}%
\pgfusepath{clip}%
\pgfsetbuttcap%
\pgfsetroundjoin%
\pgfsetlinewidth{1.003750pt}%
\definecolor{currentstroke}{rgb}{1.000000,0.498039,0.054902}%
\pgfsetstrokecolor{currentstroke}%
\pgfsetdash{}{0pt}%
\pgfpathmoveto{\pgfqpoint{4.717300in}{1.567223in}}%
\pgfpathlineto{\pgfqpoint{4.899504in}{1.567223in}}%
\pgfusepath{stroke}%
\end{pgfscope}%
\begin{pgfscope}%
\pgfpathrectangle{\pgfqpoint{0.617715in}{1.333988in}}{\pgfqpoint{5.466114in}{3.078389in}}%
\pgfusepath{clip}%
\pgfsetbuttcap%
\pgfsetroundjoin%
\pgfsetlinewidth{1.003750pt}%
\definecolor{currentstroke}{rgb}{1.000000,0.498039,0.054902}%
\pgfsetstrokecolor{currentstroke}%
\pgfsetdash{}{0pt}%
\pgfpathmoveto{\pgfqpoint{5.081708in}{1.565241in}}%
\pgfpathlineto{\pgfqpoint{5.263912in}{1.565241in}}%
\pgfusepath{stroke}%
\end{pgfscope}%
\begin{pgfscope}%
\pgfpathrectangle{\pgfqpoint{0.617715in}{1.333988in}}{\pgfqpoint{5.466114in}{3.078389in}}%
\pgfusepath{clip}%
\pgfsetbuttcap%
\pgfsetroundjoin%
\pgfsetlinewidth{1.003750pt}%
\definecolor{currentstroke}{rgb}{1.000000,0.498039,0.054902}%
\pgfsetstrokecolor{currentstroke}%
\pgfsetdash{}{0pt}%
\pgfpathmoveto{\pgfqpoint{5.446116in}{1.476602in}}%
\pgfpathlineto{\pgfqpoint{5.628319in}{1.476602in}}%
\pgfusepath{stroke}%
\end{pgfscope}%
\begin{pgfscope}%
\pgfpathrectangle{\pgfqpoint{0.617715in}{1.333988in}}{\pgfqpoint{5.466114in}{3.078389in}}%
\pgfusepath{clip}%
\pgfsetbuttcap%
\pgfsetroundjoin%
\pgfsetlinewidth{1.003750pt}%
\definecolor{currentstroke}{rgb}{1.000000,0.498039,0.054902}%
\pgfsetstrokecolor{currentstroke}%
\pgfsetdash{}{0pt}%
\pgfpathmoveto{\pgfqpoint{5.810523in}{1.478749in}}%
\pgfpathlineto{\pgfqpoint{5.992727in}{1.478749in}}%
\pgfusepath{stroke}%
\end{pgfscope}%
\begin{pgfscope}%
\pgfsetrectcap%
\pgfsetmiterjoin%
\pgfsetlinewidth{0.803000pt}%
\definecolor{currentstroke}{rgb}{0.000000,0.000000,0.000000}%
\pgfsetstrokecolor{currentstroke}%
\pgfsetdash{}{0pt}%
\pgfpathmoveto{\pgfqpoint{0.617715in}{1.333988in}}%
\pgfpathlineto{\pgfqpoint{0.617715in}{4.412376in}}%
\pgfusepath{stroke}%
\end{pgfscope}%
\begin{pgfscope}%
\pgfsetrectcap%
\pgfsetmiterjoin%
\pgfsetlinewidth{0.803000pt}%
\definecolor{currentstroke}{rgb}{0.000000,0.000000,0.000000}%
\pgfsetstrokecolor{currentstroke}%
\pgfsetdash{}{0pt}%
\pgfpathmoveto{\pgfqpoint{6.083829in}{1.333988in}}%
\pgfpathlineto{\pgfqpoint{6.083829in}{4.412376in}}%
\pgfusepath{stroke}%
\end{pgfscope}%
\begin{pgfscope}%
\pgfsetrectcap%
\pgfsetmiterjoin%
\pgfsetlinewidth{0.803000pt}%
\definecolor{currentstroke}{rgb}{0.000000,0.000000,0.000000}%
\pgfsetstrokecolor{currentstroke}%
\pgfsetdash{}{0pt}%
\pgfpathmoveto{\pgfqpoint{0.617715in}{1.333988in}}%
\pgfpathlineto{\pgfqpoint{6.083829in}{1.333988in}}%
\pgfusepath{stroke}%
\end{pgfscope}%
\begin{pgfscope}%
\pgfsetrectcap%
\pgfsetmiterjoin%
\pgfsetlinewidth{0.803000pt}%
\definecolor{currentstroke}{rgb}{0.000000,0.000000,0.000000}%
\pgfsetstrokecolor{currentstroke}%
\pgfsetdash{}{0pt}%
\pgfpathmoveto{\pgfqpoint{0.617715in}{4.412376in}}%
\pgfpathlineto{\pgfqpoint{6.083829in}{4.412376in}}%
\pgfusepath{stroke}%
\end{pgfscope}%
\begin{pgfscope}%
\definecolor{textcolor}{rgb}{0.000000,0.000000,0.000000}%
\pgfsetstrokecolor{textcolor}%
\pgfsetfillcolor{textcolor}%
\pgftext[x=3.350772in,y=4.495710in,,base]{\color{textcolor}{\sffamily\fontsize{13.200000}{15.840000}\selectfont\catcode`\^=\active\def^{\ifmmode\sp\else\^{}\fi}\catcode`\%=\active\def%{\%}IPC difference between flushing and not flushing per workload}}%
\end{pgfscope}%
\end{pgfpicture}%
\makeatother%
\endgroup%
}
    \resizebox{0.45\textwidth}{!}{%% Creator: Matplotlib, PGF backend
%%
%% To include the figure in your LaTeX document, write
%%   \input{<filename>.pgf}
%%
%% Make sure the required packages are loaded in your preamble
%%   \usepackage{pgf}
%%
%% Also ensure that all the required font packages are loaded; for instance,
%% the lmodern package is sometimes necessary when using math font.
%%   \usepackage{lmodern}
%%
%% Figures using additional raster images can only be included by \input if
%% they are in the same directory as the main LaTeX file. For loading figures
%% from other directories you can use the `import` package
%%   \usepackage{import}
%%
%% and then include the figures with
%%   \import{<path to file>}{<filename>.pgf}
%%
%% Matplotlib used the following preamble
%%   \def\mathdefault#1{#1}
%%   \everymath=\expandafter{\the\everymath\displaystyle}
%%   
%%   \usepackage{fontspec}
%%   \setmainfont{DejaVuSerif.ttf}[Path=\detokenize{/usr/lib/python3.12/site-packages/matplotlib/mpl-data/fonts/ttf/}]
%%   \setsansfont{DejaVuSans.ttf}[Path=\detokenize{/usr/lib/python3.12/site-packages/matplotlib/mpl-data/fonts/ttf/}]
%%   \setmonofont{DejaVuSansMono.ttf}[Path=\detokenize{/usr/lib/python3.12/site-packages/matplotlib/mpl-data/fonts/ttf/}]
%%   \makeatletter\@ifpackageloaded{underscore}{}{\usepackage[strings]{underscore}}\makeatother
%%
\begingroup%
\makeatletter%
\begin{pgfpicture}%
\pgfpathrectangle{\pgfpointorigin}{\pgfqpoint{6.400000in}{4.800000in}}%
\pgfusepath{use as bounding box, clip}%
\begin{pgfscope}%
\pgfsetbuttcap%
\pgfsetmiterjoin%
\definecolor{currentfill}{rgb}{1.000000,1.000000,1.000000}%
\pgfsetfillcolor{currentfill}%
\pgfsetlinewidth{0.000000pt}%
\definecolor{currentstroke}{rgb}{1.000000,1.000000,1.000000}%
\pgfsetstrokecolor{currentstroke}%
\pgfsetdash{}{0pt}%
\pgfpathmoveto{\pgfqpoint{0.000000in}{0.000000in}}%
\pgfpathlineto{\pgfqpoint{6.400000in}{0.000000in}}%
\pgfpathlineto{\pgfqpoint{6.400000in}{4.800000in}}%
\pgfpathlineto{\pgfqpoint{0.000000in}{4.800000in}}%
\pgfpathlineto{\pgfqpoint{0.000000in}{0.000000in}}%
\pgfpathclose%
\pgfusepath{fill}%
\end{pgfscope}%
\begin{pgfscope}%
\pgfsetbuttcap%
\pgfsetmiterjoin%
\definecolor{currentfill}{rgb}{1.000000,1.000000,1.000000}%
\pgfsetfillcolor{currentfill}%
\pgfsetlinewidth{0.000000pt}%
\definecolor{currentstroke}{rgb}{0.000000,0.000000,0.000000}%
\pgfsetstrokecolor{currentstroke}%
\pgfsetstrokeopacity{0.000000}%
\pgfsetdash{}{0pt}%
\pgfpathmoveto{\pgfqpoint{0.617715in}{0.901550in}}%
\pgfpathlineto{\pgfqpoint{6.235000in}{0.901550in}}%
\pgfpathlineto{\pgfqpoint{6.235000in}{4.412376in}}%
\pgfpathlineto{\pgfqpoint{0.617715in}{4.412376in}}%
\pgfpathlineto{\pgfqpoint{0.617715in}{0.901550in}}%
\pgfpathclose%
\pgfusepath{fill}%
\end{pgfscope}%
\begin{pgfscope}%
\pgfsetbuttcap%
\pgfsetroundjoin%
\definecolor{currentfill}{rgb}{0.000000,0.000000,0.000000}%
\pgfsetfillcolor{currentfill}%
\pgfsetlinewidth{0.803000pt}%
\definecolor{currentstroke}{rgb}{0.000000,0.000000,0.000000}%
\pgfsetstrokecolor{currentstroke}%
\pgfsetdash{}{0pt}%
\pgfsys@defobject{currentmarker}{\pgfqpoint{0.000000in}{-0.048611in}}{\pgfqpoint{0.000000in}{0.000000in}}{%
\pgfpathmoveto{\pgfqpoint{0.000000in}{0.000000in}}%
\pgfpathlineto{\pgfqpoint{0.000000in}{-0.048611in}}%
\pgfusepath{stroke,fill}%
}%
\begin{pgfscope}%
\pgfsys@transformshift{1.319876in}{0.901550in}%
\pgfsys@useobject{currentmarker}{}%
\end{pgfscope}%
\end{pgfscope}%
\begin{pgfscope}%
\definecolor{textcolor}{rgb}{0.000000,0.000000,0.000000}%
\pgfsetstrokecolor{textcolor}%
\pgfsetfillcolor{textcolor}%
\pgftext[x=1.183994in, y=0.390881in, left, base,rotate=45.000000]{\color{textcolor}{\sffamily\fontsize{11.000000}{13.200000}\selectfont\catcode`\^=\active\def^{\ifmmode\sp\else\^{}\fi}\catcode`\%=\active\def%{\%}cuda1}}%
\end{pgfscope}%
\begin{pgfscope}%
\pgfsetbuttcap%
\pgfsetroundjoin%
\definecolor{currentfill}{rgb}{0.000000,0.000000,0.000000}%
\pgfsetfillcolor{currentfill}%
\pgfsetlinewidth{0.803000pt}%
\definecolor{currentstroke}{rgb}{0.000000,0.000000,0.000000}%
\pgfsetstrokecolor{currentstroke}%
\pgfsetdash{}{0pt}%
\pgfsys@defobject{currentmarker}{\pgfqpoint{0.000000in}{-0.048611in}}{\pgfqpoint{0.000000in}{0.000000in}}{%
\pgfpathmoveto{\pgfqpoint{0.000000in}{0.000000in}}%
\pgfpathlineto{\pgfqpoint{0.000000in}{-0.048611in}}%
\pgfusepath{stroke,fill}%
}%
\begin{pgfscope}%
\pgfsys@transformshift{2.724197in}{0.901550in}%
\pgfsys@useobject{currentmarker}{}%
\end{pgfscope}%
\end{pgfscope}%
\begin{pgfscope}%
\definecolor{textcolor}{rgb}{0.000000,0.000000,0.000000}%
\pgfsetstrokecolor{textcolor}%
\pgfsetfillcolor{textcolor}%
\pgftext[x=2.588315in, y=0.390881in, left, base,rotate=45.000000]{\color{textcolor}{\sffamily\fontsize{11.000000}{13.200000}\selectfont\catcode`\^=\active\def^{\ifmmode\sp\else\^{}\fi}\catcode`\%=\active\def%{\%}cuda2}}%
\end{pgfscope}%
\begin{pgfscope}%
\pgfsetbuttcap%
\pgfsetroundjoin%
\definecolor{currentfill}{rgb}{0.000000,0.000000,0.000000}%
\pgfsetfillcolor{currentfill}%
\pgfsetlinewidth{0.803000pt}%
\definecolor{currentstroke}{rgb}{0.000000,0.000000,0.000000}%
\pgfsetstrokecolor{currentstroke}%
\pgfsetdash{}{0pt}%
\pgfsys@defobject{currentmarker}{\pgfqpoint{0.000000in}{-0.048611in}}{\pgfqpoint{0.000000in}{0.000000in}}{%
\pgfpathmoveto{\pgfqpoint{0.000000in}{0.000000in}}%
\pgfpathlineto{\pgfqpoint{0.000000in}{-0.048611in}}%
\pgfusepath{stroke,fill}%
}%
\begin{pgfscope}%
\pgfsys@transformshift{4.128518in}{0.901550in}%
\pgfsys@useobject{currentmarker}{}%
\end{pgfscope}%
\end{pgfscope}%
\begin{pgfscope}%
\definecolor{textcolor}{rgb}{0.000000,0.000000,0.000000}%
\pgfsetstrokecolor{textcolor}%
\pgfsetfillcolor{textcolor}%
\pgftext[x=4.007327in, y=0.420262in, left, base,rotate=45.000000]{\color{textcolor}{\sffamily\fontsize{11.000000}{13.200000}\selectfont\catcode`\^=\active\def^{\ifmmode\sp\else\^{}\fi}\catcode`\%=\active\def%{\%}gold1}}%
\end{pgfscope}%
\begin{pgfscope}%
\pgfsetbuttcap%
\pgfsetroundjoin%
\definecolor{currentfill}{rgb}{0.000000,0.000000,0.000000}%
\pgfsetfillcolor{currentfill}%
\pgfsetlinewidth{0.803000pt}%
\definecolor{currentstroke}{rgb}{0.000000,0.000000,0.000000}%
\pgfsetstrokecolor{currentstroke}%
\pgfsetdash{}{0pt}%
\pgfsys@defobject{currentmarker}{\pgfqpoint{0.000000in}{-0.048611in}}{\pgfqpoint{0.000000in}{0.000000in}}{%
\pgfpathmoveto{\pgfqpoint{0.000000in}{0.000000in}}%
\pgfpathlineto{\pgfqpoint{0.000000in}{-0.048611in}}%
\pgfusepath{stroke,fill}%
}%
\begin{pgfscope}%
\pgfsys@transformshift{5.532839in}{0.901550in}%
\pgfsys@useobject{currentmarker}{}%
\end{pgfscope}%
\end{pgfscope}%
\begin{pgfscope}%
\definecolor{textcolor}{rgb}{0.000000,0.000000,0.000000}%
\pgfsetstrokecolor{textcolor}%
\pgfsetfillcolor{textcolor}%
\pgftext[x=5.411648in, y=0.420262in, left, base,rotate=45.000000]{\color{textcolor}{\sffamily\fontsize{11.000000}{13.200000}\selectfont\catcode`\^=\active\def^{\ifmmode\sp\else\^{}\fi}\catcode`\%=\active\def%{\%}gold2}}%
\end{pgfscope}%
\begin{pgfscope}%
\definecolor{textcolor}{rgb}{0.000000,0.000000,0.000000}%
\pgfsetstrokecolor{textcolor}%
\pgfsetfillcolor{textcolor}%
\pgftext[x=3.426358in,y=0.312854in,,top]{\color{textcolor}{\sffamily\fontsize{11.000000}{13.200000}\selectfont\catcode`\^=\active\def^{\ifmmode\sp\else\^{}\fi}\catcode`\%=\active\def%{\%}Workload}}%
\end{pgfscope}%
\begin{pgfscope}%
\pgfsetbuttcap%
\pgfsetroundjoin%
\definecolor{currentfill}{rgb}{0.000000,0.000000,0.000000}%
\pgfsetfillcolor{currentfill}%
\pgfsetlinewidth{0.803000pt}%
\definecolor{currentstroke}{rgb}{0.000000,0.000000,0.000000}%
\pgfsetstrokecolor{currentstroke}%
\pgfsetdash{}{0pt}%
\pgfsys@defobject{currentmarker}{\pgfqpoint{-0.048611in}{0.000000in}}{\pgfqpoint{-0.000000in}{0.000000in}}{%
\pgfpathmoveto{\pgfqpoint{-0.000000in}{0.000000in}}%
\pgfpathlineto{\pgfqpoint{-0.048611in}{0.000000in}}%
\pgfusepath{stroke,fill}%
}%
\begin{pgfscope}%
\pgfsys@transformshift{0.617715in}{1.032014in}%
\pgfsys@useobject{currentmarker}{}%
\end{pgfscope}%
\end{pgfscope}%
\begin{pgfscope}%
\definecolor{textcolor}{rgb}{0.000000,0.000000,0.000000}%
\pgfsetstrokecolor{textcolor}%
\pgfsetfillcolor{textcolor}%
\pgftext[x=0.444451in, y=0.973977in, left, base]{\color{textcolor}{\sffamily\fontsize{11.000000}{13.200000}\selectfont\catcode`\^=\active\def^{\ifmmode\sp\else\^{}\fi}\catcode`\%=\active\def%{\%}$\mathdefault{0}$}}%
\end{pgfscope}%
\begin{pgfscope}%
\pgfsetbuttcap%
\pgfsetroundjoin%
\definecolor{currentfill}{rgb}{0.000000,0.000000,0.000000}%
\pgfsetfillcolor{currentfill}%
\pgfsetlinewidth{0.803000pt}%
\definecolor{currentstroke}{rgb}{0.000000,0.000000,0.000000}%
\pgfsetstrokecolor{currentstroke}%
\pgfsetdash{}{0pt}%
\pgfsys@defobject{currentmarker}{\pgfqpoint{-0.048611in}{0.000000in}}{\pgfqpoint{-0.000000in}{0.000000in}}{%
\pgfpathmoveto{\pgfqpoint{-0.000000in}{0.000000in}}%
\pgfpathlineto{\pgfqpoint{-0.048611in}{0.000000in}}%
\pgfusepath{stroke,fill}%
}%
\begin{pgfscope}%
\pgfsys@transformshift{0.617715in}{1.773894in}%
\pgfsys@useobject{currentmarker}{}%
\end{pgfscope}%
\end{pgfscope}%
\begin{pgfscope}%
\definecolor{textcolor}{rgb}{0.000000,0.000000,0.000000}%
\pgfsetstrokecolor{textcolor}%
\pgfsetfillcolor{textcolor}%
\pgftext[x=0.444451in, y=1.715856in, left, base]{\color{textcolor}{\sffamily\fontsize{11.000000}{13.200000}\selectfont\catcode`\^=\active\def^{\ifmmode\sp\else\^{}\fi}\catcode`\%=\active\def%{\%}$\mathdefault{5}$}}%
\end{pgfscope}%
\begin{pgfscope}%
\pgfsetbuttcap%
\pgfsetroundjoin%
\definecolor{currentfill}{rgb}{0.000000,0.000000,0.000000}%
\pgfsetfillcolor{currentfill}%
\pgfsetlinewidth{0.803000pt}%
\definecolor{currentstroke}{rgb}{0.000000,0.000000,0.000000}%
\pgfsetstrokecolor{currentstroke}%
\pgfsetdash{}{0pt}%
\pgfsys@defobject{currentmarker}{\pgfqpoint{-0.048611in}{0.000000in}}{\pgfqpoint{-0.000000in}{0.000000in}}{%
\pgfpathmoveto{\pgfqpoint{-0.000000in}{0.000000in}}%
\pgfpathlineto{\pgfqpoint{-0.048611in}{0.000000in}}%
\pgfusepath{stroke,fill}%
}%
\begin{pgfscope}%
\pgfsys@transformshift{0.617715in}{2.515773in}%
\pgfsys@useobject{currentmarker}{}%
\end{pgfscope}%
\end{pgfscope}%
\begin{pgfscope}%
\definecolor{textcolor}{rgb}{0.000000,0.000000,0.000000}%
\pgfsetstrokecolor{textcolor}%
\pgfsetfillcolor{textcolor}%
\pgftext[x=0.368410in, y=2.457736in, left, base]{\color{textcolor}{\sffamily\fontsize{11.000000}{13.200000}\selectfont\catcode`\^=\active\def^{\ifmmode\sp\else\^{}\fi}\catcode`\%=\active\def%{\%}$\mathdefault{10}$}}%
\end{pgfscope}%
\begin{pgfscope}%
\pgfsetbuttcap%
\pgfsetroundjoin%
\definecolor{currentfill}{rgb}{0.000000,0.000000,0.000000}%
\pgfsetfillcolor{currentfill}%
\pgfsetlinewidth{0.803000pt}%
\definecolor{currentstroke}{rgb}{0.000000,0.000000,0.000000}%
\pgfsetstrokecolor{currentstroke}%
\pgfsetdash{}{0pt}%
\pgfsys@defobject{currentmarker}{\pgfqpoint{-0.048611in}{0.000000in}}{\pgfqpoint{-0.000000in}{0.000000in}}{%
\pgfpathmoveto{\pgfqpoint{-0.000000in}{0.000000in}}%
\pgfpathlineto{\pgfqpoint{-0.048611in}{0.000000in}}%
\pgfusepath{stroke,fill}%
}%
\begin{pgfscope}%
\pgfsys@transformshift{0.617715in}{3.257653in}%
\pgfsys@useobject{currentmarker}{}%
\end{pgfscope}%
\end{pgfscope}%
\begin{pgfscope}%
\definecolor{textcolor}{rgb}{0.000000,0.000000,0.000000}%
\pgfsetstrokecolor{textcolor}%
\pgfsetfillcolor{textcolor}%
\pgftext[x=0.368410in, y=3.199615in, left, base]{\color{textcolor}{\sffamily\fontsize{11.000000}{13.200000}\selectfont\catcode`\^=\active\def^{\ifmmode\sp\else\^{}\fi}\catcode`\%=\active\def%{\%}$\mathdefault{15}$}}%
\end{pgfscope}%
\begin{pgfscope}%
\pgfsetbuttcap%
\pgfsetroundjoin%
\definecolor{currentfill}{rgb}{0.000000,0.000000,0.000000}%
\pgfsetfillcolor{currentfill}%
\pgfsetlinewidth{0.803000pt}%
\definecolor{currentstroke}{rgb}{0.000000,0.000000,0.000000}%
\pgfsetstrokecolor{currentstroke}%
\pgfsetdash{}{0pt}%
\pgfsys@defobject{currentmarker}{\pgfqpoint{-0.048611in}{0.000000in}}{\pgfqpoint{-0.000000in}{0.000000in}}{%
\pgfpathmoveto{\pgfqpoint{-0.000000in}{0.000000in}}%
\pgfpathlineto{\pgfqpoint{-0.048611in}{0.000000in}}%
\pgfusepath{stroke,fill}%
}%
\begin{pgfscope}%
\pgfsys@transformshift{0.617715in}{3.999532in}%
\pgfsys@useobject{currentmarker}{}%
\end{pgfscope}%
\end{pgfscope}%
\begin{pgfscope}%
\definecolor{textcolor}{rgb}{0.000000,0.000000,0.000000}%
\pgfsetstrokecolor{textcolor}%
\pgfsetfillcolor{textcolor}%
\pgftext[x=0.368410in, y=3.941494in, left, base]{\color{textcolor}{\sffamily\fontsize{11.000000}{13.200000}\selectfont\catcode`\^=\active\def^{\ifmmode\sp\else\^{}\fi}\catcode`\%=\active\def%{\%}$\mathdefault{20}$}}%
\end{pgfscope}%
\begin{pgfscope}%
\definecolor{textcolor}{rgb}{0.000000,0.000000,0.000000}%
\pgfsetstrokecolor{textcolor}%
\pgfsetfillcolor{textcolor}%
\pgftext[x=0.312854in,y=2.656963in,,bottom,rotate=90.000000]{\color{textcolor}{\sffamily\fontsize{11.000000}{13.200000}\selectfont\catcode`\^=\active\def^{\ifmmode\sp\else\^{}\fi}\catcode`\%=\active\def%{\%}Relative IPC difference (%)}}%
\end{pgfscope}%
\begin{pgfscope}%
\pgfpathrectangle{\pgfqpoint{0.617715in}{0.901550in}}{\pgfqpoint{5.617285in}{3.510826in}}%
\pgfusepath{clip}%
\pgfsetrectcap%
\pgfsetroundjoin%
\pgfsetlinewidth{1.003750pt}%
\definecolor{currentstroke}{rgb}{0.000000,0.000000,0.000000}%
\pgfsetstrokecolor{currentstroke}%
\pgfsetdash{}{0pt}%
\pgfpathmoveto{\pgfqpoint{1.003904in}{1.791572in}}%
\pgfpathlineto{\pgfqpoint{1.635848in}{1.791572in}}%
\pgfpathlineto{\pgfqpoint{1.635848in}{3.646368in}}%
\pgfpathlineto{\pgfqpoint{1.003904in}{3.646368in}}%
\pgfpathlineto{\pgfqpoint{1.003904in}{1.791572in}}%
\pgfusepath{stroke}%
\end{pgfscope}%
\begin{pgfscope}%
\pgfpathrectangle{\pgfqpoint{0.617715in}{0.901550in}}{\pgfqpoint{5.617285in}{3.510826in}}%
\pgfusepath{clip}%
\pgfsetrectcap%
\pgfsetroundjoin%
\pgfsetlinewidth{1.003750pt}%
\definecolor{currentstroke}{rgb}{0.000000,0.000000,0.000000}%
\pgfsetstrokecolor{currentstroke}%
\pgfsetdash{}{0pt}%
\pgfpathmoveto{\pgfqpoint{1.319876in}{1.791572in}}%
\pgfpathlineto{\pgfqpoint{1.319876in}{1.091361in}}%
\pgfusepath{stroke}%
\end{pgfscope}%
\begin{pgfscope}%
\pgfpathrectangle{\pgfqpoint{0.617715in}{0.901550in}}{\pgfqpoint{5.617285in}{3.510826in}}%
\pgfusepath{clip}%
\pgfsetrectcap%
\pgfsetroundjoin%
\pgfsetlinewidth{1.003750pt}%
\definecolor{currentstroke}{rgb}{0.000000,0.000000,0.000000}%
\pgfsetstrokecolor{currentstroke}%
\pgfsetdash{}{0pt}%
\pgfpathmoveto{\pgfqpoint{1.319876in}{3.646368in}}%
\pgfpathlineto{\pgfqpoint{1.319876in}{4.206533in}}%
\pgfusepath{stroke}%
\end{pgfscope}%
\begin{pgfscope}%
\pgfpathrectangle{\pgfqpoint{0.617715in}{0.901550in}}{\pgfqpoint{5.617285in}{3.510826in}}%
\pgfusepath{clip}%
\pgfsetrectcap%
\pgfsetroundjoin%
\pgfsetlinewidth{1.003750pt}%
\definecolor{currentstroke}{rgb}{0.000000,0.000000,0.000000}%
\pgfsetstrokecolor{currentstroke}%
\pgfsetdash{}{0pt}%
\pgfpathmoveto{\pgfqpoint{1.161890in}{1.091361in}}%
\pgfpathlineto{\pgfqpoint{1.477862in}{1.091361in}}%
\pgfusepath{stroke}%
\end{pgfscope}%
\begin{pgfscope}%
\pgfpathrectangle{\pgfqpoint{0.617715in}{0.901550in}}{\pgfqpoint{5.617285in}{3.510826in}}%
\pgfusepath{clip}%
\pgfsetrectcap%
\pgfsetroundjoin%
\pgfsetlinewidth{1.003750pt}%
\definecolor{currentstroke}{rgb}{0.000000,0.000000,0.000000}%
\pgfsetstrokecolor{currentstroke}%
\pgfsetdash{}{0pt}%
\pgfpathmoveto{\pgfqpoint{1.161890in}{4.206533in}}%
\pgfpathlineto{\pgfqpoint{1.477862in}{4.206533in}}%
\pgfusepath{stroke}%
\end{pgfscope}%
\begin{pgfscope}%
\pgfpathrectangle{\pgfqpoint{0.617715in}{0.901550in}}{\pgfqpoint{5.617285in}{3.510826in}}%
\pgfusepath{clip}%
\pgfsetrectcap%
\pgfsetroundjoin%
\pgfsetlinewidth{1.003750pt}%
\definecolor{currentstroke}{rgb}{0.000000,0.000000,0.000000}%
\pgfsetstrokecolor{currentstroke}%
\pgfsetdash{}{0pt}%
\pgfpathmoveto{\pgfqpoint{2.408225in}{1.728211in}}%
\pgfpathlineto{\pgfqpoint{3.040169in}{1.728211in}}%
\pgfpathlineto{\pgfqpoint{3.040169in}{3.605376in}}%
\pgfpathlineto{\pgfqpoint{2.408225in}{3.605376in}}%
\pgfpathlineto{\pgfqpoint{2.408225in}{1.728211in}}%
\pgfusepath{stroke}%
\end{pgfscope}%
\begin{pgfscope}%
\pgfpathrectangle{\pgfqpoint{0.617715in}{0.901550in}}{\pgfqpoint{5.617285in}{3.510826in}}%
\pgfusepath{clip}%
\pgfsetrectcap%
\pgfsetroundjoin%
\pgfsetlinewidth{1.003750pt}%
\definecolor{currentstroke}{rgb}{0.000000,0.000000,0.000000}%
\pgfsetstrokecolor{currentstroke}%
\pgfsetdash{}{0pt}%
\pgfpathmoveto{\pgfqpoint{2.724197in}{1.728211in}}%
\pgfpathlineto{\pgfqpoint{2.724197in}{1.091871in}}%
\pgfusepath{stroke}%
\end{pgfscope}%
\begin{pgfscope}%
\pgfpathrectangle{\pgfqpoint{0.617715in}{0.901550in}}{\pgfqpoint{5.617285in}{3.510826in}}%
\pgfusepath{clip}%
\pgfsetrectcap%
\pgfsetroundjoin%
\pgfsetlinewidth{1.003750pt}%
\definecolor{currentstroke}{rgb}{0.000000,0.000000,0.000000}%
\pgfsetstrokecolor{currentstroke}%
\pgfsetdash{}{0pt}%
\pgfpathmoveto{\pgfqpoint{2.724197in}{3.605376in}}%
\pgfpathlineto{\pgfqpoint{2.724197in}{3.717022in}}%
\pgfusepath{stroke}%
\end{pgfscope}%
\begin{pgfscope}%
\pgfpathrectangle{\pgfqpoint{0.617715in}{0.901550in}}{\pgfqpoint{5.617285in}{3.510826in}}%
\pgfusepath{clip}%
\pgfsetrectcap%
\pgfsetroundjoin%
\pgfsetlinewidth{1.003750pt}%
\definecolor{currentstroke}{rgb}{0.000000,0.000000,0.000000}%
\pgfsetstrokecolor{currentstroke}%
\pgfsetdash{}{0pt}%
\pgfpathmoveto{\pgfqpoint{2.566211in}{1.091871in}}%
\pgfpathlineto{\pgfqpoint{2.882183in}{1.091871in}}%
\pgfusepath{stroke}%
\end{pgfscope}%
\begin{pgfscope}%
\pgfpathrectangle{\pgfqpoint{0.617715in}{0.901550in}}{\pgfqpoint{5.617285in}{3.510826in}}%
\pgfusepath{clip}%
\pgfsetrectcap%
\pgfsetroundjoin%
\pgfsetlinewidth{1.003750pt}%
\definecolor{currentstroke}{rgb}{0.000000,0.000000,0.000000}%
\pgfsetstrokecolor{currentstroke}%
\pgfsetdash{}{0pt}%
\pgfpathmoveto{\pgfqpoint{2.566211in}{3.717022in}}%
\pgfpathlineto{\pgfqpoint{2.882183in}{3.717022in}}%
\pgfusepath{stroke}%
\end{pgfscope}%
\begin{pgfscope}%
\pgfpathrectangle{\pgfqpoint{0.617715in}{0.901550in}}{\pgfqpoint{5.617285in}{3.510826in}}%
\pgfusepath{clip}%
\pgfsetrectcap%
\pgfsetroundjoin%
\pgfsetlinewidth{1.003750pt}%
\definecolor{currentstroke}{rgb}{0.000000,0.000000,0.000000}%
\pgfsetstrokecolor{currentstroke}%
\pgfsetdash{}{0pt}%
\pgfpathmoveto{\pgfqpoint{3.812546in}{1.828071in}}%
\pgfpathlineto{\pgfqpoint{4.444490in}{1.828071in}}%
\pgfpathlineto{\pgfqpoint{4.444490in}{3.772543in}}%
\pgfpathlineto{\pgfqpoint{3.812546in}{3.772543in}}%
\pgfpathlineto{\pgfqpoint{3.812546in}{1.828071in}}%
\pgfusepath{stroke}%
\end{pgfscope}%
\begin{pgfscope}%
\pgfpathrectangle{\pgfqpoint{0.617715in}{0.901550in}}{\pgfqpoint{5.617285in}{3.510826in}}%
\pgfusepath{clip}%
\pgfsetrectcap%
\pgfsetroundjoin%
\pgfsetlinewidth{1.003750pt}%
\definecolor{currentstroke}{rgb}{0.000000,0.000000,0.000000}%
\pgfsetstrokecolor{currentstroke}%
\pgfsetdash{}{0pt}%
\pgfpathmoveto{\pgfqpoint{4.128518in}{1.828071in}}%
\pgfpathlineto{\pgfqpoint{4.128518in}{1.296189in}}%
\pgfusepath{stroke}%
\end{pgfscope}%
\begin{pgfscope}%
\pgfpathrectangle{\pgfqpoint{0.617715in}{0.901550in}}{\pgfqpoint{5.617285in}{3.510826in}}%
\pgfusepath{clip}%
\pgfsetrectcap%
\pgfsetroundjoin%
\pgfsetlinewidth{1.003750pt}%
\definecolor{currentstroke}{rgb}{0.000000,0.000000,0.000000}%
\pgfsetstrokecolor{currentstroke}%
\pgfsetdash{}{0pt}%
\pgfpathmoveto{\pgfqpoint{4.128518in}{3.772543in}}%
\pgfpathlineto{\pgfqpoint{4.128518in}{4.252793in}}%
\pgfusepath{stroke}%
\end{pgfscope}%
\begin{pgfscope}%
\pgfpathrectangle{\pgfqpoint{0.617715in}{0.901550in}}{\pgfqpoint{5.617285in}{3.510826in}}%
\pgfusepath{clip}%
\pgfsetrectcap%
\pgfsetroundjoin%
\pgfsetlinewidth{1.003750pt}%
\definecolor{currentstroke}{rgb}{0.000000,0.000000,0.000000}%
\pgfsetstrokecolor{currentstroke}%
\pgfsetdash{}{0pt}%
\pgfpathmoveto{\pgfqpoint{3.970532in}{1.296189in}}%
\pgfpathlineto{\pgfqpoint{4.286504in}{1.296189in}}%
\pgfusepath{stroke}%
\end{pgfscope}%
\begin{pgfscope}%
\pgfpathrectangle{\pgfqpoint{0.617715in}{0.901550in}}{\pgfqpoint{5.617285in}{3.510826in}}%
\pgfusepath{clip}%
\pgfsetrectcap%
\pgfsetroundjoin%
\pgfsetlinewidth{1.003750pt}%
\definecolor{currentstroke}{rgb}{0.000000,0.000000,0.000000}%
\pgfsetstrokecolor{currentstroke}%
\pgfsetdash{}{0pt}%
\pgfpathmoveto{\pgfqpoint{3.970532in}{4.252793in}}%
\pgfpathlineto{\pgfqpoint{4.286504in}{4.252793in}}%
\pgfusepath{stroke}%
\end{pgfscope}%
\begin{pgfscope}%
\pgfpathrectangle{\pgfqpoint{0.617715in}{0.901550in}}{\pgfqpoint{5.617285in}{3.510826in}}%
\pgfusepath{clip}%
\pgfsetrectcap%
\pgfsetroundjoin%
\pgfsetlinewidth{1.003750pt}%
\definecolor{currentstroke}{rgb}{0.000000,0.000000,0.000000}%
\pgfsetstrokecolor{currentstroke}%
\pgfsetdash{}{0pt}%
\pgfpathmoveto{\pgfqpoint{5.216867in}{1.532572in}}%
\pgfpathlineto{\pgfqpoint{5.848812in}{1.532572in}}%
\pgfpathlineto{\pgfqpoint{5.848812in}{3.596202in}}%
\pgfpathlineto{\pgfqpoint{5.216867in}{3.596202in}}%
\pgfpathlineto{\pgfqpoint{5.216867in}{1.532572in}}%
\pgfusepath{stroke}%
\end{pgfscope}%
\begin{pgfscope}%
\pgfpathrectangle{\pgfqpoint{0.617715in}{0.901550in}}{\pgfqpoint{5.617285in}{3.510826in}}%
\pgfusepath{clip}%
\pgfsetrectcap%
\pgfsetroundjoin%
\pgfsetlinewidth{1.003750pt}%
\definecolor{currentstroke}{rgb}{0.000000,0.000000,0.000000}%
\pgfsetstrokecolor{currentstroke}%
\pgfsetdash{}{0pt}%
\pgfpathmoveto{\pgfqpoint{5.532839in}{1.532572in}}%
\pgfpathlineto{\pgfqpoint{5.532839in}{1.061133in}}%
\pgfusepath{stroke}%
\end{pgfscope}%
\begin{pgfscope}%
\pgfpathrectangle{\pgfqpoint{0.617715in}{0.901550in}}{\pgfqpoint{5.617285in}{3.510826in}}%
\pgfusepath{clip}%
\pgfsetrectcap%
\pgfsetroundjoin%
\pgfsetlinewidth{1.003750pt}%
\definecolor{currentstroke}{rgb}{0.000000,0.000000,0.000000}%
\pgfsetstrokecolor{currentstroke}%
\pgfsetdash{}{0pt}%
\pgfpathmoveto{\pgfqpoint{5.532839in}{3.596202in}}%
\pgfpathlineto{\pgfqpoint{5.532839in}{3.914351in}}%
\pgfusepath{stroke}%
\end{pgfscope}%
\begin{pgfscope}%
\pgfpathrectangle{\pgfqpoint{0.617715in}{0.901550in}}{\pgfqpoint{5.617285in}{3.510826in}}%
\pgfusepath{clip}%
\pgfsetrectcap%
\pgfsetroundjoin%
\pgfsetlinewidth{1.003750pt}%
\definecolor{currentstroke}{rgb}{0.000000,0.000000,0.000000}%
\pgfsetstrokecolor{currentstroke}%
\pgfsetdash{}{0pt}%
\pgfpathmoveto{\pgfqpoint{5.374853in}{1.061133in}}%
\pgfpathlineto{\pgfqpoint{5.690826in}{1.061133in}}%
\pgfusepath{stroke}%
\end{pgfscope}%
\begin{pgfscope}%
\pgfpathrectangle{\pgfqpoint{0.617715in}{0.901550in}}{\pgfqpoint{5.617285in}{3.510826in}}%
\pgfusepath{clip}%
\pgfsetrectcap%
\pgfsetroundjoin%
\pgfsetlinewidth{1.003750pt}%
\definecolor{currentstroke}{rgb}{0.000000,0.000000,0.000000}%
\pgfsetstrokecolor{currentstroke}%
\pgfsetdash{}{0pt}%
\pgfpathmoveto{\pgfqpoint{5.374853in}{3.914351in}}%
\pgfpathlineto{\pgfqpoint{5.690826in}{3.914351in}}%
\pgfusepath{stroke}%
\end{pgfscope}%
\begin{pgfscope}%
\pgfpathrectangle{\pgfqpoint{0.617715in}{0.901550in}}{\pgfqpoint{5.617285in}{3.510826in}}%
\pgfusepath{clip}%
\pgfsetbuttcap%
\pgfsetroundjoin%
\pgfsetlinewidth{1.003750pt}%
\definecolor{currentstroke}{rgb}{1.000000,0.498039,0.054902}%
\pgfsetstrokecolor{currentstroke}%
\pgfsetdash{}{0pt}%
\pgfpathmoveto{\pgfqpoint{1.003904in}{2.855621in}}%
\pgfpathlineto{\pgfqpoint{1.635848in}{2.855621in}}%
\pgfusepath{stroke}%
\end{pgfscope}%
\begin{pgfscope}%
\pgfpathrectangle{\pgfqpoint{0.617715in}{0.901550in}}{\pgfqpoint{5.617285in}{3.510826in}}%
\pgfusepath{clip}%
\pgfsetbuttcap%
\pgfsetroundjoin%
\pgfsetlinewidth{1.003750pt}%
\definecolor{currentstroke}{rgb}{1.000000,0.498039,0.054902}%
\pgfsetstrokecolor{currentstroke}%
\pgfsetdash{}{0pt}%
\pgfpathmoveto{\pgfqpoint{2.408225in}{2.054138in}}%
\pgfpathlineto{\pgfqpoint{3.040169in}{2.054138in}}%
\pgfusepath{stroke}%
\end{pgfscope}%
\begin{pgfscope}%
\pgfpathrectangle{\pgfqpoint{0.617715in}{0.901550in}}{\pgfqpoint{5.617285in}{3.510826in}}%
\pgfusepath{clip}%
\pgfsetbuttcap%
\pgfsetroundjoin%
\pgfsetlinewidth{1.003750pt}%
\definecolor{currentstroke}{rgb}{1.000000,0.498039,0.054902}%
\pgfsetstrokecolor{currentstroke}%
\pgfsetdash{}{0pt}%
\pgfpathmoveto{\pgfqpoint{3.812546in}{2.205595in}}%
\pgfpathlineto{\pgfqpoint{4.444490in}{2.205595in}}%
\pgfusepath{stroke}%
\end{pgfscope}%
\begin{pgfscope}%
\pgfpathrectangle{\pgfqpoint{0.617715in}{0.901550in}}{\pgfqpoint{5.617285in}{3.510826in}}%
\pgfusepath{clip}%
\pgfsetbuttcap%
\pgfsetroundjoin%
\pgfsetlinewidth{1.003750pt}%
\definecolor{currentstroke}{rgb}{1.000000,0.498039,0.054902}%
\pgfsetstrokecolor{currentstroke}%
\pgfsetdash{}{0pt}%
\pgfpathmoveto{\pgfqpoint{5.216867in}{2.217238in}}%
\pgfpathlineto{\pgfqpoint{5.848812in}{2.217238in}}%
\pgfusepath{stroke}%
\end{pgfscope}%
\begin{pgfscope}%
\pgfsetrectcap%
\pgfsetmiterjoin%
\pgfsetlinewidth{0.803000pt}%
\definecolor{currentstroke}{rgb}{0.000000,0.000000,0.000000}%
\pgfsetstrokecolor{currentstroke}%
\pgfsetdash{}{0pt}%
\pgfpathmoveto{\pgfqpoint{0.617715in}{0.901550in}}%
\pgfpathlineto{\pgfqpoint{0.617715in}{4.412376in}}%
\pgfusepath{stroke}%
\end{pgfscope}%
\begin{pgfscope}%
\pgfsetrectcap%
\pgfsetmiterjoin%
\pgfsetlinewidth{0.803000pt}%
\definecolor{currentstroke}{rgb}{0.000000,0.000000,0.000000}%
\pgfsetstrokecolor{currentstroke}%
\pgfsetdash{}{0pt}%
\pgfpathmoveto{\pgfqpoint{6.235000in}{0.901550in}}%
\pgfpathlineto{\pgfqpoint{6.235000in}{4.412376in}}%
\pgfusepath{stroke}%
\end{pgfscope}%
\begin{pgfscope}%
\pgfsetrectcap%
\pgfsetmiterjoin%
\pgfsetlinewidth{0.803000pt}%
\definecolor{currentstroke}{rgb}{0.000000,0.000000,0.000000}%
\pgfsetstrokecolor{currentstroke}%
\pgfsetdash{}{0pt}%
\pgfpathmoveto{\pgfqpoint{0.617715in}{0.901550in}}%
\pgfpathlineto{\pgfqpoint{6.235000in}{0.901550in}}%
\pgfusepath{stroke}%
\end{pgfscope}%
\begin{pgfscope}%
\pgfsetrectcap%
\pgfsetmiterjoin%
\pgfsetlinewidth{0.803000pt}%
\definecolor{currentstroke}{rgb}{0.000000,0.000000,0.000000}%
\pgfsetstrokecolor{currentstroke}%
\pgfsetdash{}{0pt}%
\pgfpathmoveto{\pgfqpoint{0.617715in}{4.412376in}}%
\pgfpathlineto{\pgfqpoint{6.235000in}{4.412376in}}%
\pgfusepath{stroke}%
\end{pgfscope}%
\begin{pgfscope}%
\definecolor{textcolor}{rgb}{0.000000,0.000000,0.000000}%
\pgfsetstrokecolor{textcolor}%
\pgfsetfillcolor{textcolor}%
\pgftext[x=3.426358in,y=4.495710in,,base]{\color{textcolor}{\sffamily\fontsize{13.200000}{15.840000}\selectfont\catcode`\^=\active\def^{\ifmmode\sp\else\^{}\fi}\catcode`\%=\active\def%{\%}IPC difference between flushing and not flushing per workload}}%
\end{pgfscope}%
\end{pgfpicture}%
\makeatother%
\endgroup%
}
    \caption{Relative IPC difference}
    \label{fig:ipc_diff}
\end{figure}

From this initial analysis, the most interesting workloads seemed to be PyTorch DCGAN\cite{dcgan} and Gunrock\cite{gru} (on road traversal), from the Cactus\cite{cactus} suite.
Below is a full list of all analyzed workloads:
\begin{itemize}
    \item Gromacs\cite{gromacs} and LAMMPS\cite{LAMMPS} (with both rhodo (LMR) and colloid (LMC) inputs); two molecular simulation workloads,
    \item Gunrock on both road (GRU) and social networks (GST),
    \item DCGAN, neural style transform (NST)\cite{nst}, reinforcement learning (RFL), spatial transformer (SPT)\cite{spt} and language translation (LGT) from PyTorch, and
    \item The following MLCommons benchmarks (from their MLPerf® Inference v2.0 collection):
    \begin{itemize}
        \item The ResNet50 model\cite{resnet50},
        \item Both MobileNet and ResNet34 variants of the SSD model,
        \item The Bidirectional Encoder Representations from Transformers (BERT)\cite{bert}, and
        \item The 3D U-Net model\cite{3d-unet}
    \end{itemize}
    \item Four inputs to the 8x8 DCT implementation in the CUDA Samples (each labeled with their respective input).
\end{itemize}

In \cref{fig:ipc_diff}, you can see the relative IPC difference for each kernel in the Cactus and MLPerf workloads.
For the majority of workloads, the IPC difference is rather small (at most 10\%).
However, for some of the more interesting workloads (like DCG, GRU, NST, and DCT), we can find IPC differences of up to 70\% for some kernels.
\FloatBarrier

\section{Weighting kernels}\label{sec:weighting-kernels}
While this shows that the problem does exist in hardware, it might not relate to how modern simulation is carried out.
Modern techniques like Sieve\cite{sieve} use clustering to select a subset of kernels to simulate, generalizing the results to the entire workload.
In these sampling techniques, many kernels are clustered together based on characteristics.
The process then selects a single kernel from each cluster to simulate, using those results are representative for the entire cluster.
To combine the clusters and compute a final result, each cluster's result is weighted by its instruction count, relative to the full workload:
\begin{align*}
    w_{kernel} &= \frac{\text{kernel instruction count}}{\sum_{\text{kernel } k \in \text{ workload}}{\text{kernel}_k \text{ instruction count}}} \\
    w_{cluster} &= \sum_{\text{kernel } k \in \text{ cluster}} w_{kernel}
\end{align*}
In order to get a better view of the impact of the cold start problem, we've also computed IPC difference when each kernel is weighted by its instruction count.

\begin{figure}[t]
    \centering
    \begin{subfigure}{\textwidth}
        \begin{minipage}[c]{0.45\textwidth}
            \resizebox{\textwidth}{!}{\begin{tabular}{|c|c|c|c|c|}
\hline
  \textbf{Workload} & \textbf{5\% difference} & \textbf{10\% difference} & \textbf{15\% difference} & \textbf{20\% difference}\\
\hline
\hline
  dcg & 0.81 \% & 0.12 \% & 0.03 \% & 0.02 \%\\
  lmr & 0.52 \% & 0.01 \% & 0.00 \% & 0.00 \%\\
  resnet50 & 0.04 \% & 0.01 \% & 0.01 \% & 0.01 \%\\
  lgt & 14.57 \% & 1.80 \% & 0.40 \% & 0.21 \%\\
  gru & 16.11 \% & 1.93 \% & 0.07 \% & 0.07 \%\\
  gst & 0.08 \% & 0.04 \% & 0.00 \% & 0.00 \%\\
  bert & 0.00 \% & 0.00 \% & 0.00 \% & 0.00 \%\\
  spt & 12.00 \% & 1.53 \% & 0.91 \% & 0.13 \%\\
  ssd-mobilenet & 2.96 \% & 2.94 \% & 0.40 \% & 0.00 \%\\
  ssd-resnet34 & 0.07 \% & 0.00 \% & 0.00 \% & 0.00 \%\\
  lmc & 0.17 \% & 0.01 \% & 0.01 \% & 0.01 \%\\
  rfl & 6.01 \% & 1.10 \% & 0.36 \% & 0.25 \%\\
  3d-unet & 10.84 \% & 10.75 \% & 10.73 \% & 0.00 \%\\
  gms & 3.21 \% & 1.17 \% & 0.33 \% & 0.08 \%\\
  nst & 0.26 \% & 0.04 \% & 0.01 \% & 0.01 \%\\
\hline
\end{tabular}
}
        \end{minipage}
        \begin{minipage}[c]{0.45\textwidth}
            \resizebox{\textwidth}{!}{%% Creator: Matplotlib, PGF backend
%%
%% To include the figure in your LaTeX document, write
%%   \input{<filename>.pgf}
%%
%% Make sure the required packages are loaded in your preamble
%%   \usepackage{pgf}
%%
%% Also ensure that all the required font packages are loaded; for instance,
%% the lmodern package is sometimes necessary when using math font.
%%   \usepackage{lmodern}
%%
%% Figures using additional raster images can only be included by \input if
%% they are in the same directory as the main LaTeX file. For loading figures
%% from other directories you can use the `import` package
%%   \usepackage{import}
%%
%% and then include the figures with
%%   \import{<path to file>}{<filename>.pgf}
%%
%% Matplotlib used the following preamble
%%   \def\mathdefault#1{#1}
%%   \everymath=\expandafter{\the\everymath\displaystyle}
%%   
%%   \usepackage{fontspec}
%%   \setmainfont{DejaVuSerif.ttf}[Path=\detokenize{/home/data/ugent/thesis/4Jonas/lib/python3.11/site-packages/matplotlib/mpl-data/fonts/ttf/}]
%%   \setsansfont{DejaVuSans.ttf}[Path=\detokenize{/home/data/ugent/thesis/4Jonas/lib/python3.11/site-packages/matplotlib/mpl-data/fonts/ttf/}]
%%   \setmonofont{DejaVuSansMono.ttf}[Path=\detokenize{/home/data/ugent/thesis/4Jonas/lib/python3.11/site-packages/matplotlib/mpl-data/fonts/ttf/}]
%%   \makeatletter\@ifpackageloaded{underscore}{}{\usepackage[strings]{underscore}}\makeatother
%%
\begingroup%
\makeatletter%
\begin{pgfpicture}%
\pgfpathrectangle{\pgfpointorigin}{\pgfqpoint{6.400000in}{4.800000in}}%
\pgfusepath{use as bounding box, clip}%
\begin{pgfscope}%
\pgfsetbuttcap%
\pgfsetmiterjoin%
\definecolor{currentfill}{rgb}{1.000000,1.000000,1.000000}%
\pgfsetfillcolor{currentfill}%
\pgfsetlinewidth{0.000000pt}%
\definecolor{currentstroke}{rgb}{1.000000,1.000000,1.000000}%
\pgfsetstrokecolor{currentstroke}%
\pgfsetdash{}{0pt}%
\pgfpathmoveto{\pgfqpoint{0.000000in}{0.000000in}}%
\pgfpathlineto{\pgfqpoint{6.400000in}{0.000000in}}%
\pgfpathlineto{\pgfqpoint{6.400000in}{4.800000in}}%
\pgfpathlineto{\pgfqpoint{0.000000in}{4.800000in}}%
\pgfpathlineto{\pgfqpoint{0.000000in}{0.000000in}}%
\pgfpathclose%
\pgfusepath{fill}%
\end{pgfscope}%
\begin{pgfscope}%
\pgfsetbuttcap%
\pgfsetmiterjoin%
\definecolor{currentfill}{rgb}{1.000000,1.000000,1.000000}%
\pgfsetfillcolor{currentfill}%
\pgfsetlinewidth{0.000000pt}%
\definecolor{currentstroke}{rgb}{0.000000,0.000000,0.000000}%
\pgfsetstrokecolor{currentstroke}%
\pgfsetstrokeopacity{0.000000}%
\pgfsetdash{}{0pt}%
\pgfpathmoveto{\pgfqpoint{0.617715in}{1.130578in}}%
\pgfpathlineto{\pgfqpoint{6.235000in}{1.130578in}}%
\pgfpathlineto{\pgfqpoint{6.235000in}{4.412376in}}%
\pgfpathlineto{\pgfqpoint{0.617715in}{4.412376in}}%
\pgfpathlineto{\pgfqpoint{0.617715in}{1.130578in}}%
\pgfpathclose%
\pgfusepath{fill}%
\end{pgfscope}%
\begin{pgfscope}%
\pgfpathrectangle{\pgfqpoint{0.617715in}{1.130578in}}{\pgfqpoint{5.617285in}{3.281798in}}%
\pgfusepath{clip}%
\pgfsetbuttcap%
\pgfsetmiterjoin%
\definecolor{currentfill}{rgb}{0.121569,0.466667,0.705882}%
\pgfsetfillcolor{currentfill}%
\pgfsetlinewidth{0.000000pt}%
\definecolor{currentstroke}{rgb}{0.000000,0.000000,0.000000}%
\pgfsetstrokecolor{currentstroke}%
\pgfsetstrokeopacity{0.000000}%
\pgfsetdash{}{0pt}%
\pgfpathmoveto{\pgfqpoint{0.873046in}{1.130578in}}%
\pgfpathlineto{\pgfqpoint{0.942055in}{1.130578in}}%
\pgfpathlineto{\pgfqpoint{0.942055in}{1.288121in}}%
\pgfpathlineto{\pgfqpoint{0.873046in}{1.288121in}}%
\pgfpathlineto{\pgfqpoint{0.873046in}{1.130578in}}%
\pgfpathclose%
\pgfusepath{fill}%
\end{pgfscope}%
\begin{pgfscope}%
\pgfpathrectangle{\pgfqpoint{0.617715in}{1.130578in}}{\pgfqpoint{5.617285in}{3.281798in}}%
\pgfusepath{clip}%
\pgfsetbuttcap%
\pgfsetmiterjoin%
\definecolor{currentfill}{rgb}{0.121569,0.466667,0.705882}%
\pgfsetfillcolor{currentfill}%
\pgfsetlinewidth{0.000000pt}%
\definecolor{currentstroke}{rgb}{0.000000,0.000000,0.000000}%
\pgfsetstrokecolor{currentstroke}%
\pgfsetstrokeopacity{0.000000}%
\pgfsetdash{}{0pt}%
\pgfpathmoveto{\pgfqpoint{1.218088in}{1.130578in}}%
\pgfpathlineto{\pgfqpoint{1.287097in}{1.130578in}}%
\pgfpathlineto{\pgfqpoint{1.287097in}{1.231847in}}%
\pgfpathlineto{\pgfqpoint{1.218088in}{1.231847in}}%
\pgfpathlineto{\pgfqpoint{1.218088in}{1.130578in}}%
\pgfpathclose%
\pgfusepath{fill}%
\end{pgfscope}%
\begin{pgfscope}%
\pgfpathrectangle{\pgfqpoint{0.617715in}{1.130578in}}{\pgfqpoint{5.617285in}{3.281798in}}%
\pgfusepath{clip}%
\pgfsetbuttcap%
\pgfsetmiterjoin%
\definecolor{currentfill}{rgb}{0.121569,0.466667,0.705882}%
\pgfsetfillcolor{currentfill}%
\pgfsetlinewidth{0.000000pt}%
\definecolor{currentstroke}{rgb}{0.000000,0.000000,0.000000}%
\pgfsetstrokecolor{currentstroke}%
\pgfsetstrokeopacity{0.000000}%
\pgfsetdash{}{0pt}%
\pgfpathmoveto{\pgfqpoint{1.563131in}{1.130578in}}%
\pgfpathlineto{\pgfqpoint{1.632139in}{1.130578in}}%
\pgfpathlineto{\pgfqpoint{1.632139in}{1.138030in}}%
\pgfpathlineto{\pgfqpoint{1.563131in}{1.138030in}}%
\pgfpathlineto{\pgfqpoint{1.563131in}{1.130578in}}%
\pgfpathclose%
\pgfusepath{fill}%
\end{pgfscope}%
\begin{pgfscope}%
\pgfpathrectangle{\pgfqpoint{0.617715in}{1.130578in}}{\pgfqpoint{5.617285in}{3.281798in}}%
\pgfusepath{clip}%
\pgfsetbuttcap%
\pgfsetmiterjoin%
\definecolor{currentfill}{rgb}{0.121569,0.466667,0.705882}%
\pgfsetfillcolor{currentfill}%
\pgfsetlinewidth{0.000000pt}%
\definecolor{currentstroke}{rgb}{0.000000,0.000000,0.000000}%
\pgfsetstrokecolor{currentstroke}%
\pgfsetstrokeopacity{0.000000}%
\pgfsetdash{}{0pt}%
\pgfpathmoveto{\pgfqpoint{1.908173in}{1.130578in}}%
\pgfpathlineto{\pgfqpoint{1.977181in}{1.130578in}}%
\pgfpathlineto{\pgfqpoint{1.977181in}{3.957993in}}%
\pgfpathlineto{\pgfqpoint{1.908173in}{3.957993in}}%
\pgfpathlineto{\pgfqpoint{1.908173in}{1.130578in}}%
\pgfpathclose%
\pgfusepath{fill}%
\end{pgfscope}%
\begin{pgfscope}%
\pgfpathrectangle{\pgfqpoint{0.617715in}{1.130578in}}{\pgfqpoint{5.617285in}{3.281798in}}%
\pgfusepath{clip}%
\pgfsetbuttcap%
\pgfsetmiterjoin%
\definecolor{currentfill}{rgb}{0.121569,0.466667,0.705882}%
\pgfsetfillcolor{currentfill}%
\pgfsetlinewidth{0.000000pt}%
\definecolor{currentstroke}{rgb}{0.000000,0.000000,0.000000}%
\pgfsetstrokecolor{currentstroke}%
\pgfsetstrokeopacity{0.000000}%
\pgfsetdash{}{0pt}%
\pgfpathmoveto{\pgfqpoint{2.253215in}{1.130578in}}%
\pgfpathlineto{\pgfqpoint{2.322223in}{1.130578in}}%
\pgfpathlineto{\pgfqpoint{2.322223in}{4.256100in}}%
\pgfpathlineto{\pgfqpoint{2.253215in}{4.256100in}}%
\pgfpathlineto{\pgfqpoint{2.253215in}{1.130578in}}%
\pgfpathclose%
\pgfusepath{fill}%
\end{pgfscope}%
\begin{pgfscope}%
\pgfpathrectangle{\pgfqpoint{0.617715in}{1.130578in}}{\pgfqpoint{5.617285in}{3.281798in}}%
\pgfusepath{clip}%
\pgfsetbuttcap%
\pgfsetmiterjoin%
\definecolor{currentfill}{rgb}{0.121569,0.466667,0.705882}%
\pgfsetfillcolor{currentfill}%
\pgfsetlinewidth{0.000000pt}%
\definecolor{currentstroke}{rgb}{0.000000,0.000000,0.000000}%
\pgfsetstrokecolor{currentstroke}%
\pgfsetstrokeopacity{0.000000}%
\pgfsetdash{}{0pt}%
\pgfpathmoveto{\pgfqpoint{2.598257in}{1.130578in}}%
\pgfpathlineto{\pgfqpoint{2.667265in}{1.130578in}}%
\pgfpathlineto{\pgfqpoint{2.667265in}{1.146061in}}%
\pgfpathlineto{\pgfqpoint{2.598257in}{1.146061in}}%
\pgfpathlineto{\pgfqpoint{2.598257in}{1.130578in}}%
\pgfpathclose%
\pgfusepath{fill}%
\end{pgfscope}%
\begin{pgfscope}%
\pgfpathrectangle{\pgfqpoint{0.617715in}{1.130578in}}{\pgfqpoint{5.617285in}{3.281798in}}%
\pgfusepath{clip}%
\pgfsetbuttcap%
\pgfsetmiterjoin%
\definecolor{currentfill}{rgb}{0.121569,0.466667,0.705882}%
\pgfsetfillcolor{currentfill}%
\pgfsetlinewidth{0.000000pt}%
\definecolor{currentstroke}{rgb}{0.000000,0.000000,0.000000}%
\pgfsetstrokecolor{currentstroke}%
\pgfsetstrokeopacity{0.000000}%
\pgfsetdash{}{0pt}%
\pgfpathmoveto{\pgfqpoint{2.943299in}{1.130578in}}%
\pgfpathlineto{\pgfqpoint{3.012307in}{1.130578in}}%
\pgfpathlineto{\pgfqpoint{3.012307in}{1.130627in}}%
\pgfpathlineto{\pgfqpoint{2.943299in}{1.130627in}}%
\pgfpathlineto{\pgfqpoint{2.943299in}{1.130578in}}%
\pgfpathclose%
\pgfusepath{fill}%
\end{pgfscope}%
\begin{pgfscope}%
\pgfpathrectangle{\pgfqpoint{0.617715in}{1.130578in}}{\pgfqpoint{5.617285in}{3.281798in}}%
\pgfusepath{clip}%
\pgfsetbuttcap%
\pgfsetmiterjoin%
\definecolor{currentfill}{rgb}{0.121569,0.466667,0.705882}%
\pgfsetfillcolor{currentfill}%
\pgfsetlinewidth{0.000000pt}%
\definecolor{currentstroke}{rgb}{0.000000,0.000000,0.000000}%
\pgfsetstrokecolor{currentstroke}%
\pgfsetstrokeopacity{0.000000}%
\pgfsetdash{}{0pt}%
\pgfpathmoveto{\pgfqpoint{3.288341in}{1.130578in}}%
\pgfpathlineto{\pgfqpoint{3.357349in}{1.130578in}}%
\pgfpathlineto{\pgfqpoint{3.357349in}{3.458667in}}%
\pgfpathlineto{\pgfqpoint{3.288341in}{3.458667in}}%
\pgfpathlineto{\pgfqpoint{3.288341in}{1.130578in}}%
\pgfpathclose%
\pgfusepath{fill}%
\end{pgfscope}%
\begin{pgfscope}%
\pgfpathrectangle{\pgfqpoint{0.617715in}{1.130578in}}{\pgfqpoint{5.617285in}{3.281798in}}%
\pgfusepath{clip}%
\pgfsetbuttcap%
\pgfsetmiterjoin%
\definecolor{currentfill}{rgb}{0.121569,0.466667,0.705882}%
\pgfsetfillcolor{currentfill}%
\pgfsetlinewidth{0.000000pt}%
\definecolor{currentstroke}{rgb}{0.000000,0.000000,0.000000}%
\pgfsetstrokecolor{currentstroke}%
\pgfsetstrokeopacity{0.000000}%
\pgfsetdash{}{0pt}%
\pgfpathmoveto{\pgfqpoint{3.633383in}{1.130578in}}%
\pgfpathlineto{\pgfqpoint{3.702391in}{1.130578in}}%
\pgfpathlineto{\pgfqpoint{3.702391in}{1.705878in}}%
\pgfpathlineto{\pgfqpoint{3.633383in}{1.705878in}}%
\pgfpathlineto{\pgfqpoint{3.633383in}{1.130578in}}%
\pgfpathclose%
\pgfusepath{fill}%
\end{pgfscope}%
\begin{pgfscope}%
\pgfpathrectangle{\pgfqpoint{0.617715in}{1.130578in}}{\pgfqpoint{5.617285in}{3.281798in}}%
\pgfusepath{clip}%
\pgfsetbuttcap%
\pgfsetmiterjoin%
\definecolor{currentfill}{rgb}{0.121569,0.466667,0.705882}%
\pgfsetfillcolor{currentfill}%
\pgfsetlinewidth{0.000000pt}%
\definecolor{currentstroke}{rgb}{0.000000,0.000000,0.000000}%
\pgfsetstrokecolor{currentstroke}%
\pgfsetstrokeopacity{0.000000}%
\pgfsetdash{}{0pt}%
\pgfpathmoveto{\pgfqpoint{3.978425in}{1.130578in}}%
\pgfpathlineto{\pgfqpoint{4.047433in}{1.130578in}}%
\pgfpathlineto{\pgfqpoint{4.047433in}{1.144798in}}%
\pgfpathlineto{\pgfqpoint{3.978425in}{1.144798in}}%
\pgfpathlineto{\pgfqpoint{3.978425in}{1.130578in}}%
\pgfpathclose%
\pgfusepath{fill}%
\end{pgfscope}%
\begin{pgfscope}%
\pgfpathrectangle{\pgfqpoint{0.617715in}{1.130578in}}{\pgfqpoint{5.617285in}{3.281798in}}%
\pgfusepath{clip}%
\pgfsetbuttcap%
\pgfsetmiterjoin%
\definecolor{currentfill}{rgb}{0.121569,0.466667,0.705882}%
\pgfsetfillcolor{currentfill}%
\pgfsetlinewidth{0.000000pt}%
\definecolor{currentstroke}{rgb}{0.000000,0.000000,0.000000}%
\pgfsetstrokecolor{currentstroke}%
\pgfsetstrokeopacity{0.000000}%
\pgfsetdash{}{0pt}%
\pgfpathmoveto{\pgfqpoint{4.323467in}{1.130578in}}%
\pgfpathlineto{\pgfqpoint{4.392475in}{1.130578in}}%
\pgfpathlineto{\pgfqpoint{4.392475in}{1.163828in}}%
\pgfpathlineto{\pgfqpoint{4.323467in}{1.163828in}}%
\pgfpathlineto{\pgfqpoint{4.323467in}{1.130578in}}%
\pgfpathclose%
\pgfusepath{fill}%
\end{pgfscope}%
\begin{pgfscope}%
\pgfpathrectangle{\pgfqpoint{0.617715in}{1.130578in}}{\pgfqpoint{5.617285in}{3.281798in}}%
\pgfusepath{clip}%
\pgfsetbuttcap%
\pgfsetmiterjoin%
\definecolor{currentfill}{rgb}{0.121569,0.466667,0.705882}%
\pgfsetfillcolor{currentfill}%
\pgfsetlinewidth{0.000000pt}%
\definecolor{currentstroke}{rgb}{0.000000,0.000000,0.000000}%
\pgfsetstrokecolor{currentstroke}%
\pgfsetstrokeopacity{0.000000}%
\pgfsetdash{}{0pt}%
\pgfpathmoveto{\pgfqpoint{4.668509in}{1.130578in}}%
\pgfpathlineto{\pgfqpoint{4.737517in}{1.130578in}}%
\pgfpathlineto{\pgfqpoint{4.737517in}{2.296014in}}%
\pgfpathlineto{\pgfqpoint{4.668509in}{2.296014in}}%
\pgfpathlineto{\pgfqpoint{4.668509in}{1.130578in}}%
\pgfpathclose%
\pgfusepath{fill}%
\end{pgfscope}%
\begin{pgfscope}%
\pgfpathrectangle{\pgfqpoint{0.617715in}{1.130578in}}{\pgfqpoint{5.617285in}{3.281798in}}%
\pgfusepath{clip}%
\pgfsetbuttcap%
\pgfsetmiterjoin%
\definecolor{currentfill}{rgb}{0.121569,0.466667,0.705882}%
\pgfsetfillcolor{currentfill}%
\pgfsetlinewidth{0.000000pt}%
\definecolor{currentstroke}{rgb}{0.000000,0.000000,0.000000}%
\pgfsetstrokecolor{currentstroke}%
\pgfsetstrokeopacity{0.000000}%
\pgfsetdash{}{0pt}%
\pgfpathmoveto{\pgfqpoint{5.013551in}{1.130578in}}%
\pgfpathlineto{\pgfqpoint{5.082560in}{1.130578in}}%
\pgfpathlineto{\pgfqpoint{5.082560in}{3.234948in}}%
\pgfpathlineto{\pgfqpoint{5.013551in}{3.234948in}}%
\pgfpathlineto{\pgfqpoint{5.013551in}{1.130578in}}%
\pgfpathclose%
\pgfusepath{fill}%
\end{pgfscope}%
\begin{pgfscope}%
\pgfpathrectangle{\pgfqpoint{0.617715in}{1.130578in}}{\pgfqpoint{5.617285in}{3.281798in}}%
\pgfusepath{clip}%
\pgfsetbuttcap%
\pgfsetmiterjoin%
\definecolor{currentfill}{rgb}{0.121569,0.466667,0.705882}%
\pgfsetfillcolor{currentfill}%
\pgfsetlinewidth{0.000000pt}%
\definecolor{currentstroke}{rgb}{0.000000,0.000000,0.000000}%
\pgfsetstrokecolor{currentstroke}%
\pgfsetstrokeopacity{0.000000}%
\pgfsetdash{}{0pt}%
\pgfpathmoveto{\pgfqpoint{5.358593in}{1.130578in}}%
\pgfpathlineto{\pgfqpoint{5.427602in}{1.130578in}}%
\pgfpathlineto{\pgfqpoint{5.427602in}{1.753198in}}%
\pgfpathlineto{\pgfqpoint{5.358593in}{1.753198in}}%
\pgfpathlineto{\pgfqpoint{5.358593in}{1.130578in}}%
\pgfpathclose%
\pgfusepath{fill}%
\end{pgfscope}%
\begin{pgfscope}%
\pgfpathrectangle{\pgfqpoint{0.617715in}{1.130578in}}{\pgfqpoint{5.617285in}{3.281798in}}%
\pgfusepath{clip}%
\pgfsetbuttcap%
\pgfsetmiterjoin%
\definecolor{currentfill}{rgb}{0.121569,0.466667,0.705882}%
\pgfsetfillcolor{currentfill}%
\pgfsetlinewidth{0.000000pt}%
\definecolor{currentstroke}{rgb}{0.000000,0.000000,0.000000}%
\pgfsetstrokecolor{currentstroke}%
\pgfsetstrokeopacity{0.000000}%
\pgfsetdash{}{0pt}%
\pgfpathmoveto{\pgfqpoint{5.703635in}{1.130578in}}%
\pgfpathlineto{\pgfqpoint{5.772644in}{1.130578in}}%
\pgfpathlineto{\pgfqpoint{5.772644in}{1.180563in}}%
\pgfpathlineto{\pgfqpoint{5.703635in}{1.180563in}}%
\pgfpathlineto{\pgfqpoint{5.703635in}{1.130578in}}%
\pgfpathclose%
\pgfusepath{fill}%
\end{pgfscope}%
\begin{pgfscope}%
\pgfpathrectangle{\pgfqpoint{0.617715in}{1.130578in}}{\pgfqpoint{5.617285in}{3.281798in}}%
\pgfusepath{clip}%
\pgfsetbuttcap%
\pgfsetmiterjoin%
\definecolor{currentfill}{rgb}{1.000000,0.498039,0.054902}%
\pgfsetfillcolor{currentfill}%
\pgfsetlinewidth{0.000000pt}%
\definecolor{currentstroke}{rgb}{0.000000,0.000000,0.000000}%
\pgfsetstrokecolor{currentstroke}%
\pgfsetstrokeopacity{0.000000}%
\pgfsetdash{}{0pt}%
\pgfpathmoveto{\pgfqpoint{0.942055in}{1.130578in}}%
\pgfpathlineto{\pgfqpoint{1.011063in}{1.130578in}}%
\pgfpathlineto{\pgfqpoint{1.011063in}{1.154359in}}%
\pgfpathlineto{\pgfqpoint{0.942055in}{1.154359in}}%
\pgfpathlineto{\pgfqpoint{0.942055in}{1.130578in}}%
\pgfpathclose%
\pgfusepath{fill}%
\end{pgfscope}%
\begin{pgfscope}%
\pgfpathrectangle{\pgfqpoint{0.617715in}{1.130578in}}{\pgfqpoint{5.617285in}{3.281798in}}%
\pgfusepath{clip}%
\pgfsetbuttcap%
\pgfsetmiterjoin%
\definecolor{currentfill}{rgb}{1.000000,0.498039,0.054902}%
\pgfsetfillcolor{currentfill}%
\pgfsetlinewidth{0.000000pt}%
\definecolor{currentstroke}{rgb}{0.000000,0.000000,0.000000}%
\pgfsetstrokecolor{currentstroke}%
\pgfsetstrokeopacity{0.000000}%
\pgfsetdash{}{0pt}%
\pgfpathmoveto{\pgfqpoint{1.287097in}{1.130578in}}%
\pgfpathlineto{\pgfqpoint{1.356105in}{1.130578in}}%
\pgfpathlineto{\pgfqpoint{1.356105in}{1.133212in}}%
\pgfpathlineto{\pgfqpoint{1.287097in}{1.133212in}}%
\pgfpathlineto{\pgfqpoint{1.287097in}{1.130578in}}%
\pgfpathclose%
\pgfusepath{fill}%
\end{pgfscope}%
\begin{pgfscope}%
\pgfpathrectangle{\pgfqpoint{0.617715in}{1.130578in}}{\pgfqpoint{5.617285in}{3.281798in}}%
\pgfusepath{clip}%
\pgfsetbuttcap%
\pgfsetmiterjoin%
\definecolor{currentfill}{rgb}{1.000000,0.498039,0.054902}%
\pgfsetfillcolor{currentfill}%
\pgfsetlinewidth{0.000000pt}%
\definecolor{currentstroke}{rgb}{0.000000,0.000000,0.000000}%
\pgfsetstrokecolor{currentstroke}%
\pgfsetstrokeopacity{0.000000}%
\pgfsetdash{}{0pt}%
\pgfpathmoveto{\pgfqpoint{1.632139in}{1.130578in}}%
\pgfpathlineto{\pgfqpoint{1.701147in}{1.130578in}}%
\pgfpathlineto{\pgfqpoint{1.701147in}{1.131820in}}%
\pgfpathlineto{\pgfqpoint{1.632139in}{1.131820in}}%
\pgfpathlineto{\pgfqpoint{1.632139in}{1.130578in}}%
\pgfpathclose%
\pgfusepath{fill}%
\end{pgfscope}%
\begin{pgfscope}%
\pgfpathrectangle{\pgfqpoint{0.617715in}{1.130578in}}{\pgfqpoint{5.617285in}{3.281798in}}%
\pgfusepath{clip}%
\pgfsetbuttcap%
\pgfsetmiterjoin%
\definecolor{currentfill}{rgb}{1.000000,0.498039,0.054902}%
\pgfsetfillcolor{currentfill}%
\pgfsetlinewidth{0.000000pt}%
\definecolor{currentstroke}{rgb}{0.000000,0.000000,0.000000}%
\pgfsetstrokecolor{currentstroke}%
\pgfsetstrokeopacity{0.000000}%
\pgfsetdash{}{0pt}%
\pgfpathmoveto{\pgfqpoint{1.977181in}{1.130578in}}%
\pgfpathlineto{\pgfqpoint{2.046189in}{1.130578in}}%
\pgfpathlineto{\pgfqpoint{2.046189in}{1.478845in}}%
\pgfpathlineto{\pgfqpoint{1.977181in}{1.478845in}}%
\pgfpathlineto{\pgfqpoint{1.977181in}{1.130578in}}%
\pgfpathclose%
\pgfusepath{fill}%
\end{pgfscope}%
\begin{pgfscope}%
\pgfpathrectangle{\pgfqpoint{0.617715in}{1.130578in}}{\pgfqpoint{5.617285in}{3.281798in}}%
\pgfusepath{clip}%
\pgfsetbuttcap%
\pgfsetmiterjoin%
\definecolor{currentfill}{rgb}{1.000000,0.498039,0.054902}%
\pgfsetfillcolor{currentfill}%
\pgfsetlinewidth{0.000000pt}%
\definecolor{currentstroke}{rgb}{0.000000,0.000000,0.000000}%
\pgfsetstrokecolor{currentstroke}%
\pgfsetstrokeopacity{0.000000}%
\pgfsetdash{}{0pt}%
\pgfpathmoveto{\pgfqpoint{2.322223in}{1.130578in}}%
\pgfpathlineto{\pgfqpoint{2.391231in}{1.130578in}}%
\pgfpathlineto{\pgfqpoint{2.391231in}{1.505920in}}%
\pgfpathlineto{\pgfqpoint{2.322223in}{1.505920in}}%
\pgfpathlineto{\pgfqpoint{2.322223in}{1.130578in}}%
\pgfpathclose%
\pgfusepath{fill}%
\end{pgfscope}%
\begin{pgfscope}%
\pgfpathrectangle{\pgfqpoint{0.617715in}{1.130578in}}{\pgfqpoint{5.617285in}{3.281798in}}%
\pgfusepath{clip}%
\pgfsetbuttcap%
\pgfsetmiterjoin%
\definecolor{currentfill}{rgb}{1.000000,0.498039,0.054902}%
\pgfsetfillcolor{currentfill}%
\pgfsetlinewidth{0.000000pt}%
\definecolor{currentstroke}{rgb}{0.000000,0.000000,0.000000}%
\pgfsetstrokecolor{currentstroke}%
\pgfsetstrokeopacity{0.000000}%
\pgfsetdash{}{0pt}%
\pgfpathmoveto{\pgfqpoint{2.667265in}{1.130578in}}%
\pgfpathlineto{\pgfqpoint{2.736274in}{1.130578in}}%
\pgfpathlineto{\pgfqpoint{2.736274in}{1.137651in}}%
\pgfpathlineto{\pgfqpoint{2.667265in}{1.137651in}}%
\pgfpathlineto{\pgfqpoint{2.667265in}{1.130578in}}%
\pgfpathclose%
\pgfusepath{fill}%
\end{pgfscope}%
\begin{pgfscope}%
\pgfpathrectangle{\pgfqpoint{0.617715in}{1.130578in}}{\pgfqpoint{5.617285in}{3.281798in}}%
\pgfusepath{clip}%
\pgfsetbuttcap%
\pgfsetmiterjoin%
\definecolor{currentfill}{rgb}{1.000000,0.498039,0.054902}%
\pgfsetfillcolor{currentfill}%
\pgfsetlinewidth{0.000000pt}%
\definecolor{currentstroke}{rgb}{0.000000,0.000000,0.000000}%
\pgfsetstrokecolor{currentstroke}%
\pgfsetstrokeopacity{0.000000}%
\pgfsetdash{}{0pt}%
\pgfpathmoveto{\pgfqpoint{3.012307in}{1.130578in}}%
\pgfpathlineto{\pgfqpoint{3.081316in}{1.130578in}}%
\pgfpathlineto{\pgfqpoint{3.081316in}{1.130581in}}%
\pgfpathlineto{\pgfqpoint{3.012307in}{1.130581in}}%
\pgfpathlineto{\pgfqpoint{3.012307in}{1.130578in}}%
\pgfpathclose%
\pgfusepath{fill}%
\end{pgfscope}%
\begin{pgfscope}%
\pgfpathrectangle{\pgfqpoint{0.617715in}{1.130578in}}{\pgfqpoint{5.617285in}{3.281798in}}%
\pgfusepath{clip}%
\pgfsetbuttcap%
\pgfsetmiterjoin%
\definecolor{currentfill}{rgb}{1.000000,0.498039,0.054902}%
\pgfsetfillcolor{currentfill}%
\pgfsetlinewidth{0.000000pt}%
\definecolor{currentstroke}{rgb}{0.000000,0.000000,0.000000}%
\pgfsetstrokecolor{currentstroke}%
\pgfsetstrokeopacity{0.000000}%
\pgfsetdash{}{0pt}%
\pgfpathmoveto{\pgfqpoint{3.357349in}{1.130578in}}%
\pgfpathlineto{\pgfqpoint{3.426358in}{1.130578in}}%
\pgfpathlineto{\pgfqpoint{3.426358in}{1.427985in}}%
\pgfpathlineto{\pgfqpoint{3.357349in}{1.427985in}}%
\pgfpathlineto{\pgfqpoint{3.357349in}{1.130578in}}%
\pgfpathclose%
\pgfusepath{fill}%
\end{pgfscope}%
\begin{pgfscope}%
\pgfpathrectangle{\pgfqpoint{0.617715in}{1.130578in}}{\pgfqpoint{5.617285in}{3.281798in}}%
\pgfusepath{clip}%
\pgfsetbuttcap%
\pgfsetmiterjoin%
\definecolor{currentfill}{rgb}{1.000000,0.498039,0.054902}%
\pgfsetfillcolor{currentfill}%
\pgfsetlinewidth{0.000000pt}%
\definecolor{currentstroke}{rgb}{0.000000,0.000000,0.000000}%
\pgfsetstrokecolor{currentstroke}%
\pgfsetstrokeopacity{0.000000}%
\pgfsetdash{}{0pt}%
\pgfpathmoveto{\pgfqpoint{3.702391in}{1.130578in}}%
\pgfpathlineto{\pgfqpoint{3.771400in}{1.130578in}}%
\pgfpathlineto{\pgfqpoint{3.771400in}{1.700744in}}%
\pgfpathlineto{\pgfqpoint{3.702391in}{1.700744in}}%
\pgfpathlineto{\pgfqpoint{3.702391in}{1.130578in}}%
\pgfpathclose%
\pgfusepath{fill}%
\end{pgfscope}%
\begin{pgfscope}%
\pgfpathrectangle{\pgfqpoint{0.617715in}{1.130578in}}{\pgfqpoint{5.617285in}{3.281798in}}%
\pgfusepath{clip}%
\pgfsetbuttcap%
\pgfsetmiterjoin%
\definecolor{currentfill}{rgb}{1.000000,0.498039,0.054902}%
\pgfsetfillcolor{currentfill}%
\pgfsetlinewidth{0.000000pt}%
\definecolor{currentstroke}{rgb}{0.000000,0.000000,0.000000}%
\pgfsetstrokecolor{currentstroke}%
\pgfsetstrokeopacity{0.000000}%
\pgfsetdash{}{0pt}%
\pgfpathmoveto{\pgfqpoint{4.047433in}{1.130578in}}%
\pgfpathlineto{\pgfqpoint{4.116442in}{1.130578in}}%
\pgfpathlineto{\pgfqpoint{4.116442in}{1.131057in}}%
\pgfpathlineto{\pgfqpoint{4.047433in}{1.131057in}}%
\pgfpathlineto{\pgfqpoint{4.047433in}{1.130578in}}%
\pgfpathclose%
\pgfusepath{fill}%
\end{pgfscope}%
\begin{pgfscope}%
\pgfpathrectangle{\pgfqpoint{0.617715in}{1.130578in}}{\pgfqpoint{5.617285in}{3.281798in}}%
\pgfusepath{clip}%
\pgfsetbuttcap%
\pgfsetmiterjoin%
\definecolor{currentfill}{rgb}{1.000000,0.498039,0.054902}%
\pgfsetfillcolor{currentfill}%
\pgfsetlinewidth{0.000000pt}%
\definecolor{currentstroke}{rgb}{0.000000,0.000000,0.000000}%
\pgfsetstrokecolor{currentstroke}%
\pgfsetstrokeopacity{0.000000}%
\pgfsetdash{}{0pt}%
\pgfpathmoveto{\pgfqpoint{4.392475in}{1.130578in}}%
\pgfpathlineto{\pgfqpoint{4.461484in}{1.130578in}}%
\pgfpathlineto{\pgfqpoint{4.461484in}{1.131808in}}%
\pgfpathlineto{\pgfqpoint{4.392475in}{1.131808in}}%
\pgfpathlineto{\pgfqpoint{4.392475in}{1.130578in}}%
\pgfpathclose%
\pgfusepath{fill}%
\end{pgfscope}%
\begin{pgfscope}%
\pgfpathrectangle{\pgfqpoint{0.617715in}{1.130578in}}{\pgfqpoint{5.617285in}{3.281798in}}%
\pgfusepath{clip}%
\pgfsetbuttcap%
\pgfsetmiterjoin%
\definecolor{currentfill}{rgb}{1.000000,0.498039,0.054902}%
\pgfsetfillcolor{currentfill}%
\pgfsetlinewidth{0.000000pt}%
\definecolor{currentstroke}{rgb}{0.000000,0.000000,0.000000}%
\pgfsetstrokecolor{currentstroke}%
\pgfsetstrokeopacity{0.000000}%
\pgfsetdash{}{0pt}%
\pgfpathmoveto{\pgfqpoint{4.737517in}{1.130578in}}%
\pgfpathlineto{\pgfqpoint{4.806526in}{1.130578in}}%
\pgfpathlineto{\pgfqpoint{4.806526in}{1.343684in}}%
\pgfpathlineto{\pgfqpoint{4.737517in}{1.343684in}}%
\pgfpathlineto{\pgfqpoint{4.737517in}{1.130578in}}%
\pgfpathclose%
\pgfusepath{fill}%
\end{pgfscope}%
\begin{pgfscope}%
\pgfpathrectangle{\pgfqpoint{0.617715in}{1.130578in}}{\pgfqpoint{5.617285in}{3.281798in}}%
\pgfusepath{clip}%
\pgfsetbuttcap%
\pgfsetmiterjoin%
\definecolor{currentfill}{rgb}{1.000000,0.498039,0.054902}%
\pgfsetfillcolor{currentfill}%
\pgfsetlinewidth{0.000000pt}%
\definecolor{currentstroke}{rgb}{0.000000,0.000000,0.000000}%
\pgfsetstrokecolor{currentstroke}%
\pgfsetstrokeopacity{0.000000}%
\pgfsetdash{}{0pt}%
\pgfpathmoveto{\pgfqpoint{5.082560in}{1.130578in}}%
\pgfpathlineto{\pgfqpoint{5.151568in}{1.130578in}}%
\pgfpathlineto{\pgfqpoint{5.151568in}{3.215847in}}%
\pgfpathlineto{\pgfqpoint{5.082560in}{3.215847in}}%
\pgfpathlineto{\pgfqpoint{5.082560in}{1.130578in}}%
\pgfpathclose%
\pgfusepath{fill}%
\end{pgfscope}%
\begin{pgfscope}%
\pgfpathrectangle{\pgfqpoint{0.617715in}{1.130578in}}{\pgfqpoint{5.617285in}{3.281798in}}%
\pgfusepath{clip}%
\pgfsetbuttcap%
\pgfsetmiterjoin%
\definecolor{currentfill}{rgb}{1.000000,0.498039,0.054902}%
\pgfsetfillcolor{currentfill}%
\pgfsetlinewidth{0.000000pt}%
\definecolor{currentstroke}{rgb}{0.000000,0.000000,0.000000}%
\pgfsetstrokecolor{currentstroke}%
\pgfsetstrokeopacity{0.000000}%
\pgfsetdash{}{0pt}%
\pgfpathmoveto{\pgfqpoint{5.427602in}{1.130578in}}%
\pgfpathlineto{\pgfqpoint{5.496610in}{1.130578in}}%
\pgfpathlineto{\pgfqpoint{5.496610in}{1.356910in}}%
\pgfpathlineto{\pgfqpoint{5.427602in}{1.356910in}}%
\pgfpathlineto{\pgfqpoint{5.427602in}{1.130578in}}%
\pgfpathclose%
\pgfusepath{fill}%
\end{pgfscope}%
\begin{pgfscope}%
\pgfpathrectangle{\pgfqpoint{0.617715in}{1.130578in}}{\pgfqpoint{5.617285in}{3.281798in}}%
\pgfusepath{clip}%
\pgfsetbuttcap%
\pgfsetmiterjoin%
\definecolor{currentfill}{rgb}{1.000000,0.498039,0.054902}%
\pgfsetfillcolor{currentfill}%
\pgfsetlinewidth{0.000000pt}%
\definecolor{currentstroke}{rgb}{0.000000,0.000000,0.000000}%
\pgfsetstrokecolor{currentstroke}%
\pgfsetstrokeopacity{0.000000}%
\pgfsetdash{}{0pt}%
\pgfpathmoveto{\pgfqpoint{5.772644in}{1.130578in}}%
\pgfpathlineto{\pgfqpoint{5.841652in}{1.130578in}}%
\pgfpathlineto{\pgfqpoint{5.841652in}{1.138203in}}%
\pgfpathlineto{\pgfqpoint{5.772644in}{1.138203in}}%
\pgfpathlineto{\pgfqpoint{5.772644in}{1.130578in}}%
\pgfpathclose%
\pgfusepath{fill}%
\end{pgfscope}%
\begin{pgfscope}%
\pgfpathrectangle{\pgfqpoint{0.617715in}{1.130578in}}{\pgfqpoint{5.617285in}{3.281798in}}%
\pgfusepath{clip}%
\pgfsetbuttcap%
\pgfsetmiterjoin%
\definecolor{currentfill}{rgb}{0.172549,0.627451,0.172549}%
\pgfsetfillcolor{currentfill}%
\pgfsetlinewidth{0.000000pt}%
\definecolor{currentstroke}{rgb}{0.000000,0.000000,0.000000}%
\pgfsetstrokecolor{currentstroke}%
\pgfsetstrokeopacity{0.000000}%
\pgfsetdash{}{0pt}%
\pgfpathmoveto{\pgfqpoint{1.011063in}{1.130578in}}%
\pgfpathlineto{\pgfqpoint{1.080072in}{1.130578in}}%
\pgfpathlineto{\pgfqpoint{1.080072in}{1.135795in}}%
\pgfpathlineto{\pgfqpoint{1.011063in}{1.135795in}}%
\pgfpathlineto{\pgfqpoint{1.011063in}{1.130578in}}%
\pgfpathclose%
\pgfusepath{fill}%
\end{pgfscope}%
\begin{pgfscope}%
\pgfpathrectangle{\pgfqpoint{0.617715in}{1.130578in}}{\pgfqpoint{5.617285in}{3.281798in}}%
\pgfusepath{clip}%
\pgfsetbuttcap%
\pgfsetmiterjoin%
\definecolor{currentfill}{rgb}{0.172549,0.627451,0.172549}%
\pgfsetfillcolor{currentfill}%
\pgfsetlinewidth{0.000000pt}%
\definecolor{currentstroke}{rgb}{0.000000,0.000000,0.000000}%
\pgfsetstrokecolor{currentstroke}%
\pgfsetstrokeopacity{0.000000}%
\pgfsetdash{}{0pt}%
\pgfpathmoveto{\pgfqpoint{1.356105in}{1.130578in}}%
\pgfpathlineto{\pgfqpoint{1.425114in}{1.130578in}}%
\pgfpathlineto{\pgfqpoint{1.425114in}{1.131054in}}%
\pgfpathlineto{\pgfqpoint{1.356105in}{1.131054in}}%
\pgfpathlineto{\pgfqpoint{1.356105in}{1.130578in}}%
\pgfpathclose%
\pgfusepath{fill}%
\end{pgfscope}%
\begin{pgfscope}%
\pgfpathrectangle{\pgfqpoint{0.617715in}{1.130578in}}{\pgfqpoint{5.617285in}{3.281798in}}%
\pgfusepath{clip}%
\pgfsetbuttcap%
\pgfsetmiterjoin%
\definecolor{currentfill}{rgb}{0.172549,0.627451,0.172549}%
\pgfsetfillcolor{currentfill}%
\pgfsetlinewidth{0.000000pt}%
\definecolor{currentstroke}{rgb}{0.000000,0.000000,0.000000}%
\pgfsetstrokecolor{currentstroke}%
\pgfsetstrokeopacity{0.000000}%
\pgfsetdash{}{0pt}%
\pgfpathmoveto{\pgfqpoint{1.701147in}{1.130578in}}%
\pgfpathlineto{\pgfqpoint{1.770156in}{1.130578in}}%
\pgfpathlineto{\pgfqpoint{1.770156in}{1.131820in}}%
\pgfpathlineto{\pgfqpoint{1.701147in}{1.131820in}}%
\pgfpathlineto{\pgfqpoint{1.701147in}{1.130578in}}%
\pgfpathclose%
\pgfusepath{fill}%
\end{pgfscope}%
\begin{pgfscope}%
\pgfpathrectangle{\pgfqpoint{0.617715in}{1.130578in}}{\pgfqpoint{5.617285in}{3.281798in}}%
\pgfusepath{clip}%
\pgfsetbuttcap%
\pgfsetmiterjoin%
\definecolor{currentfill}{rgb}{0.172549,0.627451,0.172549}%
\pgfsetfillcolor{currentfill}%
\pgfsetlinewidth{0.000000pt}%
\definecolor{currentstroke}{rgb}{0.000000,0.000000,0.000000}%
\pgfsetstrokecolor{currentstroke}%
\pgfsetstrokeopacity{0.000000}%
\pgfsetdash{}{0pt}%
\pgfpathmoveto{\pgfqpoint{2.046189in}{1.130578in}}%
\pgfpathlineto{\pgfqpoint{2.115198in}{1.130578in}}%
\pgfpathlineto{\pgfqpoint{2.115198in}{1.207300in}}%
\pgfpathlineto{\pgfqpoint{2.046189in}{1.207300in}}%
\pgfpathlineto{\pgfqpoint{2.046189in}{1.130578in}}%
\pgfpathclose%
\pgfusepath{fill}%
\end{pgfscope}%
\begin{pgfscope}%
\pgfpathrectangle{\pgfqpoint{0.617715in}{1.130578in}}{\pgfqpoint{5.617285in}{3.281798in}}%
\pgfusepath{clip}%
\pgfsetbuttcap%
\pgfsetmiterjoin%
\definecolor{currentfill}{rgb}{0.172549,0.627451,0.172549}%
\pgfsetfillcolor{currentfill}%
\pgfsetlinewidth{0.000000pt}%
\definecolor{currentstroke}{rgb}{0.000000,0.000000,0.000000}%
\pgfsetstrokecolor{currentstroke}%
\pgfsetstrokeopacity{0.000000}%
\pgfsetdash{}{0pt}%
\pgfpathmoveto{\pgfqpoint{2.391231in}{1.130578in}}%
\pgfpathlineto{\pgfqpoint{2.460240in}{1.130578in}}%
\pgfpathlineto{\pgfqpoint{2.460240in}{1.144715in}}%
\pgfpathlineto{\pgfqpoint{2.391231in}{1.144715in}}%
\pgfpathlineto{\pgfqpoint{2.391231in}{1.130578in}}%
\pgfpathclose%
\pgfusepath{fill}%
\end{pgfscope}%
\begin{pgfscope}%
\pgfpathrectangle{\pgfqpoint{0.617715in}{1.130578in}}{\pgfqpoint{5.617285in}{3.281798in}}%
\pgfusepath{clip}%
\pgfsetbuttcap%
\pgfsetmiterjoin%
\definecolor{currentfill}{rgb}{0.172549,0.627451,0.172549}%
\pgfsetfillcolor{currentfill}%
\pgfsetlinewidth{0.000000pt}%
\definecolor{currentstroke}{rgb}{0.000000,0.000000,0.000000}%
\pgfsetstrokecolor{currentstroke}%
\pgfsetstrokeopacity{0.000000}%
\pgfsetdash{}{0pt}%
\pgfpathmoveto{\pgfqpoint{2.736274in}{1.130578in}}%
\pgfpathlineto{\pgfqpoint{2.805282in}{1.130578in}}%
\pgfpathlineto{\pgfqpoint{2.805282in}{1.130578in}}%
\pgfpathlineto{\pgfqpoint{2.736274in}{1.130578in}}%
\pgfpathlineto{\pgfqpoint{2.736274in}{1.130578in}}%
\pgfpathclose%
\pgfusepath{fill}%
\end{pgfscope}%
\begin{pgfscope}%
\pgfpathrectangle{\pgfqpoint{0.617715in}{1.130578in}}{\pgfqpoint{5.617285in}{3.281798in}}%
\pgfusepath{clip}%
\pgfsetbuttcap%
\pgfsetmiterjoin%
\definecolor{currentfill}{rgb}{0.172549,0.627451,0.172549}%
\pgfsetfillcolor{currentfill}%
\pgfsetlinewidth{0.000000pt}%
\definecolor{currentstroke}{rgb}{0.000000,0.000000,0.000000}%
\pgfsetstrokecolor{currentstroke}%
\pgfsetstrokeopacity{0.000000}%
\pgfsetdash{}{0pt}%
\pgfpathmoveto{\pgfqpoint{3.081316in}{1.130578in}}%
\pgfpathlineto{\pgfqpoint{3.150324in}{1.130578in}}%
\pgfpathlineto{\pgfqpoint{3.150324in}{1.130578in}}%
\pgfpathlineto{\pgfqpoint{3.081316in}{1.130578in}}%
\pgfpathlineto{\pgfqpoint{3.081316in}{1.130578in}}%
\pgfpathclose%
\pgfusepath{fill}%
\end{pgfscope}%
\begin{pgfscope}%
\pgfpathrectangle{\pgfqpoint{0.617715in}{1.130578in}}{\pgfqpoint{5.617285in}{3.281798in}}%
\pgfusepath{clip}%
\pgfsetbuttcap%
\pgfsetmiterjoin%
\definecolor{currentfill}{rgb}{0.172549,0.627451,0.172549}%
\pgfsetfillcolor{currentfill}%
\pgfsetlinewidth{0.000000pt}%
\definecolor{currentstroke}{rgb}{0.000000,0.000000,0.000000}%
\pgfsetstrokecolor{currentstroke}%
\pgfsetstrokeopacity{0.000000}%
\pgfsetdash{}{0pt}%
\pgfpathmoveto{\pgfqpoint{3.426358in}{1.130578in}}%
\pgfpathlineto{\pgfqpoint{3.495366in}{1.130578in}}%
\pgfpathlineto{\pgfqpoint{3.495366in}{1.307911in}}%
\pgfpathlineto{\pgfqpoint{3.426358in}{1.307911in}}%
\pgfpathlineto{\pgfqpoint{3.426358in}{1.130578in}}%
\pgfpathclose%
\pgfusepath{fill}%
\end{pgfscope}%
\begin{pgfscope}%
\pgfpathrectangle{\pgfqpoint{0.617715in}{1.130578in}}{\pgfqpoint{5.617285in}{3.281798in}}%
\pgfusepath{clip}%
\pgfsetbuttcap%
\pgfsetmiterjoin%
\definecolor{currentfill}{rgb}{0.172549,0.627451,0.172549}%
\pgfsetfillcolor{currentfill}%
\pgfsetlinewidth{0.000000pt}%
\definecolor{currentstroke}{rgb}{0.000000,0.000000,0.000000}%
\pgfsetstrokecolor{currentstroke}%
\pgfsetstrokeopacity{0.000000}%
\pgfsetdash{}{0pt}%
\pgfpathmoveto{\pgfqpoint{3.771400in}{1.130578in}}%
\pgfpathlineto{\pgfqpoint{3.840408in}{1.130578in}}%
\pgfpathlineto{\pgfqpoint{3.840408in}{1.209037in}}%
\pgfpathlineto{\pgfqpoint{3.771400in}{1.209037in}}%
\pgfpathlineto{\pgfqpoint{3.771400in}{1.130578in}}%
\pgfpathclose%
\pgfusepath{fill}%
\end{pgfscope}%
\begin{pgfscope}%
\pgfpathrectangle{\pgfqpoint{0.617715in}{1.130578in}}{\pgfqpoint{5.617285in}{3.281798in}}%
\pgfusepath{clip}%
\pgfsetbuttcap%
\pgfsetmiterjoin%
\definecolor{currentfill}{rgb}{0.172549,0.627451,0.172549}%
\pgfsetfillcolor{currentfill}%
\pgfsetlinewidth{0.000000pt}%
\definecolor{currentstroke}{rgb}{0.000000,0.000000,0.000000}%
\pgfsetstrokecolor{currentstroke}%
\pgfsetstrokeopacity{0.000000}%
\pgfsetdash{}{0pt}%
\pgfpathmoveto{\pgfqpoint{4.116442in}{1.130578in}}%
\pgfpathlineto{\pgfqpoint{4.185450in}{1.130578in}}%
\pgfpathlineto{\pgfqpoint{4.185450in}{1.130690in}}%
\pgfpathlineto{\pgfqpoint{4.116442in}{1.130690in}}%
\pgfpathlineto{\pgfqpoint{4.116442in}{1.130578in}}%
\pgfpathclose%
\pgfusepath{fill}%
\end{pgfscope}%
\begin{pgfscope}%
\pgfpathrectangle{\pgfqpoint{0.617715in}{1.130578in}}{\pgfqpoint{5.617285in}{3.281798in}}%
\pgfusepath{clip}%
\pgfsetbuttcap%
\pgfsetmiterjoin%
\definecolor{currentfill}{rgb}{0.172549,0.627451,0.172549}%
\pgfsetfillcolor{currentfill}%
\pgfsetlinewidth{0.000000pt}%
\definecolor{currentstroke}{rgb}{0.000000,0.000000,0.000000}%
\pgfsetstrokecolor{currentstroke}%
\pgfsetstrokeopacity{0.000000}%
\pgfsetdash{}{0pt}%
\pgfpathmoveto{\pgfqpoint{4.461484in}{1.130578in}}%
\pgfpathlineto{\pgfqpoint{4.530492in}{1.130578in}}%
\pgfpathlineto{\pgfqpoint{4.530492in}{1.131519in}}%
\pgfpathlineto{\pgfqpoint{4.461484in}{1.131519in}}%
\pgfpathlineto{\pgfqpoint{4.461484in}{1.130578in}}%
\pgfpathclose%
\pgfusepath{fill}%
\end{pgfscope}%
\begin{pgfscope}%
\pgfpathrectangle{\pgfqpoint{0.617715in}{1.130578in}}{\pgfqpoint{5.617285in}{3.281798in}}%
\pgfusepath{clip}%
\pgfsetbuttcap%
\pgfsetmiterjoin%
\definecolor{currentfill}{rgb}{0.172549,0.627451,0.172549}%
\pgfsetfillcolor{currentfill}%
\pgfsetlinewidth{0.000000pt}%
\definecolor{currentstroke}{rgb}{0.000000,0.000000,0.000000}%
\pgfsetstrokecolor{currentstroke}%
\pgfsetstrokeopacity{0.000000}%
\pgfsetdash{}{0pt}%
\pgfpathmoveto{\pgfqpoint{4.806526in}{1.130578in}}%
\pgfpathlineto{\pgfqpoint{4.875534in}{1.130578in}}%
\pgfpathlineto{\pgfqpoint{4.875534in}{1.200811in}}%
\pgfpathlineto{\pgfqpoint{4.806526in}{1.200811in}}%
\pgfpathlineto{\pgfqpoint{4.806526in}{1.130578in}}%
\pgfpathclose%
\pgfusepath{fill}%
\end{pgfscope}%
\begin{pgfscope}%
\pgfpathrectangle{\pgfqpoint{0.617715in}{1.130578in}}{\pgfqpoint{5.617285in}{3.281798in}}%
\pgfusepath{clip}%
\pgfsetbuttcap%
\pgfsetmiterjoin%
\definecolor{currentfill}{rgb}{0.172549,0.627451,0.172549}%
\pgfsetfillcolor{currentfill}%
\pgfsetlinewidth{0.000000pt}%
\definecolor{currentstroke}{rgb}{0.000000,0.000000,0.000000}%
\pgfsetstrokecolor{currentstroke}%
\pgfsetstrokeopacity{0.000000}%
\pgfsetdash{}{0pt}%
\pgfpathmoveto{\pgfqpoint{5.151568in}{1.130578in}}%
\pgfpathlineto{\pgfqpoint{5.220576in}{1.130578in}}%
\pgfpathlineto{\pgfqpoint{5.220576in}{3.213738in}}%
\pgfpathlineto{\pgfqpoint{5.151568in}{3.213738in}}%
\pgfpathlineto{\pgfqpoint{5.151568in}{1.130578in}}%
\pgfpathclose%
\pgfusepath{fill}%
\end{pgfscope}%
\begin{pgfscope}%
\pgfpathrectangle{\pgfqpoint{0.617715in}{1.130578in}}{\pgfqpoint{5.617285in}{3.281798in}}%
\pgfusepath{clip}%
\pgfsetbuttcap%
\pgfsetmiterjoin%
\definecolor{currentfill}{rgb}{0.172549,0.627451,0.172549}%
\pgfsetfillcolor{currentfill}%
\pgfsetlinewidth{0.000000pt}%
\definecolor{currentstroke}{rgb}{0.000000,0.000000,0.000000}%
\pgfsetstrokecolor{currentstroke}%
\pgfsetstrokeopacity{0.000000}%
\pgfsetdash{}{0pt}%
\pgfpathmoveto{\pgfqpoint{5.496610in}{1.130578in}}%
\pgfpathlineto{\pgfqpoint{5.565618in}{1.130578in}}%
\pgfpathlineto{\pgfqpoint{5.565618in}{1.195014in}}%
\pgfpathlineto{\pgfqpoint{5.496610in}{1.195014in}}%
\pgfpathlineto{\pgfqpoint{5.496610in}{1.130578in}}%
\pgfpathclose%
\pgfusepath{fill}%
\end{pgfscope}%
\begin{pgfscope}%
\pgfpathrectangle{\pgfqpoint{0.617715in}{1.130578in}}{\pgfqpoint{5.617285in}{3.281798in}}%
\pgfusepath{clip}%
\pgfsetbuttcap%
\pgfsetmiterjoin%
\definecolor{currentfill}{rgb}{0.172549,0.627451,0.172549}%
\pgfsetfillcolor{currentfill}%
\pgfsetlinewidth{0.000000pt}%
\definecolor{currentstroke}{rgb}{0.000000,0.000000,0.000000}%
\pgfsetstrokecolor{currentstroke}%
\pgfsetstrokeopacity{0.000000}%
\pgfsetdash{}{0pt}%
\pgfpathmoveto{\pgfqpoint{5.841652in}{1.130578in}}%
\pgfpathlineto{\pgfqpoint{5.910660in}{1.130578in}}%
\pgfpathlineto{\pgfqpoint{5.910660in}{1.132743in}}%
\pgfpathlineto{\pgfqpoint{5.841652in}{1.132743in}}%
\pgfpathlineto{\pgfqpoint{5.841652in}{1.130578in}}%
\pgfpathclose%
\pgfusepath{fill}%
\end{pgfscope}%
\begin{pgfscope}%
\pgfpathrectangle{\pgfqpoint{0.617715in}{1.130578in}}{\pgfqpoint{5.617285in}{3.281798in}}%
\pgfusepath{clip}%
\pgfsetbuttcap%
\pgfsetmiterjoin%
\definecolor{currentfill}{rgb}{0.839216,0.152941,0.156863}%
\pgfsetfillcolor{currentfill}%
\pgfsetlinewidth{0.000000pt}%
\definecolor{currentstroke}{rgb}{0.000000,0.000000,0.000000}%
\pgfsetstrokecolor{currentstroke}%
\pgfsetstrokeopacity{0.000000}%
\pgfsetdash{}{0pt}%
\pgfpathmoveto{\pgfqpoint{1.080072in}{1.130578in}}%
\pgfpathlineto{\pgfqpoint{1.149080in}{1.130578in}}%
\pgfpathlineto{\pgfqpoint{1.149080in}{1.135095in}}%
\pgfpathlineto{\pgfqpoint{1.080072in}{1.135095in}}%
\pgfpathlineto{\pgfqpoint{1.080072in}{1.130578in}}%
\pgfpathclose%
\pgfusepath{fill}%
\end{pgfscope}%
\begin{pgfscope}%
\pgfpathrectangle{\pgfqpoint{0.617715in}{1.130578in}}{\pgfqpoint{5.617285in}{3.281798in}}%
\pgfusepath{clip}%
\pgfsetbuttcap%
\pgfsetmiterjoin%
\definecolor{currentfill}{rgb}{0.839216,0.152941,0.156863}%
\pgfsetfillcolor{currentfill}%
\pgfsetlinewidth{0.000000pt}%
\definecolor{currentstroke}{rgb}{0.000000,0.000000,0.000000}%
\pgfsetstrokecolor{currentstroke}%
\pgfsetstrokeopacity{0.000000}%
\pgfsetdash{}{0pt}%
\pgfpathmoveto{\pgfqpoint{1.425114in}{1.130578in}}%
\pgfpathlineto{\pgfqpoint{1.494122in}{1.130578in}}%
\pgfpathlineto{\pgfqpoint{1.494122in}{1.130663in}}%
\pgfpathlineto{\pgfqpoint{1.425114in}{1.130663in}}%
\pgfpathlineto{\pgfqpoint{1.425114in}{1.130578in}}%
\pgfpathclose%
\pgfusepath{fill}%
\end{pgfscope}%
\begin{pgfscope}%
\pgfpathrectangle{\pgfqpoint{0.617715in}{1.130578in}}{\pgfqpoint{5.617285in}{3.281798in}}%
\pgfusepath{clip}%
\pgfsetbuttcap%
\pgfsetmiterjoin%
\definecolor{currentfill}{rgb}{0.839216,0.152941,0.156863}%
\pgfsetfillcolor{currentfill}%
\pgfsetlinewidth{0.000000pt}%
\definecolor{currentstroke}{rgb}{0.000000,0.000000,0.000000}%
\pgfsetstrokecolor{currentstroke}%
\pgfsetstrokeopacity{0.000000}%
\pgfsetdash{}{0pt}%
\pgfpathmoveto{\pgfqpoint{1.770156in}{1.130578in}}%
\pgfpathlineto{\pgfqpoint{1.839164in}{1.130578in}}%
\pgfpathlineto{\pgfqpoint{1.839164in}{1.131820in}}%
\pgfpathlineto{\pgfqpoint{1.770156in}{1.131820in}}%
\pgfpathlineto{\pgfqpoint{1.770156in}{1.130578in}}%
\pgfpathclose%
\pgfusepath{fill}%
\end{pgfscope}%
\begin{pgfscope}%
\pgfpathrectangle{\pgfqpoint{0.617715in}{1.130578in}}{\pgfqpoint{5.617285in}{3.281798in}}%
\pgfusepath{clip}%
\pgfsetbuttcap%
\pgfsetmiterjoin%
\definecolor{currentfill}{rgb}{0.839216,0.152941,0.156863}%
\pgfsetfillcolor{currentfill}%
\pgfsetlinewidth{0.000000pt}%
\definecolor{currentstroke}{rgb}{0.000000,0.000000,0.000000}%
\pgfsetstrokecolor{currentstroke}%
\pgfsetstrokeopacity{0.000000}%
\pgfsetdash{}{0pt}%
\pgfpathmoveto{\pgfqpoint{2.115198in}{1.130578in}}%
\pgfpathlineto{\pgfqpoint{2.184206in}{1.130578in}}%
\pgfpathlineto{\pgfqpoint{2.184206in}{1.170834in}}%
\pgfpathlineto{\pgfqpoint{2.115198in}{1.170834in}}%
\pgfpathlineto{\pgfqpoint{2.115198in}{1.130578in}}%
\pgfpathclose%
\pgfusepath{fill}%
\end{pgfscope}%
\begin{pgfscope}%
\pgfpathrectangle{\pgfqpoint{0.617715in}{1.130578in}}{\pgfqpoint{5.617285in}{3.281798in}}%
\pgfusepath{clip}%
\pgfsetbuttcap%
\pgfsetmiterjoin%
\definecolor{currentfill}{rgb}{0.839216,0.152941,0.156863}%
\pgfsetfillcolor{currentfill}%
\pgfsetlinewidth{0.000000pt}%
\definecolor{currentstroke}{rgb}{0.000000,0.000000,0.000000}%
\pgfsetstrokecolor{currentstroke}%
\pgfsetstrokeopacity{0.000000}%
\pgfsetdash{}{0pt}%
\pgfpathmoveto{\pgfqpoint{2.460240in}{1.130578in}}%
\pgfpathlineto{\pgfqpoint{2.529248in}{1.130578in}}%
\pgfpathlineto{\pgfqpoint{2.529248in}{1.144212in}}%
\pgfpathlineto{\pgfqpoint{2.460240in}{1.144212in}}%
\pgfpathlineto{\pgfqpoint{2.460240in}{1.130578in}}%
\pgfpathclose%
\pgfusepath{fill}%
\end{pgfscope}%
\begin{pgfscope}%
\pgfpathrectangle{\pgfqpoint{0.617715in}{1.130578in}}{\pgfqpoint{5.617285in}{3.281798in}}%
\pgfusepath{clip}%
\pgfsetbuttcap%
\pgfsetmiterjoin%
\definecolor{currentfill}{rgb}{0.839216,0.152941,0.156863}%
\pgfsetfillcolor{currentfill}%
\pgfsetlinewidth{0.000000pt}%
\definecolor{currentstroke}{rgb}{0.000000,0.000000,0.000000}%
\pgfsetstrokecolor{currentstroke}%
\pgfsetstrokeopacity{0.000000}%
\pgfsetdash{}{0pt}%
\pgfpathmoveto{\pgfqpoint{2.805282in}{1.130578in}}%
\pgfpathlineto{\pgfqpoint{2.874290in}{1.130578in}}%
\pgfpathlineto{\pgfqpoint{2.874290in}{1.130578in}}%
\pgfpathlineto{\pgfqpoint{2.805282in}{1.130578in}}%
\pgfpathlineto{\pgfqpoint{2.805282in}{1.130578in}}%
\pgfpathclose%
\pgfusepath{fill}%
\end{pgfscope}%
\begin{pgfscope}%
\pgfpathrectangle{\pgfqpoint{0.617715in}{1.130578in}}{\pgfqpoint{5.617285in}{3.281798in}}%
\pgfusepath{clip}%
\pgfsetbuttcap%
\pgfsetmiterjoin%
\definecolor{currentfill}{rgb}{0.839216,0.152941,0.156863}%
\pgfsetfillcolor{currentfill}%
\pgfsetlinewidth{0.000000pt}%
\definecolor{currentstroke}{rgb}{0.000000,0.000000,0.000000}%
\pgfsetstrokecolor{currentstroke}%
\pgfsetstrokeopacity{0.000000}%
\pgfsetdash{}{0pt}%
\pgfpathmoveto{\pgfqpoint{3.150324in}{1.130578in}}%
\pgfpathlineto{\pgfqpoint{3.219332in}{1.130578in}}%
\pgfpathlineto{\pgfqpoint{3.219332in}{1.130578in}}%
\pgfpathlineto{\pgfqpoint{3.150324in}{1.130578in}}%
\pgfpathlineto{\pgfqpoint{3.150324in}{1.130578in}}%
\pgfpathclose%
\pgfusepath{fill}%
\end{pgfscope}%
\begin{pgfscope}%
\pgfpathrectangle{\pgfqpoint{0.617715in}{1.130578in}}{\pgfqpoint{5.617285in}{3.281798in}}%
\pgfusepath{clip}%
\pgfsetbuttcap%
\pgfsetmiterjoin%
\definecolor{currentfill}{rgb}{0.839216,0.152941,0.156863}%
\pgfsetfillcolor{currentfill}%
\pgfsetlinewidth{0.000000pt}%
\definecolor{currentstroke}{rgb}{0.000000,0.000000,0.000000}%
\pgfsetstrokecolor{currentstroke}%
\pgfsetstrokeopacity{0.000000}%
\pgfsetdash{}{0pt}%
\pgfpathmoveto{\pgfqpoint{3.495366in}{1.130578in}}%
\pgfpathlineto{\pgfqpoint{3.564374in}{1.130578in}}%
\pgfpathlineto{\pgfqpoint{3.564374in}{1.155677in}}%
\pgfpathlineto{\pgfqpoint{3.495366in}{1.155677in}}%
\pgfpathlineto{\pgfqpoint{3.495366in}{1.130578in}}%
\pgfpathclose%
\pgfusepath{fill}%
\end{pgfscope}%
\begin{pgfscope}%
\pgfpathrectangle{\pgfqpoint{0.617715in}{1.130578in}}{\pgfqpoint{5.617285in}{3.281798in}}%
\pgfusepath{clip}%
\pgfsetbuttcap%
\pgfsetmiterjoin%
\definecolor{currentfill}{rgb}{0.839216,0.152941,0.156863}%
\pgfsetfillcolor{currentfill}%
\pgfsetlinewidth{0.000000pt}%
\definecolor{currentstroke}{rgb}{0.000000,0.000000,0.000000}%
\pgfsetstrokecolor{currentstroke}%
\pgfsetstrokeopacity{0.000000}%
\pgfsetdash{}{0pt}%
\pgfpathmoveto{\pgfqpoint{3.840408in}{1.130578in}}%
\pgfpathlineto{\pgfqpoint{3.909417in}{1.130578in}}%
\pgfpathlineto{\pgfqpoint{3.909417in}{1.130578in}}%
\pgfpathlineto{\pgfqpoint{3.840408in}{1.130578in}}%
\pgfpathlineto{\pgfqpoint{3.840408in}{1.130578in}}%
\pgfpathclose%
\pgfusepath{fill}%
\end{pgfscope}%
\begin{pgfscope}%
\pgfpathrectangle{\pgfqpoint{0.617715in}{1.130578in}}{\pgfqpoint{5.617285in}{3.281798in}}%
\pgfusepath{clip}%
\pgfsetbuttcap%
\pgfsetmiterjoin%
\definecolor{currentfill}{rgb}{0.839216,0.152941,0.156863}%
\pgfsetfillcolor{currentfill}%
\pgfsetlinewidth{0.000000pt}%
\definecolor{currentstroke}{rgb}{0.000000,0.000000,0.000000}%
\pgfsetstrokecolor{currentstroke}%
\pgfsetstrokeopacity{0.000000}%
\pgfsetdash{}{0pt}%
\pgfpathmoveto{\pgfqpoint{4.185450in}{1.130578in}}%
\pgfpathlineto{\pgfqpoint{4.254459in}{1.130578in}}%
\pgfpathlineto{\pgfqpoint{4.254459in}{1.130587in}}%
\pgfpathlineto{\pgfqpoint{4.185450in}{1.130587in}}%
\pgfpathlineto{\pgfqpoint{4.185450in}{1.130578in}}%
\pgfpathclose%
\pgfusepath{fill}%
\end{pgfscope}%
\begin{pgfscope}%
\pgfpathrectangle{\pgfqpoint{0.617715in}{1.130578in}}{\pgfqpoint{5.617285in}{3.281798in}}%
\pgfusepath{clip}%
\pgfsetbuttcap%
\pgfsetmiterjoin%
\definecolor{currentfill}{rgb}{0.839216,0.152941,0.156863}%
\pgfsetfillcolor{currentfill}%
\pgfsetlinewidth{0.000000pt}%
\definecolor{currentstroke}{rgb}{0.000000,0.000000,0.000000}%
\pgfsetstrokecolor{currentstroke}%
\pgfsetstrokeopacity{0.000000}%
\pgfsetdash{}{0pt}%
\pgfpathmoveto{\pgfqpoint{4.530492in}{1.130578in}}%
\pgfpathlineto{\pgfqpoint{4.599501in}{1.130578in}}%
\pgfpathlineto{\pgfqpoint{4.599501in}{1.131473in}}%
\pgfpathlineto{\pgfqpoint{4.530492in}{1.131473in}}%
\pgfpathlineto{\pgfqpoint{4.530492in}{1.130578in}}%
\pgfpathclose%
\pgfusepath{fill}%
\end{pgfscope}%
\begin{pgfscope}%
\pgfpathrectangle{\pgfqpoint{0.617715in}{1.130578in}}{\pgfqpoint{5.617285in}{3.281798in}}%
\pgfusepath{clip}%
\pgfsetbuttcap%
\pgfsetmiterjoin%
\definecolor{currentfill}{rgb}{0.839216,0.152941,0.156863}%
\pgfsetfillcolor{currentfill}%
\pgfsetlinewidth{0.000000pt}%
\definecolor{currentstroke}{rgb}{0.000000,0.000000,0.000000}%
\pgfsetstrokecolor{currentstroke}%
\pgfsetstrokeopacity{0.000000}%
\pgfsetdash{}{0pt}%
\pgfpathmoveto{\pgfqpoint{4.875534in}{1.130578in}}%
\pgfpathlineto{\pgfqpoint{4.944543in}{1.130578in}}%
\pgfpathlineto{\pgfqpoint{4.944543in}{1.179230in}}%
\pgfpathlineto{\pgfqpoint{4.875534in}{1.179230in}}%
\pgfpathlineto{\pgfqpoint{4.875534in}{1.130578in}}%
\pgfpathclose%
\pgfusepath{fill}%
\end{pgfscope}%
\begin{pgfscope}%
\pgfpathrectangle{\pgfqpoint{0.617715in}{1.130578in}}{\pgfqpoint{5.617285in}{3.281798in}}%
\pgfusepath{clip}%
\pgfsetbuttcap%
\pgfsetmiterjoin%
\definecolor{currentfill}{rgb}{0.839216,0.152941,0.156863}%
\pgfsetfillcolor{currentfill}%
\pgfsetlinewidth{0.000000pt}%
\definecolor{currentstroke}{rgb}{0.000000,0.000000,0.000000}%
\pgfsetstrokecolor{currentstroke}%
\pgfsetstrokeopacity{0.000000}%
\pgfsetdash{}{0pt}%
\pgfpathmoveto{\pgfqpoint{5.220576in}{1.130578in}}%
\pgfpathlineto{\pgfqpoint{5.289585in}{1.130578in}}%
\pgfpathlineto{\pgfqpoint{5.289585in}{1.130578in}}%
\pgfpathlineto{\pgfqpoint{5.220576in}{1.130578in}}%
\pgfpathlineto{\pgfqpoint{5.220576in}{1.130578in}}%
\pgfpathclose%
\pgfusepath{fill}%
\end{pgfscope}%
\begin{pgfscope}%
\pgfpathrectangle{\pgfqpoint{0.617715in}{1.130578in}}{\pgfqpoint{5.617285in}{3.281798in}}%
\pgfusepath{clip}%
\pgfsetbuttcap%
\pgfsetmiterjoin%
\definecolor{currentfill}{rgb}{0.839216,0.152941,0.156863}%
\pgfsetfillcolor{currentfill}%
\pgfsetlinewidth{0.000000pt}%
\definecolor{currentstroke}{rgb}{0.000000,0.000000,0.000000}%
\pgfsetstrokecolor{currentstroke}%
\pgfsetstrokeopacity{0.000000}%
\pgfsetdash{}{0pt}%
\pgfpathmoveto{\pgfqpoint{5.565618in}{1.130578in}}%
\pgfpathlineto{\pgfqpoint{5.634627in}{1.130578in}}%
\pgfpathlineto{\pgfqpoint{5.634627in}{1.145091in}}%
\pgfpathlineto{\pgfqpoint{5.565618in}{1.145091in}}%
\pgfpathlineto{\pgfqpoint{5.565618in}{1.130578in}}%
\pgfpathclose%
\pgfusepath{fill}%
\end{pgfscope}%
\begin{pgfscope}%
\pgfpathrectangle{\pgfqpoint{0.617715in}{1.130578in}}{\pgfqpoint{5.617285in}{3.281798in}}%
\pgfusepath{clip}%
\pgfsetbuttcap%
\pgfsetmiterjoin%
\definecolor{currentfill}{rgb}{0.839216,0.152941,0.156863}%
\pgfsetfillcolor{currentfill}%
\pgfsetlinewidth{0.000000pt}%
\definecolor{currentstroke}{rgb}{0.000000,0.000000,0.000000}%
\pgfsetstrokecolor{currentstroke}%
\pgfsetstrokeopacity{0.000000}%
\pgfsetdash{}{0pt}%
\pgfpathmoveto{\pgfqpoint{5.910660in}{1.130578in}}%
\pgfpathlineto{\pgfqpoint{5.979669in}{1.130578in}}%
\pgfpathlineto{\pgfqpoint{5.979669in}{1.132193in}}%
\pgfpathlineto{\pgfqpoint{5.910660in}{1.132193in}}%
\pgfpathlineto{\pgfqpoint{5.910660in}{1.130578in}}%
\pgfpathclose%
\pgfusepath{fill}%
\end{pgfscope}%
\begin{pgfscope}%
\pgfsetbuttcap%
\pgfsetroundjoin%
\definecolor{currentfill}{rgb}{0.000000,0.000000,0.000000}%
\pgfsetfillcolor{currentfill}%
\pgfsetlinewidth{0.803000pt}%
\definecolor{currentstroke}{rgb}{0.000000,0.000000,0.000000}%
\pgfsetstrokecolor{currentstroke}%
\pgfsetdash{}{0pt}%
\pgfsys@defobject{currentmarker}{\pgfqpoint{0.000000in}{-0.048611in}}{\pgfqpoint{0.000000in}{0.000000in}}{%
\pgfpathmoveto{\pgfqpoint{0.000000in}{0.000000in}}%
\pgfpathlineto{\pgfqpoint{0.000000in}{-0.048611in}}%
\pgfusepath{stroke,fill}%
}%
\begin{pgfscope}%
\pgfsys@transformshift{0.976559in}{1.130578in}%
\pgfsys@useobject{currentmarker}{}%
\end{pgfscope}%
\end{pgfscope}%
\begin{pgfscope}%
\definecolor{textcolor}{rgb}{0.000000,0.000000,0.000000}%
\pgfsetstrokecolor{textcolor}%
\pgfsetfillcolor{textcolor}%
\pgftext[x=0.908091in, y=0.754735in, left, base,rotate=45.000000]{\color{textcolor}{\sffamily\fontsize{11.000000}{13.200000}\selectfont\catcode`\^=\active\def^{\ifmmode\sp\else\^{}\fi}\catcode`\%=\active\def%{\%}dcg}}%
\end{pgfscope}%
\begin{pgfscope}%
\pgfsetbuttcap%
\pgfsetroundjoin%
\definecolor{currentfill}{rgb}{0.000000,0.000000,0.000000}%
\pgfsetfillcolor{currentfill}%
\pgfsetlinewidth{0.803000pt}%
\definecolor{currentstroke}{rgb}{0.000000,0.000000,0.000000}%
\pgfsetstrokecolor{currentstroke}%
\pgfsetdash{}{0pt}%
\pgfsys@defobject{currentmarker}{\pgfqpoint{0.000000in}{-0.048611in}}{\pgfqpoint{0.000000in}{0.000000in}}{%
\pgfpathmoveto{\pgfqpoint{0.000000in}{0.000000in}}%
\pgfpathlineto{\pgfqpoint{0.000000in}{-0.048611in}}%
\pgfusepath{stroke,fill}%
}%
\begin{pgfscope}%
\pgfsys@transformshift{1.321601in}{1.130578in}%
\pgfsys@useobject{currentmarker}{}%
\end{pgfscope}%
\end{pgfscope}%
\begin{pgfscope}%
\definecolor{textcolor}{rgb}{0.000000,0.000000,0.000000}%
\pgfsetstrokecolor{textcolor}%
\pgfsetfillcolor{textcolor}%
\pgftext[x=1.261573in, y=0.771615in, left, base,rotate=45.000000]{\color{textcolor}{\sffamily\fontsize{11.000000}{13.200000}\selectfont\catcode`\^=\active\def^{\ifmmode\sp\else\^{}\fi}\catcode`\%=\active\def%{\%}lmr}}%
\end{pgfscope}%
\begin{pgfscope}%
\pgfsetbuttcap%
\pgfsetroundjoin%
\definecolor{currentfill}{rgb}{0.000000,0.000000,0.000000}%
\pgfsetfillcolor{currentfill}%
\pgfsetlinewidth{0.803000pt}%
\definecolor{currentstroke}{rgb}{0.000000,0.000000,0.000000}%
\pgfsetstrokecolor{currentstroke}%
\pgfsetdash{}{0pt}%
\pgfsys@defobject{currentmarker}{\pgfqpoint{0.000000in}{-0.048611in}}{\pgfqpoint{0.000000in}{0.000000in}}{%
\pgfpathmoveto{\pgfqpoint{0.000000in}{0.000000in}}%
\pgfpathlineto{\pgfqpoint{0.000000in}{-0.048611in}}%
\pgfusepath{stroke,fill}%
}%
\begin{pgfscope}%
\pgfsys@transformshift{1.666643in}{1.130578in}%
\pgfsys@useobject{currentmarker}{}%
\end{pgfscope}%
\end{pgfscope}%
\begin{pgfscope}%
\definecolor{textcolor}{rgb}{0.000000,0.000000,0.000000}%
\pgfsetstrokecolor{textcolor}%
\pgfsetfillcolor{textcolor}%
\pgftext[x=1.456675in, y=0.471736in, left, base,rotate=45.000000]{\color{textcolor}{\sffamily\fontsize{11.000000}{13.200000}\selectfont\catcode`\^=\active\def^{\ifmmode\sp\else\^{}\fi}\catcode`\%=\active\def%{\%}resnet50}}%
\end{pgfscope}%
\begin{pgfscope}%
\pgfsetbuttcap%
\pgfsetroundjoin%
\definecolor{currentfill}{rgb}{0.000000,0.000000,0.000000}%
\pgfsetfillcolor{currentfill}%
\pgfsetlinewidth{0.803000pt}%
\definecolor{currentstroke}{rgb}{0.000000,0.000000,0.000000}%
\pgfsetstrokecolor{currentstroke}%
\pgfsetdash{}{0pt}%
\pgfsys@defobject{currentmarker}{\pgfqpoint{0.000000in}{-0.048611in}}{\pgfqpoint{0.000000in}{0.000000in}}{%
\pgfpathmoveto{\pgfqpoint{0.000000in}{0.000000in}}%
\pgfpathlineto{\pgfqpoint{0.000000in}{-0.048611in}}%
\pgfusepath{stroke,fill}%
}%
\begin{pgfscope}%
\pgfsys@transformshift{2.011685in}{1.130578in}%
\pgfsys@useobject{currentmarker}{}%
\end{pgfscope}%
\end{pgfscope}%
\begin{pgfscope}%
\definecolor{textcolor}{rgb}{0.000000,0.000000,0.000000}%
\pgfsetstrokecolor{textcolor}%
\pgfsetfillcolor{textcolor}%
\pgftext[x=1.971016in, y=0.810333in, left, base,rotate=45.000000]{\color{textcolor}{\sffamily\fontsize{11.000000}{13.200000}\selectfont\catcode`\^=\active\def^{\ifmmode\sp\else\^{}\fi}\catcode`\%=\active\def%{\%}lgt}}%
\end{pgfscope}%
\begin{pgfscope}%
\pgfsetbuttcap%
\pgfsetroundjoin%
\definecolor{currentfill}{rgb}{0.000000,0.000000,0.000000}%
\pgfsetfillcolor{currentfill}%
\pgfsetlinewidth{0.803000pt}%
\definecolor{currentstroke}{rgb}{0.000000,0.000000,0.000000}%
\pgfsetstrokecolor{currentstroke}%
\pgfsetdash{}{0pt}%
\pgfsys@defobject{currentmarker}{\pgfqpoint{0.000000in}{-0.048611in}}{\pgfqpoint{0.000000in}{0.000000in}}{%
\pgfpathmoveto{\pgfqpoint{0.000000in}{0.000000in}}%
\pgfpathlineto{\pgfqpoint{0.000000in}{-0.048611in}}%
\pgfusepath{stroke,fill}%
}%
\begin{pgfscope}%
\pgfsys@transformshift{2.356727in}{1.130578in}%
\pgfsys@useobject{currentmarker}{}%
\end{pgfscope}%
\end{pgfscope}%
\begin{pgfscope}%
\definecolor{textcolor}{rgb}{0.000000,0.000000,0.000000}%
\pgfsetstrokecolor{textcolor}%
\pgfsetfillcolor{textcolor}%
\pgftext[x=2.295802in, y=0.769821in, left, base,rotate=45.000000]{\color{textcolor}{\sffamily\fontsize{11.000000}{13.200000}\selectfont\catcode`\^=\active\def^{\ifmmode\sp\else\^{}\fi}\catcode`\%=\active\def%{\%}gru}}%
\end{pgfscope}%
\begin{pgfscope}%
\pgfsetbuttcap%
\pgfsetroundjoin%
\definecolor{currentfill}{rgb}{0.000000,0.000000,0.000000}%
\pgfsetfillcolor{currentfill}%
\pgfsetlinewidth{0.803000pt}%
\definecolor{currentstroke}{rgb}{0.000000,0.000000,0.000000}%
\pgfsetstrokecolor{currentstroke}%
\pgfsetdash{}{0pt}%
\pgfsys@defobject{currentmarker}{\pgfqpoint{0.000000in}{-0.048611in}}{\pgfqpoint{0.000000in}{0.000000in}}{%
\pgfpathmoveto{\pgfqpoint{0.000000in}{0.000000in}}%
\pgfpathlineto{\pgfqpoint{0.000000in}{-0.048611in}}%
\pgfusepath{stroke,fill}%
}%
\begin{pgfscope}%
\pgfsys@transformshift{2.701769in}{1.130578in}%
\pgfsys@useobject{currentmarker}{}%
\end{pgfscope}%
\end{pgfscope}%
\begin{pgfscope}%
\definecolor{textcolor}{rgb}{0.000000,0.000000,0.000000}%
\pgfsetstrokecolor{textcolor}%
\pgfsetfillcolor{textcolor}%
\pgftext[x=2.647965in, y=0.784064in, left, base,rotate=45.000000]{\color{textcolor}{\sffamily\fontsize{11.000000}{13.200000}\selectfont\catcode`\^=\active\def^{\ifmmode\sp\else\^{}\fi}\catcode`\%=\active\def%{\%}gst}}%
\end{pgfscope}%
\begin{pgfscope}%
\pgfsetbuttcap%
\pgfsetroundjoin%
\definecolor{currentfill}{rgb}{0.000000,0.000000,0.000000}%
\pgfsetfillcolor{currentfill}%
\pgfsetlinewidth{0.803000pt}%
\definecolor{currentstroke}{rgb}{0.000000,0.000000,0.000000}%
\pgfsetstrokecolor{currentstroke}%
\pgfsetdash{}{0pt}%
\pgfsys@defobject{currentmarker}{\pgfqpoint{0.000000in}{-0.048611in}}{\pgfqpoint{0.000000in}{0.000000in}}{%
\pgfpathmoveto{\pgfqpoint{0.000000in}{0.000000in}}%
\pgfpathlineto{\pgfqpoint{0.000000in}{-0.048611in}}%
\pgfusepath{stroke,fill}%
}%
\begin{pgfscope}%
\pgfsys@transformshift{3.046811in}{1.130578in}%
\pgfsys@useobject{currentmarker}{}%
\end{pgfscope}%
\end{pgfscope}%
\begin{pgfscope}%
\definecolor{textcolor}{rgb}{0.000000,0.000000,0.000000}%
\pgfsetstrokecolor{textcolor}%
\pgfsetfillcolor{textcolor}%
\pgftext[x=2.965710in, y=0.729468in, left, base,rotate=45.000000]{\color{textcolor}{\sffamily\fontsize{11.000000}{13.200000}\selectfont\catcode`\^=\active\def^{\ifmmode\sp\else\^{}\fi}\catcode`\%=\active\def%{\%}bert}}%
\end{pgfscope}%
\begin{pgfscope}%
\pgfsetbuttcap%
\pgfsetroundjoin%
\definecolor{currentfill}{rgb}{0.000000,0.000000,0.000000}%
\pgfsetfillcolor{currentfill}%
\pgfsetlinewidth{0.803000pt}%
\definecolor{currentstroke}{rgb}{0.000000,0.000000,0.000000}%
\pgfsetstrokecolor{currentstroke}%
\pgfsetdash{}{0pt}%
\pgfsys@defobject{currentmarker}{\pgfqpoint{0.000000in}{-0.048611in}}{\pgfqpoint{0.000000in}{0.000000in}}{%
\pgfpathmoveto{\pgfqpoint{0.000000in}{0.000000in}}%
\pgfpathlineto{\pgfqpoint{0.000000in}{-0.048611in}}%
\pgfusepath{stroke,fill}%
}%
\begin{pgfscope}%
\pgfsys@transformshift{3.391853in}{1.130578in}%
\pgfsys@useobject{currentmarker}{}%
\end{pgfscope}%
\end{pgfscope}%
\begin{pgfscope}%
\definecolor{textcolor}{rgb}{0.000000,0.000000,0.000000}%
\pgfsetstrokecolor{textcolor}%
\pgfsetfillcolor{textcolor}%
\pgftext[x=3.338049in, y=0.784064in, left, base,rotate=45.000000]{\color{textcolor}{\sffamily\fontsize{11.000000}{13.200000}\selectfont\catcode`\^=\active\def^{\ifmmode\sp\else\^{}\fi}\catcode`\%=\active\def%{\%}spt}}%
\end{pgfscope}%
\begin{pgfscope}%
\pgfsetbuttcap%
\pgfsetroundjoin%
\definecolor{currentfill}{rgb}{0.000000,0.000000,0.000000}%
\pgfsetfillcolor{currentfill}%
\pgfsetlinewidth{0.803000pt}%
\definecolor{currentstroke}{rgb}{0.000000,0.000000,0.000000}%
\pgfsetstrokecolor{currentstroke}%
\pgfsetdash{}{0pt}%
\pgfsys@defobject{currentmarker}{\pgfqpoint{0.000000in}{-0.048611in}}{\pgfqpoint{0.000000in}{0.000000in}}{%
\pgfpathmoveto{\pgfqpoint{0.000000in}{0.000000in}}%
\pgfpathlineto{\pgfqpoint{0.000000in}{-0.048611in}}%
\pgfusepath{stroke,fill}%
}%
\begin{pgfscope}%
\pgfsys@transformshift{3.736895in}{1.130578in}%
\pgfsys@useobject{currentmarker}{}%
\end{pgfscope}%
\end{pgfscope}%
\begin{pgfscope}%
\definecolor{textcolor}{rgb}{0.000000,0.000000,0.000000}%
\pgfsetstrokecolor{textcolor}%
\pgfsetfillcolor{textcolor}%
\pgftext[x=3.384795in, y=0.187471in, left, base,rotate=45.000000]{\color{textcolor}{\sffamily\fontsize{11.000000}{13.200000}\selectfont\catcode`\^=\active\def^{\ifmmode\sp\else\^{}\fi}\catcode`\%=\active\def%{\%}ssd-mobilenet}}%
\end{pgfscope}%
\begin{pgfscope}%
\pgfsetbuttcap%
\pgfsetroundjoin%
\definecolor{currentfill}{rgb}{0.000000,0.000000,0.000000}%
\pgfsetfillcolor{currentfill}%
\pgfsetlinewidth{0.803000pt}%
\definecolor{currentstroke}{rgb}{0.000000,0.000000,0.000000}%
\pgfsetstrokecolor{currentstroke}%
\pgfsetdash{}{0pt}%
\pgfsys@defobject{currentmarker}{\pgfqpoint{0.000000in}{-0.048611in}}{\pgfqpoint{0.000000in}{0.000000in}}{%
\pgfpathmoveto{\pgfqpoint{0.000000in}{0.000000in}}%
\pgfpathlineto{\pgfqpoint{0.000000in}{-0.048611in}}%
\pgfusepath{stroke,fill}%
}%
\begin{pgfscope}%
\pgfsys@transformshift{4.081938in}{1.130578in}%
\pgfsys@useobject{currentmarker}{}%
\end{pgfscope}%
\end{pgfscope}%
\begin{pgfscope}%
\definecolor{textcolor}{rgb}{0.000000,0.000000,0.000000}%
\pgfsetstrokecolor{textcolor}%
\pgfsetfillcolor{textcolor}%
\pgftext[x=3.761909in, y=0.251614in, left, base,rotate=45.000000]{\color{textcolor}{\sffamily\fontsize{11.000000}{13.200000}\selectfont\catcode`\^=\active\def^{\ifmmode\sp\else\^{}\fi}\catcode`\%=\active\def%{\%}ssd-resnet34}}%
\end{pgfscope}%
\begin{pgfscope}%
\pgfsetbuttcap%
\pgfsetroundjoin%
\definecolor{currentfill}{rgb}{0.000000,0.000000,0.000000}%
\pgfsetfillcolor{currentfill}%
\pgfsetlinewidth{0.803000pt}%
\definecolor{currentstroke}{rgb}{0.000000,0.000000,0.000000}%
\pgfsetstrokecolor{currentstroke}%
\pgfsetdash{}{0pt}%
\pgfsys@defobject{currentmarker}{\pgfqpoint{0.000000in}{-0.048611in}}{\pgfqpoint{0.000000in}{0.000000in}}{%
\pgfpathmoveto{\pgfqpoint{0.000000in}{0.000000in}}%
\pgfpathlineto{\pgfqpoint{0.000000in}{-0.048611in}}%
\pgfusepath{stroke,fill}%
}%
\begin{pgfscope}%
\pgfsys@transformshift{4.426980in}{1.130578in}%
\pgfsys@useobject{currentmarker}{}%
\end{pgfscope}%
\end{pgfscope}%
\begin{pgfscope}%
\definecolor{textcolor}{rgb}{0.000000,0.000000,0.000000}%
\pgfsetstrokecolor{textcolor}%
\pgfsetfillcolor{textcolor}%
\pgftext[x=4.359461in, y=0.756634in, left, base,rotate=45.000000]{\color{textcolor}{\sffamily\fontsize{11.000000}{13.200000}\selectfont\catcode`\^=\active\def^{\ifmmode\sp\else\^{}\fi}\catcode`\%=\active\def%{\%}lmc}}%
\end{pgfscope}%
\begin{pgfscope}%
\pgfsetbuttcap%
\pgfsetroundjoin%
\definecolor{currentfill}{rgb}{0.000000,0.000000,0.000000}%
\pgfsetfillcolor{currentfill}%
\pgfsetlinewidth{0.803000pt}%
\definecolor{currentstroke}{rgb}{0.000000,0.000000,0.000000}%
\pgfsetstrokecolor{currentstroke}%
\pgfsetdash{}{0pt}%
\pgfsys@defobject{currentmarker}{\pgfqpoint{0.000000in}{-0.048611in}}{\pgfqpoint{0.000000in}{0.000000in}}{%
\pgfpathmoveto{\pgfqpoint{0.000000in}{0.000000in}}%
\pgfpathlineto{\pgfqpoint{0.000000in}{-0.048611in}}%
\pgfusepath{stroke,fill}%
}%
\begin{pgfscope}%
\pgfsys@transformshift{4.772022in}{1.130578in}%
\pgfsys@useobject{currentmarker}{}%
\end{pgfscope}%
\end{pgfscope}%
\begin{pgfscope}%
\definecolor{textcolor}{rgb}{0.000000,0.000000,0.000000}%
\pgfsetstrokecolor{textcolor}%
\pgfsetfillcolor{textcolor}%
\pgftext[x=4.745594in, y=0.838817in, left, base,rotate=45.000000]{\color{textcolor}{\sffamily\fontsize{11.000000}{13.200000}\selectfont\catcode`\^=\active\def^{\ifmmode\sp\else\^{}\fi}\catcode`\%=\active\def%{\%}rfl}}%
\end{pgfscope}%
\begin{pgfscope}%
\pgfsetbuttcap%
\pgfsetroundjoin%
\definecolor{currentfill}{rgb}{0.000000,0.000000,0.000000}%
\pgfsetfillcolor{currentfill}%
\pgfsetlinewidth{0.803000pt}%
\definecolor{currentstroke}{rgb}{0.000000,0.000000,0.000000}%
\pgfsetstrokecolor{currentstroke}%
\pgfsetdash{}{0pt}%
\pgfsys@defobject{currentmarker}{\pgfqpoint{0.000000in}{-0.048611in}}{\pgfqpoint{0.000000in}{0.000000in}}{%
\pgfpathmoveto{\pgfqpoint{0.000000in}{0.000000in}}%
\pgfpathlineto{\pgfqpoint{0.000000in}{-0.048611in}}%
\pgfusepath{stroke,fill}%
}%
\begin{pgfscope}%
\pgfsys@transformshift{5.117064in}{1.130578in}%
\pgfsys@useobject{currentmarker}{}%
\end{pgfscope}%
\end{pgfscope}%
\begin{pgfscope}%
\definecolor{textcolor}{rgb}{0.000000,0.000000,0.000000}%
\pgfsetstrokecolor{textcolor}%
\pgfsetfillcolor{textcolor}%
\pgftext[x=4.935844in, y=0.529233in, left, base,rotate=45.000000]{\color{textcolor}{\sffamily\fontsize{11.000000}{13.200000}\selectfont\catcode`\^=\active\def^{\ifmmode\sp\else\^{}\fi}\catcode`\%=\active\def%{\%}3d-unet}}%
\end{pgfscope}%
\begin{pgfscope}%
\pgfsetbuttcap%
\pgfsetroundjoin%
\definecolor{currentfill}{rgb}{0.000000,0.000000,0.000000}%
\pgfsetfillcolor{currentfill}%
\pgfsetlinewidth{0.803000pt}%
\definecolor{currentstroke}{rgb}{0.000000,0.000000,0.000000}%
\pgfsetstrokecolor{currentstroke}%
\pgfsetdash{}{0pt}%
\pgfsys@defobject{currentmarker}{\pgfqpoint{0.000000in}{-0.048611in}}{\pgfqpoint{0.000000in}{0.000000in}}{%
\pgfpathmoveto{\pgfqpoint{0.000000in}{0.000000in}}%
\pgfpathlineto{\pgfqpoint{0.000000in}{-0.048611in}}%
\pgfusepath{stroke,fill}%
}%
\begin{pgfscope}%
\pgfsys@transformshift{5.462106in}{1.130578in}%
\pgfsys@useobject{currentmarker}{}%
\end{pgfscope}%
\end{pgfscope}%
\begin{pgfscope}%
\definecolor{textcolor}{rgb}{0.000000,0.000000,0.000000}%
\pgfsetstrokecolor{textcolor}%
\pgfsetfillcolor{textcolor}%
\pgftext[x=5.376863in, y=0.721187in, left, base,rotate=45.000000]{\color{textcolor}{\sffamily\fontsize{11.000000}{13.200000}\selectfont\catcode`\^=\active\def^{\ifmmode\sp\else\^{}\fi}\catcode`\%=\active\def%{\%}gms}}%
\end{pgfscope}%
\begin{pgfscope}%
\pgfsetbuttcap%
\pgfsetroundjoin%
\definecolor{currentfill}{rgb}{0.000000,0.000000,0.000000}%
\pgfsetfillcolor{currentfill}%
\pgfsetlinewidth{0.803000pt}%
\definecolor{currentstroke}{rgb}{0.000000,0.000000,0.000000}%
\pgfsetstrokecolor{currentstroke}%
\pgfsetdash{}{0pt}%
\pgfsys@defobject{currentmarker}{\pgfqpoint{0.000000in}{-0.048611in}}{\pgfqpoint{0.000000in}{0.000000in}}{%
\pgfpathmoveto{\pgfqpoint{0.000000in}{0.000000in}}%
\pgfpathlineto{\pgfqpoint{0.000000in}{-0.048611in}}%
\pgfusepath{stroke,fill}%
}%
\begin{pgfscope}%
\pgfsys@transformshift{5.807148in}{1.130578in}%
\pgfsys@useobject{currentmarker}{}%
\end{pgfscope}%
\end{pgfscope}%
\begin{pgfscope}%
\definecolor{textcolor}{rgb}{0.000000,0.000000,0.000000}%
\pgfsetstrokecolor{textcolor}%
\pgfsetfillcolor{textcolor}%
\pgftext[x=5.753397in, y=0.784169in, left, base,rotate=45.000000]{\color{textcolor}{\sffamily\fontsize{11.000000}{13.200000}\selectfont\catcode`\^=\active\def^{\ifmmode\sp\else\^{}\fi}\catcode`\%=\active\def%{\%}nst}}%
\end{pgfscope}%
\begin{pgfscope}%
\pgfsetbuttcap%
\pgfsetroundjoin%
\definecolor{currentfill}{rgb}{0.000000,0.000000,0.000000}%
\pgfsetfillcolor{currentfill}%
\pgfsetlinewidth{0.803000pt}%
\definecolor{currentstroke}{rgb}{0.000000,0.000000,0.000000}%
\pgfsetstrokecolor{currentstroke}%
\pgfsetdash{}{0pt}%
\pgfsys@defobject{currentmarker}{\pgfqpoint{-0.048611in}{0.000000in}}{\pgfqpoint{-0.000000in}{0.000000in}}{%
\pgfpathmoveto{\pgfqpoint{-0.000000in}{0.000000in}}%
\pgfpathlineto{\pgfqpoint{-0.048611in}{0.000000in}}%
\pgfusepath{stroke,fill}%
}%
\begin{pgfscope}%
\pgfsys@transformshift{0.617715in}{1.130578in}%
\pgfsys@useobject{currentmarker}{}%
\end{pgfscope}%
\end{pgfscope}%
\begin{pgfscope}%
\definecolor{textcolor}{rgb}{0.000000,0.000000,0.000000}%
\pgfsetstrokecolor{textcolor}%
\pgfsetfillcolor{textcolor}%
\pgftext[x=0.444451in, y=1.072540in, left, base]{\color{textcolor}{\sffamily\fontsize{11.000000}{13.200000}\selectfont\catcode`\^=\active\def^{\ifmmode\sp\else\^{}\fi}\catcode`\%=\active\def%{\%}$\mathdefault{0}$}}%
\end{pgfscope}%
\begin{pgfscope}%
\pgfsetbuttcap%
\pgfsetroundjoin%
\definecolor{currentfill}{rgb}{0.000000,0.000000,0.000000}%
\pgfsetfillcolor{currentfill}%
\pgfsetlinewidth{0.803000pt}%
\definecolor{currentstroke}{rgb}{0.000000,0.000000,0.000000}%
\pgfsetstrokecolor{currentstroke}%
\pgfsetdash{}{0pt}%
\pgfsys@defobject{currentmarker}{\pgfqpoint{-0.048611in}{0.000000in}}{\pgfqpoint{-0.000000in}{0.000000in}}{%
\pgfpathmoveto{\pgfqpoint{-0.000000in}{0.000000in}}%
\pgfpathlineto{\pgfqpoint{-0.048611in}{0.000000in}}%
\pgfusepath{stroke,fill}%
}%
\begin{pgfscope}%
\pgfsys@transformshift{0.617715in}{1.518724in}%
\pgfsys@useobject{currentmarker}{}%
\end{pgfscope}%
\end{pgfscope}%
\begin{pgfscope}%
\definecolor{textcolor}{rgb}{0.000000,0.000000,0.000000}%
\pgfsetstrokecolor{textcolor}%
\pgfsetfillcolor{textcolor}%
\pgftext[x=0.444451in, y=1.460686in, left, base]{\color{textcolor}{\sffamily\fontsize{11.000000}{13.200000}\selectfont\catcode`\^=\active\def^{\ifmmode\sp\else\^{}\fi}\catcode`\%=\active\def%{\%}$\mathdefault{2}$}}%
\end{pgfscope}%
\begin{pgfscope}%
\pgfsetbuttcap%
\pgfsetroundjoin%
\definecolor{currentfill}{rgb}{0.000000,0.000000,0.000000}%
\pgfsetfillcolor{currentfill}%
\pgfsetlinewidth{0.803000pt}%
\definecolor{currentstroke}{rgb}{0.000000,0.000000,0.000000}%
\pgfsetstrokecolor{currentstroke}%
\pgfsetdash{}{0pt}%
\pgfsys@defobject{currentmarker}{\pgfqpoint{-0.048611in}{0.000000in}}{\pgfqpoint{-0.000000in}{0.000000in}}{%
\pgfpathmoveto{\pgfqpoint{-0.000000in}{0.000000in}}%
\pgfpathlineto{\pgfqpoint{-0.048611in}{0.000000in}}%
\pgfusepath{stroke,fill}%
}%
\begin{pgfscope}%
\pgfsys@transformshift{0.617715in}{1.906870in}%
\pgfsys@useobject{currentmarker}{}%
\end{pgfscope}%
\end{pgfscope}%
\begin{pgfscope}%
\definecolor{textcolor}{rgb}{0.000000,0.000000,0.000000}%
\pgfsetstrokecolor{textcolor}%
\pgfsetfillcolor{textcolor}%
\pgftext[x=0.444451in, y=1.848832in, left, base]{\color{textcolor}{\sffamily\fontsize{11.000000}{13.200000}\selectfont\catcode`\^=\active\def^{\ifmmode\sp\else\^{}\fi}\catcode`\%=\active\def%{\%}$\mathdefault{4}$}}%
\end{pgfscope}%
\begin{pgfscope}%
\pgfsetbuttcap%
\pgfsetroundjoin%
\definecolor{currentfill}{rgb}{0.000000,0.000000,0.000000}%
\pgfsetfillcolor{currentfill}%
\pgfsetlinewidth{0.803000pt}%
\definecolor{currentstroke}{rgb}{0.000000,0.000000,0.000000}%
\pgfsetstrokecolor{currentstroke}%
\pgfsetdash{}{0pt}%
\pgfsys@defobject{currentmarker}{\pgfqpoint{-0.048611in}{0.000000in}}{\pgfqpoint{-0.000000in}{0.000000in}}{%
\pgfpathmoveto{\pgfqpoint{-0.000000in}{0.000000in}}%
\pgfpathlineto{\pgfqpoint{-0.048611in}{0.000000in}}%
\pgfusepath{stroke,fill}%
}%
\begin{pgfscope}%
\pgfsys@transformshift{0.617715in}{2.295015in}%
\pgfsys@useobject{currentmarker}{}%
\end{pgfscope}%
\end{pgfscope}%
\begin{pgfscope}%
\definecolor{textcolor}{rgb}{0.000000,0.000000,0.000000}%
\pgfsetstrokecolor{textcolor}%
\pgfsetfillcolor{textcolor}%
\pgftext[x=0.444451in, y=2.236978in, left, base]{\color{textcolor}{\sffamily\fontsize{11.000000}{13.200000}\selectfont\catcode`\^=\active\def^{\ifmmode\sp\else\^{}\fi}\catcode`\%=\active\def%{\%}$\mathdefault{6}$}}%
\end{pgfscope}%
\begin{pgfscope}%
\pgfsetbuttcap%
\pgfsetroundjoin%
\definecolor{currentfill}{rgb}{0.000000,0.000000,0.000000}%
\pgfsetfillcolor{currentfill}%
\pgfsetlinewidth{0.803000pt}%
\definecolor{currentstroke}{rgb}{0.000000,0.000000,0.000000}%
\pgfsetstrokecolor{currentstroke}%
\pgfsetdash{}{0pt}%
\pgfsys@defobject{currentmarker}{\pgfqpoint{-0.048611in}{0.000000in}}{\pgfqpoint{-0.000000in}{0.000000in}}{%
\pgfpathmoveto{\pgfqpoint{-0.000000in}{0.000000in}}%
\pgfpathlineto{\pgfqpoint{-0.048611in}{0.000000in}}%
\pgfusepath{stroke,fill}%
}%
\begin{pgfscope}%
\pgfsys@transformshift{0.617715in}{2.683161in}%
\pgfsys@useobject{currentmarker}{}%
\end{pgfscope}%
\end{pgfscope}%
\begin{pgfscope}%
\definecolor{textcolor}{rgb}{0.000000,0.000000,0.000000}%
\pgfsetstrokecolor{textcolor}%
\pgfsetfillcolor{textcolor}%
\pgftext[x=0.444451in, y=2.625123in, left, base]{\color{textcolor}{\sffamily\fontsize{11.000000}{13.200000}\selectfont\catcode`\^=\active\def^{\ifmmode\sp\else\^{}\fi}\catcode`\%=\active\def%{\%}$\mathdefault{8}$}}%
\end{pgfscope}%
\begin{pgfscope}%
\pgfsetbuttcap%
\pgfsetroundjoin%
\definecolor{currentfill}{rgb}{0.000000,0.000000,0.000000}%
\pgfsetfillcolor{currentfill}%
\pgfsetlinewidth{0.803000pt}%
\definecolor{currentstroke}{rgb}{0.000000,0.000000,0.000000}%
\pgfsetstrokecolor{currentstroke}%
\pgfsetdash{}{0pt}%
\pgfsys@defobject{currentmarker}{\pgfqpoint{-0.048611in}{0.000000in}}{\pgfqpoint{-0.000000in}{0.000000in}}{%
\pgfpathmoveto{\pgfqpoint{-0.000000in}{0.000000in}}%
\pgfpathlineto{\pgfqpoint{-0.048611in}{0.000000in}}%
\pgfusepath{stroke,fill}%
}%
\begin{pgfscope}%
\pgfsys@transformshift{0.617715in}{3.071307in}%
\pgfsys@useobject{currentmarker}{}%
\end{pgfscope}%
\end{pgfscope}%
\begin{pgfscope}%
\definecolor{textcolor}{rgb}{0.000000,0.000000,0.000000}%
\pgfsetstrokecolor{textcolor}%
\pgfsetfillcolor{textcolor}%
\pgftext[x=0.368410in, y=3.013269in, left, base]{\color{textcolor}{\sffamily\fontsize{11.000000}{13.200000}\selectfont\catcode`\^=\active\def^{\ifmmode\sp\else\^{}\fi}\catcode`\%=\active\def%{\%}$\mathdefault{10}$}}%
\end{pgfscope}%
\begin{pgfscope}%
\pgfsetbuttcap%
\pgfsetroundjoin%
\definecolor{currentfill}{rgb}{0.000000,0.000000,0.000000}%
\pgfsetfillcolor{currentfill}%
\pgfsetlinewidth{0.803000pt}%
\definecolor{currentstroke}{rgb}{0.000000,0.000000,0.000000}%
\pgfsetstrokecolor{currentstroke}%
\pgfsetdash{}{0pt}%
\pgfsys@defobject{currentmarker}{\pgfqpoint{-0.048611in}{0.000000in}}{\pgfqpoint{-0.000000in}{0.000000in}}{%
\pgfpathmoveto{\pgfqpoint{-0.000000in}{0.000000in}}%
\pgfpathlineto{\pgfqpoint{-0.048611in}{0.000000in}}%
\pgfusepath{stroke,fill}%
}%
\begin{pgfscope}%
\pgfsys@transformshift{0.617715in}{3.459452in}%
\pgfsys@useobject{currentmarker}{}%
\end{pgfscope}%
\end{pgfscope}%
\begin{pgfscope}%
\definecolor{textcolor}{rgb}{0.000000,0.000000,0.000000}%
\pgfsetstrokecolor{textcolor}%
\pgfsetfillcolor{textcolor}%
\pgftext[x=0.368410in, y=3.401415in, left, base]{\color{textcolor}{\sffamily\fontsize{11.000000}{13.200000}\selectfont\catcode`\^=\active\def^{\ifmmode\sp\else\^{}\fi}\catcode`\%=\active\def%{\%}$\mathdefault{12}$}}%
\end{pgfscope}%
\begin{pgfscope}%
\pgfsetbuttcap%
\pgfsetroundjoin%
\definecolor{currentfill}{rgb}{0.000000,0.000000,0.000000}%
\pgfsetfillcolor{currentfill}%
\pgfsetlinewidth{0.803000pt}%
\definecolor{currentstroke}{rgb}{0.000000,0.000000,0.000000}%
\pgfsetstrokecolor{currentstroke}%
\pgfsetdash{}{0pt}%
\pgfsys@defobject{currentmarker}{\pgfqpoint{-0.048611in}{0.000000in}}{\pgfqpoint{-0.000000in}{0.000000in}}{%
\pgfpathmoveto{\pgfqpoint{-0.000000in}{0.000000in}}%
\pgfpathlineto{\pgfqpoint{-0.048611in}{0.000000in}}%
\pgfusepath{stroke,fill}%
}%
\begin{pgfscope}%
\pgfsys@transformshift{0.617715in}{3.847598in}%
\pgfsys@useobject{currentmarker}{}%
\end{pgfscope}%
\end{pgfscope}%
\begin{pgfscope}%
\definecolor{textcolor}{rgb}{0.000000,0.000000,0.000000}%
\pgfsetstrokecolor{textcolor}%
\pgfsetfillcolor{textcolor}%
\pgftext[x=0.368410in, y=3.789560in, left, base]{\color{textcolor}{\sffamily\fontsize{11.000000}{13.200000}\selectfont\catcode`\^=\active\def^{\ifmmode\sp\else\^{}\fi}\catcode`\%=\active\def%{\%}$\mathdefault{14}$}}%
\end{pgfscope}%
\begin{pgfscope}%
\pgfsetbuttcap%
\pgfsetroundjoin%
\definecolor{currentfill}{rgb}{0.000000,0.000000,0.000000}%
\pgfsetfillcolor{currentfill}%
\pgfsetlinewidth{0.803000pt}%
\definecolor{currentstroke}{rgb}{0.000000,0.000000,0.000000}%
\pgfsetstrokecolor{currentstroke}%
\pgfsetdash{}{0pt}%
\pgfsys@defobject{currentmarker}{\pgfqpoint{-0.048611in}{0.000000in}}{\pgfqpoint{-0.000000in}{0.000000in}}{%
\pgfpathmoveto{\pgfqpoint{-0.000000in}{0.000000in}}%
\pgfpathlineto{\pgfqpoint{-0.048611in}{0.000000in}}%
\pgfusepath{stroke,fill}%
}%
\begin{pgfscope}%
\pgfsys@transformshift{0.617715in}{4.235744in}%
\pgfsys@useobject{currentmarker}{}%
\end{pgfscope}%
\end{pgfscope}%
\begin{pgfscope}%
\definecolor{textcolor}{rgb}{0.000000,0.000000,0.000000}%
\pgfsetstrokecolor{textcolor}%
\pgfsetfillcolor{textcolor}%
\pgftext[x=0.368410in, y=4.177706in, left, base]{\color{textcolor}{\sffamily\fontsize{11.000000}{13.200000}\selectfont\catcode`\^=\active\def^{\ifmmode\sp\else\^{}\fi}\catcode`\%=\active\def%{\%}$\mathdefault{16}$}}%
\end{pgfscope}%
\begin{pgfscope}%
\definecolor{textcolor}{rgb}{0.000000,0.000000,0.000000}%
\pgfsetstrokecolor{textcolor}%
\pgfsetfillcolor{textcolor}%
\pgftext[x=0.312854in,y=2.771477in,,bottom,rotate=90.000000]{\color{textcolor}{\sffamily\fontsize{11.000000}{13.200000}\selectfont\catcode`\^=\active\def^{\ifmmode\sp\else\^{}\fi}\catcode`\%=\active\def%{\%}Affected part of the workload (% insn)}}%
\end{pgfscope}%
\begin{pgfscope}%
\pgfsetrectcap%
\pgfsetmiterjoin%
\pgfsetlinewidth{0.803000pt}%
\definecolor{currentstroke}{rgb}{0.000000,0.000000,0.000000}%
\pgfsetstrokecolor{currentstroke}%
\pgfsetdash{}{0pt}%
\pgfpathmoveto{\pgfqpoint{0.617715in}{1.130578in}}%
\pgfpathlineto{\pgfqpoint{0.617715in}{4.412376in}}%
\pgfusepath{stroke}%
\end{pgfscope}%
\begin{pgfscope}%
\pgfsetrectcap%
\pgfsetmiterjoin%
\pgfsetlinewidth{0.803000pt}%
\definecolor{currentstroke}{rgb}{0.000000,0.000000,0.000000}%
\pgfsetstrokecolor{currentstroke}%
\pgfsetdash{}{0pt}%
\pgfpathmoveto{\pgfqpoint{6.235000in}{1.130578in}}%
\pgfpathlineto{\pgfqpoint{6.235000in}{4.412376in}}%
\pgfusepath{stroke}%
\end{pgfscope}%
\begin{pgfscope}%
\pgfsetrectcap%
\pgfsetmiterjoin%
\pgfsetlinewidth{0.803000pt}%
\definecolor{currentstroke}{rgb}{0.000000,0.000000,0.000000}%
\pgfsetstrokecolor{currentstroke}%
\pgfsetdash{}{0pt}%
\pgfpathmoveto{\pgfqpoint{0.617715in}{1.130578in}}%
\pgfpathlineto{\pgfqpoint{6.235000in}{1.130578in}}%
\pgfusepath{stroke}%
\end{pgfscope}%
\begin{pgfscope}%
\pgfsetrectcap%
\pgfsetmiterjoin%
\pgfsetlinewidth{0.803000pt}%
\definecolor{currentstroke}{rgb}{0.000000,0.000000,0.000000}%
\pgfsetstrokecolor{currentstroke}%
\pgfsetdash{}{0pt}%
\pgfpathmoveto{\pgfqpoint{0.617715in}{4.412376in}}%
\pgfpathlineto{\pgfqpoint{6.235000in}{4.412376in}}%
\pgfusepath{stroke}%
\end{pgfscope}%
\begin{pgfscope}%
\definecolor{textcolor}{rgb}{0.000000,0.000000,0.000000}%
\pgfsetstrokecolor{textcolor}%
\pgfsetfillcolor{textcolor}%
\pgftext[x=3.426358in,y=4.495710in,,base]{\color{textcolor}{\sffamily\fontsize{13.200000}{15.840000}\selectfont\catcode`\^=\active\def^{\ifmmode\sp\else\^{}\fi}\catcode`\%=\active\def%{\%}Effect of flushing on workloads}}%
\end{pgfscope}%
\begin{pgfscope}%
\pgfsetbuttcap%
\pgfsetmiterjoin%
\definecolor{currentfill}{rgb}{1.000000,1.000000,1.000000}%
\pgfsetfillcolor{currentfill}%
\pgfsetfillopacity{0.800000}%
\pgfsetlinewidth{1.003750pt}%
\definecolor{currentstroke}{rgb}{0.800000,0.800000,0.800000}%
\pgfsetstrokecolor{currentstroke}%
\pgfsetstrokeopacity{0.800000}%
\pgfsetdash{}{0pt}%
\pgfpathmoveto{\pgfqpoint{3.885606in}{3.393182in}}%
\pgfpathlineto{\pgfqpoint{6.128056in}{3.393182in}}%
\pgfpathquadraticcurveto{\pgfqpoint{6.158611in}{3.393182in}}{\pgfqpoint{6.158611in}{3.423738in}}%
\pgfpathlineto{\pgfqpoint{6.158611in}{4.305432in}}%
\pgfpathquadraticcurveto{\pgfqpoint{6.158611in}{4.335987in}}{\pgfqpoint{6.128056in}{4.335987in}}%
\pgfpathlineto{\pgfqpoint{3.885606in}{4.335987in}}%
\pgfpathquadraticcurveto{\pgfqpoint{3.855051in}{4.335987in}}{\pgfqpoint{3.855051in}{4.305432in}}%
\pgfpathlineto{\pgfqpoint{3.855051in}{3.423738in}}%
\pgfpathquadraticcurveto{\pgfqpoint{3.855051in}{3.393182in}}{\pgfqpoint{3.885606in}{3.393182in}}%
\pgfpathlineto{\pgfqpoint{3.885606in}{3.393182in}}%
\pgfpathclose%
\pgfusepath{stroke,fill}%
\end{pgfscope}%
\begin{pgfscope}%
\pgfsetbuttcap%
\pgfsetmiterjoin%
\definecolor{currentfill}{rgb}{0.121569,0.466667,0.705882}%
\pgfsetfillcolor{currentfill}%
\pgfsetlinewidth{0.000000pt}%
\definecolor{currentstroke}{rgb}{0.000000,0.000000,0.000000}%
\pgfsetstrokecolor{currentstroke}%
\pgfsetstrokeopacity{0.000000}%
\pgfsetdash{}{0pt}%
\pgfpathmoveto{\pgfqpoint{3.916162in}{4.158801in}}%
\pgfpathlineto{\pgfqpoint{4.221717in}{4.158801in}}%
\pgfpathlineto{\pgfqpoint{4.221717in}{4.265746in}}%
\pgfpathlineto{\pgfqpoint{3.916162in}{4.265746in}}%
\pgfpathlineto{\pgfqpoint{3.916162in}{4.158801in}}%
\pgfpathclose%
\pgfusepath{fill}%
\end{pgfscope}%
\begin{pgfscope}%
\definecolor{textcolor}{rgb}{0.000000,0.000000,0.000000}%
\pgfsetstrokecolor{textcolor}%
\pgfsetfillcolor{textcolor}%
\pgftext[x=4.343939in,y=4.158801in,left,base]{\color{textcolor}{\sffamily\fontsize{11.000000}{13.200000}\selectfont\catcode`\^=\active\def^{\ifmmode\sp\else\^{}\fi}\catcode`\%=\active\def%{\%}$\geq5%$ difference in IPC}}%
\end{pgfscope}%
\begin{pgfscope}%
\pgfsetbuttcap%
\pgfsetmiterjoin%
\definecolor{currentfill}{rgb}{1.000000,0.498039,0.054902}%
\pgfsetfillcolor{currentfill}%
\pgfsetlinewidth{0.000000pt}%
\definecolor{currentstroke}{rgb}{0.000000,0.000000,0.000000}%
\pgfsetstrokecolor{currentstroke}%
\pgfsetstrokeopacity{0.000000}%
\pgfsetdash{}{0pt}%
\pgfpathmoveto{\pgfqpoint{3.916162in}{3.934558in}}%
\pgfpathlineto{\pgfqpoint{4.221717in}{3.934558in}}%
\pgfpathlineto{\pgfqpoint{4.221717in}{4.041503in}}%
\pgfpathlineto{\pgfqpoint{3.916162in}{4.041503in}}%
\pgfpathlineto{\pgfqpoint{3.916162in}{3.934558in}}%
\pgfpathclose%
\pgfusepath{fill}%
\end{pgfscope}%
\begin{pgfscope}%
\definecolor{textcolor}{rgb}{0.000000,0.000000,0.000000}%
\pgfsetstrokecolor{textcolor}%
\pgfsetfillcolor{textcolor}%
\pgftext[x=4.343939in,y=3.934558in,left,base]{\color{textcolor}{\sffamily\fontsize{11.000000}{13.200000}\selectfont\catcode`\^=\active\def^{\ifmmode\sp\else\^{}\fi}\catcode`\%=\active\def%{\%}$\geq10%$ difference in IPC}}%
\end{pgfscope}%
\begin{pgfscope}%
\pgfsetbuttcap%
\pgfsetmiterjoin%
\definecolor{currentfill}{rgb}{0.172549,0.627451,0.172549}%
\pgfsetfillcolor{currentfill}%
\pgfsetlinewidth{0.000000pt}%
\definecolor{currentstroke}{rgb}{0.000000,0.000000,0.000000}%
\pgfsetstrokecolor{currentstroke}%
\pgfsetstrokeopacity{0.000000}%
\pgfsetdash{}{0pt}%
\pgfpathmoveto{\pgfqpoint{3.916162in}{3.710315in}}%
\pgfpathlineto{\pgfqpoint{4.221717in}{3.710315in}}%
\pgfpathlineto{\pgfqpoint{4.221717in}{3.817260in}}%
\pgfpathlineto{\pgfqpoint{3.916162in}{3.817260in}}%
\pgfpathlineto{\pgfqpoint{3.916162in}{3.710315in}}%
\pgfpathclose%
\pgfusepath{fill}%
\end{pgfscope}%
\begin{pgfscope}%
\definecolor{textcolor}{rgb}{0.000000,0.000000,0.000000}%
\pgfsetstrokecolor{textcolor}%
\pgfsetfillcolor{textcolor}%
\pgftext[x=4.343939in,y=3.710315in,left,base]{\color{textcolor}{\sffamily\fontsize{11.000000}{13.200000}\selectfont\catcode`\^=\active\def^{\ifmmode\sp\else\^{}\fi}\catcode`\%=\active\def%{\%}$\geq15%$ difference in IPC}}%
\end{pgfscope}%
\begin{pgfscope}%
\pgfsetbuttcap%
\pgfsetmiterjoin%
\definecolor{currentfill}{rgb}{0.839216,0.152941,0.156863}%
\pgfsetfillcolor{currentfill}%
\pgfsetlinewidth{0.000000pt}%
\definecolor{currentstroke}{rgb}{0.000000,0.000000,0.000000}%
\pgfsetstrokecolor{currentstroke}%
\pgfsetstrokeopacity{0.000000}%
\pgfsetdash{}{0pt}%
\pgfpathmoveto{\pgfqpoint{3.916162in}{3.486072in}}%
\pgfpathlineto{\pgfqpoint{4.221717in}{3.486072in}}%
\pgfpathlineto{\pgfqpoint{4.221717in}{3.593017in}}%
\pgfpathlineto{\pgfqpoint{3.916162in}{3.593017in}}%
\pgfpathlineto{\pgfqpoint{3.916162in}{3.486072in}}%
\pgfpathclose%
\pgfusepath{fill}%
\end{pgfscope}%
\begin{pgfscope}%
\definecolor{textcolor}{rgb}{0.000000,0.000000,0.000000}%
\pgfsetstrokecolor{textcolor}%
\pgfsetfillcolor{textcolor}%
\pgftext[x=4.343939in,y=3.486072in,left,base]{\color{textcolor}{\sffamily\fontsize{11.000000}{13.200000}\selectfont\catcode`\^=\active\def^{\ifmmode\sp\else\^{}\fi}\catcode`\%=\active\def%{\%}$\geq20%$ difference in IPC}}%
\end{pgfscope}%
\end{pgfpicture}%
\makeatother%
\endgroup%
}
        \end{minipage}
        \caption{Weighted IPC differences}
        \label{fig:weight_ipc_diff}
    \end{subfigure}
    \centering
    \begin{subfigure}{\textwidth}
        \begin{minipage}[c]{0.45\textwidth}
            \resizebox{\textwidth}{!}{\begin{tabular}{|c|c|c|c|c|}
\hline
  \textbf{Workload} & \textbf{5\% difference} & \textbf{10\% difference} & \textbf{15\% difference} & \textbf{20\% difference}\\
\hline
\hline
  gold2 & 63.62 \% & 59.17 \% & 58.00 \% & 58.00 \%\\
  cuda2 & 66.46 \% & 58.12 \% & 58.12 \% & 58.12 \%\\
  gold1 & 64.97 \% & 59.17 \% & 58.00 \% & 58.00 \%\\
  cuda1 & 63.52 \% & 58.80 \% & 58.80 \% & 58.80 \%\\
\hline
\end{tabular}
}
        \end{minipage}
        \begin{minipage}[c]{0.45\textwidth}
            \resizebox{\textwidth}{!}{%% Creator: Matplotlib, PGF backend
%%
%% To include the figure in your LaTeX document, write
%%   \input{<filename>.pgf}
%%
%% Make sure the required packages are loaded in your preamble
%%   \usepackage{pgf}
%%
%% Also ensure that all the required font packages are loaded; for instance,
%% the lmodern package is sometimes necessary when using math font.
%%   \usepackage{lmodern}
%%
%% Figures using additional raster images can only be included by \input if
%% they are in the same directory as the main LaTeX file. For loading figures
%% from other directories you can use the `import` package
%%   \usepackage{import}
%%
%% and then include the figures with
%%   \import{<path to file>}{<filename>.pgf}
%%
%% Matplotlib used the following preamble
%%   \def\mathdefault#1{#1}
%%   \everymath=\expandafter{\the\everymath\displaystyle}
%%   
%%   \usepackage{fontspec}
%%   \setmainfont{DejaVuSerif.ttf}[Path=\detokenize{/home/data/ugent/thesis/4Jonas/lib/python3.11/site-packages/matplotlib/mpl-data/fonts/ttf/}]
%%   \setsansfont{DejaVuSans.ttf}[Path=\detokenize{/home/data/ugent/thesis/4Jonas/lib/python3.11/site-packages/matplotlib/mpl-data/fonts/ttf/}]
%%   \setmonofont{DejaVuSansMono.ttf}[Path=\detokenize{/home/data/ugent/thesis/4Jonas/lib/python3.11/site-packages/matplotlib/mpl-data/fonts/ttf/}]
%%   \makeatletter\@ifpackageloaded{underscore}{}{\usepackage[strings]{underscore}}\makeatother
%%
\begingroup%
\makeatletter%
\begin{pgfpicture}%
\pgfpathrectangle{\pgfpointorigin}{\pgfqpoint{6.400000in}{4.800000in}}%
\pgfusepath{use as bounding box, clip}%
\begin{pgfscope}%
\pgfsetbuttcap%
\pgfsetmiterjoin%
\definecolor{currentfill}{rgb}{1.000000,1.000000,1.000000}%
\pgfsetfillcolor{currentfill}%
\pgfsetlinewidth{0.000000pt}%
\definecolor{currentstroke}{rgb}{1.000000,1.000000,1.000000}%
\pgfsetstrokecolor{currentstroke}%
\pgfsetdash{}{0pt}%
\pgfpathmoveto{\pgfqpoint{0.000000in}{0.000000in}}%
\pgfpathlineto{\pgfqpoint{6.400000in}{0.000000in}}%
\pgfpathlineto{\pgfqpoint{6.400000in}{4.800000in}}%
\pgfpathlineto{\pgfqpoint{0.000000in}{4.800000in}}%
\pgfpathlineto{\pgfqpoint{0.000000in}{0.000000in}}%
\pgfpathclose%
\pgfusepath{fill}%
\end{pgfscope}%
\begin{pgfscope}%
\pgfsetbuttcap%
\pgfsetmiterjoin%
\definecolor{currentfill}{rgb}{1.000000,1.000000,1.000000}%
\pgfsetfillcolor{currentfill}%
\pgfsetlinewidth{0.000000pt}%
\definecolor{currentstroke}{rgb}{0.000000,0.000000,0.000000}%
\pgfsetstrokecolor{currentstroke}%
\pgfsetstrokeopacity{0.000000}%
\pgfsetdash{}{0pt}%
\pgfpathmoveto{\pgfqpoint{0.617715in}{0.698141in}}%
\pgfpathlineto{\pgfqpoint{6.235000in}{0.698141in}}%
\pgfpathlineto{\pgfqpoint{6.235000in}{4.412376in}}%
\pgfpathlineto{\pgfqpoint{0.617715in}{4.412376in}}%
\pgfpathlineto{\pgfqpoint{0.617715in}{0.698141in}}%
\pgfpathclose%
\pgfusepath{fill}%
\end{pgfscope}%
\begin{pgfscope}%
\pgfpathrectangle{\pgfqpoint{0.617715in}{0.698141in}}{\pgfqpoint{5.617285in}{3.714236in}}%
\pgfusepath{clip}%
\pgfsetbuttcap%
\pgfsetmiterjoin%
\definecolor{currentfill}{rgb}{0.121569,0.466667,0.705882}%
\pgfsetfillcolor{currentfill}%
\pgfsetlinewidth{0.000000pt}%
\definecolor{currentstroke}{rgb}{0.000000,0.000000,0.000000}%
\pgfsetstrokecolor{currentstroke}%
\pgfsetstrokeopacity{0.000000}%
\pgfsetdash{}{0pt}%
\pgfpathmoveto{\pgfqpoint{0.873046in}{0.698141in}}%
\pgfpathlineto{\pgfqpoint{1.141816in}{0.698141in}}%
\pgfpathlineto{\pgfqpoint{1.141816in}{4.084741in}}%
\pgfpathlineto{\pgfqpoint{0.873046in}{4.084741in}}%
\pgfpathlineto{\pgfqpoint{0.873046in}{0.698141in}}%
\pgfpathclose%
\pgfusepath{fill}%
\end{pgfscope}%
\begin{pgfscope}%
\pgfpathrectangle{\pgfqpoint{0.617715in}{0.698141in}}{\pgfqpoint{5.617285in}{3.714236in}}%
\pgfusepath{clip}%
\pgfsetbuttcap%
\pgfsetmiterjoin%
\definecolor{currentfill}{rgb}{0.121569,0.466667,0.705882}%
\pgfsetfillcolor{currentfill}%
\pgfsetlinewidth{0.000000pt}%
\definecolor{currentstroke}{rgb}{0.000000,0.000000,0.000000}%
\pgfsetstrokecolor{currentstroke}%
\pgfsetstrokeopacity{0.000000}%
\pgfsetdash{}{0pt}%
\pgfpathmoveto{\pgfqpoint{2.216894in}{0.698141in}}%
\pgfpathlineto{\pgfqpoint{2.485664in}{0.698141in}}%
\pgfpathlineto{\pgfqpoint{2.485664in}{4.235508in}}%
\pgfpathlineto{\pgfqpoint{2.216894in}{4.235508in}}%
\pgfpathlineto{\pgfqpoint{2.216894in}{0.698141in}}%
\pgfpathclose%
\pgfusepath{fill}%
\end{pgfscope}%
\begin{pgfscope}%
\pgfpathrectangle{\pgfqpoint{0.617715in}{0.698141in}}{\pgfqpoint{5.617285in}{3.714236in}}%
\pgfusepath{clip}%
\pgfsetbuttcap%
\pgfsetmiterjoin%
\definecolor{currentfill}{rgb}{0.121569,0.466667,0.705882}%
\pgfsetfillcolor{currentfill}%
\pgfsetlinewidth{0.000000pt}%
\definecolor{currentstroke}{rgb}{0.000000,0.000000,0.000000}%
\pgfsetstrokecolor{currentstroke}%
\pgfsetstrokeopacity{0.000000}%
\pgfsetdash{}{0pt}%
\pgfpathmoveto{\pgfqpoint{3.560742in}{0.698141in}}%
\pgfpathlineto{\pgfqpoint{3.829512in}{0.698141in}}%
\pgfpathlineto{\pgfqpoint{3.829512in}{4.156517in}}%
\pgfpathlineto{\pgfqpoint{3.560742in}{4.156517in}}%
\pgfpathlineto{\pgfqpoint{3.560742in}{0.698141in}}%
\pgfpathclose%
\pgfusepath{fill}%
\end{pgfscope}%
\begin{pgfscope}%
\pgfpathrectangle{\pgfqpoint{0.617715in}{0.698141in}}{\pgfqpoint{5.617285in}{3.714236in}}%
\pgfusepath{clip}%
\pgfsetbuttcap%
\pgfsetmiterjoin%
\definecolor{currentfill}{rgb}{0.121569,0.466667,0.705882}%
\pgfsetfillcolor{currentfill}%
\pgfsetlinewidth{0.000000pt}%
\definecolor{currentstroke}{rgb}{0.000000,0.000000,0.000000}%
\pgfsetstrokecolor{currentstroke}%
\pgfsetstrokeopacity{0.000000}%
\pgfsetdash{}{0pt}%
\pgfpathmoveto{\pgfqpoint{4.904590in}{0.698141in}}%
\pgfpathlineto{\pgfqpoint{5.173360in}{0.698141in}}%
\pgfpathlineto{\pgfqpoint{5.173360in}{4.079382in}}%
\pgfpathlineto{\pgfqpoint{4.904590in}{4.079382in}}%
\pgfpathlineto{\pgfqpoint{4.904590in}{0.698141in}}%
\pgfpathclose%
\pgfusepath{fill}%
\end{pgfscope}%
\begin{pgfscope}%
\pgfpathrectangle{\pgfqpoint{0.617715in}{0.698141in}}{\pgfqpoint{5.617285in}{3.714236in}}%
\pgfusepath{clip}%
\pgfsetbuttcap%
\pgfsetmiterjoin%
\definecolor{currentfill}{rgb}{1.000000,0.498039,0.054902}%
\pgfsetfillcolor{currentfill}%
\pgfsetlinewidth{0.000000pt}%
\definecolor{currentstroke}{rgb}{0.000000,0.000000,0.000000}%
\pgfsetstrokecolor{currentstroke}%
\pgfsetstrokeopacity{0.000000}%
\pgfsetdash{}{0pt}%
\pgfpathmoveto{\pgfqpoint{1.141816in}{0.698141in}}%
\pgfpathlineto{\pgfqpoint{1.410586in}{0.698141in}}%
\pgfpathlineto{\pgfqpoint{1.410586in}{3.847409in}}%
\pgfpathlineto{\pgfqpoint{1.141816in}{3.847409in}}%
\pgfpathlineto{\pgfqpoint{1.141816in}{0.698141in}}%
\pgfpathclose%
\pgfusepath{fill}%
\end{pgfscope}%
\begin{pgfscope}%
\pgfpathrectangle{\pgfqpoint{0.617715in}{0.698141in}}{\pgfqpoint{5.617285in}{3.714236in}}%
\pgfusepath{clip}%
\pgfsetbuttcap%
\pgfsetmiterjoin%
\definecolor{currentfill}{rgb}{1.000000,0.498039,0.054902}%
\pgfsetfillcolor{currentfill}%
\pgfsetlinewidth{0.000000pt}%
\definecolor{currentstroke}{rgb}{0.000000,0.000000,0.000000}%
\pgfsetstrokecolor{currentstroke}%
\pgfsetstrokeopacity{0.000000}%
\pgfsetdash{}{0pt}%
\pgfpathmoveto{\pgfqpoint{2.485664in}{0.698141in}}%
\pgfpathlineto{\pgfqpoint{2.754434in}{0.698141in}}%
\pgfpathlineto{\pgfqpoint{2.754434in}{3.791736in}}%
\pgfpathlineto{\pgfqpoint{2.485664in}{3.791736in}}%
\pgfpathlineto{\pgfqpoint{2.485664in}{0.698141in}}%
\pgfpathclose%
\pgfusepath{fill}%
\end{pgfscope}%
\begin{pgfscope}%
\pgfpathrectangle{\pgfqpoint{0.617715in}{0.698141in}}{\pgfqpoint{5.617285in}{3.714236in}}%
\pgfusepath{clip}%
\pgfsetbuttcap%
\pgfsetmiterjoin%
\definecolor{currentfill}{rgb}{1.000000,0.498039,0.054902}%
\pgfsetfillcolor{currentfill}%
\pgfsetlinewidth{0.000000pt}%
\definecolor{currentstroke}{rgb}{0.000000,0.000000,0.000000}%
\pgfsetstrokecolor{currentstroke}%
\pgfsetstrokeopacity{0.000000}%
\pgfsetdash{}{0pt}%
\pgfpathmoveto{\pgfqpoint{3.829512in}{0.698141in}}%
\pgfpathlineto{\pgfqpoint{4.098282in}{0.698141in}}%
\pgfpathlineto{\pgfqpoint{4.098282in}{3.847409in}}%
\pgfpathlineto{\pgfqpoint{3.829512in}{3.847409in}}%
\pgfpathlineto{\pgfqpoint{3.829512in}{0.698141in}}%
\pgfpathclose%
\pgfusepath{fill}%
\end{pgfscope}%
\begin{pgfscope}%
\pgfpathrectangle{\pgfqpoint{0.617715in}{0.698141in}}{\pgfqpoint{5.617285in}{3.714236in}}%
\pgfusepath{clip}%
\pgfsetbuttcap%
\pgfsetmiterjoin%
\definecolor{currentfill}{rgb}{1.000000,0.498039,0.054902}%
\pgfsetfillcolor{currentfill}%
\pgfsetlinewidth{0.000000pt}%
\definecolor{currentstroke}{rgb}{0.000000,0.000000,0.000000}%
\pgfsetstrokecolor{currentstroke}%
\pgfsetstrokeopacity{0.000000}%
\pgfsetdash{}{0pt}%
\pgfpathmoveto{\pgfqpoint{5.173360in}{0.698141in}}%
\pgfpathlineto{\pgfqpoint{5.442130in}{0.698141in}}%
\pgfpathlineto{\pgfqpoint{5.442130in}{3.827995in}}%
\pgfpathlineto{\pgfqpoint{5.173360in}{3.827995in}}%
\pgfpathlineto{\pgfqpoint{5.173360in}{0.698141in}}%
\pgfpathclose%
\pgfusepath{fill}%
\end{pgfscope}%
\begin{pgfscope}%
\pgfpathrectangle{\pgfqpoint{0.617715in}{0.698141in}}{\pgfqpoint{5.617285in}{3.714236in}}%
\pgfusepath{clip}%
\pgfsetbuttcap%
\pgfsetmiterjoin%
\definecolor{currentfill}{rgb}{0.172549,0.627451,0.172549}%
\pgfsetfillcolor{currentfill}%
\pgfsetlinewidth{0.000000pt}%
\definecolor{currentstroke}{rgb}{0.000000,0.000000,0.000000}%
\pgfsetstrokecolor{currentstroke}%
\pgfsetstrokeopacity{0.000000}%
\pgfsetdash{}{0pt}%
\pgfpathmoveto{\pgfqpoint{1.410586in}{0.698141in}}%
\pgfpathlineto{\pgfqpoint{1.679355in}{0.698141in}}%
\pgfpathlineto{\pgfqpoint{1.679355in}{3.785262in}}%
\pgfpathlineto{\pgfqpoint{1.410586in}{3.785262in}}%
\pgfpathlineto{\pgfqpoint{1.410586in}{0.698141in}}%
\pgfpathclose%
\pgfusepath{fill}%
\end{pgfscope}%
\begin{pgfscope}%
\pgfpathrectangle{\pgfqpoint{0.617715in}{0.698141in}}{\pgfqpoint{5.617285in}{3.714236in}}%
\pgfusepath{clip}%
\pgfsetbuttcap%
\pgfsetmiterjoin%
\definecolor{currentfill}{rgb}{0.172549,0.627451,0.172549}%
\pgfsetfillcolor{currentfill}%
\pgfsetlinewidth{0.000000pt}%
\definecolor{currentstroke}{rgb}{0.000000,0.000000,0.000000}%
\pgfsetstrokecolor{currentstroke}%
\pgfsetstrokeopacity{0.000000}%
\pgfsetdash{}{0pt}%
\pgfpathmoveto{\pgfqpoint{2.754434in}{0.698141in}}%
\pgfpathlineto{\pgfqpoint{3.023203in}{0.698141in}}%
\pgfpathlineto{\pgfqpoint{3.023203in}{3.791736in}}%
\pgfpathlineto{\pgfqpoint{2.754434in}{3.791736in}}%
\pgfpathlineto{\pgfqpoint{2.754434in}{0.698141in}}%
\pgfpathclose%
\pgfusepath{fill}%
\end{pgfscope}%
\begin{pgfscope}%
\pgfpathrectangle{\pgfqpoint{0.617715in}{0.698141in}}{\pgfqpoint{5.617285in}{3.714236in}}%
\pgfusepath{clip}%
\pgfsetbuttcap%
\pgfsetmiterjoin%
\definecolor{currentfill}{rgb}{0.172549,0.627451,0.172549}%
\pgfsetfillcolor{currentfill}%
\pgfsetlinewidth{0.000000pt}%
\definecolor{currentstroke}{rgb}{0.000000,0.000000,0.000000}%
\pgfsetstrokecolor{currentstroke}%
\pgfsetstrokeopacity{0.000000}%
\pgfsetdash{}{0pt}%
\pgfpathmoveto{\pgfqpoint{4.098282in}{0.698141in}}%
\pgfpathlineto{\pgfqpoint{4.367051in}{0.698141in}}%
\pgfpathlineto{\pgfqpoint{4.367051in}{3.785262in}}%
\pgfpathlineto{\pgfqpoint{4.098282in}{3.785262in}}%
\pgfpathlineto{\pgfqpoint{4.098282in}{0.698141in}}%
\pgfpathclose%
\pgfusepath{fill}%
\end{pgfscope}%
\begin{pgfscope}%
\pgfpathrectangle{\pgfqpoint{0.617715in}{0.698141in}}{\pgfqpoint{5.617285in}{3.714236in}}%
\pgfusepath{clip}%
\pgfsetbuttcap%
\pgfsetmiterjoin%
\definecolor{currentfill}{rgb}{0.172549,0.627451,0.172549}%
\pgfsetfillcolor{currentfill}%
\pgfsetlinewidth{0.000000pt}%
\definecolor{currentstroke}{rgb}{0.000000,0.000000,0.000000}%
\pgfsetstrokecolor{currentstroke}%
\pgfsetstrokeopacity{0.000000}%
\pgfsetdash{}{0pt}%
\pgfpathmoveto{\pgfqpoint{5.442130in}{0.698141in}}%
\pgfpathlineto{\pgfqpoint{5.710899in}{0.698141in}}%
\pgfpathlineto{\pgfqpoint{5.710899in}{3.827995in}}%
\pgfpathlineto{\pgfqpoint{5.442130in}{3.827995in}}%
\pgfpathlineto{\pgfqpoint{5.442130in}{0.698141in}}%
\pgfpathclose%
\pgfusepath{fill}%
\end{pgfscope}%
\begin{pgfscope}%
\pgfpathrectangle{\pgfqpoint{0.617715in}{0.698141in}}{\pgfqpoint{5.617285in}{3.714236in}}%
\pgfusepath{clip}%
\pgfsetbuttcap%
\pgfsetmiterjoin%
\definecolor{currentfill}{rgb}{0.839216,0.152941,0.156863}%
\pgfsetfillcolor{currentfill}%
\pgfsetlinewidth{0.000000pt}%
\definecolor{currentstroke}{rgb}{0.000000,0.000000,0.000000}%
\pgfsetstrokecolor{currentstroke}%
\pgfsetstrokeopacity{0.000000}%
\pgfsetdash{}{0pt}%
\pgfpathmoveto{\pgfqpoint{1.679355in}{0.698141in}}%
\pgfpathlineto{\pgfqpoint{1.948125in}{0.698141in}}%
\pgfpathlineto{\pgfqpoint{1.948125in}{3.785262in}}%
\pgfpathlineto{\pgfqpoint{1.679355in}{3.785262in}}%
\pgfpathlineto{\pgfqpoint{1.679355in}{0.698141in}}%
\pgfpathclose%
\pgfusepath{fill}%
\end{pgfscope}%
\begin{pgfscope}%
\pgfpathrectangle{\pgfqpoint{0.617715in}{0.698141in}}{\pgfqpoint{5.617285in}{3.714236in}}%
\pgfusepath{clip}%
\pgfsetbuttcap%
\pgfsetmiterjoin%
\definecolor{currentfill}{rgb}{0.839216,0.152941,0.156863}%
\pgfsetfillcolor{currentfill}%
\pgfsetlinewidth{0.000000pt}%
\definecolor{currentstroke}{rgb}{0.000000,0.000000,0.000000}%
\pgfsetstrokecolor{currentstroke}%
\pgfsetstrokeopacity{0.000000}%
\pgfsetdash{}{0pt}%
\pgfpathmoveto{\pgfqpoint{3.023203in}{0.698141in}}%
\pgfpathlineto{\pgfqpoint{3.291973in}{0.698141in}}%
\pgfpathlineto{\pgfqpoint{3.291973in}{3.791736in}}%
\pgfpathlineto{\pgfqpoint{3.023203in}{3.791736in}}%
\pgfpathlineto{\pgfqpoint{3.023203in}{0.698141in}}%
\pgfpathclose%
\pgfusepath{fill}%
\end{pgfscope}%
\begin{pgfscope}%
\pgfpathrectangle{\pgfqpoint{0.617715in}{0.698141in}}{\pgfqpoint{5.617285in}{3.714236in}}%
\pgfusepath{clip}%
\pgfsetbuttcap%
\pgfsetmiterjoin%
\definecolor{currentfill}{rgb}{0.839216,0.152941,0.156863}%
\pgfsetfillcolor{currentfill}%
\pgfsetlinewidth{0.000000pt}%
\definecolor{currentstroke}{rgb}{0.000000,0.000000,0.000000}%
\pgfsetstrokecolor{currentstroke}%
\pgfsetstrokeopacity{0.000000}%
\pgfsetdash{}{0pt}%
\pgfpathmoveto{\pgfqpoint{4.367051in}{0.698141in}}%
\pgfpathlineto{\pgfqpoint{4.635821in}{0.698141in}}%
\pgfpathlineto{\pgfqpoint{4.635821in}{3.785262in}}%
\pgfpathlineto{\pgfqpoint{4.367051in}{3.785262in}}%
\pgfpathlineto{\pgfqpoint{4.367051in}{0.698141in}}%
\pgfpathclose%
\pgfusepath{fill}%
\end{pgfscope}%
\begin{pgfscope}%
\pgfpathrectangle{\pgfqpoint{0.617715in}{0.698141in}}{\pgfqpoint{5.617285in}{3.714236in}}%
\pgfusepath{clip}%
\pgfsetbuttcap%
\pgfsetmiterjoin%
\definecolor{currentfill}{rgb}{0.839216,0.152941,0.156863}%
\pgfsetfillcolor{currentfill}%
\pgfsetlinewidth{0.000000pt}%
\definecolor{currentstroke}{rgb}{0.000000,0.000000,0.000000}%
\pgfsetstrokecolor{currentstroke}%
\pgfsetstrokeopacity{0.000000}%
\pgfsetdash{}{0pt}%
\pgfpathmoveto{\pgfqpoint{5.710899in}{0.698141in}}%
\pgfpathlineto{\pgfqpoint{5.979669in}{0.698141in}}%
\pgfpathlineto{\pgfqpoint{5.979669in}{3.827995in}}%
\pgfpathlineto{\pgfqpoint{5.710899in}{3.827995in}}%
\pgfpathlineto{\pgfqpoint{5.710899in}{0.698141in}}%
\pgfpathclose%
\pgfusepath{fill}%
\end{pgfscope}%
\begin{pgfscope}%
\pgfsetbuttcap%
\pgfsetroundjoin%
\definecolor{currentfill}{rgb}{0.000000,0.000000,0.000000}%
\pgfsetfillcolor{currentfill}%
\pgfsetlinewidth{0.803000pt}%
\definecolor{currentstroke}{rgb}{0.000000,0.000000,0.000000}%
\pgfsetstrokecolor{currentstroke}%
\pgfsetdash{}{0pt}%
\pgfsys@defobject{currentmarker}{\pgfqpoint{0.000000in}{-0.048611in}}{\pgfqpoint{0.000000in}{0.000000in}}{%
\pgfpathmoveto{\pgfqpoint{0.000000in}{0.000000in}}%
\pgfpathlineto{\pgfqpoint{0.000000in}{-0.048611in}}%
\pgfusepath{stroke,fill}%
}%
\begin{pgfscope}%
\pgfsys@transformshift{1.276201in}{0.698141in}%
\pgfsys@useobject{currentmarker}{}%
\end{pgfscope}%
\end{pgfscope}%
\begin{pgfscope}%
\definecolor{textcolor}{rgb}{0.000000,0.000000,0.000000}%
\pgfsetstrokecolor{textcolor}%
\pgfsetfillcolor{textcolor}%
\pgftext[x=1.155010in, y=0.216852in, left, base,rotate=45.000000]{\color{textcolor}{\sffamily\fontsize{11.000000}{13.200000}\selectfont\catcode`\^=\active\def^{\ifmmode\sp\else\^{}\fi}\catcode`\%=\active\def%{\%}gold2}}%
\end{pgfscope}%
\begin{pgfscope}%
\pgfsetbuttcap%
\pgfsetroundjoin%
\definecolor{currentfill}{rgb}{0.000000,0.000000,0.000000}%
\pgfsetfillcolor{currentfill}%
\pgfsetlinewidth{0.803000pt}%
\definecolor{currentstroke}{rgb}{0.000000,0.000000,0.000000}%
\pgfsetstrokecolor{currentstroke}%
\pgfsetdash{}{0pt}%
\pgfsys@defobject{currentmarker}{\pgfqpoint{0.000000in}{-0.048611in}}{\pgfqpoint{0.000000in}{0.000000in}}{%
\pgfpathmoveto{\pgfqpoint{0.000000in}{0.000000in}}%
\pgfpathlineto{\pgfqpoint{0.000000in}{-0.048611in}}%
\pgfusepath{stroke,fill}%
}%
\begin{pgfscope}%
\pgfsys@transformshift{2.620049in}{0.698141in}%
\pgfsys@useobject{currentmarker}{}%
\end{pgfscope}%
\end{pgfscope}%
\begin{pgfscope}%
\definecolor{textcolor}{rgb}{0.000000,0.000000,0.000000}%
\pgfsetstrokecolor{textcolor}%
\pgfsetfillcolor{textcolor}%
\pgftext[x=2.484167in, y=0.187471in, left, base,rotate=45.000000]{\color{textcolor}{\sffamily\fontsize{11.000000}{13.200000}\selectfont\catcode`\^=\active\def^{\ifmmode\sp\else\^{}\fi}\catcode`\%=\active\def%{\%}cuda2}}%
\end{pgfscope}%
\begin{pgfscope}%
\pgfsetbuttcap%
\pgfsetroundjoin%
\definecolor{currentfill}{rgb}{0.000000,0.000000,0.000000}%
\pgfsetfillcolor{currentfill}%
\pgfsetlinewidth{0.803000pt}%
\definecolor{currentstroke}{rgb}{0.000000,0.000000,0.000000}%
\pgfsetstrokecolor{currentstroke}%
\pgfsetdash{}{0pt}%
\pgfsys@defobject{currentmarker}{\pgfqpoint{0.000000in}{-0.048611in}}{\pgfqpoint{0.000000in}{0.000000in}}{%
\pgfpathmoveto{\pgfqpoint{0.000000in}{0.000000in}}%
\pgfpathlineto{\pgfqpoint{0.000000in}{-0.048611in}}%
\pgfusepath{stroke,fill}%
}%
\begin{pgfscope}%
\pgfsys@transformshift{3.963897in}{0.698141in}%
\pgfsys@useobject{currentmarker}{}%
\end{pgfscope}%
\end{pgfscope}%
\begin{pgfscope}%
\definecolor{textcolor}{rgb}{0.000000,0.000000,0.000000}%
\pgfsetstrokecolor{textcolor}%
\pgfsetfillcolor{textcolor}%
\pgftext[x=3.842706in, y=0.216852in, left, base,rotate=45.000000]{\color{textcolor}{\sffamily\fontsize{11.000000}{13.200000}\selectfont\catcode`\^=\active\def^{\ifmmode\sp\else\^{}\fi}\catcode`\%=\active\def%{\%}gold1}}%
\end{pgfscope}%
\begin{pgfscope}%
\pgfsetbuttcap%
\pgfsetroundjoin%
\definecolor{currentfill}{rgb}{0.000000,0.000000,0.000000}%
\pgfsetfillcolor{currentfill}%
\pgfsetlinewidth{0.803000pt}%
\definecolor{currentstroke}{rgb}{0.000000,0.000000,0.000000}%
\pgfsetstrokecolor{currentstroke}%
\pgfsetdash{}{0pt}%
\pgfsys@defobject{currentmarker}{\pgfqpoint{0.000000in}{-0.048611in}}{\pgfqpoint{0.000000in}{0.000000in}}{%
\pgfpathmoveto{\pgfqpoint{0.000000in}{0.000000in}}%
\pgfpathlineto{\pgfqpoint{0.000000in}{-0.048611in}}%
\pgfusepath{stroke,fill}%
}%
\begin{pgfscope}%
\pgfsys@transformshift{5.307745in}{0.698141in}%
\pgfsys@useobject{currentmarker}{}%
\end{pgfscope}%
\end{pgfscope}%
\begin{pgfscope}%
\definecolor{textcolor}{rgb}{0.000000,0.000000,0.000000}%
\pgfsetstrokecolor{textcolor}%
\pgfsetfillcolor{textcolor}%
\pgftext[x=5.171863in, y=0.187471in, left, base,rotate=45.000000]{\color{textcolor}{\sffamily\fontsize{11.000000}{13.200000}\selectfont\catcode`\^=\active\def^{\ifmmode\sp\else\^{}\fi}\catcode`\%=\active\def%{\%}cuda1}}%
\end{pgfscope}%
\begin{pgfscope}%
\pgfsetbuttcap%
\pgfsetroundjoin%
\definecolor{currentfill}{rgb}{0.000000,0.000000,0.000000}%
\pgfsetfillcolor{currentfill}%
\pgfsetlinewidth{0.803000pt}%
\definecolor{currentstroke}{rgb}{0.000000,0.000000,0.000000}%
\pgfsetstrokecolor{currentstroke}%
\pgfsetdash{}{0pt}%
\pgfsys@defobject{currentmarker}{\pgfqpoint{-0.048611in}{0.000000in}}{\pgfqpoint{-0.000000in}{0.000000in}}{%
\pgfpathmoveto{\pgfqpoint{-0.000000in}{0.000000in}}%
\pgfpathlineto{\pgfqpoint{-0.048611in}{0.000000in}}%
\pgfusepath{stroke,fill}%
}%
\begin{pgfscope}%
\pgfsys@transformshift{0.617715in}{0.698141in}%
\pgfsys@useobject{currentmarker}{}%
\end{pgfscope}%
\end{pgfscope}%
\begin{pgfscope}%
\definecolor{textcolor}{rgb}{0.000000,0.000000,0.000000}%
\pgfsetstrokecolor{textcolor}%
\pgfsetfillcolor{textcolor}%
\pgftext[x=0.444451in, y=0.640103in, left, base]{\color{textcolor}{\sffamily\fontsize{11.000000}{13.200000}\selectfont\catcode`\^=\active\def^{\ifmmode\sp\else\^{}\fi}\catcode`\%=\active\def%{\%}$\mathdefault{0}$}}%
\end{pgfscope}%
\begin{pgfscope}%
\pgfsetbuttcap%
\pgfsetroundjoin%
\definecolor{currentfill}{rgb}{0.000000,0.000000,0.000000}%
\pgfsetfillcolor{currentfill}%
\pgfsetlinewidth{0.803000pt}%
\definecolor{currentstroke}{rgb}{0.000000,0.000000,0.000000}%
\pgfsetstrokecolor{currentstroke}%
\pgfsetdash{}{0pt}%
\pgfsys@defobject{currentmarker}{\pgfqpoint{-0.048611in}{0.000000in}}{\pgfqpoint{-0.000000in}{0.000000in}}{%
\pgfpathmoveto{\pgfqpoint{-0.000000in}{0.000000in}}%
\pgfpathlineto{\pgfqpoint{-0.048611in}{0.000000in}}%
\pgfusepath{stroke,fill}%
}%
\begin{pgfscope}%
\pgfsys@transformshift{0.617715in}{1.230430in}%
\pgfsys@useobject{currentmarker}{}%
\end{pgfscope}%
\end{pgfscope}%
\begin{pgfscope}%
\definecolor{textcolor}{rgb}{0.000000,0.000000,0.000000}%
\pgfsetstrokecolor{textcolor}%
\pgfsetfillcolor{textcolor}%
\pgftext[x=0.368410in, y=1.172393in, left, base]{\color{textcolor}{\sffamily\fontsize{11.000000}{13.200000}\selectfont\catcode`\^=\active\def^{\ifmmode\sp\else\^{}\fi}\catcode`\%=\active\def%{\%}$\mathdefault{10}$}}%
\end{pgfscope}%
\begin{pgfscope}%
\pgfsetbuttcap%
\pgfsetroundjoin%
\definecolor{currentfill}{rgb}{0.000000,0.000000,0.000000}%
\pgfsetfillcolor{currentfill}%
\pgfsetlinewidth{0.803000pt}%
\definecolor{currentstroke}{rgb}{0.000000,0.000000,0.000000}%
\pgfsetstrokecolor{currentstroke}%
\pgfsetdash{}{0pt}%
\pgfsys@defobject{currentmarker}{\pgfqpoint{-0.048611in}{0.000000in}}{\pgfqpoint{-0.000000in}{0.000000in}}{%
\pgfpathmoveto{\pgfqpoint{-0.000000in}{0.000000in}}%
\pgfpathlineto{\pgfqpoint{-0.048611in}{0.000000in}}%
\pgfusepath{stroke,fill}%
}%
\begin{pgfscope}%
\pgfsys@transformshift{0.617715in}{1.762720in}%
\pgfsys@useobject{currentmarker}{}%
\end{pgfscope}%
\end{pgfscope}%
\begin{pgfscope}%
\definecolor{textcolor}{rgb}{0.000000,0.000000,0.000000}%
\pgfsetstrokecolor{textcolor}%
\pgfsetfillcolor{textcolor}%
\pgftext[x=0.368410in, y=1.704682in, left, base]{\color{textcolor}{\sffamily\fontsize{11.000000}{13.200000}\selectfont\catcode`\^=\active\def^{\ifmmode\sp\else\^{}\fi}\catcode`\%=\active\def%{\%}$\mathdefault{20}$}}%
\end{pgfscope}%
\begin{pgfscope}%
\pgfsetbuttcap%
\pgfsetroundjoin%
\definecolor{currentfill}{rgb}{0.000000,0.000000,0.000000}%
\pgfsetfillcolor{currentfill}%
\pgfsetlinewidth{0.803000pt}%
\definecolor{currentstroke}{rgb}{0.000000,0.000000,0.000000}%
\pgfsetstrokecolor{currentstroke}%
\pgfsetdash{}{0pt}%
\pgfsys@defobject{currentmarker}{\pgfqpoint{-0.048611in}{0.000000in}}{\pgfqpoint{-0.000000in}{0.000000in}}{%
\pgfpathmoveto{\pgfqpoint{-0.000000in}{0.000000in}}%
\pgfpathlineto{\pgfqpoint{-0.048611in}{0.000000in}}%
\pgfusepath{stroke,fill}%
}%
\begin{pgfscope}%
\pgfsys@transformshift{0.617715in}{2.295009in}%
\pgfsys@useobject{currentmarker}{}%
\end{pgfscope}%
\end{pgfscope}%
\begin{pgfscope}%
\definecolor{textcolor}{rgb}{0.000000,0.000000,0.000000}%
\pgfsetstrokecolor{textcolor}%
\pgfsetfillcolor{textcolor}%
\pgftext[x=0.368410in, y=2.236972in, left, base]{\color{textcolor}{\sffamily\fontsize{11.000000}{13.200000}\selectfont\catcode`\^=\active\def^{\ifmmode\sp\else\^{}\fi}\catcode`\%=\active\def%{\%}$\mathdefault{30}$}}%
\end{pgfscope}%
\begin{pgfscope}%
\pgfsetbuttcap%
\pgfsetroundjoin%
\definecolor{currentfill}{rgb}{0.000000,0.000000,0.000000}%
\pgfsetfillcolor{currentfill}%
\pgfsetlinewidth{0.803000pt}%
\definecolor{currentstroke}{rgb}{0.000000,0.000000,0.000000}%
\pgfsetstrokecolor{currentstroke}%
\pgfsetdash{}{0pt}%
\pgfsys@defobject{currentmarker}{\pgfqpoint{-0.048611in}{0.000000in}}{\pgfqpoint{-0.000000in}{0.000000in}}{%
\pgfpathmoveto{\pgfqpoint{-0.000000in}{0.000000in}}%
\pgfpathlineto{\pgfqpoint{-0.048611in}{0.000000in}}%
\pgfusepath{stroke,fill}%
}%
\begin{pgfscope}%
\pgfsys@transformshift{0.617715in}{2.827299in}%
\pgfsys@useobject{currentmarker}{}%
\end{pgfscope}%
\end{pgfscope}%
\begin{pgfscope}%
\definecolor{textcolor}{rgb}{0.000000,0.000000,0.000000}%
\pgfsetstrokecolor{textcolor}%
\pgfsetfillcolor{textcolor}%
\pgftext[x=0.368410in, y=2.769261in, left, base]{\color{textcolor}{\sffamily\fontsize{11.000000}{13.200000}\selectfont\catcode`\^=\active\def^{\ifmmode\sp\else\^{}\fi}\catcode`\%=\active\def%{\%}$\mathdefault{40}$}}%
\end{pgfscope}%
\begin{pgfscope}%
\pgfsetbuttcap%
\pgfsetroundjoin%
\definecolor{currentfill}{rgb}{0.000000,0.000000,0.000000}%
\pgfsetfillcolor{currentfill}%
\pgfsetlinewidth{0.803000pt}%
\definecolor{currentstroke}{rgb}{0.000000,0.000000,0.000000}%
\pgfsetstrokecolor{currentstroke}%
\pgfsetdash{}{0pt}%
\pgfsys@defobject{currentmarker}{\pgfqpoint{-0.048611in}{0.000000in}}{\pgfqpoint{-0.000000in}{0.000000in}}{%
\pgfpathmoveto{\pgfqpoint{-0.000000in}{0.000000in}}%
\pgfpathlineto{\pgfqpoint{-0.048611in}{0.000000in}}%
\pgfusepath{stroke,fill}%
}%
\begin{pgfscope}%
\pgfsys@transformshift{0.617715in}{3.359588in}%
\pgfsys@useobject{currentmarker}{}%
\end{pgfscope}%
\end{pgfscope}%
\begin{pgfscope}%
\definecolor{textcolor}{rgb}{0.000000,0.000000,0.000000}%
\pgfsetstrokecolor{textcolor}%
\pgfsetfillcolor{textcolor}%
\pgftext[x=0.368410in, y=3.301551in, left, base]{\color{textcolor}{\sffamily\fontsize{11.000000}{13.200000}\selectfont\catcode`\^=\active\def^{\ifmmode\sp\else\^{}\fi}\catcode`\%=\active\def%{\%}$\mathdefault{50}$}}%
\end{pgfscope}%
\begin{pgfscope}%
\pgfsetbuttcap%
\pgfsetroundjoin%
\definecolor{currentfill}{rgb}{0.000000,0.000000,0.000000}%
\pgfsetfillcolor{currentfill}%
\pgfsetlinewidth{0.803000pt}%
\definecolor{currentstroke}{rgb}{0.000000,0.000000,0.000000}%
\pgfsetstrokecolor{currentstroke}%
\pgfsetdash{}{0pt}%
\pgfsys@defobject{currentmarker}{\pgfqpoint{-0.048611in}{0.000000in}}{\pgfqpoint{-0.000000in}{0.000000in}}{%
\pgfpathmoveto{\pgfqpoint{-0.000000in}{0.000000in}}%
\pgfpathlineto{\pgfqpoint{-0.048611in}{0.000000in}}%
\pgfusepath{stroke,fill}%
}%
\begin{pgfscope}%
\pgfsys@transformshift{0.617715in}{3.891878in}%
\pgfsys@useobject{currentmarker}{}%
\end{pgfscope}%
\end{pgfscope}%
\begin{pgfscope}%
\definecolor{textcolor}{rgb}{0.000000,0.000000,0.000000}%
\pgfsetstrokecolor{textcolor}%
\pgfsetfillcolor{textcolor}%
\pgftext[x=0.368410in, y=3.833840in, left, base]{\color{textcolor}{\sffamily\fontsize{11.000000}{13.200000}\selectfont\catcode`\^=\active\def^{\ifmmode\sp\else\^{}\fi}\catcode`\%=\active\def%{\%}$\mathdefault{60}$}}%
\end{pgfscope}%
\begin{pgfscope}%
\definecolor{textcolor}{rgb}{0.000000,0.000000,0.000000}%
\pgfsetstrokecolor{textcolor}%
\pgfsetfillcolor{textcolor}%
\pgftext[x=0.312854in,y=2.555259in,,bottom,rotate=90.000000]{\color{textcolor}{\sffamily\fontsize{11.000000}{13.200000}\selectfont\catcode`\^=\active\def^{\ifmmode\sp\else\^{}\fi}\catcode`\%=\active\def%{\%}Affected part of the workload (% insn)}}%
\end{pgfscope}%
\begin{pgfscope}%
\pgfsetrectcap%
\pgfsetmiterjoin%
\pgfsetlinewidth{0.803000pt}%
\definecolor{currentstroke}{rgb}{0.000000,0.000000,0.000000}%
\pgfsetstrokecolor{currentstroke}%
\pgfsetdash{}{0pt}%
\pgfpathmoveto{\pgfqpoint{0.617715in}{0.698141in}}%
\pgfpathlineto{\pgfqpoint{0.617715in}{4.412376in}}%
\pgfusepath{stroke}%
\end{pgfscope}%
\begin{pgfscope}%
\pgfsetrectcap%
\pgfsetmiterjoin%
\pgfsetlinewidth{0.803000pt}%
\definecolor{currentstroke}{rgb}{0.000000,0.000000,0.000000}%
\pgfsetstrokecolor{currentstroke}%
\pgfsetdash{}{0pt}%
\pgfpathmoveto{\pgfqpoint{6.235000in}{0.698141in}}%
\pgfpathlineto{\pgfqpoint{6.235000in}{4.412376in}}%
\pgfusepath{stroke}%
\end{pgfscope}%
\begin{pgfscope}%
\pgfsetrectcap%
\pgfsetmiterjoin%
\pgfsetlinewidth{0.803000pt}%
\definecolor{currentstroke}{rgb}{0.000000,0.000000,0.000000}%
\pgfsetstrokecolor{currentstroke}%
\pgfsetdash{}{0pt}%
\pgfpathmoveto{\pgfqpoint{0.617715in}{0.698141in}}%
\pgfpathlineto{\pgfqpoint{6.235000in}{0.698141in}}%
\pgfusepath{stroke}%
\end{pgfscope}%
\begin{pgfscope}%
\pgfsetrectcap%
\pgfsetmiterjoin%
\pgfsetlinewidth{0.803000pt}%
\definecolor{currentstroke}{rgb}{0.000000,0.000000,0.000000}%
\pgfsetstrokecolor{currentstroke}%
\pgfsetdash{}{0pt}%
\pgfpathmoveto{\pgfqpoint{0.617715in}{4.412376in}}%
\pgfpathlineto{\pgfqpoint{6.235000in}{4.412376in}}%
\pgfusepath{stroke}%
\end{pgfscope}%
\begin{pgfscope}%
\definecolor{textcolor}{rgb}{0.000000,0.000000,0.000000}%
\pgfsetstrokecolor{textcolor}%
\pgfsetfillcolor{textcolor}%
\pgftext[x=3.426358in,y=4.495710in,,base]{\color{textcolor}{\sffamily\fontsize{13.200000}{15.840000}\selectfont\catcode`\^=\active\def^{\ifmmode\sp\else\^{}\fi}\catcode`\%=\active\def%{\%}Effect of flushing on workloads}}%
\end{pgfscope}%
\begin{pgfscope}%
\pgfsetbuttcap%
\pgfsetmiterjoin%
\definecolor{currentfill}{rgb}{1.000000,1.000000,1.000000}%
\pgfsetfillcolor{currentfill}%
\pgfsetfillopacity{0.800000}%
\pgfsetlinewidth{1.003750pt}%
\definecolor{currentstroke}{rgb}{0.800000,0.800000,0.800000}%
\pgfsetstrokecolor{currentstroke}%
\pgfsetstrokeopacity{0.800000}%
\pgfsetdash{}{0pt}%
\pgfpathmoveto{\pgfqpoint{3.885606in}{3.393182in}}%
\pgfpathlineto{\pgfqpoint{6.128056in}{3.393182in}}%
\pgfpathquadraticcurveto{\pgfqpoint{6.158611in}{3.393182in}}{\pgfqpoint{6.158611in}{3.423738in}}%
\pgfpathlineto{\pgfqpoint{6.158611in}{4.305432in}}%
\pgfpathquadraticcurveto{\pgfqpoint{6.158611in}{4.335987in}}{\pgfqpoint{6.128056in}{4.335987in}}%
\pgfpathlineto{\pgfqpoint{3.885606in}{4.335987in}}%
\pgfpathquadraticcurveto{\pgfqpoint{3.855051in}{4.335987in}}{\pgfqpoint{3.855051in}{4.305432in}}%
\pgfpathlineto{\pgfqpoint{3.855051in}{3.423738in}}%
\pgfpathquadraticcurveto{\pgfqpoint{3.855051in}{3.393182in}}{\pgfqpoint{3.885606in}{3.393182in}}%
\pgfpathlineto{\pgfqpoint{3.885606in}{3.393182in}}%
\pgfpathclose%
\pgfusepath{stroke,fill}%
\end{pgfscope}%
\begin{pgfscope}%
\pgfsetbuttcap%
\pgfsetmiterjoin%
\definecolor{currentfill}{rgb}{0.121569,0.466667,0.705882}%
\pgfsetfillcolor{currentfill}%
\pgfsetlinewidth{0.000000pt}%
\definecolor{currentstroke}{rgb}{0.000000,0.000000,0.000000}%
\pgfsetstrokecolor{currentstroke}%
\pgfsetstrokeopacity{0.000000}%
\pgfsetdash{}{0pt}%
\pgfpathmoveto{\pgfqpoint{3.916162in}{4.158801in}}%
\pgfpathlineto{\pgfqpoint{4.221717in}{4.158801in}}%
\pgfpathlineto{\pgfqpoint{4.221717in}{4.265746in}}%
\pgfpathlineto{\pgfqpoint{3.916162in}{4.265746in}}%
\pgfpathlineto{\pgfqpoint{3.916162in}{4.158801in}}%
\pgfpathclose%
\pgfusepath{fill}%
\end{pgfscope}%
\begin{pgfscope}%
\definecolor{textcolor}{rgb}{0.000000,0.000000,0.000000}%
\pgfsetstrokecolor{textcolor}%
\pgfsetfillcolor{textcolor}%
\pgftext[x=4.343939in,y=4.158801in,left,base]{\color{textcolor}{\sffamily\fontsize{11.000000}{13.200000}\selectfont\catcode`\^=\active\def^{\ifmmode\sp\else\^{}\fi}\catcode`\%=\active\def%{\%}$\geq5%$ difference in IPC}}%
\end{pgfscope}%
\begin{pgfscope}%
\pgfsetbuttcap%
\pgfsetmiterjoin%
\definecolor{currentfill}{rgb}{1.000000,0.498039,0.054902}%
\pgfsetfillcolor{currentfill}%
\pgfsetlinewidth{0.000000pt}%
\definecolor{currentstroke}{rgb}{0.000000,0.000000,0.000000}%
\pgfsetstrokecolor{currentstroke}%
\pgfsetstrokeopacity{0.000000}%
\pgfsetdash{}{0pt}%
\pgfpathmoveto{\pgfqpoint{3.916162in}{3.934558in}}%
\pgfpathlineto{\pgfqpoint{4.221717in}{3.934558in}}%
\pgfpathlineto{\pgfqpoint{4.221717in}{4.041503in}}%
\pgfpathlineto{\pgfqpoint{3.916162in}{4.041503in}}%
\pgfpathlineto{\pgfqpoint{3.916162in}{3.934558in}}%
\pgfpathclose%
\pgfusepath{fill}%
\end{pgfscope}%
\begin{pgfscope}%
\definecolor{textcolor}{rgb}{0.000000,0.000000,0.000000}%
\pgfsetstrokecolor{textcolor}%
\pgfsetfillcolor{textcolor}%
\pgftext[x=4.343939in,y=3.934558in,left,base]{\color{textcolor}{\sffamily\fontsize{11.000000}{13.200000}\selectfont\catcode`\^=\active\def^{\ifmmode\sp\else\^{}\fi}\catcode`\%=\active\def%{\%}$\geq10%$ difference in IPC}}%
\end{pgfscope}%
\begin{pgfscope}%
\pgfsetbuttcap%
\pgfsetmiterjoin%
\definecolor{currentfill}{rgb}{0.172549,0.627451,0.172549}%
\pgfsetfillcolor{currentfill}%
\pgfsetlinewidth{0.000000pt}%
\definecolor{currentstroke}{rgb}{0.000000,0.000000,0.000000}%
\pgfsetstrokecolor{currentstroke}%
\pgfsetstrokeopacity{0.000000}%
\pgfsetdash{}{0pt}%
\pgfpathmoveto{\pgfqpoint{3.916162in}{3.710315in}}%
\pgfpathlineto{\pgfqpoint{4.221717in}{3.710315in}}%
\pgfpathlineto{\pgfqpoint{4.221717in}{3.817260in}}%
\pgfpathlineto{\pgfqpoint{3.916162in}{3.817260in}}%
\pgfpathlineto{\pgfqpoint{3.916162in}{3.710315in}}%
\pgfpathclose%
\pgfusepath{fill}%
\end{pgfscope}%
\begin{pgfscope}%
\definecolor{textcolor}{rgb}{0.000000,0.000000,0.000000}%
\pgfsetstrokecolor{textcolor}%
\pgfsetfillcolor{textcolor}%
\pgftext[x=4.343939in,y=3.710315in,left,base]{\color{textcolor}{\sffamily\fontsize{11.000000}{13.200000}\selectfont\catcode`\^=\active\def^{\ifmmode\sp\else\^{}\fi}\catcode`\%=\active\def%{\%}$\geq15%$ difference in IPC}}%
\end{pgfscope}%
\begin{pgfscope}%
\pgfsetbuttcap%
\pgfsetmiterjoin%
\definecolor{currentfill}{rgb}{0.839216,0.152941,0.156863}%
\pgfsetfillcolor{currentfill}%
\pgfsetlinewidth{0.000000pt}%
\definecolor{currentstroke}{rgb}{0.000000,0.000000,0.000000}%
\pgfsetstrokecolor{currentstroke}%
\pgfsetstrokeopacity{0.000000}%
\pgfsetdash{}{0pt}%
\pgfpathmoveto{\pgfqpoint{3.916162in}{3.486072in}}%
\pgfpathlineto{\pgfqpoint{4.221717in}{3.486072in}}%
\pgfpathlineto{\pgfqpoint{4.221717in}{3.593017in}}%
\pgfpathlineto{\pgfqpoint{3.916162in}{3.593017in}}%
\pgfpathlineto{\pgfqpoint{3.916162in}{3.486072in}}%
\pgfpathclose%
\pgfusepath{fill}%
\end{pgfscope}%
\begin{pgfscope}%
\definecolor{textcolor}{rgb}{0.000000,0.000000,0.000000}%
\pgfsetstrokecolor{textcolor}%
\pgfsetfillcolor{textcolor}%
\pgftext[x=4.343939in,y=3.486072in,left,base]{\color{textcolor}{\sffamily\fontsize{11.000000}{13.200000}\selectfont\catcode`\^=\active\def^{\ifmmode\sp\else\^{}\fi}\catcode`\%=\active\def%{\%}$\geq20%$ difference in IPC}}%
\end{pgfscope}%
\end{pgfpicture}%
\makeatother%
\endgroup%
}
        \end{minipage}
        \caption{Weighted IPC differences for different DCT inputs}
        \label{fig:weight_ipc_dct}
    \end{subfigure}
    \caption{Weighted IPC differences}
\end{figure}

In \cref{fig:weight_ipc_diff}, you can see the result of this analysis for the Cactus and MLPerf workloads.
The results for the DCT workload (with its same four input images) is shown in \cref{fig:weight_ipc_dct}.
We've set four thresholds for the relative IPC difference (5\%, 10\%, 15\%, and 20\%).
For each of these thresholds, we've summed up the weights of all kernels where the IPC difference is at least that much.
This means that e.g.\ for the GRU workload, approximately 16\% of the entire workload suffers from a difference of at least 5\%.

From these figures, we end up with a different set of affected workloads.
Most of the MLPerf workloads only suffer slightly from the cold-start problem, with the only notable exception being the 3D U-Net workload.
In the Cactus suite, we found language translation (LGT), spatial transformer (SPT), reinforcement learning (RFL), and Gunrock (with road input, GRU) to be the most affected.
The real outlier here, however, is the DCT one.
It consistently suffers from at least 20\% relative IPC difference, no matter the input.

\FloatBarrier

\section{Data reuse}\label{sec:data-reuse}
We assume that the cold-start problem is more prevalent in workloads with high data reuse.
To verify this, we've analyzed the degree of inter-kernel data reuse in the DCT workload.
For each kernel, we've profiled all memory instructions and extracted the unique memory addresses.

\begin{figure}[ht]
    \centering
    \begin{subfigure}{0.4\textwidth}
        \resizebox{\textwidth}{!}{%% Creator: Matplotlib, PGF backend
%%
%% To include the figure in your LaTeX document, write
%%   \input{<filename>.pgf}
%%
%% Make sure the required packages are loaded in your preamble
%%   \usepackage{pgf}
%%
%% Also ensure that all the required font packages are loaded; for instance,
%% the lmodern package is sometimes necessary when using math font.
%%   \usepackage{lmodern}
%%
%% Figures using additional raster images can only be included by \input if
%% they are in the same directory as the main LaTeX file. For loading figures
%% from other directories you can use the `import` package
%%   \usepackage{import}
%%
%% and then include the figures with
%%   \import{<path to file>}{<filename>.pgf}
%%
%% Matplotlib used the following preamble
%%   \def\mathdefault#1{#1}
%%   \everymath=\expandafter{\the\everymath\displaystyle}
%%   
%%   \usepackage{fontspec}
%%   \setmainfont{DejaVuSerif.ttf}[Path=\detokenize{/usr/lib/python3.11/site-packages/matplotlib/mpl-data/fonts/ttf/}]
%%   \setsansfont{DejaVuSans.ttf}[Path=\detokenize{/usr/lib/python3.11/site-packages/matplotlib/mpl-data/fonts/ttf/}]
%%   \setmonofont{DejaVuSansMono.ttf}[Path=\detokenize{/usr/lib/python3.11/site-packages/matplotlib/mpl-data/fonts/ttf/}]
%%   \makeatletter\@ifpackageloaded{underscore}{}{\usepackage[strings]{underscore}}\makeatother
%%
\begingroup%
\makeatletter%
\begin{pgfpicture}%
\pgfpathrectangle{\pgfpointorigin}{\pgfqpoint{6.400000in}{4.800000in}}%
\pgfusepath{use as bounding box, clip}%
\begin{pgfscope}%
\pgfsetbuttcap%
\pgfsetmiterjoin%
\definecolor{currentfill}{rgb}{1.000000,1.000000,1.000000}%
\pgfsetfillcolor{currentfill}%
\pgfsetlinewidth{0.000000pt}%
\definecolor{currentstroke}{rgb}{1.000000,1.000000,1.000000}%
\pgfsetstrokecolor{currentstroke}%
\pgfsetdash{}{0pt}%
\pgfpathmoveto{\pgfqpoint{0.000000in}{0.000000in}}%
\pgfpathlineto{\pgfqpoint{6.400000in}{0.000000in}}%
\pgfpathlineto{\pgfqpoint{6.400000in}{4.800000in}}%
\pgfpathlineto{\pgfqpoint{0.000000in}{4.800000in}}%
\pgfpathlineto{\pgfqpoint{0.000000in}{0.000000in}}%
\pgfpathclose%
\pgfusepath{fill}%
\end{pgfscope}%
\begin{pgfscope}%
\pgfsetbuttcap%
\pgfsetmiterjoin%
\definecolor{currentfill}{rgb}{1.000000,1.000000,1.000000}%
\pgfsetfillcolor{currentfill}%
\pgfsetlinewidth{0.000000pt}%
\definecolor{currentstroke}{rgb}{0.000000,0.000000,0.000000}%
\pgfsetstrokecolor{currentstroke}%
\pgfsetstrokeopacity{0.000000}%
\pgfsetdash{}{0pt}%
\pgfpathmoveto{\pgfqpoint{0.800000in}{0.528000in}}%
\pgfpathlineto{\pgfqpoint{5.760000in}{0.528000in}}%
\pgfpathlineto{\pgfqpoint{5.760000in}{4.224000in}}%
\pgfpathlineto{\pgfqpoint{0.800000in}{4.224000in}}%
\pgfpathlineto{\pgfqpoint{0.800000in}{0.528000in}}%
\pgfpathclose%
\pgfusepath{fill}%
\end{pgfscope}%
\begin{pgfscope}%
\pgfpathrectangle{\pgfqpoint{0.800000in}{0.528000in}}{\pgfqpoint{4.960000in}{3.696000in}}%
\pgfusepath{clip}%
\pgfsetbuttcap%
\pgfsetmiterjoin%
\definecolor{currentfill}{rgb}{0.000000,0.000000,1.000000}%
\pgfsetfillcolor{currentfill}%
\pgfsetlinewidth{0.000000pt}%
\definecolor{currentstroke}{rgb}{0.000000,0.000000,0.000000}%
\pgfsetstrokecolor{currentstroke}%
\pgfsetstrokeopacity{0.000000}%
\pgfsetdash{}{0pt}%
\pgfpathmoveto{\pgfqpoint{1.025455in}{0.528000in}}%
\pgfpathlineto{\pgfqpoint{1.056339in}{0.528000in}}%
\pgfpathlineto{\pgfqpoint{1.056339in}{4.224000in}}%
\pgfpathlineto{\pgfqpoint{1.025455in}{4.224000in}}%
\pgfpathlineto{\pgfqpoint{1.025455in}{0.528000in}}%
\pgfpathclose%
\pgfusepath{fill}%
\end{pgfscope}%
\begin{pgfscope}%
\pgfpathrectangle{\pgfqpoint{0.800000in}{0.528000in}}{\pgfqpoint{4.960000in}{3.696000in}}%
\pgfusepath{clip}%
\pgfsetbuttcap%
\pgfsetmiterjoin%
\definecolor{currentfill}{rgb}{0.000000,0.000000,1.000000}%
\pgfsetfillcolor{currentfill}%
\pgfsetlinewidth{0.000000pt}%
\definecolor{currentstroke}{rgb}{0.000000,0.000000,0.000000}%
\pgfsetstrokecolor{currentstroke}%
\pgfsetstrokeopacity{0.000000}%
\pgfsetdash{}{0pt}%
\pgfpathmoveto{\pgfqpoint{1.064060in}{0.528000in}}%
\pgfpathlineto{\pgfqpoint{1.094944in}{0.528000in}}%
\pgfpathlineto{\pgfqpoint{1.094944in}{4.224000in}}%
\pgfpathlineto{\pgfqpoint{1.064060in}{4.224000in}}%
\pgfpathlineto{\pgfqpoint{1.064060in}{0.528000in}}%
\pgfpathclose%
\pgfusepath{fill}%
\end{pgfscope}%
\begin{pgfscope}%
\pgfpathrectangle{\pgfqpoint{0.800000in}{0.528000in}}{\pgfqpoint{4.960000in}{3.696000in}}%
\pgfusepath{clip}%
\pgfsetbuttcap%
\pgfsetmiterjoin%
\definecolor{currentfill}{rgb}{0.000000,0.000000,1.000000}%
\pgfsetfillcolor{currentfill}%
\pgfsetlinewidth{0.000000pt}%
\definecolor{currentstroke}{rgb}{0.000000,0.000000,0.000000}%
\pgfsetstrokecolor{currentstroke}%
\pgfsetstrokeopacity{0.000000}%
\pgfsetdash{}{0pt}%
\pgfpathmoveto{\pgfqpoint{1.102665in}{0.528000in}}%
\pgfpathlineto{\pgfqpoint{1.133549in}{0.528000in}}%
\pgfpathlineto{\pgfqpoint{1.133549in}{4.224000in}}%
\pgfpathlineto{\pgfqpoint{1.102665in}{4.224000in}}%
\pgfpathlineto{\pgfqpoint{1.102665in}{0.528000in}}%
\pgfpathclose%
\pgfusepath{fill}%
\end{pgfscope}%
\begin{pgfscope}%
\pgfpathrectangle{\pgfqpoint{0.800000in}{0.528000in}}{\pgfqpoint{4.960000in}{3.696000in}}%
\pgfusepath{clip}%
\pgfsetbuttcap%
\pgfsetmiterjoin%
\definecolor{currentfill}{rgb}{0.000000,0.000000,1.000000}%
\pgfsetfillcolor{currentfill}%
\pgfsetlinewidth{0.000000pt}%
\definecolor{currentstroke}{rgb}{0.000000,0.000000,0.000000}%
\pgfsetstrokecolor{currentstroke}%
\pgfsetstrokeopacity{0.000000}%
\pgfsetdash{}{0pt}%
\pgfpathmoveto{\pgfqpoint{1.141270in}{0.528000in}}%
\pgfpathlineto{\pgfqpoint{1.172154in}{0.528000in}}%
\pgfpathlineto{\pgfqpoint{1.172154in}{4.224000in}}%
\pgfpathlineto{\pgfqpoint{1.141270in}{4.224000in}}%
\pgfpathlineto{\pgfqpoint{1.141270in}{0.528000in}}%
\pgfpathclose%
\pgfusepath{fill}%
\end{pgfscope}%
\begin{pgfscope}%
\pgfpathrectangle{\pgfqpoint{0.800000in}{0.528000in}}{\pgfqpoint{4.960000in}{3.696000in}}%
\pgfusepath{clip}%
\pgfsetbuttcap%
\pgfsetmiterjoin%
\definecolor{currentfill}{rgb}{0.000000,0.000000,1.000000}%
\pgfsetfillcolor{currentfill}%
\pgfsetlinewidth{0.000000pt}%
\definecolor{currentstroke}{rgb}{0.000000,0.000000,0.000000}%
\pgfsetstrokecolor{currentstroke}%
\pgfsetstrokeopacity{0.000000}%
\pgfsetdash{}{0pt}%
\pgfpathmoveto{\pgfqpoint{1.179875in}{0.528000in}}%
\pgfpathlineto{\pgfqpoint{1.210760in}{0.528000in}}%
\pgfpathlineto{\pgfqpoint{1.210760in}{4.224000in}}%
\pgfpathlineto{\pgfqpoint{1.179875in}{4.224000in}}%
\pgfpathlineto{\pgfqpoint{1.179875in}{0.528000in}}%
\pgfpathclose%
\pgfusepath{fill}%
\end{pgfscope}%
\begin{pgfscope}%
\pgfpathrectangle{\pgfqpoint{0.800000in}{0.528000in}}{\pgfqpoint{4.960000in}{3.696000in}}%
\pgfusepath{clip}%
\pgfsetbuttcap%
\pgfsetmiterjoin%
\definecolor{currentfill}{rgb}{0.000000,0.000000,1.000000}%
\pgfsetfillcolor{currentfill}%
\pgfsetlinewidth{0.000000pt}%
\definecolor{currentstroke}{rgb}{0.000000,0.000000,0.000000}%
\pgfsetstrokecolor{currentstroke}%
\pgfsetstrokeopacity{0.000000}%
\pgfsetdash{}{0pt}%
\pgfpathmoveto{\pgfqpoint{1.218481in}{0.528000in}}%
\pgfpathlineto{\pgfqpoint{1.249365in}{0.528000in}}%
\pgfpathlineto{\pgfqpoint{1.249365in}{4.224000in}}%
\pgfpathlineto{\pgfqpoint{1.218481in}{4.224000in}}%
\pgfpathlineto{\pgfqpoint{1.218481in}{0.528000in}}%
\pgfpathclose%
\pgfusepath{fill}%
\end{pgfscope}%
\begin{pgfscope}%
\pgfpathrectangle{\pgfqpoint{0.800000in}{0.528000in}}{\pgfqpoint{4.960000in}{3.696000in}}%
\pgfusepath{clip}%
\pgfsetbuttcap%
\pgfsetmiterjoin%
\definecolor{currentfill}{rgb}{0.000000,0.000000,1.000000}%
\pgfsetfillcolor{currentfill}%
\pgfsetlinewidth{0.000000pt}%
\definecolor{currentstroke}{rgb}{0.000000,0.000000,0.000000}%
\pgfsetstrokecolor{currentstroke}%
\pgfsetstrokeopacity{0.000000}%
\pgfsetdash{}{0pt}%
\pgfpathmoveto{\pgfqpoint{1.257086in}{0.528000in}}%
\pgfpathlineto{\pgfqpoint{1.287970in}{0.528000in}}%
\pgfpathlineto{\pgfqpoint{1.287970in}{4.224000in}}%
\pgfpathlineto{\pgfqpoint{1.257086in}{4.224000in}}%
\pgfpathlineto{\pgfqpoint{1.257086in}{0.528000in}}%
\pgfpathclose%
\pgfusepath{fill}%
\end{pgfscope}%
\begin{pgfscope}%
\pgfpathrectangle{\pgfqpoint{0.800000in}{0.528000in}}{\pgfqpoint{4.960000in}{3.696000in}}%
\pgfusepath{clip}%
\pgfsetbuttcap%
\pgfsetmiterjoin%
\definecolor{currentfill}{rgb}{0.000000,0.000000,1.000000}%
\pgfsetfillcolor{currentfill}%
\pgfsetlinewidth{0.000000pt}%
\definecolor{currentstroke}{rgb}{0.000000,0.000000,0.000000}%
\pgfsetstrokecolor{currentstroke}%
\pgfsetstrokeopacity{0.000000}%
\pgfsetdash{}{0pt}%
\pgfpathmoveto{\pgfqpoint{1.295691in}{0.528000in}}%
\pgfpathlineto{\pgfqpoint{1.326575in}{0.528000in}}%
\pgfpathlineto{\pgfqpoint{1.326575in}{4.224000in}}%
\pgfpathlineto{\pgfqpoint{1.295691in}{4.224000in}}%
\pgfpathlineto{\pgfqpoint{1.295691in}{0.528000in}}%
\pgfpathclose%
\pgfusepath{fill}%
\end{pgfscope}%
\begin{pgfscope}%
\pgfpathrectangle{\pgfqpoint{0.800000in}{0.528000in}}{\pgfqpoint{4.960000in}{3.696000in}}%
\pgfusepath{clip}%
\pgfsetbuttcap%
\pgfsetmiterjoin%
\definecolor{currentfill}{rgb}{0.000000,0.000000,1.000000}%
\pgfsetfillcolor{currentfill}%
\pgfsetlinewidth{0.000000pt}%
\definecolor{currentstroke}{rgb}{0.000000,0.000000,0.000000}%
\pgfsetstrokecolor{currentstroke}%
\pgfsetstrokeopacity{0.000000}%
\pgfsetdash{}{0pt}%
\pgfpathmoveto{\pgfqpoint{1.334296in}{0.528000in}}%
\pgfpathlineto{\pgfqpoint{1.365181in}{0.528000in}}%
\pgfpathlineto{\pgfqpoint{1.365181in}{4.224000in}}%
\pgfpathlineto{\pgfqpoint{1.334296in}{4.224000in}}%
\pgfpathlineto{\pgfqpoint{1.334296in}{0.528000in}}%
\pgfpathclose%
\pgfusepath{fill}%
\end{pgfscope}%
\begin{pgfscope}%
\pgfpathrectangle{\pgfqpoint{0.800000in}{0.528000in}}{\pgfqpoint{4.960000in}{3.696000in}}%
\pgfusepath{clip}%
\pgfsetbuttcap%
\pgfsetmiterjoin%
\definecolor{currentfill}{rgb}{0.000000,0.000000,1.000000}%
\pgfsetfillcolor{currentfill}%
\pgfsetlinewidth{0.000000pt}%
\definecolor{currentstroke}{rgb}{0.000000,0.000000,0.000000}%
\pgfsetstrokecolor{currentstroke}%
\pgfsetstrokeopacity{0.000000}%
\pgfsetdash{}{0pt}%
\pgfpathmoveto{\pgfqpoint{1.372902in}{0.528000in}}%
\pgfpathlineto{\pgfqpoint{1.403786in}{0.528000in}}%
\pgfpathlineto{\pgfqpoint{1.403786in}{4.223885in}}%
\pgfpathlineto{\pgfqpoint{1.372902in}{4.223885in}}%
\pgfpathlineto{\pgfqpoint{1.372902in}{0.528000in}}%
\pgfpathclose%
\pgfusepath{fill}%
\end{pgfscope}%
\begin{pgfscope}%
\pgfpathrectangle{\pgfqpoint{0.800000in}{0.528000in}}{\pgfqpoint{4.960000in}{3.696000in}}%
\pgfusepath{clip}%
\pgfsetbuttcap%
\pgfsetmiterjoin%
\definecolor{currentfill}{rgb}{0.000000,0.000000,1.000000}%
\pgfsetfillcolor{currentfill}%
\pgfsetlinewidth{0.000000pt}%
\definecolor{currentstroke}{rgb}{0.000000,0.000000,0.000000}%
\pgfsetstrokecolor{currentstroke}%
\pgfsetstrokeopacity{0.000000}%
\pgfsetdash{}{0pt}%
\pgfpathmoveto{\pgfqpoint{1.411507in}{0.528000in}}%
\pgfpathlineto{\pgfqpoint{1.442391in}{0.528000in}}%
\pgfpathlineto{\pgfqpoint{1.442391in}{4.224000in}}%
\pgfpathlineto{\pgfqpoint{1.411507in}{4.224000in}}%
\pgfpathlineto{\pgfqpoint{1.411507in}{0.528000in}}%
\pgfpathclose%
\pgfusepath{fill}%
\end{pgfscope}%
\begin{pgfscope}%
\pgfpathrectangle{\pgfqpoint{0.800000in}{0.528000in}}{\pgfqpoint{4.960000in}{3.696000in}}%
\pgfusepath{clip}%
\pgfsetbuttcap%
\pgfsetmiterjoin%
\definecolor{currentfill}{rgb}{0.000000,0.000000,1.000000}%
\pgfsetfillcolor{currentfill}%
\pgfsetlinewidth{0.000000pt}%
\definecolor{currentstroke}{rgb}{0.000000,0.000000,0.000000}%
\pgfsetstrokecolor{currentstroke}%
\pgfsetstrokeopacity{0.000000}%
\pgfsetdash{}{0pt}%
\pgfpathmoveto{\pgfqpoint{1.450112in}{0.528000in}}%
\pgfpathlineto{\pgfqpoint{1.480996in}{0.528000in}}%
\pgfpathlineto{\pgfqpoint{1.480996in}{1.452085in}}%
\pgfpathlineto{\pgfqpoint{1.450112in}{1.452085in}}%
\pgfpathlineto{\pgfqpoint{1.450112in}{0.528000in}}%
\pgfpathclose%
\pgfusepath{fill}%
\end{pgfscope}%
\begin{pgfscope}%
\pgfpathrectangle{\pgfqpoint{0.800000in}{0.528000in}}{\pgfqpoint{4.960000in}{3.696000in}}%
\pgfusepath{clip}%
\pgfsetbuttcap%
\pgfsetmiterjoin%
\definecolor{currentfill}{rgb}{0.000000,0.000000,1.000000}%
\pgfsetfillcolor{currentfill}%
\pgfsetlinewidth{0.000000pt}%
\definecolor{currentstroke}{rgb}{0.000000,0.000000,0.000000}%
\pgfsetstrokecolor{currentstroke}%
\pgfsetstrokeopacity{0.000000}%
\pgfsetdash{}{0pt}%
\pgfpathmoveto{\pgfqpoint{1.488717in}{0.528000in}}%
\pgfpathlineto{\pgfqpoint{1.519601in}{0.528000in}}%
\pgfpathlineto{\pgfqpoint{1.519601in}{4.224000in}}%
\pgfpathlineto{\pgfqpoint{1.488717in}{4.224000in}}%
\pgfpathlineto{\pgfqpoint{1.488717in}{0.528000in}}%
\pgfpathclose%
\pgfusepath{fill}%
\end{pgfscope}%
\begin{pgfscope}%
\pgfpathrectangle{\pgfqpoint{0.800000in}{0.528000in}}{\pgfqpoint{4.960000in}{3.696000in}}%
\pgfusepath{clip}%
\pgfsetbuttcap%
\pgfsetmiterjoin%
\definecolor{currentfill}{rgb}{0.000000,0.000000,1.000000}%
\pgfsetfillcolor{currentfill}%
\pgfsetlinewidth{0.000000pt}%
\definecolor{currentstroke}{rgb}{0.000000,0.000000,0.000000}%
\pgfsetstrokecolor{currentstroke}%
\pgfsetstrokeopacity{0.000000}%
\pgfsetdash{}{0pt}%
\pgfpathmoveto{\pgfqpoint{1.527323in}{0.528000in}}%
\pgfpathlineto{\pgfqpoint{1.558207in}{0.528000in}}%
\pgfpathlineto{\pgfqpoint{1.558207in}{4.224000in}}%
\pgfpathlineto{\pgfqpoint{1.527323in}{4.224000in}}%
\pgfpathlineto{\pgfqpoint{1.527323in}{0.528000in}}%
\pgfpathclose%
\pgfusepath{fill}%
\end{pgfscope}%
\begin{pgfscope}%
\pgfpathrectangle{\pgfqpoint{0.800000in}{0.528000in}}{\pgfqpoint{4.960000in}{3.696000in}}%
\pgfusepath{clip}%
\pgfsetbuttcap%
\pgfsetmiterjoin%
\definecolor{currentfill}{rgb}{0.000000,0.000000,1.000000}%
\pgfsetfillcolor{currentfill}%
\pgfsetlinewidth{0.000000pt}%
\definecolor{currentstroke}{rgb}{0.000000,0.000000,0.000000}%
\pgfsetstrokecolor{currentstroke}%
\pgfsetstrokeopacity{0.000000}%
\pgfsetdash{}{0pt}%
\pgfpathmoveto{\pgfqpoint{1.565928in}{0.528000in}}%
\pgfpathlineto{\pgfqpoint{1.596812in}{0.528000in}}%
\pgfpathlineto{\pgfqpoint{1.596812in}{4.224000in}}%
\pgfpathlineto{\pgfqpoint{1.565928in}{4.224000in}}%
\pgfpathlineto{\pgfqpoint{1.565928in}{0.528000in}}%
\pgfpathclose%
\pgfusepath{fill}%
\end{pgfscope}%
\begin{pgfscope}%
\pgfpathrectangle{\pgfqpoint{0.800000in}{0.528000in}}{\pgfqpoint{4.960000in}{3.696000in}}%
\pgfusepath{clip}%
\pgfsetbuttcap%
\pgfsetmiterjoin%
\definecolor{currentfill}{rgb}{0.000000,0.000000,1.000000}%
\pgfsetfillcolor{currentfill}%
\pgfsetlinewidth{0.000000pt}%
\definecolor{currentstroke}{rgb}{0.000000,0.000000,0.000000}%
\pgfsetstrokecolor{currentstroke}%
\pgfsetstrokeopacity{0.000000}%
\pgfsetdash{}{0pt}%
\pgfpathmoveto{\pgfqpoint{1.604533in}{0.528000in}}%
\pgfpathlineto{\pgfqpoint{1.635417in}{0.528000in}}%
\pgfpathlineto{\pgfqpoint{1.635417in}{4.224000in}}%
\pgfpathlineto{\pgfqpoint{1.604533in}{4.224000in}}%
\pgfpathlineto{\pgfqpoint{1.604533in}{0.528000in}}%
\pgfpathclose%
\pgfusepath{fill}%
\end{pgfscope}%
\begin{pgfscope}%
\pgfpathrectangle{\pgfqpoint{0.800000in}{0.528000in}}{\pgfqpoint{4.960000in}{3.696000in}}%
\pgfusepath{clip}%
\pgfsetbuttcap%
\pgfsetmiterjoin%
\definecolor{currentfill}{rgb}{0.000000,0.000000,1.000000}%
\pgfsetfillcolor{currentfill}%
\pgfsetlinewidth{0.000000pt}%
\definecolor{currentstroke}{rgb}{0.000000,0.000000,0.000000}%
\pgfsetstrokecolor{currentstroke}%
\pgfsetstrokeopacity{0.000000}%
\pgfsetdash{}{0pt}%
\pgfpathmoveto{\pgfqpoint{1.643138in}{0.528000in}}%
\pgfpathlineto{\pgfqpoint{1.674022in}{0.528000in}}%
\pgfpathlineto{\pgfqpoint{1.674022in}{4.224000in}}%
\pgfpathlineto{\pgfqpoint{1.643138in}{4.224000in}}%
\pgfpathlineto{\pgfqpoint{1.643138in}{0.528000in}}%
\pgfpathclose%
\pgfusepath{fill}%
\end{pgfscope}%
\begin{pgfscope}%
\pgfpathrectangle{\pgfqpoint{0.800000in}{0.528000in}}{\pgfqpoint{4.960000in}{3.696000in}}%
\pgfusepath{clip}%
\pgfsetbuttcap%
\pgfsetmiterjoin%
\definecolor{currentfill}{rgb}{0.000000,0.000000,1.000000}%
\pgfsetfillcolor{currentfill}%
\pgfsetlinewidth{0.000000pt}%
\definecolor{currentstroke}{rgb}{0.000000,0.000000,0.000000}%
\pgfsetstrokecolor{currentstroke}%
\pgfsetstrokeopacity{0.000000}%
\pgfsetdash{}{0pt}%
\pgfpathmoveto{\pgfqpoint{1.681743in}{0.528000in}}%
\pgfpathlineto{\pgfqpoint{1.712628in}{0.528000in}}%
\pgfpathlineto{\pgfqpoint{1.712628in}{4.224000in}}%
\pgfpathlineto{\pgfqpoint{1.681743in}{4.224000in}}%
\pgfpathlineto{\pgfqpoint{1.681743in}{0.528000in}}%
\pgfpathclose%
\pgfusepath{fill}%
\end{pgfscope}%
\begin{pgfscope}%
\pgfpathrectangle{\pgfqpoint{0.800000in}{0.528000in}}{\pgfqpoint{4.960000in}{3.696000in}}%
\pgfusepath{clip}%
\pgfsetbuttcap%
\pgfsetmiterjoin%
\definecolor{currentfill}{rgb}{0.000000,0.000000,1.000000}%
\pgfsetfillcolor{currentfill}%
\pgfsetlinewidth{0.000000pt}%
\definecolor{currentstroke}{rgb}{0.000000,0.000000,0.000000}%
\pgfsetstrokecolor{currentstroke}%
\pgfsetstrokeopacity{0.000000}%
\pgfsetdash{}{0pt}%
\pgfpathmoveto{\pgfqpoint{1.720349in}{0.528000in}}%
\pgfpathlineto{\pgfqpoint{1.751233in}{0.528000in}}%
\pgfpathlineto{\pgfqpoint{1.751233in}{4.224000in}}%
\pgfpathlineto{\pgfqpoint{1.720349in}{4.224000in}}%
\pgfpathlineto{\pgfqpoint{1.720349in}{0.528000in}}%
\pgfpathclose%
\pgfusepath{fill}%
\end{pgfscope}%
\begin{pgfscope}%
\pgfpathrectangle{\pgfqpoint{0.800000in}{0.528000in}}{\pgfqpoint{4.960000in}{3.696000in}}%
\pgfusepath{clip}%
\pgfsetbuttcap%
\pgfsetmiterjoin%
\definecolor{currentfill}{rgb}{0.000000,0.000000,1.000000}%
\pgfsetfillcolor{currentfill}%
\pgfsetlinewidth{0.000000pt}%
\definecolor{currentstroke}{rgb}{0.000000,0.000000,0.000000}%
\pgfsetstrokecolor{currentstroke}%
\pgfsetstrokeopacity{0.000000}%
\pgfsetdash{}{0pt}%
\pgfpathmoveto{\pgfqpoint{1.758954in}{0.528000in}}%
\pgfpathlineto{\pgfqpoint{1.789838in}{0.528000in}}%
\pgfpathlineto{\pgfqpoint{1.789838in}{4.224000in}}%
\pgfpathlineto{\pgfqpoint{1.758954in}{4.224000in}}%
\pgfpathlineto{\pgfqpoint{1.758954in}{0.528000in}}%
\pgfpathclose%
\pgfusepath{fill}%
\end{pgfscope}%
\begin{pgfscope}%
\pgfpathrectangle{\pgfqpoint{0.800000in}{0.528000in}}{\pgfqpoint{4.960000in}{3.696000in}}%
\pgfusepath{clip}%
\pgfsetbuttcap%
\pgfsetmiterjoin%
\definecolor{currentfill}{rgb}{0.000000,0.000000,1.000000}%
\pgfsetfillcolor{currentfill}%
\pgfsetlinewidth{0.000000pt}%
\definecolor{currentstroke}{rgb}{0.000000,0.000000,0.000000}%
\pgfsetstrokecolor{currentstroke}%
\pgfsetstrokeopacity{0.000000}%
\pgfsetdash{}{0pt}%
\pgfpathmoveto{\pgfqpoint{1.797559in}{0.528000in}}%
\pgfpathlineto{\pgfqpoint{1.828443in}{0.528000in}}%
\pgfpathlineto{\pgfqpoint{1.828443in}{4.224000in}}%
\pgfpathlineto{\pgfqpoint{1.797559in}{4.224000in}}%
\pgfpathlineto{\pgfqpoint{1.797559in}{0.528000in}}%
\pgfpathclose%
\pgfusepath{fill}%
\end{pgfscope}%
\begin{pgfscope}%
\pgfpathrectangle{\pgfqpoint{0.800000in}{0.528000in}}{\pgfqpoint{4.960000in}{3.696000in}}%
\pgfusepath{clip}%
\pgfsetbuttcap%
\pgfsetmiterjoin%
\definecolor{currentfill}{rgb}{0.000000,0.000000,1.000000}%
\pgfsetfillcolor{currentfill}%
\pgfsetlinewidth{0.000000pt}%
\definecolor{currentstroke}{rgb}{0.000000,0.000000,0.000000}%
\pgfsetstrokecolor{currentstroke}%
\pgfsetstrokeopacity{0.000000}%
\pgfsetdash{}{0pt}%
\pgfpathmoveto{\pgfqpoint{1.836164in}{0.528000in}}%
\pgfpathlineto{\pgfqpoint{1.867049in}{0.528000in}}%
\pgfpathlineto{\pgfqpoint{1.867049in}{4.224000in}}%
\pgfpathlineto{\pgfqpoint{1.836164in}{4.224000in}}%
\pgfpathlineto{\pgfqpoint{1.836164in}{0.528000in}}%
\pgfpathclose%
\pgfusepath{fill}%
\end{pgfscope}%
\begin{pgfscope}%
\pgfpathrectangle{\pgfqpoint{0.800000in}{0.528000in}}{\pgfqpoint{4.960000in}{3.696000in}}%
\pgfusepath{clip}%
\pgfsetbuttcap%
\pgfsetmiterjoin%
\definecolor{currentfill}{rgb}{0.000000,0.000000,1.000000}%
\pgfsetfillcolor{currentfill}%
\pgfsetlinewidth{0.000000pt}%
\definecolor{currentstroke}{rgb}{0.000000,0.000000,0.000000}%
\pgfsetstrokecolor{currentstroke}%
\pgfsetstrokeopacity{0.000000}%
\pgfsetdash{}{0pt}%
\pgfpathmoveto{\pgfqpoint{1.874770in}{0.528000in}}%
\pgfpathlineto{\pgfqpoint{1.905654in}{0.528000in}}%
\pgfpathlineto{\pgfqpoint{1.905654in}{4.224000in}}%
\pgfpathlineto{\pgfqpoint{1.874770in}{4.224000in}}%
\pgfpathlineto{\pgfqpoint{1.874770in}{0.528000in}}%
\pgfpathclose%
\pgfusepath{fill}%
\end{pgfscope}%
\begin{pgfscope}%
\pgfpathrectangle{\pgfqpoint{0.800000in}{0.528000in}}{\pgfqpoint{4.960000in}{3.696000in}}%
\pgfusepath{clip}%
\pgfsetbuttcap%
\pgfsetmiterjoin%
\definecolor{currentfill}{rgb}{0.000000,0.000000,1.000000}%
\pgfsetfillcolor{currentfill}%
\pgfsetlinewidth{0.000000pt}%
\definecolor{currentstroke}{rgb}{0.000000,0.000000,0.000000}%
\pgfsetstrokecolor{currentstroke}%
\pgfsetstrokeopacity{0.000000}%
\pgfsetdash{}{0pt}%
\pgfpathmoveto{\pgfqpoint{1.913375in}{0.528000in}}%
\pgfpathlineto{\pgfqpoint{1.944259in}{0.528000in}}%
\pgfpathlineto{\pgfqpoint{1.944259in}{4.224000in}}%
\pgfpathlineto{\pgfqpoint{1.913375in}{4.224000in}}%
\pgfpathlineto{\pgfqpoint{1.913375in}{0.528000in}}%
\pgfpathclose%
\pgfusepath{fill}%
\end{pgfscope}%
\begin{pgfscope}%
\pgfpathrectangle{\pgfqpoint{0.800000in}{0.528000in}}{\pgfqpoint{4.960000in}{3.696000in}}%
\pgfusepath{clip}%
\pgfsetbuttcap%
\pgfsetmiterjoin%
\definecolor{currentfill}{rgb}{0.000000,0.000000,1.000000}%
\pgfsetfillcolor{currentfill}%
\pgfsetlinewidth{0.000000pt}%
\definecolor{currentstroke}{rgb}{0.000000,0.000000,0.000000}%
\pgfsetstrokecolor{currentstroke}%
\pgfsetstrokeopacity{0.000000}%
\pgfsetdash{}{0pt}%
\pgfpathmoveto{\pgfqpoint{1.951980in}{0.528000in}}%
\pgfpathlineto{\pgfqpoint{1.982864in}{0.528000in}}%
\pgfpathlineto{\pgfqpoint{1.982864in}{4.224000in}}%
\pgfpathlineto{\pgfqpoint{1.951980in}{4.224000in}}%
\pgfpathlineto{\pgfqpoint{1.951980in}{0.528000in}}%
\pgfpathclose%
\pgfusepath{fill}%
\end{pgfscope}%
\begin{pgfscope}%
\pgfpathrectangle{\pgfqpoint{0.800000in}{0.528000in}}{\pgfqpoint{4.960000in}{3.696000in}}%
\pgfusepath{clip}%
\pgfsetbuttcap%
\pgfsetmiterjoin%
\definecolor{currentfill}{rgb}{0.000000,0.000000,1.000000}%
\pgfsetfillcolor{currentfill}%
\pgfsetlinewidth{0.000000pt}%
\definecolor{currentstroke}{rgb}{0.000000,0.000000,0.000000}%
\pgfsetstrokecolor{currentstroke}%
\pgfsetstrokeopacity{0.000000}%
\pgfsetdash{}{0pt}%
\pgfpathmoveto{\pgfqpoint{1.990585in}{0.528000in}}%
\pgfpathlineto{\pgfqpoint{2.021469in}{0.528000in}}%
\pgfpathlineto{\pgfqpoint{2.021469in}{4.224000in}}%
\pgfpathlineto{\pgfqpoint{1.990585in}{4.224000in}}%
\pgfpathlineto{\pgfqpoint{1.990585in}{0.528000in}}%
\pgfpathclose%
\pgfusepath{fill}%
\end{pgfscope}%
\begin{pgfscope}%
\pgfpathrectangle{\pgfqpoint{0.800000in}{0.528000in}}{\pgfqpoint{4.960000in}{3.696000in}}%
\pgfusepath{clip}%
\pgfsetbuttcap%
\pgfsetmiterjoin%
\definecolor{currentfill}{rgb}{0.000000,0.000000,1.000000}%
\pgfsetfillcolor{currentfill}%
\pgfsetlinewidth{0.000000pt}%
\definecolor{currentstroke}{rgb}{0.000000,0.000000,0.000000}%
\pgfsetstrokecolor{currentstroke}%
\pgfsetstrokeopacity{0.000000}%
\pgfsetdash{}{0pt}%
\pgfpathmoveto{\pgfqpoint{2.029191in}{0.528000in}}%
\pgfpathlineto{\pgfqpoint{2.060075in}{0.528000in}}%
\pgfpathlineto{\pgfqpoint{2.060075in}{4.224000in}}%
\pgfpathlineto{\pgfqpoint{2.029191in}{4.224000in}}%
\pgfpathlineto{\pgfqpoint{2.029191in}{0.528000in}}%
\pgfpathclose%
\pgfusepath{fill}%
\end{pgfscope}%
\begin{pgfscope}%
\pgfpathrectangle{\pgfqpoint{0.800000in}{0.528000in}}{\pgfqpoint{4.960000in}{3.696000in}}%
\pgfusepath{clip}%
\pgfsetbuttcap%
\pgfsetmiterjoin%
\definecolor{currentfill}{rgb}{0.000000,0.000000,1.000000}%
\pgfsetfillcolor{currentfill}%
\pgfsetlinewidth{0.000000pt}%
\definecolor{currentstroke}{rgb}{0.000000,0.000000,0.000000}%
\pgfsetstrokecolor{currentstroke}%
\pgfsetstrokeopacity{0.000000}%
\pgfsetdash{}{0pt}%
\pgfpathmoveto{\pgfqpoint{2.067796in}{0.528000in}}%
\pgfpathlineto{\pgfqpoint{2.098680in}{0.528000in}}%
\pgfpathlineto{\pgfqpoint{2.098680in}{4.224000in}}%
\pgfpathlineto{\pgfqpoint{2.067796in}{4.224000in}}%
\pgfpathlineto{\pgfqpoint{2.067796in}{0.528000in}}%
\pgfpathclose%
\pgfusepath{fill}%
\end{pgfscope}%
\begin{pgfscope}%
\pgfpathrectangle{\pgfqpoint{0.800000in}{0.528000in}}{\pgfqpoint{4.960000in}{3.696000in}}%
\pgfusepath{clip}%
\pgfsetbuttcap%
\pgfsetmiterjoin%
\definecolor{currentfill}{rgb}{0.000000,0.000000,1.000000}%
\pgfsetfillcolor{currentfill}%
\pgfsetlinewidth{0.000000pt}%
\definecolor{currentstroke}{rgb}{0.000000,0.000000,0.000000}%
\pgfsetstrokecolor{currentstroke}%
\pgfsetstrokeopacity{0.000000}%
\pgfsetdash{}{0pt}%
\pgfpathmoveto{\pgfqpoint{2.106401in}{0.528000in}}%
\pgfpathlineto{\pgfqpoint{2.137285in}{0.528000in}}%
\pgfpathlineto{\pgfqpoint{2.137285in}{4.224000in}}%
\pgfpathlineto{\pgfqpoint{2.106401in}{4.224000in}}%
\pgfpathlineto{\pgfqpoint{2.106401in}{0.528000in}}%
\pgfpathclose%
\pgfusepath{fill}%
\end{pgfscope}%
\begin{pgfscope}%
\pgfpathrectangle{\pgfqpoint{0.800000in}{0.528000in}}{\pgfqpoint{4.960000in}{3.696000in}}%
\pgfusepath{clip}%
\pgfsetbuttcap%
\pgfsetmiterjoin%
\definecolor{currentfill}{rgb}{0.000000,0.000000,1.000000}%
\pgfsetfillcolor{currentfill}%
\pgfsetlinewidth{0.000000pt}%
\definecolor{currentstroke}{rgb}{0.000000,0.000000,0.000000}%
\pgfsetstrokecolor{currentstroke}%
\pgfsetstrokeopacity{0.000000}%
\pgfsetdash{}{0pt}%
\pgfpathmoveto{\pgfqpoint{2.145006in}{0.528000in}}%
\pgfpathlineto{\pgfqpoint{2.175890in}{0.528000in}}%
\pgfpathlineto{\pgfqpoint{2.175890in}{4.224000in}}%
\pgfpathlineto{\pgfqpoint{2.145006in}{4.224000in}}%
\pgfpathlineto{\pgfqpoint{2.145006in}{0.528000in}}%
\pgfpathclose%
\pgfusepath{fill}%
\end{pgfscope}%
\begin{pgfscope}%
\pgfpathrectangle{\pgfqpoint{0.800000in}{0.528000in}}{\pgfqpoint{4.960000in}{3.696000in}}%
\pgfusepath{clip}%
\pgfsetbuttcap%
\pgfsetmiterjoin%
\definecolor{currentfill}{rgb}{0.000000,0.000000,1.000000}%
\pgfsetfillcolor{currentfill}%
\pgfsetlinewidth{0.000000pt}%
\definecolor{currentstroke}{rgb}{0.000000,0.000000,0.000000}%
\pgfsetstrokecolor{currentstroke}%
\pgfsetstrokeopacity{0.000000}%
\pgfsetdash{}{0pt}%
\pgfpathmoveto{\pgfqpoint{2.183611in}{0.528000in}}%
\pgfpathlineto{\pgfqpoint{2.214496in}{0.528000in}}%
\pgfpathlineto{\pgfqpoint{2.214496in}{4.224000in}}%
\pgfpathlineto{\pgfqpoint{2.183611in}{4.224000in}}%
\pgfpathlineto{\pgfqpoint{2.183611in}{0.528000in}}%
\pgfpathclose%
\pgfusepath{fill}%
\end{pgfscope}%
\begin{pgfscope}%
\pgfpathrectangle{\pgfqpoint{0.800000in}{0.528000in}}{\pgfqpoint{4.960000in}{3.696000in}}%
\pgfusepath{clip}%
\pgfsetbuttcap%
\pgfsetmiterjoin%
\definecolor{currentfill}{rgb}{0.000000,0.000000,1.000000}%
\pgfsetfillcolor{currentfill}%
\pgfsetlinewidth{0.000000pt}%
\definecolor{currentstroke}{rgb}{0.000000,0.000000,0.000000}%
\pgfsetstrokecolor{currentstroke}%
\pgfsetstrokeopacity{0.000000}%
\pgfsetdash{}{0pt}%
\pgfpathmoveto{\pgfqpoint{2.222217in}{0.528000in}}%
\pgfpathlineto{\pgfqpoint{2.253101in}{0.528000in}}%
\pgfpathlineto{\pgfqpoint{2.253101in}{4.224000in}}%
\pgfpathlineto{\pgfqpoint{2.222217in}{4.224000in}}%
\pgfpathlineto{\pgfqpoint{2.222217in}{0.528000in}}%
\pgfpathclose%
\pgfusepath{fill}%
\end{pgfscope}%
\begin{pgfscope}%
\pgfpathrectangle{\pgfqpoint{0.800000in}{0.528000in}}{\pgfqpoint{4.960000in}{3.696000in}}%
\pgfusepath{clip}%
\pgfsetbuttcap%
\pgfsetmiterjoin%
\definecolor{currentfill}{rgb}{0.000000,0.000000,1.000000}%
\pgfsetfillcolor{currentfill}%
\pgfsetlinewidth{0.000000pt}%
\definecolor{currentstroke}{rgb}{0.000000,0.000000,0.000000}%
\pgfsetstrokecolor{currentstroke}%
\pgfsetstrokeopacity{0.000000}%
\pgfsetdash{}{0pt}%
\pgfpathmoveto{\pgfqpoint{2.260822in}{0.528000in}}%
\pgfpathlineto{\pgfqpoint{2.291706in}{0.528000in}}%
\pgfpathlineto{\pgfqpoint{2.291706in}{4.224000in}}%
\pgfpathlineto{\pgfqpoint{2.260822in}{4.224000in}}%
\pgfpathlineto{\pgfqpoint{2.260822in}{0.528000in}}%
\pgfpathclose%
\pgfusepath{fill}%
\end{pgfscope}%
\begin{pgfscope}%
\pgfpathrectangle{\pgfqpoint{0.800000in}{0.528000in}}{\pgfqpoint{4.960000in}{3.696000in}}%
\pgfusepath{clip}%
\pgfsetbuttcap%
\pgfsetmiterjoin%
\definecolor{currentfill}{rgb}{0.000000,0.000000,1.000000}%
\pgfsetfillcolor{currentfill}%
\pgfsetlinewidth{0.000000pt}%
\definecolor{currentstroke}{rgb}{0.000000,0.000000,0.000000}%
\pgfsetstrokecolor{currentstroke}%
\pgfsetstrokeopacity{0.000000}%
\pgfsetdash{}{0pt}%
\pgfpathmoveto{\pgfqpoint{2.299427in}{0.528000in}}%
\pgfpathlineto{\pgfqpoint{2.330311in}{0.528000in}}%
\pgfpathlineto{\pgfqpoint{2.330311in}{4.224000in}}%
\pgfpathlineto{\pgfqpoint{2.299427in}{4.224000in}}%
\pgfpathlineto{\pgfqpoint{2.299427in}{0.528000in}}%
\pgfpathclose%
\pgfusepath{fill}%
\end{pgfscope}%
\begin{pgfscope}%
\pgfpathrectangle{\pgfqpoint{0.800000in}{0.528000in}}{\pgfqpoint{4.960000in}{3.696000in}}%
\pgfusepath{clip}%
\pgfsetbuttcap%
\pgfsetmiterjoin%
\definecolor{currentfill}{rgb}{0.000000,0.000000,1.000000}%
\pgfsetfillcolor{currentfill}%
\pgfsetlinewidth{0.000000pt}%
\definecolor{currentstroke}{rgb}{0.000000,0.000000,0.000000}%
\pgfsetstrokecolor{currentstroke}%
\pgfsetstrokeopacity{0.000000}%
\pgfsetdash{}{0pt}%
\pgfpathmoveto{\pgfqpoint{2.338032in}{0.528000in}}%
\pgfpathlineto{\pgfqpoint{2.368917in}{0.528000in}}%
\pgfpathlineto{\pgfqpoint{2.368917in}{4.224000in}}%
\pgfpathlineto{\pgfqpoint{2.338032in}{4.224000in}}%
\pgfpathlineto{\pgfqpoint{2.338032in}{0.528000in}}%
\pgfpathclose%
\pgfusepath{fill}%
\end{pgfscope}%
\begin{pgfscope}%
\pgfpathrectangle{\pgfqpoint{0.800000in}{0.528000in}}{\pgfqpoint{4.960000in}{3.696000in}}%
\pgfusepath{clip}%
\pgfsetbuttcap%
\pgfsetmiterjoin%
\definecolor{currentfill}{rgb}{0.000000,0.000000,1.000000}%
\pgfsetfillcolor{currentfill}%
\pgfsetlinewidth{0.000000pt}%
\definecolor{currentstroke}{rgb}{0.000000,0.000000,0.000000}%
\pgfsetstrokecolor{currentstroke}%
\pgfsetstrokeopacity{0.000000}%
\pgfsetdash{}{0pt}%
\pgfpathmoveto{\pgfqpoint{2.376638in}{0.528000in}}%
\pgfpathlineto{\pgfqpoint{2.407522in}{0.528000in}}%
\pgfpathlineto{\pgfqpoint{2.407522in}{4.224000in}}%
\pgfpathlineto{\pgfqpoint{2.376638in}{4.224000in}}%
\pgfpathlineto{\pgfqpoint{2.376638in}{0.528000in}}%
\pgfpathclose%
\pgfusepath{fill}%
\end{pgfscope}%
\begin{pgfscope}%
\pgfpathrectangle{\pgfqpoint{0.800000in}{0.528000in}}{\pgfqpoint{4.960000in}{3.696000in}}%
\pgfusepath{clip}%
\pgfsetbuttcap%
\pgfsetmiterjoin%
\definecolor{currentfill}{rgb}{0.000000,0.000000,1.000000}%
\pgfsetfillcolor{currentfill}%
\pgfsetlinewidth{0.000000pt}%
\definecolor{currentstroke}{rgb}{0.000000,0.000000,0.000000}%
\pgfsetstrokecolor{currentstroke}%
\pgfsetstrokeopacity{0.000000}%
\pgfsetdash{}{0pt}%
\pgfpathmoveto{\pgfqpoint{2.415243in}{0.528000in}}%
\pgfpathlineto{\pgfqpoint{2.446127in}{0.528000in}}%
\pgfpathlineto{\pgfqpoint{2.446127in}{4.224000in}}%
\pgfpathlineto{\pgfqpoint{2.415243in}{4.224000in}}%
\pgfpathlineto{\pgfqpoint{2.415243in}{0.528000in}}%
\pgfpathclose%
\pgfusepath{fill}%
\end{pgfscope}%
\begin{pgfscope}%
\pgfpathrectangle{\pgfqpoint{0.800000in}{0.528000in}}{\pgfqpoint{4.960000in}{3.696000in}}%
\pgfusepath{clip}%
\pgfsetbuttcap%
\pgfsetmiterjoin%
\definecolor{currentfill}{rgb}{0.000000,0.000000,1.000000}%
\pgfsetfillcolor{currentfill}%
\pgfsetlinewidth{0.000000pt}%
\definecolor{currentstroke}{rgb}{0.000000,0.000000,0.000000}%
\pgfsetstrokecolor{currentstroke}%
\pgfsetstrokeopacity{0.000000}%
\pgfsetdash{}{0pt}%
\pgfpathmoveto{\pgfqpoint{2.453848in}{0.528000in}}%
\pgfpathlineto{\pgfqpoint{2.484732in}{0.528000in}}%
\pgfpathlineto{\pgfqpoint{2.484732in}{4.224000in}}%
\pgfpathlineto{\pgfqpoint{2.453848in}{4.224000in}}%
\pgfpathlineto{\pgfqpoint{2.453848in}{0.528000in}}%
\pgfpathclose%
\pgfusepath{fill}%
\end{pgfscope}%
\begin{pgfscope}%
\pgfpathrectangle{\pgfqpoint{0.800000in}{0.528000in}}{\pgfqpoint{4.960000in}{3.696000in}}%
\pgfusepath{clip}%
\pgfsetbuttcap%
\pgfsetmiterjoin%
\definecolor{currentfill}{rgb}{0.000000,0.000000,1.000000}%
\pgfsetfillcolor{currentfill}%
\pgfsetlinewidth{0.000000pt}%
\definecolor{currentstroke}{rgb}{0.000000,0.000000,0.000000}%
\pgfsetstrokecolor{currentstroke}%
\pgfsetstrokeopacity{0.000000}%
\pgfsetdash{}{0pt}%
\pgfpathmoveto{\pgfqpoint{2.492453in}{0.528000in}}%
\pgfpathlineto{\pgfqpoint{2.523337in}{0.528000in}}%
\pgfpathlineto{\pgfqpoint{2.523337in}{4.224000in}}%
\pgfpathlineto{\pgfqpoint{2.492453in}{4.224000in}}%
\pgfpathlineto{\pgfqpoint{2.492453in}{0.528000in}}%
\pgfpathclose%
\pgfusepath{fill}%
\end{pgfscope}%
\begin{pgfscope}%
\pgfpathrectangle{\pgfqpoint{0.800000in}{0.528000in}}{\pgfqpoint{4.960000in}{3.696000in}}%
\pgfusepath{clip}%
\pgfsetbuttcap%
\pgfsetmiterjoin%
\definecolor{currentfill}{rgb}{0.000000,0.000000,1.000000}%
\pgfsetfillcolor{currentfill}%
\pgfsetlinewidth{0.000000pt}%
\definecolor{currentstroke}{rgb}{0.000000,0.000000,0.000000}%
\pgfsetstrokecolor{currentstroke}%
\pgfsetstrokeopacity{0.000000}%
\pgfsetdash{}{0pt}%
\pgfpathmoveto{\pgfqpoint{2.531059in}{0.528000in}}%
\pgfpathlineto{\pgfqpoint{2.561943in}{0.528000in}}%
\pgfpathlineto{\pgfqpoint{2.561943in}{4.224000in}}%
\pgfpathlineto{\pgfqpoint{2.531059in}{4.224000in}}%
\pgfpathlineto{\pgfqpoint{2.531059in}{0.528000in}}%
\pgfpathclose%
\pgfusepath{fill}%
\end{pgfscope}%
\begin{pgfscope}%
\pgfpathrectangle{\pgfqpoint{0.800000in}{0.528000in}}{\pgfqpoint{4.960000in}{3.696000in}}%
\pgfusepath{clip}%
\pgfsetbuttcap%
\pgfsetmiterjoin%
\definecolor{currentfill}{rgb}{0.000000,0.000000,1.000000}%
\pgfsetfillcolor{currentfill}%
\pgfsetlinewidth{0.000000pt}%
\definecolor{currentstroke}{rgb}{0.000000,0.000000,0.000000}%
\pgfsetstrokecolor{currentstroke}%
\pgfsetstrokeopacity{0.000000}%
\pgfsetdash{}{0pt}%
\pgfpathmoveto{\pgfqpoint{2.569664in}{0.528000in}}%
\pgfpathlineto{\pgfqpoint{2.600548in}{0.528000in}}%
\pgfpathlineto{\pgfqpoint{2.600548in}{4.224000in}}%
\pgfpathlineto{\pgfqpoint{2.569664in}{4.224000in}}%
\pgfpathlineto{\pgfqpoint{2.569664in}{0.528000in}}%
\pgfpathclose%
\pgfusepath{fill}%
\end{pgfscope}%
\begin{pgfscope}%
\pgfpathrectangle{\pgfqpoint{0.800000in}{0.528000in}}{\pgfqpoint{4.960000in}{3.696000in}}%
\pgfusepath{clip}%
\pgfsetbuttcap%
\pgfsetmiterjoin%
\definecolor{currentfill}{rgb}{0.000000,0.000000,1.000000}%
\pgfsetfillcolor{currentfill}%
\pgfsetlinewidth{0.000000pt}%
\definecolor{currentstroke}{rgb}{0.000000,0.000000,0.000000}%
\pgfsetstrokecolor{currentstroke}%
\pgfsetstrokeopacity{0.000000}%
\pgfsetdash{}{0pt}%
\pgfpathmoveto{\pgfqpoint{2.608269in}{0.528000in}}%
\pgfpathlineto{\pgfqpoint{2.639153in}{0.528000in}}%
\pgfpathlineto{\pgfqpoint{2.639153in}{4.224000in}}%
\pgfpathlineto{\pgfqpoint{2.608269in}{4.224000in}}%
\pgfpathlineto{\pgfqpoint{2.608269in}{0.528000in}}%
\pgfpathclose%
\pgfusepath{fill}%
\end{pgfscope}%
\begin{pgfscope}%
\pgfpathrectangle{\pgfqpoint{0.800000in}{0.528000in}}{\pgfqpoint{4.960000in}{3.696000in}}%
\pgfusepath{clip}%
\pgfsetbuttcap%
\pgfsetmiterjoin%
\definecolor{currentfill}{rgb}{0.000000,0.000000,1.000000}%
\pgfsetfillcolor{currentfill}%
\pgfsetlinewidth{0.000000pt}%
\definecolor{currentstroke}{rgb}{0.000000,0.000000,0.000000}%
\pgfsetstrokecolor{currentstroke}%
\pgfsetstrokeopacity{0.000000}%
\pgfsetdash{}{0pt}%
\pgfpathmoveto{\pgfqpoint{2.646874in}{0.528000in}}%
\pgfpathlineto{\pgfqpoint{2.677758in}{0.528000in}}%
\pgfpathlineto{\pgfqpoint{2.677758in}{4.224000in}}%
\pgfpathlineto{\pgfqpoint{2.646874in}{4.224000in}}%
\pgfpathlineto{\pgfqpoint{2.646874in}{0.528000in}}%
\pgfpathclose%
\pgfusepath{fill}%
\end{pgfscope}%
\begin{pgfscope}%
\pgfpathrectangle{\pgfqpoint{0.800000in}{0.528000in}}{\pgfqpoint{4.960000in}{3.696000in}}%
\pgfusepath{clip}%
\pgfsetbuttcap%
\pgfsetmiterjoin%
\definecolor{currentfill}{rgb}{0.000000,0.000000,1.000000}%
\pgfsetfillcolor{currentfill}%
\pgfsetlinewidth{0.000000pt}%
\definecolor{currentstroke}{rgb}{0.000000,0.000000,0.000000}%
\pgfsetstrokecolor{currentstroke}%
\pgfsetstrokeopacity{0.000000}%
\pgfsetdash{}{0pt}%
\pgfpathmoveto{\pgfqpoint{2.685479in}{0.528000in}}%
\pgfpathlineto{\pgfqpoint{2.716364in}{0.528000in}}%
\pgfpathlineto{\pgfqpoint{2.716364in}{4.224000in}}%
\pgfpathlineto{\pgfqpoint{2.685479in}{4.224000in}}%
\pgfpathlineto{\pgfqpoint{2.685479in}{0.528000in}}%
\pgfpathclose%
\pgfusepath{fill}%
\end{pgfscope}%
\begin{pgfscope}%
\pgfpathrectangle{\pgfqpoint{0.800000in}{0.528000in}}{\pgfqpoint{4.960000in}{3.696000in}}%
\pgfusepath{clip}%
\pgfsetbuttcap%
\pgfsetmiterjoin%
\definecolor{currentfill}{rgb}{0.000000,0.000000,1.000000}%
\pgfsetfillcolor{currentfill}%
\pgfsetlinewidth{0.000000pt}%
\definecolor{currentstroke}{rgb}{0.000000,0.000000,0.000000}%
\pgfsetstrokecolor{currentstroke}%
\pgfsetstrokeopacity{0.000000}%
\pgfsetdash{}{0pt}%
\pgfpathmoveto{\pgfqpoint{2.724085in}{0.528000in}}%
\pgfpathlineto{\pgfqpoint{2.754969in}{0.528000in}}%
\pgfpathlineto{\pgfqpoint{2.754969in}{4.224000in}}%
\pgfpathlineto{\pgfqpoint{2.724085in}{4.224000in}}%
\pgfpathlineto{\pgfqpoint{2.724085in}{0.528000in}}%
\pgfpathclose%
\pgfusepath{fill}%
\end{pgfscope}%
\begin{pgfscope}%
\pgfpathrectangle{\pgfqpoint{0.800000in}{0.528000in}}{\pgfqpoint{4.960000in}{3.696000in}}%
\pgfusepath{clip}%
\pgfsetbuttcap%
\pgfsetmiterjoin%
\definecolor{currentfill}{rgb}{0.000000,0.000000,1.000000}%
\pgfsetfillcolor{currentfill}%
\pgfsetlinewidth{0.000000pt}%
\definecolor{currentstroke}{rgb}{0.000000,0.000000,0.000000}%
\pgfsetstrokecolor{currentstroke}%
\pgfsetstrokeopacity{0.000000}%
\pgfsetdash{}{0pt}%
\pgfpathmoveto{\pgfqpoint{2.762690in}{0.528000in}}%
\pgfpathlineto{\pgfqpoint{2.793574in}{0.528000in}}%
\pgfpathlineto{\pgfqpoint{2.793574in}{4.224000in}}%
\pgfpathlineto{\pgfqpoint{2.762690in}{4.224000in}}%
\pgfpathlineto{\pgfqpoint{2.762690in}{0.528000in}}%
\pgfpathclose%
\pgfusepath{fill}%
\end{pgfscope}%
\begin{pgfscope}%
\pgfpathrectangle{\pgfqpoint{0.800000in}{0.528000in}}{\pgfqpoint{4.960000in}{3.696000in}}%
\pgfusepath{clip}%
\pgfsetbuttcap%
\pgfsetmiterjoin%
\definecolor{currentfill}{rgb}{0.000000,0.000000,1.000000}%
\pgfsetfillcolor{currentfill}%
\pgfsetlinewidth{0.000000pt}%
\definecolor{currentstroke}{rgb}{0.000000,0.000000,0.000000}%
\pgfsetstrokecolor{currentstroke}%
\pgfsetstrokeopacity{0.000000}%
\pgfsetdash{}{0pt}%
\pgfpathmoveto{\pgfqpoint{2.801295in}{0.528000in}}%
\pgfpathlineto{\pgfqpoint{2.832179in}{0.528000in}}%
\pgfpathlineto{\pgfqpoint{2.832179in}{4.224000in}}%
\pgfpathlineto{\pgfqpoint{2.801295in}{4.224000in}}%
\pgfpathlineto{\pgfqpoint{2.801295in}{0.528000in}}%
\pgfpathclose%
\pgfusepath{fill}%
\end{pgfscope}%
\begin{pgfscope}%
\pgfpathrectangle{\pgfqpoint{0.800000in}{0.528000in}}{\pgfqpoint{4.960000in}{3.696000in}}%
\pgfusepath{clip}%
\pgfsetbuttcap%
\pgfsetmiterjoin%
\definecolor{currentfill}{rgb}{0.000000,0.000000,1.000000}%
\pgfsetfillcolor{currentfill}%
\pgfsetlinewidth{0.000000pt}%
\definecolor{currentstroke}{rgb}{0.000000,0.000000,0.000000}%
\pgfsetstrokecolor{currentstroke}%
\pgfsetstrokeopacity{0.000000}%
\pgfsetdash{}{0pt}%
\pgfpathmoveto{\pgfqpoint{2.839900in}{0.528000in}}%
\pgfpathlineto{\pgfqpoint{2.870785in}{0.528000in}}%
\pgfpathlineto{\pgfqpoint{2.870785in}{4.224000in}}%
\pgfpathlineto{\pgfqpoint{2.839900in}{4.224000in}}%
\pgfpathlineto{\pgfqpoint{2.839900in}{0.528000in}}%
\pgfpathclose%
\pgfusepath{fill}%
\end{pgfscope}%
\begin{pgfscope}%
\pgfpathrectangle{\pgfqpoint{0.800000in}{0.528000in}}{\pgfqpoint{4.960000in}{3.696000in}}%
\pgfusepath{clip}%
\pgfsetbuttcap%
\pgfsetmiterjoin%
\definecolor{currentfill}{rgb}{0.000000,0.000000,1.000000}%
\pgfsetfillcolor{currentfill}%
\pgfsetlinewidth{0.000000pt}%
\definecolor{currentstroke}{rgb}{0.000000,0.000000,0.000000}%
\pgfsetstrokecolor{currentstroke}%
\pgfsetstrokeopacity{0.000000}%
\pgfsetdash{}{0pt}%
\pgfpathmoveto{\pgfqpoint{2.878506in}{0.528000in}}%
\pgfpathlineto{\pgfqpoint{2.909390in}{0.528000in}}%
\pgfpathlineto{\pgfqpoint{2.909390in}{4.224000in}}%
\pgfpathlineto{\pgfqpoint{2.878506in}{4.224000in}}%
\pgfpathlineto{\pgfqpoint{2.878506in}{0.528000in}}%
\pgfpathclose%
\pgfusepath{fill}%
\end{pgfscope}%
\begin{pgfscope}%
\pgfpathrectangle{\pgfqpoint{0.800000in}{0.528000in}}{\pgfqpoint{4.960000in}{3.696000in}}%
\pgfusepath{clip}%
\pgfsetbuttcap%
\pgfsetmiterjoin%
\definecolor{currentfill}{rgb}{0.000000,0.000000,1.000000}%
\pgfsetfillcolor{currentfill}%
\pgfsetlinewidth{0.000000pt}%
\definecolor{currentstroke}{rgb}{0.000000,0.000000,0.000000}%
\pgfsetstrokecolor{currentstroke}%
\pgfsetstrokeopacity{0.000000}%
\pgfsetdash{}{0pt}%
\pgfpathmoveto{\pgfqpoint{2.917111in}{0.528000in}}%
\pgfpathlineto{\pgfqpoint{2.947995in}{0.528000in}}%
\pgfpathlineto{\pgfqpoint{2.947995in}{4.224000in}}%
\pgfpathlineto{\pgfqpoint{2.917111in}{4.224000in}}%
\pgfpathlineto{\pgfqpoint{2.917111in}{0.528000in}}%
\pgfpathclose%
\pgfusepath{fill}%
\end{pgfscope}%
\begin{pgfscope}%
\pgfpathrectangle{\pgfqpoint{0.800000in}{0.528000in}}{\pgfqpoint{4.960000in}{3.696000in}}%
\pgfusepath{clip}%
\pgfsetbuttcap%
\pgfsetmiterjoin%
\definecolor{currentfill}{rgb}{0.000000,0.000000,1.000000}%
\pgfsetfillcolor{currentfill}%
\pgfsetlinewidth{0.000000pt}%
\definecolor{currentstroke}{rgb}{0.000000,0.000000,0.000000}%
\pgfsetstrokecolor{currentstroke}%
\pgfsetstrokeopacity{0.000000}%
\pgfsetdash{}{0pt}%
\pgfpathmoveto{\pgfqpoint{2.955716in}{0.528000in}}%
\pgfpathlineto{\pgfqpoint{2.986600in}{0.528000in}}%
\pgfpathlineto{\pgfqpoint{2.986600in}{4.224000in}}%
\pgfpathlineto{\pgfqpoint{2.955716in}{4.224000in}}%
\pgfpathlineto{\pgfqpoint{2.955716in}{0.528000in}}%
\pgfpathclose%
\pgfusepath{fill}%
\end{pgfscope}%
\begin{pgfscope}%
\pgfpathrectangle{\pgfqpoint{0.800000in}{0.528000in}}{\pgfqpoint{4.960000in}{3.696000in}}%
\pgfusepath{clip}%
\pgfsetbuttcap%
\pgfsetmiterjoin%
\definecolor{currentfill}{rgb}{0.000000,0.000000,1.000000}%
\pgfsetfillcolor{currentfill}%
\pgfsetlinewidth{0.000000pt}%
\definecolor{currentstroke}{rgb}{0.000000,0.000000,0.000000}%
\pgfsetstrokecolor{currentstroke}%
\pgfsetstrokeopacity{0.000000}%
\pgfsetdash{}{0pt}%
\pgfpathmoveto{\pgfqpoint{2.994321in}{0.528000in}}%
\pgfpathlineto{\pgfqpoint{3.025205in}{0.528000in}}%
\pgfpathlineto{\pgfqpoint{3.025205in}{4.224000in}}%
\pgfpathlineto{\pgfqpoint{2.994321in}{4.224000in}}%
\pgfpathlineto{\pgfqpoint{2.994321in}{0.528000in}}%
\pgfpathclose%
\pgfusepath{fill}%
\end{pgfscope}%
\begin{pgfscope}%
\pgfpathrectangle{\pgfqpoint{0.800000in}{0.528000in}}{\pgfqpoint{4.960000in}{3.696000in}}%
\pgfusepath{clip}%
\pgfsetbuttcap%
\pgfsetmiterjoin%
\definecolor{currentfill}{rgb}{0.000000,0.000000,1.000000}%
\pgfsetfillcolor{currentfill}%
\pgfsetlinewidth{0.000000pt}%
\definecolor{currentstroke}{rgb}{0.000000,0.000000,0.000000}%
\pgfsetstrokecolor{currentstroke}%
\pgfsetstrokeopacity{0.000000}%
\pgfsetdash{}{0pt}%
\pgfpathmoveto{\pgfqpoint{3.032927in}{0.528000in}}%
\pgfpathlineto{\pgfqpoint{3.063811in}{0.528000in}}%
\pgfpathlineto{\pgfqpoint{3.063811in}{4.224000in}}%
\pgfpathlineto{\pgfqpoint{3.032927in}{4.224000in}}%
\pgfpathlineto{\pgfqpoint{3.032927in}{0.528000in}}%
\pgfpathclose%
\pgfusepath{fill}%
\end{pgfscope}%
\begin{pgfscope}%
\pgfpathrectangle{\pgfqpoint{0.800000in}{0.528000in}}{\pgfqpoint{4.960000in}{3.696000in}}%
\pgfusepath{clip}%
\pgfsetbuttcap%
\pgfsetmiterjoin%
\definecolor{currentfill}{rgb}{0.000000,0.000000,1.000000}%
\pgfsetfillcolor{currentfill}%
\pgfsetlinewidth{0.000000pt}%
\definecolor{currentstroke}{rgb}{0.000000,0.000000,0.000000}%
\pgfsetstrokecolor{currentstroke}%
\pgfsetstrokeopacity{0.000000}%
\pgfsetdash{}{0pt}%
\pgfpathmoveto{\pgfqpoint{3.071532in}{0.528000in}}%
\pgfpathlineto{\pgfqpoint{3.102416in}{0.528000in}}%
\pgfpathlineto{\pgfqpoint{3.102416in}{4.224000in}}%
\pgfpathlineto{\pgfqpoint{3.071532in}{4.224000in}}%
\pgfpathlineto{\pgfqpoint{3.071532in}{0.528000in}}%
\pgfpathclose%
\pgfusepath{fill}%
\end{pgfscope}%
\begin{pgfscope}%
\pgfpathrectangle{\pgfqpoint{0.800000in}{0.528000in}}{\pgfqpoint{4.960000in}{3.696000in}}%
\pgfusepath{clip}%
\pgfsetbuttcap%
\pgfsetmiterjoin%
\definecolor{currentfill}{rgb}{0.000000,0.000000,1.000000}%
\pgfsetfillcolor{currentfill}%
\pgfsetlinewidth{0.000000pt}%
\definecolor{currentstroke}{rgb}{0.000000,0.000000,0.000000}%
\pgfsetstrokecolor{currentstroke}%
\pgfsetstrokeopacity{0.000000}%
\pgfsetdash{}{0pt}%
\pgfpathmoveto{\pgfqpoint{3.110137in}{0.528000in}}%
\pgfpathlineto{\pgfqpoint{3.141021in}{0.528000in}}%
\pgfpathlineto{\pgfqpoint{3.141021in}{4.224000in}}%
\pgfpathlineto{\pgfqpoint{3.110137in}{4.224000in}}%
\pgfpathlineto{\pgfqpoint{3.110137in}{0.528000in}}%
\pgfpathclose%
\pgfusepath{fill}%
\end{pgfscope}%
\begin{pgfscope}%
\pgfpathrectangle{\pgfqpoint{0.800000in}{0.528000in}}{\pgfqpoint{4.960000in}{3.696000in}}%
\pgfusepath{clip}%
\pgfsetbuttcap%
\pgfsetmiterjoin%
\definecolor{currentfill}{rgb}{0.000000,0.000000,1.000000}%
\pgfsetfillcolor{currentfill}%
\pgfsetlinewidth{0.000000pt}%
\definecolor{currentstroke}{rgb}{0.000000,0.000000,0.000000}%
\pgfsetstrokecolor{currentstroke}%
\pgfsetstrokeopacity{0.000000}%
\pgfsetdash{}{0pt}%
\pgfpathmoveto{\pgfqpoint{3.148742in}{0.528000in}}%
\pgfpathlineto{\pgfqpoint{3.179626in}{0.528000in}}%
\pgfpathlineto{\pgfqpoint{3.179626in}{4.224000in}}%
\pgfpathlineto{\pgfqpoint{3.148742in}{4.224000in}}%
\pgfpathlineto{\pgfqpoint{3.148742in}{0.528000in}}%
\pgfpathclose%
\pgfusepath{fill}%
\end{pgfscope}%
\begin{pgfscope}%
\pgfpathrectangle{\pgfqpoint{0.800000in}{0.528000in}}{\pgfqpoint{4.960000in}{3.696000in}}%
\pgfusepath{clip}%
\pgfsetbuttcap%
\pgfsetmiterjoin%
\definecolor{currentfill}{rgb}{0.000000,0.000000,1.000000}%
\pgfsetfillcolor{currentfill}%
\pgfsetlinewidth{0.000000pt}%
\definecolor{currentstroke}{rgb}{0.000000,0.000000,0.000000}%
\pgfsetstrokecolor{currentstroke}%
\pgfsetstrokeopacity{0.000000}%
\pgfsetdash{}{0pt}%
\pgfpathmoveto{\pgfqpoint{3.187347in}{0.528000in}}%
\pgfpathlineto{\pgfqpoint{3.218232in}{0.528000in}}%
\pgfpathlineto{\pgfqpoint{3.218232in}{4.224000in}}%
\pgfpathlineto{\pgfqpoint{3.187347in}{4.224000in}}%
\pgfpathlineto{\pgfqpoint{3.187347in}{0.528000in}}%
\pgfpathclose%
\pgfusepath{fill}%
\end{pgfscope}%
\begin{pgfscope}%
\pgfpathrectangle{\pgfqpoint{0.800000in}{0.528000in}}{\pgfqpoint{4.960000in}{3.696000in}}%
\pgfusepath{clip}%
\pgfsetbuttcap%
\pgfsetmiterjoin%
\definecolor{currentfill}{rgb}{0.000000,0.000000,1.000000}%
\pgfsetfillcolor{currentfill}%
\pgfsetlinewidth{0.000000pt}%
\definecolor{currentstroke}{rgb}{0.000000,0.000000,0.000000}%
\pgfsetstrokecolor{currentstroke}%
\pgfsetstrokeopacity{0.000000}%
\pgfsetdash{}{0pt}%
\pgfpathmoveto{\pgfqpoint{3.225953in}{0.528000in}}%
\pgfpathlineto{\pgfqpoint{3.256837in}{0.528000in}}%
\pgfpathlineto{\pgfqpoint{3.256837in}{4.224000in}}%
\pgfpathlineto{\pgfqpoint{3.225953in}{4.224000in}}%
\pgfpathlineto{\pgfqpoint{3.225953in}{0.528000in}}%
\pgfpathclose%
\pgfusepath{fill}%
\end{pgfscope}%
\begin{pgfscope}%
\pgfpathrectangle{\pgfqpoint{0.800000in}{0.528000in}}{\pgfqpoint{4.960000in}{3.696000in}}%
\pgfusepath{clip}%
\pgfsetbuttcap%
\pgfsetmiterjoin%
\definecolor{currentfill}{rgb}{0.000000,0.000000,1.000000}%
\pgfsetfillcolor{currentfill}%
\pgfsetlinewidth{0.000000pt}%
\definecolor{currentstroke}{rgb}{0.000000,0.000000,0.000000}%
\pgfsetstrokecolor{currentstroke}%
\pgfsetstrokeopacity{0.000000}%
\pgfsetdash{}{0pt}%
\pgfpathmoveto{\pgfqpoint{3.264558in}{0.528000in}}%
\pgfpathlineto{\pgfqpoint{3.295442in}{0.528000in}}%
\pgfpathlineto{\pgfqpoint{3.295442in}{4.224000in}}%
\pgfpathlineto{\pgfqpoint{3.264558in}{4.224000in}}%
\pgfpathlineto{\pgfqpoint{3.264558in}{0.528000in}}%
\pgfpathclose%
\pgfusepath{fill}%
\end{pgfscope}%
\begin{pgfscope}%
\pgfpathrectangle{\pgfqpoint{0.800000in}{0.528000in}}{\pgfqpoint{4.960000in}{3.696000in}}%
\pgfusepath{clip}%
\pgfsetbuttcap%
\pgfsetmiterjoin%
\definecolor{currentfill}{rgb}{0.000000,0.000000,1.000000}%
\pgfsetfillcolor{currentfill}%
\pgfsetlinewidth{0.000000pt}%
\definecolor{currentstroke}{rgb}{0.000000,0.000000,0.000000}%
\pgfsetstrokecolor{currentstroke}%
\pgfsetstrokeopacity{0.000000}%
\pgfsetdash{}{0pt}%
\pgfpathmoveto{\pgfqpoint{3.303163in}{0.528000in}}%
\pgfpathlineto{\pgfqpoint{3.334047in}{0.528000in}}%
\pgfpathlineto{\pgfqpoint{3.334047in}{4.224000in}}%
\pgfpathlineto{\pgfqpoint{3.303163in}{4.224000in}}%
\pgfpathlineto{\pgfqpoint{3.303163in}{0.528000in}}%
\pgfpathclose%
\pgfusepath{fill}%
\end{pgfscope}%
\begin{pgfscope}%
\pgfpathrectangle{\pgfqpoint{0.800000in}{0.528000in}}{\pgfqpoint{4.960000in}{3.696000in}}%
\pgfusepath{clip}%
\pgfsetbuttcap%
\pgfsetmiterjoin%
\definecolor{currentfill}{rgb}{0.000000,0.000000,1.000000}%
\pgfsetfillcolor{currentfill}%
\pgfsetlinewidth{0.000000pt}%
\definecolor{currentstroke}{rgb}{0.000000,0.000000,0.000000}%
\pgfsetstrokecolor{currentstroke}%
\pgfsetstrokeopacity{0.000000}%
\pgfsetdash{}{0pt}%
\pgfpathmoveto{\pgfqpoint{3.341768in}{0.528000in}}%
\pgfpathlineto{\pgfqpoint{3.372653in}{0.528000in}}%
\pgfpathlineto{\pgfqpoint{3.372653in}{4.224000in}}%
\pgfpathlineto{\pgfqpoint{3.341768in}{4.224000in}}%
\pgfpathlineto{\pgfqpoint{3.341768in}{0.528000in}}%
\pgfpathclose%
\pgfusepath{fill}%
\end{pgfscope}%
\begin{pgfscope}%
\pgfpathrectangle{\pgfqpoint{0.800000in}{0.528000in}}{\pgfqpoint{4.960000in}{3.696000in}}%
\pgfusepath{clip}%
\pgfsetbuttcap%
\pgfsetmiterjoin%
\definecolor{currentfill}{rgb}{0.000000,0.000000,1.000000}%
\pgfsetfillcolor{currentfill}%
\pgfsetlinewidth{0.000000pt}%
\definecolor{currentstroke}{rgb}{0.000000,0.000000,0.000000}%
\pgfsetstrokecolor{currentstroke}%
\pgfsetstrokeopacity{0.000000}%
\pgfsetdash{}{0pt}%
\pgfpathmoveto{\pgfqpoint{3.380374in}{0.528000in}}%
\pgfpathlineto{\pgfqpoint{3.411258in}{0.528000in}}%
\pgfpathlineto{\pgfqpoint{3.411258in}{4.224000in}}%
\pgfpathlineto{\pgfqpoint{3.380374in}{4.224000in}}%
\pgfpathlineto{\pgfqpoint{3.380374in}{0.528000in}}%
\pgfpathclose%
\pgfusepath{fill}%
\end{pgfscope}%
\begin{pgfscope}%
\pgfpathrectangle{\pgfqpoint{0.800000in}{0.528000in}}{\pgfqpoint{4.960000in}{3.696000in}}%
\pgfusepath{clip}%
\pgfsetbuttcap%
\pgfsetmiterjoin%
\definecolor{currentfill}{rgb}{0.000000,0.000000,1.000000}%
\pgfsetfillcolor{currentfill}%
\pgfsetlinewidth{0.000000pt}%
\definecolor{currentstroke}{rgb}{0.000000,0.000000,0.000000}%
\pgfsetstrokecolor{currentstroke}%
\pgfsetstrokeopacity{0.000000}%
\pgfsetdash{}{0pt}%
\pgfpathmoveto{\pgfqpoint{3.418979in}{0.528000in}}%
\pgfpathlineto{\pgfqpoint{3.449863in}{0.528000in}}%
\pgfpathlineto{\pgfqpoint{3.449863in}{4.224000in}}%
\pgfpathlineto{\pgfqpoint{3.418979in}{4.224000in}}%
\pgfpathlineto{\pgfqpoint{3.418979in}{0.528000in}}%
\pgfpathclose%
\pgfusepath{fill}%
\end{pgfscope}%
\begin{pgfscope}%
\pgfpathrectangle{\pgfqpoint{0.800000in}{0.528000in}}{\pgfqpoint{4.960000in}{3.696000in}}%
\pgfusepath{clip}%
\pgfsetbuttcap%
\pgfsetmiterjoin%
\definecolor{currentfill}{rgb}{0.000000,0.000000,1.000000}%
\pgfsetfillcolor{currentfill}%
\pgfsetlinewidth{0.000000pt}%
\definecolor{currentstroke}{rgb}{0.000000,0.000000,0.000000}%
\pgfsetstrokecolor{currentstroke}%
\pgfsetstrokeopacity{0.000000}%
\pgfsetdash{}{0pt}%
\pgfpathmoveto{\pgfqpoint{3.457584in}{0.528000in}}%
\pgfpathlineto{\pgfqpoint{3.488468in}{0.528000in}}%
\pgfpathlineto{\pgfqpoint{3.488468in}{4.224000in}}%
\pgfpathlineto{\pgfqpoint{3.457584in}{4.224000in}}%
\pgfpathlineto{\pgfqpoint{3.457584in}{0.528000in}}%
\pgfpathclose%
\pgfusepath{fill}%
\end{pgfscope}%
\begin{pgfscope}%
\pgfpathrectangle{\pgfqpoint{0.800000in}{0.528000in}}{\pgfqpoint{4.960000in}{3.696000in}}%
\pgfusepath{clip}%
\pgfsetbuttcap%
\pgfsetmiterjoin%
\definecolor{currentfill}{rgb}{0.000000,0.000000,1.000000}%
\pgfsetfillcolor{currentfill}%
\pgfsetlinewidth{0.000000pt}%
\definecolor{currentstroke}{rgb}{0.000000,0.000000,0.000000}%
\pgfsetstrokecolor{currentstroke}%
\pgfsetstrokeopacity{0.000000}%
\pgfsetdash{}{0pt}%
\pgfpathmoveto{\pgfqpoint{3.496189in}{0.528000in}}%
\pgfpathlineto{\pgfqpoint{3.527073in}{0.528000in}}%
\pgfpathlineto{\pgfqpoint{3.527073in}{4.224000in}}%
\pgfpathlineto{\pgfqpoint{3.496189in}{4.224000in}}%
\pgfpathlineto{\pgfqpoint{3.496189in}{0.528000in}}%
\pgfpathclose%
\pgfusepath{fill}%
\end{pgfscope}%
\begin{pgfscope}%
\pgfpathrectangle{\pgfqpoint{0.800000in}{0.528000in}}{\pgfqpoint{4.960000in}{3.696000in}}%
\pgfusepath{clip}%
\pgfsetbuttcap%
\pgfsetmiterjoin%
\definecolor{currentfill}{rgb}{0.000000,0.000000,1.000000}%
\pgfsetfillcolor{currentfill}%
\pgfsetlinewidth{0.000000pt}%
\definecolor{currentstroke}{rgb}{0.000000,0.000000,0.000000}%
\pgfsetstrokecolor{currentstroke}%
\pgfsetstrokeopacity{0.000000}%
\pgfsetdash{}{0pt}%
\pgfpathmoveto{\pgfqpoint{3.534795in}{0.528000in}}%
\pgfpathlineto{\pgfqpoint{3.565679in}{0.528000in}}%
\pgfpathlineto{\pgfqpoint{3.565679in}{4.224000in}}%
\pgfpathlineto{\pgfqpoint{3.534795in}{4.224000in}}%
\pgfpathlineto{\pgfqpoint{3.534795in}{0.528000in}}%
\pgfpathclose%
\pgfusepath{fill}%
\end{pgfscope}%
\begin{pgfscope}%
\pgfpathrectangle{\pgfqpoint{0.800000in}{0.528000in}}{\pgfqpoint{4.960000in}{3.696000in}}%
\pgfusepath{clip}%
\pgfsetbuttcap%
\pgfsetmiterjoin%
\definecolor{currentfill}{rgb}{0.000000,0.000000,1.000000}%
\pgfsetfillcolor{currentfill}%
\pgfsetlinewidth{0.000000pt}%
\definecolor{currentstroke}{rgb}{0.000000,0.000000,0.000000}%
\pgfsetstrokecolor{currentstroke}%
\pgfsetstrokeopacity{0.000000}%
\pgfsetdash{}{0pt}%
\pgfpathmoveto{\pgfqpoint{3.573400in}{0.528000in}}%
\pgfpathlineto{\pgfqpoint{3.604284in}{0.528000in}}%
\pgfpathlineto{\pgfqpoint{3.604284in}{4.224000in}}%
\pgfpathlineto{\pgfqpoint{3.573400in}{4.224000in}}%
\pgfpathlineto{\pgfqpoint{3.573400in}{0.528000in}}%
\pgfpathclose%
\pgfusepath{fill}%
\end{pgfscope}%
\begin{pgfscope}%
\pgfpathrectangle{\pgfqpoint{0.800000in}{0.528000in}}{\pgfqpoint{4.960000in}{3.696000in}}%
\pgfusepath{clip}%
\pgfsetbuttcap%
\pgfsetmiterjoin%
\definecolor{currentfill}{rgb}{0.000000,0.000000,1.000000}%
\pgfsetfillcolor{currentfill}%
\pgfsetlinewidth{0.000000pt}%
\definecolor{currentstroke}{rgb}{0.000000,0.000000,0.000000}%
\pgfsetstrokecolor{currentstroke}%
\pgfsetstrokeopacity{0.000000}%
\pgfsetdash{}{0pt}%
\pgfpathmoveto{\pgfqpoint{3.612005in}{0.528000in}}%
\pgfpathlineto{\pgfqpoint{3.642889in}{0.528000in}}%
\pgfpathlineto{\pgfqpoint{3.642889in}{4.224000in}}%
\pgfpathlineto{\pgfqpoint{3.612005in}{4.224000in}}%
\pgfpathlineto{\pgfqpoint{3.612005in}{0.528000in}}%
\pgfpathclose%
\pgfusepath{fill}%
\end{pgfscope}%
\begin{pgfscope}%
\pgfpathrectangle{\pgfqpoint{0.800000in}{0.528000in}}{\pgfqpoint{4.960000in}{3.696000in}}%
\pgfusepath{clip}%
\pgfsetbuttcap%
\pgfsetmiterjoin%
\definecolor{currentfill}{rgb}{0.000000,0.000000,1.000000}%
\pgfsetfillcolor{currentfill}%
\pgfsetlinewidth{0.000000pt}%
\definecolor{currentstroke}{rgb}{0.000000,0.000000,0.000000}%
\pgfsetstrokecolor{currentstroke}%
\pgfsetstrokeopacity{0.000000}%
\pgfsetdash{}{0pt}%
\pgfpathmoveto{\pgfqpoint{3.650610in}{0.528000in}}%
\pgfpathlineto{\pgfqpoint{3.681494in}{0.528000in}}%
\pgfpathlineto{\pgfqpoint{3.681494in}{4.224000in}}%
\pgfpathlineto{\pgfqpoint{3.650610in}{4.224000in}}%
\pgfpathlineto{\pgfqpoint{3.650610in}{0.528000in}}%
\pgfpathclose%
\pgfusepath{fill}%
\end{pgfscope}%
\begin{pgfscope}%
\pgfpathrectangle{\pgfqpoint{0.800000in}{0.528000in}}{\pgfqpoint{4.960000in}{3.696000in}}%
\pgfusepath{clip}%
\pgfsetbuttcap%
\pgfsetmiterjoin%
\definecolor{currentfill}{rgb}{0.000000,0.000000,1.000000}%
\pgfsetfillcolor{currentfill}%
\pgfsetlinewidth{0.000000pt}%
\definecolor{currentstroke}{rgb}{0.000000,0.000000,0.000000}%
\pgfsetstrokecolor{currentstroke}%
\pgfsetstrokeopacity{0.000000}%
\pgfsetdash{}{0pt}%
\pgfpathmoveto{\pgfqpoint{3.689215in}{0.528000in}}%
\pgfpathlineto{\pgfqpoint{3.720100in}{0.528000in}}%
\pgfpathlineto{\pgfqpoint{3.720100in}{4.224000in}}%
\pgfpathlineto{\pgfqpoint{3.689215in}{4.224000in}}%
\pgfpathlineto{\pgfqpoint{3.689215in}{0.528000in}}%
\pgfpathclose%
\pgfusepath{fill}%
\end{pgfscope}%
\begin{pgfscope}%
\pgfpathrectangle{\pgfqpoint{0.800000in}{0.528000in}}{\pgfqpoint{4.960000in}{3.696000in}}%
\pgfusepath{clip}%
\pgfsetbuttcap%
\pgfsetmiterjoin%
\definecolor{currentfill}{rgb}{0.000000,0.000000,1.000000}%
\pgfsetfillcolor{currentfill}%
\pgfsetlinewidth{0.000000pt}%
\definecolor{currentstroke}{rgb}{0.000000,0.000000,0.000000}%
\pgfsetstrokecolor{currentstroke}%
\pgfsetstrokeopacity{0.000000}%
\pgfsetdash{}{0pt}%
\pgfpathmoveto{\pgfqpoint{3.727821in}{0.528000in}}%
\pgfpathlineto{\pgfqpoint{3.758705in}{0.528000in}}%
\pgfpathlineto{\pgfqpoint{3.758705in}{4.224000in}}%
\pgfpathlineto{\pgfqpoint{3.727821in}{4.224000in}}%
\pgfpathlineto{\pgfqpoint{3.727821in}{0.528000in}}%
\pgfpathclose%
\pgfusepath{fill}%
\end{pgfscope}%
\begin{pgfscope}%
\pgfpathrectangle{\pgfqpoint{0.800000in}{0.528000in}}{\pgfqpoint{4.960000in}{3.696000in}}%
\pgfusepath{clip}%
\pgfsetbuttcap%
\pgfsetmiterjoin%
\definecolor{currentfill}{rgb}{0.000000,0.000000,1.000000}%
\pgfsetfillcolor{currentfill}%
\pgfsetlinewidth{0.000000pt}%
\definecolor{currentstroke}{rgb}{0.000000,0.000000,0.000000}%
\pgfsetstrokecolor{currentstroke}%
\pgfsetstrokeopacity{0.000000}%
\pgfsetdash{}{0pt}%
\pgfpathmoveto{\pgfqpoint{3.766426in}{0.528000in}}%
\pgfpathlineto{\pgfqpoint{3.797310in}{0.528000in}}%
\pgfpathlineto{\pgfqpoint{3.797310in}{4.224000in}}%
\pgfpathlineto{\pgfqpoint{3.766426in}{4.224000in}}%
\pgfpathlineto{\pgfqpoint{3.766426in}{0.528000in}}%
\pgfpathclose%
\pgfusepath{fill}%
\end{pgfscope}%
\begin{pgfscope}%
\pgfpathrectangle{\pgfqpoint{0.800000in}{0.528000in}}{\pgfqpoint{4.960000in}{3.696000in}}%
\pgfusepath{clip}%
\pgfsetbuttcap%
\pgfsetmiterjoin%
\definecolor{currentfill}{rgb}{0.000000,0.000000,1.000000}%
\pgfsetfillcolor{currentfill}%
\pgfsetlinewidth{0.000000pt}%
\definecolor{currentstroke}{rgb}{0.000000,0.000000,0.000000}%
\pgfsetstrokecolor{currentstroke}%
\pgfsetstrokeopacity{0.000000}%
\pgfsetdash{}{0pt}%
\pgfpathmoveto{\pgfqpoint{3.805031in}{0.528000in}}%
\pgfpathlineto{\pgfqpoint{3.835915in}{0.528000in}}%
\pgfpathlineto{\pgfqpoint{3.835915in}{4.224000in}}%
\pgfpathlineto{\pgfqpoint{3.805031in}{4.224000in}}%
\pgfpathlineto{\pgfqpoint{3.805031in}{0.528000in}}%
\pgfpathclose%
\pgfusepath{fill}%
\end{pgfscope}%
\begin{pgfscope}%
\pgfpathrectangle{\pgfqpoint{0.800000in}{0.528000in}}{\pgfqpoint{4.960000in}{3.696000in}}%
\pgfusepath{clip}%
\pgfsetbuttcap%
\pgfsetmiterjoin%
\definecolor{currentfill}{rgb}{0.000000,0.000000,1.000000}%
\pgfsetfillcolor{currentfill}%
\pgfsetlinewidth{0.000000pt}%
\definecolor{currentstroke}{rgb}{0.000000,0.000000,0.000000}%
\pgfsetstrokecolor{currentstroke}%
\pgfsetstrokeopacity{0.000000}%
\pgfsetdash{}{0pt}%
\pgfpathmoveto{\pgfqpoint{3.843636in}{0.528000in}}%
\pgfpathlineto{\pgfqpoint{3.874521in}{0.528000in}}%
\pgfpathlineto{\pgfqpoint{3.874521in}{4.224000in}}%
\pgfpathlineto{\pgfqpoint{3.843636in}{4.224000in}}%
\pgfpathlineto{\pgfqpoint{3.843636in}{0.528000in}}%
\pgfpathclose%
\pgfusepath{fill}%
\end{pgfscope}%
\begin{pgfscope}%
\pgfpathrectangle{\pgfqpoint{0.800000in}{0.528000in}}{\pgfqpoint{4.960000in}{3.696000in}}%
\pgfusepath{clip}%
\pgfsetbuttcap%
\pgfsetmiterjoin%
\definecolor{currentfill}{rgb}{0.000000,0.000000,1.000000}%
\pgfsetfillcolor{currentfill}%
\pgfsetlinewidth{0.000000pt}%
\definecolor{currentstroke}{rgb}{0.000000,0.000000,0.000000}%
\pgfsetstrokecolor{currentstroke}%
\pgfsetstrokeopacity{0.000000}%
\pgfsetdash{}{0pt}%
\pgfpathmoveto{\pgfqpoint{3.882242in}{0.528000in}}%
\pgfpathlineto{\pgfqpoint{3.913126in}{0.528000in}}%
\pgfpathlineto{\pgfqpoint{3.913126in}{4.224000in}}%
\pgfpathlineto{\pgfqpoint{3.882242in}{4.224000in}}%
\pgfpathlineto{\pgfqpoint{3.882242in}{0.528000in}}%
\pgfpathclose%
\pgfusepath{fill}%
\end{pgfscope}%
\begin{pgfscope}%
\pgfpathrectangle{\pgfqpoint{0.800000in}{0.528000in}}{\pgfqpoint{4.960000in}{3.696000in}}%
\pgfusepath{clip}%
\pgfsetbuttcap%
\pgfsetmiterjoin%
\definecolor{currentfill}{rgb}{0.000000,0.000000,1.000000}%
\pgfsetfillcolor{currentfill}%
\pgfsetlinewidth{0.000000pt}%
\definecolor{currentstroke}{rgb}{0.000000,0.000000,0.000000}%
\pgfsetstrokecolor{currentstroke}%
\pgfsetstrokeopacity{0.000000}%
\pgfsetdash{}{0pt}%
\pgfpathmoveto{\pgfqpoint{3.920847in}{0.528000in}}%
\pgfpathlineto{\pgfqpoint{3.951731in}{0.528000in}}%
\pgfpathlineto{\pgfqpoint{3.951731in}{4.224000in}}%
\pgfpathlineto{\pgfqpoint{3.920847in}{4.224000in}}%
\pgfpathlineto{\pgfqpoint{3.920847in}{0.528000in}}%
\pgfpathclose%
\pgfusepath{fill}%
\end{pgfscope}%
\begin{pgfscope}%
\pgfpathrectangle{\pgfqpoint{0.800000in}{0.528000in}}{\pgfqpoint{4.960000in}{3.696000in}}%
\pgfusepath{clip}%
\pgfsetbuttcap%
\pgfsetmiterjoin%
\definecolor{currentfill}{rgb}{0.000000,0.000000,1.000000}%
\pgfsetfillcolor{currentfill}%
\pgfsetlinewidth{0.000000pt}%
\definecolor{currentstroke}{rgb}{0.000000,0.000000,0.000000}%
\pgfsetstrokecolor{currentstroke}%
\pgfsetstrokeopacity{0.000000}%
\pgfsetdash{}{0pt}%
\pgfpathmoveto{\pgfqpoint{3.959452in}{0.528000in}}%
\pgfpathlineto{\pgfqpoint{3.990336in}{0.528000in}}%
\pgfpathlineto{\pgfqpoint{3.990336in}{4.224000in}}%
\pgfpathlineto{\pgfqpoint{3.959452in}{4.224000in}}%
\pgfpathlineto{\pgfqpoint{3.959452in}{0.528000in}}%
\pgfpathclose%
\pgfusepath{fill}%
\end{pgfscope}%
\begin{pgfscope}%
\pgfpathrectangle{\pgfqpoint{0.800000in}{0.528000in}}{\pgfqpoint{4.960000in}{3.696000in}}%
\pgfusepath{clip}%
\pgfsetbuttcap%
\pgfsetmiterjoin%
\definecolor{currentfill}{rgb}{0.000000,0.000000,1.000000}%
\pgfsetfillcolor{currentfill}%
\pgfsetlinewidth{0.000000pt}%
\definecolor{currentstroke}{rgb}{0.000000,0.000000,0.000000}%
\pgfsetstrokecolor{currentstroke}%
\pgfsetstrokeopacity{0.000000}%
\pgfsetdash{}{0pt}%
\pgfpathmoveto{\pgfqpoint{3.998057in}{0.528000in}}%
\pgfpathlineto{\pgfqpoint{4.028941in}{0.528000in}}%
\pgfpathlineto{\pgfqpoint{4.028941in}{4.224000in}}%
\pgfpathlineto{\pgfqpoint{3.998057in}{4.224000in}}%
\pgfpathlineto{\pgfqpoint{3.998057in}{0.528000in}}%
\pgfpathclose%
\pgfusepath{fill}%
\end{pgfscope}%
\begin{pgfscope}%
\pgfpathrectangle{\pgfqpoint{0.800000in}{0.528000in}}{\pgfqpoint{4.960000in}{3.696000in}}%
\pgfusepath{clip}%
\pgfsetbuttcap%
\pgfsetmiterjoin%
\definecolor{currentfill}{rgb}{0.000000,0.000000,1.000000}%
\pgfsetfillcolor{currentfill}%
\pgfsetlinewidth{0.000000pt}%
\definecolor{currentstroke}{rgb}{0.000000,0.000000,0.000000}%
\pgfsetstrokecolor{currentstroke}%
\pgfsetstrokeopacity{0.000000}%
\pgfsetdash{}{0pt}%
\pgfpathmoveto{\pgfqpoint{4.036663in}{0.528000in}}%
\pgfpathlineto{\pgfqpoint{4.067547in}{0.528000in}}%
\pgfpathlineto{\pgfqpoint{4.067547in}{4.224000in}}%
\pgfpathlineto{\pgfqpoint{4.036663in}{4.224000in}}%
\pgfpathlineto{\pgfqpoint{4.036663in}{0.528000in}}%
\pgfpathclose%
\pgfusepath{fill}%
\end{pgfscope}%
\begin{pgfscope}%
\pgfpathrectangle{\pgfqpoint{0.800000in}{0.528000in}}{\pgfqpoint{4.960000in}{3.696000in}}%
\pgfusepath{clip}%
\pgfsetbuttcap%
\pgfsetmiterjoin%
\definecolor{currentfill}{rgb}{0.000000,0.000000,1.000000}%
\pgfsetfillcolor{currentfill}%
\pgfsetlinewidth{0.000000pt}%
\definecolor{currentstroke}{rgb}{0.000000,0.000000,0.000000}%
\pgfsetstrokecolor{currentstroke}%
\pgfsetstrokeopacity{0.000000}%
\pgfsetdash{}{0pt}%
\pgfpathmoveto{\pgfqpoint{4.075268in}{0.528000in}}%
\pgfpathlineto{\pgfqpoint{4.106152in}{0.528000in}}%
\pgfpathlineto{\pgfqpoint{4.106152in}{4.224000in}}%
\pgfpathlineto{\pgfqpoint{4.075268in}{4.224000in}}%
\pgfpathlineto{\pgfqpoint{4.075268in}{0.528000in}}%
\pgfpathclose%
\pgfusepath{fill}%
\end{pgfscope}%
\begin{pgfscope}%
\pgfpathrectangle{\pgfqpoint{0.800000in}{0.528000in}}{\pgfqpoint{4.960000in}{3.696000in}}%
\pgfusepath{clip}%
\pgfsetbuttcap%
\pgfsetmiterjoin%
\definecolor{currentfill}{rgb}{0.000000,0.000000,1.000000}%
\pgfsetfillcolor{currentfill}%
\pgfsetlinewidth{0.000000pt}%
\definecolor{currentstroke}{rgb}{0.000000,0.000000,0.000000}%
\pgfsetstrokecolor{currentstroke}%
\pgfsetstrokeopacity{0.000000}%
\pgfsetdash{}{0pt}%
\pgfpathmoveto{\pgfqpoint{4.113873in}{0.528000in}}%
\pgfpathlineto{\pgfqpoint{4.144757in}{0.528000in}}%
\pgfpathlineto{\pgfqpoint{4.144757in}{4.224000in}}%
\pgfpathlineto{\pgfqpoint{4.113873in}{4.224000in}}%
\pgfpathlineto{\pgfqpoint{4.113873in}{0.528000in}}%
\pgfpathclose%
\pgfusepath{fill}%
\end{pgfscope}%
\begin{pgfscope}%
\pgfpathrectangle{\pgfqpoint{0.800000in}{0.528000in}}{\pgfqpoint{4.960000in}{3.696000in}}%
\pgfusepath{clip}%
\pgfsetbuttcap%
\pgfsetmiterjoin%
\definecolor{currentfill}{rgb}{0.000000,0.000000,1.000000}%
\pgfsetfillcolor{currentfill}%
\pgfsetlinewidth{0.000000pt}%
\definecolor{currentstroke}{rgb}{0.000000,0.000000,0.000000}%
\pgfsetstrokecolor{currentstroke}%
\pgfsetstrokeopacity{0.000000}%
\pgfsetdash{}{0pt}%
\pgfpathmoveto{\pgfqpoint{4.152478in}{0.528000in}}%
\pgfpathlineto{\pgfqpoint{4.183362in}{0.528000in}}%
\pgfpathlineto{\pgfqpoint{4.183362in}{4.224000in}}%
\pgfpathlineto{\pgfqpoint{4.152478in}{4.224000in}}%
\pgfpathlineto{\pgfqpoint{4.152478in}{0.528000in}}%
\pgfpathclose%
\pgfusepath{fill}%
\end{pgfscope}%
\begin{pgfscope}%
\pgfpathrectangle{\pgfqpoint{0.800000in}{0.528000in}}{\pgfqpoint{4.960000in}{3.696000in}}%
\pgfusepath{clip}%
\pgfsetbuttcap%
\pgfsetmiterjoin%
\definecolor{currentfill}{rgb}{0.000000,0.000000,1.000000}%
\pgfsetfillcolor{currentfill}%
\pgfsetlinewidth{0.000000pt}%
\definecolor{currentstroke}{rgb}{0.000000,0.000000,0.000000}%
\pgfsetstrokecolor{currentstroke}%
\pgfsetstrokeopacity{0.000000}%
\pgfsetdash{}{0pt}%
\pgfpathmoveto{\pgfqpoint{4.191083in}{0.528000in}}%
\pgfpathlineto{\pgfqpoint{4.221968in}{0.528000in}}%
\pgfpathlineto{\pgfqpoint{4.221968in}{4.224000in}}%
\pgfpathlineto{\pgfqpoint{4.191083in}{4.224000in}}%
\pgfpathlineto{\pgfqpoint{4.191083in}{0.528000in}}%
\pgfpathclose%
\pgfusepath{fill}%
\end{pgfscope}%
\begin{pgfscope}%
\pgfpathrectangle{\pgfqpoint{0.800000in}{0.528000in}}{\pgfqpoint{4.960000in}{3.696000in}}%
\pgfusepath{clip}%
\pgfsetbuttcap%
\pgfsetmiterjoin%
\definecolor{currentfill}{rgb}{0.000000,0.000000,1.000000}%
\pgfsetfillcolor{currentfill}%
\pgfsetlinewidth{0.000000pt}%
\definecolor{currentstroke}{rgb}{0.000000,0.000000,0.000000}%
\pgfsetstrokecolor{currentstroke}%
\pgfsetstrokeopacity{0.000000}%
\pgfsetdash{}{0pt}%
\pgfpathmoveto{\pgfqpoint{4.229689in}{0.528000in}}%
\pgfpathlineto{\pgfqpoint{4.260573in}{0.528000in}}%
\pgfpathlineto{\pgfqpoint{4.260573in}{4.224000in}}%
\pgfpathlineto{\pgfqpoint{4.229689in}{4.224000in}}%
\pgfpathlineto{\pgfqpoint{4.229689in}{0.528000in}}%
\pgfpathclose%
\pgfusepath{fill}%
\end{pgfscope}%
\begin{pgfscope}%
\pgfpathrectangle{\pgfqpoint{0.800000in}{0.528000in}}{\pgfqpoint{4.960000in}{3.696000in}}%
\pgfusepath{clip}%
\pgfsetbuttcap%
\pgfsetmiterjoin%
\definecolor{currentfill}{rgb}{0.000000,0.000000,1.000000}%
\pgfsetfillcolor{currentfill}%
\pgfsetlinewidth{0.000000pt}%
\definecolor{currentstroke}{rgb}{0.000000,0.000000,0.000000}%
\pgfsetstrokecolor{currentstroke}%
\pgfsetstrokeopacity{0.000000}%
\pgfsetdash{}{0pt}%
\pgfpathmoveto{\pgfqpoint{4.268294in}{0.528000in}}%
\pgfpathlineto{\pgfqpoint{4.299178in}{0.528000in}}%
\pgfpathlineto{\pgfqpoint{4.299178in}{4.224000in}}%
\pgfpathlineto{\pgfqpoint{4.268294in}{4.224000in}}%
\pgfpathlineto{\pgfqpoint{4.268294in}{0.528000in}}%
\pgfpathclose%
\pgfusepath{fill}%
\end{pgfscope}%
\begin{pgfscope}%
\pgfpathrectangle{\pgfqpoint{0.800000in}{0.528000in}}{\pgfqpoint{4.960000in}{3.696000in}}%
\pgfusepath{clip}%
\pgfsetbuttcap%
\pgfsetmiterjoin%
\definecolor{currentfill}{rgb}{0.000000,0.000000,1.000000}%
\pgfsetfillcolor{currentfill}%
\pgfsetlinewidth{0.000000pt}%
\definecolor{currentstroke}{rgb}{0.000000,0.000000,0.000000}%
\pgfsetstrokecolor{currentstroke}%
\pgfsetstrokeopacity{0.000000}%
\pgfsetdash{}{0pt}%
\pgfpathmoveto{\pgfqpoint{4.306899in}{0.528000in}}%
\pgfpathlineto{\pgfqpoint{4.337783in}{0.528000in}}%
\pgfpathlineto{\pgfqpoint{4.337783in}{4.224000in}}%
\pgfpathlineto{\pgfqpoint{4.306899in}{4.224000in}}%
\pgfpathlineto{\pgfqpoint{4.306899in}{0.528000in}}%
\pgfpathclose%
\pgfusepath{fill}%
\end{pgfscope}%
\begin{pgfscope}%
\pgfpathrectangle{\pgfqpoint{0.800000in}{0.528000in}}{\pgfqpoint{4.960000in}{3.696000in}}%
\pgfusepath{clip}%
\pgfsetbuttcap%
\pgfsetmiterjoin%
\definecolor{currentfill}{rgb}{0.000000,0.000000,1.000000}%
\pgfsetfillcolor{currentfill}%
\pgfsetlinewidth{0.000000pt}%
\definecolor{currentstroke}{rgb}{0.000000,0.000000,0.000000}%
\pgfsetstrokecolor{currentstroke}%
\pgfsetstrokeopacity{0.000000}%
\pgfsetdash{}{0pt}%
\pgfpathmoveto{\pgfqpoint{4.345504in}{0.528000in}}%
\pgfpathlineto{\pgfqpoint{4.376389in}{0.528000in}}%
\pgfpathlineto{\pgfqpoint{4.376389in}{4.224000in}}%
\pgfpathlineto{\pgfqpoint{4.345504in}{4.224000in}}%
\pgfpathlineto{\pgfqpoint{4.345504in}{0.528000in}}%
\pgfpathclose%
\pgfusepath{fill}%
\end{pgfscope}%
\begin{pgfscope}%
\pgfpathrectangle{\pgfqpoint{0.800000in}{0.528000in}}{\pgfqpoint{4.960000in}{3.696000in}}%
\pgfusepath{clip}%
\pgfsetbuttcap%
\pgfsetmiterjoin%
\definecolor{currentfill}{rgb}{0.000000,0.000000,1.000000}%
\pgfsetfillcolor{currentfill}%
\pgfsetlinewidth{0.000000pt}%
\definecolor{currentstroke}{rgb}{0.000000,0.000000,0.000000}%
\pgfsetstrokecolor{currentstroke}%
\pgfsetstrokeopacity{0.000000}%
\pgfsetdash{}{0pt}%
\pgfpathmoveto{\pgfqpoint{4.384110in}{0.528000in}}%
\pgfpathlineto{\pgfqpoint{4.414994in}{0.528000in}}%
\pgfpathlineto{\pgfqpoint{4.414994in}{4.224000in}}%
\pgfpathlineto{\pgfqpoint{4.384110in}{4.224000in}}%
\pgfpathlineto{\pgfqpoint{4.384110in}{0.528000in}}%
\pgfpathclose%
\pgfusepath{fill}%
\end{pgfscope}%
\begin{pgfscope}%
\pgfpathrectangle{\pgfqpoint{0.800000in}{0.528000in}}{\pgfqpoint{4.960000in}{3.696000in}}%
\pgfusepath{clip}%
\pgfsetbuttcap%
\pgfsetmiterjoin%
\definecolor{currentfill}{rgb}{0.000000,0.000000,1.000000}%
\pgfsetfillcolor{currentfill}%
\pgfsetlinewidth{0.000000pt}%
\definecolor{currentstroke}{rgb}{0.000000,0.000000,0.000000}%
\pgfsetstrokecolor{currentstroke}%
\pgfsetstrokeopacity{0.000000}%
\pgfsetdash{}{0pt}%
\pgfpathmoveto{\pgfqpoint{4.422715in}{0.528000in}}%
\pgfpathlineto{\pgfqpoint{4.453599in}{0.528000in}}%
\pgfpathlineto{\pgfqpoint{4.453599in}{4.224000in}}%
\pgfpathlineto{\pgfqpoint{4.422715in}{4.224000in}}%
\pgfpathlineto{\pgfqpoint{4.422715in}{0.528000in}}%
\pgfpathclose%
\pgfusepath{fill}%
\end{pgfscope}%
\begin{pgfscope}%
\pgfpathrectangle{\pgfqpoint{0.800000in}{0.528000in}}{\pgfqpoint{4.960000in}{3.696000in}}%
\pgfusepath{clip}%
\pgfsetbuttcap%
\pgfsetmiterjoin%
\definecolor{currentfill}{rgb}{0.000000,0.000000,1.000000}%
\pgfsetfillcolor{currentfill}%
\pgfsetlinewidth{0.000000pt}%
\definecolor{currentstroke}{rgb}{0.000000,0.000000,0.000000}%
\pgfsetstrokecolor{currentstroke}%
\pgfsetstrokeopacity{0.000000}%
\pgfsetdash{}{0pt}%
\pgfpathmoveto{\pgfqpoint{4.461320in}{0.528000in}}%
\pgfpathlineto{\pgfqpoint{4.492204in}{0.528000in}}%
\pgfpathlineto{\pgfqpoint{4.492204in}{4.224000in}}%
\pgfpathlineto{\pgfqpoint{4.461320in}{4.224000in}}%
\pgfpathlineto{\pgfqpoint{4.461320in}{0.528000in}}%
\pgfpathclose%
\pgfusepath{fill}%
\end{pgfscope}%
\begin{pgfscope}%
\pgfpathrectangle{\pgfqpoint{0.800000in}{0.528000in}}{\pgfqpoint{4.960000in}{3.696000in}}%
\pgfusepath{clip}%
\pgfsetbuttcap%
\pgfsetmiterjoin%
\definecolor{currentfill}{rgb}{0.000000,0.000000,1.000000}%
\pgfsetfillcolor{currentfill}%
\pgfsetlinewidth{0.000000pt}%
\definecolor{currentstroke}{rgb}{0.000000,0.000000,0.000000}%
\pgfsetstrokecolor{currentstroke}%
\pgfsetstrokeopacity{0.000000}%
\pgfsetdash{}{0pt}%
\pgfpathmoveto{\pgfqpoint{4.499925in}{0.528000in}}%
\pgfpathlineto{\pgfqpoint{4.530809in}{0.528000in}}%
\pgfpathlineto{\pgfqpoint{4.530809in}{4.224000in}}%
\pgfpathlineto{\pgfqpoint{4.499925in}{4.224000in}}%
\pgfpathlineto{\pgfqpoint{4.499925in}{0.528000in}}%
\pgfpathclose%
\pgfusepath{fill}%
\end{pgfscope}%
\begin{pgfscope}%
\pgfpathrectangle{\pgfqpoint{0.800000in}{0.528000in}}{\pgfqpoint{4.960000in}{3.696000in}}%
\pgfusepath{clip}%
\pgfsetbuttcap%
\pgfsetmiterjoin%
\definecolor{currentfill}{rgb}{0.000000,0.000000,1.000000}%
\pgfsetfillcolor{currentfill}%
\pgfsetlinewidth{0.000000pt}%
\definecolor{currentstroke}{rgb}{0.000000,0.000000,0.000000}%
\pgfsetstrokecolor{currentstroke}%
\pgfsetstrokeopacity{0.000000}%
\pgfsetdash{}{0pt}%
\pgfpathmoveto{\pgfqpoint{4.538531in}{0.528000in}}%
\pgfpathlineto{\pgfqpoint{4.569415in}{0.528000in}}%
\pgfpathlineto{\pgfqpoint{4.569415in}{4.224000in}}%
\pgfpathlineto{\pgfqpoint{4.538531in}{4.224000in}}%
\pgfpathlineto{\pgfqpoint{4.538531in}{0.528000in}}%
\pgfpathclose%
\pgfusepath{fill}%
\end{pgfscope}%
\begin{pgfscope}%
\pgfpathrectangle{\pgfqpoint{0.800000in}{0.528000in}}{\pgfqpoint{4.960000in}{3.696000in}}%
\pgfusepath{clip}%
\pgfsetbuttcap%
\pgfsetmiterjoin%
\definecolor{currentfill}{rgb}{0.000000,0.000000,1.000000}%
\pgfsetfillcolor{currentfill}%
\pgfsetlinewidth{0.000000pt}%
\definecolor{currentstroke}{rgb}{0.000000,0.000000,0.000000}%
\pgfsetstrokecolor{currentstroke}%
\pgfsetstrokeopacity{0.000000}%
\pgfsetdash{}{0pt}%
\pgfpathmoveto{\pgfqpoint{4.577136in}{0.528000in}}%
\pgfpathlineto{\pgfqpoint{4.608020in}{0.528000in}}%
\pgfpathlineto{\pgfqpoint{4.608020in}{4.224000in}}%
\pgfpathlineto{\pgfqpoint{4.577136in}{4.224000in}}%
\pgfpathlineto{\pgfqpoint{4.577136in}{0.528000in}}%
\pgfpathclose%
\pgfusepath{fill}%
\end{pgfscope}%
\begin{pgfscope}%
\pgfpathrectangle{\pgfqpoint{0.800000in}{0.528000in}}{\pgfqpoint{4.960000in}{3.696000in}}%
\pgfusepath{clip}%
\pgfsetbuttcap%
\pgfsetmiterjoin%
\definecolor{currentfill}{rgb}{0.000000,0.000000,1.000000}%
\pgfsetfillcolor{currentfill}%
\pgfsetlinewidth{0.000000pt}%
\definecolor{currentstroke}{rgb}{0.000000,0.000000,0.000000}%
\pgfsetstrokecolor{currentstroke}%
\pgfsetstrokeopacity{0.000000}%
\pgfsetdash{}{0pt}%
\pgfpathmoveto{\pgfqpoint{4.615741in}{0.528000in}}%
\pgfpathlineto{\pgfqpoint{4.646625in}{0.528000in}}%
\pgfpathlineto{\pgfqpoint{4.646625in}{4.224000in}}%
\pgfpathlineto{\pgfqpoint{4.615741in}{4.224000in}}%
\pgfpathlineto{\pgfqpoint{4.615741in}{0.528000in}}%
\pgfpathclose%
\pgfusepath{fill}%
\end{pgfscope}%
\begin{pgfscope}%
\pgfpathrectangle{\pgfqpoint{0.800000in}{0.528000in}}{\pgfqpoint{4.960000in}{3.696000in}}%
\pgfusepath{clip}%
\pgfsetbuttcap%
\pgfsetmiterjoin%
\definecolor{currentfill}{rgb}{0.000000,0.000000,1.000000}%
\pgfsetfillcolor{currentfill}%
\pgfsetlinewidth{0.000000pt}%
\definecolor{currentstroke}{rgb}{0.000000,0.000000,0.000000}%
\pgfsetstrokecolor{currentstroke}%
\pgfsetstrokeopacity{0.000000}%
\pgfsetdash{}{0pt}%
\pgfpathmoveto{\pgfqpoint{4.654346in}{0.528000in}}%
\pgfpathlineto{\pgfqpoint{4.685230in}{0.528000in}}%
\pgfpathlineto{\pgfqpoint{4.685230in}{4.224000in}}%
\pgfpathlineto{\pgfqpoint{4.654346in}{4.224000in}}%
\pgfpathlineto{\pgfqpoint{4.654346in}{0.528000in}}%
\pgfpathclose%
\pgfusepath{fill}%
\end{pgfscope}%
\begin{pgfscope}%
\pgfpathrectangle{\pgfqpoint{0.800000in}{0.528000in}}{\pgfqpoint{4.960000in}{3.696000in}}%
\pgfusepath{clip}%
\pgfsetbuttcap%
\pgfsetmiterjoin%
\definecolor{currentfill}{rgb}{0.000000,0.000000,1.000000}%
\pgfsetfillcolor{currentfill}%
\pgfsetlinewidth{0.000000pt}%
\definecolor{currentstroke}{rgb}{0.000000,0.000000,0.000000}%
\pgfsetstrokecolor{currentstroke}%
\pgfsetstrokeopacity{0.000000}%
\pgfsetdash{}{0pt}%
\pgfpathmoveto{\pgfqpoint{4.692951in}{0.528000in}}%
\pgfpathlineto{\pgfqpoint{4.723836in}{0.528000in}}%
\pgfpathlineto{\pgfqpoint{4.723836in}{4.224000in}}%
\pgfpathlineto{\pgfqpoint{4.692951in}{4.224000in}}%
\pgfpathlineto{\pgfqpoint{4.692951in}{0.528000in}}%
\pgfpathclose%
\pgfusepath{fill}%
\end{pgfscope}%
\begin{pgfscope}%
\pgfpathrectangle{\pgfqpoint{0.800000in}{0.528000in}}{\pgfqpoint{4.960000in}{3.696000in}}%
\pgfusepath{clip}%
\pgfsetbuttcap%
\pgfsetmiterjoin%
\definecolor{currentfill}{rgb}{0.000000,0.000000,1.000000}%
\pgfsetfillcolor{currentfill}%
\pgfsetlinewidth{0.000000pt}%
\definecolor{currentstroke}{rgb}{0.000000,0.000000,0.000000}%
\pgfsetstrokecolor{currentstroke}%
\pgfsetstrokeopacity{0.000000}%
\pgfsetdash{}{0pt}%
\pgfpathmoveto{\pgfqpoint{4.731557in}{0.528000in}}%
\pgfpathlineto{\pgfqpoint{4.762441in}{0.528000in}}%
\pgfpathlineto{\pgfqpoint{4.762441in}{4.224000in}}%
\pgfpathlineto{\pgfqpoint{4.731557in}{4.224000in}}%
\pgfpathlineto{\pgfqpoint{4.731557in}{0.528000in}}%
\pgfpathclose%
\pgfusepath{fill}%
\end{pgfscope}%
\begin{pgfscope}%
\pgfpathrectangle{\pgfqpoint{0.800000in}{0.528000in}}{\pgfqpoint{4.960000in}{3.696000in}}%
\pgfusepath{clip}%
\pgfsetbuttcap%
\pgfsetmiterjoin%
\definecolor{currentfill}{rgb}{0.000000,0.000000,1.000000}%
\pgfsetfillcolor{currentfill}%
\pgfsetlinewidth{0.000000pt}%
\definecolor{currentstroke}{rgb}{0.000000,0.000000,0.000000}%
\pgfsetstrokecolor{currentstroke}%
\pgfsetstrokeopacity{0.000000}%
\pgfsetdash{}{0pt}%
\pgfpathmoveto{\pgfqpoint{4.770162in}{0.528000in}}%
\pgfpathlineto{\pgfqpoint{4.801046in}{0.528000in}}%
\pgfpathlineto{\pgfqpoint{4.801046in}{4.224000in}}%
\pgfpathlineto{\pgfqpoint{4.770162in}{4.224000in}}%
\pgfpathlineto{\pgfqpoint{4.770162in}{0.528000in}}%
\pgfpathclose%
\pgfusepath{fill}%
\end{pgfscope}%
\begin{pgfscope}%
\pgfpathrectangle{\pgfqpoint{0.800000in}{0.528000in}}{\pgfqpoint{4.960000in}{3.696000in}}%
\pgfusepath{clip}%
\pgfsetbuttcap%
\pgfsetmiterjoin%
\definecolor{currentfill}{rgb}{0.000000,0.000000,1.000000}%
\pgfsetfillcolor{currentfill}%
\pgfsetlinewidth{0.000000pt}%
\definecolor{currentstroke}{rgb}{0.000000,0.000000,0.000000}%
\pgfsetstrokecolor{currentstroke}%
\pgfsetstrokeopacity{0.000000}%
\pgfsetdash{}{0pt}%
\pgfpathmoveto{\pgfqpoint{4.808767in}{0.528000in}}%
\pgfpathlineto{\pgfqpoint{4.839651in}{0.528000in}}%
\pgfpathlineto{\pgfqpoint{4.839651in}{4.224000in}}%
\pgfpathlineto{\pgfqpoint{4.808767in}{4.224000in}}%
\pgfpathlineto{\pgfqpoint{4.808767in}{0.528000in}}%
\pgfpathclose%
\pgfusepath{fill}%
\end{pgfscope}%
\begin{pgfscope}%
\pgfpathrectangle{\pgfqpoint{0.800000in}{0.528000in}}{\pgfqpoint{4.960000in}{3.696000in}}%
\pgfusepath{clip}%
\pgfsetbuttcap%
\pgfsetmiterjoin%
\definecolor{currentfill}{rgb}{0.000000,0.000000,1.000000}%
\pgfsetfillcolor{currentfill}%
\pgfsetlinewidth{0.000000pt}%
\definecolor{currentstroke}{rgb}{0.000000,0.000000,0.000000}%
\pgfsetstrokecolor{currentstroke}%
\pgfsetstrokeopacity{0.000000}%
\pgfsetdash{}{0pt}%
\pgfpathmoveto{\pgfqpoint{4.847372in}{0.528000in}}%
\pgfpathlineto{\pgfqpoint{4.878257in}{0.528000in}}%
\pgfpathlineto{\pgfqpoint{4.878257in}{4.224000in}}%
\pgfpathlineto{\pgfqpoint{4.847372in}{4.224000in}}%
\pgfpathlineto{\pgfqpoint{4.847372in}{0.528000in}}%
\pgfpathclose%
\pgfusepath{fill}%
\end{pgfscope}%
\begin{pgfscope}%
\pgfpathrectangle{\pgfqpoint{0.800000in}{0.528000in}}{\pgfqpoint{4.960000in}{3.696000in}}%
\pgfusepath{clip}%
\pgfsetbuttcap%
\pgfsetmiterjoin%
\definecolor{currentfill}{rgb}{0.000000,0.000000,1.000000}%
\pgfsetfillcolor{currentfill}%
\pgfsetlinewidth{0.000000pt}%
\definecolor{currentstroke}{rgb}{0.000000,0.000000,0.000000}%
\pgfsetstrokecolor{currentstroke}%
\pgfsetstrokeopacity{0.000000}%
\pgfsetdash{}{0pt}%
\pgfpathmoveto{\pgfqpoint{4.885978in}{0.528000in}}%
\pgfpathlineto{\pgfqpoint{4.916862in}{0.528000in}}%
\pgfpathlineto{\pgfqpoint{4.916862in}{4.224000in}}%
\pgfpathlineto{\pgfqpoint{4.885978in}{4.224000in}}%
\pgfpathlineto{\pgfqpoint{4.885978in}{0.528000in}}%
\pgfpathclose%
\pgfusepath{fill}%
\end{pgfscope}%
\begin{pgfscope}%
\pgfpathrectangle{\pgfqpoint{0.800000in}{0.528000in}}{\pgfqpoint{4.960000in}{3.696000in}}%
\pgfusepath{clip}%
\pgfsetbuttcap%
\pgfsetmiterjoin%
\definecolor{currentfill}{rgb}{0.000000,0.000000,1.000000}%
\pgfsetfillcolor{currentfill}%
\pgfsetlinewidth{0.000000pt}%
\definecolor{currentstroke}{rgb}{0.000000,0.000000,0.000000}%
\pgfsetstrokecolor{currentstroke}%
\pgfsetstrokeopacity{0.000000}%
\pgfsetdash{}{0pt}%
\pgfpathmoveto{\pgfqpoint{4.924583in}{0.528000in}}%
\pgfpathlineto{\pgfqpoint{4.955467in}{0.528000in}}%
\pgfpathlineto{\pgfqpoint{4.955467in}{4.224000in}}%
\pgfpathlineto{\pgfqpoint{4.924583in}{4.224000in}}%
\pgfpathlineto{\pgfqpoint{4.924583in}{0.528000in}}%
\pgfpathclose%
\pgfusepath{fill}%
\end{pgfscope}%
\begin{pgfscope}%
\pgfpathrectangle{\pgfqpoint{0.800000in}{0.528000in}}{\pgfqpoint{4.960000in}{3.696000in}}%
\pgfusepath{clip}%
\pgfsetbuttcap%
\pgfsetmiterjoin%
\definecolor{currentfill}{rgb}{0.000000,0.000000,1.000000}%
\pgfsetfillcolor{currentfill}%
\pgfsetlinewidth{0.000000pt}%
\definecolor{currentstroke}{rgb}{0.000000,0.000000,0.000000}%
\pgfsetstrokecolor{currentstroke}%
\pgfsetstrokeopacity{0.000000}%
\pgfsetdash{}{0pt}%
\pgfpathmoveto{\pgfqpoint{4.963188in}{0.528000in}}%
\pgfpathlineto{\pgfqpoint{4.994072in}{0.528000in}}%
\pgfpathlineto{\pgfqpoint{4.994072in}{4.224000in}}%
\pgfpathlineto{\pgfqpoint{4.963188in}{4.224000in}}%
\pgfpathlineto{\pgfqpoint{4.963188in}{0.528000in}}%
\pgfpathclose%
\pgfusepath{fill}%
\end{pgfscope}%
\begin{pgfscope}%
\pgfpathrectangle{\pgfqpoint{0.800000in}{0.528000in}}{\pgfqpoint{4.960000in}{3.696000in}}%
\pgfusepath{clip}%
\pgfsetbuttcap%
\pgfsetmiterjoin%
\definecolor{currentfill}{rgb}{0.000000,0.000000,1.000000}%
\pgfsetfillcolor{currentfill}%
\pgfsetlinewidth{0.000000pt}%
\definecolor{currentstroke}{rgb}{0.000000,0.000000,0.000000}%
\pgfsetstrokecolor{currentstroke}%
\pgfsetstrokeopacity{0.000000}%
\pgfsetdash{}{0pt}%
\pgfpathmoveto{\pgfqpoint{5.001793in}{0.528000in}}%
\pgfpathlineto{\pgfqpoint{5.032677in}{0.528000in}}%
\pgfpathlineto{\pgfqpoint{5.032677in}{4.224000in}}%
\pgfpathlineto{\pgfqpoint{5.001793in}{4.224000in}}%
\pgfpathlineto{\pgfqpoint{5.001793in}{0.528000in}}%
\pgfpathclose%
\pgfusepath{fill}%
\end{pgfscope}%
\begin{pgfscope}%
\pgfpathrectangle{\pgfqpoint{0.800000in}{0.528000in}}{\pgfqpoint{4.960000in}{3.696000in}}%
\pgfusepath{clip}%
\pgfsetbuttcap%
\pgfsetmiterjoin%
\definecolor{currentfill}{rgb}{0.000000,0.000000,1.000000}%
\pgfsetfillcolor{currentfill}%
\pgfsetlinewidth{0.000000pt}%
\definecolor{currentstroke}{rgb}{0.000000,0.000000,0.000000}%
\pgfsetstrokecolor{currentstroke}%
\pgfsetstrokeopacity{0.000000}%
\pgfsetdash{}{0pt}%
\pgfpathmoveto{\pgfqpoint{5.040399in}{0.528000in}}%
\pgfpathlineto{\pgfqpoint{5.071283in}{0.528000in}}%
\pgfpathlineto{\pgfqpoint{5.071283in}{4.224000in}}%
\pgfpathlineto{\pgfqpoint{5.040399in}{4.224000in}}%
\pgfpathlineto{\pgfqpoint{5.040399in}{0.528000in}}%
\pgfpathclose%
\pgfusepath{fill}%
\end{pgfscope}%
\begin{pgfscope}%
\pgfpathrectangle{\pgfqpoint{0.800000in}{0.528000in}}{\pgfqpoint{4.960000in}{3.696000in}}%
\pgfusepath{clip}%
\pgfsetbuttcap%
\pgfsetmiterjoin%
\definecolor{currentfill}{rgb}{0.000000,0.000000,1.000000}%
\pgfsetfillcolor{currentfill}%
\pgfsetlinewidth{0.000000pt}%
\definecolor{currentstroke}{rgb}{0.000000,0.000000,0.000000}%
\pgfsetstrokecolor{currentstroke}%
\pgfsetstrokeopacity{0.000000}%
\pgfsetdash{}{0pt}%
\pgfpathmoveto{\pgfqpoint{5.079004in}{0.528000in}}%
\pgfpathlineto{\pgfqpoint{5.109888in}{0.528000in}}%
\pgfpathlineto{\pgfqpoint{5.109888in}{4.224000in}}%
\pgfpathlineto{\pgfqpoint{5.079004in}{4.224000in}}%
\pgfpathlineto{\pgfqpoint{5.079004in}{0.528000in}}%
\pgfpathclose%
\pgfusepath{fill}%
\end{pgfscope}%
\begin{pgfscope}%
\pgfpathrectangle{\pgfqpoint{0.800000in}{0.528000in}}{\pgfqpoint{4.960000in}{3.696000in}}%
\pgfusepath{clip}%
\pgfsetbuttcap%
\pgfsetmiterjoin%
\definecolor{currentfill}{rgb}{0.000000,0.000000,1.000000}%
\pgfsetfillcolor{currentfill}%
\pgfsetlinewidth{0.000000pt}%
\definecolor{currentstroke}{rgb}{0.000000,0.000000,0.000000}%
\pgfsetstrokecolor{currentstroke}%
\pgfsetstrokeopacity{0.000000}%
\pgfsetdash{}{0pt}%
\pgfpathmoveto{\pgfqpoint{5.117609in}{0.528000in}}%
\pgfpathlineto{\pgfqpoint{5.148493in}{0.528000in}}%
\pgfpathlineto{\pgfqpoint{5.148493in}{4.224000in}}%
\pgfpathlineto{\pgfqpoint{5.117609in}{4.224000in}}%
\pgfpathlineto{\pgfqpoint{5.117609in}{0.528000in}}%
\pgfpathclose%
\pgfusepath{fill}%
\end{pgfscope}%
\begin{pgfscope}%
\pgfpathrectangle{\pgfqpoint{0.800000in}{0.528000in}}{\pgfqpoint{4.960000in}{3.696000in}}%
\pgfusepath{clip}%
\pgfsetbuttcap%
\pgfsetmiterjoin%
\definecolor{currentfill}{rgb}{0.000000,0.000000,1.000000}%
\pgfsetfillcolor{currentfill}%
\pgfsetlinewidth{0.000000pt}%
\definecolor{currentstroke}{rgb}{0.000000,0.000000,0.000000}%
\pgfsetstrokecolor{currentstroke}%
\pgfsetstrokeopacity{0.000000}%
\pgfsetdash{}{0pt}%
\pgfpathmoveto{\pgfqpoint{5.156214in}{0.528000in}}%
\pgfpathlineto{\pgfqpoint{5.187098in}{0.528000in}}%
\pgfpathlineto{\pgfqpoint{5.187098in}{4.224000in}}%
\pgfpathlineto{\pgfqpoint{5.156214in}{4.224000in}}%
\pgfpathlineto{\pgfqpoint{5.156214in}{0.528000in}}%
\pgfpathclose%
\pgfusepath{fill}%
\end{pgfscope}%
\begin{pgfscope}%
\pgfpathrectangle{\pgfqpoint{0.800000in}{0.528000in}}{\pgfqpoint{4.960000in}{3.696000in}}%
\pgfusepath{clip}%
\pgfsetbuttcap%
\pgfsetmiterjoin%
\definecolor{currentfill}{rgb}{0.000000,0.000000,1.000000}%
\pgfsetfillcolor{currentfill}%
\pgfsetlinewidth{0.000000pt}%
\definecolor{currentstroke}{rgb}{0.000000,0.000000,0.000000}%
\pgfsetstrokecolor{currentstroke}%
\pgfsetstrokeopacity{0.000000}%
\pgfsetdash{}{0pt}%
\pgfpathmoveto{\pgfqpoint{5.194819in}{0.528000in}}%
\pgfpathlineto{\pgfqpoint{5.225704in}{0.528000in}}%
\pgfpathlineto{\pgfqpoint{5.225704in}{4.224000in}}%
\pgfpathlineto{\pgfqpoint{5.194819in}{4.224000in}}%
\pgfpathlineto{\pgfqpoint{5.194819in}{0.528000in}}%
\pgfpathclose%
\pgfusepath{fill}%
\end{pgfscope}%
\begin{pgfscope}%
\pgfpathrectangle{\pgfqpoint{0.800000in}{0.528000in}}{\pgfqpoint{4.960000in}{3.696000in}}%
\pgfusepath{clip}%
\pgfsetbuttcap%
\pgfsetmiterjoin%
\definecolor{currentfill}{rgb}{0.000000,0.000000,1.000000}%
\pgfsetfillcolor{currentfill}%
\pgfsetlinewidth{0.000000pt}%
\definecolor{currentstroke}{rgb}{0.000000,0.000000,0.000000}%
\pgfsetstrokecolor{currentstroke}%
\pgfsetstrokeopacity{0.000000}%
\pgfsetdash{}{0pt}%
\pgfpathmoveto{\pgfqpoint{5.233425in}{0.528000in}}%
\pgfpathlineto{\pgfqpoint{5.264309in}{0.528000in}}%
\pgfpathlineto{\pgfqpoint{5.264309in}{4.224000in}}%
\pgfpathlineto{\pgfqpoint{5.233425in}{4.224000in}}%
\pgfpathlineto{\pgfqpoint{5.233425in}{0.528000in}}%
\pgfpathclose%
\pgfusepath{fill}%
\end{pgfscope}%
\begin{pgfscope}%
\pgfpathrectangle{\pgfqpoint{0.800000in}{0.528000in}}{\pgfqpoint{4.960000in}{3.696000in}}%
\pgfusepath{clip}%
\pgfsetbuttcap%
\pgfsetmiterjoin%
\definecolor{currentfill}{rgb}{0.000000,0.000000,1.000000}%
\pgfsetfillcolor{currentfill}%
\pgfsetlinewidth{0.000000pt}%
\definecolor{currentstroke}{rgb}{0.000000,0.000000,0.000000}%
\pgfsetstrokecolor{currentstroke}%
\pgfsetstrokeopacity{0.000000}%
\pgfsetdash{}{0pt}%
\pgfpathmoveto{\pgfqpoint{5.272030in}{0.528000in}}%
\pgfpathlineto{\pgfqpoint{5.302914in}{0.528000in}}%
\pgfpathlineto{\pgfqpoint{5.302914in}{4.224000in}}%
\pgfpathlineto{\pgfqpoint{5.272030in}{4.224000in}}%
\pgfpathlineto{\pgfqpoint{5.272030in}{0.528000in}}%
\pgfpathclose%
\pgfusepath{fill}%
\end{pgfscope}%
\begin{pgfscope}%
\pgfpathrectangle{\pgfqpoint{0.800000in}{0.528000in}}{\pgfqpoint{4.960000in}{3.696000in}}%
\pgfusepath{clip}%
\pgfsetbuttcap%
\pgfsetmiterjoin%
\definecolor{currentfill}{rgb}{0.000000,0.000000,1.000000}%
\pgfsetfillcolor{currentfill}%
\pgfsetlinewidth{0.000000pt}%
\definecolor{currentstroke}{rgb}{0.000000,0.000000,0.000000}%
\pgfsetstrokecolor{currentstroke}%
\pgfsetstrokeopacity{0.000000}%
\pgfsetdash{}{0pt}%
\pgfpathmoveto{\pgfqpoint{5.310635in}{0.528000in}}%
\pgfpathlineto{\pgfqpoint{5.341519in}{0.528000in}}%
\pgfpathlineto{\pgfqpoint{5.341519in}{4.224000in}}%
\pgfpathlineto{\pgfqpoint{5.310635in}{4.224000in}}%
\pgfpathlineto{\pgfqpoint{5.310635in}{0.528000in}}%
\pgfpathclose%
\pgfusepath{fill}%
\end{pgfscope}%
\begin{pgfscope}%
\pgfpathrectangle{\pgfqpoint{0.800000in}{0.528000in}}{\pgfqpoint{4.960000in}{3.696000in}}%
\pgfusepath{clip}%
\pgfsetbuttcap%
\pgfsetmiterjoin%
\definecolor{currentfill}{rgb}{0.000000,0.000000,1.000000}%
\pgfsetfillcolor{currentfill}%
\pgfsetlinewidth{0.000000pt}%
\definecolor{currentstroke}{rgb}{0.000000,0.000000,0.000000}%
\pgfsetstrokecolor{currentstroke}%
\pgfsetstrokeopacity{0.000000}%
\pgfsetdash{}{0pt}%
\pgfpathmoveto{\pgfqpoint{5.349240in}{0.528000in}}%
\pgfpathlineto{\pgfqpoint{5.380125in}{0.528000in}}%
\pgfpathlineto{\pgfqpoint{5.380125in}{0.989993in}}%
\pgfpathlineto{\pgfqpoint{5.349240in}{0.989993in}}%
\pgfpathlineto{\pgfqpoint{5.349240in}{0.528000in}}%
\pgfpathclose%
\pgfusepath{fill}%
\end{pgfscope}%
\begin{pgfscope}%
\pgfpathrectangle{\pgfqpoint{0.800000in}{0.528000in}}{\pgfqpoint{4.960000in}{3.696000in}}%
\pgfusepath{clip}%
\pgfsetbuttcap%
\pgfsetmiterjoin%
\definecolor{currentfill}{rgb}{0.000000,0.000000,1.000000}%
\pgfsetfillcolor{currentfill}%
\pgfsetlinewidth{0.000000pt}%
\definecolor{currentstroke}{rgb}{0.000000,0.000000,0.000000}%
\pgfsetstrokecolor{currentstroke}%
\pgfsetstrokeopacity{0.000000}%
\pgfsetdash{}{0pt}%
\pgfpathmoveto{\pgfqpoint{5.387846in}{0.528000in}}%
\pgfpathlineto{\pgfqpoint{5.418730in}{0.528000in}}%
\pgfpathlineto{\pgfqpoint{5.418730in}{1.452000in}}%
\pgfpathlineto{\pgfqpoint{5.387846in}{1.452000in}}%
\pgfpathlineto{\pgfqpoint{5.387846in}{0.528000in}}%
\pgfpathclose%
\pgfusepath{fill}%
\end{pgfscope}%
\begin{pgfscope}%
\pgfpathrectangle{\pgfqpoint{0.800000in}{0.528000in}}{\pgfqpoint{4.960000in}{3.696000in}}%
\pgfusepath{clip}%
\pgfsetbuttcap%
\pgfsetmiterjoin%
\definecolor{currentfill}{rgb}{0.000000,0.000000,1.000000}%
\pgfsetfillcolor{currentfill}%
\pgfsetlinewidth{0.000000pt}%
\definecolor{currentstroke}{rgb}{0.000000,0.000000,0.000000}%
\pgfsetstrokecolor{currentstroke}%
\pgfsetstrokeopacity{0.000000}%
\pgfsetdash{}{0pt}%
\pgfpathmoveto{\pgfqpoint{5.426451in}{0.528000in}}%
\pgfpathlineto{\pgfqpoint{5.457335in}{0.528000in}}%
\pgfpathlineto{\pgfqpoint{5.457335in}{1.452041in}}%
\pgfpathlineto{\pgfqpoint{5.426451in}{1.452041in}}%
\pgfpathlineto{\pgfqpoint{5.426451in}{0.528000in}}%
\pgfpathclose%
\pgfusepath{fill}%
\end{pgfscope}%
\begin{pgfscope}%
\pgfpathrectangle{\pgfqpoint{0.800000in}{0.528000in}}{\pgfqpoint{4.960000in}{3.696000in}}%
\pgfusepath{clip}%
\pgfsetbuttcap%
\pgfsetmiterjoin%
\definecolor{currentfill}{rgb}{0.000000,0.000000,1.000000}%
\pgfsetfillcolor{currentfill}%
\pgfsetlinewidth{0.000000pt}%
\definecolor{currentstroke}{rgb}{0.000000,0.000000,0.000000}%
\pgfsetstrokecolor{currentstroke}%
\pgfsetstrokeopacity{0.000000}%
\pgfsetdash{}{0pt}%
\pgfpathmoveto{\pgfqpoint{5.465056in}{0.528000in}}%
\pgfpathlineto{\pgfqpoint{5.495940in}{0.528000in}}%
\pgfpathlineto{\pgfqpoint{5.495940in}{0.758993in}}%
\pgfpathlineto{\pgfqpoint{5.465056in}{0.758993in}}%
\pgfpathlineto{\pgfqpoint{5.465056in}{0.528000in}}%
\pgfpathclose%
\pgfusepath{fill}%
\end{pgfscope}%
\begin{pgfscope}%
\pgfpathrectangle{\pgfqpoint{0.800000in}{0.528000in}}{\pgfqpoint{4.960000in}{3.696000in}}%
\pgfusepath{clip}%
\pgfsetbuttcap%
\pgfsetmiterjoin%
\definecolor{currentfill}{rgb}{0.000000,0.000000,1.000000}%
\pgfsetfillcolor{currentfill}%
\pgfsetlinewidth{0.000000pt}%
\definecolor{currentstroke}{rgb}{0.000000,0.000000,0.000000}%
\pgfsetstrokecolor{currentstroke}%
\pgfsetstrokeopacity{0.000000}%
\pgfsetdash{}{0pt}%
\pgfpathmoveto{\pgfqpoint{5.503661in}{0.528000in}}%
\pgfpathlineto{\pgfqpoint{5.534545in}{0.528000in}}%
\pgfpathlineto{\pgfqpoint{5.534545in}{0.990000in}}%
\pgfpathlineto{\pgfqpoint{5.503661in}{0.990000in}}%
\pgfpathlineto{\pgfqpoint{5.503661in}{0.528000in}}%
\pgfpathclose%
\pgfusepath{fill}%
\end{pgfscope}%
\begin{pgfscope}%
\pgfsetbuttcap%
\pgfsetroundjoin%
\definecolor{currentfill}{rgb}{0.000000,0.000000,0.000000}%
\pgfsetfillcolor{currentfill}%
\pgfsetlinewidth{0.803000pt}%
\definecolor{currentstroke}{rgb}{0.000000,0.000000,0.000000}%
\pgfsetstrokecolor{currentstroke}%
\pgfsetdash{}{0pt}%
\pgfsys@defobject{currentmarker}{\pgfqpoint{0.000000in}{-0.048611in}}{\pgfqpoint{0.000000in}{0.000000in}}{%
\pgfpathmoveto{\pgfqpoint{0.000000in}{0.000000in}}%
\pgfpathlineto{\pgfqpoint{0.000000in}{-0.048611in}}%
\pgfusepath{stroke,fill}%
}%
\begin{pgfscope}%
\pgfsys@transformshift{1.002291in}{0.528000in}%
\pgfsys@useobject{currentmarker}{}%
\end{pgfscope}%
\end{pgfscope}%
\begin{pgfscope}%
\definecolor{textcolor}{rgb}{0.000000,0.000000,0.000000}%
\pgfsetstrokecolor{textcolor}%
\pgfsetfillcolor{textcolor}%
\pgftext[x=1.002291in,y=0.430778in,,top]{\color{textcolor}{\sffamily\fontsize{11.000000}{13.200000}\selectfont\catcode`\^=\active\def^{\ifmmode\sp\else\^{}\fi}\catcode`\%=\active\def%{\%}$\mathdefault{0}$}}%
\end{pgfscope}%
\begin{pgfscope}%
\pgfsetbuttcap%
\pgfsetroundjoin%
\definecolor{currentfill}{rgb}{0.000000,0.000000,0.000000}%
\pgfsetfillcolor{currentfill}%
\pgfsetlinewidth{0.803000pt}%
\definecolor{currentstroke}{rgb}{0.000000,0.000000,0.000000}%
\pgfsetstrokecolor{currentstroke}%
\pgfsetdash{}{0pt}%
\pgfsys@defobject{currentmarker}{\pgfqpoint{0.000000in}{-0.048611in}}{\pgfqpoint{0.000000in}{0.000000in}}{%
\pgfpathmoveto{\pgfqpoint{0.000000in}{0.000000in}}%
\pgfpathlineto{\pgfqpoint{0.000000in}{-0.048611in}}%
\pgfusepath{stroke,fill}%
}%
\begin{pgfscope}%
\pgfsys@transformshift{1.774396in}{0.528000in}%
\pgfsys@useobject{currentmarker}{}%
\end{pgfscope}%
\end{pgfscope}%
\begin{pgfscope}%
\definecolor{textcolor}{rgb}{0.000000,0.000000,0.000000}%
\pgfsetstrokecolor{textcolor}%
\pgfsetfillcolor{textcolor}%
\pgftext[x=1.774396in,y=0.430778in,,top]{\color{textcolor}{\sffamily\fontsize{11.000000}{13.200000}\selectfont\catcode`\^=\active\def^{\ifmmode\sp\else\^{}\fi}\catcode`\%=\active\def%{\%}$\mathdefault{20}$}}%
\end{pgfscope}%
\begin{pgfscope}%
\pgfsetbuttcap%
\pgfsetroundjoin%
\definecolor{currentfill}{rgb}{0.000000,0.000000,0.000000}%
\pgfsetfillcolor{currentfill}%
\pgfsetlinewidth{0.803000pt}%
\definecolor{currentstroke}{rgb}{0.000000,0.000000,0.000000}%
\pgfsetstrokecolor{currentstroke}%
\pgfsetdash{}{0pt}%
\pgfsys@defobject{currentmarker}{\pgfqpoint{0.000000in}{-0.048611in}}{\pgfqpoint{0.000000in}{0.000000in}}{%
\pgfpathmoveto{\pgfqpoint{0.000000in}{0.000000in}}%
\pgfpathlineto{\pgfqpoint{0.000000in}{-0.048611in}}%
\pgfusepath{stroke,fill}%
}%
\begin{pgfscope}%
\pgfsys@transformshift{2.546501in}{0.528000in}%
\pgfsys@useobject{currentmarker}{}%
\end{pgfscope}%
\end{pgfscope}%
\begin{pgfscope}%
\definecolor{textcolor}{rgb}{0.000000,0.000000,0.000000}%
\pgfsetstrokecolor{textcolor}%
\pgfsetfillcolor{textcolor}%
\pgftext[x=2.546501in,y=0.430778in,,top]{\color{textcolor}{\sffamily\fontsize{11.000000}{13.200000}\selectfont\catcode`\^=\active\def^{\ifmmode\sp\else\^{}\fi}\catcode`\%=\active\def%{\%}$\mathdefault{40}$}}%
\end{pgfscope}%
\begin{pgfscope}%
\pgfsetbuttcap%
\pgfsetroundjoin%
\definecolor{currentfill}{rgb}{0.000000,0.000000,0.000000}%
\pgfsetfillcolor{currentfill}%
\pgfsetlinewidth{0.803000pt}%
\definecolor{currentstroke}{rgb}{0.000000,0.000000,0.000000}%
\pgfsetstrokecolor{currentstroke}%
\pgfsetdash{}{0pt}%
\pgfsys@defobject{currentmarker}{\pgfqpoint{0.000000in}{-0.048611in}}{\pgfqpoint{0.000000in}{0.000000in}}{%
\pgfpathmoveto{\pgfqpoint{0.000000in}{0.000000in}}%
\pgfpathlineto{\pgfqpoint{0.000000in}{-0.048611in}}%
\pgfusepath{stroke,fill}%
}%
\begin{pgfscope}%
\pgfsys@transformshift{3.318605in}{0.528000in}%
\pgfsys@useobject{currentmarker}{}%
\end{pgfscope}%
\end{pgfscope}%
\begin{pgfscope}%
\definecolor{textcolor}{rgb}{0.000000,0.000000,0.000000}%
\pgfsetstrokecolor{textcolor}%
\pgfsetfillcolor{textcolor}%
\pgftext[x=3.318605in,y=0.430778in,,top]{\color{textcolor}{\sffamily\fontsize{11.000000}{13.200000}\selectfont\catcode`\^=\active\def^{\ifmmode\sp\else\^{}\fi}\catcode`\%=\active\def%{\%}$\mathdefault{60}$}}%
\end{pgfscope}%
\begin{pgfscope}%
\pgfsetbuttcap%
\pgfsetroundjoin%
\definecolor{currentfill}{rgb}{0.000000,0.000000,0.000000}%
\pgfsetfillcolor{currentfill}%
\pgfsetlinewidth{0.803000pt}%
\definecolor{currentstroke}{rgb}{0.000000,0.000000,0.000000}%
\pgfsetstrokecolor{currentstroke}%
\pgfsetdash{}{0pt}%
\pgfsys@defobject{currentmarker}{\pgfqpoint{0.000000in}{-0.048611in}}{\pgfqpoint{0.000000in}{0.000000in}}{%
\pgfpathmoveto{\pgfqpoint{0.000000in}{0.000000in}}%
\pgfpathlineto{\pgfqpoint{0.000000in}{-0.048611in}}%
\pgfusepath{stroke,fill}%
}%
\begin{pgfscope}%
\pgfsys@transformshift{4.090710in}{0.528000in}%
\pgfsys@useobject{currentmarker}{}%
\end{pgfscope}%
\end{pgfscope}%
\begin{pgfscope}%
\definecolor{textcolor}{rgb}{0.000000,0.000000,0.000000}%
\pgfsetstrokecolor{textcolor}%
\pgfsetfillcolor{textcolor}%
\pgftext[x=4.090710in,y=0.430778in,,top]{\color{textcolor}{\sffamily\fontsize{11.000000}{13.200000}\selectfont\catcode`\^=\active\def^{\ifmmode\sp\else\^{}\fi}\catcode`\%=\active\def%{\%}$\mathdefault{80}$}}%
\end{pgfscope}%
\begin{pgfscope}%
\pgfsetbuttcap%
\pgfsetroundjoin%
\definecolor{currentfill}{rgb}{0.000000,0.000000,0.000000}%
\pgfsetfillcolor{currentfill}%
\pgfsetlinewidth{0.803000pt}%
\definecolor{currentstroke}{rgb}{0.000000,0.000000,0.000000}%
\pgfsetstrokecolor{currentstroke}%
\pgfsetdash{}{0pt}%
\pgfsys@defobject{currentmarker}{\pgfqpoint{0.000000in}{-0.048611in}}{\pgfqpoint{0.000000in}{0.000000in}}{%
\pgfpathmoveto{\pgfqpoint{0.000000in}{0.000000in}}%
\pgfpathlineto{\pgfqpoint{0.000000in}{-0.048611in}}%
\pgfusepath{stroke,fill}%
}%
\begin{pgfscope}%
\pgfsys@transformshift{4.862814in}{0.528000in}%
\pgfsys@useobject{currentmarker}{}%
\end{pgfscope}%
\end{pgfscope}%
\begin{pgfscope}%
\definecolor{textcolor}{rgb}{0.000000,0.000000,0.000000}%
\pgfsetstrokecolor{textcolor}%
\pgfsetfillcolor{textcolor}%
\pgftext[x=4.862814in,y=0.430778in,,top]{\color{textcolor}{\sffamily\fontsize{11.000000}{13.200000}\selectfont\catcode`\^=\active\def^{\ifmmode\sp\else\^{}\fi}\catcode`\%=\active\def%{\%}$\mathdefault{100}$}}%
\end{pgfscope}%
\begin{pgfscope}%
\pgfsetbuttcap%
\pgfsetroundjoin%
\definecolor{currentfill}{rgb}{0.000000,0.000000,0.000000}%
\pgfsetfillcolor{currentfill}%
\pgfsetlinewidth{0.803000pt}%
\definecolor{currentstroke}{rgb}{0.000000,0.000000,0.000000}%
\pgfsetstrokecolor{currentstroke}%
\pgfsetdash{}{0pt}%
\pgfsys@defobject{currentmarker}{\pgfqpoint{0.000000in}{-0.048611in}}{\pgfqpoint{0.000000in}{0.000000in}}{%
\pgfpathmoveto{\pgfqpoint{0.000000in}{0.000000in}}%
\pgfpathlineto{\pgfqpoint{0.000000in}{-0.048611in}}%
\pgfusepath{stroke,fill}%
}%
\begin{pgfscope}%
\pgfsys@transformshift{5.634919in}{0.528000in}%
\pgfsys@useobject{currentmarker}{}%
\end{pgfscope}%
\end{pgfscope}%
\begin{pgfscope}%
\definecolor{textcolor}{rgb}{0.000000,0.000000,0.000000}%
\pgfsetstrokecolor{textcolor}%
\pgfsetfillcolor{textcolor}%
\pgftext[x=5.634919in,y=0.430778in,,top]{\color{textcolor}{\sffamily\fontsize{11.000000}{13.200000}\selectfont\catcode`\^=\active\def^{\ifmmode\sp\else\^{}\fi}\catcode`\%=\active\def%{\%}$\mathdefault{120}$}}%
\end{pgfscope}%
\begin{pgfscope}%
\definecolor{textcolor}{rgb}{0.000000,0.000000,0.000000}%
\pgfsetstrokecolor{textcolor}%
\pgfsetfillcolor{textcolor}%
\pgftext[x=3.280000in,y=0.227368in,,top]{\color{textcolor}{\sffamily\fontsize{11.000000}{13.200000}\selectfont\catcode`\^=\active\def^{\ifmmode\sp\else\^{}\fi}\catcode`\%=\active\def%{\%}Kernel index}}%
\end{pgfscope}%
\begin{pgfscope}%
\pgfsetbuttcap%
\pgfsetroundjoin%
\definecolor{currentfill}{rgb}{0.000000,0.000000,0.000000}%
\pgfsetfillcolor{currentfill}%
\pgfsetlinewidth{0.803000pt}%
\definecolor{currentstroke}{rgb}{0.000000,0.000000,0.000000}%
\pgfsetstrokecolor{currentstroke}%
\pgfsetdash{}{0pt}%
\pgfsys@defobject{currentmarker}{\pgfqpoint{-0.048611in}{0.000000in}}{\pgfqpoint{-0.000000in}{0.000000in}}{%
\pgfpathmoveto{\pgfqpoint{-0.000000in}{0.000000in}}%
\pgfpathlineto{\pgfqpoint{-0.048611in}{0.000000in}}%
\pgfusepath{stroke,fill}%
}%
\begin{pgfscope}%
\pgfsys@transformshift{0.800000in}{0.528000in}%
\pgfsys@useobject{currentmarker}{}%
\end{pgfscope}%
\end{pgfscope}%
\begin{pgfscope}%
\definecolor{textcolor}{rgb}{0.000000,0.000000,0.000000}%
\pgfsetstrokecolor{textcolor}%
\pgfsetfillcolor{textcolor}%
\pgftext[x=0.626736in, y=0.469962in, left, base]{\color{textcolor}{\sffamily\fontsize{11.000000}{13.200000}\selectfont\catcode`\^=\active\def^{\ifmmode\sp\else\^{}\fi}\catcode`\%=\active\def%{\%}$\mathdefault{0}$}}%
\end{pgfscope}%
\begin{pgfscope}%
\pgfsetbuttcap%
\pgfsetroundjoin%
\definecolor{currentfill}{rgb}{0.000000,0.000000,0.000000}%
\pgfsetfillcolor{currentfill}%
\pgfsetlinewidth{0.803000pt}%
\definecolor{currentstroke}{rgb}{0.000000,0.000000,0.000000}%
\pgfsetstrokecolor{currentstroke}%
\pgfsetdash{}{0pt}%
\pgfsys@defobject{currentmarker}{\pgfqpoint{-0.048611in}{0.000000in}}{\pgfqpoint{-0.000000in}{0.000000in}}{%
\pgfpathmoveto{\pgfqpoint{-0.000000in}{0.000000in}}%
\pgfpathlineto{\pgfqpoint{-0.048611in}{0.000000in}}%
\pgfusepath{stroke,fill}%
}%
\begin{pgfscope}%
\pgfsys@transformshift{0.800000in}{1.267200in}%
\pgfsys@useobject{currentmarker}{}%
\end{pgfscope}%
\end{pgfscope}%
\begin{pgfscope}%
\definecolor{textcolor}{rgb}{0.000000,0.000000,0.000000}%
\pgfsetstrokecolor{textcolor}%
\pgfsetfillcolor{textcolor}%
\pgftext[x=0.550694in, y=1.209162in, left, base]{\color{textcolor}{\sffamily\fontsize{11.000000}{13.200000}\selectfont\catcode`\^=\active\def^{\ifmmode\sp\else\^{}\fi}\catcode`\%=\active\def%{\%}$\mathdefault{20}$}}%
\end{pgfscope}%
\begin{pgfscope}%
\pgfsetbuttcap%
\pgfsetroundjoin%
\definecolor{currentfill}{rgb}{0.000000,0.000000,0.000000}%
\pgfsetfillcolor{currentfill}%
\pgfsetlinewidth{0.803000pt}%
\definecolor{currentstroke}{rgb}{0.000000,0.000000,0.000000}%
\pgfsetstrokecolor{currentstroke}%
\pgfsetdash{}{0pt}%
\pgfsys@defobject{currentmarker}{\pgfqpoint{-0.048611in}{0.000000in}}{\pgfqpoint{-0.000000in}{0.000000in}}{%
\pgfpathmoveto{\pgfqpoint{-0.000000in}{0.000000in}}%
\pgfpathlineto{\pgfqpoint{-0.048611in}{0.000000in}}%
\pgfusepath{stroke,fill}%
}%
\begin{pgfscope}%
\pgfsys@transformshift{0.800000in}{2.006400in}%
\pgfsys@useobject{currentmarker}{}%
\end{pgfscope}%
\end{pgfscope}%
\begin{pgfscope}%
\definecolor{textcolor}{rgb}{0.000000,0.000000,0.000000}%
\pgfsetstrokecolor{textcolor}%
\pgfsetfillcolor{textcolor}%
\pgftext[x=0.550694in, y=1.948362in, left, base]{\color{textcolor}{\sffamily\fontsize{11.000000}{13.200000}\selectfont\catcode`\^=\active\def^{\ifmmode\sp\else\^{}\fi}\catcode`\%=\active\def%{\%}$\mathdefault{40}$}}%
\end{pgfscope}%
\begin{pgfscope}%
\pgfsetbuttcap%
\pgfsetroundjoin%
\definecolor{currentfill}{rgb}{0.000000,0.000000,0.000000}%
\pgfsetfillcolor{currentfill}%
\pgfsetlinewidth{0.803000pt}%
\definecolor{currentstroke}{rgb}{0.000000,0.000000,0.000000}%
\pgfsetstrokecolor{currentstroke}%
\pgfsetdash{}{0pt}%
\pgfsys@defobject{currentmarker}{\pgfqpoint{-0.048611in}{0.000000in}}{\pgfqpoint{-0.000000in}{0.000000in}}{%
\pgfpathmoveto{\pgfqpoint{-0.000000in}{0.000000in}}%
\pgfpathlineto{\pgfqpoint{-0.048611in}{0.000000in}}%
\pgfusepath{stroke,fill}%
}%
\begin{pgfscope}%
\pgfsys@transformshift{0.800000in}{2.745600in}%
\pgfsys@useobject{currentmarker}{}%
\end{pgfscope}%
\end{pgfscope}%
\begin{pgfscope}%
\definecolor{textcolor}{rgb}{0.000000,0.000000,0.000000}%
\pgfsetstrokecolor{textcolor}%
\pgfsetfillcolor{textcolor}%
\pgftext[x=0.550694in, y=2.687562in, left, base]{\color{textcolor}{\sffamily\fontsize{11.000000}{13.200000}\selectfont\catcode`\^=\active\def^{\ifmmode\sp\else\^{}\fi}\catcode`\%=\active\def%{\%}$\mathdefault{60}$}}%
\end{pgfscope}%
\begin{pgfscope}%
\pgfsetbuttcap%
\pgfsetroundjoin%
\definecolor{currentfill}{rgb}{0.000000,0.000000,0.000000}%
\pgfsetfillcolor{currentfill}%
\pgfsetlinewidth{0.803000pt}%
\definecolor{currentstroke}{rgb}{0.000000,0.000000,0.000000}%
\pgfsetstrokecolor{currentstroke}%
\pgfsetdash{}{0pt}%
\pgfsys@defobject{currentmarker}{\pgfqpoint{-0.048611in}{0.000000in}}{\pgfqpoint{-0.000000in}{0.000000in}}{%
\pgfpathmoveto{\pgfqpoint{-0.000000in}{0.000000in}}%
\pgfpathlineto{\pgfqpoint{-0.048611in}{0.000000in}}%
\pgfusepath{stroke,fill}%
}%
\begin{pgfscope}%
\pgfsys@transformshift{0.800000in}{3.484800in}%
\pgfsys@useobject{currentmarker}{}%
\end{pgfscope}%
\end{pgfscope}%
\begin{pgfscope}%
\definecolor{textcolor}{rgb}{0.000000,0.000000,0.000000}%
\pgfsetstrokecolor{textcolor}%
\pgfsetfillcolor{textcolor}%
\pgftext[x=0.550694in, y=3.426762in, left, base]{\color{textcolor}{\sffamily\fontsize{11.000000}{13.200000}\selectfont\catcode`\^=\active\def^{\ifmmode\sp\else\^{}\fi}\catcode`\%=\active\def%{\%}$\mathdefault{80}$}}%
\end{pgfscope}%
\begin{pgfscope}%
\pgfsetbuttcap%
\pgfsetroundjoin%
\definecolor{currentfill}{rgb}{0.000000,0.000000,0.000000}%
\pgfsetfillcolor{currentfill}%
\pgfsetlinewidth{0.803000pt}%
\definecolor{currentstroke}{rgb}{0.000000,0.000000,0.000000}%
\pgfsetstrokecolor{currentstroke}%
\pgfsetdash{}{0pt}%
\pgfsys@defobject{currentmarker}{\pgfqpoint{-0.048611in}{0.000000in}}{\pgfqpoint{-0.000000in}{0.000000in}}{%
\pgfpathmoveto{\pgfqpoint{-0.000000in}{0.000000in}}%
\pgfpathlineto{\pgfqpoint{-0.048611in}{0.000000in}}%
\pgfusepath{stroke,fill}%
}%
\begin{pgfscope}%
\pgfsys@transformshift{0.800000in}{4.224000in}%
\pgfsys@useobject{currentmarker}{}%
\end{pgfscope}%
\end{pgfscope}%
\begin{pgfscope}%
\definecolor{textcolor}{rgb}{0.000000,0.000000,0.000000}%
\pgfsetstrokecolor{textcolor}%
\pgfsetfillcolor{textcolor}%
\pgftext[x=0.474652in, y=4.165962in, left, base]{\color{textcolor}{\sffamily\fontsize{11.000000}{13.200000}\selectfont\catcode`\^=\active\def^{\ifmmode\sp\else\^{}\fi}\catcode`\%=\active\def%{\%}$\mathdefault{100}$}}%
\end{pgfscope}%
\begin{pgfscope}%
\definecolor{textcolor}{rgb}{0.000000,0.000000,0.000000}%
\pgfsetstrokecolor{textcolor}%
\pgfsetfillcolor{textcolor}%
\pgftext[x=0.419097in,y=2.376000in,,bottom,rotate=90.000000]{\color{textcolor}{\sffamily\fontsize{11.000000}{13.200000}\selectfont\catcode`\^=\active\def^{\ifmmode\sp\else\^{}\fi}\catcode`\%=\active\def%{\%}Data reuse (in %)}}%
\end{pgfscope}%
\begin{pgfscope}%
\pgfsetrectcap%
\pgfsetmiterjoin%
\pgfsetlinewidth{0.803000pt}%
\definecolor{currentstroke}{rgb}{0.000000,0.000000,0.000000}%
\pgfsetstrokecolor{currentstroke}%
\pgfsetdash{}{0pt}%
\pgfpathmoveto{\pgfqpoint{0.800000in}{0.528000in}}%
\pgfpathlineto{\pgfqpoint{0.800000in}{4.224000in}}%
\pgfusepath{stroke}%
\end{pgfscope}%
\begin{pgfscope}%
\pgfsetrectcap%
\pgfsetmiterjoin%
\pgfsetlinewidth{0.803000pt}%
\definecolor{currentstroke}{rgb}{0.000000,0.000000,0.000000}%
\pgfsetstrokecolor{currentstroke}%
\pgfsetdash{}{0pt}%
\pgfpathmoveto{\pgfqpoint{5.760000in}{0.528000in}}%
\pgfpathlineto{\pgfqpoint{5.760000in}{4.224000in}}%
\pgfusepath{stroke}%
\end{pgfscope}%
\begin{pgfscope}%
\pgfsetrectcap%
\pgfsetmiterjoin%
\pgfsetlinewidth{0.803000pt}%
\definecolor{currentstroke}{rgb}{0.000000,0.000000,0.000000}%
\pgfsetstrokecolor{currentstroke}%
\pgfsetdash{}{0pt}%
\pgfpathmoveto{\pgfqpoint{0.800000in}{0.528000in}}%
\pgfpathlineto{\pgfqpoint{5.760000in}{0.528000in}}%
\pgfusepath{stroke}%
\end{pgfscope}%
\begin{pgfscope}%
\pgfsetrectcap%
\pgfsetmiterjoin%
\pgfsetlinewidth{0.803000pt}%
\definecolor{currentstroke}{rgb}{0.000000,0.000000,0.000000}%
\pgfsetstrokecolor{currentstroke}%
\pgfsetdash{}{0pt}%
\pgfpathmoveto{\pgfqpoint{0.800000in}{4.224000in}}%
\pgfpathlineto{\pgfqpoint{5.760000in}{4.224000in}}%
\pgfusepath{stroke}%
\end{pgfscope}%
\end{pgfpicture}%
\makeatother%
\endgroup%
}
        \caption{Forward data reuse}
        \label{fig:dct_forward_reuse}
    \end{subfigure}
    \begin{subfigure}{0.4\textwidth}
        \resizebox{\textwidth}{!}{%% Creator: Matplotlib, PGF backend
%%
%% To include the figure in your LaTeX document, write
%%   \input{<filename>.pgf}
%%
%% Make sure the required packages are loaded in your preamble
%%   \usepackage{pgf}
%%
%% Also ensure that all the required font packages are loaded; for instance,
%% the lmodern package is sometimes necessary when using math font.
%%   \usepackage{lmodern}
%%
%% Figures using additional raster images can only be included by \input if
%% they are in the same directory as the main LaTeX file. For loading figures
%% from other directories you can use the `import` package
%%   \usepackage{import}
%%
%% and then include the figures with
%%   \import{<path to file>}{<filename>.pgf}
%%
%% Matplotlib used the following preamble
%%   \def\mathdefault#1{#1}
%%   \everymath=\expandafter{\the\everymath\displaystyle}
%%   
%%   \usepackage{fontspec}
%%   \setmainfont{DejaVuSerif.ttf}[Path=\detokenize{/usr/lib/python3.11/site-packages/matplotlib/mpl-data/fonts/ttf/}]
%%   \setsansfont{DejaVuSans.ttf}[Path=\detokenize{/usr/lib/python3.11/site-packages/matplotlib/mpl-data/fonts/ttf/}]
%%   \setmonofont{DejaVuSansMono.ttf}[Path=\detokenize{/usr/lib/python3.11/site-packages/matplotlib/mpl-data/fonts/ttf/}]
%%   \makeatletter\@ifpackageloaded{underscore}{}{\usepackage[strings]{underscore}}\makeatother
%%
\begingroup%
\makeatletter%
\begin{pgfpicture}%
\pgfpathrectangle{\pgfpointorigin}{\pgfqpoint{6.400000in}{4.800000in}}%
\pgfusepath{use as bounding box, clip}%
\begin{pgfscope}%
\pgfsetbuttcap%
\pgfsetmiterjoin%
\definecolor{currentfill}{rgb}{1.000000,1.000000,1.000000}%
\pgfsetfillcolor{currentfill}%
\pgfsetlinewidth{0.000000pt}%
\definecolor{currentstroke}{rgb}{1.000000,1.000000,1.000000}%
\pgfsetstrokecolor{currentstroke}%
\pgfsetdash{}{0pt}%
\pgfpathmoveto{\pgfqpoint{0.000000in}{0.000000in}}%
\pgfpathlineto{\pgfqpoint{6.400000in}{0.000000in}}%
\pgfpathlineto{\pgfqpoint{6.400000in}{4.800000in}}%
\pgfpathlineto{\pgfqpoint{0.000000in}{4.800000in}}%
\pgfpathlineto{\pgfqpoint{0.000000in}{0.000000in}}%
\pgfpathclose%
\pgfusepath{fill}%
\end{pgfscope}%
\begin{pgfscope}%
\pgfsetbuttcap%
\pgfsetmiterjoin%
\definecolor{currentfill}{rgb}{1.000000,1.000000,1.000000}%
\pgfsetfillcolor{currentfill}%
\pgfsetlinewidth{0.000000pt}%
\definecolor{currentstroke}{rgb}{0.000000,0.000000,0.000000}%
\pgfsetstrokecolor{currentstroke}%
\pgfsetstrokeopacity{0.000000}%
\pgfsetdash{}{0pt}%
\pgfpathmoveto{\pgfqpoint{0.693757in}{0.613486in}}%
\pgfpathlineto{\pgfqpoint{6.235000in}{0.613486in}}%
\pgfpathlineto{\pgfqpoint{6.235000in}{4.576962in}}%
\pgfpathlineto{\pgfqpoint{0.693757in}{4.576962in}}%
\pgfpathlineto{\pgfqpoint{0.693757in}{0.613486in}}%
\pgfpathclose%
\pgfusepath{fill}%
\end{pgfscope}%
\begin{pgfscope}%
\pgfpathrectangle{\pgfqpoint{0.693757in}{0.613486in}}{\pgfqpoint{5.541243in}{3.963477in}}%
\pgfusepath{clip}%
\pgfsetbuttcap%
\pgfsetmiterjoin%
\definecolor{currentfill}{rgb}{0.000000,0.000000,1.000000}%
\pgfsetfillcolor{currentfill}%
\pgfsetlinewidth{0.000000pt}%
\definecolor{currentstroke}{rgb}{0.000000,0.000000,0.000000}%
\pgfsetstrokecolor{currentstroke}%
\pgfsetstrokeopacity{0.000000}%
\pgfsetdash{}{0pt}%
\pgfpathmoveto{\pgfqpoint{0.945632in}{0.613486in}}%
\pgfpathlineto{\pgfqpoint{0.980135in}{0.613486in}}%
\pgfpathlineto{\pgfqpoint{0.980135in}{4.576962in}}%
\pgfpathlineto{\pgfqpoint{0.945632in}{4.576962in}}%
\pgfpathlineto{\pgfqpoint{0.945632in}{0.613486in}}%
\pgfpathclose%
\pgfusepath{fill}%
\end{pgfscope}%
\begin{pgfscope}%
\pgfpathrectangle{\pgfqpoint{0.693757in}{0.613486in}}{\pgfqpoint{5.541243in}{3.963477in}}%
\pgfusepath{clip}%
\pgfsetbuttcap%
\pgfsetmiterjoin%
\definecolor{currentfill}{rgb}{0.000000,0.000000,1.000000}%
\pgfsetfillcolor{currentfill}%
\pgfsetlinewidth{0.000000pt}%
\definecolor{currentstroke}{rgb}{0.000000,0.000000,0.000000}%
\pgfsetstrokecolor{currentstroke}%
\pgfsetstrokeopacity{0.000000}%
\pgfsetdash{}{0pt}%
\pgfpathmoveto{\pgfqpoint{0.988761in}{0.613486in}}%
\pgfpathlineto{\pgfqpoint{1.023264in}{0.613486in}}%
\pgfpathlineto{\pgfqpoint{1.023264in}{4.576962in}}%
\pgfpathlineto{\pgfqpoint{0.988761in}{4.576962in}}%
\pgfpathlineto{\pgfqpoint{0.988761in}{0.613486in}}%
\pgfpathclose%
\pgfusepath{fill}%
\end{pgfscope}%
\begin{pgfscope}%
\pgfpathrectangle{\pgfqpoint{0.693757in}{0.613486in}}{\pgfqpoint{5.541243in}{3.963477in}}%
\pgfusepath{clip}%
\pgfsetbuttcap%
\pgfsetmiterjoin%
\definecolor{currentfill}{rgb}{0.000000,0.000000,1.000000}%
\pgfsetfillcolor{currentfill}%
\pgfsetlinewidth{0.000000pt}%
\definecolor{currentstroke}{rgb}{0.000000,0.000000,0.000000}%
\pgfsetstrokecolor{currentstroke}%
\pgfsetstrokeopacity{0.000000}%
\pgfsetdash{}{0pt}%
\pgfpathmoveto{\pgfqpoint{1.031890in}{0.613486in}}%
\pgfpathlineto{\pgfqpoint{1.066394in}{0.613486in}}%
\pgfpathlineto{\pgfqpoint{1.066394in}{4.576962in}}%
\pgfpathlineto{\pgfqpoint{1.031890in}{4.576962in}}%
\pgfpathlineto{\pgfqpoint{1.031890in}{0.613486in}}%
\pgfpathclose%
\pgfusepath{fill}%
\end{pgfscope}%
\begin{pgfscope}%
\pgfpathrectangle{\pgfqpoint{0.693757in}{0.613486in}}{\pgfqpoint{5.541243in}{3.963477in}}%
\pgfusepath{clip}%
\pgfsetbuttcap%
\pgfsetmiterjoin%
\definecolor{currentfill}{rgb}{0.000000,0.000000,1.000000}%
\pgfsetfillcolor{currentfill}%
\pgfsetlinewidth{0.000000pt}%
\definecolor{currentstroke}{rgb}{0.000000,0.000000,0.000000}%
\pgfsetstrokecolor{currentstroke}%
\pgfsetstrokeopacity{0.000000}%
\pgfsetdash{}{0pt}%
\pgfpathmoveto{\pgfqpoint{1.075019in}{0.613486in}}%
\pgfpathlineto{\pgfqpoint{1.109523in}{0.613486in}}%
\pgfpathlineto{\pgfqpoint{1.109523in}{4.576962in}}%
\pgfpathlineto{\pgfqpoint{1.075019in}{4.576962in}}%
\pgfpathlineto{\pgfqpoint{1.075019in}{0.613486in}}%
\pgfpathclose%
\pgfusepath{fill}%
\end{pgfscope}%
\begin{pgfscope}%
\pgfpathrectangle{\pgfqpoint{0.693757in}{0.613486in}}{\pgfqpoint{5.541243in}{3.963477in}}%
\pgfusepath{clip}%
\pgfsetbuttcap%
\pgfsetmiterjoin%
\definecolor{currentfill}{rgb}{0.000000,0.000000,1.000000}%
\pgfsetfillcolor{currentfill}%
\pgfsetlinewidth{0.000000pt}%
\definecolor{currentstroke}{rgb}{0.000000,0.000000,0.000000}%
\pgfsetstrokecolor{currentstroke}%
\pgfsetstrokeopacity{0.000000}%
\pgfsetdash{}{0pt}%
\pgfpathmoveto{\pgfqpoint{1.118149in}{0.613486in}}%
\pgfpathlineto{\pgfqpoint{1.152652in}{0.613486in}}%
\pgfpathlineto{\pgfqpoint{1.152652in}{4.576962in}}%
\pgfpathlineto{\pgfqpoint{1.118149in}{4.576962in}}%
\pgfpathlineto{\pgfqpoint{1.118149in}{0.613486in}}%
\pgfpathclose%
\pgfusepath{fill}%
\end{pgfscope}%
\begin{pgfscope}%
\pgfpathrectangle{\pgfqpoint{0.693757in}{0.613486in}}{\pgfqpoint{5.541243in}{3.963477in}}%
\pgfusepath{clip}%
\pgfsetbuttcap%
\pgfsetmiterjoin%
\definecolor{currentfill}{rgb}{0.000000,0.000000,1.000000}%
\pgfsetfillcolor{currentfill}%
\pgfsetlinewidth{0.000000pt}%
\definecolor{currentstroke}{rgb}{0.000000,0.000000,0.000000}%
\pgfsetstrokecolor{currentstroke}%
\pgfsetstrokeopacity{0.000000}%
\pgfsetdash{}{0pt}%
\pgfpathmoveto{\pgfqpoint{1.161278in}{0.613486in}}%
\pgfpathlineto{\pgfqpoint{1.195781in}{0.613486in}}%
\pgfpathlineto{\pgfqpoint{1.195781in}{4.576962in}}%
\pgfpathlineto{\pgfqpoint{1.161278in}{4.576962in}}%
\pgfpathlineto{\pgfqpoint{1.161278in}{0.613486in}}%
\pgfpathclose%
\pgfusepath{fill}%
\end{pgfscope}%
\begin{pgfscope}%
\pgfpathrectangle{\pgfqpoint{0.693757in}{0.613486in}}{\pgfqpoint{5.541243in}{3.963477in}}%
\pgfusepath{clip}%
\pgfsetbuttcap%
\pgfsetmiterjoin%
\definecolor{currentfill}{rgb}{0.000000,0.000000,1.000000}%
\pgfsetfillcolor{currentfill}%
\pgfsetlinewidth{0.000000pt}%
\definecolor{currentstroke}{rgb}{0.000000,0.000000,0.000000}%
\pgfsetstrokecolor{currentstroke}%
\pgfsetstrokeopacity{0.000000}%
\pgfsetdash{}{0pt}%
\pgfpathmoveto{\pgfqpoint{1.204407in}{0.613486in}}%
\pgfpathlineto{\pgfqpoint{1.238910in}{0.613486in}}%
\pgfpathlineto{\pgfqpoint{1.238910in}{4.576962in}}%
\pgfpathlineto{\pgfqpoint{1.204407in}{4.576962in}}%
\pgfpathlineto{\pgfqpoint{1.204407in}{0.613486in}}%
\pgfpathclose%
\pgfusepath{fill}%
\end{pgfscope}%
\begin{pgfscope}%
\pgfpathrectangle{\pgfqpoint{0.693757in}{0.613486in}}{\pgfqpoint{5.541243in}{3.963477in}}%
\pgfusepath{clip}%
\pgfsetbuttcap%
\pgfsetmiterjoin%
\definecolor{currentfill}{rgb}{0.000000,0.000000,1.000000}%
\pgfsetfillcolor{currentfill}%
\pgfsetlinewidth{0.000000pt}%
\definecolor{currentstroke}{rgb}{0.000000,0.000000,0.000000}%
\pgfsetstrokecolor{currentstroke}%
\pgfsetstrokeopacity{0.000000}%
\pgfsetdash{}{0pt}%
\pgfpathmoveto{\pgfqpoint{1.247536in}{0.613486in}}%
\pgfpathlineto{\pgfqpoint{1.282040in}{0.613486in}}%
\pgfpathlineto{\pgfqpoint{1.282040in}{4.576962in}}%
\pgfpathlineto{\pgfqpoint{1.247536in}{4.576962in}}%
\pgfpathlineto{\pgfqpoint{1.247536in}{0.613486in}}%
\pgfpathclose%
\pgfusepath{fill}%
\end{pgfscope}%
\begin{pgfscope}%
\pgfpathrectangle{\pgfqpoint{0.693757in}{0.613486in}}{\pgfqpoint{5.541243in}{3.963477in}}%
\pgfusepath{clip}%
\pgfsetbuttcap%
\pgfsetmiterjoin%
\definecolor{currentfill}{rgb}{0.000000,0.000000,1.000000}%
\pgfsetfillcolor{currentfill}%
\pgfsetlinewidth{0.000000pt}%
\definecolor{currentstroke}{rgb}{0.000000,0.000000,0.000000}%
\pgfsetstrokecolor{currentstroke}%
\pgfsetstrokeopacity{0.000000}%
\pgfsetdash{}{0pt}%
\pgfpathmoveto{\pgfqpoint{1.290666in}{0.613486in}}%
\pgfpathlineto{\pgfqpoint{1.325169in}{0.613486in}}%
\pgfpathlineto{\pgfqpoint{1.325169in}{4.576962in}}%
\pgfpathlineto{\pgfqpoint{1.290666in}{4.576962in}}%
\pgfpathlineto{\pgfqpoint{1.290666in}{0.613486in}}%
\pgfpathclose%
\pgfusepath{fill}%
\end{pgfscope}%
\begin{pgfscope}%
\pgfpathrectangle{\pgfqpoint{0.693757in}{0.613486in}}{\pgfqpoint{5.541243in}{3.963477in}}%
\pgfusepath{clip}%
\pgfsetbuttcap%
\pgfsetmiterjoin%
\definecolor{currentfill}{rgb}{0.000000,0.000000,1.000000}%
\pgfsetfillcolor{currentfill}%
\pgfsetlinewidth{0.000000pt}%
\definecolor{currentstroke}{rgb}{0.000000,0.000000,0.000000}%
\pgfsetstrokecolor{currentstroke}%
\pgfsetstrokeopacity{0.000000}%
\pgfsetdash{}{0pt}%
\pgfpathmoveto{\pgfqpoint{1.333795in}{0.613486in}}%
\pgfpathlineto{\pgfqpoint{1.368298in}{0.613486in}}%
\pgfpathlineto{\pgfqpoint{1.368298in}{4.576962in}}%
\pgfpathlineto{\pgfqpoint{1.333795in}{4.576962in}}%
\pgfpathlineto{\pgfqpoint{1.333795in}{0.613486in}}%
\pgfpathclose%
\pgfusepath{fill}%
\end{pgfscope}%
\begin{pgfscope}%
\pgfpathrectangle{\pgfqpoint{0.693757in}{0.613486in}}{\pgfqpoint{5.541243in}{3.963477in}}%
\pgfusepath{clip}%
\pgfsetbuttcap%
\pgfsetmiterjoin%
\definecolor{currentfill}{rgb}{0.000000,0.000000,1.000000}%
\pgfsetfillcolor{currentfill}%
\pgfsetlinewidth{0.000000pt}%
\definecolor{currentstroke}{rgb}{0.000000,0.000000,0.000000}%
\pgfsetstrokecolor{currentstroke}%
\pgfsetstrokeopacity{0.000000}%
\pgfsetdash{}{0pt}%
\pgfpathmoveto{\pgfqpoint{1.376924in}{0.613486in}}%
\pgfpathlineto{\pgfqpoint{1.411427in}{0.613486in}}%
\pgfpathlineto{\pgfqpoint{1.411427in}{4.576839in}}%
\pgfpathlineto{\pgfqpoint{1.376924in}{4.576839in}}%
\pgfpathlineto{\pgfqpoint{1.376924in}{0.613486in}}%
\pgfpathclose%
\pgfusepath{fill}%
\end{pgfscope}%
\begin{pgfscope}%
\pgfpathrectangle{\pgfqpoint{0.693757in}{0.613486in}}{\pgfqpoint{5.541243in}{3.963477in}}%
\pgfusepath{clip}%
\pgfsetbuttcap%
\pgfsetmiterjoin%
\definecolor{currentfill}{rgb}{0.000000,0.000000,1.000000}%
\pgfsetfillcolor{currentfill}%
\pgfsetlinewidth{0.000000pt}%
\definecolor{currentstroke}{rgb}{0.000000,0.000000,0.000000}%
\pgfsetstrokecolor{currentstroke}%
\pgfsetstrokeopacity{0.000000}%
\pgfsetdash{}{0pt}%
\pgfpathmoveto{\pgfqpoint{1.420053in}{0.613486in}}%
\pgfpathlineto{\pgfqpoint{1.454557in}{0.613486in}}%
\pgfpathlineto{\pgfqpoint{1.454557in}{1.108972in}}%
\pgfpathlineto{\pgfqpoint{1.420053in}{1.108972in}}%
\pgfpathlineto{\pgfqpoint{1.420053in}{0.613486in}}%
\pgfpathclose%
\pgfusepath{fill}%
\end{pgfscope}%
\begin{pgfscope}%
\pgfpathrectangle{\pgfqpoint{0.693757in}{0.613486in}}{\pgfqpoint{5.541243in}{3.963477in}}%
\pgfusepath{clip}%
\pgfsetbuttcap%
\pgfsetmiterjoin%
\definecolor{currentfill}{rgb}{0.000000,0.000000,1.000000}%
\pgfsetfillcolor{currentfill}%
\pgfsetlinewidth{0.000000pt}%
\definecolor{currentstroke}{rgb}{0.000000,0.000000,0.000000}%
\pgfsetstrokecolor{currentstroke}%
\pgfsetstrokeopacity{0.000000}%
\pgfsetdash{}{0pt}%
\pgfpathmoveto{\pgfqpoint{1.463182in}{0.613486in}}%
\pgfpathlineto{\pgfqpoint{1.497686in}{0.613486in}}%
\pgfpathlineto{\pgfqpoint{1.497686in}{4.576962in}}%
\pgfpathlineto{\pgfqpoint{1.463182in}{4.576962in}}%
\pgfpathlineto{\pgfqpoint{1.463182in}{0.613486in}}%
\pgfpathclose%
\pgfusepath{fill}%
\end{pgfscope}%
\begin{pgfscope}%
\pgfpathrectangle{\pgfqpoint{0.693757in}{0.613486in}}{\pgfqpoint{5.541243in}{3.963477in}}%
\pgfusepath{clip}%
\pgfsetbuttcap%
\pgfsetmiterjoin%
\definecolor{currentfill}{rgb}{0.000000,0.000000,1.000000}%
\pgfsetfillcolor{currentfill}%
\pgfsetlinewidth{0.000000pt}%
\definecolor{currentstroke}{rgb}{0.000000,0.000000,0.000000}%
\pgfsetstrokecolor{currentstroke}%
\pgfsetstrokeopacity{0.000000}%
\pgfsetdash{}{0pt}%
\pgfpathmoveto{\pgfqpoint{1.506312in}{0.613486in}}%
\pgfpathlineto{\pgfqpoint{1.540815in}{0.613486in}}%
\pgfpathlineto{\pgfqpoint{1.540815in}{4.576962in}}%
\pgfpathlineto{\pgfqpoint{1.506312in}{4.576962in}}%
\pgfpathlineto{\pgfqpoint{1.506312in}{0.613486in}}%
\pgfpathclose%
\pgfusepath{fill}%
\end{pgfscope}%
\begin{pgfscope}%
\pgfpathrectangle{\pgfqpoint{0.693757in}{0.613486in}}{\pgfqpoint{5.541243in}{3.963477in}}%
\pgfusepath{clip}%
\pgfsetbuttcap%
\pgfsetmiterjoin%
\definecolor{currentfill}{rgb}{0.000000,0.000000,1.000000}%
\pgfsetfillcolor{currentfill}%
\pgfsetlinewidth{0.000000pt}%
\definecolor{currentstroke}{rgb}{0.000000,0.000000,0.000000}%
\pgfsetstrokecolor{currentstroke}%
\pgfsetstrokeopacity{0.000000}%
\pgfsetdash{}{0pt}%
\pgfpathmoveto{\pgfqpoint{1.549441in}{0.613486in}}%
\pgfpathlineto{\pgfqpoint{1.583944in}{0.613486in}}%
\pgfpathlineto{\pgfqpoint{1.583944in}{4.576962in}}%
\pgfpathlineto{\pgfqpoint{1.549441in}{4.576962in}}%
\pgfpathlineto{\pgfqpoint{1.549441in}{0.613486in}}%
\pgfpathclose%
\pgfusepath{fill}%
\end{pgfscope}%
\begin{pgfscope}%
\pgfpathrectangle{\pgfqpoint{0.693757in}{0.613486in}}{\pgfqpoint{5.541243in}{3.963477in}}%
\pgfusepath{clip}%
\pgfsetbuttcap%
\pgfsetmiterjoin%
\definecolor{currentfill}{rgb}{0.000000,0.000000,1.000000}%
\pgfsetfillcolor{currentfill}%
\pgfsetlinewidth{0.000000pt}%
\definecolor{currentstroke}{rgb}{0.000000,0.000000,0.000000}%
\pgfsetstrokecolor{currentstroke}%
\pgfsetstrokeopacity{0.000000}%
\pgfsetdash{}{0pt}%
\pgfpathmoveto{\pgfqpoint{1.592570in}{0.613486in}}%
\pgfpathlineto{\pgfqpoint{1.627074in}{0.613486in}}%
\pgfpathlineto{\pgfqpoint{1.627074in}{4.576962in}}%
\pgfpathlineto{\pgfqpoint{1.592570in}{4.576962in}}%
\pgfpathlineto{\pgfqpoint{1.592570in}{0.613486in}}%
\pgfpathclose%
\pgfusepath{fill}%
\end{pgfscope}%
\begin{pgfscope}%
\pgfpathrectangle{\pgfqpoint{0.693757in}{0.613486in}}{\pgfqpoint{5.541243in}{3.963477in}}%
\pgfusepath{clip}%
\pgfsetbuttcap%
\pgfsetmiterjoin%
\definecolor{currentfill}{rgb}{0.000000,0.000000,1.000000}%
\pgfsetfillcolor{currentfill}%
\pgfsetlinewidth{0.000000pt}%
\definecolor{currentstroke}{rgb}{0.000000,0.000000,0.000000}%
\pgfsetstrokecolor{currentstroke}%
\pgfsetstrokeopacity{0.000000}%
\pgfsetdash{}{0pt}%
\pgfpathmoveto{\pgfqpoint{1.635699in}{0.613486in}}%
\pgfpathlineto{\pgfqpoint{1.670203in}{0.613486in}}%
\pgfpathlineto{\pgfqpoint{1.670203in}{4.576962in}}%
\pgfpathlineto{\pgfqpoint{1.635699in}{4.576962in}}%
\pgfpathlineto{\pgfqpoint{1.635699in}{0.613486in}}%
\pgfpathclose%
\pgfusepath{fill}%
\end{pgfscope}%
\begin{pgfscope}%
\pgfpathrectangle{\pgfqpoint{0.693757in}{0.613486in}}{\pgfqpoint{5.541243in}{3.963477in}}%
\pgfusepath{clip}%
\pgfsetbuttcap%
\pgfsetmiterjoin%
\definecolor{currentfill}{rgb}{0.000000,0.000000,1.000000}%
\pgfsetfillcolor{currentfill}%
\pgfsetlinewidth{0.000000pt}%
\definecolor{currentstroke}{rgb}{0.000000,0.000000,0.000000}%
\pgfsetstrokecolor{currentstroke}%
\pgfsetstrokeopacity{0.000000}%
\pgfsetdash{}{0pt}%
\pgfpathmoveto{\pgfqpoint{1.678829in}{0.613486in}}%
\pgfpathlineto{\pgfqpoint{1.713332in}{0.613486in}}%
\pgfpathlineto{\pgfqpoint{1.713332in}{4.576962in}}%
\pgfpathlineto{\pgfqpoint{1.678829in}{4.576962in}}%
\pgfpathlineto{\pgfqpoint{1.678829in}{0.613486in}}%
\pgfpathclose%
\pgfusepath{fill}%
\end{pgfscope}%
\begin{pgfscope}%
\pgfpathrectangle{\pgfqpoint{0.693757in}{0.613486in}}{\pgfqpoint{5.541243in}{3.963477in}}%
\pgfusepath{clip}%
\pgfsetbuttcap%
\pgfsetmiterjoin%
\definecolor{currentfill}{rgb}{0.000000,0.000000,1.000000}%
\pgfsetfillcolor{currentfill}%
\pgfsetlinewidth{0.000000pt}%
\definecolor{currentstroke}{rgb}{0.000000,0.000000,0.000000}%
\pgfsetstrokecolor{currentstroke}%
\pgfsetstrokeopacity{0.000000}%
\pgfsetdash{}{0pt}%
\pgfpathmoveto{\pgfqpoint{1.721958in}{0.613486in}}%
\pgfpathlineto{\pgfqpoint{1.756461in}{0.613486in}}%
\pgfpathlineto{\pgfqpoint{1.756461in}{4.576962in}}%
\pgfpathlineto{\pgfqpoint{1.721958in}{4.576962in}}%
\pgfpathlineto{\pgfqpoint{1.721958in}{0.613486in}}%
\pgfpathclose%
\pgfusepath{fill}%
\end{pgfscope}%
\begin{pgfscope}%
\pgfpathrectangle{\pgfqpoint{0.693757in}{0.613486in}}{\pgfqpoint{5.541243in}{3.963477in}}%
\pgfusepath{clip}%
\pgfsetbuttcap%
\pgfsetmiterjoin%
\definecolor{currentfill}{rgb}{0.000000,0.000000,1.000000}%
\pgfsetfillcolor{currentfill}%
\pgfsetlinewidth{0.000000pt}%
\definecolor{currentstroke}{rgb}{0.000000,0.000000,0.000000}%
\pgfsetstrokecolor{currentstroke}%
\pgfsetstrokeopacity{0.000000}%
\pgfsetdash{}{0pt}%
\pgfpathmoveto{\pgfqpoint{1.765087in}{0.613486in}}%
\pgfpathlineto{\pgfqpoint{1.799590in}{0.613486in}}%
\pgfpathlineto{\pgfqpoint{1.799590in}{4.576962in}}%
\pgfpathlineto{\pgfqpoint{1.765087in}{4.576962in}}%
\pgfpathlineto{\pgfqpoint{1.765087in}{0.613486in}}%
\pgfpathclose%
\pgfusepath{fill}%
\end{pgfscope}%
\begin{pgfscope}%
\pgfpathrectangle{\pgfqpoint{0.693757in}{0.613486in}}{\pgfqpoint{5.541243in}{3.963477in}}%
\pgfusepath{clip}%
\pgfsetbuttcap%
\pgfsetmiterjoin%
\definecolor{currentfill}{rgb}{0.000000,0.000000,1.000000}%
\pgfsetfillcolor{currentfill}%
\pgfsetlinewidth{0.000000pt}%
\definecolor{currentstroke}{rgb}{0.000000,0.000000,0.000000}%
\pgfsetstrokecolor{currentstroke}%
\pgfsetstrokeopacity{0.000000}%
\pgfsetdash{}{0pt}%
\pgfpathmoveto{\pgfqpoint{1.808216in}{0.613486in}}%
\pgfpathlineto{\pgfqpoint{1.842720in}{0.613486in}}%
\pgfpathlineto{\pgfqpoint{1.842720in}{4.576962in}}%
\pgfpathlineto{\pgfqpoint{1.808216in}{4.576962in}}%
\pgfpathlineto{\pgfqpoint{1.808216in}{0.613486in}}%
\pgfpathclose%
\pgfusepath{fill}%
\end{pgfscope}%
\begin{pgfscope}%
\pgfpathrectangle{\pgfqpoint{0.693757in}{0.613486in}}{\pgfqpoint{5.541243in}{3.963477in}}%
\pgfusepath{clip}%
\pgfsetbuttcap%
\pgfsetmiterjoin%
\definecolor{currentfill}{rgb}{0.000000,0.000000,1.000000}%
\pgfsetfillcolor{currentfill}%
\pgfsetlinewidth{0.000000pt}%
\definecolor{currentstroke}{rgb}{0.000000,0.000000,0.000000}%
\pgfsetstrokecolor{currentstroke}%
\pgfsetstrokeopacity{0.000000}%
\pgfsetdash{}{0pt}%
\pgfpathmoveto{\pgfqpoint{1.851346in}{0.613486in}}%
\pgfpathlineto{\pgfqpoint{1.885849in}{0.613486in}}%
\pgfpathlineto{\pgfqpoint{1.885849in}{4.576962in}}%
\pgfpathlineto{\pgfqpoint{1.851346in}{4.576962in}}%
\pgfpathlineto{\pgfqpoint{1.851346in}{0.613486in}}%
\pgfpathclose%
\pgfusepath{fill}%
\end{pgfscope}%
\begin{pgfscope}%
\pgfpathrectangle{\pgfqpoint{0.693757in}{0.613486in}}{\pgfqpoint{5.541243in}{3.963477in}}%
\pgfusepath{clip}%
\pgfsetbuttcap%
\pgfsetmiterjoin%
\definecolor{currentfill}{rgb}{0.000000,0.000000,1.000000}%
\pgfsetfillcolor{currentfill}%
\pgfsetlinewidth{0.000000pt}%
\definecolor{currentstroke}{rgb}{0.000000,0.000000,0.000000}%
\pgfsetstrokecolor{currentstroke}%
\pgfsetstrokeopacity{0.000000}%
\pgfsetdash{}{0pt}%
\pgfpathmoveto{\pgfqpoint{1.894475in}{0.613486in}}%
\pgfpathlineto{\pgfqpoint{1.928978in}{0.613486in}}%
\pgfpathlineto{\pgfqpoint{1.928978in}{4.576962in}}%
\pgfpathlineto{\pgfqpoint{1.894475in}{4.576962in}}%
\pgfpathlineto{\pgfqpoint{1.894475in}{0.613486in}}%
\pgfpathclose%
\pgfusepath{fill}%
\end{pgfscope}%
\begin{pgfscope}%
\pgfpathrectangle{\pgfqpoint{0.693757in}{0.613486in}}{\pgfqpoint{5.541243in}{3.963477in}}%
\pgfusepath{clip}%
\pgfsetbuttcap%
\pgfsetmiterjoin%
\definecolor{currentfill}{rgb}{0.000000,0.000000,1.000000}%
\pgfsetfillcolor{currentfill}%
\pgfsetlinewidth{0.000000pt}%
\definecolor{currentstroke}{rgb}{0.000000,0.000000,0.000000}%
\pgfsetstrokecolor{currentstroke}%
\pgfsetstrokeopacity{0.000000}%
\pgfsetdash{}{0pt}%
\pgfpathmoveto{\pgfqpoint{1.937604in}{0.613486in}}%
\pgfpathlineto{\pgfqpoint{1.972107in}{0.613486in}}%
\pgfpathlineto{\pgfqpoint{1.972107in}{4.576962in}}%
\pgfpathlineto{\pgfqpoint{1.937604in}{4.576962in}}%
\pgfpathlineto{\pgfqpoint{1.937604in}{0.613486in}}%
\pgfpathclose%
\pgfusepath{fill}%
\end{pgfscope}%
\begin{pgfscope}%
\pgfpathrectangle{\pgfqpoint{0.693757in}{0.613486in}}{\pgfqpoint{5.541243in}{3.963477in}}%
\pgfusepath{clip}%
\pgfsetbuttcap%
\pgfsetmiterjoin%
\definecolor{currentfill}{rgb}{0.000000,0.000000,1.000000}%
\pgfsetfillcolor{currentfill}%
\pgfsetlinewidth{0.000000pt}%
\definecolor{currentstroke}{rgb}{0.000000,0.000000,0.000000}%
\pgfsetstrokecolor{currentstroke}%
\pgfsetstrokeopacity{0.000000}%
\pgfsetdash{}{0pt}%
\pgfpathmoveto{\pgfqpoint{1.980733in}{0.613486in}}%
\pgfpathlineto{\pgfqpoint{2.015237in}{0.613486in}}%
\pgfpathlineto{\pgfqpoint{2.015237in}{4.576962in}}%
\pgfpathlineto{\pgfqpoint{1.980733in}{4.576962in}}%
\pgfpathlineto{\pgfqpoint{1.980733in}{0.613486in}}%
\pgfpathclose%
\pgfusepath{fill}%
\end{pgfscope}%
\begin{pgfscope}%
\pgfpathrectangle{\pgfqpoint{0.693757in}{0.613486in}}{\pgfqpoint{5.541243in}{3.963477in}}%
\pgfusepath{clip}%
\pgfsetbuttcap%
\pgfsetmiterjoin%
\definecolor{currentfill}{rgb}{0.000000,0.000000,1.000000}%
\pgfsetfillcolor{currentfill}%
\pgfsetlinewidth{0.000000pt}%
\definecolor{currentstroke}{rgb}{0.000000,0.000000,0.000000}%
\pgfsetstrokecolor{currentstroke}%
\pgfsetstrokeopacity{0.000000}%
\pgfsetdash{}{0pt}%
\pgfpathmoveto{\pgfqpoint{2.023862in}{0.613486in}}%
\pgfpathlineto{\pgfqpoint{2.058366in}{0.613486in}}%
\pgfpathlineto{\pgfqpoint{2.058366in}{4.576962in}}%
\pgfpathlineto{\pgfqpoint{2.023862in}{4.576962in}}%
\pgfpathlineto{\pgfqpoint{2.023862in}{0.613486in}}%
\pgfpathclose%
\pgfusepath{fill}%
\end{pgfscope}%
\begin{pgfscope}%
\pgfpathrectangle{\pgfqpoint{0.693757in}{0.613486in}}{\pgfqpoint{5.541243in}{3.963477in}}%
\pgfusepath{clip}%
\pgfsetbuttcap%
\pgfsetmiterjoin%
\definecolor{currentfill}{rgb}{0.000000,0.000000,1.000000}%
\pgfsetfillcolor{currentfill}%
\pgfsetlinewidth{0.000000pt}%
\definecolor{currentstroke}{rgb}{0.000000,0.000000,0.000000}%
\pgfsetstrokecolor{currentstroke}%
\pgfsetstrokeopacity{0.000000}%
\pgfsetdash{}{0pt}%
\pgfpathmoveto{\pgfqpoint{2.066992in}{0.613486in}}%
\pgfpathlineto{\pgfqpoint{2.101495in}{0.613486in}}%
\pgfpathlineto{\pgfqpoint{2.101495in}{4.576962in}}%
\pgfpathlineto{\pgfqpoint{2.066992in}{4.576962in}}%
\pgfpathlineto{\pgfqpoint{2.066992in}{0.613486in}}%
\pgfpathclose%
\pgfusepath{fill}%
\end{pgfscope}%
\begin{pgfscope}%
\pgfpathrectangle{\pgfqpoint{0.693757in}{0.613486in}}{\pgfqpoint{5.541243in}{3.963477in}}%
\pgfusepath{clip}%
\pgfsetbuttcap%
\pgfsetmiterjoin%
\definecolor{currentfill}{rgb}{0.000000,0.000000,1.000000}%
\pgfsetfillcolor{currentfill}%
\pgfsetlinewidth{0.000000pt}%
\definecolor{currentstroke}{rgb}{0.000000,0.000000,0.000000}%
\pgfsetstrokecolor{currentstroke}%
\pgfsetstrokeopacity{0.000000}%
\pgfsetdash{}{0pt}%
\pgfpathmoveto{\pgfqpoint{2.110121in}{0.613486in}}%
\pgfpathlineto{\pgfqpoint{2.144624in}{0.613486in}}%
\pgfpathlineto{\pgfqpoint{2.144624in}{4.576962in}}%
\pgfpathlineto{\pgfqpoint{2.110121in}{4.576962in}}%
\pgfpathlineto{\pgfqpoint{2.110121in}{0.613486in}}%
\pgfpathclose%
\pgfusepath{fill}%
\end{pgfscope}%
\begin{pgfscope}%
\pgfpathrectangle{\pgfqpoint{0.693757in}{0.613486in}}{\pgfqpoint{5.541243in}{3.963477in}}%
\pgfusepath{clip}%
\pgfsetbuttcap%
\pgfsetmiterjoin%
\definecolor{currentfill}{rgb}{0.000000,0.000000,1.000000}%
\pgfsetfillcolor{currentfill}%
\pgfsetlinewidth{0.000000pt}%
\definecolor{currentstroke}{rgb}{0.000000,0.000000,0.000000}%
\pgfsetstrokecolor{currentstroke}%
\pgfsetstrokeopacity{0.000000}%
\pgfsetdash{}{0pt}%
\pgfpathmoveto{\pgfqpoint{2.153250in}{0.613486in}}%
\pgfpathlineto{\pgfqpoint{2.187753in}{0.613486in}}%
\pgfpathlineto{\pgfqpoint{2.187753in}{4.576962in}}%
\pgfpathlineto{\pgfqpoint{2.153250in}{4.576962in}}%
\pgfpathlineto{\pgfqpoint{2.153250in}{0.613486in}}%
\pgfpathclose%
\pgfusepath{fill}%
\end{pgfscope}%
\begin{pgfscope}%
\pgfpathrectangle{\pgfqpoint{0.693757in}{0.613486in}}{\pgfqpoint{5.541243in}{3.963477in}}%
\pgfusepath{clip}%
\pgfsetbuttcap%
\pgfsetmiterjoin%
\definecolor{currentfill}{rgb}{0.000000,0.000000,1.000000}%
\pgfsetfillcolor{currentfill}%
\pgfsetlinewidth{0.000000pt}%
\definecolor{currentstroke}{rgb}{0.000000,0.000000,0.000000}%
\pgfsetstrokecolor{currentstroke}%
\pgfsetstrokeopacity{0.000000}%
\pgfsetdash{}{0pt}%
\pgfpathmoveto{\pgfqpoint{2.196379in}{0.613486in}}%
\pgfpathlineto{\pgfqpoint{2.230883in}{0.613486in}}%
\pgfpathlineto{\pgfqpoint{2.230883in}{4.576962in}}%
\pgfpathlineto{\pgfqpoint{2.196379in}{4.576962in}}%
\pgfpathlineto{\pgfqpoint{2.196379in}{0.613486in}}%
\pgfpathclose%
\pgfusepath{fill}%
\end{pgfscope}%
\begin{pgfscope}%
\pgfpathrectangle{\pgfqpoint{0.693757in}{0.613486in}}{\pgfqpoint{5.541243in}{3.963477in}}%
\pgfusepath{clip}%
\pgfsetbuttcap%
\pgfsetmiterjoin%
\definecolor{currentfill}{rgb}{0.000000,0.000000,1.000000}%
\pgfsetfillcolor{currentfill}%
\pgfsetlinewidth{0.000000pt}%
\definecolor{currentstroke}{rgb}{0.000000,0.000000,0.000000}%
\pgfsetstrokecolor{currentstroke}%
\pgfsetstrokeopacity{0.000000}%
\pgfsetdash{}{0pt}%
\pgfpathmoveto{\pgfqpoint{2.239509in}{0.613486in}}%
\pgfpathlineto{\pgfqpoint{2.274012in}{0.613486in}}%
\pgfpathlineto{\pgfqpoint{2.274012in}{4.576962in}}%
\pgfpathlineto{\pgfqpoint{2.239509in}{4.576962in}}%
\pgfpathlineto{\pgfqpoint{2.239509in}{0.613486in}}%
\pgfpathclose%
\pgfusepath{fill}%
\end{pgfscope}%
\begin{pgfscope}%
\pgfpathrectangle{\pgfqpoint{0.693757in}{0.613486in}}{\pgfqpoint{5.541243in}{3.963477in}}%
\pgfusepath{clip}%
\pgfsetbuttcap%
\pgfsetmiterjoin%
\definecolor{currentfill}{rgb}{0.000000,0.000000,1.000000}%
\pgfsetfillcolor{currentfill}%
\pgfsetlinewidth{0.000000pt}%
\definecolor{currentstroke}{rgb}{0.000000,0.000000,0.000000}%
\pgfsetstrokecolor{currentstroke}%
\pgfsetstrokeopacity{0.000000}%
\pgfsetdash{}{0pt}%
\pgfpathmoveto{\pgfqpoint{2.282638in}{0.613486in}}%
\pgfpathlineto{\pgfqpoint{2.317141in}{0.613486in}}%
\pgfpathlineto{\pgfqpoint{2.317141in}{4.576962in}}%
\pgfpathlineto{\pgfqpoint{2.282638in}{4.576962in}}%
\pgfpathlineto{\pgfqpoint{2.282638in}{0.613486in}}%
\pgfpathclose%
\pgfusepath{fill}%
\end{pgfscope}%
\begin{pgfscope}%
\pgfpathrectangle{\pgfqpoint{0.693757in}{0.613486in}}{\pgfqpoint{5.541243in}{3.963477in}}%
\pgfusepath{clip}%
\pgfsetbuttcap%
\pgfsetmiterjoin%
\definecolor{currentfill}{rgb}{0.000000,0.000000,1.000000}%
\pgfsetfillcolor{currentfill}%
\pgfsetlinewidth{0.000000pt}%
\definecolor{currentstroke}{rgb}{0.000000,0.000000,0.000000}%
\pgfsetstrokecolor{currentstroke}%
\pgfsetstrokeopacity{0.000000}%
\pgfsetdash{}{0pt}%
\pgfpathmoveto{\pgfqpoint{2.325767in}{0.613486in}}%
\pgfpathlineto{\pgfqpoint{2.360270in}{0.613486in}}%
\pgfpathlineto{\pgfqpoint{2.360270in}{4.576962in}}%
\pgfpathlineto{\pgfqpoint{2.325767in}{4.576962in}}%
\pgfpathlineto{\pgfqpoint{2.325767in}{0.613486in}}%
\pgfpathclose%
\pgfusepath{fill}%
\end{pgfscope}%
\begin{pgfscope}%
\pgfpathrectangle{\pgfqpoint{0.693757in}{0.613486in}}{\pgfqpoint{5.541243in}{3.963477in}}%
\pgfusepath{clip}%
\pgfsetbuttcap%
\pgfsetmiterjoin%
\definecolor{currentfill}{rgb}{0.000000,0.000000,1.000000}%
\pgfsetfillcolor{currentfill}%
\pgfsetlinewidth{0.000000pt}%
\definecolor{currentstroke}{rgb}{0.000000,0.000000,0.000000}%
\pgfsetstrokecolor{currentstroke}%
\pgfsetstrokeopacity{0.000000}%
\pgfsetdash{}{0pt}%
\pgfpathmoveto{\pgfqpoint{2.368896in}{0.613486in}}%
\pgfpathlineto{\pgfqpoint{2.403400in}{0.613486in}}%
\pgfpathlineto{\pgfqpoint{2.403400in}{4.576962in}}%
\pgfpathlineto{\pgfqpoint{2.368896in}{4.576962in}}%
\pgfpathlineto{\pgfqpoint{2.368896in}{0.613486in}}%
\pgfpathclose%
\pgfusepath{fill}%
\end{pgfscope}%
\begin{pgfscope}%
\pgfpathrectangle{\pgfqpoint{0.693757in}{0.613486in}}{\pgfqpoint{5.541243in}{3.963477in}}%
\pgfusepath{clip}%
\pgfsetbuttcap%
\pgfsetmiterjoin%
\definecolor{currentfill}{rgb}{0.000000,0.000000,1.000000}%
\pgfsetfillcolor{currentfill}%
\pgfsetlinewidth{0.000000pt}%
\definecolor{currentstroke}{rgb}{0.000000,0.000000,0.000000}%
\pgfsetstrokecolor{currentstroke}%
\pgfsetstrokeopacity{0.000000}%
\pgfsetdash{}{0pt}%
\pgfpathmoveto{\pgfqpoint{2.412025in}{0.613486in}}%
\pgfpathlineto{\pgfqpoint{2.446529in}{0.613486in}}%
\pgfpathlineto{\pgfqpoint{2.446529in}{4.576962in}}%
\pgfpathlineto{\pgfqpoint{2.412025in}{4.576962in}}%
\pgfpathlineto{\pgfqpoint{2.412025in}{0.613486in}}%
\pgfpathclose%
\pgfusepath{fill}%
\end{pgfscope}%
\begin{pgfscope}%
\pgfpathrectangle{\pgfqpoint{0.693757in}{0.613486in}}{\pgfqpoint{5.541243in}{3.963477in}}%
\pgfusepath{clip}%
\pgfsetbuttcap%
\pgfsetmiterjoin%
\definecolor{currentfill}{rgb}{0.000000,0.000000,1.000000}%
\pgfsetfillcolor{currentfill}%
\pgfsetlinewidth{0.000000pt}%
\definecolor{currentstroke}{rgb}{0.000000,0.000000,0.000000}%
\pgfsetstrokecolor{currentstroke}%
\pgfsetstrokeopacity{0.000000}%
\pgfsetdash{}{0pt}%
\pgfpathmoveto{\pgfqpoint{2.455155in}{0.613486in}}%
\pgfpathlineto{\pgfqpoint{2.489658in}{0.613486in}}%
\pgfpathlineto{\pgfqpoint{2.489658in}{4.576962in}}%
\pgfpathlineto{\pgfqpoint{2.455155in}{4.576962in}}%
\pgfpathlineto{\pgfqpoint{2.455155in}{0.613486in}}%
\pgfpathclose%
\pgfusepath{fill}%
\end{pgfscope}%
\begin{pgfscope}%
\pgfpathrectangle{\pgfqpoint{0.693757in}{0.613486in}}{\pgfqpoint{5.541243in}{3.963477in}}%
\pgfusepath{clip}%
\pgfsetbuttcap%
\pgfsetmiterjoin%
\definecolor{currentfill}{rgb}{0.000000,0.000000,1.000000}%
\pgfsetfillcolor{currentfill}%
\pgfsetlinewidth{0.000000pt}%
\definecolor{currentstroke}{rgb}{0.000000,0.000000,0.000000}%
\pgfsetstrokecolor{currentstroke}%
\pgfsetstrokeopacity{0.000000}%
\pgfsetdash{}{0pt}%
\pgfpathmoveto{\pgfqpoint{2.498284in}{0.613486in}}%
\pgfpathlineto{\pgfqpoint{2.532787in}{0.613486in}}%
\pgfpathlineto{\pgfqpoint{2.532787in}{4.576962in}}%
\pgfpathlineto{\pgfqpoint{2.498284in}{4.576962in}}%
\pgfpathlineto{\pgfqpoint{2.498284in}{0.613486in}}%
\pgfpathclose%
\pgfusepath{fill}%
\end{pgfscope}%
\begin{pgfscope}%
\pgfpathrectangle{\pgfqpoint{0.693757in}{0.613486in}}{\pgfqpoint{5.541243in}{3.963477in}}%
\pgfusepath{clip}%
\pgfsetbuttcap%
\pgfsetmiterjoin%
\definecolor{currentfill}{rgb}{0.000000,0.000000,1.000000}%
\pgfsetfillcolor{currentfill}%
\pgfsetlinewidth{0.000000pt}%
\definecolor{currentstroke}{rgb}{0.000000,0.000000,0.000000}%
\pgfsetstrokecolor{currentstroke}%
\pgfsetstrokeopacity{0.000000}%
\pgfsetdash{}{0pt}%
\pgfpathmoveto{\pgfqpoint{2.541413in}{0.613486in}}%
\pgfpathlineto{\pgfqpoint{2.575916in}{0.613486in}}%
\pgfpathlineto{\pgfqpoint{2.575916in}{4.576962in}}%
\pgfpathlineto{\pgfqpoint{2.541413in}{4.576962in}}%
\pgfpathlineto{\pgfqpoint{2.541413in}{0.613486in}}%
\pgfpathclose%
\pgfusepath{fill}%
\end{pgfscope}%
\begin{pgfscope}%
\pgfpathrectangle{\pgfqpoint{0.693757in}{0.613486in}}{\pgfqpoint{5.541243in}{3.963477in}}%
\pgfusepath{clip}%
\pgfsetbuttcap%
\pgfsetmiterjoin%
\definecolor{currentfill}{rgb}{0.000000,0.000000,1.000000}%
\pgfsetfillcolor{currentfill}%
\pgfsetlinewidth{0.000000pt}%
\definecolor{currentstroke}{rgb}{0.000000,0.000000,0.000000}%
\pgfsetstrokecolor{currentstroke}%
\pgfsetstrokeopacity{0.000000}%
\pgfsetdash{}{0pt}%
\pgfpathmoveto{\pgfqpoint{2.584542in}{0.613486in}}%
\pgfpathlineto{\pgfqpoint{2.619046in}{0.613486in}}%
\pgfpathlineto{\pgfqpoint{2.619046in}{4.576962in}}%
\pgfpathlineto{\pgfqpoint{2.584542in}{4.576962in}}%
\pgfpathlineto{\pgfqpoint{2.584542in}{0.613486in}}%
\pgfpathclose%
\pgfusepath{fill}%
\end{pgfscope}%
\begin{pgfscope}%
\pgfpathrectangle{\pgfqpoint{0.693757in}{0.613486in}}{\pgfqpoint{5.541243in}{3.963477in}}%
\pgfusepath{clip}%
\pgfsetbuttcap%
\pgfsetmiterjoin%
\definecolor{currentfill}{rgb}{0.000000,0.000000,1.000000}%
\pgfsetfillcolor{currentfill}%
\pgfsetlinewidth{0.000000pt}%
\definecolor{currentstroke}{rgb}{0.000000,0.000000,0.000000}%
\pgfsetstrokecolor{currentstroke}%
\pgfsetstrokeopacity{0.000000}%
\pgfsetdash{}{0pt}%
\pgfpathmoveto{\pgfqpoint{2.627672in}{0.613486in}}%
\pgfpathlineto{\pgfqpoint{2.662175in}{0.613486in}}%
\pgfpathlineto{\pgfqpoint{2.662175in}{4.576962in}}%
\pgfpathlineto{\pgfqpoint{2.627672in}{4.576962in}}%
\pgfpathlineto{\pgfqpoint{2.627672in}{0.613486in}}%
\pgfpathclose%
\pgfusepath{fill}%
\end{pgfscope}%
\begin{pgfscope}%
\pgfpathrectangle{\pgfqpoint{0.693757in}{0.613486in}}{\pgfqpoint{5.541243in}{3.963477in}}%
\pgfusepath{clip}%
\pgfsetbuttcap%
\pgfsetmiterjoin%
\definecolor{currentfill}{rgb}{0.000000,0.000000,1.000000}%
\pgfsetfillcolor{currentfill}%
\pgfsetlinewidth{0.000000pt}%
\definecolor{currentstroke}{rgb}{0.000000,0.000000,0.000000}%
\pgfsetstrokecolor{currentstroke}%
\pgfsetstrokeopacity{0.000000}%
\pgfsetdash{}{0pt}%
\pgfpathmoveto{\pgfqpoint{2.670801in}{0.613486in}}%
\pgfpathlineto{\pgfqpoint{2.705304in}{0.613486in}}%
\pgfpathlineto{\pgfqpoint{2.705304in}{4.576962in}}%
\pgfpathlineto{\pgfqpoint{2.670801in}{4.576962in}}%
\pgfpathlineto{\pgfqpoint{2.670801in}{0.613486in}}%
\pgfpathclose%
\pgfusepath{fill}%
\end{pgfscope}%
\begin{pgfscope}%
\pgfpathrectangle{\pgfqpoint{0.693757in}{0.613486in}}{\pgfqpoint{5.541243in}{3.963477in}}%
\pgfusepath{clip}%
\pgfsetbuttcap%
\pgfsetmiterjoin%
\definecolor{currentfill}{rgb}{0.000000,0.000000,1.000000}%
\pgfsetfillcolor{currentfill}%
\pgfsetlinewidth{0.000000pt}%
\definecolor{currentstroke}{rgb}{0.000000,0.000000,0.000000}%
\pgfsetstrokecolor{currentstroke}%
\pgfsetstrokeopacity{0.000000}%
\pgfsetdash{}{0pt}%
\pgfpathmoveto{\pgfqpoint{2.713930in}{0.613486in}}%
\pgfpathlineto{\pgfqpoint{2.748433in}{0.613486in}}%
\pgfpathlineto{\pgfqpoint{2.748433in}{4.576962in}}%
\pgfpathlineto{\pgfqpoint{2.713930in}{4.576962in}}%
\pgfpathlineto{\pgfqpoint{2.713930in}{0.613486in}}%
\pgfpathclose%
\pgfusepath{fill}%
\end{pgfscope}%
\begin{pgfscope}%
\pgfpathrectangle{\pgfqpoint{0.693757in}{0.613486in}}{\pgfqpoint{5.541243in}{3.963477in}}%
\pgfusepath{clip}%
\pgfsetbuttcap%
\pgfsetmiterjoin%
\definecolor{currentfill}{rgb}{0.000000,0.000000,1.000000}%
\pgfsetfillcolor{currentfill}%
\pgfsetlinewidth{0.000000pt}%
\definecolor{currentstroke}{rgb}{0.000000,0.000000,0.000000}%
\pgfsetstrokecolor{currentstroke}%
\pgfsetstrokeopacity{0.000000}%
\pgfsetdash{}{0pt}%
\pgfpathmoveto{\pgfqpoint{2.757059in}{0.613486in}}%
\pgfpathlineto{\pgfqpoint{2.791563in}{0.613486in}}%
\pgfpathlineto{\pgfqpoint{2.791563in}{4.576962in}}%
\pgfpathlineto{\pgfqpoint{2.757059in}{4.576962in}}%
\pgfpathlineto{\pgfqpoint{2.757059in}{0.613486in}}%
\pgfpathclose%
\pgfusepath{fill}%
\end{pgfscope}%
\begin{pgfscope}%
\pgfpathrectangle{\pgfqpoint{0.693757in}{0.613486in}}{\pgfqpoint{5.541243in}{3.963477in}}%
\pgfusepath{clip}%
\pgfsetbuttcap%
\pgfsetmiterjoin%
\definecolor{currentfill}{rgb}{0.000000,0.000000,1.000000}%
\pgfsetfillcolor{currentfill}%
\pgfsetlinewidth{0.000000pt}%
\definecolor{currentstroke}{rgb}{0.000000,0.000000,0.000000}%
\pgfsetstrokecolor{currentstroke}%
\pgfsetstrokeopacity{0.000000}%
\pgfsetdash{}{0pt}%
\pgfpathmoveto{\pgfqpoint{2.800188in}{0.613486in}}%
\pgfpathlineto{\pgfqpoint{2.834692in}{0.613486in}}%
\pgfpathlineto{\pgfqpoint{2.834692in}{4.576962in}}%
\pgfpathlineto{\pgfqpoint{2.800188in}{4.576962in}}%
\pgfpathlineto{\pgfqpoint{2.800188in}{0.613486in}}%
\pgfpathclose%
\pgfusepath{fill}%
\end{pgfscope}%
\begin{pgfscope}%
\pgfpathrectangle{\pgfqpoint{0.693757in}{0.613486in}}{\pgfqpoint{5.541243in}{3.963477in}}%
\pgfusepath{clip}%
\pgfsetbuttcap%
\pgfsetmiterjoin%
\definecolor{currentfill}{rgb}{0.000000,0.000000,1.000000}%
\pgfsetfillcolor{currentfill}%
\pgfsetlinewidth{0.000000pt}%
\definecolor{currentstroke}{rgb}{0.000000,0.000000,0.000000}%
\pgfsetstrokecolor{currentstroke}%
\pgfsetstrokeopacity{0.000000}%
\pgfsetdash{}{0pt}%
\pgfpathmoveto{\pgfqpoint{2.843318in}{0.613486in}}%
\pgfpathlineto{\pgfqpoint{2.877821in}{0.613486in}}%
\pgfpathlineto{\pgfqpoint{2.877821in}{4.576962in}}%
\pgfpathlineto{\pgfqpoint{2.843318in}{4.576962in}}%
\pgfpathlineto{\pgfqpoint{2.843318in}{0.613486in}}%
\pgfpathclose%
\pgfusepath{fill}%
\end{pgfscope}%
\begin{pgfscope}%
\pgfpathrectangle{\pgfqpoint{0.693757in}{0.613486in}}{\pgfqpoint{5.541243in}{3.963477in}}%
\pgfusepath{clip}%
\pgfsetbuttcap%
\pgfsetmiterjoin%
\definecolor{currentfill}{rgb}{0.000000,0.000000,1.000000}%
\pgfsetfillcolor{currentfill}%
\pgfsetlinewidth{0.000000pt}%
\definecolor{currentstroke}{rgb}{0.000000,0.000000,0.000000}%
\pgfsetstrokecolor{currentstroke}%
\pgfsetstrokeopacity{0.000000}%
\pgfsetdash{}{0pt}%
\pgfpathmoveto{\pgfqpoint{2.886447in}{0.613486in}}%
\pgfpathlineto{\pgfqpoint{2.920950in}{0.613486in}}%
\pgfpathlineto{\pgfqpoint{2.920950in}{4.576962in}}%
\pgfpathlineto{\pgfqpoint{2.886447in}{4.576962in}}%
\pgfpathlineto{\pgfqpoint{2.886447in}{0.613486in}}%
\pgfpathclose%
\pgfusepath{fill}%
\end{pgfscope}%
\begin{pgfscope}%
\pgfpathrectangle{\pgfqpoint{0.693757in}{0.613486in}}{\pgfqpoint{5.541243in}{3.963477in}}%
\pgfusepath{clip}%
\pgfsetbuttcap%
\pgfsetmiterjoin%
\definecolor{currentfill}{rgb}{0.000000,0.000000,1.000000}%
\pgfsetfillcolor{currentfill}%
\pgfsetlinewidth{0.000000pt}%
\definecolor{currentstroke}{rgb}{0.000000,0.000000,0.000000}%
\pgfsetstrokecolor{currentstroke}%
\pgfsetstrokeopacity{0.000000}%
\pgfsetdash{}{0pt}%
\pgfpathmoveto{\pgfqpoint{2.929576in}{0.613486in}}%
\pgfpathlineto{\pgfqpoint{2.964080in}{0.613486in}}%
\pgfpathlineto{\pgfqpoint{2.964080in}{4.576962in}}%
\pgfpathlineto{\pgfqpoint{2.929576in}{4.576962in}}%
\pgfpathlineto{\pgfqpoint{2.929576in}{0.613486in}}%
\pgfpathclose%
\pgfusepath{fill}%
\end{pgfscope}%
\begin{pgfscope}%
\pgfpathrectangle{\pgfqpoint{0.693757in}{0.613486in}}{\pgfqpoint{5.541243in}{3.963477in}}%
\pgfusepath{clip}%
\pgfsetbuttcap%
\pgfsetmiterjoin%
\definecolor{currentfill}{rgb}{0.000000,0.000000,1.000000}%
\pgfsetfillcolor{currentfill}%
\pgfsetlinewidth{0.000000pt}%
\definecolor{currentstroke}{rgb}{0.000000,0.000000,0.000000}%
\pgfsetstrokecolor{currentstroke}%
\pgfsetstrokeopacity{0.000000}%
\pgfsetdash{}{0pt}%
\pgfpathmoveto{\pgfqpoint{2.972705in}{0.613486in}}%
\pgfpathlineto{\pgfqpoint{3.007209in}{0.613486in}}%
\pgfpathlineto{\pgfqpoint{3.007209in}{4.576962in}}%
\pgfpathlineto{\pgfqpoint{2.972705in}{4.576962in}}%
\pgfpathlineto{\pgfqpoint{2.972705in}{0.613486in}}%
\pgfpathclose%
\pgfusepath{fill}%
\end{pgfscope}%
\begin{pgfscope}%
\pgfpathrectangle{\pgfqpoint{0.693757in}{0.613486in}}{\pgfqpoint{5.541243in}{3.963477in}}%
\pgfusepath{clip}%
\pgfsetbuttcap%
\pgfsetmiterjoin%
\definecolor{currentfill}{rgb}{0.000000,0.000000,1.000000}%
\pgfsetfillcolor{currentfill}%
\pgfsetlinewidth{0.000000pt}%
\definecolor{currentstroke}{rgb}{0.000000,0.000000,0.000000}%
\pgfsetstrokecolor{currentstroke}%
\pgfsetstrokeopacity{0.000000}%
\pgfsetdash{}{0pt}%
\pgfpathmoveto{\pgfqpoint{3.015835in}{0.613486in}}%
\pgfpathlineto{\pgfqpoint{3.050338in}{0.613486in}}%
\pgfpathlineto{\pgfqpoint{3.050338in}{4.576962in}}%
\pgfpathlineto{\pgfqpoint{3.015835in}{4.576962in}}%
\pgfpathlineto{\pgfqpoint{3.015835in}{0.613486in}}%
\pgfpathclose%
\pgfusepath{fill}%
\end{pgfscope}%
\begin{pgfscope}%
\pgfpathrectangle{\pgfqpoint{0.693757in}{0.613486in}}{\pgfqpoint{5.541243in}{3.963477in}}%
\pgfusepath{clip}%
\pgfsetbuttcap%
\pgfsetmiterjoin%
\definecolor{currentfill}{rgb}{0.000000,0.000000,1.000000}%
\pgfsetfillcolor{currentfill}%
\pgfsetlinewidth{0.000000pt}%
\definecolor{currentstroke}{rgb}{0.000000,0.000000,0.000000}%
\pgfsetstrokecolor{currentstroke}%
\pgfsetstrokeopacity{0.000000}%
\pgfsetdash{}{0pt}%
\pgfpathmoveto{\pgfqpoint{3.058964in}{0.613486in}}%
\pgfpathlineto{\pgfqpoint{3.093467in}{0.613486in}}%
\pgfpathlineto{\pgfqpoint{3.093467in}{4.576962in}}%
\pgfpathlineto{\pgfqpoint{3.058964in}{4.576962in}}%
\pgfpathlineto{\pgfqpoint{3.058964in}{0.613486in}}%
\pgfpathclose%
\pgfusepath{fill}%
\end{pgfscope}%
\begin{pgfscope}%
\pgfpathrectangle{\pgfqpoint{0.693757in}{0.613486in}}{\pgfqpoint{5.541243in}{3.963477in}}%
\pgfusepath{clip}%
\pgfsetbuttcap%
\pgfsetmiterjoin%
\definecolor{currentfill}{rgb}{0.000000,0.000000,1.000000}%
\pgfsetfillcolor{currentfill}%
\pgfsetlinewidth{0.000000pt}%
\definecolor{currentstroke}{rgb}{0.000000,0.000000,0.000000}%
\pgfsetstrokecolor{currentstroke}%
\pgfsetstrokeopacity{0.000000}%
\pgfsetdash{}{0pt}%
\pgfpathmoveto{\pgfqpoint{3.102093in}{0.613486in}}%
\pgfpathlineto{\pgfqpoint{3.136596in}{0.613486in}}%
\pgfpathlineto{\pgfqpoint{3.136596in}{4.576962in}}%
\pgfpathlineto{\pgfqpoint{3.102093in}{4.576962in}}%
\pgfpathlineto{\pgfqpoint{3.102093in}{0.613486in}}%
\pgfpathclose%
\pgfusepath{fill}%
\end{pgfscope}%
\begin{pgfscope}%
\pgfpathrectangle{\pgfqpoint{0.693757in}{0.613486in}}{\pgfqpoint{5.541243in}{3.963477in}}%
\pgfusepath{clip}%
\pgfsetbuttcap%
\pgfsetmiterjoin%
\definecolor{currentfill}{rgb}{0.000000,0.000000,1.000000}%
\pgfsetfillcolor{currentfill}%
\pgfsetlinewidth{0.000000pt}%
\definecolor{currentstroke}{rgb}{0.000000,0.000000,0.000000}%
\pgfsetstrokecolor{currentstroke}%
\pgfsetstrokeopacity{0.000000}%
\pgfsetdash{}{0pt}%
\pgfpathmoveto{\pgfqpoint{3.145222in}{0.613486in}}%
\pgfpathlineto{\pgfqpoint{3.179726in}{0.613486in}}%
\pgfpathlineto{\pgfqpoint{3.179726in}{4.576962in}}%
\pgfpathlineto{\pgfqpoint{3.145222in}{4.576962in}}%
\pgfpathlineto{\pgfqpoint{3.145222in}{0.613486in}}%
\pgfpathclose%
\pgfusepath{fill}%
\end{pgfscope}%
\begin{pgfscope}%
\pgfpathrectangle{\pgfqpoint{0.693757in}{0.613486in}}{\pgfqpoint{5.541243in}{3.963477in}}%
\pgfusepath{clip}%
\pgfsetbuttcap%
\pgfsetmiterjoin%
\definecolor{currentfill}{rgb}{0.000000,0.000000,1.000000}%
\pgfsetfillcolor{currentfill}%
\pgfsetlinewidth{0.000000pt}%
\definecolor{currentstroke}{rgb}{0.000000,0.000000,0.000000}%
\pgfsetstrokecolor{currentstroke}%
\pgfsetstrokeopacity{0.000000}%
\pgfsetdash{}{0pt}%
\pgfpathmoveto{\pgfqpoint{3.188351in}{0.613486in}}%
\pgfpathlineto{\pgfqpoint{3.222855in}{0.613486in}}%
\pgfpathlineto{\pgfqpoint{3.222855in}{4.576962in}}%
\pgfpathlineto{\pgfqpoint{3.188351in}{4.576962in}}%
\pgfpathlineto{\pgfqpoint{3.188351in}{0.613486in}}%
\pgfpathclose%
\pgfusepath{fill}%
\end{pgfscope}%
\begin{pgfscope}%
\pgfpathrectangle{\pgfqpoint{0.693757in}{0.613486in}}{\pgfqpoint{5.541243in}{3.963477in}}%
\pgfusepath{clip}%
\pgfsetbuttcap%
\pgfsetmiterjoin%
\definecolor{currentfill}{rgb}{0.000000,0.000000,1.000000}%
\pgfsetfillcolor{currentfill}%
\pgfsetlinewidth{0.000000pt}%
\definecolor{currentstroke}{rgb}{0.000000,0.000000,0.000000}%
\pgfsetstrokecolor{currentstroke}%
\pgfsetstrokeopacity{0.000000}%
\pgfsetdash{}{0pt}%
\pgfpathmoveto{\pgfqpoint{3.231481in}{0.613486in}}%
\pgfpathlineto{\pgfqpoint{3.265984in}{0.613486in}}%
\pgfpathlineto{\pgfqpoint{3.265984in}{4.576962in}}%
\pgfpathlineto{\pgfqpoint{3.231481in}{4.576962in}}%
\pgfpathlineto{\pgfqpoint{3.231481in}{0.613486in}}%
\pgfpathclose%
\pgfusepath{fill}%
\end{pgfscope}%
\begin{pgfscope}%
\pgfpathrectangle{\pgfqpoint{0.693757in}{0.613486in}}{\pgfqpoint{5.541243in}{3.963477in}}%
\pgfusepath{clip}%
\pgfsetbuttcap%
\pgfsetmiterjoin%
\definecolor{currentfill}{rgb}{0.000000,0.000000,1.000000}%
\pgfsetfillcolor{currentfill}%
\pgfsetlinewidth{0.000000pt}%
\definecolor{currentstroke}{rgb}{0.000000,0.000000,0.000000}%
\pgfsetstrokecolor{currentstroke}%
\pgfsetstrokeopacity{0.000000}%
\pgfsetdash{}{0pt}%
\pgfpathmoveto{\pgfqpoint{3.274610in}{0.613486in}}%
\pgfpathlineto{\pgfqpoint{3.309113in}{0.613486in}}%
\pgfpathlineto{\pgfqpoint{3.309113in}{4.576962in}}%
\pgfpathlineto{\pgfqpoint{3.274610in}{4.576962in}}%
\pgfpathlineto{\pgfqpoint{3.274610in}{0.613486in}}%
\pgfpathclose%
\pgfusepath{fill}%
\end{pgfscope}%
\begin{pgfscope}%
\pgfpathrectangle{\pgfqpoint{0.693757in}{0.613486in}}{\pgfqpoint{5.541243in}{3.963477in}}%
\pgfusepath{clip}%
\pgfsetbuttcap%
\pgfsetmiterjoin%
\definecolor{currentfill}{rgb}{0.000000,0.000000,1.000000}%
\pgfsetfillcolor{currentfill}%
\pgfsetlinewidth{0.000000pt}%
\definecolor{currentstroke}{rgb}{0.000000,0.000000,0.000000}%
\pgfsetstrokecolor{currentstroke}%
\pgfsetstrokeopacity{0.000000}%
\pgfsetdash{}{0pt}%
\pgfpathmoveto{\pgfqpoint{3.317739in}{0.613486in}}%
\pgfpathlineto{\pgfqpoint{3.352243in}{0.613486in}}%
\pgfpathlineto{\pgfqpoint{3.352243in}{4.576962in}}%
\pgfpathlineto{\pgfqpoint{3.317739in}{4.576962in}}%
\pgfpathlineto{\pgfqpoint{3.317739in}{0.613486in}}%
\pgfpathclose%
\pgfusepath{fill}%
\end{pgfscope}%
\begin{pgfscope}%
\pgfpathrectangle{\pgfqpoint{0.693757in}{0.613486in}}{\pgfqpoint{5.541243in}{3.963477in}}%
\pgfusepath{clip}%
\pgfsetbuttcap%
\pgfsetmiterjoin%
\definecolor{currentfill}{rgb}{0.000000,0.000000,1.000000}%
\pgfsetfillcolor{currentfill}%
\pgfsetlinewidth{0.000000pt}%
\definecolor{currentstroke}{rgb}{0.000000,0.000000,0.000000}%
\pgfsetstrokecolor{currentstroke}%
\pgfsetstrokeopacity{0.000000}%
\pgfsetdash{}{0pt}%
\pgfpathmoveto{\pgfqpoint{3.360868in}{0.613486in}}%
\pgfpathlineto{\pgfqpoint{3.395372in}{0.613486in}}%
\pgfpathlineto{\pgfqpoint{3.395372in}{4.576962in}}%
\pgfpathlineto{\pgfqpoint{3.360868in}{4.576962in}}%
\pgfpathlineto{\pgfqpoint{3.360868in}{0.613486in}}%
\pgfpathclose%
\pgfusepath{fill}%
\end{pgfscope}%
\begin{pgfscope}%
\pgfpathrectangle{\pgfqpoint{0.693757in}{0.613486in}}{\pgfqpoint{5.541243in}{3.963477in}}%
\pgfusepath{clip}%
\pgfsetbuttcap%
\pgfsetmiterjoin%
\definecolor{currentfill}{rgb}{0.000000,0.000000,1.000000}%
\pgfsetfillcolor{currentfill}%
\pgfsetlinewidth{0.000000pt}%
\definecolor{currentstroke}{rgb}{0.000000,0.000000,0.000000}%
\pgfsetstrokecolor{currentstroke}%
\pgfsetstrokeopacity{0.000000}%
\pgfsetdash{}{0pt}%
\pgfpathmoveto{\pgfqpoint{3.403998in}{0.613486in}}%
\pgfpathlineto{\pgfqpoint{3.438501in}{0.613486in}}%
\pgfpathlineto{\pgfqpoint{3.438501in}{4.576962in}}%
\pgfpathlineto{\pgfqpoint{3.403998in}{4.576962in}}%
\pgfpathlineto{\pgfqpoint{3.403998in}{0.613486in}}%
\pgfpathclose%
\pgfusepath{fill}%
\end{pgfscope}%
\begin{pgfscope}%
\pgfpathrectangle{\pgfqpoint{0.693757in}{0.613486in}}{\pgfqpoint{5.541243in}{3.963477in}}%
\pgfusepath{clip}%
\pgfsetbuttcap%
\pgfsetmiterjoin%
\definecolor{currentfill}{rgb}{0.000000,0.000000,1.000000}%
\pgfsetfillcolor{currentfill}%
\pgfsetlinewidth{0.000000pt}%
\definecolor{currentstroke}{rgb}{0.000000,0.000000,0.000000}%
\pgfsetstrokecolor{currentstroke}%
\pgfsetstrokeopacity{0.000000}%
\pgfsetdash{}{0pt}%
\pgfpathmoveto{\pgfqpoint{3.447127in}{0.613486in}}%
\pgfpathlineto{\pgfqpoint{3.481630in}{0.613486in}}%
\pgfpathlineto{\pgfqpoint{3.481630in}{4.576962in}}%
\pgfpathlineto{\pgfqpoint{3.447127in}{4.576962in}}%
\pgfpathlineto{\pgfqpoint{3.447127in}{0.613486in}}%
\pgfpathclose%
\pgfusepath{fill}%
\end{pgfscope}%
\begin{pgfscope}%
\pgfpathrectangle{\pgfqpoint{0.693757in}{0.613486in}}{\pgfqpoint{5.541243in}{3.963477in}}%
\pgfusepath{clip}%
\pgfsetbuttcap%
\pgfsetmiterjoin%
\definecolor{currentfill}{rgb}{0.000000,0.000000,1.000000}%
\pgfsetfillcolor{currentfill}%
\pgfsetlinewidth{0.000000pt}%
\definecolor{currentstroke}{rgb}{0.000000,0.000000,0.000000}%
\pgfsetstrokecolor{currentstroke}%
\pgfsetstrokeopacity{0.000000}%
\pgfsetdash{}{0pt}%
\pgfpathmoveto{\pgfqpoint{3.490256in}{0.613486in}}%
\pgfpathlineto{\pgfqpoint{3.524759in}{0.613486in}}%
\pgfpathlineto{\pgfqpoint{3.524759in}{4.576962in}}%
\pgfpathlineto{\pgfqpoint{3.490256in}{4.576962in}}%
\pgfpathlineto{\pgfqpoint{3.490256in}{0.613486in}}%
\pgfpathclose%
\pgfusepath{fill}%
\end{pgfscope}%
\begin{pgfscope}%
\pgfpathrectangle{\pgfqpoint{0.693757in}{0.613486in}}{\pgfqpoint{5.541243in}{3.963477in}}%
\pgfusepath{clip}%
\pgfsetbuttcap%
\pgfsetmiterjoin%
\definecolor{currentfill}{rgb}{0.000000,0.000000,1.000000}%
\pgfsetfillcolor{currentfill}%
\pgfsetlinewidth{0.000000pt}%
\definecolor{currentstroke}{rgb}{0.000000,0.000000,0.000000}%
\pgfsetstrokecolor{currentstroke}%
\pgfsetstrokeopacity{0.000000}%
\pgfsetdash{}{0pt}%
\pgfpathmoveto{\pgfqpoint{3.533385in}{0.613486in}}%
\pgfpathlineto{\pgfqpoint{3.567889in}{0.613486in}}%
\pgfpathlineto{\pgfqpoint{3.567889in}{4.576962in}}%
\pgfpathlineto{\pgfqpoint{3.533385in}{4.576962in}}%
\pgfpathlineto{\pgfqpoint{3.533385in}{0.613486in}}%
\pgfpathclose%
\pgfusepath{fill}%
\end{pgfscope}%
\begin{pgfscope}%
\pgfpathrectangle{\pgfqpoint{0.693757in}{0.613486in}}{\pgfqpoint{5.541243in}{3.963477in}}%
\pgfusepath{clip}%
\pgfsetbuttcap%
\pgfsetmiterjoin%
\definecolor{currentfill}{rgb}{0.000000,0.000000,1.000000}%
\pgfsetfillcolor{currentfill}%
\pgfsetlinewidth{0.000000pt}%
\definecolor{currentstroke}{rgb}{0.000000,0.000000,0.000000}%
\pgfsetstrokecolor{currentstroke}%
\pgfsetstrokeopacity{0.000000}%
\pgfsetdash{}{0pt}%
\pgfpathmoveto{\pgfqpoint{3.576515in}{0.613486in}}%
\pgfpathlineto{\pgfqpoint{3.611018in}{0.613486in}}%
\pgfpathlineto{\pgfqpoint{3.611018in}{4.576962in}}%
\pgfpathlineto{\pgfqpoint{3.576515in}{4.576962in}}%
\pgfpathlineto{\pgfqpoint{3.576515in}{0.613486in}}%
\pgfpathclose%
\pgfusepath{fill}%
\end{pgfscope}%
\begin{pgfscope}%
\pgfpathrectangle{\pgfqpoint{0.693757in}{0.613486in}}{\pgfqpoint{5.541243in}{3.963477in}}%
\pgfusepath{clip}%
\pgfsetbuttcap%
\pgfsetmiterjoin%
\definecolor{currentfill}{rgb}{0.000000,0.000000,1.000000}%
\pgfsetfillcolor{currentfill}%
\pgfsetlinewidth{0.000000pt}%
\definecolor{currentstroke}{rgb}{0.000000,0.000000,0.000000}%
\pgfsetstrokecolor{currentstroke}%
\pgfsetstrokeopacity{0.000000}%
\pgfsetdash{}{0pt}%
\pgfpathmoveto{\pgfqpoint{3.619644in}{0.613486in}}%
\pgfpathlineto{\pgfqpoint{3.654147in}{0.613486in}}%
\pgfpathlineto{\pgfqpoint{3.654147in}{4.576962in}}%
\pgfpathlineto{\pgfqpoint{3.619644in}{4.576962in}}%
\pgfpathlineto{\pgfqpoint{3.619644in}{0.613486in}}%
\pgfpathclose%
\pgfusepath{fill}%
\end{pgfscope}%
\begin{pgfscope}%
\pgfpathrectangle{\pgfqpoint{0.693757in}{0.613486in}}{\pgfqpoint{5.541243in}{3.963477in}}%
\pgfusepath{clip}%
\pgfsetbuttcap%
\pgfsetmiterjoin%
\definecolor{currentfill}{rgb}{0.000000,0.000000,1.000000}%
\pgfsetfillcolor{currentfill}%
\pgfsetlinewidth{0.000000pt}%
\definecolor{currentstroke}{rgb}{0.000000,0.000000,0.000000}%
\pgfsetstrokecolor{currentstroke}%
\pgfsetstrokeopacity{0.000000}%
\pgfsetdash{}{0pt}%
\pgfpathmoveto{\pgfqpoint{3.662773in}{0.613486in}}%
\pgfpathlineto{\pgfqpoint{3.697276in}{0.613486in}}%
\pgfpathlineto{\pgfqpoint{3.697276in}{4.576962in}}%
\pgfpathlineto{\pgfqpoint{3.662773in}{4.576962in}}%
\pgfpathlineto{\pgfqpoint{3.662773in}{0.613486in}}%
\pgfpathclose%
\pgfusepath{fill}%
\end{pgfscope}%
\begin{pgfscope}%
\pgfpathrectangle{\pgfqpoint{0.693757in}{0.613486in}}{\pgfqpoint{5.541243in}{3.963477in}}%
\pgfusepath{clip}%
\pgfsetbuttcap%
\pgfsetmiterjoin%
\definecolor{currentfill}{rgb}{0.000000,0.000000,1.000000}%
\pgfsetfillcolor{currentfill}%
\pgfsetlinewidth{0.000000pt}%
\definecolor{currentstroke}{rgb}{0.000000,0.000000,0.000000}%
\pgfsetstrokecolor{currentstroke}%
\pgfsetstrokeopacity{0.000000}%
\pgfsetdash{}{0pt}%
\pgfpathmoveto{\pgfqpoint{3.705902in}{0.613486in}}%
\pgfpathlineto{\pgfqpoint{3.740406in}{0.613486in}}%
\pgfpathlineto{\pgfqpoint{3.740406in}{4.576962in}}%
\pgfpathlineto{\pgfqpoint{3.705902in}{4.576962in}}%
\pgfpathlineto{\pgfqpoint{3.705902in}{0.613486in}}%
\pgfpathclose%
\pgfusepath{fill}%
\end{pgfscope}%
\begin{pgfscope}%
\pgfpathrectangle{\pgfqpoint{0.693757in}{0.613486in}}{\pgfqpoint{5.541243in}{3.963477in}}%
\pgfusepath{clip}%
\pgfsetbuttcap%
\pgfsetmiterjoin%
\definecolor{currentfill}{rgb}{0.000000,0.000000,1.000000}%
\pgfsetfillcolor{currentfill}%
\pgfsetlinewidth{0.000000pt}%
\definecolor{currentstroke}{rgb}{0.000000,0.000000,0.000000}%
\pgfsetstrokecolor{currentstroke}%
\pgfsetstrokeopacity{0.000000}%
\pgfsetdash{}{0pt}%
\pgfpathmoveto{\pgfqpoint{3.749031in}{0.613486in}}%
\pgfpathlineto{\pgfqpoint{3.783535in}{0.613486in}}%
\pgfpathlineto{\pgfqpoint{3.783535in}{4.576962in}}%
\pgfpathlineto{\pgfqpoint{3.749031in}{4.576962in}}%
\pgfpathlineto{\pgfqpoint{3.749031in}{0.613486in}}%
\pgfpathclose%
\pgfusepath{fill}%
\end{pgfscope}%
\begin{pgfscope}%
\pgfpathrectangle{\pgfqpoint{0.693757in}{0.613486in}}{\pgfqpoint{5.541243in}{3.963477in}}%
\pgfusepath{clip}%
\pgfsetbuttcap%
\pgfsetmiterjoin%
\definecolor{currentfill}{rgb}{0.000000,0.000000,1.000000}%
\pgfsetfillcolor{currentfill}%
\pgfsetlinewidth{0.000000pt}%
\definecolor{currentstroke}{rgb}{0.000000,0.000000,0.000000}%
\pgfsetstrokecolor{currentstroke}%
\pgfsetstrokeopacity{0.000000}%
\pgfsetdash{}{0pt}%
\pgfpathmoveto{\pgfqpoint{3.792161in}{0.613486in}}%
\pgfpathlineto{\pgfqpoint{3.826664in}{0.613486in}}%
\pgfpathlineto{\pgfqpoint{3.826664in}{4.576962in}}%
\pgfpathlineto{\pgfqpoint{3.792161in}{4.576962in}}%
\pgfpathlineto{\pgfqpoint{3.792161in}{0.613486in}}%
\pgfpathclose%
\pgfusepath{fill}%
\end{pgfscope}%
\begin{pgfscope}%
\pgfpathrectangle{\pgfqpoint{0.693757in}{0.613486in}}{\pgfqpoint{5.541243in}{3.963477in}}%
\pgfusepath{clip}%
\pgfsetbuttcap%
\pgfsetmiterjoin%
\definecolor{currentfill}{rgb}{0.000000,0.000000,1.000000}%
\pgfsetfillcolor{currentfill}%
\pgfsetlinewidth{0.000000pt}%
\definecolor{currentstroke}{rgb}{0.000000,0.000000,0.000000}%
\pgfsetstrokecolor{currentstroke}%
\pgfsetstrokeopacity{0.000000}%
\pgfsetdash{}{0pt}%
\pgfpathmoveto{\pgfqpoint{3.835290in}{0.613486in}}%
\pgfpathlineto{\pgfqpoint{3.869793in}{0.613486in}}%
\pgfpathlineto{\pgfqpoint{3.869793in}{4.576962in}}%
\pgfpathlineto{\pgfqpoint{3.835290in}{4.576962in}}%
\pgfpathlineto{\pgfqpoint{3.835290in}{0.613486in}}%
\pgfpathclose%
\pgfusepath{fill}%
\end{pgfscope}%
\begin{pgfscope}%
\pgfpathrectangle{\pgfqpoint{0.693757in}{0.613486in}}{\pgfqpoint{5.541243in}{3.963477in}}%
\pgfusepath{clip}%
\pgfsetbuttcap%
\pgfsetmiterjoin%
\definecolor{currentfill}{rgb}{0.000000,0.000000,1.000000}%
\pgfsetfillcolor{currentfill}%
\pgfsetlinewidth{0.000000pt}%
\definecolor{currentstroke}{rgb}{0.000000,0.000000,0.000000}%
\pgfsetstrokecolor{currentstroke}%
\pgfsetstrokeopacity{0.000000}%
\pgfsetdash{}{0pt}%
\pgfpathmoveto{\pgfqpoint{3.878419in}{0.613486in}}%
\pgfpathlineto{\pgfqpoint{3.912922in}{0.613486in}}%
\pgfpathlineto{\pgfqpoint{3.912922in}{4.576962in}}%
\pgfpathlineto{\pgfqpoint{3.878419in}{4.576962in}}%
\pgfpathlineto{\pgfqpoint{3.878419in}{0.613486in}}%
\pgfpathclose%
\pgfusepath{fill}%
\end{pgfscope}%
\begin{pgfscope}%
\pgfpathrectangle{\pgfqpoint{0.693757in}{0.613486in}}{\pgfqpoint{5.541243in}{3.963477in}}%
\pgfusepath{clip}%
\pgfsetbuttcap%
\pgfsetmiterjoin%
\definecolor{currentfill}{rgb}{0.000000,0.000000,1.000000}%
\pgfsetfillcolor{currentfill}%
\pgfsetlinewidth{0.000000pt}%
\definecolor{currentstroke}{rgb}{0.000000,0.000000,0.000000}%
\pgfsetstrokecolor{currentstroke}%
\pgfsetstrokeopacity{0.000000}%
\pgfsetdash{}{0pt}%
\pgfpathmoveto{\pgfqpoint{3.921548in}{0.613486in}}%
\pgfpathlineto{\pgfqpoint{3.956052in}{0.613486in}}%
\pgfpathlineto{\pgfqpoint{3.956052in}{4.576962in}}%
\pgfpathlineto{\pgfqpoint{3.921548in}{4.576962in}}%
\pgfpathlineto{\pgfqpoint{3.921548in}{0.613486in}}%
\pgfpathclose%
\pgfusepath{fill}%
\end{pgfscope}%
\begin{pgfscope}%
\pgfpathrectangle{\pgfqpoint{0.693757in}{0.613486in}}{\pgfqpoint{5.541243in}{3.963477in}}%
\pgfusepath{clip}%
\pgfsetbuttcap%
\pgfsetmiterjoin%
\definecolor{currentfill}{rgb}{0.000000,0.000000,1.000000}%
\pgfsetfillcolor{currentfill}%
\pgfsetlinewidth{0.000000pt}%
\definecolor{currentstroke}{rgb}{0.000000,0.000000,0.000000}%
\pgfsetstrokecolor{currentstroke}%
\pgfsetstrokeopacity{0.000000}%
\pgfsetdash{}{0pt}%
\pgfpathmoveto{\pgfqpoint{3.964678in}{0.613486in}}%
\pgfpathlineto{\pgfqpoint{3.999181in}{0.613486in}}%
\pgfpathlineto{\pgfqpoint{3.999181in}{4.576962in}}%
\pgfpathlineto{\pgfqpoint{3.964678in}{4.576962in}}%
\pgfpathlineto{\pgfqpoint{3.964678in}{0.613486in}}%
\pgfpathclose%
\pgfusepath{fill}%
\end{pgfscope}%
\begin{pgfscope}%
\pgfpathrectangle{\pgfqpoint{0.693757in}{0.613486in}}{\pgfqpoint{5.541243in}{3.963477in}}%
\pgfusepath{clip}%
\pgfsetbuttcap%
\pgfsetmiterjoin%
\definecolor{currentfill}{rgb}{0.000000,0.000000,1.000000}%
\pgfsetfillcolor{currentfill}%
\pgfsetlinewidth{0.000000pt}%
\definecolor{currentstroke}{rgb}{0.000000,0.000000,0.000000}%
\pgfsetstrokecolor{currentstroke}%
\pgfsetstrokeopacity{0.000000}%
\pgfsetdash{}{0pt}%
\pgfpathmoveto{\pgfqpoint{4.007807in}{0.613486in}}%
\pgfpathlineto{\pgfqpoint{4.042310in}{0.613486in}}%
\pgfpathlineto{\pgfqpoint{4.042310in}{4.576962in}}%
\pgfpathlineto{\pgfqpoint{4.007807in}{4.576962in}}%
\pgfpathlineto{\pgfqpoint{4.007807in}{0.613486in}}%
\pgfpathclose%
\pgfusepath{fill}%
\end{pgfscope}%
\begin{pgfscope}%
\pgfpathrectangle{\pgfqpoint{0.693757in}{0.613486in}}{\pgfqpoint{5.541243in}{3.963477in}}%
\pgfusepath{clip}%
\pgfsetbuttcap%
\pgfsetmiterjoin%
\definecolor{currentfill}{rgb}{0.000000,0.000000,1.000000}%
\pgfsetfillcolor{currentfill}%
\pgfsetlinewidth{0.000000pt}%
\definecolor{currentstroke}{rgb}{0.000000,0.000000,0.000000}%
\pgfsetstrokecolor{currentstroke}%
\pgfsetstrokeopacity{0.000000}%
\pgfsetdash{}{0pt}%
\pgfpathmoveto{\pgfqpoint{4.050936in}{0.613486in}}%
\pgfpathlineto{\pgfqpoint{4.085439in}{0.613486in}}%
\pgfpathlineto{\pgfqpoint{4.085439in}{4.576962in}}%
\pgfpathlineto{\pgfqpoint{4.050936in}{4.576962in}}%
\pgfpathlineto{\pgfqpoint{4.050936in}{0.613486in}}%
\pgfpathclose%
\pgfusepath{fill}%
\end{pgfscope}%
\begin{pgfscope}%
\pgfpathrectangle{\pgfqpoint{0.693757in}{0.613486in}}{\pgfqpoint{5.541243in}{3.963477in}}%
\pgfusepath{clip}%
\pgfsetbuttcap%
\pgfsetmiterjoin%
\definecolor{currentfill}{rgb}{0.000000,0.000000,1.000000}%
\pgfsetfillcolor{currentfill}%
\pgfsetlinewidth{0.000000pt}%
\definecolor{currentstroke}{rgb}{0.000000,0.000000,0.000000}%
\pgfsetstrokecolor{currentstroke}%
\pgfsetstrokeopacity{0.000000}%
\pgfsetdash{}{0pt}%
\pgfpathmoveto{\pgfqpoint{4.094065in}{0.613486in}}%
\pgfpathlineto{\pgfqpoint{4.128569in}{0.613486in}}%
\pgfpathlineto{\pgfqpoint{4.128569in}{4.576962in}}%
\pgfpathlineto{\pgfqpoint{4.094065in}{4.576962in}}%
\pgfpathlineto{\pgfqpoint{4.094065in}{0.613486in}}%
\pgfpathclose%
\pgfusepath{fill}%
\end{pgfscope}%
\begin{pgfscope}%
\pgfpathrectangle{\pgfqpoint{0.693757in}{0.613486in}}{\pgfqpoint{5.541243in}{3.963477in}}%
\pgfusepath{clip}%
\pgfsetbuttcap%
\pgfsetmiterjoin%
\definecolor{currentfill}{rgb}{0.000000,0.000000,1.000000}%
\pgfsetfillcolor{currentfill}%
\pgfsetlinewidth{0.000000pt}%
\definecolor{currentstroke}{rgb}{0.000000,0.000000,0.000000}%
\pgfsetstrokecolor{currentstroke}%
\pgfsetstrokeopacity{0.000000}%
\pgfsetdash{}{0pt}%
\pgfpathmoveto{\pgfqpoint{4.137194in}{0.613486in}}%
\pgfpathlineto{\pgfqpoint{4.171698in}{0.613486in}}%
\pgfpathlineto{\pgfqpoint{4.171698in}{4.576962in}}%
\pgfpathlineto{\pgfqpoint{4.137194in}{4.576962in}}%
\pgfpathlineto{\pgfqpoint{4.137194in}{0.613486in}}%
\pgfpathclose%
\pgfusepath{fill}%
\end{pgfscope}%
\begin{pgfscope}%
\pgfpathrectangle{\pgfqpoint{0.693757in}{0.613486in}}{\pgfqpoint{5.541243in}{3.963477in}}%
\pgfusepath{clip}%
\pgfsetbuttcap%
\pgfsetmiterjoin%
\definecolor{currentfill}{rgb}{0.000000,0.000000,1.000000}%
\pgfsetfillcolor{currentfill}%
\pgfsetlinewidth{0.000000pt}%
\definecolor{currentstroke}{rgb}{0.000000,0.000000,0.000000}%
\pgfsetstrokecolor{currentstroke}%
\pgfsetstrokeopacity{0.000000}%
\pgfsetdash{}{0pt}%
\pgfpathmoveto{\pgfqpoint{4.180324in}{0.613486in}}%
\pgfpathlineto{\pgfqpoint{4.214827in}{0.613486in}}%
\pgfpathlineto{\pgfqpoint{4.214827in}{4.576962in}}%
\pgfpathlineto{\pgfqpoint{4.180324in}{4.576962in}}%
\pgfpathlineto{\pgfqpoint{4.180324in}{0.613486in}}%
\pgfpathclose%
\pgfusepath{fill}%
\end{pgfscope}%
\begin{pgfscope}%
\pgfpathrectangle{\pgfqpoint{0.693757in}{0.613486in}}{\pgfqpoint{5.541243in}{3.963477in}}%
\pgfusepath{clip}%
\pgfsetbuttcap%
\pgfsetmiterjoin%
\definecolor{currentfill}{rgb}{0.000000,0.000000,1.000000}%
\pgfsetfillcolor{currentfill}%
\pgfsetlinewidth{0.000000pt}%
\definecolor{currentstroke}{rgb}{0.000000,0.000000,0.000000}%
\pgfsetstrokecolor{currentstroke}%
\pgfsetstrokeopacity{0.000000}%
\pgfsetdash{}{0pt}%
\pgfpathmoveto{\pgfqpoint{4.223453in}{0.613486in}}%
\pgfpathlineto{\pgfqpoint{4.257956in}{0.613486in}}%
\pgfpathlineto{\pgfqpoint{4.257956in}{4.576962in}}%
\pgfpathlineto{\pgfqpoint{4.223453in}{4.576962in}}%
\pgfpathlineto{\pgfqpoint{4.223453in}{0.613486in}}%
\pgfpathclose%
\pgfusepath{fill}%
\end{pgfscope}%
\begin{pgfscope}%
\pgfpathrectangle{\pgfqpoint{0.693757in}{0.613486in}}{\pgfqpoint{5.541243in}{3.963477in}}%
\pgfusepath{clip}%
\pgfsetbuttcap%
\pgfsetmiterjoin%
\definecolor{currentfill}{rgb}{0.000000,0.000000,1.000000}%
\pgfsetfillcolor{currentfill}%
\pgfsetlinewidth{0.000000pt}%
\definecolor{currentstroke}{rgb}{0.000000,0.000000,0.000000}%
\pgfsetstrokecolor{currentstroke}%
\pgfsetstrokeopacity{0.000000}%
\pgfsetdash{}{0pt}%
\pgfpathmoveto{\pgfqpoint{4.266582in}{0.613486in}}%
\pgfpathlineto{\pgfqpoint{4.301086in}{0.613486in}}%
\pgfpathlineto{\pgfqpoint{4.301086in}{4.576962in}}%
\pgfpathlineto{\pgfqpoint{4.266582in}{4.576962in}}%
\pgfpathlineto{\pgfqpoint{4.266582in}{0.613486in}}%
\pgfpathclose%
\pgfusepath{fill}%
\end{pgfscope}%
\begin{pgfscope}%
\pgfpathrectangle{\pgfqpoint{0.693757in}{0.613486in}}{\pgfqpoint{5.541243in}{3.963477in}}%
\pgfusepath{clip}%
\pgfsetbuttcap%
\pgfsetmiterjoin%
\definecolor{currentfill}{rgb}{0.000000,0.000000,1.000000}%
\pgfsetfillcolor{currentfill}%
\pgfsetlinewidth{0.000000pt}%
\definecolor{currentstroke}{rgb}{0.000000,0.000000,0.000000}%
\pgfsetstrokecolor{currentstroke}%
\pgfsetstrokeopacity{0.000000}%
\pgfsetdash{}{0pt}%
\pgfpathmoveto{\pgfqpoint{4.309711in}{0.613486in}}%
\pgfpathlineto{\pgfqpoint{4.344215in}{0.613486in}}%
\pgfpathlineto{\pgfqpoint{4.344215in}{4.576962in}}%
\pgfpathlineto{\pgfqpoint{4.309711in}{4.576962in}}%
\pgfpathlineto{\pgfqpoint{4.309711in}{0.613486in}}%
\pgfpathclose%
\pgfusepath{fill}%
\end{pgfscope}%
\begin{pgfscope}%
\pgfpathrectangle{\pgfqpoint{0.693757in}{0.613486in}}{\pgfqpoint{5.541243in}{3.963477in}}%
\pgfusepath{clip}%
\pgfsetbuttcap%
\pgfsetmiterjoin%
\definecolor{currentfill}{rgb}{0.000000,0.000000,1.000000}%
\pgfsetfillcolor{currentfill}%
\pgfsetlinewidth{0.000000pt}%
\definecolor{currentstroke}{rgb}{0.000000,0.000000,0.000000}%
\pgfsetstrokecolor{currentstroke}%
\pgfsetstrokeopacity{0.000000}%
\pgfsetdash{}{0pt}%
\pgfpathmoveto{\pgfqpoint{4.352841in}{0.613486in}}%
\pgfpathlineto{\pgfqpoint{4.387344in}{0.613486in}}%
\pgfpathlineto{\pgfqpoint{4.387344in}{4.576962in}}%
\pgfpathlineto{\pgfqpoint{4.352841in}{4.576962in}}%
\pgfpathlineto{\pgfqpoint{4.352841in}{0.613486in}}%
\pgfpathclose%
\pgfusepath{fill}%
\end{pgfscope}%
\begin{pgfscope}%
\pgfpathrectangle{\pgfqpoint{0.693757in}{0.613486in}}{\pgfqpoint{5.541243in}{3.963477in}}%
\pgfusepath{clip}%
\pgfsetbuttcap%
\pgfsetmiterjoin%
\definecolor{currentfill}{rgb}{0.000000,0.000000,1.000000}%
\pgfsetfillcolor{currentfill}%
\pgfsetlinewidth{0.000000pt}%
\definecolor{currentstroke}{rgb}{0.000000,0.000000,0.000000}%
\pgfsetstrokecolor{currentstroke}%
\pgfsetstrokeopacity{0.000000}%
\pgfsetdash{}{0pt}%
\pgfpathmoveto{\pgfqpoint{4.395970in}{0.613486in}}%
\pgfpathlineto{\pgfqpoint{4.430473in}{0.613486in}}%
\pgfpathlineto{\pgfqpoint{4.430473in}{4.576962in}}%
\pgfpathlineto{\pgfqpoint{4.395970in}{4.576962in}}%
\pgfpathlineto{\pgfqpoint{4.395970in}{0.613486in}}%
\pgfpathclose%
\pgfusepath{fill}%
\end{pgfscope}%
\begin{pgfscope}%
\pgfpathrectangle{\pgfqpoint{0.693757in}{0.613486in}}{\pgfqpoint{5.541243in}{3.963477in}}%
\pgfusepath{clip}%
\pgfsetbuttcap%
\pgfsetmiterjoin%
\definecolor{currentfill}{rgb}{0.000000,0.000000,1.000000}%
\pgfsetfillcolor{currentfill}%
\pgfsetlinewidth{0.000000pt}%
\definecolor{currentstroke}{rgb}{0.000000,0.000000,0.000000}%
\pgfsetstrokecolor{currentstroke}%
\pgfsetstrokeopacity{0.000000}%
\pgfsetdash{}{0pt}%
\pgfpathmoveto{\pgfqpoint{4.439099in}{0.613486in}}%
\pgfpathlineto{\pgfqpoint{4.473602in}{0.613486in}}%
\pgfpathlineto{\pgfqpoint{4.473602in}{4.576962in}}%
\pgfpathlineto{\pgfqpoint{4.439099in}{4.576962in}}%
\pgfpathlineto{\pgfqpoint{4.439099in}{0.613486in}}%
\pgfpathclose%
\pgfusepath{fill}%
\end{pgfscope}%
\begin{pgfscope}%
\pgfpathrectangle{\pgfqpoint{0.693757in}{0.613486in}}{\pgfqpoint{5.541243in}{3.963477in}}%
\pgfusepath{clip}%
\pgfsetbuttcap%
\pgfsetmiterjoin%
\definecolor{currentfill}{rgb}{0.000000,0.000000,1.000000}%
\pgfsetfillcolor{currentfill}%
\pgfsetlinewidth{0.000000pt}%
\definecolor{currentstroke}{rgb}{0.000000,0.000000,0.000000}%
\pgfsetstrokecolor{currentstroke}%
\pgfsetstrokeopacity{0.000000}%
\pgfsetdash{}{0pt}%
\pgfpathmoveto{\pgfqpoint{4.482228in}{0.613486in}}%
\pgfpathlineto{\pgfqpoint{4.516732in}{0.613486in}}%
\pgfpathlineto{\pgfqpoint{4.516732in}{4.576962in}}%
\pgfpathlineto{\pgfqpoint{4.482228in}{4.576962in}}%
\pgfpathlineto{\pgfqpoint{4.482228in}{0.613486in}}%
\pgfpathclose%
\pgfusepath{fill}%
\end{pgfscope}%
\begin{pgfscope}%
\pgfpathrectangle{\pgfqpoint{0.693757in}{0.613486in}}{\pgfqpoint{5.541243in}{3.963477in}}%
\pgfusepath{clip}%
\pgfsetbuttcap%
\pgfsetmiterjoin%
\definecolor{currentfill}{rgb}{0.000000,0.000000,1.000000}%
\pgfsetfillcolor{currentfill}%
\pgfsetlinewidth{0.000000pt}%
\definecolor{currentstroke}{rgb}{0.000000,0.000000,0.000000}%
\pgfsetstrokecolor{currentstroke}%
\pgfsetstrokeopacity{0.000000}%
\pgfsetdash{}{0pt}%
\pgfpathmoveto{\pgfqpoint{4.525357in}{0.613486in}}%
\pgfpathlineto{\pgfqpoint{4.559861in}{0.613486in}}%
\pgfpathlineto{\pgfqpoint{4.559861in}{4.576962in}}%
\pgfpathlineto{\pgfqpoint{4.525357in}{4.576962in}}%
\pgfpathlineto{\pgfqpoint{4.525357in}{0.613486in}}%
\pgfpathclose%
\pgfusepath{fill}%
\end{pgfscope}%
\begin{pgfscope}%
\pgfpathrectangle{\pgfqpoint{0.693757in}{0.613486in}}{\pgfqpoint{5.541243in}{3.963477in}}%
\pgfusepath{clip}%
\pgfsetbuttcap%
\pgfsetmiterjoin%
\definecolor{currentfill}{rgb}{0.000000,0.000000,1.000000}%
\pgfsetfillcolor{currentfill}%
\pgfsetlinewidth{0.000000pt}%
\definecolor{currentstroke}{rgb}{0.000000,0.000000,0.000000}%
\pgfsetstrokecolor{currentstroke}%
\pgfsetstrokeopacity{0.000000}%
\pgfsetdash{}{0pt}%
\pgfpathmoveto{\pgfqpoint{4.568487in}{0.613486in}}%
\pgfpathlineto{\pgfqpoint{4.602990in}{0.613486in}}%
\pgfpathlineto{\pgfqpoint{4.602990in}{4.576962in}}%
\pgfpathlineto{\pgfqpoint{4.568487in}{4.576962in}}%
\pgfpathlineto{\pgfqpoint{4.568487in}{0.613486in}}%
\pgfpathclose%
\pgfusepath{fill}%
\end{pgfscope}%
\begin{pgfscope}%
\pgfpathrectangle{\pgfqpoint{0.693757in}{0.613486in}}{\pgfqpoint{5.541243in}{3.963477in}}%
\pgfusepath{clip}%
\pgfsetbuttcap%
\pgfsetmiterjoin%
\definecolor{currentfill}{rgb}{0.000000,0.000000,1.000000}%
\pgfsetfillcolor{currentfill}%
\pgfsetlinewidth{0.000000pt}%
\definecolor{currentstroke}{rgb}{0.000000,0.000000,0.000000}%
\pgfsetstrokecolor{currentstroke}%
\pgfsetstrokeopacity{0.000000}%
\pgfsetdash{}{0pt}%
\pgfpathmoveto{\pgfqpoint{4.611616in}{0.613486in}}%
\pgfpathlineto{\pgfqpoint{4.646119in}{0.613486in}}%
\pgfpathlineto{\pgfqpoint{4.646119in}{4.576962in}}%
\pgfpathlineto{\pgfqpoint{4.611616in}{4.576962in}}%
\pgfpathlineto{\pgfqpoint{4.611616in}{0.613486in}}%
\pgfpathclose%
\pgfusepath{fill}%
\end{pgfscope}%
\begin{pgfscope}%
\pgfpathrectangle{\pgfqpoint{0.693757in}{0.613486in}}{\pgfqpoint{5.541243in}{3.963477in}}%
\pgfusepath{clip}%
\pgfsetbuttcap%
\pgfsetmiterjoin%
\definecolor{currentfill}{rgb}{0.000000,0.000000,1.000000}%
\pgfsetfillcolor{currentfill}%
\pgfsetlinewidth{0.000000pt}%
\definecolor{currentstroke}{rgb}{0.000000,0.000000,0.000000}%
\pgfsetstrokecolor{currentstroke}%
\pgfsetstrokeopacity{0.000000}%
\pgfsetdash{}{0pt}%
\pgfpathmoveto{\pgfqpoint{4.654745in}{0.613486in}}%
\pgfpathlineto{\pgfqpoint{4.689249in}{0.613486in}}%
\pgfpathlineto{\pgfqpoint{4.689249in}{4.576962in}}%
\pgfpathlineto{\pgfqpoint{4.654745in}{4.576962in}}%
\pgfpathlineto{\pgfqpoint{4.654745in}{0.613486in}}%
\pgfpathclose%
\pgfusepath{fill}%
\end{pgfscope}%
\begin{pgfscope}%
\pgfpathrectangle{\pgfqpoint{0.693757in}{0.613486in}}{\pgfqpoint{5.541243in}{3.963477in}}%
\pgfusepath{clip}%
\pgfsetbuttcap%
\pgfsetmiterjoin%
\definecolor{currentfill}{rgb}{0.000000,0.000000,1.000000}%
\pgfsetfillcolor{currentfill}%
\pgfsetlinewidth{0.000000pt}%
\definecolor{currentstroke}{rgb}{0.000000,0.000000,0.000000}%
\pgfsetstrokecolor{currentstroke}%
\pgfsetstrokeopacity{0.000000}%
\pgfsetdash{}{0pt}%
\pgfpathmoveto{\pgfqpoint{4.697874in}{0.613486in}}%
\pgfpathlineto{\pgfqpoint{4.732378in}{0.613486in}}%
\pgfpathlineto{\pgfqpoint{4.732378in}{4.576962in}}%
\pgfpathlineto{\pgfqpoint{4.697874in}{4.576962in}}%
\pgfpathlineto{\pgfqpoint{4.697874in}{0.613486in}}%
\pgfpathclose%
\pgfusepath{fill}%
\end{pgfscope}%
\begin{pgfscope}%
\pgfpathrectangle{\pgfqpoint{0.693757in}{0.613486in}}{\pgfqpoint{5.541243in}{3.963477in}}%
\pgfusepath{clip}%
\pgfsetbuttcap%
\pgfsetmiterjoin%
\definecolor{currentfill}{rgb}{0.000000,0.000000,1.000000}%
\pgfsetfillcolor{currentfill}%
\pgfsetlinewidth{0.000000pt}%
\definecolor{currentstroke}{rgb}{0.000000,0.000000,0.000000}%
\pgfsetstrokecolor{currentstroke}%
\pgfsetstrokeopacity{0.000000}%
\pgfsetdash{}{0pt}%
\pgfpathmoveto{\pgfqpoint{4.741004in}{0.613486in}}%
\pgfpathlineto{\pgfqpoint{4.775507in}{0.613486in}}%
\pgfpathlineto{\pgfqpoint{4.775507in}{4.576962in}}%
\pgfpathlineto{\pgfqpoint{4.741004in}{4.576962in}}%
\pgfpathlineto{\pgfqpoint{4.741004in}{0.613486in}}%
\pgfpathclose%
\pgfusepath{fill}%
\end{pgfscope}%
\begin{pgfscope}%
\pgfpathrectangle{\pgfqpoint{0.693757in}{0.613486in}}{\pgfqpoint{5.541243in}{3.963477in}}%
\pgfusepath{clip}%
\pgfsetbuttcap%
\pgfsetmiterjoin%
\definecolor{currentfill}{rgb}{0.000000,0.000000,1.000000}%
\pgfsetfillcolor{currentfill}%
\pgfsetlinewidth{0.000000pt}%
\definecolor{currentstroke}{rgb}{0.000000,0.000000,0.000000}%
\pgfsetstrokecolor{currentstroke}%
\pgfsetstrokeopacity{0.000000}%
\pgfsetdash{}{0pt}%
\pgfpathmoveto{\pgfqpoint{4.784133in}{0.613486in}}%
\pgfpathlineto{\pgfqpoint{4.818636in}{0.613486in}}%
\pgfpathlineto{\pgfqpoint{4.818636in}{4.576962in}}%
\pgfpathlineto{\pgfqpoint{4.784133in}{4.576962in}}%
\pgfpathlineto{\pgfqpoint{4.784133in}{0.613486in}}%
\pgfpathclose%
\pgfusepath{fill}%
\end{pgfscope}%
\begin{pgfscope}%
\pgfpathrectangle{\pgfqpoint{0.693757in}{0.613486in}}{\pgfqpoint{5.541243in}{3.963477in}}%
\pgfusepath{clip}%
\pgfsetbuttcap%
\pgfsetmiterjoin%
\definecolor{currentfill}{rgb}{0.000000,0.000000,1.000000}%
\pgfsetfillcolor{currentfill}%
\pgfsetlinewidth{0.000000pt}%
\definecolor{currentstroke}{rgb}{0.000000,0.000000,0.000000}%
\pgfsetstrokecolor{currentstroke}%
\pgfsetstrokeopacity{0.000000}%
\pgfsetdash{}{0pt}%
\pgfpathmoveto{\pgfqpoint{4.827262in}{0.613486in}}%
\pgfpathlineto{\pgfqpoint{4.861765in}{0.613486in}}%
\pgfpathlineto{\pgfqpoint{4.861765in}{4.576962in}}%
\pgfpathlineto{\pgfqpoint{4.827262in}{4.576962in}}%
\pgfpathlineto{\pgfqpoint{4.827262in}{0.613486in}}%
\pgfpathclose%
\pgfusepath{fill}%
\end{pgfscope}%
\begin{pgfscope}%
\pgfpathrectangle{\pgfqpoint{0.693757in}{0.613486in}}{\pgfqpoint{5.541243in}{3.963477in}}%
\pgfusepath{clip}%
\pgfsetbuttcap%
\pgfsetmiterjoin%
\definecolor{currentfill}{rgb}{0.000000,0.000000,1.000000}%
\pgfsetfillcolor{currentfill}%
\pgfsetlinewidth{0.000000pt}%
\definecolor{currentstroke}{rgb}{0.000000,0.000000,0.000000}%
\pgfsetstrokecolor{currentstroke}%
\pgfsetstrokeopacity{0.000000}%
\pgfsetdash{}{0pt}%
\pgfpathmoveto{\pgfqpoint{4.870391in}{0.613486in}}%
\pgfpathlineto{\pgfqpoint{4.904895in}{0.613486in}}%
\pgfpathlineto{\pgfqpoint{4.904895in}{4.576962in}}%
\pgfpathlineto{\pgfqpoint{4.870391in}{4.576962in}}%
\pgfpathlineto{\pgfqpoint{4.870391in}{0.613486in}}%
\pgfpathclose%
\pgfusepath{fill}%
\end{pgfscope}%
\begin{pgfscope}%
\pgfpathrectangle{\pgfqpoint{0.693757in}{0.613486in}}{\pgfqpoint{5.541243in}{3.963477in}}%
\pgfusepath{clip}%
\pgfsetbuttcap%
\pgfsetmiterjoin%
\definecolor{currentfill}{rgb}{0.000000,0.000000,1.000000}%
\pgfsetfillcolor{currentfill}%
\pgfsetlinewidth{0.000000pt}%
\definecolor{currentstroke}{rgb}{0.000000,0.000000,0.000000}%
\pgfsetstrokecolor{currentstroke}%
\pgfsetstrokeopacity{0.000000}%
\pgfsetdash{}{0pt}%
\pgfpathmoveto{\pgfqpoint{4.913521in}{0.613486in}}%
\pgfpathlineto{\pgfqpoint{4.948024in}{0.613486in}}%
\pgfpathlineto{\pgfqpoint{4.948024in}{4.576962in}}%
\pgfpathlineto{\pgfqpoint{4.913521in}{4.576962in}}%
\pgfpathlineto{\pgfqpoint{4.913521in}{0.613486in}}%
\pgfpathclose%
\pgfusepath{fill}%
\end{pgfscope}%
\begin{pgfscope}%
\pgfpathrectangle{\pgfqpoint{0.693757in}{0.613486in}}{\pgfqpoint{5.541243in}{3.963477in}}%
\pgfusepath{clip}%
\pgfsetbuttcap%
\pgfsetmiterjoin%
\definecolor{currentfill}{rgb}{0.000000,0.000000,1.000000}%
\pgfsetfillcolor{currentfill}%
\pgfsetlinewidth{0.000000pt}%
\definecolor{currentstroke}{rgb}{0.000000,0.000000,0.000000}%
\pgfsetstrokecolor{currentstroke}%
\pgfsetstrokeopacity{0.000000}%
\pgfsetdash{}{0pt}%
\pgfpathmoveto{\pgfqpoint{4.956650in}{0.613486in}}%
\pgfpathlineto{\pgfqpoint{4.991153in}{0.613486in}}%
\pgfpathlineto{\pgfqpoint{4.991153in}{4.576962in}}%
\pgfpathlineto{\pgfqpoint{4.956650in}{4.576962in}}%
\pgfpathlineto{\pgfqpoint{4.956650in}{0.613486in}}%
\pgfpathclose%
\pgfusepath{fill}%
\end{pgfscope}%
\begin{pgfscope}%
\pgfpathrectangle{\pgfqpoint{0.693757in}{0.613486in}}{\pgfqpoint{5.541243in}{3.963477in}}%
\pgfusepath{clip}%
\pgfsetbuttcap%
\pgfsetmiterjoin%
\definecolor{currentfill}{rgb}{0.000000,0.000000,1.000000}%
\pgfsetfillcolor{currentfill}%
\pgfsetlinewidth{0.000000pt}%
\definecolor{currentstroke}{rgb}{0.000000,0.000000,0.000000}%
\pgfsetstrokecolor{currentstroke}%
\pgfsetstrokeopacity{0.000000}%
\pgfsetdash{}{0pt}%
\pgfpathmoveto{\pgfqpoint{4.999779in}{0.613486in}}%
\pgfpathlineto{\pgfqpoint{5.034282in}{0.613486in}}%
\pgfpathlineto{\pgfqpoint{5.034282in}{4.576962in}}%
\pgfpathlineto{\pgfqpoint{4.999779in}{4.576962in}}%
\pgfpathlineto{\pgfqpoint{4.999779in}{0.613486in}}%
\pgfpathclose%
\pgfusepath{fill}%
\end{pgfscope}%
\begin{pgfscope}%
\pgfpathrectangle{\pgfqpoint{0.693757in}{0.613486in}}{\pgfqpoint{5.541243in}{3.963477in}}%
\pgfusepath{clip}%
\pgfsetbuttcap%
\pgfsetmiterjoin%
\definecolor{currentfill}{rgb}{0.000000,0.000000,1.000000}%
\pgfsetfillcolor{currentfill}%
\pgfsetlinewidth{0.000000pt}%
\definecolor{currentstroke}{rgb}{0.000000,0.000000,0.000000}%
\pgfsetstrokecolor{currentstroke}%
\pgfsetstrokeopacity{0.000000}%
\pgfsetdash{}{0pt}%
\pgfpathmoveto{\pgfqpoint{5.042908in}{0.613486in}}%
\pgfpathlineto{\pgfqpoint{5.077412in}{0.613486in}}%
\pgfpathlineto{\pgfqpoint{5.077412in}{4.576962in}}%
\pgfpathlineto{\pgfqpoint{5.042908in}{4.576962in}}%
\pgfpathlineto{\pgfqpoint{5.042908in}{0.613486in}}%
\pgfpathclose%
\pgfusepath{fill}%
\end{pgfscope}%
\begin{pgfscope}%
\pgfpathrectangle{\pgfqpoint{0.693757in}{0.613486in}}{\pgfqpoint{5.541243in}{3.963477in}}%
\pgfusepath{clip}%
\pgfsetbuttcap%
\pgfsetmiterjoin%
\definecolor{currentfill}{rgb}{0.000000,0.000000,1.000000}%
\pgfsetfillcolor{currentfill}%
\pgfsetlinewidth{0.000000pt}%
\definecolor{currentstroke}{rgb}{0.000000,0.000000,0.000000}%
\pgfsetstrokecolor{currentstroke}%
\pgfsetstrokeopacity{0.000000}%
\pgfsetdash{}{0pt}%
\pgfpathmoveto{\pgfqpoint{5.086037in}{0.613486in}}%
\pgfpathlineto{\pgfqpoint{5.120541in}{0.613486in}}%
\pgfpathlineto{\pgfqpoint{5.120541in}{4.576962in}}%
\pgfpathlineto{\pgfqpoint{5.086037in}{4.576962in}}%
\pgfpathlineto{\pgfqpoint{5.086037in}{0.613486in}}%
\pgfpathclose%
\pgfusepath{fill}%
\end{pgfscope}%
\begin{pgfscope}%
\pgfpathrectangle{\pgfqpoint{0.693757in}{0.613486in}}{\pgfqpoint{5.541243in}{3.963477in}}%
\pgfusepath{clip}%
\pgfsetbuttcap%
\pgfsetmiterjoin%
\definecolor{currentfill}{rgb}{0.000000,0.000000,1.000000}%
\pgfsetfillcolor{currentfill}%
\pgfsetlinewidth{0.000000pt}%
\definecolor{currentstroke}{rgb}{0.000000,0.000000,0.000000}%
\pgfsetstrokecolor{currentstroke}%
\pgfsetstrokeopacity{0.000000}%
\pgfsetdash{}{0pt}%
\pgfpathmoveto{\pgfqpoint{5.129167in}{0.613486in}}%
\pgfpathlineto{\pgfqpoint{5.163670in}{0.613486in}}%
\pgfpathlineto{\pgfqpoint{5.163670in}{4.576962in}}%
\pgfpathlineto{\pgfqpoint{5.129167in}{4.576962in}}%
\pgfpathlineto{\pgfqpoint{5.129167in}{0.613486in}}%
\pgfpathclose%
\pgfusepath{fill}%
\end{pgfscope}%
\begin{pgfscope}%
\pgfpathrectangle{\pgfqpoint{0.693757in}{0.613486in}}{\pgfqpoint{5.541243in}{3.963477in}}%
\pgfusepath{clip}%
\pgfsetbuttcap%
\pgfsetmiterjoin%
\definecolor{currentfill}{rgb}{0.000000,0.000000,1.000000}%
\pgfsetfillcolor{currentfill}%
\pgfsetlinewidth{0.000000pt}%
\definecolor{currentstroke}{rgb}{0.000000,0.000000,0.000000}%
\pgfsetstrokecolor{currentstroke}%
\pgfsetstrokeopacity{0.000000}%
\pgfsetdash{}{0pt}%
\pgfpathmoveto{\pgfqpoint{5.172296in}{0.613486in}}%
\pgfpathlineto{\pgfqpoint{5.206799in}{0.613486in}}%
\pgfpathlineto{\pgfqpoint{5.206799in}{4.576962in}}%
\pgfpathlineto{\pgfqpoint{5.172296in}{4.576962in}}%
\pgfpathlineto{\pgfqpoint{5.172296in}{0.613486in}}%
\pgfpathclose%
\pgfusepath{fill}%
\end{pgfscope}%
\begin{pgfscope}%
\pgfpathrectangle{\pgfqpoint{0.693757in}{0.613486in}}{\pgfqpoint{5.541243in}{3.963477in}}%
\pgfusepath{clip}%
\pgfsetbuttcap%
\pgfsetmiterjoin%
\definecolor{currentfill}{rgb}{0.000000,0.000000,1.000000}%
\pgfsetfillcolor{currentfill}%
\pgfsetlinewidth{0.000000pt}%
\definecolor{currentstroke}{rgb}{0.000000,0.000000,0.000000}%
\pgfsetstrokecolor{currentstroke}%
\pgfsetstrokeopacity{0.000000}%
\pgfsetdash{}{0pt}%
\pgfpathmoveto{\pgfqpoint{5.215425in}{0.613486in}}%
\pgfpathlineto{\pgfqpoint{5.249928in}{0.613486in}}%
\pgfpathlineto{\pgfqpoint{5.249928in}{4.576962in}}%
\pgfpathlineto{\pgfqpoint{5.215425in}{4.576962in}}%
\pgfpathlineto{\pgfqpoint{5.215425in}{0.613486in}}%
\pgfpathclose%
\pgfusepath{fill}%
\end{pgfscope}%
\begin{pgfscope}%
\pgfpathrectangle{\pgfqpoint{0.693757in}{0.613486in}}{\pgfqpoint{5.541243in}{3.963477in}}%
\pgfusepath{clip}%
\pgfsetbuttcap%
\pgfsetmiterjoin%
\definecolor{currentfill}{rgb}{0.000000,0.000000,1.000000}%
\pgfsetfillcolor{currentfill}%
\pgfsetlinewidth{0.000000pt}%
\definecolor{currentstroke}{rgb}{0.000000,0.000000,0.000000}%
\pgfsetstrokecolor{currentstroke}%
\pgfsetstrokeopacity{0.000000}%
\pgfsetdash{}{0pt}%
\pgfpathmoveto{\pgfqpoint{5.258554in}{0.613486in}}%
\pgfpathlineto{\pgfqpoint{5.293058in}{0.613486in}}%
\pgfpathlineto{\pgfqpoint{5.293058in}{4.576962in}}%
\pgfpathlineto{\pgfqpoint{5.258554in}{4.576962in}}%
\pgfpathlineto{\pgfqpoint{5.258554in}{0.613486in}}%
\pgfpathclose%
\pgfusepath{fill}%
\end{pgfscope}%
\begin{pgfscope}%
\pgfpathrectangle{\pgfqpoint{0.693757in}{0.613486in}}{\pgfqpoint{5.541243in}{3.963477in}}%
\pgfusepath{clip}%
\pgfsetbuttcap%
\pgfsetmiterjoin%
\definecolor{currentfill}{rgb}{0.000000,0.000000,1.000000}%
\pgfsetfillcolor{currentfill}%
\pgfsetlinewidth{0.000000pt}%
\definecolor{currentstroke}{rgb}{0.000000,0.000000,0.000000}%
\pgfsetstrokecolor{currentstroke}%
\pgfsetstrokeopacity{0.000000}%
\pgfsetdash{}{0pt}%
\pgfpathmoveto{\pgfqpoint{5.301684in}{0.613486in}}%
\pgfpathlineto{\pgfqpoint{5.336187in}{0.613486in}}%
\pgfpathlineto{\pgfqpoint{5.336187in}{4.576962in}}%
\pgfpathlineto{\pgfqpoint{5.301684in}{4.576962in}}%
\pgfpathlineto{\pgfqpoint{5.301684in}{0.613486in}}%
\pgfpathclose%
\pgfusepath{fill}%
\end{pgfscope}%
\begin{pgfscope}%
\pgfpathrectangle{\pgfqpoint{0.693757in}{0.613486in}}{\pgfqpoint{5.541243in}{3.963477in}}%
\pgfusepath{clip}%
\pgfsetbuttcap%
\pgfsetmiterjoin%
\definecolor{currentfill}{rgb}{0.000000,0.000000,1.000000}%
\pgfsetfillcolor{currentfill}%
\pgfsetlinewidth{0.000000pt}%
\definecolor{currentstroke}{rgb}{0.000000,0.000000,0.000000}%
\pgfsetstrokecolor{currentstroke}%
\pgfsetstrokeopacity{0.000000}%
\pgfsetdash{}{0pt}%
\pgfpathmoveto{\pgfqpoint{5.344813in}{0.613486in}}%
\pgfpathlineto{\pgfqpoint{5.379316in}{0.613486in}}%
\pgfpathlineto{\pgfqpoint{5.379316in}{4.576962in}}%
\pgfpathlineto{\pgfqpoint{5.344813in}{4.576962in}}%
\pgfpathlineto{\pgfqpoint{5.344813in}{0.613486in}}%
\pgfpathclose%
\pgfusepath{fill}%
\end{pgfscope}%
\begin{pgfscope}%
\pgfpathrectangle{\pgfqpoint{0.693757in}{0.613486in}}{\pgfqpoint{5.541243in}{3.963477in}}%
\pgfusepath{clip}%
\pgfsetbuttcap%
\pgfsetmiterjoin%
\definecolor{currentfill}{rgb}{0.000000,0.000000,1.000000}%
\pgfsetfillcolor{currentfill}%
\pgfsetlinewidth{0.000000pt}%
\definecolor{currentstroke}{rgb}{0.000000,0.000000,0.000000}%
\pgfsetstrokecolor{currentstroke}%
\pgfsetstrokeopacity{0.000000}%
\pgfsetdash{}{0pt}%
\pgfpathmoveto{\pgfqpoint{5.387942in}{0.613486in}}%
\pgfpathlineto{\pgfqpoint{5.422445in}{0.613486in}}%
\pgfpathlineto{\pgfqpoint{5.422445in}{4.576962in}}%
\pgfpathlineto{\pgfqpoint{5.387942in}{4.576962in}}%
\pgfpathlineto{\pgfqpoint{5.387942in}{0.613486in}}%
\pgfpathclose%
\pgfusepath{fill}%
\end{pgfscope}%
\begin{pgfscope}%
\pgfpathrectangle{\pgfqpoint{0.693757in}{0.613486in}}{\pgfqpoint{5.541243in}{3.963477in}}%
\pgfusepath{clip}%
\pgfsetbuttcap%
\pgfsetmiterjoin%
\definecolor{currentfill}{rgb}{0.000000,0.000000,1.000000}%
\pgfsetfillcolor{currentfill}%
\pgfsetlinewidth{0.000000pt}%
\definecolor{currentstroke}{rgb}{0.000000,0.000000,0.000000}%
\pgfsetstrokecolor{currentstroke}%
\pgfsetstrokeopacity{0.000000}%
\pgfsetdash{}{0pt}%
\pgfpathmoveto{\pgfqpoint{5.431071in}{0.613486in}}%
\pgfpathlineto{\pgfqpoint{5.465575in}{0.613486in}}%
\pgfpathlineto{\pgfqpoint{5.465575in}{4.576962in}}%
\pgfpathlineto{\pgfqpoint{5.431071in}{4.576962in}}%
\pgfpathlineto{\pgfqpoint{5.431071in}{0.613486in}}%
\pgfpathclose%
\pgfusepath{fill}%
\end{pgfscope}%
\begin{pgfscope}%
\pgfpathrectangle{\pgfqpoint{0.693757in}{0.613486in}}{\pgfqpoint{5.541243in}{3.963477in}}%
\pgfusepath{clip}%
\pgfsetbuttcap%
\pgfsetmiterjoin%
\definecolor{currentfill}{rgb}{0.000000,0.000000,1.000000}%
\pgfsetfillcolor{currentfill}%
\pgfsetlinewidth{0.000000pt}%
\definecolor{currentstroke}{rgb}{0.000000,0.000000,0.000000}%
\pgfsetstrokecolor{currentstroke}%
\pgfsetstrokeopacity{0.000000}%
\pgfsetdash{}{0pt}%
\pgfpathmoveto{\pgfqpoint{5.474200in}{0.613486in}}%
\pgfpathlineto{\pgfqpoint{5.508704in}{0.613486in}}%
\pgfpathlineto{\pgfqpoint{5.508704in}{4.576962in}}%
\pgfpathlineto{\pgfqpoint{5.474200in}{4.576962in}}%
\pgfpathlineto{\pgfqpoint{5.474200in}{0.613486in}}%
\pgfpathclose%
\pgfusepath{fill}%
\end{pgfscope}%
\begin{pgfscope}%
\pgfpathrectangle{\pgfqpoint{0.693757in}{0.613486in}}{\pgfqpoint{5.541243in}{3.963477in}}%
\pgfusepath{clip}%
\pgfsetbuttcap%
\pgfsetmiterjoin%
\definecolor{currentfill}{rgb}{0.000000,0.000000,1.000000}%
\pgfsetfillcolor{currentfill}%
\pgfsetlinewidth{0.000000pt}%
\definecolor{currentstroke}{rgb}{0.000000,0.000000,0.000000}%
\pgfsetstrokecolor{currentstroke}%
\pgfsetstrokeopacity{0.000000}%
\pgfsetdash{}{0pt}%
\pgfpathmoveto{\pgfqpoint{5.517330in}{0.613486in}}%
\pgfpathlineto{\pgfqpoint{5.551833in}{0.613486in}}%
\pgfpathlineto{\pgfqpoint{5.551833in}{4.576962in}}%
\pgfpathlineto{\pgfqpoint{5.517330in}{4.576962in}}%
\pgfpathlineto{\pgfqpoint{5.517330in}{0.613486in}}%
\pgfpathclose%
\pgfusepath{fill}%
\end{pgfscope}%
\begin{pgfscope}%
\pgfpathrectangle{\pgfqpoint{0.693757in}{0.613486in}}{\pgfqpoint{5.541243in}{3.963477in}}%
\pgfusepath{clip}%
\pgfsetbuttcap%
\pgfsetmiterjoin%
\definecolor{currentfill}{rgb}{0.000000,0.000000,1.000000}%
\pgfsetfillcolor{currentfill}%
\pgfsetlinewidth{0.000000pt}%
\definecolor{currentstroke}{rgb}{0.000000,0.000000,0.000000}%
\pgfsetstrokecolor{currentstroke}%
\pgfsetstrokeopacity{0.000000}%
\pgfsetdash{}{0pt}%
\pgfpathmoveto{\pgfqpoint{5.560459in}{0.613486in}}%
\pgfpathlineto{\pgfqpoint{5.594962in}{0.613486in}}%
\pgfpathlineto{\pgfqpoint{5.594962in}{4.576962in}}%
\pgfpathlineto{\pgfqpoint{5.560459in}{4.576962in}}%
\pgfpathlineto{\pgfqpoint{5.560459in}{0.613486in}}%
\pgfpathclose%
\pgfusepath{fill}%
\end{pgfscope}%
\begin{pgfscope}%
\pgfpathrectangle{\pgfqpoint{0.693757in}{0.613486in}}{\pgfqpoint{5.541243in}{3.963477in}}%
\pgfusepath{clip}%
\pgfsetbuttcap%
\pgfsetmiterjoin%
\definecolor{currentfill}{rgb}{0.000000,0.000000,1.000000}%
\pgfsetfillcolor{currentfill}%
\pgfsetlinewidth{0.000000pt}%
\definecolor{currentstroke}{rgb}{0.000000,0.000000,0.000000}%
\pgfsetstrokecolor{currentstroke}%
\pgfsetstrokeopacity{0.000000}%
\pgfsetdash{}{0pt}%
\pgfpathmoveto{\pgfqpoint{5.603588in}{0.613486in}}%
\pgfpathlineto{\pgfqpoint{5.638092in}{0.613486in}}%
\pgfpathlineto{\pgfqpoint{5.638092in}{4.576962in}}%
\pgfpathlineto{\pgfqpoint{5.603588in}{4.576962in}}%
\pgfpathlineto{\pgfqpoint{5.603588in}{0.613486in}}%
\pgfpathclose%
\pgfusepath{fill}%
\end{pgfscope}%
\begin{pgfscope}%
\pgfpathrectangle{\pgfqpoint{0.693757in}{0.613486in}}{\pgfqpoint{5.541243in}{3.963477in}}%
\pgfusepath{clip}%
\pgfsetbuttcap%
\pgfsetmiterjoin%
\definecolor{currentfill}{rgb}{0.000000,0.000000,1.000000}%
\pgfsetfillcolor{currentfill}%
\pgfsetlinewidth{0.000000pt}%
\definecolor{currentstroke}{rgb}{0.000000,0.000000,0.000000}%
\pgfsetstrokecolor{currentstroke}%
\pgfsetstrokeopacity{0.000000}%
\pgfsetdash{}{0pt}%
\pgfpathmoveto{\pgfqpoint{5.646717in}{0.613486in}}%
\pgfpathlineto{\pgfqpoint{5.681221in}{0.613486in}}%
\pgfpathlineto{\pgfqpoint{5.681221in}{4.576962in}}%
\pgfpathlineto{\pgfqpoint{5.646717in}{4.576962in}}%
\pgfpathlineto{\pgfqpoint{5.646717in}{0.613486in}}%
\pgfpathclose%
\pgfusepath{fill}%
\end{pgfscope}%
\begin{pgfscope}%
\pgfpathrectangle{\pgfqpoint{0.693757in}{0.613486in}}{\pgfqpoint{5.541243in}{3.963477in}}%
\pgfusepath{clip}%
\pgfsetbuttcap%
\pgfsetmiterjoin%
\definecolor{currentfill}{rgb}{0.000000,0.000000,1.000000}%
\pgfsetfillcolor{currentfill}%
\pgfsetlinewidth{0.000000pt}%
\definecolor{currentstroke}{rgb}{0.000000,0.000000,0.000000}%
\pgfsetstrokecolor{currentstroke}%
\pgfsetstrokeopacity{0.000000}%
\pgfsetdash{}{0pt}%
\pgfpathmoveto{\pgfqpoint{5.689847in}{0.613486in}}%
\pgfpathlineto{\pgfqpoint{5.724350in}{0.613486in}}%
\pgfpathlineto{\pgfqpoint{5.724350in}{4.576962in}}%
\pgfpathlineto{\pgfqpoint{5.689847in}{4.576962in}}%
\pgfpathlineto{\pgfqpoint{5.689847in}{0.613486in}}%
\pgfpathclose%
\pgfusepath{fill}%
\end{pgfscope}%
\begin{pgfscope}%
\pgfpathrectangle{\pgfqpoint{0.693757in}{0.613486in}}{\pgfqpoint{5.541243in}{3.963477in}}%
\pgfusepath{clip}%
\pgfsetbuttcap%
\pgfsetmiterjoin%
\definecolor{currentfill}{rgb}{0.000000,0.000000,1.000000}%
\pgfsetfillcolor{currentfill}%
\pgfsetlinewidth{0.000000pt}%
\definecolor{currentstroke}{rgb}{0.000000,0.000000,0.000000}%
\pgfsetstrokecolor{currentstroke}%
\pgfsetstrokeopacity{0.000000}%
\pgfsetdash{}{0pt}%
\pgfpathmoveto{\pgfqpoint{5.732976in}{0.613486in}}%
\pgfpathlineto{\pgfqpoint{5.767479in}{0.613486in}}%
\pgfpathlineto{\pgfqpoint{5.767479in}{4.576962in}}%
\pgfpathlineto{\pgfqpoint{5.732976in}{4.576962in}}%
\pgfpathlineto{\pgfqpoint{5.732976in}{0.613486in}}%
\pgfpathclose%
\pgfusepath{fill}%
\end{pgfscope}%
\begin{pgfscope}%
\pgfpathrectangle{\pgfqpoint{0.693757in}{0.613486in}}{\pgfqpoint{5.541243in}{3.963477in}}%
\pgfusepath{clip}%
\pgfsetbuttcap%
\pgfsetmiterjoin%
\definecolor{currentfill}{rgb}{0.000000,0.000000,1.000000}%
\pgfsetfillcolor{currentfill}%
\pgfsetlinewidth{0.000000pt}%
\definecolor{currentstroke}{rgb}{0.000000,0.000000,0.000000}%
\pgfsetstrokecolor{currentstroke}%
\pgfsetstrokeopacity{0.000000}%
\pgfsetdash{}{0pt}%
\pgfpathmoveto{\pgfqpoint{5.776105in}{0.613486in}}%
\pgfpathlineto{\pgfqpoint{5.810608in}{0.613486in}}%
\pgfpathlineto{\pgfqpoint{5.810608in}{1.604355in}}%
\pgfpathlineto{\pgfqpoint{5.776105in}{1.604355in}}%
\pgfpathlineto{\pgfqpoint{5.776105in}{0.613486in}}%
\pgfpathclose%
\pgfusepath{fill}%
\end{pgfscope}%
\begin{pgfscope}%
\pgfpathrectangle{\pgfqpoint{0.693757in}{0.613486in}}{\pgfqpoint{5.541243in}{3.963477in}}%
\pgfusepath{clip}%
\pgfsetbuttcap%
\pgfsetmiterjoin%
\definecolor{currentfill}{rgb}{0.000000,0.000000,1.000000}%
\pgfsetfillcolor{currentfill}%
\pgfsetlinewidth{0.000000pt}%
\definecolor{currentstroke}{rgb}{0.000000,0.000000,0.000000}%
\pgfsetstrokecolor{currentstroke}%
\pgfsetstrokeopacity{0.000000}%
\pgfsetdash{}{0pt}%
\pgfpathmoveto{\pgfqpoint{5.819234in}{0.613486in}}%
\pgfpathlineto{\pgfqpoint{5.853738in}{0.613486in}}%
\pgfpathlineto{\pgfqpoint{5.853738in}{1.108912in}}%
\pgfpathlineto{\pgfqpoint{5.819234in}{1.108912in}}%
\pgfpathlineto{\pgfqpoint{5.819234in}{0.613486in}}%
\pgfpathclose%
\pgfusepath{fill}%
\end{pgfscope}%
\begin{pgfscope}%
\pgfpathrectangle{\pgfqpoint{0.693757in}{0.613486in}}{\pgfqpoint{5.541243in}{3.963477in}}%
\pgfusepath{clip}%
\pgfsetbuttcap%
\pgfsetmiterjoin%
\definecolor{currentfill}{rgb}{0.000000,0.000000,1.000000}%
\pgfsetfillcolor{currentfill}%
\pgfsetlinewidth{0.000000pt}%
\definecolor{currentstroke}{rgb}{0.000000,0.000000,0.000000}%
\pgfsetstrokecolor{currentstroke}%
\pgfsetstrokeopacity{0.000000}%
\pgfsetdash{}{0pt}%
\pgfpathmoveto{\pgfqpoint{5.862363in}{0.613486in}}%
\pgfpathlineto{\pgfqpoint{5.896867in}{0.613486in}}%
\pgfpathlineto{\pgfqpoint{5.896867in}{2.595284in}}%
\pgfpathlineto{\pgfqpoint{5.862363in}{2.595284in}}%
\pgfpathlineto{\pgfqpoint{5.862363in}{0.613486in}}%
\pgfpathclose%
\pgfusepath{fill}%
\end{pgfscope}%
\begin{pgfscope}%
\pgfpathrectangle{\pgfqpoint{0.693757in}{0.613486in}}{\pgfqpoint{5.541243in}{3.963477in}}%
\pgfusepath{clip}%
\pgfsetbuttcap%
\pgfsetmiterjoin%
\definecolor{currentfill}{rgb}{0.000000,0.000000,1.000000}%
\pgfsetfillcolor{currentfill}%
\pgfsetlinewidth{0.000000pt}%
\definecolor{currentstroke}{rgb}{0.000000,0.000000,0.000000}%
\pgfsetstrokecolor{currentstroke}%
\pgfsetstrokeopacity{0.000000}%
\pgfsetdash{}{0pt}%
\pgfpathmoveto{\pgfqpoint{5.905493in}{0.613486in}}%
\pgfpathlineto{\pgfqpoint{5.939996in}{0.613486in}}%
\pgfpathlineto{\pgfqpoint{5.939996in}{1.108920in}}%
\pgfpathlineto{\pgfqpoint{5.905493in}{1.108920in}}%
\pgfpathlineto{\pgfqpoint{5.905493in}{0.613486in}}%
\pgfpathclose%
\pgfusepath{fill}%
\end{pgfscope}%
\begin{pgfscope}%
\pgfpathrectangle{\pgfqpoint{0.693757in}{0.613486in}}{\pgfqpoint{5.541243in}{3.963477in}}%
\pgfusepath{clip}%
\pgfsetbuttcap%
\pgfsetmiterjoin%
\definecolor{currentfill}{rgb}{0.000000,0.000000,1.000000}%
\pgfsetfillcolor{currentfill}%
\pgfsetlinewidth{0.000000pt}%
\definecolor{currentstroke}{rgb}{0.000000,0.000000,0.000000}%
\pgfsetstrokecolor{currentstroke}%
\pgfsetstrokeopacity{0.000000}%
\pgfsetdash{}{0pt}%
\pgfpathmoveto{\pgfqpoint{5.948622in}{0.613486in}}%
\pgfpathlineto{\pgfqpoint{5.983125in}{0.613486in}}%
\pgfpathlineto{\pgfqpoint{5.983125in}{0.861196in}}%
\pgfpathlineto{\pgfqpoint{5.948622in}{0.861196in}}%
\pgfpathlineto{\pgfqpoint{5.948622in}{0.613486in}}%
\pgfpathclose%
\pgfusepath{fill}%
\end{pgfscope}%
\begin{pgfscope}%
\pgfsetbuttcap%
\pgfsetroundjoin%
\definecolor{currentfill}{rgb}{0.000000,0.000000,0.000000}%
\pgfsetfillcolor{currentfill}%
\pgfsetlinewidth{0.803000pt}%
\definecolor{currentstroke}{rgb}{0.000000,0.000000,0.000000}%
\pgfsetstrokecolor{currentstroke}%
\pgfsetdash{}{0pt}%
\pgfsys@defobject{currentmarker}{\pgfqpoint{0.000000in}{-0.048611in}}{\pgfqpoint{0.000000in}{0.000000in}}{%
\pgfpathmoveto{\pgfqpoint{0.000000in}{0.000000in}}%
\pgfpathlineto{\pgfqpoint{0.000000in}{-0.048611in}}%
\pgfusepath{stroke,fill}%
}%
\begin{pgfscope}%
\pgfsys@transformshift{0.876625in}{0.613486in}%
\pgfsys@useobject{currentmarker}{}%
\end{pgfscope}%
\end{pgfscope}%
\begin{pgfscope}%
\definecolor{textcolor}{rgb}{0.000000,0.000000,0.000000}%
\pgfsetstrokecolor{textcolor}%
\pgfsetfillcolor{textcolor}%
\pgftext[x=0.876625in,y=0.516264in,,top]{\color{textcolor}{\sffamily\fontsize{11.000000}{13.200000}\selectfont\catcode`\^=\active\def^{\ifmmode\sp\else\^{}\fi}\catcode`\%=\active\def%{\%}$\mathdefault{0}$}}%
\end{pgfscope}%
\begin{pgfscope}%
\pgfsetbuttcap%
\pgfsetroundjoin%
\definecolor{currentfill}{rgb}{0.000000,0.000000,0.000000}%
\pgfsetfillcolor{currentfill}%
\pgfsetlinewidth{0.803000pt}%
\definecolor{currentstroke}{rgb}{0.000000,0.000000,0.000000}%
\pgfsetstrokecolor{currentstroke}%
\pgfsetdash{}{0pt}%
\pgfsys@defobject{currentmarker}{\pgfqpoint{0.000000in}{-0.048611in}}{\pgfqpoint{0.000000in}{0.000000in}}{%
\pgfpathmoveto{\pgfqpoint{0.000000in}{0.000000in}}%
\pgfpathlineto{\pgfqpoint{0.000000in}{-0.048611in}}%
\pgfusepath{stroke,fill}%
}%
\begin{pgfscope}%
\pgfsys@transformshift{1.739210in}{0.613486in}%
\pgfsys@useobject{currentmarker}{}%
\end{pgfscope}%
\end{pgfscope}%
\begin{pgfscope}%
\definecolor{textcolor}{rgb}{0.000000,0.000000,0.000000}%
\pgfsetstrokecolor{textcolor}%
\pgfsetfillcolor{textcolor}%
\pgftext[x=1.739210in,y=0.516264in,,top]{\color{textcolor}{\sffamily\fontsize{11.000000}{13.200000}\selectfont\catcode`\^=\active\def^{\ifmmode\sp\else\^{}\fi}\catcode`\%=\active\def%{\%}$\mathdefault{20}$}}%
\end{pgfscope}%
\begin{pgfscope}%
\pgfsetbuttcap%
\pgfsetroundjoin%
\definecolor{currentfill}{rgb}{0.000000,0.000000,0.000000}%
\pgfsetfillcolor{currentfill}%
\pgfsetlinewidth{0.803000pt}%
\definecolor{currentstroke}{rgb}{0.000000,0.000000,0.000000}%
\pgfsetstrokecolor{currentstroke}%
\pgfsetdash{}{0pt}%
\pgfsys@defobject{currentmarker}{\pgfqpoint{0.000000in}{-0.048611in}}{\pgfqpoint{0.000000in}{0.000000in}}{%
\pgfpathmoveto{\pgfqpoint{0.000000in}{0.000000in}}%
\pgfpathlineto{\pgfqpoint{0.000000in}{-0.048611in}}%
\pgfusepath{stroke,fill}%
}%
\begin{pgfscope}%
\pgfsys@transformshift{2.601794in}{0.613486in}%
\pgfsys@useobject{currentmarker}{}%
\end{pgfscope}%
\end{pgfscope}%
\begin{pgfscope}%
\definecolor{textcolor}{rgb}{0.000000,0.000000,0.000000}%
\pgfsetstrokecolor{textcolor}%
\pgfsetfillcolor{textcolor}%
\pgftext[x=2.601794in,y=0.516264in,,top]{\color{textcolor}{\sffamily\fontsize{11.000000}{13.200000}\selectfont\catcode`\^=\active\def^{\ifmmode\sp\else\^{}\fi}\catcode`\%=\active\def%{\%}$\mathdefault{40}$}}%
\end{pgfscope}%
\begin{pgfscope}%
\pgfsetbuttcap%
\pgfsetroundjoin%
\definecolor{currentfill}{rgb}{0.000000,0.000000,0.000000}%
\pgfsetfillcolor{currentfill}%
\pgfsetlinewidth{0.803000pt}%
\definecolor{currentstroke}{rgb}{0.000000,0.000000,0.000000}%
\pgfsetstrokecolor{currentstroke}%
\pgfsetdash{}{0pt}%
\pgfsys@defobject{currentmarker}{\pgfqpoint{0.000000in}{-0.048611in}}{\pgfqpoint{0.000000in}{0.000000in}}{%
\pgfpathmoveto{\pgfqpoint{0.000000in}{0.000000in}}%
\pgfpathlineto{\pgfqpoint{0.000000in}{-0.048611in}}%
\pgfusepath{stroke,fill}%
}%
\begin{pgfscope}%
\pgfsys@transformshift{3.464379in}{0.613486in}%
\pgfsys@useobject{currentmarker}{}%
\end{pgfscope}%
\end{pgfscope}%
\begin{pgfscope}%
\definecolor{textcolor}{rgb}{0.000000,0.000000,0.000000}%
\pgfsetstrokecolor{textcolor}%
\pgfsetfillcolor{textcolor}%
\pgftext[x=3.464379in,y=0.516264in,,top]{\color{textcolor}{\sffamily\fontsize{11.000000}{13.200000}\selectfont\catcode`\^=\active\def^{\ifmmode\sp\else\^{}\fi}\catcode`\%=\active\def%{\%}$\mathdefault{60}$}}%
\end{pgfscope}%
\begin{pgfscope}%
\pgfsetbuttcap%
\pgfsetroundjoin%
\definecolor{currentfill}{rgb}{0.000000,0.000000,0.000000}%
\pgfsetfillcolor{currentfill}%
\pgfsetlinewidth{0.803000pt}%
\definecolor{currentstroke}{rgb}{0.000000,0.000000,0.000000}%
\pgfsetstrokecolor{currentstroke}%
\pgfsetdash{}{0pt}%
\pgfsys@defobject{currentmarker}{\pgfqpoint{0.000000in}{-0.048611in}}{\pgfqpoint{0.000000in}{0.000000in}}{%
\pgfpathmoveto{\pgfqpoint{0.000000in}{0.000000in}}%
\pgfpathlineto{\pgfqpoint{0.000000in}{-0.048611in}}%
\pgfusepath{stroke,fill}%
}%
\begin{pgfscope}%
\pgfsys@transformshift{4.326963in}{0.613486in}%
\pgfsys@useobject{currentmarker}{}%
\end{pgfscope}%
\end{pgfscope}%
\begin{pgfscope}%
\definecolor{textcolor}{rgb}{0.000000,0.000000,0.000000}%
\pgfsetstrokecolor{textcolor}%
\pgfsetfillcolor{textcolor}%
\pgftext[x=4.326963in,y=0.516264in,,top]{\color{textcolor}{\sffamily\fontsize{11.000000}{13.200000}\selectfont\catcode`\^=\active\def^{\ifmmode\sp\else\^{}\fi}\catcode`\%=\active\def%{\%}$\mathdefault{80}$}}%
\end{pgfscope}%
\begin{pgfscope}%
\pgfsetbuttcap%
\pgfsetroundjoin%
\definecolor{currentfill}{rgb}{0.000000,0.000000,0.000000}%
\pgfsetfillcolor{currentfill}%
\pgfsetlinewidth{0.803000pt}%
\definecolor{currentstroke}{rgb}{0.000000,0.000000,0.000000}%
\pgfsetstrokecolor{currentstroke}%
\pgfsetdash{}{0pt}%
\pgfsys@defobject{currentmarker}{\pgfqpoint{0.000000in}{-0.048611in}}{\pgfqpoint{0.000000in}{0.000000in}}{%
\pgfpathmoveto{\pgfqpoint{0.000000in}{0.000000in}}%
\pgfpathlineto{\pgfqpoint{0.000000in}{-0.048611in}}%
\pgfusepath{stroke,fill}%
}%
\begin{pgfscope}%
\pgfsys@transformshift{5.189548in}{0.613486in}%
\pgfsys@useobject{currentmarker}{}%
\end{pgfscope}%
\end{pgfscope}%
\begin{pgfscope}%
\definecolor{textcolor}{rgb}{0.000000,0.000000,0.000000}%
\pgfsetstrokecolor{textcolor}%
\pgfsetfillcolor{textcolor}%
\pgftext[x=5.189548in,y=0.516264in,,top]{\color{textcolor}{\sffamily\fontsize{11.000000}{13.200000}\selectfont\catcode`\^=\active\def^{\ifmmode\sp\else\^{}\fi}\catcode`\%=\active\def%{\%}$\mathdefault{100}$}}%
\end{pgfscope}%
\begin{pgfscope}%
\pgfsetbuttcap%
\pgfsetroundjoin%
\definecolor{currentfill}{rgb}{0.000000,0.000000,0.000000}%
\pgfsetfillcolor{currentfill}%
\pgfsetlinewidth{0.803000pt}%
\definecolor{currentstroke}{rgb}{0.000000,0.000000,0.000000}%
\pgfsetstrokecolor{currentstroke}%
\pgfsetdash{}{0pt}%
\pgfsys@defobject{currentmarker}{\pgfqpoint{0.000000in}{-0.048611in}}{\pgfqpoint{0.000000in}{0.000000in}}{%
\pgfpathmoveto{\pgfqpoint{0.000000in}{0.000000in}}%
\pgfpathlineto{\pgfqpoint{0.000000in}{-0.048611in}}%
\pgfusepath{stroke,fill}%
}%
\begin{pgfscope}%
\pgfsys@transformshift{6.052132in}{0.613486in}%
\pgfsys@useobject{currentmarker}{}%
\end{pgfscope}%
\end{pgfscope}%
\begin{pgfscope}%
\definecolor{textcolor}{rgb}{0.000000,0.000000,0.000000}%
\pgfsetstrokecolor{textcolor}%
\pgfsetfillcolor{textcolor}%
\pgftext[x=6.052132in,y=0.516264in,,top]{\color{textcolor}{\sffamily\fontsize{11.000000}{13.200000}\selectfont\catcode`\^=\active\def^{\ifmmode\sp\else\^{}\fi}\catcode`\%=\active\def%{\%}$\mathdefault{120}$}}%
\end{pgfscope}%
\begin{pgfscope}%
\definecolor{textcolor}{rgb}{0.000000,0.000000,0.000000}%
\pgfsetstrokecolor{textcolor}%
\pgfsetfillcolor{textcolor}%
\pgftext[x=3.464379in,y=0.312854in,,top]{\color{textcolor}{\sffamily\fontsize{11.000000}{13.200000}\selectfont\catcode`\^=\active\def^{\ifmmode\sp\else\^{}\fi}\catcode`\%=\active\def%{\%}Kernel index}}%
\end{pgfscope}%
\begin{pgfscope}%
\pgfsetbuttcap%
\pgfsetroundjoin%
\definecolor{currentfill}{rgb}{0.000000,0.000000,0.000000}%
\pgfsetfillcolor{currentfill}%
\pgfsetlinewidth{0.803000pt}%
\definecolor{currentstroke}{rgb}{0.000000,0.000000,0.000000}%
\pgfsetstrokecolor{currentstroke}%
\pgfsetdash{}{0pt}%
\pgfsys@defobject{currentmarker}{\pgfqpoint{-0.048611in}{0.000000in}}{\pgfqpoint{-0.000000in}{0.000000in}}{%
\pgfpathmoveto{\pgfqpoint{-0.000000in}{0.000000in}}%
\pgfpathlineto{\pgfqpoint{-0.048611in}{0.000000in}}%
\pgfusepath{stroke,fill}%
}%
\begin{pgfscope}%
\pgfsys@transformshift{0.693757in}{0.613486in}%
\pgfsys@useobject{currentmarker}{}%
\end{pgfscope}%
\end{pgfscope}%
\begin{pgfscope}%
\definecolor{textcolor}{rgb}{0.000000,0.000000,0.000000}%
\pgfsetstrokecolor{textcolor}%
\pgfsetfillcolor{textcolor}%
\pgftext[x=0.520493in, y=0.555448in, left, base]{\color{textcolor}{\sffamily\fontsize{11.000000}{13.200000}\selectfont\catcode`\^=\active\def^{\ifmmode\sp\else\^{}\fi}\catcode`\%=\active\def%{\%}$\mathdefault{0}$}}%
\end{pgfscope}%
\begin{pgfscope}%
\pgfsetbuttcap%
\pgfsetroundjoin%
\definecolor{currentfill}{rgb}{0.000000,0.000000,0.000000}%
\pgfsetfillcolor{currentfill}%
\pgfsetlinewidth{0.803000pt}%
\definecolor{currentstroke}{rgb}{0.000000,0.000000,0.000000}%
\pgfsetstrokecolor{currentstroke}%
\pgfsetdash{}{0pt}%
\pgfsys@defobject{currentmarker}{\pgfqpoint{-0.048611in}{0.000000in}}{\pgfqpoint{-0.000000in}{0.000000in}}{%
\pgfpathmoveto{\pgfqpoint{-0.000000in}{0.000000in}}%
\pgfpathlineto{\pgfqpoint{-0.048611in}{0.000000in}}%
\pgfusepath{stroke,fill}%
}%
\begin{pgfscope}%
\pgfsys@transformshift{0.693757in}{1.406181in}%
\pgfsys@useobject{currentmarker}{}%
\end{pgfscope}%
\end{pgfscope}%
\begin{pgfscope}%
\definecolor{textcolor}{rgb}{0.000000,0.000000,0.000000}%
\pgfsetstrokecolor{textcolor}%
\pgfsetfillcolor{textcolor}%
\pgftext[x=0.444451in, y=1.348144in, left, base]{\color{textcolor}{\sffamily\fontsize{11.000000}{13.200000}\selectfont\catcode`\^=\active\def^{\ifmmode\sp\else\^{}\fi}\catcode`\%=\active\def%{\%}$\mathdefault{20}$}}%
\end{pgfscope}%
\begin{pgfscope}%
\pgfsetbuttcap%
\pgfsetroundjoin%
\definecolor{currentfill}{rgb}{0.000000,0.000000,0.000000}%
\pgfsetfillcolor{currentfill}%
\pgfsetlinewidth{0.803000pt}%
\definecolor{currentstroke}{rgb}{0.000000,0.000000,0.000000}%
\pgfsetstrokecolor{currentstroke}%
\pgfsetdash{}{0pt}%
\pgfsys@defobject{currentmarker}{\pgfqpoint{-0.048611in}{0.000000in}}{\pgfqpoint{-0.000000in}{0.000000in}}{%
\pgfpathmoveto{\pgfqpoint{-0.000000in}{0.000000in}}%
\pgfpathlineto{\pgfqpoint{-0.048611in}{0.000000in}}%
\pgfusepath{stroke,fill}%
}%
\begin{pgfscope}%
\pgfsys@transformshift{0.693757in}{2.198876in}%
\pgfsys@useobject{currentmarker}{}%
\end{pgfscope}%
\end{pgfscope}%
\begin{pgfscope}%
\definecolor{textcolor}{rgb}{0.000000,0.000000,0.000000}%
\pgfsetstrokecolor{textcolor}%
\pgfsetfillcolor{textcolor}%
\pgftext[x=0.444451in, y=2.140839in, left, base]{\color{textcolor}{\sffamily\fontsize{11.000000}{13.200000}\selectfont\catcode`\^=\active\def^{\ifmmode\sp\else\^{}\fi}\catcode`\%=\active\def%{\%}$\mathdefault{40}$}}%
\end{pgfscope}%
\begin{pgfscope}%
\pgfsetbuttcap%
\pgfsetroundjoin%
\definecolor{currentfill}{rgb}{0.000000,0.000000,0.000000}%
\pgfsetfillcolor{currentfill}%
\pgfsetlinewidth{0.803000pt}%
\definecolor{currentstroke}{rgb}{0.000000,0.000000,0.000000}%
\pgfsetstrokecolor{currentstroke}%
\pgfsetdash{}{0pt}%
\pgfsys@defobject{currentmarker}{\pgfqpoint{-0.048611in}{0.000000in}}{\pgfqpoint{-0.000000in}{0.000000in}}{%
\pgfpathmoveto{\pgfqpoint{-0.000000in}{0.000000in}}%
\pgfpathlineto{\pgfqpoint{-0.048611in}{0.000000in}}%
\pgfusepath{stroke,fill}%
}%
\begin{pgfscope}%
\pgfsys@transformshift{0.693757in}{2.991572in}%
\pgfsys@useobject{currentmarker}{}%
\end{pgfscope}%
\end{pgfscope}%
\begin{pgfscope}%
\definecolor{textcolor}{rgb}{0.000000,0.000000,0.000000}%
\pgfsetstrokecolor{textcolor}%
\pgfsetfillcolor{textcolor}%
\pgftext[x=0.444451in, y=2.933534in, left, base]{\color{textcolor}{\sffamily\fontsize{11.000000}{13.200000}\selectfont\catcode`\^=\active\def^{\ifmmode\sp\else\^{}\fi}\catcode`\%=\active\def%{\%}$\mathdefault{60}$}}%
\end{pgfscope}%
\begin{pgfscope}%
\pgfsetbuttcap%
\pgfsetroundjoin%
\definecolor{currentfill}{rgb}{0.000000,0.000000,0.000000}%
\pgfsetfillcolor{currentfill}%
\pgfsetlinewidth{0.803000pt}%
\definecolor{currentstroke}{rgb}{0.000000,0.000000,0.000000}%
\pgfsetstrokecolor{currentstroke}%
\pgfsetdash{}{0pt}%
\pgfsys@defobject{currentmarker}{\pgfqpoint{-0.048611in}{0.000000in}}{\pgfqpoint{-0.000000in}{0.000000in}}{%
\pgfpathmoveto{\pgfqpoint{-0.000000in}{0.000000in}}%
\pgfpathlineto{\pgfqpoint{-0.048611in}{0.000000in}}%
\pgfusepath{stroke,fill}%
}%
\begin{pgfscope}%
\pgfsys@transformshift{0.693757in}{3.784267in}%
\pgfsys@useobject{currentmarker}{}%
\end{pgfscope}%
\end{pgfscope}%
\begin{pgfscope}%
\definecolor{textcolor}{rgb}{0.000000,0.000000,0.000000}%
\pgfsetstrokecolor{textcolor}%
\pgfsetfillcolor{textcolor}%
\pgftext[x=0.444451in, y=3.726229in, left, base]{\color{textcolor}{\sffamily\fontsize{11.000000}{13.200000}\selectfont\catcode`\^=\active\def^{\ifmmode\sp\else\^{}\fi}\catcode`\%=\active\def%{\%}$\mathdefault{80}$}}%
\end{pgfscope}%
\begin{pgfscope}%
\pgfsetbuttcap%
\pgfsetroundjoin%
\definecolor{currentfill}{rgb}{0.000000,0.000000,0.000000}%
\pgfsetfillcolor{currentfill}%
\pgfsetlinewidth{0.803000pt}%
\definecolor{currentstroke}{rgb}{0.000000,0.000000,0.000000}%
\pgfsetstrokecolor{currentstroke}%
\pgfsetdash{}{0pt}%
\pgfsys@defobject{currentmarker}{\pgfqpoint{-0.048611in}{0.000000in}}{\pgfqpoint{-0.000000in}{0.000000in}}{%
\pgfpathmoveto{\pgfqpoint{-0.000000in}{0.000000in}}%
\pgfpathlineto{\pgfqpoint{-0.048611in}{0.000000in}}%
\pgfusepath{stroke,fill}%
}%
\begin{pgfscope}%
\pgfsys@transformshift{0.693757in}{4.576962in}%
\pgfsys@useobject{currentmarker}{}%
\end{pgfscope}%
\end{pgfscope}%
\begin{pgfscope}%
\definecolor{textcolor}{rgb}{0.000000,0.000000,0.000000}%
\pgfsetstrokecolor{textcolor}%
\pgfsetfillcolor{textcolor}%
\pgftext[x=0.368410in, y=4.518925in, left, base]{\color{textcolor}{\sffamily\fontsize{11.000000}{13.200000}\selectfont\catcode`\^=\active\def^{\ifmmode\sp\else\^{}\fi}\catcode`\%=\active\def%{\%}$\mathdefault{100}$}}%
\end{pgfscope}%
\begin{pgfscope}%
\definecolor{textcolor}{rgb}{0.000000,0.000000,0.000000}%
\pgfsetstrokecolor{textcolor}%
\pgfsetfillcolor{textcolor}%
\pgftext[x=0.312854in,y=2.595224in,,bottom,rotate=90.000000]{\color{textcolor}{\sffamily\fontsize{11.000000}{13.200000}\selectfont\catcode`\^=\active\def^{\ifmmode\sp\else\^{}\fi}\catcode`\%=\active\def%{\%}Data reuse (in %)}}%
\end{pgfscope}%
\begin{pgfscope}%
\pgfsetrectcap%
\pgfsetmiterjoin%
\pgfsetlinewidth{0.803000pt}%
\definecolor{currentstroke}{rgb}{0.000000,0.000000,0.000000}%
\pgfsetstrokecolor{currentstroke}%
\pgfsetdash{}{0pt}%
\pgfpathmoveto{\pgfqpoint{0.693757in}{0.613486in}}%
\pgfpathlineto{\pgfqpoint{0.693757in}{4.576962in}}%
\pgfusepath{stroke}%
\end{pgfscope}%
\begin{pgfscope}%
\pgfsetrectcap%
\pgfsetmiterjoin%
\pgfsetlinewidth{0.803000pt}%
\definecolor{currentstroke}{rgb}{0.000000,0.000000,0.000000}%
\pgfsetstrokecolor{currentstroke}%
\pgfsetdash{}{0pt}%
\pgfpathmoveto{\pgfqpoint{6.235000in}{0.613486in}}%
\pgfpathlineto{\pgfqpoint{6.235000in}{4.576962in}}%
\pgfusepath{stroke}%
\end{pgfscope}%
\begin{pgfscope}%
\pgfsetrectcap%
\pgfsetmiterjoin%
\pgfsetlinewidth{0.803000pt}%
\definecolor{currentstroke}{rgb}{0.000000,0.000000,0.000000}%
\pgfsetstrokecolor{currentstroke}%
\pgfsetdash{}{0pt}%
\pgfpathmoveto{\pgfqpoint{0.693757in}{0.613486in}}%
\pgfpathlineto{\pgfqpoint{6.235000in}{0.613486in}}%
\pgfusepath{stroke}%
\end{pgfscope}%
\begin{pgfscope}%
\pgfsetrectcap%
\pgfsetmiterjoin%
\pgfsetlinewidth{0.803000pt}%
\definecolor{currentstroke}{rgb}{0.000000,0.000000,0.000000}%
\pgfsetstrokecolor{currentstroke}%
\pgfsetdash{}{0pt}%
\pgfpathmoveto{\pgfqpoint{0.693757in}{4.576962in}}%
\pgfpathlineto{\pgfqpoint{6.235000in}{4.576962in}}%
\pgfusepath{stroke}%
\end{pgfscope}%
\end{pgfpicture}%
\makeatother%
\endgroup%
}
        \caption{Backward data reuse}
        \label{fig:dct_backward_reuse}
    \end{subfigure}
    \caption{Inter-kernel data reuse in DCT}
    \label{fig:dct_reuse}
\end{figure}

\begin{figure}[ht]
    \centering
    \begin{subfigure}{0.4\textwidth}
        \resizebox{\textwidth}{!}{%% Creator: Matplotlib, PGF backend
%%
%% To include the figure in your LaTeX document, write
%%   \input{<filename>.pgf}
%%
%% Make sure the required packages are loaded in your preamble
%%   \usepackage{pgf}
%%
%% Also ensure that all the required font packages are loaded; for instance,
%% the lmodern package is sometimes necessary when using math font.
%%   \usepackage{lmodern}
%%
%% Figures using additional raster images can only be included by \input if
%% they are in the same directory as the main LaTeX file. For loading figures
%% from other directories you can use the `import` package
%%   \usepackage{import}
%%
%% and then include the figures with
%%   \import{<path to file>}{<filename>.pgf}
%%
%% Matplotlib used the following preamble
%%   \def\mathdefault#1{#1}
%%   \everymath=\expandafter{\the\everymath\displaystyle}
%%   
%%   \usepackage{fontspec}
%%   \setmainfont{DejaVuSerif.ttf}[Path=\detokenize{/usr/lib/python3.11/site-packages/matplotlib/mpl-data/fonts/ttf/}]
%%   \setsansfont{DejaVuSans.ttf}[Path=\detokenize{/usr/lib/python3.11/site-packages/matplotlib/mpl-data/fonts/ttf/}]
%%   \setmonofont{DejaVuSansMono.ttf}[Path=\detokenize{/usr/lib/python3.11/site-packages/matplotlib/mpl-data/fonts/ttf/}]
%%   \makeatletter\@ifpackageloaded{underscore}{}{\usepackage[strings]{underscore}}\makeatother
%%
\begingroup%
\makeatletter%
\begin{pgfpicture}%
\pgfpathrectangle{\pgfpointorigin}{\pgfqpoint{6.400000in}{4.800000in}}%
\pgfusepath{use as bounding box, clip}%
\begin{pgfscope}%
\pgfsetbuttcap%
\pgfsetmiterjoin%
\definecolor{currentfill}{rgb}{1.000000,1.000000,1.000000}%
\pgfsetfillcolor{currentfill}%
\pgfsetlinewidth{0.000000pt}%
\definecolor{currentstroke}{rgb}{1.000000,1.000000,1.000000}%
\pgfsetstrokecolor{currentstroke}%
\pgfsetdash{}{0pt}%
\pgfpathmoveto{\pgfqpoint{0.000000in}{0.000000in}}%
\pgfpathlineto{\pgfqpoint{6.400000in}{0.000000in}}%
\pgfpathlineto{\pgfqpoint{6.400000in}{4.800000in}}%
\pgfpathlineto{\pgfqpoint{0.000000in}{4.800000in}}%
\pgfpathlineto{\pgfqpoint{0.000000in}{0.000000in}}%
\pgfpathclose%
\pgfusepath{fill}%
\end{pgfscope}%
\begin{pgfscope}%
\pgfsetbuttcap%
\pgfsetmiterjoin%
\definecolor{currentfill}{rgb}{1.000000,1.000000,1.000000}%
\pgfsetfillcolor{currentfill}%
\pgfsetlinewidth{0.000000pt}%
\definecolor{currentstroke}{rgb}{0.000000,0.000000,0.000000}%
\pgfsetstrokecolor{currentstroke}%
\pgfsetstrokeopacity{0.000000}%
\pgfsetdash{}{0pt}%
\pgfpathmoveto{\pgfqpoint{0.693757in}{0.613486in}}%
\pgfpathlineto{\pgfqpoint{6.235000in}{0.613486in}}%
\pgfpathlineto{\pgfqpoint{6.235000in}{4.576962in}}%
\pgfpathlineto{\pgfqpoint{0.693757in}{4.576962in}}%
\pgfpathlineto{\pgfqpoint{0.693757in}{0.613486in}}%
\pgfpathclose%
\pgfusepath{fill}%
\end{pgfscope}%
\begin{pgfscope}%
\pgfpathrectangle{\pgfqpoint{0.693757in}{0.613486in}}{\pgfqpoint{5.541243in}{3.963477in}}%
\pgfusepath{clip}%
\pgfsetbuttcap%
\pgfsetmiterjoin%
\definecolor{currentfill}{rgb}{0.000000,0.000000,1.000000}%
\pgfsetfillcolor{currentfill}%
\pgfsetlinewidth{0.000000pt}%
\definecolor{currentstroke}{rgb}{0.000000,0.000000,0.000000}%
\pgfsetstrokecolor{currentstroke}%
\pgfsetstrokeopacity{0.000000}%
\pgfsetdash{}{0pt}%
\pgfpathmoveto{\pgfqpoint{0.945632in}{0.613486in}}%
\pgfpathlineto{\pgfqpoint{0.955637in}{0.613486in}}%
\pgfpathlineto{\pgfqpoint{0.955637in}{1.604355in}}%
\pgfpathlineto{\pgfqpoint{0.945632in}{1.604355in}}%
\pgfpathlineto{\pgfqpoint{0.945632in}{0.613486in}}%
\pgfpathclose%
\pgfusepath{fill}%
\end{pgfscope}%
\begin{pgfscope}%
\pgfpathrectangle{\pgfqpoint{0.693757in}{0.613486in}}{\pgfqpoint{5.541243in}{3.963477in}}%
\pgfusepath{clip}%
\pgfsetbuttcap%
\pgfsetmiterjoin%
\definecolor{currentfill}{rgb}{0.000000,0.000000,1.000000}%
\pgfsetfillcolor{currentfill}%
\pgfsetlinewidth{0.000000pt}%
\definecolor{currentstroke}{rgb}{0.000000,0.000000,0.000000}%
\pgfsetstrokecolor{currentstroke}%
\pgfsetstrokeopacity{0.000000}%
\pgfsetdash{}{0pt}%
\pgfpathmoveto{\pgfqpoint{0.958138in}{0.613486in}}%
\pgfpathlineto{\pgfqpoint{0.968143in}{0.613486in}}%
\pgfpathlineto{\pgfqpoint{0.968143in}{2.611074in}}%
\pgfpathlineto{\pgfqpoint{0.958138in}{2.611074in}}%
\pgfpathlineto{\pgfqpoint{0.958138in}{0.613486in}}%
\pgfpathclose%
\pgfusepath{fill}%
\end{pgfscope}%
\begin{pgfscope}%
\pgfpathrectangle{\pgfqpoint{0.693757in}{0.613486in}}{\pgfqpoint{5.541243in}{3.963477in}}%
\pgfusepath{clip}%
\pgfsetbuttcap%
\pgfsetmiterjoin%
\definecolor{currentfill}{rgb}{0.000000,0.000000,1.000000}%
\pgfsetfillcolor{currentfill}%
\pgfsetlinewidth{0.000000pt}%
\definecolor{currentstroke}{rgb}{0.000000,0.000000,0.000000}%
\pgfsetstrokecolor{currentstroke}%
\pgfsetstrokeopacity{0.000000}%
\pgfsetdash{}{0pt}%
\pgfpathmoveto{\pgfqpoint{0.970644in}{0.613486in}}%
\pgfpathlineto{\pgfqpoint{0.980649in}{0.613486in}}%
\pgfpathlineto{\pgfqpoint{0.980649in}{2.595224in}}%
\pgfpathlineto{\pgfqpoint{0.970644in}{2.595224in}}%
\pgfpathlineto{\pgfqpoint{0.970644in}{0.613486in}}%
\pgfpathclose%
\pgfusepath{fill}%
\end{pgfscope}%
\begin{pgfscope}%
\pgfpathrectangle{\pgfqpoint{0.693757in}{0.613486in}}{\pgfqpoint{5.541243in}{3.963477in}}%
\pgfusepath{clip}%
\pgfsetbuttcap%
\pgfsetmiterjoin%
\definecolor{currentfill}{rgb}{0.000000,0.000000,1.000000}%
\pgfsetfillcolor{currentfill}%
\pgfsetlinewidth{0.000000pt}%
\definecolor{currentstroke}{rgb}{0.000000,0.000000,0.000000}%
\pgfsetstrokecolor{currentstroke}%
\pgfsetstrokeopacity{0.000000}%
\pgfsetdash{}{0pt}%
\pgfpathmoveto{\pgfqpoint{0.983150in}{0.613486in}}%
\pgfpathlineto{\pgfqpoint{0.993155in}{0.613486in}}%
\pgfpathlineto{\pgfqpoint{0.993155in}{1.612282in}}%
\pgfpathlineto{\pgfqpoint{0.983150in}{1.612282in}}%
\pgfpathlineto{\pgfqpoint{0.983150in}{0.613486in}}%
\pgfpathclose%
\pgfusepath{fill}%
\end{pgfscope}%
\begin{pgfscope}%
\pgfpathrectangle{\pgfqpoint{0.693757in}{0.613486in}}{\pgfqpoint{5.541243in}{3.963477in}}%
\pgfusepath{clip}%
\pgfsetbuttcap%
\pgfsetmiterjoin%
\definecolor{currentfill}{rgb}{0.000000,0.000000,1.000000}%
\pgfsetfillcolor{currentfill}%
\pgfsetlinewidth{0.000000pt}%
\definecolor{currentstroke}{rgb}{0.000000,0.000000,0.000000}%
\pgfsetstrokecolor{currentstroke}%
\pgfsetstrokeopacity{0.000000}%
\pgfsetdash{}{0pt}%
\pgfpathmoveto{\pgfqpoint{0.995657in}{0.613486in}}%
\pgfpathlineto{\pgfqpoint{1.005661in}{0.613486in}}%
\pgfpathlineto{\pgfqpoint{1.005661in}{1.604355in}}%
\pgfpathlineto{\pgfqpoint{0.995657in}{1.604355in}}%
\pgfpathlineto{\pgfqpoint{0.995657in}{0.613486in}}%
\pgfpathclose%
\pgfusepath{fill}%
\end{pgfscope}%
\begin{pgfscope}%
\pgfpathrectangle{\pgfqpoint{0.693757in}{0.613486in}}{\pgfqpoint{5.541243in}{3.963477in}}%
\pgfusepath{clip}%
\pgfsetbuttcap%
\pgfsetmiterjoin%
\definecolor{currentfill}{rgb}{0.000000,0.000000,1.000000}%
\pgfsetfillcolor{currentfill}%
\pgfsetlinewidth{0.000000pt}%
\definecolor{currentstroke}{rgb}{0.000000,0.000000,0.000000}%
\pgfsetstrokecolor{currentstroke}%
\pgfsetstrokeopacity{0.000000}%
\pgfsetdash{}{0pt}%
\pgfpathmoveto{\pgfqpoint{1.008163in}{0.613486in}}%
\pgfpathlineto{\pgfqpoint{1.018168in}{0.613486in}}%
\pgfpathlineto{\pgfqpoint{1.018168in}{2.611074in}}%
\pgfpathlineto{\pgfqpoint{1.008163in}{2.611074in}}%
\pgfpathlineto{\pgfqpoint{1.008163in}{0.613486in}}%
\pgfpathclose%
\pgfusepath{fill}%
\end{pgfscope}%
\begin{pgfscope}%
\pgfpathrectangle{\pgfqpoint{0.693757in}{0.613486in}}{\pgfqpoint{5.541243in}{3.963477in}}%
\pgfusepath{clip}%
\pgfsetbuttcap%
\pgfsetmiterjoin%
\definecolor{currentfill}{rgb}{0.000000,0.000000,1.000000}%
\pgfsetfillcolor{currentfill}%
\pgfsetlinewidth{0.000000pt}%
\definecolor{currentstroke}{rgb}{0.000000,0.000000,0.000000}%
\pgfsetstrokecolor{currentstroke}%
\pgfsetstrokeopacity{0.000000}%
\pgfsetdash{}{0pt}%
\pgfpathmoveto{\pgfqpoint{1.020669in}{0.613486in}}%
\pgfpathlineto{\pgfqpoint{1.030674in}{0.613486in}}%
\pgfpathlineto{\pgfqpoint{1.030674in}{2.595224in}}%
\pgfpathlineto{\pgfqpoint{1.020669in}{2.595224in}}%
\pgfpathlineto{\pgfqpoint{1.020669in}{0.613486in}}%
\pgfpathclose%
\pgfusepath{fill}%
\end{pgfscope}%
\begin{pgfscope}%
\pgfpathrectangle{\pgfqpoint{0.693757in}{0.613486in}}{\pgfqpoint{5.541243in}{3.963477in}}%
\pgfusepath{clip}%
\pgfsetbuttcap%
\pgfsetmiterjoin%
\definecolor{currentfill}{rgb}{0.000000,0.000000,1.000000}%
\pgfsetfillcolor{currentfill}%
\pgfsetlinewidth{0.000000pt}%
\definecolor{currentstroke}{rgb}{0.000000,0.000000,0.000000}%
\pgfsetstrokecolor{currentstroke}%
\pgfsetstrokeopacity{0.000000}%
\pgfsetdash{}{0pt}%
\pgfpathmoveto{\pgfqpoint{1.033175in}{0.613486in}}%
\pgfpathlineto{\pgfqpoint{1.043180in}{0.613486in}}%
\pgfpathlineto{\pgfqpoint{1.043180in}{1.612282in}}%
\pgfpathlineto{\pgfqpoint{1.033175in}{1.612282in}}%
\pgfpathlineto{\pgfqpoint{1.033175in}{0.613486in}}%
\pgfpathclose%
\pgfusepath{fill}%
\end{pgfscope}%
\begin{pgfscope}%
\pgfpathrectangle{\pgfqpoint{0.693757in}{0.613486in}}{\pgfqpoint{5.541243in}{3.963477in}}%
\pgfusepath{clip}%
\pgfsetbuttcap%
\pgfsetmiterjoin%
\definecolor{currentfill}{rgb}{0.000000,0.000000,1.000000}%
\pgfsetfillcolor{currentfill}%
\pgfsetlinewidth{0.000000pt}%
\definecolor{currentstroke}{rgb}{0.000000,0.000000,0.000000}%
\pgfsetstrokecolor{currentstroke}%
\pgfsetstrokeopacity{0.000000}%
\pgfsetdash{}{0pt}%
\pgfpathmoveto{\pgfqpoint{1.045681in}{0.613486in}}%
\pgfpathlineto{\pgfqpoint{1.055686in}{0.613486in}}%
\pgfpathlineto{\pgfqpoint{1.055686in}{1.604355in}}%
\pgfpathlineto{\pgfqpoint{1.045681in}{1.604355in}}%
\pgfpathlineto{\pgfqpoint{1.045681in}{0.613486in}}%
\pgfpathclose%
\pgfusepath{fill}%
\end{pgfscope}%
\begin{pgfscope}%
\pgfpathrectangle{\pgfqpoint{0.693757in}{0.613486in}}{\pgfqpoint{5.541243in}{3.963477in}}%
\pgfusepath{clip}%
\pgfsetbuttcap%
\pgfsetmiterjoin%
\definecolor{currentfill}{rgb}{0.000000,0.000000,1.000000}%
\pgfsetfillcolor{currentfill}%
\pgfsetlinewidth{0.000000pt}%
\definecolor{currentstroke}{rgb}{0.000000,0.000000,0.000000}%
\pgfsetstrokecolor{currentstroke}%
\pgfsetstrokeopacity{0.000000}%
\pgfsetdash{}{0pt}%
\pgfpathmoveto{\pgfqpoint{1.058187in}{0.613486in}}%
\pgfpathlineto{\pgfqpoint{1.068192in}{0.613486in}}%
\pgfpathlineto{\pgfqpoint{1.068192in}{2.611074in}}%
\pgfpathlineto{\pgfqpoint{1.058187in}{2.611074in}}%
\pgfpathlineto{\pgfqpoint{1.058187in}{0.613486in}}%
\pgfpathclose%
\pgfusepath{fill}%
\end{pgfscope}%
\begin{pgfscope}%
\pgfpathrectangle{\pgfqpoint{0.693757in}{0.613486in}}{\pgfqpoint{5.541243in}{3.963477in}}%
\pgfusepath{clip}%
\pgfsetbuttcap%
\pgfsetmiterjoin%
\definecolor{currentfill}{rgb}{0.000000,0.000000,1.000000}%
\pgfsetfillcolor{currentfill}%
\pgfsetlinewidth{0.000000pt}%
\definecolor{currentstroke}{rgb}{0.000000,0.000000,0.000000}%
\pgfsetstrokecolor{currentstroke}%
\pgfsetstrokeopacity{0.000000}%
\pgfsetdash{}{0pt}%
\pgfpathmoveto{\pgfqpoint{1.070694in}{0.613486in}}%
\pgfpathlineto{\pgfqpoint{1.080699in}{0.613486in}}%
\pgfpathlineto{\pgfqpoint{1.080699in}{2.595224in}}%
\pgfpathlineto{\pgfqpoint{1.070694in}{2.595224in}}%
\pgfpathlineto{\pgfqpoint{1.070694in}{0.613486in}}%
\pgfpathclose%
\pgfusepath{fill}%
\end{pgfscope}%
\begin{pgfscope}%
\pgfpathrectangle{\pgfqpoint{0.693757in}{0.613486in}}{\pgfqpoint{5.541243in}{3.963477in}}%
\pgfusepath{clip}%
\pgfsetbuttcap%
\pgfsetmiterjoin%
\definecolor{currentfill}{rgb}{0.000000,0.000000,1.000000}%
\pgfsetfillcolor{currentfill}%
\pgfsetlinewidth{0.000000pt}%
\definecolor{currentstroke}{rgb}{0.000000,0.000000,0.000000}%
\pgfsetstrokecolor{currentstroke}%
\pgfsetstrokeopacity{0.000000}%
\pgfsetdash{}{0pt}%
\pgfpathmoveto{\pgfqpoint{1.083200in}{0.613486in}}%
\pgfpathlineto{\pgfqpoint{1.093205in}{0.613486in}}%
\pgfpathlineto{\pgfqpoint{1.093205in}{1.612282in}}%
\pgfpathlineto{\pgfqpoint{1.083200in}{1.612282in}}%
\pgfpathlineto{\pgfqpoint{1.083200in}{0.613486in}}%
\pgfpathclose%
\pgfusepath{fill}%
\end{pgfscope}%
\begin{pgfscope}%
\pgfpathrectangle{\pgfqpoint{0.693757in}{0.613486in}}{\pgfqpoint{5.541243in}{3.963477in}}%
\pgfusepath{clip}%
\pgfsetbuttcap%
\pgfsetmiterjoin%
\definecolor{currentfill}{rgb}{0.000000,0.000000,1.000000}%
\pgfsetfillcolor{currentfill}%
\pgfsetlinewidth{0.000000pt}%
\definecolor{currentstroke}{rgb}{0.000000,0.000000,0.000000}%
\pgfsetstrokecolor{currentstroke}%
\pgfsetstrokeopacity{0.000000}%
\pgfsetdash{}{0pt}%
\pgfpathmoveto{\pgfqpoint{1.095706in}{0.613486in}}%
\pgfpathlineto{\pgfqpoint{1.105711in}{0.613486in}}%
\pgfpathlineto{\pgfqpoint{1.105711in}{1.604355in}}%
\pgfpathlineto{\pgfqpoint{1.095706in}{1.604355in}}%
\pgfpathlineto{\pgfqpoint{1.095706in}{0.613486in}}%
\pgfpathclose%
\pgfusepath{fill}%
\end{pgfscope}%
\begin{pgfscope}%
\pgfpathrectangle{\pgfqpoint{0.693757in}{0.613486in}}{\pgfqpoint{5.541243in}{3.963477in}}%
\pgfusepath{clip}%
\pgfsetbuttcap%
\pgfsetmiterjoin%
\definecolor{currentfill}{rgb}{0.000000,0.000000,1.000000}%
\pgfsetfillcolor{currentfill}%
\pgfsetlinewidth{0.000000pt}%
\definecolor{currentstroke}{rgb}{0.000000,0.000000,0.000000}%
\pgfsetstrokecolor{currentstroke}%
\pgfsetstrokeopacity{0.000000}%
\pgfsetdash{}{0pt}%
\pgfpathmoveto{\pgfqpoint{1.108212in}{0.613486in}}%
\pgfpathlineto{\pgfqpoint{1.118217in}{0.613486in}}%
\pgfpathlineto{\pgfqpoint{1.118217in}{2.611074in}}%
\pgfpathlineto{\pgfqpoint{1.108212in}{2.611074in}}%
\pgfpathlineto{\pgfqpoint{1.108212in}{0.613486in}}%
\pgfpathclose%
\pgfusepath{fill}%
\end{pgfscope}%
\begin{pgfscope}%
\pgfpathrectangle{\pgfqpoint{0.693757in}{0.613486in}}{\pgfqpoint{5.541243in}{3.963477in}}%
\pgfusepath{clip}%
\pgfsetbuttcap%
\pgfsetmiterjoin%
\definecolor{currentfill}{rgb}{0.000000,0.000000,1.000000}%
\pgfsetfillcolor{currentfill}%
\pgfsetlinewidth{0.000000pt}%
\definecolor{currentstroke}{rgb}{0.000000,0.000000,0.000000}%
\pgfsetstrokecolor{currentstroke}%
\pgfsetstrokeopacity{0.000000}%
\pgfsetdash{}{0pt}%
\pgfpathmoveto{\pgfqpoint{1.120718in}{0.613486in}}%
\pgfpathlineto{\pgfqpoint{1.130723in}{0.613486in}}%
\pgfpathlineto{\pgfqpoint{1.130723in}{2.595224in}}%
\pgfpathlineto{\pgfqpoint{1.120718in}{2.595224in}}%
\pgfpathlineto{\pgfqpoint{1.120718in}{0.613486in}}%
\pgfpathclose%
\pgfusepath{fill}%
\end{pgfscope}%
\begin{pgfscope}%
\pgfpathrectangle{\pgfqpoint{0.693757in}{0.613486in}}{\pgfqpoint{5.541243in}{3.963477in}}%
\pgfusepath{clip}%
\pgfsetbuttcap%
\pgfsetmiterjoin%
\definecolor{currentfill}{rgb}{0.000000,0.000000,1.000000}%
\pgfsetfillcolor{currentfill}%
\pgfsetlinewidth{0.000000pt}%
\definecolor{currentstroke}{rgb}{0.000000,0.000000,0.000000}%
\pgfsetstrokecolor{currentstroke}%
\pgfsetstrokeopacity{0.000000}%
\pgfsetdash{}{0pt}%
\pgfpathmoveto{\pgfqpoint{1.133225in}{0.613486in}}%
\pgfpathlineto{\pgfqpoint{1.143230in}{0.613486in}}%
\pgfpathlineto{\pgfqpoint{1.143230in}{1.612282in}}%
\pgfpathlineto{\pgfqpoint{1.133225in}{1.612282in}}%
\pgfpathlineto{\pgfqpoint{1.133225in}{0.613486in}}%
\pgfpathclose%
\pgfusepath{fill}%
\end{pgfscope}%
\begin{pgfscope}%
\pgfpathrectangle{\pgfqpoint{0.693757in}{0.613486in}}{\pgfqpoint{5.541243in}{3.963477in}}%
\pgfusepath{clip}%
\pgfsetbuttcap%
\pgfsetmiterjoin%
\definecolor{currentfill}{rgb}{0.000000,0.000000,1.000000}%
\pgfsetfillcolor{currentfill}%
\pgfsetlinewidth{0.000000pt}%
\definecolor{currentstroke}{rgb}{0.000000,0.000000,0.000000}%
\pgfsetstrokecolor{currentstroke}%
\pgfsetstrokeopacity{0.000000}%
\pgfsetdash{}{0pt}%
\pgfpathmoveto{\pgfqpoint{1.145731in}{0.613486in}}%
\pgfpathlineto{\pgfqpoint{1.155736in}{0.613486in}}%
\pgfpathlineto{\pgfqpoint{1.155736in}{1.604355in}}%
\pgfpathlineto{\pgfqpoint{1.145731in}{1.604355in}}%
\pgfpathlineto{\pgfqpoint{1.145731in}{0.613486in}}%
\pgfpathclose%
\pgfusepath{fill}%
\end{pgfscope}%
\begin{pgfscope}%
\pgfpathrectangle{\pgfqpoint{0.693757in}{0.613486in}}{\pgfqpoint{5.541243in}{3.963477in}}%
\pgfusepath{clip}%
\pgfsetbuttcap%
\pgfsetmiterjoin%
\definecolor{currentfill}{rgb}{0.000000,0.000000,1.000000}%
\pgfsetfillcolor{currentfill}%
\pgfsetlinewidth{0.000000pt}%
\definecolor{currentstroke}{rgb}{0.000000,0.000000,0.000000}%
\pgfsetstrokecolor{currentstroke}%
\pgfsetstrokeopacity{0.000000}%
\pgfsetdash{}{0pt}%
\pgfpathmoveto{\pgfqpoint{1.158237in}{0.613486in}}%
\pgfpathlineto{\pgfqpoint{1.168242in}{0.613486in}}%
\pgfpathlineto{\pgfqpoint{1.168242in}{2.611074in}}%
\pgfpathlineto{\pgfqpoint{1.158237in}{2.611074in}}%
\pgfpathlineto{\pgfqpoint{1.158237in}{0.613486in}}%
\pgfpathclose%
\pgfusepath{fill}%
\end{pgfscope}%
\begin{pgfscope}%
\pgfpathrectangle{\pgfqpoint{0.693757in}{0.613486in}}{\pgfqpoint{5.541243in}{3.963477in}}%
\pgfusepath{clip}%
\pgfsetbuttcap%
\pgfsetmiterjoin%
\definecolor{currentfill}{rgb}{0.000000,0.000000,1.000000}%
\pgfsetfillcolor{currentfill}%
\pgfsetlinewidth{0.000000pt}%
\definecolor{currentstroke}{rgb}{0.000000,0.000000,0.000000}%
\pgfsetstrokecolor{currentstroke}%
\pgfsetstrokeopacity{0.000000}%
\pgfsetdash{}{0pt}%
\pgfpathmoveto{\pgfqpoint{1.170743in}{0.613486in}}%
\pgfpathlineto{\pgfqpoint{1.180748in}{0.613486in}}%
\pgfpathlineto{\pgfqpoint{1.180748in}{2.595224in}}%
\pgfpathlineto{\pgfqpoint{1.170743in}{2.595224in}}%
\pgfpathlineto{\pgfqpoint{1.170743in}{0.613486in}}%
\pgfpathclose%
\pgfusepath{fill}%
\end{pgfscope}%
\begin{pgfscope}%
\pgfpathrectangle{\pgfqpoint{0.693757in}{0.613486in}}{\pgfqpoint{5.541243in}{3.963477in}}%
\pgfusepath{clip}%
\pgfsetbuttcap%
\pgfsetmiterjoin%
\definecolor{currentfill}{rgb}{0.000000,0.000000,1.000000}%
\pgfsetfillcolor{currentfill}%
\pgfsetlinewidth{0.000000pt}%
\definecolor{currentstroke}{rgb}{0.000000,0.000000,0.000000}%
\pgfsetstrokecolor{currentstroke}%
\pgfsetstrokeopacity{0.000000}%
\pgfsetdash{}{0pt}%
\pgfpathmoveto{\pgfqpoint{1.183249in}{0.613486in}}%
\pgfpathlineto{\pgfqpoint{1.193254in}{0.613486in}}%
\pgfpathlineto{\pgfqpoint{1.193254in}{1.612282in}}%
\pgfpathlineto{\pgfqpoint{1.183249in}{1.612282in}}%
\pgfpathlineto{\pgfqpoint{1.183249in}{0.613486in}}%
\pgfpathclose%
\pgfusepath{fill}%
\end{pgfscope}%
\begin{pgfscope}%
\pgfpathrectangle{\pgfqpoint{0.693757in}{0.613486in}}{\pgfqpoint{5.541243in}{3.963477in}}%
\pgfusepath{clip}%
\pgfsetbuttcap%
\pgfsetmiterjoin%
\definecolor{currentfill}{rgb}{0.000000,0.000000,1.000000}%
\pgfsetfillcolor{currentfill}%
\pgfsetlinewidth{0.000000pt}%
\definecolor{currentstroke}{rgb}{0.000000,0.000000,0.000000}%
\pgfsetstrokecolor{currentstroke}%
\pgfsetstrokeopacity{0.000000}%
\pgfsetdash{}{0pt}%
\pgfpathmoveto{\pgfqpoint{1.195756in}{0.613486in}}%
\pgfpathlineto{\pgfqpoint{1.205761in}{0.613486in}}%
\pgfpathlineto{\pgfqpoint{1.205761in}{1.604355in}}%
\pgfpathlineto{\pgfqpoint{1.195756in}{1.604355in}}%
\pgfpathlineto{\pgfqpoint{1.195756in}{0.613486in}}%
\pgfpathclose%
\pgfusepath{fill}%
\end{pgfscope}%
\begin{pgfscope}%
\pgfpathrectangle{\pgfqpoint{0.693757in}{0.613486in}}{\pgfqpoint{5.541243in}{3.963477in}}%
\pgfusepath{clip}%
\pgfsetbuttcap%
\pgfsetmiterjoin%
\definecolor{currentfill}{rgb}{0.000000,0.000000,1.000000}%
\pgfsetfillcolor{currentfill}%
\pgfsetlinewidth{0.000000pt}%
\definecolor{currentstroke}{rgb}{0.000000,0.000000,0.000000}%
\pgfsetstrokecolor{currentstroke}%
\pgfsetstrokeopacity{0.000000}%
\pgfsetdash{}{0pt}%
\pgfpathmoveto{\pgfqpoint{1.208262in}{0.613486in}}%
\pgfpathlineto{\pgfqpoint{1.218267in}{0.613486in}}%
\pgfpathlineto{\pgfqpoint{1.218267in}{2.611074in}}%
\pgfpathlineto{\pgfqpoint{1.208262in}{2.611074in}}%
\pgfpathlineto{\pgfqpoint{1.208262in}{0.613486in}}%
\pgfpathclose%
\pgfusepath{fill}%
\end{pgfscope}%
\begin{pgfscope}%
\pgfpathrectangle{\pgfqpoint{0.693757in}{0.613486in}}{\pgfqpoint{5.541243in}{3.963477in}}%
\pgfusepath{clip}%
\pgfsetbuttcap%
\pgfsetmiterjoin%
\definecolor{currentfill}{rgb}{0.000000,0.000000,1.000000}%
\pgfsetfillcolor{currentfill}%
\pgfsetlinewidth{0.000000pt}%
\definecolor{currentstroke}{rgb}{0.000000,0.000000,0.000000}%
\pgfsetstrokecolor{currentstroke}%
\pgfsetstrokeopacity{0.000000}%
\pgfsetdash{}{0pt}%
\pgfpathmoveto{\pgfqpoint{1.220768in}{0.613486in}}%
\pgfpathlineto{\pgfqpoint{1.230773in}{0.613486in}}%
\pgfpathlineto{\pgfqpoint{1.230773in}{2.595224in}}%
\pgfpathlineto{\pgfqpoint{1.220768in}{2.595224in}}%
\pgfpathlineto{\pgfqpoint{1.220768in}{0.613486in}}%
\pgfpathclose%
\pgfusepath{fill}%
\end{pgfscope}%
\begin{pgfscope}%
\pgfpathrectangle{\pgfqpoint{0.693757in}{0.613486in}}{\pgfqpoint{5.541243in}{3.963477in}}%
\pgfusepath{clip}%
\pgfsetbuttcap%
\pgfsetmiterjoin%
\definecolor{currentfill}{rgb}{0.000000,0.000000,1.000000}%
\pgfsetfillcolor{currentfill}%
\pgfsetlinewidth{0.000000pt}%
\definecolor{currentstroke}{rgb}{0.000000,0.000000,0.000000}%
\pgfsetstrokecolor{currentstroke}%
\pgfsetstrokeopacity{0.000000}%
\pgfsetdash{}{0pt}%
\pgfpathmoveto{\pgfqpoint{1.233274in}{0.613486in}}%
\pgfpathlineto{\pgfqpoint{1.243279in}{0.613486in}}%
\pgfpathlineto{\pgfqpoint{1.243279in}{1.612282in}}%
\pgfpathlineto{\pgfqpoint{1.233274in}{1.612282in}}%
\pgfpathlineto{\pgfqpoint{1.233274in}{0.613486in}}%
\pgfpathclose%
\pgfusepath{fill}%
\end{pgfscope}%
\begin{pgfscope}%
\pgfpathrectangle{\pgfqpoint{0.693757in}{0.613486in}}{\pgfqpoint{5.541243in}{3.963477in}}%
\pgfusepath{clip}%
\pgfsetbuttcap%
\pgfsetmiterjoin%
\definecolor{currentfill}{rgb}{0.000000,0.000000,1.000000}%
\pgfsetfillcolor{currentfill}%
\pgfsetlinewidth{0.000000pt}%
\definecolor{currentstroke}{rgb}{0.000000,0.000000,0.000000}%
\pgfsetstrokecolor{currentstroke}%
\pgfsetstrokeopacity{0.000000}%
\pgfsetdash{}{0pt}%
\pgfpathmoveto{\pgfqpoint{1.245780in}{0.613486in}}%
\pgfpathlineto{\pgfqpoint{1.255785in}{0.613486in}}%
\pgfpathlineto{\pgfqpoint{1.255785in}{1.604355in}}%
\pgfpathlineto{\pgfqpoint{1.245780in}{1.604355in}}%
\pgfpathlineto{\pgfqpoint{1.245780in}{0.613486in}}%
\pgfpathclose%
\pgfusepath{fill}%
\end{pgfscope}%
\begin{pgfscope}%
\pgfpathrectangle{\pgfqpoint{0.693757in}{0.613486in}}{\pgfqpoint{5.541243in}{3.963477in}}%
\pgfusepath{clip}%
\pgfsetbuttcap%
\pgfsetmiterjoin%
\definecolor{currentfill}{rgb}{0.000000,0.000000,1.000000}%
\pgfsetfillcolor{currentfill}%
\pgfsetlinewidth{0.000000pt}%
\definecolor{currentstroke}{rgb}{0.000000,0.000000,0.000000}%
\pgfsetstrokecolor{currentstroke}%
\pgfsetstrokeopacity{0.000000}%
\pgfsetdash{}{0pt}%
\pgfpathmoveto{\pgfqpoint{1.258287in}{0.613486in}}%
\pgfpathlineto{\pgfqpoint{1.268291in}{0.613486in}}%
\pgfpathlineto{\pgfqpoint{1.268291in}{2.611074in}}%
\pgfpathlineto{\pgfqpoint{1.258287in}{2.611074in}}%
\pgfpathlineto{\pgfqpoint{1.258287in}{0.613486in}}%
\pgfpathclose%
\pgfusepath{fill}%
\end{pgfscope}%
\begin{pgfscope}%
\pgfpathrectangle{\pgfqpoint{0.693757in}{0.613486in}}{\pgfqpoint{5.541243in}{3.963477in}}%
\pgfusepath{clip}%
\pgfsetbuttcap%
\pgfsetmiterjoin%
\definecolor{currentfill}{rgb}{0.000000,0.000000,1.000000}%
\pgfsetfillcolor{currentfill}%
\pgfsetlinewidth{0.000000pt}%
\definecolor{currentstroke}{rgb}{0.000000,0.000000,0.000000}%
\pgfsetstrokecolor{currentstroke}%
\pgfsetstrokeopacity{0.000000}%
\pgfsetdash{}{0pt}%
\pgfpathmoveto{\pgfqpoint{1.270793in}{0.613486in}}%
\pgfpathlineto{\pgfqpoint{1.280798in}{0.613486in}}%
\pgfpathlineto{\pgfqpoint{1.280798in}{2.595224in}}%
\pgfpathlineto{\pgfqpoint{1.270793in}{2.595224in}}%
\pgfpathlineto{\pgfqpoint{1.270793in}{0.613486in}}%
\pgfpathclose%
\pgfusepath{fill}%
\end{pgfscope}%
\begin{pgfscope}%
\pgfpathrectangle{\pgfqpoint{0.693757in}{0.613486in}}{\pgfqpoint{5.541243in}{3.963477in}}%
\pgfusepath{clip}%
\pgfsetbuttcap%
\pgfsetmiterjoin%
\definecolor{currentfill}{rgb}{0.000000,0.000000,1.000000}%
\pgfsetfillcolor{currentfill}%
\pgfsetlinewidth{0.000000pt}%
\definecolor{currentstroke}{rgb}{0.000000,0.000000,0.000000}%
\pgfsetstrokecolor{currentstroke}%
\pgfsetstrokeopacity{0.000000}%
\pgfsetdash{}{0pt}%
\pgfpathmoveto{\pgfqpoint{1.283299in}{0.613486in}}%
\pgfpathlineto{\pgfqpoint{1.293304in}{0.613486in}}%
\pgfpathlineto{\pgfqpoint{1.293304in}{1.612282in}}%
\pgfpathlineto{\pgfqpoint{1.283299in}{1.612282in}}%
\pgfpathlineto{\pgfqpoint{1.283299in}{0.613486in}}%
\pgfpathclose%
\pgfusepath{fill}%
\end{pgfscope}%
\begin{pgfscope}%
\pgfpathrectangle{\pgfqpoint{0.693757in}{0.613486in}}{\pgfqpoint{5.541243in}{3.963477in}}%
\pgfusepath{clip}%
\pgfsetbuttcap%
\pgfsetmiterjoin%
\definecolor{currentfill}{rgb}{0.000000,0.000000,1.000000}%
\pgfsetfillcolor{currentfill}%
\pgfsetlinewidth{0.000000pt}%
\definecolor{currentstroke}{rgb}{0.000000,0.000000,0.000000}%
\pgfsetstrokecolor{currentstroke}%
\pgfsetstrokeopacity{0.000000}%
\pgfsetdash{}{0pt}%
\pgfpathmoveto{\pgfqpoint{1.295805in}{0.613486in}}%
\pgfpathlineto{\pgfqpoint{1.305810in}{0.613486in}}%
\pgfpathlineto{\pgfqpoint{1.305810in}{1.604355in}}%
\pgfpathlineto{\pgfqpoint{1.295805in}{1.604355in}}%
\pgfpathlineto{\pgfqpoint{1.295805in}{0.613486in}}%
\pgfpathclose%
\pgfusepath{fill}%
\end{pgfscope}%
\begin{pgfscope}%
\pgfpathrectangle{\pgfqpoint{0.693757in}{0.613486in}}{\pgfqpoint{5.541243in}{3.963477in}}%
\pgfusepath{clip}%
\pgfsetbuttcap%
\pgfsetmiterjoin%
\definecolor{currentfill}{rgb}{0.000000,0.000000,1.000000}%
\pgfsetfillcolor{currentfill}%
\pgfsetlinewidth{0.000000pt}%
\definecolor{currentstroke}{rgb}{0.000000,0.000000,0.000000}%
\pgfsetstrokecolor{currentstroke}%
\pgfsetstrokeopacity{0.000000}%
\pgfsetdash{}{0pt}%
\pgfpathmoveto{\pgfqpoint{1.308311in}{0.613486in}}%
\pgfpathlineto{\pgfqpoint{1.318316in}{0.613486in}}%
\pgfpathlineto{\pgfqpoint{1.318316in}{2.611074in}}%
\pgfpathlineto{\pgfqpoint{1.308311in}{2.611074in}}%
\pgfpathlineto{\pgfqpoint{1.308311in}{0.613486in}}%
\pgfpathclose%
\pgfusepath{fill}%
\end{pgfscope}%
\begin{pgfscope}%
\pgfpathrectangle{\pgfqpoint{0.693757in}{0.613486in}}{\pgfqpoint{5.541243in}{3.963477in}}%
\pgfusepath{clip}%
\pgfsetbuttcap%
\pgfsetmiterjoin%
\definecolor{currentfill}{rgb}{0.000000,0.000000,1.000000}%
\pgfsetfillcolor{currentfill}%
\pgfsetlinewidth{0.000000pt}%
\definecolor{currentstroke}{rgb}{0.000000,0.000000,0.000000}%
\pgfsetstrokecolor{currentstroke}%
\pgfsetstrokeopacity{0.000000}%
\pgfsetdash{}{0pt}%
\pgfpathmoveto{\pgfqpoint{1.320817in}{0.613486in}}%
\pgfpathlineto{\pgfqpoint{1.330822in}{0.613486in}}%
\pgfpathlineto{\pgfqpoint{1.330822in}{2.595224in}}%
\pgfpathlineto{\pgfqpoint{1.320817in}{2.595224in}}%
\pgfpathlineto{\pgfqpoint{1.320817in}{0.613486in}}%
\pgfpathclose%
\pgfusepath{fill}%
\end{pgfscope}%
\begin{pgfscope}%
\pgfpathrectangle{\pgfqpoint{0.693757in}{0.613486in}}{\pgfqpoint{5.541243in}{3.963477in}}%
\pgfusepath{clip}%
\pgfsetbuttcap%
\pgfsetmiterjoin%
\definecolor{currentfill}{rgb}{0.000000,0.000000,1.000000}%
\pgfsetfillcolor{currentfill}%
\pgfsetlinewidth{0.000000pt}%
\definecolor{currentstroke}{rgb}{0.000000,0.000000,0.000000}%
\pgfsetstrokecolor{currentstroke}%
\pgfsetstrokeopacity{0.000000}%
\pgfsetdash{}{0pt}%
\pgfpathmoveto{\pgfqpoint{1.333324in}{0.613486in}}%
\pgfpathlineto{\pgfqpoint{1.343329in}{0.613486in}}%
\pgfpathlineto{\pgfqpoint{1.343329in}{1.612282in}}%
\pgfpathlineto{\pgfqpoint{1.333324in}{1.612282in}}%
\pgfpathlineto{\pgfqpoint{1.333324in}{0.613486in}}%
\pgfpathclose%
\pgfusepath{fill}%
\end{pgfscope}%
\begin{pgfscope}%
\pgfpathrectangle{\pgfqpoint{0.693757in}{0.613486in}}{\pgfqpoint{5.541243in}{3.963477in}}%
\pgfusepath{clip}%
\pgfsetbuttcap%
\pgfsetmiterjoin%
\definecolor{currentfill}{rgb}{0.000000,0.000000,1.000000}%
\pgfsetfillcolor{currentfill}%
\pgfsetlinewidth{0.000000pt}%
\definecolor{currentstroke}{rgb}{0.000000,0.000000,0.000000}%
\pgfsetstrokecolor{currentstroke}%
\pgfsetstrokeopacity{0.000000}%
\pgfsetdash{}{0pt}%
\pgfpathmoveto{\pgfqpoint{1.345830in}{0.613486in}}%
\pgfpathlineto{\pgfqpoint{1.355835in}{0.613486in}}%
\pgfpathlineto{\pgfqpoint{1.355835in}{1.604355in}}%
\pgfpathlineto{\pgfqpoint{1.345830in}{1.604355in}}%
\pgfpathlineto{\pgfqpoint{1.345830in}{0.613486in}}%
\pgfpathclose%
\pgfusepath{fill}%
\end{pgfscope}%
\begin{pgfscope}%
\pgfpathrectangle{\pgfqpoint{0.693757in}{0.613486in}}{\pgfqpoint{5.541243in}{3.963477in}}%
\pgfusepath{clip}%
\pgfsetbuttcap%
\pgfsetmiterjoin%
\definecolor{currentfill}{rgb}{0.000000,0.000000,1.000000}%
\pgfsetfillcolor{currentfill}%
\pgfsetlinewidth{0.000000pt}%
\definecolor{currentstroke}{rgb}{0.000000,0.000000,0.000000}%
\pgfsetstrokecolor{currentstroke}%
\pgfsetstrokeopacity{0.000000}%
\pgfsetdash{}{0pt}%
\pgfpathmoveto{\pgfqpoint{1.358336in}{0.613486in}}%
\pgfpathlineto{\pgfqpoint{1.368341in}{0.613486in}}%
\pgfpathlineto{\pgfqpoint{1.368341in}{2.611074in}}%
\pgfpathlineto{\pgfqpoint{1.358336in}{2.611074in}}%
\pgfpathlineto{\pgfqpoint{1.358336in}{0.613486in}}%
\pgfpathclose%
\pgfusepath{fill}%
\end{pgfscope}%
\begin{pgfscope}%
\pgfpathrectangle{\pgfqpoint{0.693757in}{0.613486in}}{\pgfqpoint{5.541243in}{3.963477in}}%
\pgfusepath{clip}%
\pgfsetbuttcap%
\pgfsetmiterjoin%
\definecolor{currentfill}{rgb}{0.000000,0.000000,1.000000}%
\pgfsetfillcolor{currentfill}%
\pgfsetlinewidth{0.000000pt}%
\definecolor{currentstroke}{rgb}{0.000000,0.000000,0.000000}%
\pgfsetstrokecolor{currentstroke}%
\pgfsetstrokeopacity{0.000000}%
\pgfsetdash{}{0pt}%
\pgfpathmoveto{\pgfqpoint{1.370842in}{0.613486in}}%
\pgfpathlineto{\pgfqpoint{1.380847in}{0.613486in}}%
\pgfpathlineto{\pgfqpoint{1.380847in}{2.595224in}}%
\pgfpathlineto{\pgfqpoint{1.370842in}{2.595224in}}%
\pgfpathlineto{\pgfqpoint{1.370842in}{0.613486in}}%
\pgfpathclose%
\pgfusepath{fill}%
\end{pgfscope}%
\begin{pgfscope}%
\pgfpathrectangle{\pgfqpoint{0.693757in}{0.613486in}}{\pgfqpoint{5.541243in}{3.963477in}}%
\pgfusepath{clip}%
\pgfsetbuttcap%
\pgfsetmiterjoin%
\definecolor{currentfill}{rgb}{0.000000,0.000000,1.000000}%
\pgfsetfillcolor{currentfill}%
\pgfsetlinewidth{0.000000pt}%
\definecolor{currentstroke}{rgb}{0.000000,0.000000,0.000000}%
\pgfsetstrokecolor{currentstroke}%
\pgfsetstrokeopacity{0.000000}%
\pgfsetdash{}{0pt}%
\pgfpathmoveto{\pgfqpoint{1.383348in}{0.613486in}}%
\pgfpathlineto{\pgfqpoint{1.393353in}{0.613486in}}%
\pgfpathlineto{\pgfqpoint{1.393353in}{1.612282in}}%
\pgfpathlineto{\pgfqpoint{1.383348in}{1.612282in}}%
\pgfpathlineto{\pgfqpoint{1.383348in}{0.613486in}}%
\pgfpathclose%
\pgfusepath{fill}%
\end{pgfscope}%
\begin{pgfscope}%
\pgfpathrectangle{\pgfqpoint{0.693757in}{0.613486in}}{\pgfqpoint{5.541243in}{3.963477in}}%
\pgfusepath{clip}%
\pgfsetbuttcap%
\pgfsetmiterjoin%
\definecolor{currentfill}{rgb}{0.000000,0.000000,1.000000}%
\pgfsetfillcolor{currentfill}%
\pgfsetlinewidth{0.000000pt}%
\definecolor{currentstroke}{rgb}{0.000000,0.000000,0.000000}%
\pgfsetstrokecolor{currentstroke}%
\pgfsetstrokeopacity{0.000000}%
\pgfsetdash{}{0pt}%
\pgfpathmoveto{\pgfqpoint{1.395855in}{0.613486in}}%
\pgfpathlineto{\pgfqpoint{1.405860in}{0.613486in}}%
\pgfpathlineto{\pgfqpoint{1.405860in}{1.604355in}}%
\pgfpathlineto{\pgfqpoint{1.395855in}{1.604355in}}%
\pgfpathlineto{\pgfqpoint{1.395855in}{0.613486in}}%
\pgfpathclose%
\pgfusepath{fill}%
\end{pgfscope}%
\begin{pgfscope}%
\pgfpathrectangle{\pgfqpoint{0.693757in}{0.613486in}}{\pgfqpoint{5.541243in}{3.963477in}}%
\pgfusepath{clip}%
\pgfsetbuttcap%
\pgfsetmiterjoin%
\definecolor{currentfill}{rgb}{0.000000,0.000000,1.000000}%
\pgfsetfillcolor{currentfill}%
\pgfsetlinewidth{0.000000pt}%
\definecolor{currentstroke}{rgb}{0.000000,0.000000,0.000000}%
\pgfsetstrokecolor{currentstroke}%
\pgfsetstrokeopacity{0.000000}%
\pgfsetdash{}{0pt}%
\pgfpathmoveto{\pgfqpoint{1.408361in}{0.613486in}}%
\pgfpathlineto{\pgfqpoint{1.418366in}{0.613486in}}%
\pgfpathlineto{\pgfqpoint{1.418366in}{2.611074in}}%
\pgfpathlineto{\pgfqpoint{1.408361in}{2.611074in}}%
\pgfpathlineto{\pgfqpoint{1.408361in}{0.613486in}}%
\pgfpathclose%
\pgfusepath{fill}%
\end{pgfscope}%
\begin{pgfscope}%
\pgfpathrectangle{\pgfqpoint{0.693757in}{0.613486in}}{\pgfqpoint{5.541243in}{3.963477in}}%
\pgfusepath{clip}%
\pgfsetbuttcap%
\pgfsetmiterjoin%
\definecolor{currentfill}{rgb}{0.000000,0.000000,1.000000}%
\pgfsetfillcolor{currentfill}%
\pgfsetlinewidth{0.000000pt}%
\definecolor{currentstroke}{rgb}{0.000000,0.000000,0.000000}%
\pgfsetstrokecolor{currentstroke}%
\pgfsetstrokeopacity{0.000000}%
\pgfsetdash{}{0pt}%
\pgfpathmoveto{\pgfqpoint{1.420867in}{0.613486in}}%
\pgfpathlineto{\pgfqpoint{1.430872in}{0.613486in}}%
\pgfpathlineto{\pgfqpoint{1.430872in}{2.595224in}}%
\pgfpathlineto{\pgfqpoint{1.420867in}{2.595224in}}%
\pgfpathlineto{\pgfqpoint{1.420867in}{0.613486in}}%
\pgfpathclose%
\pgfusepath{fill}%
\end{pgfscope}%
\begin{pgfscope}%
\pgfpathrectangle{\pgfqpoint{0.693757in}{0.613486in}}{\pgfqpoint{5.541243in}{3.963477in}}%
\pgfusepath{clip}%
\pgfsetbuttcap%
\pgfsetmiterjoin%
\definecolor{currentfill}{rgb}{0.000000,0.000000,1.000000}%
\pgfsetfillcolor{currentfill}%
\pgfsetlinewidth{0.000000pt}%
\definecolor{currentstroke}{rgb}{0.000000,0.000000,0.000000}%
\pgfsetstrokecolor{currentstroke}%
\pgfsetstrokeopacity{0.000000}%
\pgfsetdash{}{0pt}%
\pgfpathmoveto{\pgfqpoint{1.433373in}{0.613486in}}%
\pgfpathlineto{\pgfqpoint{1.443378in}{0.613486in}}%
\pgfpathlineto{\pgfqpoint{1.443378in}{1.612282in}}%
\pgfpathlineto{\pgfqpoint{1.433373in}{1.612282in}}%
\pgfpathlineto{\pgfqpoint{1.433373in}{0.613486in}}%
\pgfpathclose%
\pgfusepath{fill}%
\end{pgfscope}%
\begin{pgfscope}%
\pgfpathrectangle{\pgfqpoint{0.693757in}{0.613486in}}{\pgfqpoint{5.541243in}{3.963477in}}%
\pgfusepath{clip}%
\pgfsetbuttcap%
\pgfsetmiterjoin%
\definecolor{currentfill}{rgb}{0.000000,0.000000,1.000000}%
\pgfsetfillcolor{currentfill}%
\pgfsetlinewidth{0.000000pt}%
\definecolor{currentstroke}{rgb}{0.000000,0.000000,0.000000}%
\pgfsetstrokecolor{currentstroke}%
\pgfsetstrokeopacity{0.000000}%
\pgfsetdash{}{0pt}%
\pgfpathmoveto{\pgfqpoint{1.445879in}{0.613486in}}%
\pgfpathlineto{\pgfqpoint{1.455884in}{0.613486in}}%
\pgfpathlineto{\pgfqpoint{1.455884in}{1.604355in}}%
\pgfpathlineto{\pgfqpoint{1.445879in}{1.604355in}}%
\pgfpathlineto{\pgfqpoint{1.445879in}{0.613486in}}%
\pgfpathclose%
\pgfusepath{fill}%
\end{pgfscope}%
\begin{pgfscope}%
\pgfpathrectangle{\pgfqpoint{0.693757in}{0.613486in}}{\pgfqpoint{5.541243in}{3.963477in}}%
\pgfusepath{clip}%
\pgfsetbuttcap%
\pgfsetmiterjoin%
\definecolor{currentfill}{rgb}{0.000000,0.000000,1.000000}%
\pgfsetfillcolor{currentfill}%
\pgfsetlinewidth{0.000000pt}%
\definecolor{currentstroke}{rgb}{0.000000,0.000000,0.000000}%
\pgfsetstrokecolor{currentstroke}%
\pgfsetstrokeopacity{0.000000}%
\pgfsetdash{}{0pt}%
\pgfpathmoveto{\pgfqpoint{1.458386in}{0.613486in}}%
\pgfpathlineto{\pgfqpoint{1.468391in}{0.613486in}}%
\pgfpathlineto{\pgfqpoint{1.468391in}{2.611074in}}%
\pgfpathlineto{\pgfqpoint{1.458386in}{2.611074in}}%
\pgfpathlineto{\pgfqpoint{1.458386in}{0.613486in}}%
\pgfpathclose%
\pgfusepath{fill}%
\end{pgfscope}%
\begin{pgfscope}%
\pgfpathrectangle{\pgfqpoint{0.693757in}{0.613486in}}{\pgfqpoint{5.541243in}{3.963477in}}%
\pgfusepath{clip}%
\pgfsetbuttcap%
\pgfsetmiterjoin%
\definecolor{currentfill}{rgb}{0.000000,0.000000,1.000000}%
\pgfsetfillcolor{currentfill}%
\pgfsetlinewidth{0.000000pt}%
\definecolor{currentstroke}{rgb}{0.000000,0.000000,0.000000}%
\pgfsetstrokecolor{currentstroke}%
\pgfsetstrokeopacity{0.000000}%
\pgfsetdash{}{0pt}%
\pgfpathmoveto{\pgfqpoint{1.470892in}{0.613486in}}%
\pgfpathlineto{\pgfqpoint{1.480897in}{0.613486in}}%
\pgfpathlineto{\pgfqpoint{1.480897in}{2.595224in}}%
\pgfpathlineto{\pgfqpoint{1.470892in}{2.595224in}}%
\pgfpathlineto{\pgfqpoint{1.470892in}{0.613486in}}%
\pgfpathclose%
\pgfusepath{fill}%
\end{pgfscope}%
\begin{pgfscope}%
\pgfpathrectangle{\pgfqpoint{0.693757in}{0.613486in}}{\pgfqpoint{5.541243in}{3.963477in}}%
\pgfusepath{clip}%
\pgfsetbuttcap%
\pgfsetmiterjoin%
\definecolor{currentfill}{rgb}{0.000000,0.000000,1.000000}%
\pgfsetfillcolor{currentfill}%
\pgfsetlinewidth{0.000000pt}%
\definecolor{currentstroke}{rgb}{0.000000,0.000000,0.000000}%
\pgfsetstrokecolor{currentstroke}%
\pgfsetstrokeopacity{0.000000}%
\pgfsetdash{}{0pt}%
\pgfpathmoveto{\pgfqpoint{1.483398in}{0.613486in}}%
\pgfpathlineto{\pgfqpoint{1.493403in}{0.613486in}}%
\pgfpathlineto{\pgfqpoint{1.493403in}{1.612282in}}%
\pgfpathlineto{\pgfqpoint{1.483398in}{1.612282in}}%
\pgfpathlineto{\pgfqpoint{1.483398in}{0.613486in}}%
\pgfpathclose%
\pgfusepath{fill}%
\end{pgfscope}%
\begin{pgfscope}%
\pgfpathrectangle{\pgfqpoint{0.693757in}{0.613486in}}{\pgfqpoint{5.541243in}{3.963477in}}%
\pgfusepath{clip}%
\pgfsetbuttcap%
\pgfsetmiterjoin%
\definecolor{currentfill}{rgb}{0.000000,0.000000,1.000000}%
\pgfsetfillcolor{currentfill}%
\pgfsetlinewidth{0.000000pt}%
\definecolor{currentstroke}{rgb}{0.000000,0.000000,0.000000}%
\pgfsetstrokecolor{currentstroke}%
\pgfsetstrokeopacity{0.000000}%
\pgfsetdash{}{0pt}%
\pgfpathmoveto{\pgfqpoint{1.495904in}{0.613486in}}%
\pgfpathlineto{\pgfqpoint{1.505909in}{0.613486in}}%
\pgfpathlineto{\pgfqpoint{1.505909in}{1.604355in}}%
\pgfpathlineto{\pgfqpoint{1.495904in}{1.604355in}}%
\pgfpathlineto{\pgfqpoint{1.495904in}{0.613486in}}%
\pgfpathclose%
\pgfusepath{fill}%
\end{pgfscope}%
\begin{pgfscope}%
\pgfpathrectangle{\pgfqpoint{0.693757in}{0.613486in}}{\pgfqpoint{5.541243in}{3.963477in}}%
\pgfusepath{clip}%
\pgfsetbuttcap%
\pgfsetmiterjoin%
\definecolor{currentfill}{rgb}{0.000000,0.000000,1.000000}%
\pgfsetfillcolor{currentfill}%
\pgfsetlinewidth{0.000000pt}%
\definecolor{currentstroke}{rgb}{0.000000,0.000000,0.000000}%
\pgfsetstrokecolor{currentstroke}%
\pgfsetstrokeopacity{0.000000}%
\pgfsetdash{}{0pt}%
\pgfpathmoveto{\pgfqpoint{1.508410in}{0.613486in}}%
\pgfpathlineto{\pgfqpoint{1.518415in}{0.613486in}}%
\pgfpathlineto{\pgfqpoint{1.518415in}{2.611074in}}%
\pgfpathlineto{\pgfqpoint{1.508410in}{2.611074in}}%
\pgfpathlineto{\pgfqpoint{1.508410in}{0.613486in}}%
\pgfpathclose%
\pgfusepath{fill}%
\end{pgfscope}%
\begin{pgfscope}%
\pgfpathrectangle{\pgfqpoint{0.693757in}{0.613486in}}{\pgfqpoint{5.541243in}{3.963477in}}%
\pgfusepath{clip}%
\pgfsetbuttcap%
\pgfsetmiterjoin%
\definecolor{currentfill}{rgb}{0.000000,0.000000,1.000000}%
\pgfsetfillcolor{currentfill}%
\pgfsetlinewidth{0.000000pt}%
\definecolor{currentstroke}{rgb}{0.000000,0.000000,0.000000}%
\pgfsetstrokecolor{currentstroke}%
\pgfsetstrokeopacity{0.000000}%
\pgfsetdash{}{0pt}%
\pgfpathmoveto{\pgfqpoint{1.520917in}{0.613486in}}%
\pgfpathlineto{\pgfqpoint{1.530921in}{0.613486in}}%
\pgfpathlineto{\pgfqpoint{1.530921in}{2.595224in}}%
\pgfpathlineto{\pgfqpoint{1.520917in}{2.595224in}}%
\pgfpathlineto{\pgfqpoint{1.520917in}{0.613486in}}%
\pgfpathclose%
\pgfusepath{fill}%
\end{pgfscope}%
\begin{pgfscope}%
\pgfpathrectangle{\pgfqpoint{0.693757in}{0.613486in}}{\pgfqpoint{5.541243in}{3.963477in}}%
\pgfusepath{clip}%
\pgfsetbuttcap%
\pgfsetmiterjoin%
\definecolor{currentfill}{rgb}{0.000000,0.000000,1.000000}%
\pgfsetfillcolor{currentfill}%
\pgfsetlinewidth{0.000000pt}%
\definecolor{currentstroke}{rgb}{0.000000,0.000000,0.000000}%
\pgfsetstrokecolor{currentstroke}%
\pgfsetstrokeopacity{0.000000}%
\pgfsetdash{}{0pt}%
\pgfpathmoveto{\pgfqpoint{1.533423in}{0.613486in}}%
\pgfpathlineto{\pgfqpoint{1.543428in}{0.613486in}}%
\pgfpathlineto{\pgfqpoint{1.543428in}{1.612282in}}%
\pgfpathlineto{\pgfqpoint{1.533423in}{1.612282in}}%
\pgfpathlineto{\pgfqpoint{1.533423in}{0.613486in}}%
\pgfpathclose%
\pgfusepath{fill}%
\end{pgfscope}%
\begin{pgfscope}%
\pgfpathrectangle{\pgfqpoint{0.693757in}{0.613486in}}{\pgfqpoint{5.541243in}{3.963477in}}%
\pgfusepath{clip}%
\pgfsetbuttcap%
\pgfsetmiterjoin%
\definecolor{currentfill}{rgb}{0.000000,0.000000,1.000000}%
\pgfsetfillcolor{currentfill}%
\pgfsetlinewidth{0.000000pt}%
\definecolor{currentstroke}{rgb}{0.000000,0.000000,0.000000}%
\pgfsetstrokecolor{currentstroke}%
\pgfsetstrokeopacity{0.000000}%
\pgfsetdash{}{0pt}%
\pgfpathmoveto{\pgfqpoint{1.545929in}{0.613486in}}%
\pgfpathlineto{\pgfqpoint{1.555934in}{0.613486in}}%
\pgfpathlineto{\pgfqpoint{1.555934in}{1.604355in}}%
\pgfpathlineto{\pgfqpoint{1.545929in}{1.604355in}}%
\pgfpathlineto{\pgfqpoint{1.545929in}{0.613486in}}%
\pgfpathclose%
\pgfusepath{fill}%
\end{pgfscope}%
\begin{pgfscope}%
\pgfpathrectangle{\pgfqpoint{0.693757in}{0.613486in}}{\pgfqpoint{5.541243in}{3.963477in}}%
\pgfusepath{clip}%
\pgfsetbuttcap%
\pgfsetmiterjoin%
\definecolor{currentfill}{rgb}{0.000000,0.000000,1.000000}%
\pgfsetfillcolor{currentfill}%
\pgfsetlinewidth{0.000000pt}%
\definecolor{currentstroke}{rgb}{0.000000,0.000000,0.000000}%
\pgfsetstrokecolor{currentstroke}%
\pgfsetstrokeopacity{0.000000}%
\pgfsetdash{}{0pt}%
\pgfpathmoveto{\pgfqpoint{1.558435in}{0.613486in}}%
\pgfpathlineto{\pgfqpoint{1.568440in}{0.613486in}}%
\pgfpathlineto{\pgfqpoint{1.568440in}{2.611074in}}%
\pgfpathlineto{\pgfqpoint{1.558435in}{2.611074in}}%
\pgfpathlineto{\pgfqpoint{1.558435in}{0.613486in}}%
\pgfpathclose%
\pgfusepath{fill}%
\end{pgfscope}%
\begin{pgfscope}%
\pgfpathrectangle{\pgfqpoint{0.693757in}{0.613486in}}{\pgfqpoint{5.541243in}{3.963477in}}%
\pgfusepath{clip}%
\pgfsetbuttcap%
\pgfsetmiterjoin%
\definecolor{currentfill}{rgb}{0.000000,0.000000,1.000000}%
\pgfsetfillcolor{currentfill}%
\pgfsetlinewidth{0.000000pt}%
\definecolor{currentstroke}{rgb}{0.000000,0.000000,0.000000}%
\pgfsetstrokecolor{currentstroke}%
\pgfsetstrokeopacity{0.000000}%
\pgfsetdash{}{0pt}%
\pgfpathmoveto{\pgfqpoint{1.570941in}{0.613486in}}%
\pgfpathlineto{\pgfqpoint{1.580946in}{0.613486in}}%
\pgfpathlineto{\pgfqpoint{1.580946in}{2.595224in}}%
\pgfpathlineto{\pgfqpoint{1.570941in}{2.595224in}}%
\pgfpathlineto{\pgfqpoint{1.570941in}{0.613486in}}%
\pgfpathclose%
\pgfusepath{fill}%
\end{pgfscope}%
\begin{pgfscope}%
\pgfpathrectangle{\pgfqpoint{0.693757in}{0.613486in}}{\pgfqpoint{5.541243in}{3.963477in}}%
\pgfusepath{clip}%
\pgfsetbuttcap%
\pgfsetmiterjoin%
\definecolor{currentfill}{rgb}{0.000000,0.000000,1.000000}%
\pgfsetfillcolor{currentfill}%
\pgfsetlinewidth{0.000000pt}%
\definecolor{currentstroke}{rgb}{0.000000,0.000000,0.000000}%
\pgfsetstrokecolor{currentstroke}%
\pgfsetstrokeopacity{0.000000}%
\pgfsetdash{}{0pt}%
\pgfpathmoveto{\pgfqpoint{1.583447in}{0.613486in}}%
\pgfpathlineto{\pgfqpoint{1.593452in}{0.613486in}}%
\pgfpathlineto{\pgfqpoint{1.593452in}{1.612282in}}%
\pgfpathlineto{\pgfqpoint{1.583447in}{1.612282in}}%
\pgfpathlineto{\pgfqpoint{1.583447in}{0.613486in}}%
\pgfpathclose%
\pgfusepath{fill}%
\end{pgfscope}%
\begin{pgfscope}%
\pgfpathrectangle{\pgfqpoint{0.693757in}{0.613486in}}{\pgfqpoint{5.541243in}{3.963477in}}%
\pgfusepath{clip}%
\pgfsetbuttcap%
\pgfsetmiterjoin%
\definecolor{currentfill}{rgb}{0.000000,0.000000,1.000000}%
\pgfsetfillcolor{currentfill}%
\pgfsetlinewidth{0.000000pt}%
\definecolor{currentstroke}{rgb}{0.000000,0.000000,0.000000}%
\pgfsetstrokecolor{currentstroke}%
\pgfsetstrokeopacity{0.000000}%
\pgfsetdash{}{0pt}%
\pgfpathmoveto{\pgfqpoint{1.595954in}{0.613486in}}%
\pgfpathlineto{\pgfqpoint{1.605959in}{0.613486in}}%
\pgfpathlineto{\pgfqpoint{1.605959in}{1.604355in}}%
\pgfpathlineto{\pgfqpoint{1.595954in}{1.604355in}}%
\pgfpathlineto{\pgfqpoint{1.595954in}{0.613486in}}%
\pgfpathclose%
\pgfusepath{fill}%
\end{pgfscope}%
\begin{pgfscope}%
\pgfpathrectangle{\pgfqpoint{0.693757in}{0.613486in}}{\pgfqpoint{5.541243in}{3.963477in}}%
\pgfusepath{clip}%
\pgfsetbuttcap%
\pgfsetmiterjoin%
\definecolor{currentfill}{rgb}{0.000000,0.000000,1.000000}%
\pgfsetfillcolor{currentfill}%
\pgfsetlinewidth{0.000000pt}%
\definecolor{currentstroke}{rgb}{0.000000,0.000000,0.000000}%
\pgfsetstrokecolor{currentstroke}%
\pgfsetstrokeopacity{0.000000}%
\pgfsetdash{}{0pt}%
\pgfpathmoveto{\pgfqpoint{1.608460in}{0.613486in}}%
\pgfpathlineto{\pgfqpoint{1.618465in}{0.613486in}}%
\pgfpathlineto{\pgfqpoint{1.618465in}{2.611074in}}%
\pgfpathlineto{\pgfqpoint{1.608460in}{2.611074in}}%
\pgfpathlineto{\pgfqpoint{1.608460in}{0.613486in}}%
\pgfpathclose%
\pgfusepath{fill}%
\end{pgfscope}%
\begin{pgfscope}%
\pgfpathrectangle{\pgfqpoint{0.693757in}{0.613486in}}{\pgfqpoint{5.541243in}{3.963477in}}%
\pgfusepath{clip}%
\pgfsetbuttcap%
\pgfsetmiterjoin%
\definecolor{currentfill}{rgb}{0.000000,0.000000,1.000000}%
\pgfsetfillcolor{currentfill}%
\pgfsetlinewidth{0.000000pt}%
\definecolor{currentstroke}{rgb}{0.000000,0.000000,0.000000}%
\pgfsetstrokecolor{currentstroke}%
\pgfsetstrokeopacity{0.000000}%
\pgfsetdash{}{0pt}%
\pgfpathmoveto{\pgfqpoint{1.620966in}{0.613486in}}%
\pgfpathlineto{\pgfqpoint{1.630971in}{0.613486in}}%
\pgfpathlineto{\pgfqpoint{1.630971in}{2.595224in}}%
\pgfpathlineto{\pgfqpoint{1.620966in}{2.595224in}}%
\pgfpathlineto{\pgfqpoint{1.620966in}{0.613486in}}%
\pgfpathclose%
\pgfusepath{fill}%
\end{pgfscope}%
\begin{pgfscope}%
\pgfpathrectangle{\pgfqpoint{0.693757in}{0.613486in}}{\pgfqpoint{5.541243in}{3.963477in}}%
\pgfusepath{clip}%
\pgfsetbuttcap%
\pgfsetmiterjoin%
\definecolor{currentfill}{rgb}{0.000000,0.000000,1.000000}%
\pgfsetfillcolor{currentfill}%
\pgfsetlinewidth{0.000000pt}%
\definecolor{currentstroke}{rgb}{0.000000,0.000000,0.000000}%
\pgfsetstrokecolor{currentstroke}%
\pgfsetstrokeopacity{0.000000}%
\pgfsetdash{}{0pt}%
\pgfpathmoveto{\pgfqpoint{1.633472in}{0.613486in}}%
\pgfpathlineto{\pgfqpoint{1.643477in}{0.613486in}}%
\pgfpathlineto{\pgfqpoint{1.643477in}{1.612282in}}%
\pgfpathlineto{\pgfqpoint{1.633472in}{1.612282in}}%
\pgfpathlineto{\pgfqpoint{1.633472in}{0.613486in}}%
\pgfpathclose%
\pgfusepath{fill}%
\end{pgfscope}%
\begin{pgfscope}%
\pgfpathrectangle{\pgfqpoint{0.693757in}{0.613486in}}{\pgfqpoint{5.541243in}{3.963477in}}%
\pgfusepath{clip}%
\pgfsetbuttcap%
\pgfsetmiterjoin%
\definecolor{currentfill}{rgb}{0.000000,0.000000,1.000000}%
\pgfsetfillcolor{currentfill}%
\pgfsetlinewidth{0.000000pt}%
\definecolor{currentstroke}{rgb}{0.000000,0.000000,0.000000}%
\pgfsetstrokecolor{currentstroke}%
\pgfsetstrokeopacity{0.000000}%
\pgfsetdash{}{0pt}%
\pgfpathmoveto{\pgfqpoint{1.645978in}{0.613486in}}%
\pgfpathlineto{\pgfqpoint{1.655983in}{0.613486in}}%
\pgfpathlineto{\pgfqpoint{1.655983in}{1.604355in}}%
\pgfpathlineto{\pgfqpoint{1.645978in}{1.604355in}}%
\pgfpathlineto{\pgfqpoint{1.645978in}{0.613486in}}%
\pgfpathclose%
\pgfusepath{fill}%
\end{pgfscope}%
\begin{pgfscope}%
\pgfpathrectangle{\pgfqpoint{0.693757in}{0.613486in}}{\pgfqpoint{5.541243in}{3.963477in}}%
\pgfusepath{clip}%
\pgfsetbuttcap%
\pgfsetmiterjoin%
\definecolor{currentfill}{rgb}{0.000000,0.000000,1.000000}%
\pgfsetfillcolor{currentfill}%
\pgfsetlinewidth{0.000000pt}%
\definecolor{currentstroke}{rgb}{0.000000,0.000000,0.000000}%
\pgfsetstrokecolor{currentstroke}%
\pgfsetstrokeopacity{0.000000}%
\pgfsetdash{}{0pt}%
\pgfpathmoveto{\pgfqpoint{1.658485in}{0.613486in}}%
\pgfpathlineto{\pgfqpoint{1.668490in}{0.613486in}}%
\pgfpathlineto{\pgfqpoint{1.668490in}{2.611074in}}%
\pgfpathlineto{\pgfqpoint{1.658485in}{2.611074in}}%
\pgfpathlineto{\pgfqpoint{1.658485in}{0.613486in}}%
\pgfpathclose%
\pgfusepath{fill}%
\end{pgfscope}%
\begin{pgfscope}%
\pgfpathrectangle{\pgfqpoint{0.693757in}{0.613486in}}{\pgfqpoint{5.541243in}{3.963477in}}%
\pgfusepath{clip}%
\pgfsetbuttcap%
\pgfsetmiterjoin%
\definecolor{currentfill}{rgb}{0.000000,0.000000,1.000000}%
\pgfsetfillcolor{currentfill}%
\pgfsetlinewidth{0.000000pt}%
\definecolor{currentstroke}{rgb}{0.000000,0.000000,0.000000}%
\pgfsetstrokecolor{currentstroke}%
\pgfsetstrokeopacity{0.000000}%
\pgfsetdash{}{0pt}%
\pgfpathmoveto{\pgfqpoint{1.670991in}{0.613486in}}%
\pgfpathlineto{\pgfqpoint{1.680996in}{0.613486in}}%
\pgfpathlineto{\pgfqpoint{1.680996in}{2.595224in}}%
\pgfpathlineto{\pgfqpoint{1.670991in}{2.595224in}}%
\pgfpathlineto{\pgfqpoint{1.670991in}{0.613486in}}%
\pgfpathclose%
\pgfusepath{fill}%
\end{pgfscope}%
\begin{pgfscope}%
\pgfpathrectangle{\pgfqpoint{0.693757in}{0.613486in}}{\pgfqpoint{5.541243in}{3.963477in}}%
\pgfusepath{clip}%
\pgfsetbuttcap%
\pgfsetmiterjoin%
\definecolor{currentfill}{rgb}{0.000000,0.000000,1.000000}%
\pgfsetfillcolor{currentfill}%
\pgfsetlinewidth{0.000000pt}%
\definecolor{currentstroke}{rgb}{0.000000,0.000000,0.000000}%
\pgfsetstrokecolor{currentstroke}%
\pgfsetstrokeopacity{0.000000}%
\pgfsetdash{}{0pt}%
\pgfpathmoveto{\pgfqpoint{1.683497in}{0.613486in}}%
\pgfpathlineto{\pgfqpoint{1.693502in}{0.613486in}}%
\pgfpathlineto{\pgfqpoint{1.693502in}{1.612282in}}%
\pgfpathlineto{\pgfqpoint{1.683497in}{1.612282in}}%
\pgfpathlineto{\pgfqpoint{1.683497in}{0.613486in}}%
\pgfpathclose%
\pgfusepath{fill}%
\end{pgfscope}%
\begin{pgfscope}%
\pgfpathrectangle{\pgfqpoint{0.693757in}{0.613486in}}{\pgfqpoint{5.541243in}{3.963477in}}%
\pgfusepath{clip}%
\pgfsetbuttcap%
\pgfsetmiterjoin%
\definecolor{currentfill}{rgb}{0.000000,0.000000,1.000000}%
\pgfsetfillcolor{currentfill}%
\pgfsetlinewidth{0.000000pt}%
\definecolor{currentstroke}{rgb}{0.000000,0.000000,0.000000}%
\pgfsetstrokecolor{currentstroke}%
\pgfsetstrokeopacity{0.000000}%
\pgfsetdash{}{0pt}%
\pgfpathmoveto{\pgfqpoint{1.696003in}{0.613486in}}%
\pgfpathlineto{\pgfqpoint{1.706008in}{0.613486in}}%
\pgfpathlineto{\pgfqpoint{1.706008in}{1.604355in}}%
\pgfpathlineto{\pgfqpoint{1.696003in}{1.604355in}}%
\pgfpathlineto{\pgfqpoint{1.696003in}{0.613486in}}%
\pgfpathclose%
\pgfusepath{fill}%
\end{pgfscope}%
\begin{pgfscope}%
\pgfpathrectangle{\pgfqpoint{0.693757in}{0.613486in}}{\pgfqpoint{5.541243in}{3.963477in}}%
\pgfusepath{clip}%
\pgfsetbuttcap%
\pgfsetmiterjoin%
\definecolor{currentfill}{rgb}{0.000000,0.000000,1.000000}%
\pgfsetfillcolor{currentfill}%
\pgfsetlinewidth{0.000000pt}%
\definecolor{currentstroke}{rgb}{0.000000,0.000000,0.000000}%
\pgfsetstrokecolor{currentstroke}%
\pgfsetstrokeopacity{0.000000}%
\pgfsetdash{}{0pt}%
\pgfpathmoveto{\pgfqpoint{1.708509in}{0.613486in}}%
\pgfpathlineto{\pgfqpoint{1.718514in}{0.613486in}}%
\pgfpathlineto{\pgfqpoint{1.718514in}{2.611074in}}%
\pgfpathlineto{\pgfqpoint{1.708509in}{2.611074in}}%
\pgfpathlineto{\pgfqpoint{1.708509in}{0.613486in}}%
\pgfpathclose%
\pgfusepath{fill}%
\end{pgfscope}%
\begin{pgfscope}%
\pgfpathrectangle{\pgfqpoint{0.693757in}{0.613486in}}{\pgfqpoint{5.541243in}{3.963477in}}%
\pgfusepath{clip}%
\pgfsetbuttcap%
\pgfsetmiterjoin%
\definecolor{currentfill}{rgb}{0.000000,0.000000,1.000000}%
\pgfsetfillcolor{currentfill}%
\pgfsetlinewidth{0.000000pt}%
\definecolor{currentstroke}{rgb}{0.000000,0.000000,0.000000}%
\pgfsetstrokecolor{currentstroke}%
\pgfsetstrokeopacity{0.000000}%
\pgfsetdash{}{0pt}%
\pgfpathmoveto{\pgfqpoint{1.721016in}{0.613486in}}%
\pgfpathlineto{\pgfqpoint{1.731021in}{0.613486in}}%
\pgfpathlineto{\pgfqpoint{1.731021in}{2.595224in}}%
\pgfpathlineto{\pgfqpoint{1.721016in}{2.595224in}}%
\pgfpathlineto{\pgfqpoint{1.721016in}{0.613486in}}%
\pgfpathclose%
\pgfusepath{fill}%
\end{pgfscope}%
\begin{pgfscope}%
\pgfpathrectangle{\pgfqpoint{0.693757in}{0.613486in}}{\pgfqpoint{5.541243in}{3.963477in}}%
\pgfusepath{clip}%
\pgfsetbuttcap%
\pgfsetmiterjoin%
\definecolor{currentfill}{rgb}{0.000000,0.000000,1.000000}%
\pgfsetfillcolor{currentfill}%
\pgfsetlinewidth{0.000000pt}%
\definecolor{currentstroke}{rgb}{0.000000,0.000000,0.000000}%
\pgfsetstrokecolor{currentstroke}%
\pgfsetstrokeopacity{0.000000}%
\pgfsetdash{}{0pt}%
\pgfpathmoveto{\pgfqpoint{1.733522in}{0.613486in}}%
\pgfpathlineto{\pgfqpoint{1.743527in}{0.613486in}}%
\pgfpathlineto{\pgfqpoint{1.743527in}{1.612282in}}%
\pgfpathlineto{\pgfqpoint{1.733522in}{1.612282in}}%
\pgfpathlineto{\pgfqpoint{1.733522in}{0.613486in}}%
\pgfpathclose%
\pgfusepath{fill}%
\end{pgfscope}%
\begin{pgfscope}%
\pgfpathrectangle{\pgfqpoint{0.693757in}{0.613486in}}{\pgfqpoint{5.541243in}{3.963477in}}%
\pgfusepath{clip}%
\pgfsetbuttcap%
\pgfsetmiterjoin%
\definecolor{currentfill}{rgb}{0.000000,0.000000,1.000000}%
\pgfsetfillcolor{currentfill}%
\pgfsetlinewidth{0.000000pt}%
\definecolor{currentstroke}{rgb}{0.000000,0.000000,0.000000}%
\pgfsetstrokecolor{currentstroke}%
\pgfsetstrokeopacity{0.000000}%
\pgfsetdash{}{0pt}%
\pgfpathmoveto{\pgfqpoint{1.746028in}{0.613486in}}%
\pgfpathlineto{\pgfqpoint{1.756033in}{0.613486in}}%
\pgfpathlineto{\pgfqpoint{1.756033in}{1.604355in}}%
\pgfpathlineto{\pgfqpoint{1.746028in}{1.604355in}}%
\pgfpathlineto{\pgfqpoint{1.746028in}{0.613486in}}%
\pgfpathclose%
\pgfusepath{fill}%
\end{pgfscope}%
\begin{pgfscope}%
\pgfpathrectangle{\pgfqpoint{0.693757in}{0.613486in}}{\pgfqpoint{5.541243in}{3.963477in}}%
\pgfusepath{clip}%
\pgfsetbuttcap%
\pgfsetmiterjoin%
\definecolor{currentfill}{rgb}{0.000000,0.000000,1.000000}%
\pgfsetfillcolor{currentfill}%
\pgfsetlinewidth{0.000000pt}%
\definecolor{currentstroke}{rgb}{0.000000,0.000000,0.000000}%
\pgfsetstrokecolor{currentstroke}%
\pgfsetstrokeopacity{0.000000}%
\pgfsetdash{}{0pt}%
\pgfpathmoveto{\pgfqpoint{1.758534in}{0.613486in}}%
\pgfpathlineto{\pgfqpoint{1.768539in}{0.613486in}}%
\pgfpathlineto{\pgfqpoint{1.768539in}{2.611074in}}%
\pgfpathlineto{\pgfqpoint{1.758534in}{2.611074in}}%
\pgfpathlineto{\pgfqpoint{1.758534in}{0.613486in}}%
\pgfpathclose%
\pgfusepath{fill}%
\end{pgfscope}%
\begin{pgfscope}%
\pgfpathrectangle{\pgfqpoint{0.693757in}{0.613486in}}{\pgfqpoint{5.541243in}{3.963477in}}%
\pgfusepath{clip}%
\pgfsetbuttcap%
\pgfsetmiterjoin%
\definecolor{currentfill}{rgb}{0.000000,0.000000,1.000000}%
\pgfsetfillcolor{currentfill}%
\pgfsetlinewidth{0.000000pt}%
\definecolor{currentstroke}{rgb}{0.000000,0.000000,0.000000}%
\pgfsetstrokecolor{currentstroke}%
\pgfsetstrokeopacity{0.000000}%
\pgfsetdash{}{0pt}%
\pgfpathmoveto{\pgfqpoint{1.771040in}{0.613486in}}%
\pgfpathlineto{\pgfqpoint{1.781045in}{0.613486in}}%
\pgfpathlineto{\pgfqpoint{1.781045in}{2.595224in}}%
\pgfpathlineto{\pgfqpoint{1.771040in}{2.595224in}}%
\pgfpathlineto{\pgfqpoint{1.771040in}{0.613486in}}%
\pgfpathclose%
\pgfusepath{fill}%
\end{pgfscope}%
\begin{pgfscope}%
\pgfpathrectangle{\pgfqpoint{0.693757in}{0.613486in}}{\pgfqpoint{5.541243in}{3.963477in}}%
\pgfusepath{clip}%
\pgfsetbuttcap%
\pgfsetmiterjoin%
\definecolor{currentfill}{rgb}{0.000000,0.000000,1.000000}%
\pgfsetfillcolor{currentfill}%
\pgfsetlinewidth{0.000000pt}%
\definecolor{currentstroke}{rgb}{0.000000,0.000000,0.000000}%
\pgfsetstrokecolor{currentstroke}%
\pgfsetstrokeopacity{0.000000}%
\pgfsetdash{}{0pt}%
\pgfpathmoveto{\pgfqpoint{1.783547in}{0.613486in}}%
\pgfpathlineto{\pgfqpoint{1.793551in}{0.613486in}}%
\pgfpathlineto{\pgfqpoint{1.793551in}{1.612282in}}%
\pgfpathlineto{\pgfqpoint{1.783547in}{1.612282in}}%
\pgfpathlineto{\pgfqpoint{1.783547in}{0.613486in}}%
\pgfpathclose%
\pgfusepath{fill}%
\end{pgfscope}%
\begin{pgfscope}%
\pgfpathrectangle{\pgfqpoint{0.693757in}{0.613486in}}{\pgfqpoint{5.541243in}{3.963477in}}%
\pgfusepath{clip}%
\pgfsetbuttcap%
\pgfsetmiterjoin%
\definecolor{currentfill}{rgb}{0.000000,0.000000,1.000000}%
\pgfsetfillcolor{currentfill}%
\pgfsetlinewidth{0.000000pt}%
\definecolor{currentstroke}{rgb}{0.000000,0.000000,0.000000}%
\pgfsetstrokecolor{currentstroke}%
\pgfsetstrokeopacity{0.000000}%
\pgfsetdash{}{0pt}%
\pgfpathmoveto{\pgfqpoint{1.796053in}{0.613486in}}%
\pgfpathlineto{\pgfqpoint{1.806058in}{0.613486in}}%
\pgfpathlineto{\pgfqpoint{1.806058in}{1.604355in}}%
\pgfpathlineto{\pgfqpoint{1.796053in}{1.604355in}}%
\pgfpathlineto{\pgfqpoint{1.796053in}{0.613486in}}%
\pgfpathclose%
\pgfusepath{fill}%
\end{pgfscope}%
\begin{pgfscope}%
\pgfpathrectangle{\pgfqpoint{0.693757in}{0.613486in}}{\pgfqpoint{5.541243in}{3.963477in}}%
\pgfusepath{clip}%
\pgfsetbuttcap%
\pgfsetmiterjoin%
\definecolor{currentfill}{rgb}{0.000000,0.000000,1.000000}%
\pgfsetfillcolor{currentfill}%
\pgfsetlinewidth{0.000000pt}%
\definecolor{currentstroke}{rgb}{0.000000,0.000000,0.000000}%
\pgfsetstrokecolor{currentstroke}%
\pgfsetstrokeopacity{0.000000}%
\pgfsetdash{}{0pt}%
\pgfpathmoveto{\pgfqpoint{1.808559in}{0.613486in}}%
\pgfpathlineto{\pgfqpoint{1.818564in}{0.613486in}}%
\pgfpathlineto{\pgfqpoint{1.818564in}{2.611074in}}%
\pgfpathlineto{\pgfqpoint{1.808559in}{2.611074in}}%
\pgfpathlineto{\pgfqpoint{1.808559in}{0.613486in}}%
\pgfpathclose%
\pgfusepath{fill}%
\end{pgfscope}%
\begin{pgfscope}%
\pgfpathrectangle{\pgfqpoint{0.693757in}{0.613486in}}{\pgfqpoint{5.541243in}{3.963477in}}%
\pgfusepath{clip}%
\pgfsetbuttcap%
\pgfsetmiterjoin%
\definecolor{currentfill}{rgb}{0.000000,0.000000,1.000000}%
\pgfsetfillcolor{currentfill}%
\pgfsetlinewidth{0.000000pt}%
\definecolor{currentstroke}{rgb}{0.000000,0.000000,0.000000}%
\pgfsetstrokecolor{currentstroke}%
\pgfsetstrokeopacity{0.000000}%
\pgfsetdash{}{0pt}%
\pgfpathmoveto{\pgfqpoint{1.821065in}{0.613486in}}%
\pgfpathlineto{\pgfqpoint{1.831070in}{0.613486in}}%
\pgfpathlineto{\pgfqpoint{1.831070in}{2.595224in}}%
\pgfpathlineto{\pgfqpoint{1.821065in}{2.595224in}}%
\pgfpathlineto{\pgfqpoint{1.821065in}{0.613486in}}%
\pgfpathclose%
\pgfusepath{fill}%
\end{pgfscope}%
\begin{pgfscope}%
\pgfpathrectangle{\pgfqpoint{0.693757in}{0.613486in}}{\pgfqpoint{5.541243in}{3.963477in}}%
\pgfusepath{clip}%
\pgfsetbuttcap%
\pgfsetmiterjoin%
\definecolor{currentfill}{rgb}{0.000000,0.000000,1.000000}%
\pgfsetfillcolor{currentfill}%
\pgfsetlinewidth{0.000000pt}%
\definecolor{currentstroke}{rgb}{0.000000,0.000000,0.000000}%
\pgfsetstrokecolor{currentstroke}%
\pgfsetstrokeopacity{0.000000}%
\pgfsetdash{}{0pt}%
\pgfpathmoveto{\pgfqpoint{1.833571in}{0.613486in}}%
\pgfpathlineto{\pgfqpoint{1.843576in}{0.613486in}}%
\pgfpathlineto{\pgfqpoint{1.843576in}{1.612282in}}%
\pgfpathlineto{\pgfqpoint{1.833571in}{1.612282in}}%
\pgfpathlineto{\pgfqpoint{1.833571in}{0.613486in}}%
\pgfpathclose%
\pgfusepath{fill}%
\end{pgfscope}%
\begin{pgfscope}%
\pgfpathrectangle{\pgfqpoint{0.693757in}{0.613486in}}{\pgfqpoint{5.541243in}{3.963477in}}%
\pgfusepath{clip}%
\pgfsetbuttcap%
\pgfsetmiterjoin%
\definecolor{currentfill}{rgb}{0.000000,0.000000,1.000000}%
\pgfsetfillcolor{currentfill}%
\pgfsetlinewidth{0.000000pt}%
\definecolor{currentstroke}{rgb}{0.000000,0.000000,0.000000}%
\pgfsetstrokecolor{currentstroke}%
\pgfsetstrokeopacity{0.000000}%
\pgfsetdash{}{0pt}%
\pgfpathmoveto{\pgfqpoint{1.846077in}{0.613486in}}%
\pgfpathlineto{\pgfqpoint{1.856082in}{0.613486in}}%
\pgfpathlineto{\pgfqpoint{1.856082in}{1.604355in}}%
\pgfpathlineto{\pgfqpoint{1.846077in}{1.604355in}}%
\pgfpathlineto{\pgfqpoint{1.846077in}{0.613486in}}%
\pgfpathclose%
\pgfusepath{fill}%
\end{pgfscope}%
\begin{pgfscope}%
\pgfpathrectangle{\pgfqpoint{0.693757in}{0.613486in}}{\pgfqpoint{5.541243in}{3.963477in}}%
\pgfusepath{clip}%
\pgfsetbuttcap%
\pgfsetmiterjoin%
\definecolor{currentfill}{rgb}{0.000000,0.000000,1.000000}%
\pgfsetfillcolor{currentfill}%
\pgfsetlinewidth{0.000000pt}%
\definecolor{currentstroke}{rgb}{0.000000,0.000000,0.000000}%
\pgfsetstrokecolor{currentstroke}%
\pgfsetstrokeopacity{0.000000}%
\pgfsetdash{}{0pt}%
\pgfpathmoveto{\pgfqpoint{1.858584in}{0.613486in}}%
\pgfpathlineto{\pgfqpoint{1.868589in}{0.613486in}}%
\pgfpathlineto{\pgfqpoint{1.868589in}{2.611074in}}%
\pgfpathlineto{\pgfqpoint{1.858584in}{2.611074in}}%
\pgfpathlineto{\pgfqpoint{1.858584in}{0.613486in}}%
\pgfpathclose%
\pgfusepath{fill}%
\end{pgfscope}%
\begin{pgfscope}%
\pgfpathrectangle{\pgfqpoint{0.693757in}{0.613486in}}{\pgfqpoint{5.541243in}{3.963477in}}%
\pgfusepath{clip}%
\pgfsetbuttcap%
\pgfsetmiterjoin%
\definecolor{currentfill}{rgb}{0.000000,0.000000,1.000000}%
\pgfsetfillcolor{currentfill}%
\pgfsetlinewidth{0.000000pt}%
\definecolor{currentstroke}{rgb}{0.000000,0.000000,0.000000}%
\pgfsetstrokecolor{currentstroke}%
\pgfsetstrokeopacity{0.000000}%
\pgfsetdash{}{0pt}%
\pgfpathmoveto{\pgfqpoint{1.871090in}{0.613486in}}%
\pgfpathlineto{\pgfqpoint{1.881095in}{0.613486in}}%
\pgfpathlineto{\pgfqpoint{1.881095in}{2.595224in}}%
\pgfpathlineto{\pgfqpoint{1.871090in}{2.595224in}}%
\pgfpathlineto{\pgfqpoint{1.871090in}{0.613486in}}%
\pgfpathclose%
\pgfusepath{fill}%
\end{pgfscope}%
\begin{pgfscope}%
\pgfpathrectangle{\pgfqpoint{0.693757in}{0.613486in}}{\pgfqpoint{5.541243in}{3.963477in}}%
\pgfusepath{clip}%
\pgfsetbuttcap%
\pgfsetmiterjoin%
\definecolor{currentfill}{rgb}{0.000000,0.000000,1.000000}%
\pgfsetfillcolor{currentfill}%
\pgfsetlinewidth{0.000000pt}%
\definecolor{currentstroke}{rgb}{0.000000,0.000000,0.000000}%
\pgfsetstrokecolor{currentstroke}%
\pgfsetstrokeopacity{0.000000}%
\pgfsetdash{}{0pt}%
\pgfpathmoveto{\pgfqpoint{1.883596in}{0.613486in}}%
\pgfpathlineto{\pgfqpoint{1.893601in}{0.613486in}}%
\pgfpathlineto{\pgfqpoint{1.893601in}{1.612282in}}%
\pgfpathlineto{\pgfqpoint{1.883596in}{1.612282in}}%
\pgfpathlineto{\pgfqpoint{1.883596in}{0.613486in}}%
\pgfpathclose%
\pgfusepath{fill}%
\end{pgfscope}%
\begin{pgfscope}%
\pgfpathrectangle{\pgfqpoint{0.693757in}{0.613486in}}{\pgfqpoint{5.541243in}{3.963477in}}%
\pgfusepath{clip}%
\pgfsetbuttcap%
\pgfsetmiterjoin%
\definecolor{currentfill}{rgb}{0.000000,0.000000,1.000000}%
\pgfsetfillcolor{currentfill}%
\pgfsetlinewidth{0.000000pt}%
\definecolor{currentstroke}{rgb}{0.000000,0.000000,0.000000}%
\pgfsetstrokecolor{currentstroke}%
\pgfsetstrokeopacity{0.000000}%
\pgfsetdash{}{0pt}%
\pgfpathmoveto{\pgfqpoint{1.896102in}{0.613486in}}%
\pgfpathlineto{\pgfqpoint{1.906107in}{0.613486in}}%
\pgfpathlineto{\pgfqpoint{1.906107in}{1.604355in}}%
\pgfpathlineto{\pgfqpoint{1.896102in}{1.604355in}}%
\pgfpathlineto{\pgfqpoint{1.896102in}{0.613486in}}%
\pgfpathclose%
\pgfusepath{fill}%
\end{pgfscope}%
\begin{pgfscope}%
\pgfpathrectangle{\pgfqpoint{0.693757in}{0.613486in}}{\pgfqpoint{5.541243in}{3.963477in}}%
\pgfusepath{clip}%
\pgfsetbuttcap%
\pgfsetmiterjoin%
\definecolor{currentfill}{rgb}{0.000000,0.000000,1.000000}%
\pgfsetfillcolor{currentfill}%
\pgfsetlinewidth{0.000000pt}%
\definecolor{currentstroke}{rgb}{0.000000,0.000000,0.000000}%
\pgfsetstrokecolor{currentstroke}%
\pgfsetstrokeopacity{0.000000}%
\pgfsetdash{}{0pt}%
\pgfpathmoveto{\pgfqpoint{1.908608in}{0.613486in}}%
\pgfpathlineto{\pgfqpoint{1.918613in}{0.613486in}}%
\pgfpathlineto{\pgfqpoint{1.918613in}{2.611074in}}%
\pgfpathlineto{\pgfqpoint{1.908608in}{2.611074in}}%
\pgfpathlineto{\pgfqpoint{1.908608in}{0.613486in}}%
\pgfpathclose%
\pgfusepath{fill}%
\end{pgfscope}%
\begin{pgfscope}%
\pgfpathrectangle{\pgfqpoint{0.693757in}{0.613486in}}{\pgfqpoint{5.541243in}{3.963477in}}%
\pgfusepath{clip}%
\pgfsetbuttcap%
\pgfsetmiterjoin%
\definecolor{currentfill}{rgb}{0.000000,0.000000,1.000000}%
\pgfsetfillcolor{currentfill}%
\pgfsetlinewidth{0.000000pt}%
\definecolor{currentstroke}{rgb}{0.000000,0.000000,0.000000}%
\pgfsetstrokecolor{currentstroke}%
\pgfsetstrokeopacity{0.000000}%
\pgfsetdash{}{0pt}%
\pgfpathmoveto{\pgfqpoint{1.921115in}{0.613486in}}%
\pgfpathlineto{\pgfqpoint{1.931120in}{0.613486in}}%
\pgfpathlineto{\pgfqpoint{1.931120in}{2.595224in}}%
\pgfpathlineto{\pgfqpoint{1.921115in}{2.595224in}}%
\pgfpathlineto{\pgfqpoint{1.921115in}{0.613486in}}%
\pgfpathclose%
\pgfusepath{fill}%
\end{pgfscope}%
\begin{pgfscope}%
\pgfpathrectangle{\pgfqpoint{0.693757in}{0.613486in}}{\pgfqpoint{5.541243in}{3.963477in}}%
\pgfusepath{clip}%
\pgfsetbuttcap%
\pgfsetmiterjoin%
\definecolor{currentfill}{rgb}{0.000000,0.000000,1.000000}%
\pgfsetfillcolor{currentfill}%
\pgfsetlinewidth{0.000000pt}%
\definecolor{currentstroke}{rgb}{0.000000,0.000000,0.000000}%
\pgfsetstrokecolor{currentstroke}%
\pgfsetstrokeopacity{0.000000}%
\pgfsetdash{}{0pt}%
\pgfpathmoveto{\pgfqpoint{1.933621in}{0.613486in}}%
\pgfpathlineto{\pgfqpoint{1.943626in}{0.613486in}}%
\pgfpathlineto{\pgfqpoint{1.943626in}{1.612282in}}%
\pgfpathlineto{\pgfqpoint{1.933621in}{1.612282in}}%
\pgfpathlineto{\pgfqpoint{1.933621in}{0.613486in}}%
\pgfpathclose%
\pgfusepath{fill}%
\end{pgfscope}%
\begin{pgfscope}%
\pgfpathrectangle{\pgfqpoint{0.693757in}{0.613486in}}{\pgfqpoint{5.541243in}{3.963477in}}%
\pgfusepath{clip}%
\pgfsetbuttcap%
\pgfsetmiterjoin%
\definecolor{currentfill}{rgb}{0.000000,0.000000,1.000000}%
\pgfsetfillcolor{currentfill}%
\pgfsetlinewidth{0.000000pt}%
\definecolor{currentstroke}{rgb}{0.000000,0.000000,0.000000}%
\pgfsetstrokecolor{currentstroke}%
\pgfsetstrokeopacity{0.000000}%
\pgfsetdash{}{0pt}%
\pgfpathmoveto{\pgfqpoint{1.946127in}{0.613486in}}%
\pgfpathlineto{\pgfqpoint{1.956132in}{0.613486in}}%
\pgfpathlineto{\pgfqpoint{1.956132in}{1.604355in}}%
\pgfpathlineto{\pgfqpoint{1.946127in}{1.604355in}}%
\pgfpathlineto{\pgfqpoint{1.946127in}{0.613486in}}%
\pgfpathclose%
\pgfusepath{fill}%
\end{pgfscope}%
\begin{pgfscope}%
\pgfpathrectangle{\pgfqpoint{0.693757in}{0.613486in}}{\pgfqpoint{5.541243in}{3.963477in}}%
\pgfusepath{clip}%
\pgfsetbuttcap%
\pgfsetmiterjoin%
\definecolor{currentfill}{rgb}{0.000000,0.000000,1.000000}%
\pgfsetfillcolor{currentfill}%
\pgfsetlinewidth{0.000000pt}%
\definecolor{currentstroke}{rgb}{0.000000,0.000000,0.000000}%
\pgfsetstrokecolor{currentstroke}%
\pgfsetstrokeopacity{0.000000}%
\pgfsetdash{}{0pt}%
\pgfpathmoveto{\pgfqpoint{1.958633in}{0.613486in}}%
\pgfpathlineto{\pgfqpoint{1.968638in}{0.613486in}}%
\pgfpathlineto{\pgfqpoint{1.968638in}{2.611074in}}%
\pgfpathlineto{\pgfqpoint{1.958633in}{2.611074in}}%
\pgfpathlineto{\pgfqpoint{1.958633in}{0.613486in}}%
\pgfpathclose%
\pgfusepath{fill}%
\end{pgfscope}%
\begin{pgfscope}%
\pgfpathrectangle{\pgfqpoint{0.693757in}{0.613486in}}{\pgfqpoint{5.541243in}{3.963477in}}%
\pgfusepath{clip}%
\pgfsetbuttcap%
\pgfsetmiterjoin%
\definecolor{currentfill}{rgb}{0.000000,0.000000,1.000000}%
\pgfsetfillcolor{currentfill}%
\pgfsetlinewidth{0.000000pt}%
\definecolor{currentstroke}{rgb}{0.000000,0.000000,0.000000}%
\pgfsetstrokecolor{currentstroke}%
\pgfsetstrokeopacity{0.000000}%
\pgfsetdash{}{0pt}%
\pgfpathmoveto{\pgfqpoint{1.971139in}{0.613486in}}%
\pgfpathlineto{\pgfqpoint{1.981144in}{0.613486in}}%
\pgfpathlineto{\pgfqpoint{1.981144in}{2.595224in}}%
\pgfpathlineto{\pgfqpoint{1.971139in}{2.595224in}}%
\pgfpathlineto{\pgfqpoint{1.971139in}{0.613486in}}%
\pgfpathclose%
\pgfusepath{fill}%
\end{pgfscope}%
\begin{pgfscope}%
\pgfpathrectangle{\pgfqpoint{0.693757in}{0.613486in}}{\pgfqpoint{5.541243in}{3.963477in}}%
\pgfusepath{clip}%
\pgfsetbuttcap%
\pgfsetmiterjoin%
\definecolor{currentfill}{rgb}{0.000000,0.000000,1.000000}%
\pgfsetfillcolor{currentfill}%
\pgfsetlinewidth{0.000000pt}%
\definecolor{currentstroke}{rgb}{0.000000,0.000000,0.000000}%
\pgfsetstrokecolor{currentstroke}%
\pgfsetstrokeopacity{0.000000}%
\pgfsetdash{}{0pt}%
\pgfpathmoveto{\pgfqpoint{1.983646in}{0.613486in}}%
\pgfpathlineto{\pgfqpoint{1.993651in}{0.613486in}}%
\pgfpathlineto{\pgfqpoint{1.993651in}{1.612282in}}%
\pgfpathlineto{\pgfqpoint{1.983646in}{1.612282in}}%
\pgfpathlineto{\pgfqpoint{1.983646in}{0.613486in}}%
\pgfpathclose%
\pgfusepath{fill}%
\end{pgfscope}%
\begin{pgfscope}%
\pgfpathrectangle{\pgfqpoint{0.693757in}{0.613486in}}{\pgfqpoint{5.541243in}{3.963477in}}%
\pgfusepath{clip}%
\pgfsetbuttcap%
\pgfsetmiterjoin%
\definecolor{currentfill}{rgb}{0.000000,0.000000,1.000000}%
\pgfsetfillcolor{currentfill}%
\pgfsetlinewidth{0.000000pt}%
\definecolor{currentstroke}{rgb}{0.000000,0.000000,0.000000}%
\pgfsetstrokecolor{currentstroke}%
\pgfsetstrokeopacity{0.000000}%
\pgfsetdash{}{0pt}%
\pgfpathmoveto{\pgfqpoint{1.996152in}{0.613486in}}%
\pgfpathlineto{\pgfqpoint{2.006157in}{0.613486in}}%
\pgfpathlineto{\pgfqpoint{2.006157in}{1.604355in}}%
\pgfpathlineto{\pgfqpoint{1.996152in}{1.604355in}}%
\pgfpathlineto{\pgfqpoint{1.996152in}{0.613486in}}%
\pgfpathclose%
\pgfusepath{fill}%
\end{pgfscope}%
\begin{pgfscope}%
\pgfpathrectangle{\pgfqpoint{0.693757in}{0.613486in}}{\pgfqpoint{5.541243in}{3.963477in}}%
\pgfusepath{clip}%
\pgfsetbuttcap%
\pgfsetmiterjoin%
\definecolor{currentfill}{rgb}{0.000000,0.000000,1.000000}%
\pgfsetfillcolor{currentfill}%
\pgfsetlinewidth{0.000000pt}%
\definecolor{currentstroke}{rgb}{0.000000,0.000000,0.000000}%
\pgfsetstrokecolor{currentstroke}%
\pgfsetstrokeopacity{0.000000}%
\pgfsetdash{}{0pt}%
\pgfpathmoveto{\pgfqpoint{2.008658in}{0.613486in}}%
\pgfpathlineto{\pgfqpoint{2.018663in}{0.613486in}}%
\pgfpathlineto{\pgfqpoint{2.018663in}{2.611074in}}%
\pgfpathlineto{\pgfqpoint{2.008658in}{2.611074in}}%
\pgfpathlineto{\pgfqpoint{2.008658in}{0.613486in}}%
\pgfpathclose%
\pgfusepath{fill}%
\end{pgfscope}%
\begin{pgfscope}%
\pgfpathrectangle{\pgfqpoint{0.693757in}{0.613486in}}{\pgfqpoint{5.541243in}{3.963477in}}%
\pgfusepath{clip}%
\pgfsetbuttcap%
\pgfsetmiterjoin%
\definecolor{currentfill}{rgb}{0.000000,0.000000,1.000000}%
\pgfsetfillcolor{currentfill}%
\pgfsetlinewidth{0.000000pt}%
\definecolor{currentstroke}{rgb}{0.000000,0.000000,0.000000}%
\pgfsetstrokecolor{currentstroke}%
\pgfsetstrokeopacity{0.000000}%
\pgfsetdash{}{0pt}%
\pgfpathmoveto{\pgfqpoint{2.021164in}{0.613486in}}%
\pgfpathlineto{\pgfqpoint{2.031169in}{0.613486in}}%
\pgfpathlineto{\pgfqpoint{2.031169in}{2.595224in}}%
\pgfpathlineto{\pgfqpoint{2.021164in}{2.595224in}}%
\pgfpathlineto{\pgfqpoint{2.021164in}{0.613486in}}%
\pgfpathclose%
\pgfusepath{fill}%
\end{pgfscope}%
\begin{pgfscope}%
\pgfpathrectangle{\pgfqpoint{0.693757in}{0.613486in}}{\pgfqpoint{5.541243in}{3.963477in}}%
\pgfusepath{clip}%
\pgfsetbuttcap%
\pgfsetmiterjoin%
\definecolor{currentfill}{rgb}{0.000000,0.000000,1.000000}%
\pgfsetfillcolor{currentfill}%
\pgfsetlinewidth{0.000000pt}%
\definecolor{currentstroke}{rgb}{0.000000,0.000000,0.000000}%
\pgfsetstrokecolor{currentstroke}%
\pgfsetstrokeopacity{0.000000}%
\pgfsetdash{}{0pt}%
\pgfpathmoveto{\pgfqpoint{2.033670in}{0.613486in}}%
\pgfpathlineto{\pgfqpoint{2.043675in}{0.613486in}}%
\pgfpathlineto{\pgfqpoint{2.043675in}{1.612282in}}%
\pgfpathlineto{\pgfqpoint{2.033670in}{1.612282in}}%
\pgfpathlineto{\pgfqpoint{2.033670in}{0.613486in}}%
\pgfpathclose%
\pgfusepath{fill}%
\end{pgfscope}%
\begin{pgfscope}%
\pgfpathrectangle{\pgfqpoint{0.693757in}{0.613486in}}{\pgfqpoint{5.541243in}{3.963477in}}%
\pgfusepath{clip}%
\pgfsetbuttcap%
\pgfsetmiterjoin%
\definecolor{currentfill}{rgb}{0.000000,0.000000,1.000000}%
\pgfsetfillcolor{currentfill}%
\pgfsetlinewidth{0.000000pt}%
\definecolor{currentstroke}{rgb}{0.000000,0.000000,0.000000}%
\pgfsetstrokecolor{currentstroke}%
\pgfsetstrokeopacity{0.000000}%
\pgfsetdash{}{0pt}%
\pgfpathmoveto{\pgfqpoint{2.046177in}{0.613486in}}%
\pgfpathlineto{\pgfqpoint{2.056181in}{0.613486in}}%
\pgfpathlineto{\pgfqpoint{2.056181in}{1.604355in}}%
\pgfpathlineto{\pgfqpoint{2.046177in}{1.604355in}}%
\pgfpathlineto{\pgfqpoint{2.046177in}{0.613486in}}%
\pgfpathclose%
\pgfusepath{fill}%
\end{pgfscope}%
\begin{pgfscope}%
\pgfpathrectangle{\pgfqpoint{0.693757in}{0.613486in}}{\pgfqpoint{5.541243in}{3.963477in}}%
\pgfusepath{clip}%
\pgfsetbuttcap%
\pgfsetmiterjoin%
\definecolor{currentfill}{rgb}{0.000000,0.000000,1.000000}%
\pgfsetfillcolor{currentfill}%
\pgfsetlinewidth{0.000000pt}%
\definecolor{currentstroke}{rgb}{0.000000,0.000000,0.000000}%
\pgfsetstrokecolor{currentstroke}%
\pgfsetstrokeopacity{0.000000}%
\pgfsetdash{}{0pt}%
\pgfpathmoveto{\pgfqpoint{2.058683in}{0.613486in}}%
\pgfpathlineto{\pgfqpoint{2.068688in}{0.613486in}}%
\pgfpathlineto{\pgfqpoint{2.068688in}{2.611074in}}%
\pgfpathlineto{\pgfqpoint{2.058683in}{2.611074in}}%
\pgfpathlineto{\pgfqpoint{2.058683in}{0.613486in}}%
\pgfpathclose%
\pgfusepath{fill}%
\end{pgfscope}%
\begin{pgfscope}%
\pgfpathrectangle{\pgfqpoint{0.693757in}{0.613486in}}{\pgfqpoint{5.541243in}{3.963477in}}%
\pgfusepath{clip}%
\pgfsetbuttcap%
\pgfsetmiterjoin%
\definecolor{currentfill}{rgb}{0.000000,0.000000,1.000000}%
\pgfsetfillcolor{currentfill}%
\pgfsetlinewidth{0.000000pt}%
\definecolor{currentstroke}{rgb}{0.000000,0.000000,0.000000}%
\pgfsetstrokecolor{currentstroke}%
\pgfsetstrokeopacity{0.000000}%
\pgfsetdash{}{0pt}%
\pgfpathmoveto{\pgfqpoint{2.071189in}{0.613486in}}%
\pgfpathlineto{\pgfqpoint{2.081194in}{0.613486in}}%
\pgfpathlineto{\pgfqpoint{2.081194in}{2.595224in}}%
\pgfpathlineto{\pgfqpoint{2.071189in}{2.595224in}}%
\pgfpathlineto{\pgfqpoint{2.071189in}{0.613486in}}%
\pgfpathclose%
\pgfusepath{fill}%
\end{pgfscope}%
\begin{pgfscope}%
\pgfpathrectangle{\pgfqpoint{0.693757in}{0.613486in}}{\pgfqpoint{5.541243in}{3.963477in}}%
\pgfusepath{clip}%
\pgfsetbuttcap%
\pgfsetmiterjoin%
\definecolor{currentfill}{rgb}{0.000000,0.000000,1.000000}%
\pgfsetfillcolor{currentfill}%
\pgfsetlinewidth{0.000000pt}%
\definecolor{currentstroke}{rgb}{0.000000,0.000000,0.000000}%
\pgfsetstrokecolor{currentstroke}%
\pgfsetstrokeopacity{0.000000}%
\pgfsetdash{}{0pt}%
\pgfpathmoveto{\pgfqpoint{2.083695in}{0.613486in}}%
\pgfpathlineto{\pgfqpoint{2.093700in}{0.613486in}}%
\pgfpathlineto{\pgfqpoint{2.093700in}{1.612282in}}%
\pgfpathlineto{\pgfqpoint{2.083695in}{1.612282in}}%
\pgfpathlineto{\pgfqpoint{2.083695in}{0.613486in}}%
\pgfpathclose%
\pgfusepath{fill}%
\end{pgfscope}%
\begin{pgfscope}%
\pgfpathrectangle{\pgfqpoint{0.693757in}{0.613486in}}{\pgfqpoint{5.541243in}{3.963477in}}%
\pgfusepath{clip}%
\pgfsetbuttcap%
\pgfsetmiterjoin%
\definecolor{currentfill}{rgb}{0.000000,0.000000,1.000000}%
\pgfsetfillcolor{currentfill}%
\pgfsetlinewidth{0.000000pt}%
\definecolor{currentstroke}{rgb}{0.000000,0.000000,0.000000}%
\pgfsetstrokecolor{currentstroke}%
\pgfsetstrokeopacity{0.000000}%
\pgfsetdash{}{0pt}%
\pgfpathmoveto{\pgfqpoint{2.096201in}{0.613486in}}%
\pgfpathlineto{\pgfqpoint{2.106206in}{0.613486in}}%
\pgfpathlineto{\pgfqpoint{2.106206in}{1.604355in}}%
\pgfpathlineto{\pgfqpoint{2.096201in}{1.604355in}}%
\pgfpathlineto{\pgfqpoint{2.096201in}{0.613486in}}%
\pgfpathclose%
\pgfusepath{fill}%
\end{pgfscope}%
\begin{pgfscope}%
\pgfpathrectangle{\pgfqpoint{0.693757in}{0.613486in}}{\pgfqpoint{5.541243in}{3.963477in}}%
\pgfusepath{clip}%
\pgfsetbuttcap%
\pgfsetmiterjoin%
\definecolor{currentfill}{rgb}{0.000000,0.000000,1.000000}%
\pgfsetfillcolor{currentfill}%
\pgfsetlinewidth{0.000000pt}%
\definecolor{currentstroke}{rgb}{0.000000,0.000000,0.000000}%
\pgfsetstrokecolor{currentstroke}%
\pgfsetstrokeopacity{0.000000}%
\pgfsetdash{}{0pt}%
\pgfpathmoveto{\pgfqpoint{2.108707in}{0.613486in}}%
\pgfpathlineto{\pgfqpoint{2.118712in}{0.613486in}}%
\pgfpathlineto{\pgfqpoint{2.118712in}{2.611074in}}%
\pgfpathlineto{\pgfqpoint{2.108707in}{2.611074in}}%
\pgfpathlineto{\pgfqpoint{2.108707in}{0.613486in}}%
\pgfpathclose%
\pgfusepath{fill}%
\end{pgfscope}%
\begin{pgfscope}%
\pgfpathrectangle{\pgfqpoint{0.693757in}{0.613486in}}{\pgfqpoint{5.541243in}{3.963477in}}%
\pgfusepath{clip}%
\pgfsetbuttcap%
\pgfsetmiterjoin%
\definecolor{currentfill}{rgb}{0.000000,0.000000,1.000000}%
\pgfsetfillcolor{currentfill}%
\pgfsetlinewidth{0.000000pt}%
\definecolor{currentstroke}{rgb}{0.000000,0.000000,0.000000}%
\pgfsetstrokecolor{currentstroke}%
\pgfsetstrokeopacity{0.000000}%
\pgfsetdash{}{0pt}%
\pgfpathmoveto{\pgfqpoint{2.121214in}{0.613486in}}%
\pgfpathlineto{\pgfqpoint{2.131219in}{0.613486in}}%
\pgfpathlineto{\pgfqpoint{2.131219in}{2.595224in}}%
\pgfpathlineto{\pgfqpoint{2.121214in}{2.595224in}}%
\pgfpathlineto{\pgfqpoint{2.121214in}{0.613486in}}%
\pgfpathclose%
\pgfusepath{fill}%
\end{pgfscope}%
\begin{pgfscope}%
\pgfpathrectangle{\pgfqpoint{0.693757in}{0.613486in}}{\pgfqpoint{5.541243in}{3.963477in}}%
\pgfusepath{clip}%
\pgfsetbuttcap%
\pgfsetmiterjoin%
\definecolor{currentfill}{rgb}{0.000000,0.000000,1.000000}%
\pgfsetfillcolor{currentfill}%
\pgfsetlinewidth{0.000000pt}%
\definecolor{currentstroke}{rgb}{0.000000,0.000000,0.000000}%
\pgfsetstrokecolor{currentstroke}%
\pgfsetstrokeopacity{0.000000}%
\pgfsetdash{}{0pt}%
\pgfpathmoveto{\pgfqpoint{2.133720in}{0.613486in}}%
\pgfpathlineto{\pgfqpoint{2.143725in}{0.613486in}}%
\pgfpathlineto{\pgfqpoint{2.143725in}{1.612282in}}%
\pgfpathlineto{\pgfqpoint{2.133720in}{1.612282in}}%
\pgfpathlineto{\pgfqpoint{2.133720in}{0.613486in}}%
\pgfpathclose%
\pgfusepath{fill}%
\end{pgfscope}%
\begin{pgfscope}%
\pgfpathrectangle{\pgfqpoint{0.693757in}{0.613486in}}{\pgfqpoint{5.541243in}{3.963477in}}%
\pgfusepath{clip}%
\pgfsetbuttcap%
\pgfsetmiterjoin%
\definecolor{currentfill}{rgb}{0.000000,0.000000,1.000000}%
\pgfsetfillcolor{currentfill}%
\pgfsetlinewidth{0.000000pt}%
\definecolor{currentstroke}{rgb}{0.000000,0.000000,0.000000}%
\pgfsetstrokecolor{currentstroke}%
\pgfsetstrokeopacity{0.000000}%
\pgfsetdash{}{0pt}%
\pgfpathmoveto{\pgfqpoint{2.146226in}{0.613486in}}%
\pgfpathlineto{\pgfqpoint{2.156231in}{0.613486in}}%
\pgfpathlineto{\pgfqpoint{2.156231in}{1.604355in}}%
\pgfpathlineto{\pgfqpoint{2.146226in}{1.604355in}}%
\pgfpathlineto{\pgfqpoint{2.146226in}{0.613486in}}%
\pgfpathclose%
\pgfusepath{fill}%
\end{pgfscope}%
\begin{pgfscope}%
\pgfpathrectangle{\pgfqpoint{0.693757in}{0.613486in}}{\pgfqpoint{5.541243in}{3.963477in}}%
\pgfusepath{clip}%
\pgfsetbuttcap%
\pgfsetmiterjoin%
\definecolor{currentfill}{rgb}{0.000000,0.000000,1.000000}%
\pgfsetfillcolor{currentfill}%
\pgfsetlinewidth{0.000000pt}%
\definecolor{currentstroke}{rgb}{0.000000,0.000000,0.000000}%
\pgfsetstrokecolor{currentstroke}%
\pgfsetstrokeopacity{0.000000}%
\pgfsetdash{}{0pt}%
\pgfpathmoveto{\pgfqpoint{2.158732in}{0.613486in}}%
\pgfpathlineto{\pgfqpoint{2.168737in}{0.613486in}}%
\pgfpathlineto{\pgfqpoint{2.168737in}{2.611074in}}%
\pgfpathlineto{\pgfqpoint{2.158732in}{2.611074in}}%
\pgfpathlineto{\pgfqpoint{2.158732in}{0.613486in}}%
\pgfpathclose%
\pgfusepath{fill}%
\end{pgfscope}%
\begin{pgfscope}%
\pgfpathrectangle{\pgfqpoint{0.693757in}{0.613486in}}{\pgfqpoint{5.541243in}{3.963477in}}%
\pgfusepath{clip}%
\pgfsetbuttcap%
\pgfsetmiterjoin%
\definecolor{currentfill}{rgb}{0.000000,0.000000,1.000000}%
\pgfsetfillcolor{currentfill}%
\pgfsetlinewidth{0.000000pt}%
\definecolor{currentstroke}{rgb}{0.000000,0.000000,0.000000}%
\pgfsetstrokecolor{currentstroke}%
\pgfsetstrokeopacity{0.000000}%
\pgfsetdash{}{0pt}%
\pgfpathmoveto{\pgfqpoint{2.171238in}{0.613486in}}%
\pgfpathlineto{\pgfqpoint{2.181243in}{0.613486in}}%
\pgfpathlineto{\pgfqpoint{2.181243in}{2.595224in}}%
\pgfpathlineto{\pgfqpoint{2.171238in}{2.595224in}}%
\pgfpathlineto{\pgfqpoint{2.171238in}{0.613486in}}%
\pgfpathclose%
\pgfusepath{fill}%
\end{pgfscope}%
\begin{pgfscope}%
\pgfpathrectangle{\pgfqpoint{0.693757in}{0.613486in}}{\pgfqpoint{5.541243in}{3.963477in}}%
\pgfusepath{clip}%
\pgfsetbuttcap%
\pgfsetmiterjoin%
\definecolor{currentfill}{rgb}{0.000000,0.000000,1.000000}%
\pgfsetfillcolor{currentfill}%
\pgfsetlinewidth{0.000000pt}%
\definecolor{currentstroke}{rgb}{0.000000,0.000000,0.000000}%
\pgfsetstrokecolor{currentstroke}%
\pgfsetstrokeopacity{0.000000}%
\pgfsetdash{}{0pt}%
\pgfpathmoveto{\pgfqpoint{2.183745in}{0.613486in}}%
\pgfpathlineto{\pgfqpoint{2.193750in}{0.613486in}}%
\pgfpathlineto{\pgfqpoint{2.193750in}{1.612282in}}%
\pgfpathlineto{\pgfqpoint{2.183745in}{1.612282in}}%
\pgfpathlineto{\pgfqpoint{2.183745in}{0.613486in}}%
\pgfpathclose%
\pgfusepath{fill}%
\end{pgfscope}%
\begin{pgfscope}%
\pgfpathrectangle{\pgfqpoint{0.693757in}{0.613486in}}{\pgfqpoint{5.541243in}{3.963477in}}%
\pgfusepath{clip}%
\pgfsetbuttcap%
\pgfsetmiterjoin%
\definecolor{currentfill}{rgb}{0.000000,0.000000,1.000000}%
\pgfsetfillcolor{currentfill}%
\pgfsetlinewidth{0.000000pt}%
\definecolor{currentstroke}{rgb}{0.000000,0.000000,0.000000}%
\pgfsetstrokecolor{currentstroke}%
\pgfsetstrokeopacity{0.000000}%
\pgfsetdash{}{0pt}%
\pgfpathmoveto{\pgfqpoint{2.196251in}{0.613486in}}%
\pgfpathlineto{\pgfqpoint{2.206256in}{0.613486in}}%
\pgfpathlineto{\pgfqpoint{2.206256in}{1.604355in}}%
\pgfpathlineto{\pgfqpoint{2.196251in}{1.604355in}}%
\pgfpathlineto{\pgfqpoint{2.196251in}{0.613486in}}%
\pgfpathclose%
\pgfusepath{fill}%
\end{pgfscope}%
\begin{pgfscope}%
\pgfpathrectangle{\pgfqpoint{0.693757in}{0.613486in}}{\pgfqpoint{5.541243in}{3.963477in}}%
\pgfusepath{clip}%
\pgfsetbuttcap%
\pgfsetmiterjoin%
\definecolor{currentfill}{rgb}{0.000000,0.000000,1.000000}%
\pgfsetfillcolor{currentfill}%
\pgfsetlinewidth{0.000000pt}%
\definecolor{currentstroke}{rgb}{0.000000,0.000000,0.000000}%
\pgfsetstrokecolor{currentstroke}%
\pgfsetstrokeopacity{0.000000}%
\pgfsetdash{}{0pt}%
\pgfpathmoveto{\pgfqpoint{2.208757in}{0.613486in}}%
\pgfpathlineto{\pgfqpoint{2.218762in}{0.613486in}}%
\pgfpathlineto{\pgfqpoint{2.218762in}{2.611074in}}%
\pgfpathlineto{\pgfqpoint{2.208757in}{2.611074in}}%
\pgfpathlineto{\pgfqpoint{2.208757in}{0.613486in}}%
\pgfpathclose%
\pgfusepath{fill}%
\end{pgfscope}%
\begin{pgfscope}%
\pgfpathrectangle{\pgfqpoint{0.693757in}{0.613486in}}{\pgfqpoint{5.541243in}{3.963477in}}%
\pgfusepath{clip}%
\pgfsetbuttcap%
\pgfsetmiterjoin%
\definecolor{currentfill}{rgb}{0.000000,0.000000,1.000000}%
\pgfsetfillcolor{currentfill}%
\pgfsetlinewidth{0.000000pt}%
\definecolor{currentstroke}{rgb}{0.000000,0.000000,0.000000}%
\pgfsetstrokecolor{currentstroke}%
\pgfsetstrokeopacity{0.000000}%
\pgfsetdash{}{0pt}%
\pgfpathmoveto{\pgfqpoint{2.221263in}{0.613486in}}%
\pgfpathlineto{\pgfqpoint{2.231268in}{0.613486in}}%
\pgfpathlineto{\pgfqpoint{2.231268in}{2.595224in}}%
\pgfpathlineto{\pgfqpoint{2.221263in}{2.595224in}}%
\pgfpathlineto{\pgfqpoint{2.221263in}{0.613486in}}%
\pgfpathclose%
\pgfusepath{fill}%
\end{pgfscope}%
\begin{pgfscope}%
\pgfpathrectangle{\pgfqpoint{0.693757in}{0.613486in}}{\pgfqpoint{5.541243in}{3.963477in}}%
\pgfusepath{clip}%
\pgfsetbuttcap%
\pgfsetmiterjoin%
\definecolor{currentfill}{rgb}{0.000000,0.000000,1.000000}%
\pgfsetfillcolor{currentfill}%
\pgfsetlinewidth{0.000000pt}%
\definecolor{currentstroke}{rgb}{0.000000,0.000000,0.000000}%
\pgfsetstrokecolor{currentstroke}%
\pgfsetstrokeopacity{0.000000}%
\pgfsetdash{}{0pt}%
\pgfpathmoveto{\pgfqpoint{2.233769in}{0.613486in}}%
\pgfpathlineto{\pgfqpoint{2.243774in}{0.613486in}}%
\pgfpathlineto{\pgfqpoint{2.243774in}{1.612282in}}%
\pgfpathlineto{\pgfqpoint{2.233769in}{1.612282in}}%
\pgfpathlineto{\pgfqpoint{2.233769in}{0.613486in}}%
\pgfpathclose%
\pgfusepath{fill}%
\end{pgfscope}%
\begin{pgfscope}%
\pgfpathrectangle{\pgfqpoint{0.693757in}{0.613486in}}{\pgfqpoint{5.541243in}{3.963477in}}%
\pgfusepath{clip}%
\pgfsetbuttcap%
\pgfsetmiterjoin%
\definecolor{currentfill}{rgb}{0.000000,0.000000,1.000000}%
\pgfsetfillcolor{currentfill}%
\pgfsetlinewidth{0.000000pt}%
\definecolor{currentstroke}{rgb}{0.000000,0.000000,0.000000}%
\pgfsetstrokecolor{currentstroke}%
\pgfsetstrokeopacity{0.000000}%
\pgfsetdash{}{0pt}%
\pgfpathmoveto{\pgfqpoint{2.246276in}{0.613486in}}%
\pgfpathlineto{\pgfqpoint{2.256281in}{0.613486in}}%
\pgfpathlineto{\pgfqpoint{2.256281in}{1.604355in}}%
\pgfpathlineto{\pgfqpoint{2.246276in}{1.604355in}}%
\pgfpathlineto{\pgfqpoint{2.246276in}{0.613486in}}%
\pgfpathclose%
\pgfusepath{fill}%
\end{pgfscope}%
\begin{pgfscope}%
\pgfpathrectangle{\pgfqpoint{0.693757in}{0.613486in}}{\pgfqpoint{5.541243in}{3.963477in}}%
\pgfusepath{clip}%
\pgfsetbuttcap%
\pgfsetmiterjoin%
\definecolor{currentfill}{rgb}{0.000000,0.000000,1.000000}%
\pgfsetfillcolor{currentfill}%
\pgfsetlinewidth{0.000000pt}%
\definecolor{currentstroke}{rgb}{0.000000,0.000000,0.000000}%
\pgfsetstrokecolor{currentstroke}%
\pgfsetstrokeopacity{0.000000}%
\pgfsetdash{}{0pt}%
\pgfpathmoveto{\pgfqpoint{2.258782in}{0.613486in}}%
\pgfpathlineto{\pgfqpoint{2.268787in}{0.613486in}}%
\pgfpathlineto{\pgfqpoint{2.268787in}{2.611074in}}%
\pgfpathlineto{\pgfqpoint{2.258782in}{2.611074in}}%
\pgfpathlineto{\pgfqpoint{2.258782in}{0.613486in}}%
\pgfpathclose%
\pgfusepath{fill}%
\end{pgfscope}%
\begin{pgfscope}%
\pgfpathrectangle{\pgfqpoint{0.693757in}{0.613486in}}{\pgfqpoint{5.541243in}{3.963477in}}%
\pgfusepath{clip}%
\pgfsetbuttcap%
\pgfsetmiterjoin%
\definecolor{currentfill}{rgb}{0.000000,0.000000,1.000000}%
\pgfsetfillcolor{currentfill}%
\pgfsetlinewidth{0.000000pt}%
\definecolor{currentstroke}{rgb}{0.000000,0.000000,0.000000}%
\pgfsetstrokecolor{currentstroke}%
\pgfsetstrokeopacity{0.000000}%
\pgfsetdash{}{0pt}%
\pgfpathmoveto{\pgfqpoint{2.271288in}{0.613486in}}%
\pgfpathlineto{\pgfqpoint{2.281293in}{0.613486in}}%
\pgfpathlineto{\pgfqpoint{2.281293in}{2.595224in}}%
\pgfpathlineto{\pgfqpoint{2.271288in}{2.595224in}}%
\pgfpathlineto{\pgfqpoint{2.271288in}{0.613486in}}%
\pgfpathclose%
\pgfusepath{fill}%
\end{pgfscope}%
\begin{pgfscope}%
\pgfpathrectangle{\pgfqpoint{0.693757in}{0.613486in}}{\pgfqpoint{5.541243in}{3.963477in}}%
\pgfusepath{clip}%
\pgfsetbuttcap%
\pgfsetmiterjoin%
\definecolor{currentfill}{rgb}{0.000000,0.000000,1.000000}%
\pgfsetfillcolor{currentfill}%
\pgfsetlinewidth{0.000000pt}%
\definecolor{currentstroke}{rgb}{0.000000,0.000000,0.000000}%
\pgfsetstrokecolor{currentstroke}%
\pgfsetstrokeopacity{0.000000}%
\pgfsetdash{}{0pt}%
\pgfpathmoveto{\pgfqpoint{2.283794in}{0.613486in}}%
\pgfpathlineto{\pgfqpoint{2.293799in}{0.613486in}}%
\pgfpathlineto{\pgfqpoint{2.293799in}{1.612282in}}%
\pgfpathlineto{\pgfqpoint{2.283794in}{1.612282in}}%
\pgfpathlineto{\pgfqpoint{2.283794in}{0.613486in}}%
\pgfpathclose%
\pgfusepath{fill}%
\end{pgfscope}%
\begin{pgfscope}%
\pgfpathrectangle{\pgfqpoint{0.693757in}{0.613486in}}{\pgfqpoint{5.541243in}{3.963477in}}%
\pgfusepath{clip}%
\pgfsetbuttcap%
\pgfsetmiterjoin%
\definecolor{currentfill}{rgb}{0.000000,0.000000,1.000000}%
\pgfsetfillcolor{currentfill}%
\pgfsetlinewidth{0.000000pt}%
\definecolor{currentstroke}{rgb}{0.000000,0.000000,0.000000}%
\pgfsetstrokecolor{currentstroke}%
\pgfsetstrokeopacity{0.000000}%
\pgfsetdash{}{0pt}%
\pgfpathmoveto{\pgfqpoint{2.296300in}{0.613486in}}%
\pgfpathlineto{\pgfqpoint{2.306305in}{0.613486in}}%
\pgfpathlineto{\pgfqpoint{2.306305in}{1.604355in}}%
\pgfpathlineto{\pgfqpoint{2.296300in}{1.604355in}}%
\pgfpathlineto{\pgfqpoint{2.296300in}{0.613486in}}%
\pgfpathclose%
\pgfusepath{fill}%
\end{pgfscope}%
\begin{pgfscope}%
\pgfpathrectangle{\pgfqpoint{0.693757in}{0.613486in}}{\pgfqpoint{5.541243in}{3.963477in}}%
\pgfusepath{clip}%
\pgfsetbuttcap%
\pgfsetmiterjoin%
\definecolor{currentfill}{rgb}{0.000000,0.000000,1.000000}%
\pgfsetfillcolor{currentfill}%
\pgfsetlinewidth{0.000000pt}%
\definecolor{currentstroke}{rgb}{0.000000,0.000000,0.000000}%
\pgfsetstrokecolor{currentstroke}%
\pgfsetstrokeopacity{0.000000}%
\pgfsetdash{}{0pt}%
\pgfpathmoveto{\pgfqpoint{2.308807in}{0.613486in}}%
\pgfpathlineto{\pgfqpoint{2.318811in}{0.613486in}}%
\pgfpathlineto{\pgfqpoint{2.318811in}{2.611074in}}%
\pgfpathlineto{\pgfqpoint{2.308807in}{2.611074in}}%
\pgfpathlineto{\pgfqpoint{2.308807in}{0.613486in}}%
\pgfpathclose%
\pgfusepath{fill}%
\end{pgfscope}%
\begin{pgfscope}%
\pgfpathrectangle{\pgfqpoint{0.693757in}{0.613486in}}{\pgfqpoint{5.541243in}{3.963477in}}%
\pgfusepath{clip}%
\pgfsetbuttcap%
\pgfsetmiterjoin%
\definecolor{currentfill}{rgb}{0.000000,0.000000,1.000000}%
\pgfsetfillcolor{currentfill}%
\pgfsetlinewidth{0.000000pt}%
\definecolor{currentstroke}{rgb}{0.000000,0.000000,0.000000}%
\pgfsetstrokecolor{currentstroke}%
\pgfsetstrokeopacity{0.000000}%
\pgfsetdash{}{0pt}%
\pgfpathmoveto{\pgfqpoint{2.321313in}{0.613486in}}%
\pgfpathlineto{\pgfqpoint{2.331318in}{0.613486in}}%
\pgfpathlineto{\pgfqpoint{2.331318in}{2.595224in}}%
\pgfpathlineto{\pgfqpoint{2.321313in}{2.595224in}}%
\pgfpathlineto{\pgfqpoint{2.321313in}{0.613486in}}%
\pgfpathclose%
\pgfusepath{fill}%
\end{pgfscope}%
\begin{pgfscope}%
\pgfpathrectangle{\pgfqpoint{0.693757in}{0.613486in}}{\pgfqpoint{5.541243in}{3.963477in}}%
\pgfusepath{clip}%
\pgfsetbuttcap%
\pgfsetmiterjoin%
\definecolor{currentfill}{rgb}{0.000000,0.000000,1.000000}%
\pgfsetfillcolor{currentfill}%
\pgfsetlinewidth{0.000000pt}%
\definecolor{currentstroke}{rgb}{0.000000,0.000000,0.000000}%
\pgfsetstrokecolor{currentstroke}%
\pgfsetstrokeopacity{0.000000}%
\pgfsetdash{}{0pt}%
\pgfpathmoveto{\pgfqpoint{2.333819in}{0.613486in}}%
\pgfpathlineto{\pgfqpoint{2.343824in}{0.613486in}}%
\pgfpathlineto{\pgfqpoint{2.343824in}{1.612282in}}%
\pgfpathlineto{\pgfqpoint{2.333819in}{1.612282in}}%
\pgfpathlineto{\pgfqpoint{2.333819in}{0.613486in}}%
\pgfpathclose%
\pgfusepath{fill}%
\end{pgfscope}%
\begin{pgfscope}%
\pgfpathrectangle{\pgfqpoint{0.693757in}{0.613486in}}{\pgfqpoint{5.541243in}{3.963477in}}%
\pgfusepath{clip}%
\pgfsetbuttcap%
\pgfsetmiterjoin%
\definecolor{currentfill}{rgb}{0.000000,0.000000,1.000000}%
\pgfsetfillcolor{currentfill}%
\pgfsetlinewidth{0.000000pt}%
\definecolor{currentstroke}{rgb}{0.000000,0.000000,0.000000}%
\pgfsetstrokecolor{currentstroke}%
\pgfsetstrokeopacity{0.000000}%
\pgfsetdash{}{0pt}%
\pgfpathmoveto{\pgfqpoint{2.346325in}{0.613486in}}%
\pgfpathlineto{\pgfqpoint{2.356330in}{0.613486in}}%
\pgfpathlineto{\pgfqpoint{2.356330in}{1.604355in}}%
\pgfpathlineto{\pgfqpoint{2.346325in}{1.604355in}}%
\pgfpathlineto{\pgfqpoint{2.346325in}{0.613486in}}%
\pgfpathclose%
\pgfusepath{fill}%
\end{pgfscope}%
\begin{pgfscope}%
\pgfpathrectangle{\pgfqpoint{0.693757in}{0.613486in}}{\pgfqpoint{5.541243in}{3.963477in}}%
\pgfusepath{clip}%
\pgfsetbuttcap%
\pgfsetmiterjoin%
\definecolor{currentfill}{rgb}{0.000000,0.000000,1.000000}%
\pgfsetfillcolor{currentfill}%
\pgfsetlinewidth{0.000000pt}%
\definecolor{currentstroke}{rgb}{0.000000,0.000000,0.000000}%
\pgfsetstrokecolor{currentstroke}%
\pgfsetstrokeopacity{0.000000}%
\pgfsetdash{}{0pt}%
\pgfpathmoveto{\pgfqpoint{2.358831in}{0.613486in}}%
\pgfpathlineto{\pgfqpoint{2.368836in}{0.613486in}}%
\pgfpathlineto{\pgfqpoint{2.368836in}{2.611074in}}%
\pgfpathlineto{\pgfqpoint{2.358831in}{2.611074in}}%
\pgfpathlineto{\pgfqpoint{2.358831in}{0.613486in}}%
\pgfpathclose%
\pgfusepath{fill}%
\end{pgfscope}%
\begin{pgfscope}%
\pgfpathrectangle{\pgfqpoint{0.693757in}{0.613486in}}{\pgfqpoint{5.541243in}{3.963477in}}%
\pgfusepath{clip}%
\pgfsetbuttcap%
\pgfsetmiterjoin%
\definecolor{currentfill}{rgb}{0.000000,0.000000,1.000000}%
\pgfsetfillcolor{currentfill}%
\pgfsetlinewidth{0.000000pt}%
\definecolor{currentstroke}{rgb}{0.000000,0.000000,0.000000}%
\pgfsetstrokecolor{currentstroke}%
\pgfsetstrokeopacity{0.000000}%
\pgfsetdash{}{0pt}%
\pgfpathmoveto{\pgfqpoint{2.371337in}{0.613486in}}%
\pgfpathlineto{\pgfqpoint{2.381342in}{0.613486in}}%
\pgfpathlineto{\pgfqpoint{2.381342in}{2.595224in}}%
\pgfpathlineto{\pgfqpoint{2.371337in}{2.595224in}}%
\pgfpathlineto{\pgfqpoint{2.371337in}{0.613486in}}%
\pgfpathclose%
\pgfusepath{fill}%
\end{pgfscope}%
\begin{pgfscope}%
\pgfpathrectangle{\pgfqpoint{0.693757in}{0.613486in}}{\pgfqpoint{5.541243in}{3.963477in}}%
\pgfusepath{clip}%
\pgfsetbuttcap%
\pgfsetmiterjoin%
\definecolor{currentfill}{rgb}{0.000000,0.000000,1.000000}%
\pgfsetfillcolor{currentfill}%
\pgfsetlinewidth{0.000000pt}%
\definecolor{currentstroke}{rgb}{0.000000,0.000000,0.000000}%
\pgfsetstrokecolor{currentstroke}%
\pgfsetstrokeopacity{0.000000}%
\pgfsetdash{}{0pt}%
\pgfpathmoveto{\pgfqpoint{2.383844in}{0.613486in}}%
\pgfpathlineto{\pgfqpoint{2.393849in}{0.613486in}}%
\pgfpathlineto{\pgfqpoint{2.393849in}{1.612282in}}%
\pgfpathlineto{\pgfqpoint{2.383844in}{1.612282in}}%
\pgfpathlineto{\pgfqpoint{2.383844in}{0.613486in}}%
\pgfpathclose%
\pgfusepath{fill}%
\end{pgfscope}%
\begin{pgfscope}%
\pgfpathrectangle{\pgfqpoint{0.693757in}{0.613486in}}{\pgfqpoint{5.541243in}{3.963477in}}%
\pgfusepath{clip}%
\pgfsetbuttcap%
\pgfsetmiterjoin%
\definecolor{currentfill}{rgb}{0.000000,0.000000,1.000000}%
\pgfsetfillcolor{currentfill}%
\pgfsetlinewidth{0.000000pt}%
\definecolor{currentstroke}{rgb}{0.000000,0.000000,0.000000}%
\pgfsetstrokecolor{currentstroke}%
\pgfsetstrokeopacity{0.000000}%
\pgfsetdash{}{0pt}%
\pgfpathmoveto{\pgfqpoint{2.396350in}{0.613486in}}%
\pgfpathlineto{\pgfqpoint{2.406355in}{0.613486in}}%
\pgfpathlineto{\pgfqpoint{2.406355in}{1.604355in}}%
\pgfpathlineto{\pgfqpoint{2.396350in}{1.604355in}}%
\pgfpathlineto{\pgfqpoint{2.396350in}{0.613486in}}%
\pgfpathclose%
\pgfusepath{fill}%
\end{pgfscope}%
\begin{pgfscope}%
\pgfpathrectangle{\pgfqpoint{0.693757in}{0.613486in}}{\pgfqpoint{5.541243in}{3.963477in}}%
\pgfusepath{clip}%
\pgfsetbuttcap%
\pgfsetmiterjoin%
\definecolor{currentfill}{rgb}{0.000000,0.000000,1.000000}%
\pgfsetfillcolor{currentfill}%
\pgfsetlinewidth{0.000000pt}%
\definecolor{currentstroke}{rgb}{0.000000,0.000000,0.000000}%
\pgfsetstrokecolor{currentstroke}%
\pgfsetstrokeopacity{0.000000}%
\pgfsetdash{}{0pt}%
\pgfpathmoveto{\pgfqpoint{2.408856in}{0.613486in}}%
\pgfpathlineto{\pgfqpoint{2.418861in}{0.613486in}}%
\pgfpathlineto{\pgfqpoint{2.418861in}{2.611074in}}%
\pgfpathlineto{\pgfqpoint{2.408856in}{2.611074in}}%
\pgfpathlineto{\pgfqpoint{2.408856in}{0.613486in}}%
\pgfpathclose%
\pgfusepath{fill}%
\end{pgfscope}%
\begin{pgfscope}%
\pgfpathrectangle{\pgfqpoint{0.693757in}{0.613486in}}{\pgfqpoint{5.541243in}{3.963477in}}%
\pgfusepath{clip}%
\pgfsetbuttcap%
\pgfsetmiterjoin%
\definecolor{currentfill}{rgb}{0.000000,0.000000,1.000000}%
\pgfsetfillcolor{currentfill}%
\pgfsetlinewidth{0.000000pt}%
\definecolor{currentstroke}{rgb}{0.000000,0.000000,0.000000}%
\pgfsetstrokecolor{currentstroke}%
\pgfsetstrokeopacity{0.000000}%
\pgfsetdash{}{0pt}%
\pgfpathmoveto{\pgfqpoint{2.421362in}{0.613486in}}%
\pgfpathlineto{\pgfqpoint{2.431367in}{0.613486in}}%
\pgfpathlineto{\pgfqpoint{2.431367in}{2.595224in}}%
\pgfpathlineto{\pgfqpoint{2.421362in}{2.595224in}}%
\pgfpathlineto{\pgfqpoint{2.421362in}{0.613486in}}%
\pgfpathclose%
\pgfusepath{fill}%
\end{pgfscope}%
\begin{pgfscope}%
\pgfpathrectangle{\pgfqpoint{0.693757in}{0.613486in}}{\pgfqpoint{5.541243in}{3.963477in}}%
\pgfusepath{clip}%
\pgfsetbuttcap%
\pgfsetmiterjoin%
\definecolor{currentfill}{rgb}{0.000000,0.000000,1.000000}%
\pgfsetfillcolor{currentfill}%
\pgfsetlinewidth{0.000000pt}%
\definecolor{currentstroke}{rgb}{0.000000,0.000000,0.000000}%
\pgfsetstrokecolor{currentstroke}%
\pgfsetstrokeopacity{0.000000}%
\pgfsetdash{}{0pt}%
\pgfpathmoveto{\pgfqpoint{2.433868in}{0.613486in}}%
\pgfpathlineto{\pgfqpoint{2.443873in}{0.613486in}}%
\pgfpathlineto{\pgfqpoint{2.443873in}{1.612282in}}%
\pgfpathlineto{\pgfqpoint{2.433868in}{1.612282in}}%
\pgfpathlineto{\pgfqpoint{2.433868in}{0.613486in}}%
\pgfpathclose%
\pgfusepath{fill}%
\end{pgfscope}%
\begin{pgfscope}%
\pgfpathrectangle{\pgfqpoint{0.693757in}{0.613486in}}{\pgfqpoint{5.541243in}{3.963477in}}%
\pgfusepath{clip}%
\pgfsetbuttcap%
\pgfsetmiterjoin%
\definecolor{currentfill}{rgb}{0.000000,0.000000,1.000000}%
\pgfsetfillcolor{currentfill}%
\pgfsetlinewidth{0.000000pt}%
\definecolor{currentstroke}{rgb}{0.000000,0.000000,0.000000}%
\pgfsetstrokecolor{currentstroke}%
\pgfsetstrokeopacity{0.000000}%
\pgfsetdash{}{0pt}%
\pgfpathmoveto{\pgfqpoint{2.446375in}{0.613486in}}%
\pgfpathlineto{\pgfqpoint{2.456380in}{0.613486in}}%
\pgfpathlineto{\pgfqpoint{2.456380in}{1.604355in}}%
\pgfpathlineto{\pgfqpoint{2.446375in}{1.604355in}}%
\pgfpathlineto{\pgfqpoint{2.446375in}{0.613486in}}%
\pgfpathclose%
\pgfusepath{fill}%
\end{pgfscope}%
\begin{pgfscope}%
\pgfpathrectangle{\pgfqpoint{0.693757in}{0.613486in}}{\pgfqpoint{5.541243in}{3.963477in}}%
\pgfusepath{clip}%
\pgfsetbuttcap%
\pgfsetmiterjoin%
\definecolor{currentfill}{rgb}{0.000000,0.000000,1.000000}%
\pgfsetfillcolor{currentfill}%
\pgfsetlinewidth{0.000000pt}%
\definecolor{currentstroke}{rgb}{0.000000,0.000000,0.000000}%
\pgfsetstrokecolor{currentstroke}%
\pgfsetstrokeopacity{0.000000}%
\pgfsetdash{}{0pt}%
\pgfpathmoveto{\pgfqpoint{2.458881in}{0.613486in}}%
\pgfpathlineto{\pgfqpoint{2.468886in}{0.613486in}}%
\pgfpathlineto{\pgfqpoint{2.468886in}{2.611074in}}%
\pgfpathlineto{\pgfqpoint{2.458881in}{2.611074in}}%
\pgfpathlineto{\pgfqpoint{2.458881in}{0.613486in}}%
\pgfpathclose%
\pgfusepath{fill}%
\end{pgfscope}%
\begin{pgfscope}%
\pgfpathrectangle{\pgfqpoint{0.693757in}{0.613486in}}{\pgfqpoint{5.541243in}{3.963477in}}%
\pgfusepath{clip}%
\pgfsetbuttcap%
\pgfsetmiterjoin%
\definecolor{currentfill}{rgb}{0.000000,0.000000,1.000000}%
\pgfsetfillcolor{currentfill}%
\pgfsetlinewidth{0.000000pt}%
\definecolor{currentstroke}{rgb}{0.000000,0.000000,0.000000}%
\pgfsetstrokecolor{currentstroke}%
\pgfsetstrokeopacity{0.000000}%
\pgfsetdash{}{0pt}%
\pgfpathmoveto{\pgfqpoint{2.471387in}{0.613486in}}%
\pgfpathlineto{\pgfqpoint{2.481392in}{0.613486in}}%
\pgfpathlineto{\pgfqpoint{2.481392in}{2.595224in}}%
\pgfpathlineto{\pgfqpoint{2.471387in}{2.595224in}}%
\pgfpathlineto{\pgfqpoint{2.471387in}{0.613486in}}%
\pgfpathclose%
\pgfusepath{fill}%
\end{pgfscope}%
\begin{pgfscope}%
\pgfpathrectangle{\pgfqpoint{0.693757in}{0.613486in}}{\pgfqpoint{5.541243in}{3.963477in}}%
\pgfusepath{clip}%
\pgfsetbuttcap%
\pgfsetmiterjoin%
\definecolor{currentfill}{rgb}{0.000000,0.000000,1.000000}%
\pgfsetfillcolor{currentfill}%
\pgfsetlinewidth{0.000000pt}%
\definecolor{currentstroke}{rgb}{0.000000,0.000000,0.000000}%
\pgfsetstrokecolor{currentstroke}%
\pgfsetstrokeopacity{0.000000}%
\pgfsetdash{}{0pt}%
\pgfpathmoveto{\pgfqpoint{2.483893in}{0.613486in}}%
\pgfpathlineto{\pgfqpoint{2.493898in}{0.613486in}}%
\pgfpathlineto{\pgfqpoint{2.493898in}{1.612282in}}%
\pgfpathlineto{\pgfqpoint{2.483893in}{1.612282in}}%
\pgfpathlineto{\pgfqpoint{2.483893in}{0.613486in}}%
\pgfpathclose%
\pgfusepath{fill}%
\end{pgfscope}%
\begin{pgfscope}%
\pgfpathrectangle{\pgfqpoint{0.693757in}{0.613486in}}{\pgfqpoint{5.541243in}{3.963477in}}%
\pgfusepath{clip}%
\pgfsetbuttcap%
\pgfsetmiterjoin%
\definecolor{currentfill}{rgb}{0.000000,0.000000,1.000000}%
\pgfsetfillcolor{currentfill}%
\pgfsetlinewidth{0.000000pt}%
\definecolor{currentstroke}{rgb}{0.000000,0.000000,0.000000}%
\pgfsetstrokecolor{currentstroke}%
\pgfsetstrokeopacity{0.000000}%
\pgfsetdash{}{0pt}%
\pgfpathmoveto{\pgfqpoint{2.496399in}{0.613486in}}%
\pgfpathlineto{\pgfqpoint{2.506404in}{0.613486in}}%
\pgfpathlineto{\pgfqpoint{2.506404in}{1.604355in}}%
\pgfpathlineto{\pgfqpoint{2.496399in}{1.604355in}}%
\pgfpathlineto{\pgfqpoint{2.496399in}{0.613486in}}%
\pgfpathclose%
\pgfusepath{fill}%
\end{pgfscope}%
\begin{pgfscope}%
\pgfpathrectangle{\pgfqpoint{0.693757in}{0.613486in}}{\pgfqpoint{5.541243in}{3.963477in}}%
\pgfusepath{clip}%
\pgfsetbuttcap%
\pgfsetmiterjoin%
\definecolor{currentfill}{rgb}{0.000000,0.000000,1.000000}%
\pgfsetfillcolor{currentfill}%
\pgfsetlinewidth{0.000000pt}%
\definecolor{currentstroke}{rgb}{0.000000,0.000000,0.000000}%
\pgfsetstrokecolor{currentstroke}%
\pgfsetstrokeopacity{0.000000}%
\pgfsetdash{}{0pt}%
\pgfpathmoveto{\pgfqpoint{2.508906in}{0.613486in}}%
\pgfpathlineto{\pgfqpoint{2.518911in}{0.613486in}}%
\pgfpathlineto{\pgfqpoint{2.518911in}{2.611074in}}%
\pgfpathlineto{\pgfqpoint{2.508906in}{2.611074in}}%
\pgfpathlineto{\pgfqpoint{2.508906in}{0.613486in}}%
\pgfpathclose%
\pgfusepath{fill}%
\end{pgfscope}%
\begin{pgfscope}%
\pgfpathrectangle{\pgfqpoint{0.693757in}{0.613486in}}{\pgfqpoint{5.541243in}{3.963477in}}%
\pgfusepath{clip}%
\pgfsetbuttcap%
\pgfsetmiterjoin%
\definecolor{currentfill}{rgb}{0.000000,0.000000,1.000000}%
\pgfsetfillcolor{currentfill}%
\pgfsetlinewidth{0.000000pt}%
\definecolor{currentstroke}{rgb}{0.000000,0.000000,0.000000}%
\pgfsetstrokecolor{currentstroke}%
\pgfsetstrokeopacity{0.000000}%
\pgfsetdash{}{0pt}%
\pgfpathmoveto{\pgfqpoint{2.521412in}{0.613486in}}%
\pgfpathlineto{\pgfqpoint{2.531417in}{0.613486in}}%
\pgfpathlineto{\pgfqpoint{2.531417in}{2.595224in}}%
\pgfpathlineto{\pgfqpoint{2.521412in}{2.595224in}}%
\pgfpathlineto{\pgfqpoint{2.521412in}{0.613486in}}%
\pgfpathclose%
\pgfusepath{fill}%
\end{pgfscope}%
\begin{pgfscope}%
\pgfpathrectangle{\pgfqpoint{0.693757in}{0.613486in}}{\pgfqpoint{5.541243in}{3.963477in}}%
\pgfusepath{clip}%
\pgfsetbuttcap%
\pgfsetmiterjoin%
\definecolor{currentfill}{rgb}{0.000000,0.000000,1.000000}%
\pgfsetfillcolor{currentfill}%
\pgfsetlinewidth{0.000000pt}%
\definecolor{currentstroke}{rgb}{0.000000,0.000000,0.000000}%
\pgfsetstrokecolor{currentstroke}%
\pgfsetstrokeopacity{0.000000}%
\pgfsetdash{}{0pt}%
\pgfpathmoveto{\pgfqpoint{2.533918in}{0.613486in}}%
\pgfpathlineto{\pgfqpoint{2.543923in}{0.613486in}}%
\pgfpathlineto{\pgfqpoint{2.543923in}{1.612282in}}%
\pgfpathlineto{\pgfqpoint{2.533918in}{1.612282in}}%
\pgfpathlineto{\pgfqpoint{2.533918in}{0.613486in}}%
\pgfpathclose%
\pgfusepath{fill}%
\end{pgfscope}%
\begin{pgfscope}%
\pgfpathrectangle{\pgfqpoint{0.693757in}{0.613486in}}{\pgfqpoint{5.541243in}{3.963477in}}%
\pgfusepath{clip}%
\pgfsetbuttcap%
\pgfsetmiterjoin%
\definecolor{currentfill}{rgb}{0.000000,0.000000,1.000000}%
\pgfsetfillcolor{currentfill}%
\pgfsetlinewidth{0.000000pt}%
\definecolor{currentstroke}{rgb}{0.000000,0.000000,0.000000}%
\pgfsetstrokecolor{currentstroke}%
\pgfsetstrokeopacity{0.000000}%
\pgfsetdash{}{0pt}%
\pgfpathmoveto{\pgfqpoint{2.546424in}{0.613486in}}%
\pgfpathlineto{\pgfqpoint{2.556429in}{0.613486in}}%
\pgfpathlineto{\pgfqpoint{2.556429in}{1.604355in}}%
\pgfpathlineto{\pgfqpoint{2.546424in}{1.604355in}}%
\pgfpathlineto{\pgfqpoint{2.546424in}{0.613486in}}%
\pgfpathclose%
\pgfusepath{fill}%
\end{pgfscope}%
\begin{pgfscope}%
\pgfpathrectangle{\pgfqpoint{0.693757in}{0.613486in}}{\pgfqpoint{5.541243in}{3.963477in}}%
\pgfusepath{clip}%
\pgfsetbuttcap%
\pgfsetmiterjoin%
\definecolor{currentfill}{rgb}{0.000000,0.000000,1.000000}%
\pgfsetfillcolor{currentfill}%
\pgfsetlinewidth{0.000000pt}%
\definecolor{currentstroke}{rgb}{0.000000,0.000000,0.000000}%
\pgfsetstrokecolor{currentstroke}%
\pgfsetstrokeopacity{0.000000}%
\pgfsetdash{}{0pt}%
\pgfpathmoveto{\pgfqpoint{2.558930in}{0.613486in}}%
\pgfpathlineto{\pgfqpoint{2.568935in}{0.613486in}}%
\pgfpathlineto{\pgfqpoint{2.568935in}{2.611074in}}%
\pgfpathlineto{\pgfqpoint{2.558930in}{2.611074in}}%
\pgfpathlineto{\pgfqpoint{2.558930in}{0.613486in}}%
\pgfpathclose%
\pgfusepath{fill}%
\end{pgfscope}%
\begin{pgfscope}%
\pgfpathrectangle{\pgfqpoint{0.693757in}{0.613486in}}{\pgfqpoint{5.541243in}{3.963477in}}%
\pgfusepath{clip}%
\pgfsetbuttcap%
\pgfsetmiterjoin%
\definecolor{currentfill}{rgb}{0.000000,0.000000,1.000000}%
\pgfsetfillcolor{currentfill}%
\pgfsetlinewidth{0.000000pt}%
\definecolor{currentstroke}{rgb}{0.000000,0.000000,0.000000}%
\pgfsetstrokecolor{currentstroke}%
\pgfsetstrokeopacity{0.000000}%
\pgfsetdash{}{0pt}%
\pgfpathmoveto{\pgfqpoint{2.571437in}{0.613486in}}%
\pgfpathlineto{\pgfqpoint{2.581441in}{0.613486in}}%
\pgfpathlineto{\pgfqpoint{2.581441in}{2.595224in}}%
\pgfpathlineto{\pgfqpoint{2.571437in}{2.595224in}}%
\pgfpathlineto{\pgfqpoint{2.571437in}{0.613486in}}%
\pgfpathclose%
\pgfusepath{fill}%
\end{pgfscope}%
\begin{pgfscope}%
\pgfpathrectangle{\pgfqpoint{0.693757in}{0.613486in}}{\pgfqpoint{5.541243in}{3.963477in}}%
\pgfusepath{clip}%
\pgfsetbuttcap%
\pgfsetmiterjoin%
\definecolor{currentfill}{rgb}{0.000000,0.000000,1.000000}%
\pgfsetfillcolor{currentfill}%
\pgfsetlinewidth{0.000000pt}%
\definecolor{currentstroke}{rgb}{0.000000,0.000000,0.000000}%
\pgfsetstrokecolor{currentstroke}%
\pgfsetstrokeopacity{0.000000}%
\pgfsetdash{}{0pt}%
\pgfpathmoveto{\pgfqpoint{2.583943in}{0.613486in}}%
\pgfpathlineto{\pgfqpoint{2.593948in}{0.613486in}}%
\pgfpathlineto{\pgfqpoint{2.593948in}{1.612282in}}%
\pgfpathlineto{\pgfqpoint{2.583943in}{1.612282in}}%
\pgfpathlineto{\pgfqpoint{2.583943in}{0.613486in}}%
\pgfpathclose%
\pgfusepath{fill}%
\end{pgfscope}%
\begin{pgfscope}%
\pgfpathrectangle{\pgfqpoint{0.693757in}{0.613486in}}{\pgfqpoint{5.541243in}{3.963477in}}%
\pgfusepath{clip}%
\pgfsetbuttcap%
\pgfsetmiterjoin%
\definecolor{currentfill}{rgb}{0.000000,0.000000,1.000000}%
\pgfsetfillcolor{currentfill}%
\pgfsetlinewidth{0.000000pt}%
\definecolor{currentstroke}{rgb}{0.000000,0.000000,0.000000}%
\pgfsetstrokecolor{currentstroke}%
\pgfsetstrokeopacity{0.000000}%
\pgfsetdash{}{0pt}%
\pgfpathmoveto{\pgfqpoint{2.596449in}{0.613486in}}%
\pgfpathlineto{\pgfqpoint{2.606454in}{0.613486in}}%
\pgfpathlineto{\pgfqpoint{2.606454in}{1.604355in}}%
\pgfpathlineto{\pgfqpoint{2.596449in}{1.604355in}}%
\pgfpathlineto{\pgfqpoint{2.596449in}{0.613486in}}%
\pgfpathclose%
\pgfusepath{fill}%
\end{pgfscope}%
\begin{pgfscope}%
\pgfpathrectangle{\pgfqpoint{0.693757in}{0.613486in}}{\pgfqpoint{5.541243in}{3.963477in}}%
\pgfusepath{clip}%
\pgfsetbuttcap%
\pgfsetmiterjoin%
\definecolor{currentfill}{rgb}{0.000000,0.000000,1.000000}%
\pgfsetfillcolor{currentfill}%
\pgfsetlinewidth{0.000000pt}%
\definecolor{currentstroke}{rgb}{0.000000,0.000000,0.000000}%
\pgfsetstrokecolor{currentstroke}%
\pgfsetstrokeopacity{0.000000}%
\pgfsetdash{}{0pt}%
\pgfpathmoveto{\pgfqpoint{2.608955in}{0.613486in}}%
\pgfpathlineto{\pgfqpoint{2.618960in}{0.613486in}}%
\pgfpathlineto{\pgfqpoint{2.618960in}{2.611074in}}%
\pgfpathlineto{\pgfqpoint{2.608955in}{2.611074in}}%
\pgfpathlineto{\pgfqpoint{2.608955in}{0.613486in}}%
\pgfpathclose%
\pgfusepath{fill}%
\end{pgfscope}%
\begin{pgfscope}%
\pgfpathrectangle{\pgfqpoint{0.693757in}{0.613486in}}{\pgfqpoint{5.541243in}{3.963477in}}%
\pgfusepath{clip}%
\pgfsetbuttcap%
\pgfsetmiterjoin%
\definecolor{currentfill}{rgb}{0.000000,0.000000,1.000000}%
\pgfsetfillcolor{currentfill}%
\pgfsetlinewidth{0.000000pt}%
\definecolor{currentstroke}{rgb}{0.000000,0.000000,0.000000}%
\pgfsetstrokecolor{currentstroke}%
\pgfsetstrokeopacity{0.000000}%
\pgfsetdash{}{0pt}%
\pgfpathmoveto{\pgfqpoint{2.621461in}{0.613486in}}%
\pgfpathlineto{\pgfqpoint{2.631466in}{0.613486in}}%
\pgfpathlineto{\pgfqpoint{2.631466in}{2.595224in}}%
\pgfpathlineto{\pgfqpoint{2.621461in}{2.595224in}}%
\pgfpathlineto{\pgfqpoint{2.621461in}{0.613486in}}%
\pgfpathclose%
\pgfusepath{fill}%
\end{pgfscope}%
\begin{pgfscope}%
\pgfpathrectangle{\pgfqpoint{0.693757in}{0.613486in}}{\pgfqpoint{5.541243in}{3.963477in}}%
\pgfusepath{clip}%
\pgfsetbuttcap%
\pgfsetmiterjoin%
\definecolor{currentfill}{rgb}{0.000000,0.000000,1.000000}%
\pgfsetfillcolor{currentfill}%
\pgfsetlinewidth{0.000000pt}%
\definecolor{currentstroke}{rgb}{0.000000,0.000000,0.000000}%
\pgfsetstrokecolor{currentstroke}%
\pgfsetstrokeopacity{0.000000}%
\pgfsetdash{}{0pt}%
\pgfpathmoveto{\pgfqpoint{2.633967in}{0.613486in}}%
\pgfpathlineto{\pgfqpoint{2.643972in}{0.613486in}}%
\pgfpathlineto{\pgfqpoint{2.643972in}{1.612282in}}%
\pgfpathlineto{\pgfqpoint{2.633967in}{1.612282in}}%
\pgfpathlineto{\pgfqpoint{2.633967in}{0.613486in}}%
\pgfpathclose%
\pgfusepath{fill}%
\end{pgfscope}%
\begin{pgfscope}%
\pgfpathrectangle{\pgfqpoint{0.693757in}{0.613486in}}{\pgfqpoint{5.541243in}{3.963477in}}%
\pgfusepath{clip}%
\pgfsetbuttcap%
\pgfsetmiterjoin%
\definecolor{currentfill}{rgb}{0.000000,0.000000,1.000000}%
\pgfsetfillcolor{currentfill}%
\pgfsetlinewidth{0.000000pt}%
\definecolor{currentstroke}{rgb}{0.000000,0.000000,0.000000}%
\pgfsetstrokecolor{currentstroke}%
\pgfsetstrokeopacity{0.000000}%
\pgfsetdash{}{0pt}%
\pgfpathmoveto{\pgfqpoint{2.646474in}{0.613486in}}%
\pgfpathlineto{\pgfqpoint{2.656479in}{0.613486in}}%
\pgfpathlineto{\pgfqpoint{2.656479in}{1.604355in}}%
\pgfpathlineto{\pgfqpoint{2.646474in}{1.604355in}}%
\pgfpathlineto{\pgfqpoint{2.646474in}{0.613486in}}%
\pgfpathclose%
\pgfusepath{fill}%
\end{pgfscope}%
\begin{pgfscope}%
\pgfpathrectangle{\pgfqpoint{0.693757in}{0.613486in}}{\pgfqpoint{5.541243in}{3.963477in}}%
\pgfusepath{clip}%
\pgfsetbuttcap%
\pgfsetmiterjoin%
\definecolor{currentfill}{rgb}{0.000000,0.000000,1.000000}%
\pgfsetfillcolor{currentfill}%
\pgfsetlinewidth{0.000000pt}%
\definecolor{currentstroke}{rgb}{0.000000,0.000000,0.000000}%
\pgfsetstrokecolor{currentstroke}%
\pgfsetstrokeopacity{0.000000}%
\pgfsetdash{}{0pt}%
\pgfpathmoveto{\pgfqpoint{2.658980in}{0.613486in}}%
\pgfpathlineto{\pgfqpoint{2.668985in}{0.613486in}}%
\pgfpathlineto{\pgfqpoint{2.668985in}{2.611074in}}%
\pgfpathlineto{\pgfqpoint{2.658980in}{2.611074in}}%
\pgfpathlineto{\pgfqpoint{2.658980in}{0.613486in}}%
\pgfpathclose%
\pgfusepath{fill}%
\end{pgfscope}%
\begin{pgfscope}%
\pgfpathrectangle{\pgfqpoint{0.693757in}{0.613486in}}{\pgfqpoint{5.541243in}{3.963477in}}%
\pgfusepath{clip}%
\pgfsetbuttcap%
\pgfsetmiterjoin%
\definecolor{currentfill}{rgb}{0.000000,0.000000,1.000000}%
\pgfsetfillcolor{currentfill}%
\pgfsetlinewidth{0.000000pt}%
\definecolor{currentstroke}{rgb}{0.000000,0.000000,0.000000}%
\pgfsetstrokecolor{currentstroke}%
\pgfsetstrokeopacity{0.000000}%
\pgfsetdash{}{0pt}%
\pgfpathmoveto{\pgfqpoint{2.671486in}{0.613486in}}%
\pgfpathlineto{\pgfqpoint{2.681491in}{0.613486in}}%
\pgfpathlineto{\pgfqpoint{2.681491in}{2.595224in}}%
\pgfpathlineto{\pgfqpoint{2.671486in}{2.595224in}}%
\pgfpathlineto{\pgfqpoint{2.671486in}{0.613486in}}%
\pgfpathclose%
\pgfusepath{fill}%
\end{pgfscope}%
\begin{pgfscope}%
\pgfpathrectangle{\pgfqpoint{0.693757in}{0.613486in}}{\pgfqpoint{5.541243in}{3.963477in}}%
\pgfusepath{clip}%
\pgfsetbuttcap%
\pgfsetmiterjoin%
\definecolor{currentfill}{rgb}{0.000000,0.000000,1.000000}%
\pgfsetfillcolor{currentfill}%
\pgfsetlinewidth{0.000000pt}%
\definecolor{currentstroke}{rgb}{0.000000,0.000000,0.000000}%
\pgfsetstrokecolor{currentstroke}%
\pgfsetstrokeopacity{0.000000}%
\pgfsetdash{}{0pt}%
\pgfpathmoveto{\pgfqpoint{2.683992in}{0.613486in}}%
\pgfpathlineto{\pgfqpoint{2.693997in}{0.613486in}}%
\pgfpathlineto{\pgfqpoint{2.693997in}{1.612282in}}%
\pgfpathlineto{\pgfqpoint{2.683992in}{1.612282in}}%
\pgfpathlineto{\pgfqpoint{2.683992in}{0.613486in}}%
\pgfpathclose%
\pgfusepath{fill}%
\end{pgfscope}%
\begin{pgfscope}%
\pgfpathrectangle{\pgfqpoint{0.693757in}{0.613486in}}{\pgfqpoint{5.541243in}{3.963477in}}%
\pgfusepath{clip}%
\pgfsetbuttcap%
\pgfsetmiterjoin%
\definecolor{currentfill}{rgb}{0.000000,0.000000,1.000000}%
\pgfsetfillcolor{currentfill}%
\pgfsetlinewidth{0.000000pt}%
\definecolor{currentstroke}{rgb}{0.000000,0.000000,0.000000}%
\pgfsetstrokecolor{currentstroke}%
\pgfsetstrokeopacity{0.000000}%
\pgfsetdash{}{0pt}%
\pgfpathmoveto{\pgfqpoint{2.696498in}{0.613486in}}%
\pgfpathlineto{\pgfqpoint{2.706503in}{0.613486in}}%
\pgfpathlineto{\pgfqpoint{2.706503in}{1.604355in}}%
\pgfpathlineto{\pgfqpoint{2.696498in}{1.604355in}}%
\pgfpathlineto{\pgfqpoint{2.696498in}{0.613486in}}%
\pgfpathclose%
\pgfusepath{fill}%
\end{pgfscope}%
\begin{pgfscope}%
\pgfpathrectangle{\pgfqpoint{0.693757in}{0.613486in}}{\pgfqpoint{5.541243in}{3.963477in}}%
\pgfusepath{clip}%
\pgfsetbuttcap%
\pgfsetmiterjoin%
\definecolor{currentfill}{rgb}{0.000000,0.000000,1.000000}%
\pgfsetfillcolor{currentfill}%
\pgfsetlinewidth{0.000000pt}%
\definecolor{currentstroke}{rgb}{0.000000,0.000000,0.000000}%
\pgfsetstrokecolor{currentstroke}%
\pgfsetstrokeopacity{0.000000}%
\pgfsetdash{}{0pt}%
\pgfpathmoveto{\pgfqpoint{2.709005in}{0.613486in}}%
\pgfpathlineto{\pgfqpoint{2.719010in}{0.613486in}}%
\pgfpathlineto{\pgfqpoint{2.719010in}{2.611074in}}%
\pgfpathlineto{\pgfqpoint{2.709005in}{2.611074in}}%
\pgfpathlineto{\pgfqpoint{2.709005in}{0.613486in}}%
\pgfpathclose%
\pgfusepath{fill}%
\end{pgfscope}%
\begin{pgfscope}%
\pgfpathrectangle{\pgfqpoint{0.693757in}{0.613486in}}{\pgfqpoint{5.541243in}{3.963477in}}%
\pgfusepath{clip}%
\pgfsetbuttcap%
\pgfsetmiterjoin%
\definecolor{currentfill}{rgb}{0.000000,0.000000,1.000000}%
\pgfsetfillcolor{currentfill}%
\pgfsetlinewidth{0.000000pt}%
\definecolor{currentstroke}{rgb}{0.000000,0.000000,0.000000}%
\pgfsetstrokecolor{currentstroke}%
\pgfsetstrokeopacity{0.000000}%
\pgfsetdash{}{0pt}%
\pgfpathmoveto{\pgfqpoint{2.721511in}{0.613486in}}%
\pgfpathlineto{\pgfqpoint{2.731516in}{0.613486in}}%
\pgfpathlineto{\pgfqpoint{2.731516in}{2.595224in}}%
\pgfpathlineto{\pgfqpoint{2.721511in}{2.595224in}}%
\pgfpathlineto{\pgfqpoint{2.721511in}{0.613486in}}%
\pgfpathclose%
\pgfusepath{fill}%
\end{pgfscope}%
\begin{pgfscope}%
\pgfpathrectangle{\pgfqpoint{0.693757in}{0.613486in}}{\pgfqpoint{5.541243in}{3.963477in}}%
\pgfusepath{clip}%
\pgfsetbuttcap%
\pgfsetmiterjoin%
\definecolor{currentfill}{rgb}{0.000000,0.000000,1.000000}%
\pgfsetfillcolor{currentfill}%
\pgfsetlinewidth{0.000000pt}%
\definecolor{currentstroke}{rgb}{0.000000,0.000000,0.000000}%
\pgfsetstrokecolor{currentstroke}%
\pgfsetstrokeopacity{0.000000}%
\pgfsetdash{}{0pt}%
\pgfpathmoveto{\pgfqpoint{2.734017in}{0.613486in}}%
\pgfpathlineto{\pgfqpoint{2.744022in}{0.613486in}}%
\pgfpathlineto{\pgfqpoint{2.744022in}{1.612282in}}%
\pgfpathlineto{\pgfqpoint{2.734017in}{1.612282in}}%
\pgfpathlineto{\pgfqpoint{2.734017in}{0.613486in}}%
\pgfpathclose%
\pgfusepath{fill}%
\end{pgfscope}%
\begin{pgfscope}%
\pgfpathrectangle{\pgfqpoint{0.693757in}{0.613486in}}{\pgfqpoint{5.541243in}{3.963477in}}%
\pgfusepath{clip}%
\pgfsetbuttcap%
\pgfsetmiterjoin%
\definecolor{currentfill}{rgb}{0.000000,0.000000,1.000000}%
\pgfsetfillcolor{currentfill}%
\pgfsetlinewidth{0.000000pt}%
\definecolor{currentstroke}{rgb}{0.000000,0.000000,0.000000}%
\pgfsetstrokecolor{currentstroke}%
\pgfsetstrokeopacity{0.000000}%
\pgfsetdash{}{0pt}%
\pgfpathmoveto{\pgfqpoint{2.746523in}{0.613486in}}%
\pgfpathlineto{\pgfqpoint{2.756528in}{0.613486in}}%
\pgfpathlineto{\pgfqpoint{2.756528in}{1.604355in}}%
\pgfpathlineto{\pgfqpoint{2.746523in}{1.604355in}}%
\pgfpathlineto{\pgfqpoint{2.746523in}{0.613486in}}%
\pgfpathclose%
\pgfusepath{fill}%
\end{pgfscope}%
\begin{pgfscope}%
\pgfpathrectangle{\pgfqpoint{0.693757in}{0.613486in}}{\pgfqpoint{5.541243in}{3.963477in}}%
\pgfusepath{clip}%
\pgfsetbuttcap%
\pgfsetmiterjoin%
\definecolor{currentfill}{rgb}{0.000000,0.000000,1.000000}%
\pgfsetfillcolor{currentfill}%
\pgfsetlinewidth{0.000000pt}%
\definecolor{currentstroke}{rgb}{0.000000,0.000000,0.000000}%
\pgfsetstrokecolor{currentstroke}%
\pgfsetstrokeopacity{0.000000}%
\pgfsetdash{}{0pt}%
\pgfpathmoveto{\pgfqpoint{2.759029in}{0.613486in}}%
\pgfpathlineto{\pgfqpoint{2.769034in}{0.613486in}}%
\pgfpathlineto{\pgfqpoint{2.769034in}{2.611074in}}%
\pgfpathlineto{\pgfqpoint{2.759029in}{2.611074in}}%
\pgfpathlineto{\pgfqpoint{2.759029in}{0.613486in}}%
\pgfpathclose%
\pgfusepath{fill}%
\end{pgfscope}%
\begin{pgfscope}%
\pgfpathrectangle{\pgfqpoint{0.693757in}{0.613486in}}{\pgfqpoint{5.541243in}{3.963477in}}%
\pgfusepath{clip}%
\pgfsetbuttcap%
\pgfsetmiterjoin%
\definecolor{currentfill}{rgb}{0.000000,0.000000,1.000000}%
\pgfsetfillcolor{currentfill}%
\pgfsetlinewidth{0.000000pt}%
\definecolor{currentstroke}{rgb}{0.000000,0.000000,0.000000}%
\pgfsetstrokecolor{currentstroke}%
\pgfsetstrokeopacity{0.000000}%
\pgfsetdash{}{0pt}%
\pgfpathmoveto{\pgfqpoint{2.771536in}{0.613486in}}%
\pgfpathlineto{\pgfqpoint{2.781541in}{0.613486in}}%
\pgfpathlineto{\pgfqpoint{2.781541in}{2.595224in}}%
\pgfpathlineto{\pgfqpoint{2.771536in}{2.595224in}}%
\pgfpathlineto{\pgfqpoint{2.771536in}{0.613486in}}%
\pgfpathclose%
\pgfusepath{fill}%
\end{pgfscope}%
\begin{pgfscope}%
\pgfpathrectangle{\pgfqpoint{0.693757in}{0.613486in}}{\pgfqpoint{5.541243in}{3.963477in}}%
\pgfusepath{clip}%
\pgfsetbuttcap%
\pgfsetmiterjoin%
\definecolor{currentfill}{rgb}{0.000000,0.000000,1.000000}%
\pgfsetfillcolor{currentfill}%
\pgfsetlinewidth{0.000000pt}%
\definecolor{currentstroke}{rgb}{0.000000,0.000000,0.000000}%
\pgfsetstrokecolor{currentstroke}%
\pgfsetstrokeopacity{0.000000}%
\pgfsetdash{}{0pt}%
\pgfpathmoveto{\pgfqpoint{2.784042in}{0.613486in}}%
\pgfpathlineto{\pgfqpoint{2.794047in}{0.613486in}}%
\pgfpathlineto{\pgfqpoint{2.794047in}{1.612282in}}%
\pgfpathlineto{\pgfqpoint{2.784042in}{1.612282in}}%
\pgfpathlineto{\pgfqpoint{2.784042in}{0.613486in}}%
\pgfpathclose%
\pgfusepath{fill}%
\end{pgfscope}%
\begin{pgfscope}%
\pgfpathrectangle{\pgfqpoint{0.693757in}{0.613486in}}{\pgfqpoint{5.541243in}{3.963477in}}%
\pgfusepath{clip}%
\pgfsetbuttcap%
\pgfsetmiterjoin%
\definecolor{currentfill}{rgb}{0.000000,0.000000,1.000000}%
\pgfsetfillcolor{currentfill}%
\pgfsetlinewidth{0.000000pt}%
\definecolor{currentstroke}{rgb}{0.000000,0.000000,0.000000}%
\pgfsetstrokecolor{currentstroke}%
\pgfsetstrokeopacity{0.000000}%
\pgfsetdash{}{0pt}%
\pgfpathmoveto{\pgfqpoint{2.796548in}{0.613486in}}%
\pgfpathlineto{\pgfqpoint{2.806553in}{0.613486in}}%
\pgfpathlineto{\pgfqpoint{2.806553in}{1.604355in}}%
\pgfpathlineto{\pgfqpoint{2.796548in}{1.604355in}}%
\pgfpathlineto{\pgfqpoint{2.796548in}{0.613486in}}%
\pgfpathclose%
\pgfusepath{fill}%
\end{pgfscope}%
\begin{pgfscope}%
\pgfpathrectangle{\pgfqpoint{0.693757in}{0.613486in}}{\pgfqpoint{5.541243in}{3.963477in}}%
\pgfusepath{clip}%
\pgfsetbuttcap%
\pgfsetmiterjoin%
\definecolor{currentfill}{rgb}{0.000000,0.000000,1.000000}%
\pgfsetfillcolor{currentfill}%
\pgfsetlinewidth{0.000000pt}%
\definecolor{currentstroke}{rgb}{0.000000,0.000000,0.000000}%
\pgfsetstrokecolor{currentstroke}%
\pgfsetstrokeopacity{0.000000}%
\pgfsetdash{}{0pt}%
\pgfpathmoveto{\pgfqpoint{2.809054in}{0.613486in}}%
\pgfpathlineto{\pgfqpoint{2.819059in}{0.613486in}}%
\pgfpathlineto{\pgfqpoint{2.819059in}{2.611074in}}%
\pgfpathlineto{\pgfqpoint{2.809054in}{2.611074in}}%
\pgfpathlineto{\pgfqpoint{2.809054in}{0.613486in}}%
\pgfpathclose%
\pgfusepath{fill}%
\end{pgfscope}%
\begin{pgfscope}%
\pgfpathrectangle{\pgfqpoint{0.693757in}{0.613486in}}{\pgfqpoint{5.541243in}{3.963477in}}%
\pgfusepath{clip}%
\pgfsetbuttcap%
\pgfsetmiterjoin%
\definecolor{currentfill}{rgb}{0.000000,0.000000,1.000000}%
\pgfsetfillcolor{currentfill}%
\pgfsetlinewidth{0.000000pt}%
\definecolor{currentstroke}{rgb}{0.000000,0.000000,0.000000}%
\pgfsetstrokecolor{currentstroke}%
\pgfsetstrokeopacity{0.000000}%
\pgfsetdash{}{0pt}%
\pgfpathmoveto{\pgfqpoint{2.821560in}{0.613486in}}%
\pgfpathlineto{\pgfqpoint{2.831565in}{0.613486in}}%
\pgfpathlineto{\pgfqpoint{2.831565in}{2.595224in}}%
\pgfpathlineto{\pgfqpoint{2.821560in}{2.595224in}}%
\pgfpathlineto{\pgfqpoint{2.821560in}{0.613486in}}%
\pgfpathclose%
\pgfusepath{fill}%
\end{pgfscope}%
\begin{pgfscope}%
\pgfpathrectangle{\pgfqpoint{0.693757in}{0.613486in}}{\pgfqpoint{5.541243in}{3.963477in}}%
\pgfusepath{clip}%
\pgfsetbuttcap%
\pgfsetmiterjoin%
\definecolor{currentfill}{rgb}{0.000000,0.000000,1.000000}%
\pgfsetfillcolor{currentfill}%
\pgfsetlinewidth{0.000000pt}%
\definecolor{currentstroke}{rgb}{0.000000,0.000000,0.000000}%
\pgfsetstrokecolor{currentstroke}%
\pgfsetstrokeopacity{0.000000}%
\pgfsetdash{}{0pt}%
\pgfpathmoveto{\pgfqpoint{2.834067in}{0.613486in}}%
\pgfpathlineto{\pgfqpoint{2.844071in}{0.613486in}}%
\pgfpathlineto{\pgfqpoint{2.844071in}{1.612282in}}%
\pgfpathlineto{\pgfqpoint{2.834067in}{1.612282in}}%
\pgfpathlineto{\pgfqpoint{2.834067in}{0.613486in}}%
\pgfpathclose%
\pgfusepath{fill}%
\end{pgfscope}%
\begin{pgfscope}%
\pgfpathrectangle{\pgfqpoint{0.693757in}{0.613486in}}{\pgfqpoint{5.541243in}{3.963477in}}%
\pgfusepath{clip}%
\pgfsetbuttcap%
\pgfsetmiterjoin%
\definecolor{currentfill}{rgb}{0.000000,0.000000,1.000000}%
\pgfsetfillcolor{currentfill}%
\pgfsetlinewidth{0.000000pt}%
\definecolor{currentstroke}{rgb}{0.000000,0.000000,0.000000}%
\pgfsetstrokecolor{currentstroke}%
\pgfsetstrokeopacity{0.000000}%
\pgfsetdash{}{0pt}%
\pgfpathmoveto{\pgfqpoint{2.846573in}{0.613486in}}%
\pgfpathlineto{\pgfqpoint{2.856578in}{0.613486in}}%
\pgfpathlineto{\pgfqpoint{2.856578in}{1.604355in}}%
\pgfpathlineto{\pgfqpoint{2.846573in}{1.604355in}}%
\pgfpathlineto{\pgfqpoint{2.846573in}{0.613486in}}%
\pgfpathclose%
\pgfusepath{fill}%
\end{pgfscope}%
\begin{pgfscope}%
\pgfpathrectangle{\pgfqpoint{0.693757in}{0.613486in}}{\pgfqpoint{5.541243in}{3.963477in}}%
\pgfusepath{clip}%
\pgfsetbuttcap%
\pgfsetmiterjoin%
\definecolor{currentfill}{rgb}{0.000000,0.000000,1.000000}%
\pgfsetfillcolor{currentfill}%
\pgfsetlinewidth{0.000000pt}%
\definecolor{currentstroke}{rgb}{0.000000,0.000000,0.000000}%
\pgfsetstrokecolor{currentstroke}%
\pgfsetstrokeopacity{0.000000}%
\pgfsetdash{}{0pt}%
\pgfpathmoveto{\pgfqpoint{2.859079in}{0.613486in}}%
\pgfpathlineto{\pgfqpoint{2.869084in}{0.613486in}}%
\pgfpathlineto{\pgfqpoint{2.869084in}{2.611074in}}%
\pgfpathlineto{\pgfqpoint{2.859079in}{2.611074in}}%
\pgfpathlineto{\pgfqpoint{2.859079in}{0.613486in}}%
\pgfpathclose%
\pgfusepath{fill}%
\end{pgfscope}%
\begin{pgfscope}%
\pgfpathrectangle{\pgfqpoint{0.693757in}{0.613486in}}{\pgfqpoint{5.541243in}{3.963477in}}%
\pgfusepath{clip}%
\pgfsetbuttcap%
\pgfsetmiterjoin%
\definecolor{currentfill}{rgb}{0.000000,0.000000,1.000000}%
\pgfsetfillcolor{currentfill}%
\pgfsetlinewidth{0.000000pt}%
\definecolor{currentstroke}{rgb}{0.000000,0.000000,0.000000}%
\pgfsetstrokecolor{currentstroke}%
\pgfsetstrokeopacity{0.000000}%
\pgfsetdash{}{0pt}%
\pgfpathmoveto{\pgfqpoint{2.871585in}{0.613486in}}%
\pgfpathlineto{\pgfqpoint{2.881590in}{0.613486in}}%
\pgfpathlineto{\pgfqpoint{2.881590in}{2.595224in}}%
\pgfpathlineto{\pgfqpoint{2.871585in}{2.595224in}}%
\pgfpathlineto{\pgfqpoint{2.871585in}{0.613486in}}%
\pgfpathclose%
\pgfusepath{fill}%
\end{pgfscope}%
\begin{pgfscope}%
\pgfpathrectangle{\pgfqpoint{0.693757in}{0.613486in}}{\pgfqpoint{5.541243in}{3.963477in}}%
\pgfusepath{clip}%
\pgfsetbuttcap%
\pgfsetmiterjoin%
\definecolor{currentfill}{rgb}{0.000000,0.000000,1.000000}%
\pgfsetfillcolor{currentfill}%
\pgfsetlinewidth{0.000000pt}%
\definecolor{currentstroke}{rgb}{0.000000,0.000000,0.000000}%
\pgfsetstrokecolor{currentstroke}%
\pgfsetstrokeopacity{0.000000}%
\pgfsetdash{}{0pt}%
\pgfpathmoveto{\pgfqpoint{2.884091in}{0.613486in}}%
\pgfpathlineto{\pgfqpoint{2.894096in}{0.613486in}}%
\pgfpathlineto{\pgfqpoint{2.894096in}{1.612282in}}%
\pgfpathlineto{\pgfqpoint{2.884091in}{1.612282in}}%
\pgfpathlineto{\pgfqpoint{2.884091in}{0.613486in}}%
\pgfpathclose%
\pgfusepath{fill}%
\end{pgfscope}%
\begin{pgfscope}%
\pgfpathrectangle{\pgfqpoint{0.693757in}{0.613486in}}{\pgfqpoint{5.541243in}{3.963477in}}%
\pgfusepath{clip}%
\pgfsetbuttcap%
\pgfsetmiterjoin%
\definecolor{currentfill}{rgb}{0.000000,0.000000,1.000000}%
\pgfsetfillcolor{currentfill}%
\pgfsetlinewidth{0.000000pt}%
\definecolor{currentstroke}{rgb}{0.000000,0.000000,0.000000}%
\pgfsetstrokecolor{currentstroke}%
\pgfsetstrokeopacity{0.000000}%
\pgfsetdash{}{0pt}%
\pgfpathmoveto{\pgfqpoint{2.896597in}{0.613486in}}%
\pgfpathlineto{\pgfqpoint{2.906602in}{0.613486in}}%
\pgfpathlineto{\pgfqpoint{2.906602in}{1.604355in}}%
\pgfpathlineto{\pgfqpoint{2.896597in}{1.604355in}}%
\pgfpathlineto{\pgfqpoint{2.896597in}{0.613486in}}%
\pgfpathclose%
\pgfusepath{fill}%
\end{pgfscope}%
\begin{pgfscope}%
\pgfpathrectangle{\pgfqpoint{0.693757in}{0.613486in}}{\pgfqpoint{5.541243in}{3.963477in}}%
\pgfusepath{clip}%
\pgfsetbuttcap%
\pgfsetmiterjoin%
\definecolor{currentfill}{rgb}{0.000000,0.000000,1.000000}%
\pgfsetfillcolor{currentfill}%
\pgfsetlinewidth{0.000000pt}%
\definecolor{currentstroke}{rgb}{0.000000,0.000000,0.000000}%
\pgfsetstrokecolor{currentstroke}%
\pgfsetstrokeopacity{0.000000}%
\pgfsetdash{}{0pt}%
\pgfpathmoveto{\pgfqpoint{2.909104in}{0.613486in}}%
\pgfpathlineto{\pgfqpoint{2.919109in}{0.613486in}}%
\pgfpathlineto{\pgfqpoint{2.919109in}{2.611074in}}%
\pgfpathlineto{\pgfqpoint{2.909104in}{2.611074in}}%
\pgfpathlineto{\pgfqpoint{2.909104in}{0.613486in}}%
\pgfpathclose%
\pgfusepath{fill}%
\end{pgfscope}%
\begin{pgfscope}%
\pgfpathrectangle{\pgfqpoint{0.693757in}{0.613486in}}{\pgfqpoint{5.541243in}{3.963477in}}%
\pgfusepath{clip}%
\pgfsetbuttcap%
\pgfsetmiterjoin%
\definecolor{currentfill}{rgb}{0.000000,0.000000,1.000000}%
\pgfsetfillcolor{currentfill}%
\pgfsetlinewidth{0.000000pt}%
\definecolor{currentstroke}{rgb}{0.000000,0.000000,0.000000}%
\pgfsetstrokecolor{currentstroke}%
\pgfsetstrokeopacity{0.000000}%
\pgfsetdash{}{0pt}%
\pgfpathmoveto{\pgfqpoint{2.921610in}{0.613486in}}%
\pgfpathlineto{\pgfqpoint{2.931615in}{0.613486in}}%
\pgfpathlineto{\pgfqpoint{2.931615in}{2.595224in}}%
\pgfpathlineto{\pgfqpoint{2.921610in}{2.595224in}}%
\pgfpathlineto{\pgfqpoint{2.921610in}{0.613486in}}%
\pgfpathclose%
\pgfusepath{fill}%
\end{pgfscope}%
\begin{pgfscope}%
\pgfpathrectangle{\pgfqpoint{0.693757in}{0.613486in}}{\pgfqpoint{5.541243in}{3.963477in}}%
\pgfusepath{clip}%
\pgfsetbuttcap%
\pgfsetmiterjoin%
\definecolor{currentfill}{rgb}{0.000000,0.000000,1.000000}%
\pgfsetfillcolor{currentfill}%
\pgfsetlinewidth{0.000000pt}%
\definecolor{currentstroke}{rgb}{0.000000,0.000000,0.000000}%
\pgfsetstrokecolor{currentstroke}%
\pgfsetstrokeopacity{0.000000}%
\pgfsetdash{}{0pt}%
\pgfpathmoveto{\pgfqpoint{2.934116in}{0.613486in}}%
\pgfpathlineto{\pgfqpoint{2.944121in}{0.613486in}}%
\pgfpathlineto{\pgfqpoint{2.944121in}{1.612282in}}%
\pgfpathlineto{\pgfqpoint{2.934116in}{1.612282in}}%
\pgfpathlineto{\pgfqpoint{2.934116in}{0.613486in}}%
\pgfpathclose%
\pgfusepath{fill}%
\end{pgfscope}%
\begin{pgfscope}%
\pgfpathrectangle{\pgfqpoint{0.693757in}{0.613486in}}{\pgfqpoint{5.541243in}{3.963477in}}%
\pgfusepath{clip}%
\pgfsetbuttcap%
\pgfsetmiterjoin%
\definecolor{currentfill}{rgb}{0.000000,0.000000,1.000000}%
\pgfsetfillcolor{currentfill}%
\pgfsetlinewidth{0.000000pt}%
\definecolor{currentstroke}{rgb}{0.000000,0.000000,0.000000}%
\pgfsetstrokecolor{currentstroke}%
\pgfsetstrokeopacity{0.000000}%
\pgfsetdash{}{0pt}%
\pgfpathmoveto{\pgfqpoint{2.946622in}{0.613486in}}%
\pgfpathlineto{\pgfqpoint{2.956627in}{0.613486in}}%
\pgfpathlineto{\pgfqpoint{2.956627in}{1.604355in}}%
\pgfpathlineto{\pgfqpoint{2.946622in}{1.604355in}}%
\pgfpathlineto{\pgfqpoint{2.946622in}{0.613486in}}%
\pgfpathclose%
\pgfusepath{fill}%
\end{pgfscope}%
\begin{pgfscope}%
\pgfpathrectangle{\pgfqpoint{0.693757in}{0.613486in}}{\pgfqpoint{5.541243in}{3.963477in}}%
\pgfusepath{clip}%
\pgfsetbuttcap%
\pgfsetmiterjoin%
\definecolor{currentfill}{rgb}{0.000000,0.000000,1.000000}%
\pgfsetfillcolor{currentfill}%
\pgfsetlinewidth{0.000000pt}%
\definecolor{currentstroke}{rgb}{0.000000,0.000000,0.000000}%
\pgfsetstrokecolor{currentstroke}%
\pgfsetstrokeopacity{0.000000}%
\pgfsetdash{}{0pt}%
\pgfpathmoveto{\pgfqpoint{2.959128in}{0.613486in}}%
\pgfpathlineto{\pgfqpoint{2.969133in}{0.613486in}}%
\pgfpathlineto{\pgfqpoint{2.969133in}{2.611074in}}%
\pgfpathlineto{\pgfqpoint{2.959128in}{2.611074in}}%
\pgfpathlineto{\pgfqpoint{2.959128in}{0.613486in}}%
\pgfpathclose%
\pgfusepath{fill}%
\end{pgfscope}%
\begin{pgfscope}%
\pgfpathrectangle{\pgfqpoint{0.693757in}{0.613486in}}{\pgfqpoint{5.541243in}{3.963477in}}%
\pgfusepath{clip}%
\pgfsetbuttcap%
\pgfsetmiterjoin%
\definecolor{currentfill}{rgb}{0.000000,0.000000,1.000000}%
\pgfsetfillcolor{currentfill}%
\pgfsetlinewidth{0.000000pt}%
\definecolor{currentstroke}{rgb}{0.000000,0.000000,0.000000}%
\pgfsetstrokecolor{currentstroke}%
\pgfsetstrokeopacity{0.000000}%
\pgfsetdash{}{0pt}%
\pgfpathmoveto{\pgfqpoint{2.971635in}{0.613486in}}%
\pgfpathlineto{\pgfqpoint{2.981640in}{0.613486in}}%
\pgfpathlineto{\pgfqpoint{2.981640in}{2.595224in}}%
\pgfpathlineto{\pgfqpoint{2.971635in}{2.595224in}}%
\pgfpathlineto{\pgfqpoint{2.971635in}{0.613486in}}%
\pgfpathclose%
\pgfusepath{fill}%
\end{pgfscope}%
\begin{pgfscope}%
\pgfpathrectangle{\pgfqpoint{0.693757in}{0.613486in}}{\pgfqpoint{5.541243in}{3.963477in}}%
\pgfusepath{clip}%
\pgfsetbuttcap%
\pgfsetmiterjoin%
\definecolor{currentfill}{rgb}{0.000000,0.000000,1.000000}%
\pgfsetfillcolor{currentfill}%
\pgfsetlinewidth{0.000000pt}%
\definecolor{currentstroke}{rgb}{0.000000,0.000000,0.000000}%
\pgfsetstrokecolor{currentstroke}%
\pgfsetstrokeopacity{0.000000}%
\pgfsetdash{}{0pt}%
\pgfpathmoveto{\pgfqpoint{2.984141in}{0.613486in}}%
\pgfpathlineto{\pgfqpoint{2.994146in}{0.613486in}}%
\pgfpathlineto{\pgfqpoint{2.994146in}{1.612282in}}%
\pgfpathlineto{\pgfqpoint{2.984141in}{1.612282in}}%
\pgfpathlineto{\pgfqpoint{2.984141in}{0.613486in}}%
\pgfpathclose%
\pgfusepath{fill}%
\end{pgfscope}%
\begin{pgfscope}%
\pgfpathrectangle{\pgfqpoint{0.693757in}{0.613486in}}{\pgfqpoint{5.541243in}{3.963477in}}%
\pgfusepath{clip}%
\pgfsetbuttcap%
\pgfsetmiterjoin%
\definecolor{currentfill}{rgb}{0.000000,0.000000,1.000000}%
\pgfsetfillcolor{currentfill}%
\pgfsetlinewidth{0.000000pt}%
\definecolor{currentstroke}{rgb}{0.000000,0.000000,0.000000}%
\pgfsetstrokecolor{currentstroke}%
\pgfsetstrokeopacity{0.000000}%
\pgfsetdash{}{0pt}%
\pgfpathmoveto{\pgfqpoint{2.996647in}{0.613486in}}%
\pgfpathlineto{\pgfqpoint{3.006652in}{0.613486in}}%
\pgfpathlineto{\pgfqpoint{3.006652in}{1.604355in}}%
\pgfpathlineto{\pgfqpoint{2.996647in}{1.604355in}}%
\pgfpathlineto{\pgfqpoint{2.996647in}{0.613486in}}%
\pgfpathclose%
\pgfusepath{fill}%
\end{pgfscope}%
\begin{pgfscope}%
\pgfpathrectangle{\pgfqpoint{0.693757in}{0.613486in}}{\pgfqpoint{5.541243in}{3.963477in}}%
\pgfusepath{clip}%
\pgfsetbuttcap%
\pgfsetmiterjoin%
\definecolor{currentfill}{rgb}{0.000000,0.000000,1.000000}%
\pgfsetfillcolor{currentfill}%
\pgfsetlinewidth{0.000000pt}%
\definecolor{currentstroke}{rgb}{0.000000,0.000000,0.000000}%
\pgfsetstrokecolor{currentstroke}%
\pgfsetstrokeopacity{0.000000}%
\pgfsetdash{}{0pt}%
\pgfpathmoveto{\pgfqpoint{3.009153in}{0.613486in}}%
\pgfpathlineto{\pgfqpoint{3.019158in}{0.613486in}}%
\pgfpathlineto{\pgfqpoint{3.019158in}{2.611074in}}%
\pgfpathlineto{\pgfqpoint{3.009153in}{2.611074in}}%
\pgfpathlineto{\pgfqpoint{3.009153in}{0.613486in}}%
\pgfpathclose%
\pgfusepath{fill}%
\end{pgfscope}%
\begin{pgfscope}%
\pgfpathrectangle{\pgfqpoint{0.693757in}{0.613486in}}{\pgfqpoint{5.541243in}{3.963477in}}%
\pgfusepath{clip}%
\pgfsetbuttcap%
\pgfsetmiterjoin%
\definecolor{currentfill}{rgb}{0.000000,0.000000,1.000000}%
\pgfsetfillcolor{currentfill}%
\pgfsetlinewidth{0.000000pt}%
\definecolor{currentstroke}{rgb}{0.000000,0.000000,0.000000}%
\pgfsetstrokecolor{currentstroke}%
\pgfsetstrokeopacity{0.000000}%
\pgfsetdash{}{0pt}%
\pgfpathmoveto{\pgfqpoint{3.021659in}{0.613486in}}%
\pgfpathlineto{\pgfqpoint{3.031664in}{0.613486in}}%
\pgfpathlineto{\pgfqpoint{3.031664in}{2.595224in}}%
\pgfpathlineto{\pgfqpoint{3.021659in}{2.595224in}}%
\pgfpathlineto{\pgfqpoint{3.021659in}{0.613486in}}%
\pgfpathclose%
\pgfusepath{fill}%
\end{pgfscope}%
\begin{pgfscope}%
\pgfpathrectangle{\pgfqpoint{0.693757in}{0.613486in}}{\pgfqpoint{5.541243in}{3.963477in}}%
\pgfusepath{clip}%
\pgfsetbuttcap%
\pgfsetmiterjoin%
\definecolor{currentfill}{rgb}{0.000000,0.000000,1.000000}%
\pgfsetfillcolor{currentfill}%
\pgfsetlinewidth{0.000000pt}%
\definecolor{currentstroke}{rgb}{0.000000,0.000000,0.000000}%
\pgfsetstrokecolor{currentstroke}%
\pgfsetstrokeopacity{0.000000}%
\pgfsetdash{}{0pt}%
\pgfpathmoveto{\pgfqpoint{3.034166in}{0.613486in}}%
\pgfpathlineto{\pgfqpoint{3.044171in}{0.613486in}}%
\pgfpathlineto{\pgfqpoint{3.044171in}{1.612282in}}%
\pgfpathlineto{\pgfqpoint{3.034166in}{1.612282in}}%
\pgfpathlineto{\pgfqpoint{3.034166in}{0.613486in}}%
\pgfpathclose%
\pgfusepath{fill}%
\end{pgfscope}%
\begin{pgfscope}%
\pgfpathrectangle{\pgfqpoint{0.693757in}{0.613486in}}{\pgfqpoint{5.541243in}{3.963477in}}%
\pgfusepath{clip}%
\pgfsetbuttcap%
\pgfsetmiterjoin%
\definecolor{currentfill}{rgb}{0.000000,0.000000,1.000000}%
\pgfsetfillcolor{currentfill}%
\pgfsetlinewidth{0.000000pt}%
\definecolor{currentstroke}{rgb}{0.000000,0.000000,0.000000}%
\pgfsetstrokecolor{currentstroke}%
\pgfsetstrokeopacity{0.000000}%
\pgfsetdash{}{0pt}%
\pgfpathmoveto{\pgfqpoint{3.046672in}{0.613486in}}%
\pgfpathlineto{\pgfqpoint{3.056677in}{0.613486in}}%
\pgfpathlineto{\pgfqpoint{3.056677in}{1.604355in}}%
\pgfpathlineto{\pgfqpoint{3.046672in}{1.604355in}}%
\pgfpathlineto{\pgfqpoint{3.046672in}{0.613486in}}%
\pgfpathclose%
\pgfusepath{fill}%
\end{pgfscope}%
\begin{pgfscope}%
\pgfpathrectangle{\pgfqpoint{0.693757in}{0.613486in}}{\pgfqpoint{5.541243in}{3.963477in}}%
\pgfusepath{clip}%
\pgfsetbuttcap%
\pgfsetmiterjoin%
\definecolor{currentfill}{rgb}{0.000000,0.000000,1.000000}%
\pgfsetfillcolor{currentfill}%
\pgfsetlinewidth{0.000000pt}%
\definecolor{currentstroke}{rgb}{0.000000,0.000000,0.000000}%
\pgfsetstrokecolor{currentstroke}%
\pgfsetstrokeopacity{0.000000}%
\pgfsetdash{}{0pt}%
\pgfpathmoveto{\pgfqpoint{3.059178in}{0.613486in}}%
\pgfpathlineto{\pgfqpoint{3.069183in}{0.613486in}}%
\pgfpathlineto{\pgfqpoint{3.069183in}{2.611074in}}%
\pgfpathlineto{\pgfqpoint{3.059178in}{2.611074in}}%
\pgfpathlineto{\pgfqpoint{3.059178in}{0.613486in}}%
\pgfpathclose%
\pgfusepath{fill}%
\end{pgfscope}%
\begin{pgfscope}%
\pgfpathrectangle{\pgfqpoint{0.693757in}{0.613486in}}{\pgfqpoint{5.541243in}{3.963477in}}%
\pgfusepath{clip}%
\pgfsetbuttcap%
\pgfsetmiterjoin%
\definecolor{currentfill}{rgb}{0.000000,0.000000,1.000000}%
\pgfsetfillcolor{currentfill}%
\pgfsetlinewidth{0.000000pt}%
\definecolor{currentstroke}{rgb}{0.000000,0.000000,0.000000}%
\pgfsetstrokecolor{currentstroke}%
\pgfsetstrokeopacity{0.000000}%
\pgfsetdash{}{0pt}%
\pgfpathmoveto{\pgfqpoint{3.071684in}{0.613486in}}%
\pgfpathlineto{\pgfqpoint{3.081689in}{0.613486in}}%
\pgfpathlineto{\pgfqpoint{3.081689in}{2.595224in}}%
\pgfpathlineto{\pgfqpoint{3.071684in}{2.595224in}}%
\pgfpathlineto{\pgfqpoint{3.071684in}{0.613486in}}%
\pgfpathclose%
\pgfusepath{fill}%
\end{pgfscope}%
\begin{pgfscope}%
\pgfpathrectangle{\pgfqpoint{0.693757in}{0.613486in}}{\pgfqpoint{5.541243in}{3.963477in}}%
\pgfusepath{clip}%
\pgfsetbuttcap%
\pgfsetmiterjoin%
\definecolor{currentfill}{rgb}{0.000000,0.000000,1.000000}%
\pgfsetfillcolor{currentfill}%
\pgfsetlinewidth{0.000000pt}%
\definecolor{currentstroke}{rgb}{0.000000,0.000000,0.000000}%
\pgfsetstrokecolor{currentstroke}%
\pgfsetstrokeopacity{0.000000}%
\pgfsetdash{}{0pt}%
\pgfpathmoveto{\pgfqpoint{3.084190in}{0.613486in}}%
\pgfpathlineto{\pgfqpoint{3.094195in}{0.613486in}}%
\pgfpathlineto{\pgfqpoint{3.094195in}{1.612282in}}%
\pgfpathlineto{\pgfqpoint{3.084190in}{1.612282in}}%
\pgfpathlineto{\pgfqpoint{3.084190in}{0.613486in}}%
\pgfpathclose%
\pgfusepath{fill}%
\end{pgfscope}%
\begin{pgfscope}%
\pgfpathrectangle{\pgfqpoint{0.693757in}{0.613486in}}{\pgfqpoint{5.541243in}{3.963477in}}%
\pgfusepath{clip}%
\pgfsetbuttcap%
\pgfsetmiterjoin%
\definecolor{currentfill}{rgb}{0.000000,0.000000,1.000000}%
\pgfsetfillcolor{currentfill}%
\pgfsetlinewidth{0.000000pt}%
\definecolor{currentstroke}{rgb}{0.000000,0.000000,0.000000}%
\pgfsetstrokecolor{currentstroke}%
\pgfsetstrokeopacity{0.000000}%
\pgfsetdash{}{0pt}%
\pgfpathmoveto{\pgfqpoint{3.096697in}{0.613486in}}%
\pgfpathlineto{\pgfqpoint{3.106701in}{0.613486in}}%
\pgfpathlineto{\pgfqpoint{3.106701in}{1.604355in}}%
\pgfpathlineto{\pgfqpoint{3.096697in}{1.604355in}}%
\pgfpathlineto{\pgfqpoint{3.096697in}{0.613486in}}%
\pgfpathclose%
\pgfusepath{fill}%
\end{pgfscope}%
\begin{pgfscope}%
\pgfpathrectangle{\pgfqpoint{0.693757in}{0.613486in}}{\pgfqpoint{5.541243in}{3.963477in}}%
\pgfusepath{clip}%
\pgfsetbuttcap%
\pgfsetmiterjoin%
\definecolor{currentfill}{rgb}{0.000000,0.000000,1.000000}%
\pgfsetfillcolor{currentfill}%
\pgfsetlinewidth{0.000000pt}%
\definecolor{currentstroke}{rgb}{0.000000,0.000000,0.000000}%
\pgfsetstrokecolor{currentstroke}%
\pgfsetstrokeopacity{0.000000}%
\pgfsetdash{}{0pt}%
\pgfpathmoveto{\pgfqpoint{3.109203in}{0.613486in}}%
\pgfpathlineto{\pgfqpoint{3.119208in}{0.613486in}}%
\pgfpathlineto{\pgfqpoint{3.119208in}{2.611074in}}%
\pgfpathlineto{\pgfqpoint{3.109203in}{2.611074in}}%
\pgfpathlineto{\pgfqpoint{3.109203in}{0.613486in}}%
\pgfpathclose%
\pgfusepath{fill}%
\end{pgfscope}%
\begin{pgfscope}%
\pgfpathrectangle{\pgfqpoint{0.693757in}{0.613486in}}{\pgfqpoint{5.541243in}{3.963477in}}%
\pgfusepath{clip}%
\pgfsetbuttcap%
\pgfsetmiterjoin%
\definecolor{currentfill}{rgb}{0.000000,0.000000,1.000000}%
\pgfsetfillcolor{currentfill}%
\pgfsetlinewidth{0.000000pt}%
\definecolor{currentstroke}{rgb}{0.000000,0.000000,0.000000}%
\pgfsetstrokecolor{currentstroke}%
\pgfsetstrokeopacity{0.000000}%
\pgfsetdash{}{0pt}%
\pgfpathmoveto{\pgfqpoint{3.121709in}{0.613486in}}%
\pgfpathlineto{\pgfqpoint{3.131714in}{0.613486in}}%
\pgfpathlineto{\pgfqpoint{3.131714in}{2.595224in}}%
\pgfpathlineto{\pgfqpoint{3.121709in}{2.595224in}}%
\pgfpathlineto{\pgfqpoint{3.121709in}{0.613486in}}%
\pgfpathclose%
\pgfusepath{fill}%
\end{pgfscope}%
\begin{pgfscope}%
\pgfpathrectangle{\pgfqpoint{0.693757in}{0.613486in}}{\pgfqpoint{5.541243in}{3.963477in}}%
\pgfusepath{clip}%
\pgfsetbuttcap%
\pgfsetmiterjoin%
\definecolor{currentfill}{rgb}{0.000000,0.000000,1.000000}%
\pgfsetfillcolor{currentfill}%
\pgfsetlinewidth{0.000000pt}%
\definecolor{currentstroke}{rgb}{0.000000,0.000000,0.000000}%
\pgfsetstrokecolor{currentstroke}%
\pgfsetstrokeopacity{0.000000}%
\pgfsetdash{}{0pt}%
\pgfpathmoveto{\pgfqpoint{3.134215in}{0.613486in}}%
\pgfpathlineto{\pgfqpoint{3.144220in}{0.613486in}}%
\pgfpathlineto{\pgfqpoint{3.144220in}{1.612282in}}%
\pgfpathlineto{\pgfqpoint{3.134215in}{1.612282in}}%
\pgfpathlineto{\pgfqpoint{3.134215in}{0.613486in}}%
\pgfpathclose%
\pgfusepath{fill}%
\end{pgfscope}%
\begin{pgfscope}%
\pgfpathrectangle{\pgfqpoint{0.693757in}{0.613486in}}{\pgfqpoint{5.541243in}{3.963477in}}%
\pgfusepath{clip}%
\pgfsetbuttcap%
\pgfsetmiterjoin%
\definecolor{currentfill}{rgb}{0.000000,0.000000,1.000000}%
\pgfsetfillcolor{currentfill}%
\pgfsetlinewidth{0.000000pt}%
\definecolor{currentstroke}{rgb}{0.000000,0.000000,0.000000}%
\pgfsetstrokecolor{currentstroke}%
\pgfsetstrokeopacity{0.000000}%
\pgfsetdash{}{0pt}%
\pgfpathmoveto{\pgfqpoint{3.146721in}{0.613486in}}%
\pgfpathlineto{\pgfqpoint{3.156726in}{0.613486in}}%
\pgfpathlineto{\pgfqpoint{3.156726in}{1.604355in}}%
\pgfpathlineto{\pgfqpoint{3.146721in}{1.604355in}}%
\pgfpathlineto{\pgfqpoint{3.146721in}{0.613486in}}%
\pgfpathclose%
\pgfusepath{fill}%
\end{pgfscope}%
\begin{pgfscope}%
\pgfpathrectangle{\pgfqpoint{0.693757in}{0.613486in}}{\pgfqpoint{5.541243in}{3.963477in}}%
\pgfusepath{clip}%
\pgfsetbuttcap%
\pgfsetmiterjoin%
\definecolor{currentfill}{rgb}{0.000000,0.000000,1.000000}%
\pgfsetfillcolor{currentfill}%
\pgfsetlinewidth{0.000000pt}%
\definecolor{currentstroke}{rgb}{0.000000,0.000000,0.000000}%
\pgfsetstrokecolor{currentstroke}%
\pgfsetstrokeopacity{0.000000}%
\pgfsetdash{}{0pt}%
\pgfpathmoveto{\pgfqpoint{3.159227in}{0.613486in}}%
\pgfpathlineto{\pgfqpoint{3.169232in}{0.613486in}}%
\pgfpathlineto{\pgfqpoint{3.169232in}{2.611074in}}%
\pgfpathlineto{\pgfqpoint{3.159227in}{2.611074in}}%
\pgfpathlineto{\pgfqpoint{3.159227in}{0.613486in}}%
\pgfpathclose%
\pgfusepath{fill}%
\end{pgfscope}%
\begin{pgfscope}%
\pgfpathrectangle{\pgfqpoint{0.693757in}{0.613486in}}{\pgfqpoint{5.541243in}{3.963477in}}%
\pgfusepath{clip}%
\pgfsetbuttcap%
\pgfsetmiterjoin%
\definecolor{currentfill}{rgb}{0.000000,0.000000,1.000000}%
\pgfsetfillcolor{currentfill}%
\pgfsetlinewidth{0.000000pt}%
\definecolor{currentstroke}{rgb}{0.000000,0.000000,0.000000}%
\pgfsetstrokecolor{currentstroke}%
\pgfsetstrokeopacity{0.000000}%
\pgfsetdash{}{0pt}%
\pgfpathmoveto{\pgfqpoint{3.171734in}{0.613486in}}%
\pgfpathlineto{\pgfqpoint{3.181739in}{0.613486in}}%
\pgfpathlineto{\pgfqpoint{3.181739in}{2.595224in}}%
\pgfpathlineto{\pgfqpoint{3.171734in}{2.595224in}}%
\pgfpathlineto{\pgfqpoint{3.171734in}{0.613486in}}%
\pgfpathclose%
\pgfusepath{fill}%
\end{pgfscope}%
\begin{pgfscope}%
\pgfpathrectangle{\pgfqpoint{0.693757in}{0.613486in}}{\pgfqpoint{5.541243in}{3.963477in}}%
\pgfusepath{clip}%
\pgfsetbuttcap%
\pgfsetmiterjoin%
\definecolor{currentfill}{rgb}{0.000000,0.000000,1.000000}%
\pgfsetfillcolor{currentfill}%
\pgfsetlinewidth{0.000000pt}%
\definecolor{currentstroke}{rgb}{0.000000,0.000000,0.000000}%
\pgfsetstrokecolor{currentstroke}%
\pgfsetstrokeopacity{0.000000}%
\pgfsetdash{}{0pt}%
\pgfpathmoveto{\pgfqpoint{3.184240in}{0.613486in}}%
\pgfpathlineto{\pgfqpoint{3.194245in}{0.613486in}}%
\pgfpathlineto{\pgfqpoint{3.194245in}{1.612282in}}%
\pgfpathlineto{\pgfqpoint{3.184240in}{1.612282in}}%
\pgfpathlineto{\pgfqpoint{3.184240in}{0.613486in}}%
\pgfpathclose%
\pgfusepath{fill}%
\end{pgfscope}%
\begin{pgfscope}%
\pgfpathrectangle{\pgfqpoint{0.693757in}{0.613486in}}{\pgfqpoint{5.541243in}{3.963477in}}%
\pgfusepath{clip}%
\pgfsetbuttcap%
\pgfsetmiterjoin%
\definecolor{currentfill}{rgb}{0.000000,0.000000,1.000000}%
\pgfsetfillcolor{currentfill}%
\pgfsetlinewidth{0.000000pt}%
\definecolor{currentstroke}{rgb}{0.000000,0.000000,0.000000}%
\pgfsetstrokecolor{currentstroke}%
\pgfsetstrokeopacity{0.000000}%
\pgfsetdash{}{0pt}%
\pgfpathmoveto{\pgfqpoint{3.196746in}{0.613486in}}%
\pgfpathlineto{\pgfqpoint{3.206751in}{0.613486in}}%
\pgfpathlineto{\pgfqpoint{3.206751in}{1.604355in}}%
\pgfpathlineto{\pgfqpoint{3.196746in}{1.604355in}}%
\pgfpathlineto{\pgfqpoint{3.196746in}{0.613486in}}%
\pgfpathclose%
\pgfusepath{fill}%
\end{pgfscope}%
\begin{pgfscope}%
\pgfpathrectangle{\pgfqpoint{0.693757in}{0.613486in}}{\pgfqpoint{5.541243in}{3.963477in}}%
\pgfusepath{clip}%
\pgfsetbuttcap%
\pgfsetmiterjoin%
\definecolor{currentfill}{rgb}{0.000000,0.000000,1.000000}%
\pgfsetfillcolor{currentfill}%
\pgfsetlinewidth{0.000000pt}%
\definecolor{currentstroke}{rgb}{0.000000,0.000000,0.000000}%
\pgfsetstrokecolor{currentstroke}%
\pgfsetstrokeopacity{0.000000}%
\pgfsetdash{}{0pt}%
\pgfpathmoveto{\pgfqpoint{3.209252in}{0.613486in}}%
\pgfpathlineto{\pgfqpoint{3.219257in}{0.613486in}}%
\pgfpathlineto{\pgfqpoint{3.219257in}{2.611074in}}%
\pgfpathlineto{\pgfqpoint{3.209252in}{2.611074in}}%
\pgfpathlineto{\pgfqpoint{3.209252in}{0.613486in}}%
\pgfpathclose%
\pgfusepath{fill}%
\end{pgfscope}%
\begin{pgfscope}%
\pgfpathrectangle{\pgfqpoint{0.693757in}{0.613486in}}{\pgfqpoint{5.541243in}{3.963477in}}%
\pgfusepath{clip}%
\pgfsetbuttcap%
\pgfsetmiterjoin%
\definecolor{currentfill}{rgb}{0.000000,0.000000,1.000000}%
\pgfsetfillcolor{currentfill}%
\pgfsetlinewidth{0.000000pt}%
\definecolor{currentstroke}{rgb}{0.000000,0.000000,0.000000}%
\pgfsetstrokecolor{currentstroke}%
\pgfsetstrokeopacity{0.000000}%
\pgfsetdash{}{0pt}%
\pgfpathmoveto{\pgfqpoint{3.221758in}{0.613486in}}%
\pgfpathlineto{\pgfqpoint{3.231763in}{0.613486in}}%
\pgfpathlineto{\pgfqpoint{3.231763in}{2.595224in}}%
\pgfpathlineto{\pgfqpoint{3.221758in}{2.595224in}}%
\pgfpathlineto{\pgfqpoint{3.221758in}{0.613486in}}%
\pgfpathclose%
\pgfusepath{fill}%
\end{pgfscope}%
\begin{pgfscope}%
\pgfpathrectangle{\pgfqpoint{0.693757in}{0.613486in}}{\pgfqpoint{5.541243in}{3.963477in}}%
\pgfusepath{clip}%
\pgfsetbuttcap%
\pgfsetmiterjoin%
\definecolor{currentfill}{rgb}{0.000000,0.000000,1.000000}%
\pgfsetfillcolor{currentfill}%
\pgfsetlinewidth{0.000000pt}%
\definecolor{currentstroke}{rgb}{0.000000,0.000000,0.000000}%
\pgfsetstrokecolor{currentstroke}%
\pgfsetstrokeopacity{0.000000}%
\pgfsetdash{}{0pt}%
\pgfpathmoveto{\pgfqpoint{3.234265in}{0.613486in}}%
\pgfpathlineto{\pgfqpoint{3.244270in}{0.613486in}}%
\pgfpathlineto{\pgfqpoint{3.244270in}{1.612282in}}%
\pgfpathlineto{\pgfqpoint{3.234265in}{1.612282in}}%
\pgfpathlineto{\pgfqpoint{3.234265in}{0.613486in}}%
\pgfpathclose%
\pgfusepath{fill}%
\end{pgfscope}%
\begin{pgfscope}%
\pgfpathrectangle{\pgfqpoint{0.693757in}{0.613486in}}{\pgfqpoint{5.541243in}{3.963477in}}%
\pgfusepath{clip}%
\pgfsetbuttcap%
\pgfsetmiterjoin%
\definecolor{currentfill}{rgb}{0.000000,0.000000,1.000000}%
\pgfsetfillcolor{currentfill}%
\pgfsetlinewidth{0.000000pt}%
\definecolor{currentstroke}{rgb}{0.000000,0.000000,0.000000}%
\pgfsetstrokecolor{currentstroke}%
\pgfsetstrokeopacity{0.000000}%
\pgfsetdash{}{0pt}%
\pgfpathmoveto{\pgfqpoint{3.246771in}{0.613486in}}%
\pgfpathlineto{\pgfqpoint{3.256776in}{0.613486in}}%
\pgfpathlineto{\pgfqpoint{3.256776in}{1.604355in}}%
\pgfpathlineto{\pgfqpoint{3.246771in}{1.604355in}}%
\pgfpathlineto{\pgfqpoint{3.246771in}{0.613486in}}%
\pgfpathclose%
\pgfusepath{fill}%
\end{pgfscope}%
\begin{pgfscope}%
\pgfpathrectangle{\pgfqpoint{0.693757in}{0.613486in}}{\pgfqpoint{5.541243in}{3.963477in}}%
\pgfusepath{clip}%
\pgfsetbuttcap%
\pgfsetmiterjoin%
\definecolor{currentfill}{rgb}{0.000000,0.000000,1.000000}%
\pgfsetfillcolor{currentfill}%
\pgfsetlinewidth{0.000000pt}%
\definecolor{currentstroke}{rgb}{0.000000,0.000000,0.000000}%
\pgfsetstrokecolor{currentstroke}%
\pgfsetstrokeopacity{0.000000}%
\pgfsetdash{}{0pt}%
\pgfpathmoveto{\pgfqpoint{3.259277in}{0.613486in}}%
\pgfpathlineto{\pgfqpoint{3.269282in}{0.613486in}}%
\pgfpathlineto{\pgfqpoint{3.269282in}{2.611074in}}%
\pgfpathlineto{\pgfqpoint{3.259277in}{2.611074in}}%
\pgfpathlineto{\pgfqpoint{3.259277in}{0.613486in}}%
\pgfpathclose%
\pgfusepath{fill}%
\end{pgfscope}%
\begin{pgfscope}%
\pgfpathrectangle{\pgfqpoint{0.693757in}{0.613486in}}{\pgfqpoint{5.541243in}{3.963477in}}%
\pgfusepath{clip}%
\pgfsetbuttcap%
\pgfsetmiterjoin%
\definecolor{currentfill}{rgb}{0.000000,0.000000,1.000000}%
\pgfsetfillcolor{currentfill}%
\pgfsetlinewidth{0.000000pt}%
\definecolor{currentstroke}{rgb}{0.000000,0.000000,0.000000}%
\pgfsetstrokecolor{currentstroke}%
\pgfsetstrokeopacity{0.000000}%
\pgfsetdash{}{0pt}%
\pgfpathmoveto{\pgfqpoint{3.271783in}{0.613486in}}%
\pgfpathlineto{\pgfqpoint{3.281788in}{0.613486in}}%
\pgfpathlineto{\pgfqpoint{3.281788in}{2.595224in}}%
\pgfpathlineto{\pgfqpoint{3.271783in}{2.595224in}}%
\pgfpathlineto{\pgfqpoint{3.271783in}{0.613486in}}%
\pgfpathclose%
\pgfusepath{fill}%
\end{pgfscope}%
\begin{pgfscope}%
\pgfpathrectangle{\pgfqpoint{0.693757in}{0.613486in}}{\pgfqpoint{5.541243in}{3.963477in}}%
\pgfusepath{clip}%
\pgfsetbuttcap%
\pgfsetmiterjoin%
\definecolor{currentfill}{rgb}{0.000000,0.000000,1.000000}%
\pgfsetfillcolor{currentfill}%
\pgfsetlinewidth{0.000000pt}%
\definecolor{currentstroke}{rgb}{0.000000,0.000000,0.000000}%
\pgfsetstrokecolor{currentstroke}%
\pgfsetstrokeopacity{0.000000}%
\pgfsetdash{}{0pt}%
\pgfpathmoveto{\pgfqpoint{3.284289in}{0.613486in}}%
\pgfpathlineto{\pgfqpoint{3.294294in}{0.613486in}}%
\pgfpathlineto{\pgfqpoint{3.294294in}{1.612282in}}%
\pgfpathlineto{\pgfqpoint{3.284289in}{1.612282in}}%
\pgfpathlineto{\pgfqpoint{3.284289in}{0.613486in}}%
\pgfpathclose%
\pgfusepath{fill}%
\end{pgfscope}%
\begin{pgfscope}%
\pgfpathrectangle{\pgfqpoint{0.693757in}{0.613486in}}{\pgfqpoint{5.541243in}{3.963477in}}%
\pgfusepath{clip}%
\pgfsetbuttcap%
\pgfsetmiterjoin%
\definecolor{currentfill}{rgb}{0.000000,0.000000,1.000000}%
\pgfsetfillcolor{currentfill}%
\pgfsetlinewidth{0.000000pt}%
\definecolor{currentstroke}{rgb}{0.000000,0.000000,0.000000}%
\pgfsetstrokecolor{currentstroke}%
\pgfsetstrokeopacity{0.000000}%
\pgfsetdash{}{0pt}%
\pgfpathmoveto{\pgfqpoint{3.296796in}{0.613486in}}%
\pgfpathlineto{\pgfqpoint{3.306801in}{0.613486in}}%
\pgfpathlineto{\pgfqpoint{3.306801in}{1.604355in}}%
\pgfpathlineto{\pgfqpoint{3.296796in}{1.604355in}}%
\pgfpathlineto{\pgfqpoint{3.296796in}{0.613486in}}%
\pgfpathclose%
\pgfusepath{fill}%
\end{pgfscope}%
\begin{pgfscope}%
\pgfpathrectangle{\pgfqpoint{0.693757in}{0.613486in}}{\pgfqpoint{5.541243in}{3.963477in}}%
\pgfusepath{clip}%
\pgfsetbuttcap%
\pgfsetmiterjoin%
\definecolor{currentfill}{rgb}{0.000000,0.000000,1.000000}%
\pgfsetfillcolor{currentfill}%
\pgfsetlinewidth{0.000000pt}%
\definecolor{currentstroke}{rgb}{0.000000,0.000000,0.000000}%
\pgfsetstrokecolor{currentstroke}%
\pgfsetstrokeopacity{0.000000}%
\pgfsetdash{}{0pt}%
\pgfpathmoveto{\pgfqpoint{3.309302in}{0.613486in}}%
\pgfpathlineto{\pgfqpoint{3.319307in}{0.613486in}}%
\pgfpathlineto{\pgfqpoint{3.319307in}{2.611074in}}%
\pgfpathlineto{\pgfqpoint{3.309302in}{2.611074in}}%
\pgfpathlineto{\pgfqpoint{3.309302in}{0.613486in}}%
\pgfpathclose%
\pgfusepath{fill}%
\end{pgfscope}%
\begin{pgfscope}%
\pgfpathrectangle{\pgfqpoint{0.693757in}{0.613486in}}{\pgfqpoint{5.541243in}{3.963477in}}%
\pgfusepath{clip}%
\pgfsetbuttcap%
\pgfsetmiterjoin%
\definecolor{currentfill}{rgb}{0.000000,0.000000,1.000000}%
\pgfsetfillcolor{currentfill}%
\pgfsetlinewidth{0.000000pt}%
\definecolor{currentstroke}{rgb}{0.000000,0.000000,0.000000}%
\pgfsetstrokecolor{currentstroke}%
\pgfsetstrokeopacity{0.000000}%
\pgfsetdash{}{0pt}%
\pgfpathmoveto{\pgfqpoint{3.321808in}{0.613486in}}%
\pgfpathlineto{\pgfqpoint{3.331813in}{0.613486in}}%
\pgfpathlineto{\pgfqpoint{3.331813in}{2.595224in}}%
\pgfpathlineto{\pgfqpoint{3.321808in}{2.595224in}}%
\pgfpathlineto{\pgfqpoint{3.321808in}{0.613486in}}%
\pgfpathclose%
\pgfusepath{fill}%
\end{pgfscope}%
\begin{pgfscope}%
\pgfpathrectangle{\pgfqpoint{0.693757in}{0.613486in}}{\pgfqpoint{5.541243in}{3.963477in}}%
\pgfusepath{clip}%
\pgfsetbuttcap%
\pgfsetmiterjoin%
\definecolor{currentfill}{rgb}{0.000000,0.000000,1.000000}%
\pgfsetfillcolor{currentfill}%
\pgfsetlinewidth{0.000000pt}%
\definecolor{currentstroke}{rgb}{0.000000,0.000000,0.000000}%
\pgfsetstrokecolor{currentstroke}%
\pgfsetstrokeopacity{0.000000}%
\pgfsetdash{}{0pt}%
\pgfpathmoveto{\pgfqpoint{3.334314in}{0.613486in}}%
\pgfpathlineto{\pgfqpoint{3.344319in}{0.613486in}}%
\pgfpathlineto{\pgfqpoint{3.344319in}{1.612282in}}%
\pgfpathlineto{\pgfqpoint{3.334314in}{1.612282in}}%
\pgfpathlineto{\pgfqpoint{3.334314in}{0.613486in}}%
\pgfpathclose%
\pgfusepath{fill}%
\end{pgfscope}%
\begin{pgfscope}%
\pgfpathrectangle{\pgfqpoint{0.693757in}{0.613486in}}{\pgfqpoint{5.541243in}{3.963477in}}%
\pgfusepath{clip}%
\pgfsetbuttcap%
\pgfsetmiterjoin%
\definecolor{currentfill}{rgb}{0.000000,0.000000,1.000000}%
\pgfsetfillcolor{currentfill}%
\pgfsetlinewidth{0.000000pt}%
\definecolor{currentstroke}{rgb}{0.000000,0.000000,0.000000}%
\pgfsetstrokecolor{currentstroke}%
\pgfsetstrokeopacity{0.000000}%
\pgfsetdash{}{0pt}%
\pgfpathmoveto{\pgfqpoint{3.346820in}{0.613486in}}%
\pgfpathlineto{\pgfqpoint{3.356825in}{0.613486in}}%
\pgfpathlineto{\pgfqpoint{3.356825in}{1.604355in}}%
\pgfpathlineto{\pgfqpoint{3.346820in}{1.604355in}}%
\pgfpathlineto{\pgfqpoint{3.346820in}{0.613486in}}%
\pgfpathclose%
\pgfusepath{fill}%
\end{pgfscope}%
\begin{pgfscope}%
\pgfpathrectangle{\pgfqpoint{0.693757in}{0.613486in}}{\pgfqpoint{5.541243in}{3.963477in}}%
\pgfusepath{clip}%
\pgfsetbuttcap%
\pgfsetmiterjoin%
\definecolor{currentfill}{rgb}{0.000000,0.000000,1.000000}%
\pgfsetfillcolor{currentfill}%
\pgfsetlinewidth{0.000000pt}%
\definecolor{currentstroke}{rgb}{0.000000,0.000000,0.000000}%
\pgfsetstrokecolor{currentstroke}%
\pgfsetstrokeopacity{0.000000}%
\pgfsetdash{}{0pt}%
\pgfpathmoveto{\pgfqpoint{3.359327in}{0.613486in}}%
\pgfpathlineto{\pgfqpoint{3.369331in}{0.613486in}}%
\pgfpathlineto{\pgfqpoint{3.369331in}{2.611074in}}%
\pgfpathlineto{\pgfqpoint{3.359327in}{2.611074in}}%
\pgfpathlineto{\pgfqpoint{3.359327in}{0.613486in}}%
\pgfpathclose%
\pgfusepath{fill}%
\end{pgfscope}%
\begin{pgfscope}%
\pgfpathrectangle{\pgfqpoint{0.693757in}{0.613486in}}{\pgfqpoint{5.541243in}{3.963477in}}%
\pgfusepath{clip}%
\pgfsetbuttcap%
\pgfsetmiterjoin%
\definecolor{currentfill}{rgb}{0.000000,0.000000,1.000000}%
\pgfsetfillcolor{currentfill}%
\pgfsetlinewidth{0.000000pt}%
\definecolor{currentstroke}{rgb}{0.000000,0.000000,0.000000}%
\pgfsetstrokecolor{currentstroke}%
\pgfsetstrokeopacity{0.000000}%
\pgfsetdash{}{0pt}%
\pgfpathmoveto{\pgfqpoint{3.371833in}{0.613486in}}%
\pgfpathlineto{\pgfqpoint{3.381838in}{0.613486in}}%
\pgfpathlineto{\pgfqpoint{3.381838in}{2.595224in}}%
\pgfpathlineto{\pgfqpoint{3.371833in}{2.595224in}}%
\pgfpathlineto{\pgfqpoint{3.371833in}{0.613486in}}%
\pgfpathclose%
\pgfusepath{fill}%
\end{pgfscope}%
\begin{pgfscope}%
\pgfpathrectangle{\pgfqpoint{0.693757in}{0.613486in}}{\pgfqpoint{5.541243in}{3.963477in}}%
\pgfusepath{clip}%
\pgfsetbuttcap%
\pgfsetmiterjoin%
\definecolor{currentfill}{rgb}{0.000000,0.000000,1.000000}%
\pgfsetfillcolor{currentfill}%
\pgfsetlinewidth{0.000000pt}%
\definecolor{currentstroke}{rgb}{0.000000,0.000000,0.000000}%
\pgfsetstrokecolor{currentstroke}%
\pgfsetstrokeopacity{0.000000}%
\pgfsetdash{}{0pt}%
\pgfpathmoveto{\pgfqpoint{3.384339in}{0.613486in}}%
\pgfpathlineto{\pgfqpoint{3.394344in}{0.613486in}}%
\pgfpathlineto{\pgfqpoint{3.394344in}{1.612282in}}%
\pgfpathlineto{\pgfqpoint{3.384339in}{1.612282in}}%
\pgfpathlineto{\pgfqpoint{3.384339in}{0.613486in}}%
\pgfpathclose%
\pgfusepath{fill}%
\end{pgfscope}%
\begin{pgfscope}%
\pgfpathrectangle{\pgfqpoint{0.693757in}{0.613486in}}{\pgfqpoint{5.541243in}{3.963477in}}%
\pgfusepath{clip}%
\pgfsetbuttcap%
\pgfsetmiterjoin%
\definecolor{currentfill}{rgb}{0.000000,0.000000,1.000000}%
\pgfsetfillcolor{currentfill}%
\pgfsetlinewidth{0.000000pt}%
\definecolor{currentstroke}{rgb}{0.000000,0.000000,0.000000}%
\pgfsetstrokecolor{currentstroke}%
\pgfsetstrokeopacity{0.000000}%
\pgfsetdash{}{0pt}%
\pgfpathmoveto{\pgfqpoint{3.396845in}{0.613486in}}%
\pgfpathlineto{\pgfqpoint{3.406850in}{0.613486in}}%
\pgfpathlineto{\pgfqpoint{3.406850in}{1.604355in}}%
\pgfpathlineto{\pgfqpoint{3.396845in}{1.604355in}}%
\pgfpathlineto{\pgfqpoint{3.396845in}{0.613486in}}%
\pgfpathclose%
\pgfusepath{fill}%
\end{pgfscope}%
\begin{pgfscope}%
\pgfpathrectangle{\pgfqpoint{0.693757in}{0.613486in}}{\pgfqpoint{5.541243in}{3.963477in}}%
\pgfusepath{clip}%
\pgfsetbuttcap%
\pgfsetmiterjoin%
\definecolor{currentfill}{rgb}{0.000000,0.000000,1.000000}%
\pgfsetfillcolor{currentfill}%
\pgfsetlinewidth{0.000000pt}%
\definecolor{currentstroke}{rgb}{0.000000,0.000000,0.000000}%
\pgfsetstrokecolor{currentstroke}%
\pgfsetstrokeopacity{0.000000}%
\pgfsetdash{}{0pt}%
\pgfpathmoveto{\pgfqpoint{3.409351in}{0.613486in}}%
\pgfpathlineto{\pgfqpoint{3.419356in}{0.613486in}}%
\pgfpathlineto{\pgfqpoint{3.419356in}{2.611074in}}%
\pgfpathlineto{\pgfqpoint{3.409351in}{2.611074in}}%
\pgfpathlineto{\pgfqpoint{3.409351in}{0.613486in}}%
\pgfpathclose%
\pgfusepath{fill}%
\end{pgfscope}%
\begin{pgfscope}%
\pgfpathrectangle{\pgfqpoint{0.693757in}{0.613486in}}{\pgfqpoint{5.541243in}{3.963477in}}%
\pgfusepath{clip}%
\pgfsetbuttcap%
\pgfsetmiterjoin%
\definecolor{currentfill}{rgb}{0.000000,0.000000,1.000000}%
\pgfsetfillcolor{currentfill}%
\pgfsetlinewidth{0.000000pt}%
\definecolor{currentstroke}{rgb}{0.000000,0.000000,0.000000}%
\pgfsetstrokecolor{currentstroke}%
\pgfsetstrokeopacity{0.000000}%
\pgfsetdash{}{0pt}%
\pgfpathmoveto{\pgfqpoint{3.421857in}{0.613486in}}%
\pgfpathlineto{\pgfqpoint{3.431862in}{0.613486in}}%
\pgfpathlineto{\pgfqpoint{3.431862in}{2.595224in}}%
\pgfpathlineto{\pgfqpoint{3.421857in}{2.595224in}}%
\pgfpathlineto{\pgfqpoint{3.421857in}{0.613486in}}%
\pgfpathclose%
\pgfusepath{fill}%
\end{pgfscope}%
\begin{pgfscope}%
\pgfpathrectangle{\pgfqpoint{0.693757in}{0.613486in}}{\pgfqpoint{5.541243in}{3.963477in}}%
\pgfusepath{clip}%
\pgfsetbuttcap%
\pgfsetmiterjoin%
\definecolor{currentfill}{rgb}{0.000000,0.000000,1.000000}%
\pgfsetfillcolor{currentfill}%
\pgfsetlinewidth{0.000000pt}%
\definecolor{currentstroke}{rgb}{0.000000,0.000000,0.000000}%
\pgfsetstrokecolor{currentstroke}%
\pgfsetstrokeopacity{0.000000}%
\pgfsetdash{}{0pt}%
\pgfpathmoveto{\pgfqpoint{3.434364in}{0.613486in}}%
\pgfpathlineto{\pgfqpoint{3.444369in}{0.613486in}}%
\pgfpathlineto{\pgfqpoint{3.444369in}{1.612282in}}%
\pgfpathlineto{\pgfqpoint{3.434364in}{1.612282in}}%
\pgfpathlineto{\pgfqpoint{3.434364in}{0.613486in}}%
\pgfpathclose%
\pgfusepath{fill}%
\end{pgfscope}%
\begin{pgfscope}%
\pgfpathrectangle{\pgfqpoint{0.693757in}{0.613486in}}{\pgfqpoint{5.541243in}{3.963477in}}%
\pgfusepath{clip}%
\pgfsetbuttcap%
\pgfsetmiterjoin%
\definecolor{currentfill}{rgb}{0.000000,0.000000,1.000000}%
\pgfsetfillcolor{currentfill}%
\pgfsetlinewidth{0.000000pt}%
\definecolor{currentstroke}{rgb}{0.000000,0.000000,0.000000}%
\pgfsetstrokecolor{currentstroke}%
\pgfsetstrokeopacity{0.000000}%
\pgfsetdash{}{0pt}%
\pgfpathmoveto{\pgfqpoint{3.446870in}{0.613486in}}%
\pgfpathlineto{\pgfqpoint{3.456875in}{0.613486in}}%
\pgfpathlineto{\pgfqpoint{3.456875in}{1.604355in}}%
\pgfpathlineto{\pgfqpoint{3.446870in}{1.604355in}}%
\pgfpathlineto{\pgfqpoint{3.446870in}{0.613486in}}%
\pgfpathclose%
\pgfusepath{fill}%
\end{pgfscope}%
\begin{pgfscope}%
\pgfpathrectangle{\pgfqpoint{0.693757in}{0.613486in}}{\pgfqpoint{5.541243in}{3.963477in}}%
\pgfusepath{clip}%
\pgfsetbuttcap%
\pgfsetmiterjoin%
\definecolor{currentfill}{rgb}{0.000000,0.000000,1.000000}%
\pgfsetfillcolor{currentfill}%
\pgfsetlinewidth{0.000000pt}%
\definecolor{currentstroke}{rgb}{0.000000,0.000000,0.000000}%
\pgfsetstrokecolor{currentstroke}%
\pgfsetstrokeopacity{0.000000}%
\pgfsetdash{}{0pt}%
\pgfpathmoveto{\pgfqpoint{3.459376in}{0.613486in}}%
\pgfpathlineto{\pgfqpoint{3.469381in}{0.613486in}}%
\pgfpathlineto{\pgfqpoint{3.469381in}{2.611074in}}%
\pgfpathlineto{\pgfqpoint{3.459376in}{2.611074in}}%
\pgfpathlineto{\pgfqpoint{3.459376in}{0.613486in}}%
\pgfpathclose%
\pgfusepath{fill}%
\end{pgfscope}%
\begin{pgfscope}%
\pgfpathrectangle{\pgfqpoint{0.693757in}{0.613486in}}{\pgfqpoint{5.541243in}{3.963477in}}%
\pgfusepath{clip}%
\pgfsetbuttcap%
\pgfsetmiterjoin%
\definecolor{currentfill}{rgb}{0.000000,0.000000,1.000000}%
\pgfsetfillcolor{currentfill}%
\pgfsetlinewidth{0.000000pt}%
\definecolor{currentstroke}{rgb}{0.000000,0.000000,0.000000}%
\pgfsetstrokecolor{currentstroke}%
\pgfsetstrokeopacity{0.000000}%
\pgfsetdash{}{0pt}%
\pgfpathmoveto{\pgfqpoint{3.471882in}{0.613486in}}%
\pgfpathlineto{\pgfqpoint{3.481887in}{0.613486in}}%
\pgfpathlineto{\pgfqpoint{3.481887in}{2.595224in}}%
\pgfpathlineto{\pgfqpoint{3.471882in}{2.595224in}}%
\pgfpathlineto{\pgfqpoint{3.471882in}{0.613486in}}%
\pgfpathclose%
\pgfusepath{fill}%
\end{pgfscope}%
\begin{pgfscope}%
\pgfpathrectangle{\pgfqpoint{0.693757in}{0.613486in}}{\pgfqpoint{5.541243in}{3.963477in}}%
\pgfusepath{clip}%
\pgfsetbuttcap%
\pgfsetmiterjoin%
\definecolor{currentfill}{rgb}{0.000000,0.000000,1.000000}%
\pgfsetfillcolor{currentfill}%
\pgfsetlinewidth{0.000000pt}%
\definecolor{currentstroke}{rgb}{0.000000,0.000000,0.000000}%
\pgfsetstrokecolor{currentstroke}%
\pgfsetstrokeopacity{0.000000}%
\pgfsetdash{}{0pt}%
\pgfpathmoveto{\pgfqpoint{3.484388in}{0.613486in}}%
\pgfpathlineto{\pgfqpoint{3.494393in}{0.613486in}}%
\pgfpathlineto{\pgfqpoint{3.494393in}{1.612282in}}%
\pgfpathlineto{\pgfqpoint{3.484388in}{1.612282in}}%
\pgfpathlineto{\pgfqpoint{3.484388in}{0.613486in}}%
\pgfpathclose%
\pgfusepath{fill}%
\end{pgfscope}%
\begin{pgfscope}%
\pgfpathrectangle{\pgfqpoint{0.693757in}{0.613486in}}{\pgfqpoint{5.541243in}{3.963477in}}%
\pgfusepath{clip}%
\pgfsetbuttcap%
\pgfsetmiterjoin%
\definecolor{currentfill}{rgb}{0.000000,0.000000,1.000000}%
\pgfsetfillcolor{currentfill}%
\pgfsetlinewidth{0.000000pt}%
\definecolor{currentstroke}{rgb}{0.000000,0.000000,0.000000}%
\pgfsetstrokecolor{currentstroke}%
\pgfsetstrokeopacity{0.000000}%
\pgfsetdash{}{0pt}%
\pgfpathmoveto{\pgfqpoint{3.496895in}{0.613486in}}%
\pgfpathlineto{\pgfqpoint{3.506900in}{0.613486in}}%
\pgfpathlineto{\pgfqpoint{3.506900in}{1.604355in}}%
\pgfpathlineto{\pgfqpoint{3.496895in}{1.604355in}}%
\pgfpathlineto{\pgfqpoint{3.496895in}{0.613486in}}%
\pgfpathclose%
\pgfusepath{fill}%
\end{pgfscope}%
\begin{pgfscope}%
\pgfpathrectangle{\pgfqpoint{0.693757in}{0.613486in}}{\pgfqpoint{5.541243in}{3.963477in}}%
\pgfusepath{clip}%
\pgfsetbuttcap%
\pgfsetmiterjoin%
\definecolor{currentfill}{rgb}{0.000000,0.000000,1.000000}%
\pgfsetfillcolor{currentfill}%
\pgfsetlinewidth{0.000000pt}%
\definecolor{currentstroke}{rgb}{0.000000,0.000000,0.000000}%
\pgfsetstrokecolor{currentstroke}%
\pgfsetstrokeopacity{0.000000}%
\pgfsetdash{}{0pt}%
\pgfpathmoveto{\pgfqpoint{3.509401in}{0.613486in}}%
\pgfpathlineto{\pgfqpoint{3.519406in}{0.613486in}}%
\pgfpathlineto{\pgfqpoint{3.519406in}{2.611074in}}%
\pgfpathlineto{\pgfqpoint{3.509401in}{2.611074in}}%
\pgfpathlineto{\pgfqpoint{3.509401in}{0.613486in}}%
\pgfpathclose%
\pgfusepath{fill}%
\end{pgfscope}%
\begin{pgfscope}%
\pgfpathrectangle{\pgfqpoint{0.693757in}{0.613486in}}{\pgfqpoint{5.541243in}{3.963477in}}%
\pgfusepath{clip}%
\pgfsetbuttcap%
\pgfsetmiterjoin%
\definecolor{currentfill}{rgb}{0.000000,0.000000,1.000000}%
\pgfsetfillcolor{currentfill}%
\pgfsetlinewidth{0.000000pt}%
\definecolor{currentstroke}{rgb}{0.000000,0.000000,0.000000}%
\pgfsetstrokecolor{currentstroke}%
\pgfsetstrokeopacity{0.000000}%
\pgfsetdash{}{0pt}%
\pgfpathmoveto{\pgfqpoint{3.521907in}{0.613486in}}%
\pgfpathlineto{\pgfqpoint{3.531912in}{0.613486in}}%
\pgfpathlineto{\pgfqpoint{3.531912in}{2.595224in}}%
\pgfpathlineto{\pgfqpoint{3.521907in}{2.595224in}}%
\pgfpathlineto{\pgfqpoint{3.521907in}{0.613486in}}%
\pgfpathclose%
\pgfusepath{fill}%
\end{pgfscope}%
\begin{pgfscope}%
\pgfpathrectangle{\pgfqpoint{0.693757in}{0.613486in}}{\pgfqpoint{5.541243in}{3.963477in}}%
\pgfusepath{clip}%
\pgfsetbuttcap%
\pgfsetmiterjoin%
\definecolor{currentfill}{rgb}{0.000000,0.000000,1.000000}%
\pgfsetfillcolor{currentfill}%
\pgfsetlinewidth{0.000000pt}%
\definecolor{currentstroke}{rgb}{0.000000,0.000000,0.000000}%
\pgfsetstrokecolor{currentstroke}%
\pgfsetstrokeopacity{0.000000}%
\pgfsetdash{}{0pt}%
\pgfpathmoveto{\pgfqpoint{3.534413in}{0.613486in}}%
\pgfpathlineto{\pgfqpoint{3.544418in}{0.613486in}}%
\pgfpathlineto{\pgfqpoint{3.544418in}{1.612282in}}%
\pgfpathlineto{\pgfqpoint{3.534413in}{1.612282in}}%
\pgfpathlineto{\pgfqpoint{3.534413in}{0.613486in}}%
\pgfpathclose%
\pgfusepath{fill}%
\end{pgfscope}%
\begin{pgfscope}%
\pgfpathrectangle{\pgfqpoint{0.693757in}{0.613486in}}{\pgfqpoint{5.541243in}{3.963477in}}%
\pgfusepath{clip}%
\pgfsetbuttcap%
\pgfsetmiterjoin%
\definecolor{currentfill}{rgb}{0.000000,0.000000,1.000000}%
\pgfsetfillcolor{currentfill}%
\pgfsetlinewidth{0.000000pt}%
\definecolor{currentstroke}{rgb}{0.000000,0.000000,0.000000}%
\pgfsetstrokecolor{currentstroke}%
\pgfsetstrokeopacity{0.000000}%
\pgfsetdash{}{0pt}%
\pgfpathmoveto{\pgfqpoint{3.546919in}{0.613486in}}%
\pgfpathlineto{\pgfqpoint{3.556924in}{0.613486in}}%
\pgfpathlineto{\pgfqpoint{3.556924in}{1.604355in}}%
\pgfpathlineto{\pgfqpoint{3.546919in}{1.604355in}}%
\pgfpathlineto{\pgfqpoint{3.546919in}{0.613486in}}%
\pgfpathclose%
\pgfusepath{fill}%
\end{pgfscope}%
\begin{pgfscope}%
\pgfpathrectangle{\pgfqpoint{0.693757in}{0.613486in}}{\pgfqpoint{5.541243in}{3.963477in}}%
\pgfusepath{clip}%
\pgfsetbuttcap%
\pgfsetmiterjoin%
\definecolor{currentfill}{rgb}{0.000000,0.000000,1.000000}%
\pgfsetfillcolor{currentfill}%
\pgfsetlinewidth{0.000000pt}%
\definecolor{currentstroke}{rgb}{0.000000,0.000000,0.000000}%
\pgfsetstrokecolor{currentstroke}%
\pgfsetstrokeopacity{0.000000}%
\pgfsetdash{}{0pt}%
\pgfpathmoveto{\pgfqpoint{3.559426in}{0.613486in}}%
\pgfpathlineto{\pgfqpoint{3.569431in}{0.613486in}}%
\pgfpathlineto{\pgfqpoint{3.569431in}{2.611074in}}%
\pgfpathlineto{\pgfqpoint{3.559426in}{2.611074in}}%
\pgfpathlineto{\pgfqpoint{3.559426in}{0.613486in}}%
\pgfpathclose%
\pgfusepath{fill}%
\end{pgfscope}%
\begin{pgfscope}%
\pgfpathrectangle{\pgfqpoint{0.693757in}{0.613486in}}{\pgfqpoint{5.541243in}{3.963477in}}%
\pgfusepath{clip}%
\pgfsetbuttcap%
\pgfsetmiterjoin%
\definecolor{currentfill}{rgb}{0.000000,0.000000,1.000000}%
\pgfsetfillcolor{currentfill}%
\pgfsetlinewidth{0.000000pt}%
\definecolor{currentstroke}{rgb}{0.000000,0.000000,0.000000}%
\pgfsetstrokecolor{currentstroke}%
\pgfsetstrokeopacity{0.000000}%
\pgfsetdash{}{0pt}%
\pgfpathmoveto{\pgfqpoint{3.571932in}{0.613486in}}%
\pgfpathlineto{\pgfqpoint{3.581937in}{0.613486in}}%
\pgfpathlineto{\pgfqpoint{3.581937in}{2.595224in}}%
\pgfpathlineto{\pgfqpoint{3.571932in}{2.595224in}}%
\pgfpathlineto{\pgfqpoint{3.571932in}{0.613486in}}%
\pgfpathclose%
\pgfusepath{fill}%
\end{pgfscope}%
\begin{pgfscope}%
\pgfpathrectangle{\pgfqpoint{0.693757in}{0.613486in}}{\pgfqpoint{5.541243in}{3.963477in}}%
\pgfusepath{clip}%
\pgfsetbuttcap%
\pgfsetmiterjoin%
\definecolor{currentfill}{rgb}{0.000000,0.000000,1.000000}%
\pgfsetfillcolor{currentfill}%
\pgfsetlinewidth{0.000000pt}%
\definecolor{currentstroke}{rgb}{0.000000,0.000000,0.000000}%
\pgfsetstrokecolor{currentstroke}%
\pgfsetstrokeopacity{0.000000}%
\pgfsetdash{}{0pt}%
\pgfpathmoveto{\pgfqpoint{3.584438in}{0.613486in}}%
\pgfpathlineto{\pgfqpoint{3.594443in}{0.613486in}}%
\pgfpathlineto{\pgfqpoint{3.594443in}{1.612282in}}%
\pgfpathlineto{\pgfqpoint{3.584438in}{1.612282in}}%
\pgfpathlineto{\pgfqpoint{3.584438in}{0.613486in}}%
\pgfpathclose%
\pgfusepath{fill}%
\end{pgfscope}%
\begin{pgfscope}%
\pgfpathrectangle{\pgfqpoint{0.693757in}{0.613486in}}{\pgfqpoint{5.541243in}{3.963477in}}%
\pgfusepath{clip}%
\pgfsetbuttcap%
\pgfsetmiterjoin%
\definecolor{currentfill}{rgb}{0.000000,0.000000,1.000000}%
\pgfsetfillcolor{currentfill}%
\pgfsetlinewidth{0.000000pt}%
\definecolor{currentstroke}{rgb}{0.000000,0.000000,0.000000}%
\pgfsetstrokecolor{currentstroke}%
\pgfsetstrokeopacity{0.000000}%
\pgfsetdash{}{0pt}%
\pgfpathmoveto{\pgfqpoint{3.596944in}{0.613486in}}%
\pgfpathlineto{\pgfqpoint{3.606949in}{0.613486in}}%
\pgfpathlineto{\pgfqpoint{3.606949in}{1.604355in}}%
\pgfpathlineto{\pgfqpoint{3.596944in}{1.604355in}}%
\pgfpathlineto{\pgfqpoint{3.596944in}{0.613486in}}%
\pgfpathclose%
\pgfusepath{fill}%
\end{pgfscope}%
\begin{pgfscope}%
\pgfpathrectangle{\pgfqpoint{0.693757in}{0.613486in}}{\pgfqpoint{5.541243in}{3.963477in}}%
\pgfusepath{clip}%
\pgfsetbuttcap%
\pgfsetmiterjoin%
\definecolor{currentfill}{rgb}{0.000000,0.000000,1.000000}%
\pgfsetfillcolor{currentfill}%
\pgfsetlinewidth{0.000000pt}%
\definecolor{currentstroke}{rgb}{0.000000,0.000000,0.000000}%
\pgfsetstrokecolor{currentstroke}%
\pgfsetstrokeopacity{0.000000}%
\pgfsetdash{}{0pt}%
\pgfpathmoveto{\pgfqpoint{3.609450in}{0.613486in}}%
\pgfpathlineto{\pgfqpoint{3.619455in}{0.613486in}}%
\pgfpathlineto{\pgfqpoint{3.619455in}{2.611074in}}%
\pgfpathlineto{\pgfqpoint{3.609450in}{2.611074in}}%
\pgfpathlineto{\pgfqpoint{3.609450in}{0.613486in}}%
\pgfpathclose%
\pgfusepath{fill}%
\end{pgfscope}%
\begin{pgfscope}%
\pgfpathrectangle{\pgfqpoint{0.693757in}{0.613486in}}{\pgfqpoint{5.541243in}{3.963477in}}%
\pgfusepath{clip}%
\pgfsetbuttcap%
\pgfsetmiterjoin%
\definecolor{currentfill}{rgb}{0.000000,0.000000,1.000000}%
\pgfsetfillcolor{currentfill}%
\pgfsetlinewidth{0.000000pt}%
\definecolor{currentstroke}{rgb}{0.000000,0.000000,0.000000}%
\pgfsetstrokecolor{currentstroke}%
\pgfsetstrokeopacity{0.000000}%
\pgfsetdash{}{0pt}%
\pgfpathmoveto{\pgfqpoint{3.621957in}{0.613486in}}%
\pgfpathlineto{\pgfqpoint{3.631961in}{0.613486in}}%
\pgfpathlineto{\pgfqpoint{3.631961in}{2.595224in}}%
\pgfpathlineto{\pgfqpoint{3.621957in}{2.595224in}}%
\pgfpathlineto{\pgfqpoint{3.621957in}{0.613486in}}%
\pgfpathclose%
\pgfusepath{fill}%
\end{pgfscope}%
\begin{pgfscope}%
\pgfpathrectangle{\pgfqpoint{0.693757in}{0.613486in}}{\pgfqpoint{5.541243in}{3.963477in}}%
\pgfusepath{clip}%
\pgfsetbuttcap%
\pgfsetmiterjoin%
\definecolor{currentfill}{rgb}{0.000000,0.000000,1.000000}%
\pgfsetfillcolor{currentfill}%
\pgfsetlinewidth{0.000000pt}%
\definecolor{currentstroke}{rgb}{0.000000,0.000000,0.000000}%
\pgfsetstrokecolor{currentstroke}%
\pgfsetstrokeopacity{0.000000}%
\pgfsetdash{}{0pt}%
\pgfpathmoveto{\pgfqpoint{3.634463in}{0.613486in}}%
\pgfpathlineto{\pgfqpoint{3.644468in}{0.613486in}}%
\pgfpathlineto{\pgfqpoint{3.644468in}{1.612282in}}%
\pgfpathlineto{\pgfqpoint{3.634463in}{1.612282in}}%
\pgfpathlineto{\pgfqpoint{3.634463in}{0.613486in}}%
\pgfpathclose%
\pgfusepath{fill}%
\end{pgfscope}%
\begin{pgfscope}%
\pgfpathrectangle{\pgfqpoint{0.693757in}{0.613486in}}{\pgfqpoint{5.541243in}{3.963477in}}%
\pgfusepath{clip}%
\pgfsetbuttcap%
\pgfsetmiterjoin%
\definecolor{currentfill}{rgb}{0.000000,0.000000,1.000000}%
\pgfsetfillcolor{currentfill}%
\pgfsetlinewidth{0.000000pt}%
\definecolor{currentstroke}{rgb}{0.000000,0.000000,0.000000}%
\pgfsetstrokecolor{currentstroke}%
\pgfsetstrokeopacity{0.000000}%
\pgfsetdash{}{0pt}%
\pgfpathmoveto{\pgfqpoint{3.646969in}{0.613486in}}%
\pgfpathlineto{\pgfqpoint{3.656974in}{0.613486in}}%
\pgfpathlineto{\pgfqpoint{3.656974in}{1.604355in}}%
\pgfpathlineto{\pgfqpoint{3.646969in}{1.604355in}}%
\pgfpathlineto{\pgfqpoint{3.646969in}{0.613486in}}%
\pgfpathclose%
\pgfusepath{fill}%
\end{pgfscope}%
\begin{pgfscope}%
\pgfpathrectangle{\pgfqpoint{0.693757in}{0.613486in}}{\pgfqpoint{5.541243in}{3.963477in}}%
\pgfusepath{clip}%
\pgfsetbuttcap%
\pgfsetmiterjoin%
\definecolor{currentfill}{rgb}{0.000000,0.000000,1.000000}%
\pgfsetfillcolor{currentfill}%
\pgfsetlinewidth{0.000000pt}%
\definecolor{currentstroke}{rgb}{0.000000,0.000000,0.000000}%
\pgfsetstrokecolor{currentstroke}%
\pgfsetstrokeopacity{0.000000}%
\pgfsetdash{}{0pt}%
\pgfpathmoveto{\pgfqpoint{3.659475in}{0.613486in}}%
\pgfpathlineto{\pgfqpoint{3.669480in}{0.613486in}}%
\pgfpathlineto{\pgfqpoint{3.669480in}{2.611074in}}%
\pgfpathlineto{\pgfqpoint{3.659475in}{2.611074in}}%
\pgfpathlineto{\pgfqpoint{3.659475in}{0.613486in}}%
\pgfpathclose%
\pgfusepath{fill}%
\end{pgfscope}%
\begin{pgfscope}%
\pgfpathrectangle{\pgfqpoint{0.693757in}{0.613486in}}{\pgfqpoint{5.541243in}{3.963477in}}%
\pgfusepath{clip}%
\pgfsetbuttcap%
\pgfsetmiterjoin%
\definecolor{currentfill}{rgb}{0.000000,0.000000,1.000000}%
\pgfsetfillcolor{currentfill}%
\pgfsetlinewidth{0.000000pt}%
\definecolor{currentstroke}{rgb}{0.000000,0.000000,0.000000}%
\pgfsetstrokecolor{currentstroke}%
\pgfsetstrokeopacity{0.000000}%
\pgfsetdash{}{0pt}%
\pgfpathmoveto{\pgfqpoint{3.671981in}{0.613486in}}%
\pgfpathlineto{\pgfqpoint{3.681986in}{0.613486in}}%
\pgfpathlineto{\pgfqpoint{3.681986in}{2.595224in}}%
\pgfpathlineto{\pgfqpoint{3.671981in}{2.595224in}}%
\pgfpathlineto{\pgfqpoint{3.671981in}{0.613486in}}%
\pgfpathclose%
\pgfusepath{fill}%
\end{pgfscope}%
\begin{pgfscope}%
\pgfpathrectangle{\pgfqpoint{0.693757in}{0.613486in}}{\pgfqpoint{5.541243in}{3.963477in}}%
\pgfusepath{clip}%
\pgfsetbuttcap%
\pgfsetmiterjoin%
\definecolor{currentfill}{rgb}{0.000000,0.000000,1.000000}%
\pgfsetfillcolor{currentfill}%
\pgfsetlinewidth{0.000000pt}%
\definecolor{currentstroke}{rgb}{0.000000,0.000000,0.000000}%
\pgfsetstrokecolor{currentstroke}%
\pgfsetstrokeopacity{0.000000}%
\pgfsetdash{}{0pt}%
\pgfpathmoveto{\pgfqpoint{3.684487in}{0.613486in}}%
\pgfpathlineto{\pgfqpoint{3.694492in}{0.613486in}}%
\pgfpathlineto{\pgfqpoint{3.694492in}{1.612282in}}%
\pgfpathlineto{\pgfqpoint{3.684487in}{1.612282in}}%
\pgfpathlineto{\pgfqpoint{3.684487in}{0.613486in}}%
\pgfpathclose%
\pgfusepath{fill}%
\end{pgfscope}%
\begin{pgfscope}%
\pgfpathrectangle{\pgfqpoint{0.693757in}{0.613486in}}{\pgfqpoint{5.541243in}{3.963477in}}%
\pgfusepath{clip}%
\pgfsetbuttcap%
\pgfsetmiterjoin%
\definecolor{currentfill}{rgb}{0.000000,0.000000,1.000000}%
\pgfsetfillcolor{currentfill}%
\pgfsetlinewidth{0.000000pt}%
\definecolor{currentstroke}{rgb}{0.000000,0.000000,0.000000}%
\pgfsetstrokecolor{currentstroke}%
\pgfsetstrokeopacity{0.000000}%
\pgfsetdash{}{0pt}%
\pgfpathmoveto{\pgfqpoint{3.696994in}{0.613486in}}%
\pgfpathlineto{\pgfqpoint{3.706999in}{0.613486in}}%
\pgfpathlineto{\pgfqpoint{3.706999in}{1.604355in}}%
\pgfpathlineto{\pgfqpoint{3.696994in}{1.604355in}}%
\pgfpathlineto{\pgfqpoint{3.696994in}{0.613486in}}%
\pgfpathclose%
\pgfusepath{fill}%
\end{pgfscope}%
\begin{pgfscope}%
\pgfpathrectangle{\pgfqpoint{0.693757in}{0.613486in}}{\pgfqpoint{5.541243in}{3.963477in}}%
\pgfusepath{clip}%
\pgfsetbuttcap%
\pgfsetmiterjoin%
\definecolor{currentfill}{rgb}{0.000000,0.000000,1.000000}%
\pgfsetfillcolor{currentfill}%
\pgfsetlinewidth{0.000000pt}%
\definecolor{currentstroke}{rgb}{0.000000,0.000000,0.000000}%
\pgfsetstrokecolor{currentstroke}%
\pgfsetstrokeopacity{0.000000}%
\pgfsetdash{}{0pt}%
\pgfpathmoveto{\pgfqpoint{3.709500in}{0.613486in}}%
\pgfpathlineto{\pgfqpoint{3.719505in}{0.613486in}}%
\pgfpathlineto{\pgfqpoint{3.719505in}{2.611074in}}%
\pgfpathlineto{\pgfqpoint{3.709500in}{2.611074in}}%
\pgfpathlineto{\pgfqpoint{3.709500in}{0.613486in}}%
\pgfpathclose%
\pgfusepath{fill}%
\end{pgfscope}%
\begin{pgfscope}%
\pgfpathrectangle{\pgfqpoint{0.693757in}{0.613486in}}{\pgfqpoint{5.541243in}{3.963477in}}%
\pgfusepath{clip}%
\pgfsetbuttcap%
\pgfsetmiterjoin%
\definecolor{currentfill}{rgb}{0.000000,0.000000,1.000000}%
\pgfsetfillcolor{currentfill}%
\pgfsetlinewidth{0.000000pt}%
\definecolor{currentstroke}{rgb}{0.000000,0.000000,0.000000}%
\pgfsetstrokecolor{currentstroke}%
\pgfsetstrokeopacity{0.000000}%
\pgfsetdash{}{0pt}%
\pgfpathmoveto{\pgfqpoint{3.722006in}{0.613486in}}%
\pgfpathlineto{\pgfqpoint{3.732011in}{0.613486in}}%
\pgfpathlineto{\pgfqpoint{3.732011in}{2.595224in}}%
\pgfpathlineto{\pgfqpoint{3.722006in}{2.595224in}}%
\pgfpathlineto{\pgfqpoint{3.722006in}{0.613486in}}%
\pgfpathclose%
\pgfusepath{fill}%
\end{pgfscope}%
\begin{pgfscope}%
\pgfpathrectangle{\pgfqpoint{0.693757in}{0.613486in}}{\pgfqpoint{5.541243in}{3.963477in}}%
\pgfusepath{clip}%
\pgfsetbuttcap%
\pgfsetmiterjoin%
\definecolor{currentfill}{rgb}{0.000000,0.000000,1.000000}%
\pgfsetfillcolor{currentfill}%
\pgfsetlinewidth{0.000000pt}%
\definecolor{currentstroke}{rgb}{0.000000,0.000000,0.000000}%
\pgfsetstrokecolor{currentstroke}%
\pgfsetstrokeopacity{0.000000}%
\pgfsetdash{}{0pt}%
\pgfpathmoveto{\pgfqpoint{3.734512in}{0.613486in}}%
\pgfpathlineto{\pgfqpoint{3.744517in}{0.613486in}}%
\pgfpathlineto{\pgfqpoint{3.744517in}{1.612282in}}%
\pgfpathlineto{\pgfqpoint{3.734512in}{1.612282in}}%
\pgfpathlineto{\pgfqpoint{3.734512in}{0.613486in}}%
\pgfpathclose%
\pgfusepath{fill}%
\end{pgfscope}%
\begin{pgfscope}%
\pgfpathrectangle{\pgfqpoint{0.693757in}{0.613486in}}{\pgfqpoint{5.541243in}{3.963477in}}%
\pgfusepath{clip}%
\pgfsetbuttcap%
\pgfsetmiterjoin%
\definecolor{currentfill}{rgb}{0.000000,0.000000,1.000000}%
\pgfsetfillcolor{currentfill}%
\pgfsetlinewidth{0.000000pt}%
\definecolor{currentstroke}{rgb}{0.000000,0.000000,0.000000}%
\pgfsetstrokecolor{currentstroke}%
\pgfsetstrokeopacity{0.000000}%
\pgfsetdash{}{0pt}%
\pgfpathmoveto{\pgfqpoint{3.747018in}{0.613486in}}%
\pgfpathlineto{\pgfqpoint{3.757023in}{0.613486in}}%
\pgfpathlineto{\pgfqpoint{3.757023in}{1.604355in}}%
\pgfpathlineto{\pgfqpoint{3.747018in}{1.604355in}}%
\pgfpathlineto{\pgfqpoint{3.747018in}{0.613486in}}%
\pgfpathclose%
\pgfusepath{fill}%
\end{pgfscope}%
\begin{pgfscope}%
\pgfpathrectangle{\pgfqpoint{0.693757in}{0.613486in}}{\pgfqpoint{5.541243in}{3.963477in}}%
\pgfusepath{clip}%
\pgfsetbuttcap%
\pgfsetmiterjoin%
\definecolor{currentfill}{rgb}{0.000000,0.000000,1.000000}%
\pgfsetfillcolor{currentfill}%
\pgfsetlinewidth{0.000000pt}%
\definecolor{currentstroke}{rgb}{0.000000,0.000000,0.000000}%
\pgfsetstrokecolor{currentstroke}%
\pgfsetstrokeopacity{0.000000}%
\pgfsetdash{}{0pt}%
\pgfpathmoveto{\pgfqpoint{3.759525in}{0.613486in}}%
\pgfpathlineto{\pgfqpoint{3.769530in}{0.613486in}}%
\pgfpathlineto{\pgfqpoint{3.769530in}{2.611074in}}%
\pgfpathlineto{\pgfqpoint{3.759525in}{2.611074in}}%
\pgfpathlineto{\pgfqpoint{3.759525in}{0.613486in}}%
\pgfpathclose%
\pgfusepath{fill}%
\end{pgfscope}%
\begin{pgfscope}%
\pgfpathrectangle{\pgfqpoint{0.693757in}{0.613486in}}{\pgfqpoint{5.541243in}{3.963477in}}%
\pgfusepath{clip}%
\pgfsetbuttcap%
\pgfsetmiterjoin%
\definecolor{currentfill}{rgb}{0.000000,0.000000,1.000000}%
\pgfsetfillcolor{currentfill}%
\pgfsetlinewidth{0.000000pt}%
\definecolor{currentstroke}{rgb}{0.000000,0.000000,0.000000}%
\pgfsetstrokecolor{currentstroke}%
\pgfsetstrokeopacity{0.000000}%
\pgfsetdash{}{0pt}%
\pgfpathmoveto{\pgfqpoint{3.772031in}{0.613486in}}%
\pgfpathlineto{\pgfqpoint{3.782036in}{0.613486in}}%
\pgfpathlineto{\pgfqpoint{3.782036in}{2.595224in}}%
\pgfpathlineto{\pgfqpoint{3.772031in}{2.595224in}}%
\pgfpathlineto{\pgfqpoint{3.772031in}{0.613486in}}%
\pgfpathclose%
\pgfusepath{fill}%
\end{pgfscope}%
\begin{pgfscope}%
\pgfpathrectangle{\pgfqpoint{0.693757in}{0.613486in}}{\pgfqpoint{5.541243in}{3.963477in}}%
\pgfusepath{clip}%
\pgfsetbuttcap%
\pgfsetmiterjoin%
\definecolor{currentfill}{rgb}{0.000000,0.000000,1.000000}%
\pgfsetfillcolor{currentfill}%
\pgfsetlinewidth{0.000000pt}%
\definecolor{currentstroke}{rgb}{0.000000,0.000000,0.000000}%
\pgfsetstrokecolor{currentstroke}%
\pgfsetstrokeopacity{0.000000}%
\pgfsetdash{}{0pt}%
\pgfpathmoveto{\pgfqpoint{3.784537in}{0.613486in}}%
\pgfpathlineto{\pgfqpoint{3.794542in}{0.613486in}}%
\pgfpathlineto{\pgfqpoint{3.794542in}{1.612282in}}%
\pgfpathlineto{\pgfqpoint{3.784537in}{1.612282in}}%
\pgfpathlineto{\pgfqpoint{3.784537in}{0.613486in}}%
\pgfpathclose%
\pgfusepath{fill}%
\end{pgfscope}%
\begin{pgfscope}%
\pgfpathrectangle{\pgfqpoint{0.693757in}{0.613486in}}{\pgfqpoint{5.541243in}{3.963477in}}%
\pgfusepath{clip}%
\pgfsetbuttcap%
\pgfsetmiterjoin%
\definecolor{currentfill}{rgb}{0.000000,0.000000,1.000000}%
\pgfsetfillcolor{currentfill}%
\pgfsetlinewidth{0.000000pt}%
\definecolor{currentstroke}{rgb}{0.000000,0.000000,0.000000}%
\pgfsetstrokecolor{currentstroke}%
\pgfsetstrokeopacity{0.000000}%
\pgfsetdash{}{0pt}%
\pgfpathmoveto{\pgfqpoint{3.797043in}{0.613486in}}%
\pgfpathlineto{\pgfqpoint{3.807048in}{0.613486in}}%
\pgfpathlineto{\pgfqpoint{3.807048in}{1.604355in}}%
\pgfpathlineto{\pgfqpoint{3.797043in}{1.604355in}}%
\pgfpathlineto{\pgfqpoint{3.797043in}{0.613486in}}%
\pgfpathclose%
\pgfusepath{fill}%
\end{pgfscope}%
\begin{pgfscope}%
\pgfpathrectangle{\pgfqpoint{0.693757in}{0.613486in}}{\pgfqpoint{5.541243in}{3.963477in}}%
\pgfusepath{clip}%
\pgfsetbuttcap%
\pgfsetmiterjoin%
\definecolor{currentfill}{rgb}{0.000000,0.000000,1.000000}%
\pgfsetfillcolor{currentfill}%
\pgfsetlinewidth{0.000000pt}%
\definecolor{currentstroke}{rgb}{0.000000,0.000000,0.000000}%
\pgfsetstrokecolor{currentstroke}%
\pgfsetstrokeopacity{0.000000}%
\pgfsetdash{}{0pt}%
\pgfpathmoveto{\pgfqpoint{3.809549in}{0.613486in}}%
\pgfpathlineto{\pgfqpoint{3.819554in}{0.613486in}}%
\pgfpathlineto{\pgfqpoint{3.819554in}{2.611074in}}%
\pgfpathlineto{\pgfqpoint{3.809549in}{2.611074in}}%
\pgfpathlineto{\pgfqpoint{3.809549in}{0.613486in}}%
\pgfpathclose%
\pgfusepath{fill}%
\end{pgfscope}%
\begin{pgfscope}%
\pgfpathrectangle{\pgfqpoint{0.693757in}{0.613486in}}{\pgfqpoint{5.541243in}{3.963477in}}%
\pgfusepath{clip}%
\pgfsetbuttcap%
\pgfsetmiterjoin%
\definecolor{currentfill}{rgb}{0.000000,0.000000,1.000000}%
\pgfsetfillcolor{currentfill}%
\pgfsetlinewidth{0.000000pt}%
\definecolor{currentstroke}{rgb}{0.000000,0.000000,0.000000}%
\pgfsetstrokecolor{currentstroke}%
\pgfsetstrokeopacity{0.000000}%
\pgfsetdash{}{0pt}%
\pgfpathmoveto{\pgfqpoint{3.822056in}{0.613486in}}%
\pgfpathlineto{\pgfqpoint{3.832061in}{0.613486in}}%
\pgfpathlineto{\pgfqpoint{3.832061in}{2.595224in}}%
\pgfpathlineto{\pgfqpoint{3.822056in}{2.595224in}}%
\pgfpathlineto{\pgfqpoint{3.822056in}{0.613486in}}%
\pgfpathclose%
\pgfusepath{fill}%
\end{pgfscope}%
\begin{pgfscope}%
\pgfpathrectangle{\pgfqpoint{0.693757in}{0.613486in}}{\pgfqpoint{5.541243in}{3.963477in}}%
\pgfusepath{clip}%
\pgfsetbuttcap%
\pgfsetmiterjoin%
\definecolor{currentfill}{rgb}{0.000000,0.000000,1.000000}%
\pgfsetfillcolor{currentfill}%
\pgfsetlinewidth{0.000000pt}%
\definecolor{currentstroke}{rgb}{0.000000,0.000000,0.000000}%
\pgfsetstrokecolor{currentstroke}%
\pgfsetstrokeopacity{0.000000}%
\pgfsetdash{}{0pt}%
\pgfpathmoveto{\pgfqpoint{3.834562in}{0.613486in}}%
\pgfpathlineto{\pgfqpoint{3.844567in}{0.613486in}}%
\pgfpathlineto{\pgfqpoint{3.844567in}{1.612282in}}%
\pgfpathlineto{\pgfqpoint{3.834562in}{1.612282in}}%
\pgfpathlineto{\pgfqpoint{3.834562in}{0.613486in}}%
\pgfpathclose%
\pgfusepath{fill}%
\end{pgfscope}%
\begin{pgfscope}%
\pgfpathrectangle{\pgfqpoint{0.693757in}{0.613486in}}{\pgfqpoint{5.541243in}{3.963477in}}%
\pgfusepath{clip}%
\pgfsetbuttcap%
\pgfsetmiterjoin%
\definecolor{currentfill}{rgb}{0.000000,0.000000,1.000000}%
\pgfsetfillcolor{currentfill}%
\pgfsetlinewidth{0.000000pt}%
\definecolor{currentstroke}{rgb}{0.000000,0.000000,0.000000}%
\pgfsetstrokecolor{currentstroke}%
\pgfsetstrokeopacity{0.000000}%
\pgfsetdash{}{0pt}%
\pgfpathmoveto{\pgfqpoint{3.847068in}{0.613486in}}%
\pgfpathlineto{\pgfqpoint{3.857073in}{0.613486in}}%
\pgfpathlineto{\pgfqpoint{3.857073in}{1.604355in}}%
\pgfpathlineto{\pgfqpoint{3.847068in}{1.604355in}}%
\pgfpathlineto{\pgfqpoint{3.847068in}{0.613486in}}%
\pgfpathclose%
\pgfusepath{fill}%
\end{pgfscope}%
\begin{pgfscope}%
\pgfpathrectangle{\pgfqpoint{0.693757in}{0.613486in}}{\pgfqpoint{5.541243in}{3.963477in}}%
\pgfusepath{clip}%
\pgfsetbuttcap%
\pgfsetmiterjoin%
\definecolor{currentfill}{rgb}{0.000000,0.000000,1.000000}%
\pgfsetfillcolor{currentfill}%
\pgfsetlinewidth{0.000000pt}%
\definecolor{currentstroke}{rgb}{0.000000,0.000000,0.000000}%
\pgfsetstrokecolor{currentstroke}%
\pgfsetstrokeopacity{0.000000}%
\pgfsetdash{}{0pt}%
\pgfpathmoveto{\pgfqpoint{3.859574in}{0.613486in}}%
\pgfpathlineto{\pgfqpoint{3.869579in}{0.613486in}}%
\pgfpathlineto{\pgfqpoint{3.869579in}{2.611074in}}%
\pgfpathlineto{\pgfqpoint{3.859574in}{2.611074in}}%
\pgfpathlineto{\pgfqpoint{3.859574in}{0.613486in}}%
\pgfpathclose%
\pgfusepath{fill}%
\end{pgfscope}%
\begin{pgfscope}%
\pgfpathrectangle{\pgfqpoint{0.693757in}{0.613486in}}{\pgfqpoint{5.541243in}{3.963477in}}%
\pgfusepath{clip}%
\pgfsetbuttcap%
\pgfsetmiterjoin%
\definecolor{currentfill}{rgb}{0.000000,0.000000,1.000000}%
\pgfsetfillcolor{currentfill}%
\pgfsetlinewidth{0.000000pt}%
\definecolor{currentstroke}{rgb}{0.000000,0.000000,0.000000}%
\pgfsetstrokecolor{currentstroke}%
\pgfsetstrokeopacity{0.000000}%
\pgfsetdash{}{0pt}%
\pgfpathmoveto{\pgfqpoint{3.872080in}{0.613486in}}%
\pgfpathlineto{\pgfqpoint{3.882085in}{0.613486in}}%
\pgfpathlineto{\pgfqpoint{3.882085in}{2.595224in}}%
\pgfpathlineto{\pgfqpoint{3.872080in}{2.595224in}}%
\pgfpathlineto{\pgfqpoint{3.872080in}{0.613486in}}%
\pgfpathclose%
\pgfusepath{fill}%
\end{pgfscope}%
\begin{pgfscope}%
\pgfpathrectangle{\pgfqpoint{0.693757in}{0.613486in}}{\pgfqpoint{5.541243in}{3.963477in}}%
\pgfusepath{clip}%
\pgfsetbuttcap%
\pgfsetmiterjoin%
\definecolor{currentfill}{rgb}{0.000000,0.000000,1.000000}%
\pgfsetfillcolor{currentfill}%
\pgfsetlinewidth{0.000000pt}%
\definecolor{currentstroke}{rgb}{0.000000,0.000000,0.000000}%
\pgfsetstrokecolor{currentstroke}%
\pgfsetstrokeopacity{0.000000}%
\pgfsetdash{}{0pt}%
\pgfpathmoveto{\pgfqpoint{3.884587in}{0.613486in}}%
\pgfpathlineto{\pgfqpoint{3.894591in}{0.613486in}}%
\pgfpathlineto{\pgfqpoint{3.894591in}{1.612282in}}%
\pgfpathlineto{\pgfqpoint{3.884587in}{1.612282in}}%
\pgfpathlineto{\pgfqpoint{3.884587in}{0.613486in}}%
\pgfpathclose%
\pgfusepath{fill}%
\end{pgfscope}%
\begin{pgfscope}%
\pgfpathrectangle{\pgfqpoint{0.693757in}{0.613486in}}{\pgfqpoint{5.541243in}{3.963477in}}%
\pgfusepath{clip}%
\pgfsetbuttcap%
\pgfsetmiterjoin%
\definecolor{currentfill}{rgb}{0.000000,0.000000,1.000000}%
\pgfsetfillcolor{currentfill}%
\pgfsetlinewidth{0.000000pt}%
\definecolor{currentstroke}{rgb}{0.000000,0.000000,0.000000}%
\pgfsetstrokecolor{currentstroke}%
\pgfsetstrokeopacity{0.000000}%
\pgfsetdash{}{0pt}%
\pgfpathmoveto{\pgfqpoint{3.897093in}{0.613486in}}%
\pgfpathlineto{\pgfqpoint{3.907098in}{0.613486in}}%
\pgfpathlineto{\pgfqpoint{3.907098in}{1.604355in}}%
\pgfpathlineto{\pgfqpoint{3.897093in}{1.604355in}}%
\pgfpathlineto{\pgfqpoint{3.897093in}{0.613486in}}%
\pgfpathclose%
\pgfusepath{fill}%
\end{pgfscope}%
\begin{pgfscope}%
\pgfpathrectangle{\pgfqpoint{0.693757in}{0.613486in}}{\pgfqpoint{5.541243in}{3.963477in}}%
\pgfusepath{clip}%
\pgfsetbuttcap%
\pgfsetmiterjoin%
\definecolor{currentfill}{rgb}{0.000000,0.000000,1.000000}%
\pgfsetfillcolor{currentfill}%
\pgfsetlinewidth{0.000000pt}%
\definecolor{currentstroke}{rgb}{0.000000,0.000000,0.000000}%
\pgfsetstrokecolor{currentstroke}%
\pgfsetstrokeopacity{0.000000}%
\pgfsetdash{}{0pt}%
\pgfpathmoveto{\pgfqpoint{3.909599in}{0.613486in}}%
\pgfpathlineto{\pgfqpoint{3.919604in}{0.613486in}}%
\pgfpathlineto{\pgfqpoint{3.919604in}{2.611074in}}%
\pgfpathlineto{\pgfqpoint{3.909599in}{2.611074in}}%
\pgfpathlineto{\pgfqpoint{3.909599in}{0.613486in}}%
\pgfpathclose%
\pgfusepath{fill}%
\end{pgfscope}%
\begin{pgfscope}%
\pgfpathrectangle{\pgfqpoint{0.693757in}{0.613486in}}{\pgfqpoint{5.541243in}{3.963477in}}%
\pgfusepath{clip}%
\pgfsetbuttcap%
\pgfsetmiterjoin%
\definecolor{currentfill}{rgb}{0.000000,0.000000,1.000000}%
\pgfsetfillcolor{currentfill}%
\pgfsetlinewidth{0.000000pt}%
\definecolor{currentstroke}{rgb}{0.000000,0.000000,0.000000}%
\pgfsetstrokecolor{currentstroke}%
\pgfsetstrokeopacity{0.000000}%
\pgfsetdash{}{0pt}%
\pgfpathmoveto{\pgfqpoint{3.922105in}{0.613486in}}%
\pgfpathlineto{\pgfqpoint{3.932110in}{0.613486in}}%
\pgfpathlineto{\pgfqpoint{3.932110in}{2.595224in}}%
\pgfpathlineto{\pgfqpoint{3.922105in}{2.595224in}}%
\pgfpathlineto{\pgfqpoint{3.922105in}{0.613486in}}%
\pgfpathclose%
\pgfusepath{fill}%
\end{pgfscope}%
\begin{pgfscope}%
\pgfpathrectangle{\pgfqpoint{0.693757in}{0.613486in}}{\pgfqpoint{5.541243in}{3.963477in}}%
\pgfusepath{clip}%
\pgfsetbuttcap%
\pgfsetmiterjoin%
\definecolor{currentfill}{rgb}{0.000000,0.000000,1.000000}%
\pgfsetfillcolor{currentfill}%
\pgfsetlinewidth{0.000000pt}%
\definecolor{currentstroke}{rgb}{0.000000,0.000000,0.000000}%
\pgfsetstrokecolor{currentstroke}%
\pgfsetstrokeopacity{0.000000}%
\pgfsetdash{}{0pt}%
\pgfpathmoveto{\pgfqpoint{3.934611in}{0.613486in}}%
\pgfpathlineto{\pgfqpoint{3.944616in}{0.613486in}}%
\pgfpathlineto{\pgfqpoint{3.944616in}{1.612282in}}%
\pgfpathlineto{\pgfqpoint{3.934611in}{1.612282in}}%
\pgfpathlineto{\pgfqpoint{3.934611in}{0.613486in}}%
\pgfpathclose%
\pgfusepath{fill}%
\end{pgfscope}%
\begin{pgfscope}%
\pgfpathrectangle{\pgfqpoint{0.693757in}{0.613486in}}{\pgfqpoint{5.541243in}{3.963477in}}%
\pgfusepath{clip}%
\pgfsetbuttcap%
\pgfsetmiterjoin%
\definecolor{currentfill}{rgb}{0.000000,0.000000,1.000000}%
\pgfsetfillcolor{currentfill}%
\pgfsetlinewidth{0.000000pt}%
\definecolor{currentstroke}{rgb}{0.000000,0.000000,0.000000}%
\pgfsetstrokecolor{currentstroke}%
\pgfsetstrokeopacity{0.000000}%
\pgfsetdash{}{0pt}%
\pgfpathmoveto{\pgfqpoint{3.947117in}{0.613486in}}%
\pgfpathlineto{\pgfqpoint{3.957122in}{0.613486in}}%
\pgfpathlineto{\pgfqpoint{3.957122in}{1.604355in}}%
\pgfpathlineto{\pgfqpoint{3.947117in}{1.604355in}}%
\pgfpathlineto{\pgfqpoint{3.947117in}{0.613486in}}%
\pgfpathclose%
\pgfusepath{fill}%
\end{pgfscope}%
\begin{pgfscope}%
\pgfpathrectangle{\pgfqpoint{0.693757in}{0.613486in}}{\pgfqpoint{5.541243in}{3.963477in}}%
\pgfusepath{clip}%
\pgfsetbuttcap%
\pgfsetmiterjoin%
\definecolor{currentfill}{rgb}{0.000000,0.000000,1.000000}%
\pgfsetfillcolor{currentfill}%
\pgfsetlinewidth{0.000000pt}%
\definecolor{currentstroke}{rgb}{0.000000,0.000000,0.000000}%
\pgfsetstrokecolor{currentstroke}%
\pgfsetstrokeopacity{0.000000}%
\pgfsetdash{}{0pt}%
\pgfpathmoveto{\pgfqpoint{3.959624in}{0.613486in}}%
\pgfpathlineto{\pgfqpoint{3.969629in}{0.613486in}}%
\pgfpathlineto{\pgfqpoint{3.969629in}{2.611074in}}%
\pgfpathlineto{\pgfqpoint{3.959624in}{2.611074in}}%
\pgfpathlineto{\pgfqpoint{3.959624in}{0.613486in}}%
\pgfpathclose%
\pgfusepath{fill}%
\end{pgfscope}%
\begin{pgfscope}%
\pgfpathrectangle{\pgfqpoint{0.693757in}{0.613486in}}{\pgfqpoint{5.541243in}{3.963477in}}%
\pgfusepath{clip}%
\pgfsetbuttcap%
\pgfsetmiterjoin%
\definecolor{currentfill}{rgb}{0.000000,0.000000,1.000000}%
\pgfsetfillcolor{currentfill}%
\pgfsetlinewidth{0.000000pt}%
\definecolor{currentstroke}{rgb}{0.000000,0.000000,0.000000}%
\pgfsetstrokecolor{currentstroke}%
\pgfsetstrokeopacity{0.000000}%
\pgfsetdash{}{0pt}%
\pgfpathmoveto{\pgfqpoint{3.972130in}{0.613486in}}%
\pgfpathlineto{\pgfqpoint{3.982135in}{0.613486in}}%
\pgfpathlineto{\pgfqpoint{3.982135in}{2.595224in}}%
\pgfpathlineto{\pgfqpoint{3.972130in}{2.595224in}}%
\pgfpathlineto{\pgfqpoint{3.972130in}{0.613486in}}%
\pgfpathclose%
\pgfusepath{fill}%
\end{pgfscope}%
\begin{pgfscope}%
\pgfpathrectangle{\pgfqpoint{0.693757in}{0.613486in}}{\pgfqpoint{5.541243in}{3.963477in}}%
\pgfusepath{clip}%
\pgfsetbuttcap%
\pgfsetmiterjoin%
\definecolor{currentfill}{rgb}{0.000000,0.000000,1.000000}%
\pgfsetfillcolor{currentfill}%
\pgfsetlinewidth{0.000000pt}%
\definecolor{currentstroke}{rgb}{0.000000,0.000000,0.000000}%
\pgfsetstrokecolor{currentstroke}%
\pgfsetstrokeopacity{0.000000}%
\pgfsetdash{}{0pt}%
\pgfpathmoveto{\pgfqpoint{3.984636in}{0.613486in}}%
\pgfpathlineto{\pgfqpoint{3.994641in}{0.613486in}}%
\pgfpathlineto{\pgfqpoint{3.994641in}{1.612282in}}%
\pgfpathlineto{\pgfqpoint{3.984636in}{1.612282in}}%
\pgfpathlineto{\pgfqpoint{3.984636in}{0.613486in}}%
\pgfpathclose%
\pgfusepath{fill}%
\end{pgfscope}%
\begin{pgfscope}%
\pgfpathrectangle{\pgfqpoint{0.693757in}{0.613486in}}{\pgfqpoint{5.541243in}{3.963477in}}%
\pgfusepath{clip}%
\pgfsetbuttcap%
\pgfsetmiterjoin%
\definecolor{currentfill}{rgb}{0.000000,0.000000,1.000000}%
\pgfsetfillcolor{currentfill}%
\pgfsetlinewidth{0.000000pt}%
\definecolor{currentstroke}{rgb}{0.000000,0.000000,0.000000}%
\pgfsetstrokecolor{currentstroke}%
\pgfsetstrokeopacity{0.000000}%
\pgfsetdash{}{0pt}%
\pgfpathmoveto{\pgfqpoint{3.997142in}{0.613486in}}%
\pgfpathlineto{\pgfqpoint{4.007147in}{0.613486in}}%
\pgfpathlineto{\pgfqpoint{4.007147in}{1.604355in}}%
\pgfpathlineto{\pgfqpoint{3.997142in}{1.604355in}}%
\pgfpathlineto{\pgfqpoint{3.997142in}{0.613486in}}%
\pgfpathclose%
\pgfusepath{fill}%
\end{pgfscope}%
\begin{pgfscope}%
\pgfpathrectangle{\pgfqpoint{0.693757in}{0.613486in}}{\pgfqpoint{5.541243in}{3.963477in}}%
\pgfusepath{clip}%
\pgfsetbuttcap%
\pgfsetmiterjoin%
\definecolor{currentfill}{rgb}{0.000000,0.000000,1.000000}%
\pgfsetfillcolor{currentfill}%
\pgfsetlinewidth{0.000000pt}%
\definecolor{currentstroke}{rgb}{0.000000,0.000000,0.000000}%
\pgfsetstrokecolor{currentstroke}%
\pgfsetstrokeopacity{0.000000}%
\pgfsetdash{}{0pt}%
\pgfpathmoveto{\pgfqpoint{4.009648in}{0.613486in}}%
\pgfpathlineto{\pgfqpoint{4.019653in}{0.613486in}}%
\pgfpathlineto{\pgfqpoint{4.019653in}{2.611074in}}%
\pgfpathlineto{\pgfqpoint{4.009648in}{2.611074in}}%
\pgfpathlineto{\pgfqpoint{4.009648in}{0.613486in}}%
\pgfpathclose%
\pgfusepath{fill}%
\end{pgfscope}%
\begin{pgfscope}%
\pgfpathrectangle{\pgfqpoint{0.693757in}{0.613486in}}{\pgfqpoint{5.541243in}{3.963477in}}%
\pgfusepath{clip}%
\pgfsetbuttcap%
\pgfsetmiterjoin%
\definecolor{currentfill}{rgb}{0.000000,0.000000,1.000000}%
\pgfsetfillcolor{currentfill}%
\pgfsetlinewidth{0.000000pt}%
\definecolor{currentstroke}{rgb}{0.000000,0.000000,0.000000}%
\pgfsetstrokecolor{currentstroke}%
\pgfsetstrokeopacity{0.000000}%
\pgfsetdash{}{0pt}%
\pgfpathmoveto{\pgfqpoint{4.022155in}{0.613486in}}%
\pgfpathlineto{\pgfqpoint{4.032160in}{0.613486in}}%
\pgfpathlineto{\pgfqpoint{4.032160in}{2.595224in}}%
\pgfpathlineto{\pgfqpoint{4.022155in}{2.595224in}}%
\pgfpathlineto{\pgfqpoint{4.022155in}{0.613486in}}%
\pgfpathclose%
\pgfusepath{fill}%
\end{pgfscope}%
\begin{pgfscope}%
\pgfpathrectangle{\pgfqpoint{0.693757in}{0.613486in}}{\pgfqpoint{5.541243in}{3.963477in}}%
\pgfusepath{clip}%
\pgfsetbuttcap%
\pgfsetmiterjoin%
\definecolor{currentfill}{rgb}{0.000000,0.000000,1.000000}%
\pgfsetfillcolor{currentfill}%
\pgfsetlinewidth{0.000000pt}%
\definecolor{currentstroke}{rgb}{0.000000,0.000000,0.000000}%
\pgfsetstrokecolor{currentstroke}%
\pgfsetstrokeopacity{0.000000}%
\pgfsetdash{}{0pt}%
\pgfpathmoveto{\pgfqpoint{4.034661in}{0.613486in}}%
\pgfpathlineto{\pgfqpoint{4.044666in}{0.613486in}}%
\pgfpathlineto{\pgfqpoint{4.044666in}{1.612282in}}%
\pgfpathlineto{\pgfqpoint{4.034661in}{1.612282in}}%
\pgfpathlineto{\pgfqpoint{4.034661in}{0.613486in}}%
\pgfpathclose%
\pgfusepath{fill}%
\end{pgfscope}%
\begin{pgfscope}%
\pgfpathrectangle{\pgfqpoint{0.693757in}{0.613486in}}{\pgfqpoint{5.541243in}{3.963477in}}%
\pgfusepath{clip}%
\pgfsetbuttcap%
\pgfsetmiterjoin%
\definecolor{currentfill}{rgb}{0.000000,0.000000,1.000000}%
\pgfsetfillcolor{currentfill}%
\pgfsetlinewidth{0.000000pt}%
\definecolor{currentstroke}{rgb}{0.000000,0.000000,0.000000}%
\pgfsetstrokecolor{currentstroke}%
\pgfsetstrokeopacity{0.000000}%
\pgfsetdash{}{0pt}%
\pgfpathmoveto{\pgfqpoint{4.047167in}{0.613486in}}%
\pgfpathlineto{\pgfqpoint{4.057172in}{0.613486in}}%
\pgfpathlineto{\pgfqpoint{4.057172in}{1.604355in}}%
\pgfpathlineto{\pgfqpoint{4.047167in}{1.604355in}}%
\pgfpathlineto{\pgfqpoint{4.047167in}{0.613486in}}%
\pgfpathclose%
\pgfusepath{fill}%
\end{pgfscope}%
\begin{pgfscope}%
\pgfpathrectangle{\pgfqpoint{0.693757in}{0.613486in}}{\pgfqpoint{5.541243in}{3.963477in}}%
\pgfusepath{clip}%
\pgfsetbuttcap%
\pgfsetmiterjoin%
\definecolor{currentfill}{rgb}{0.000000,0.000000,1.000000}%
\pgfsetfillcolor{currentfill}%
\pgfsetlinewidth{0.000000pt}%
\definecolor{currentstroke}{rgb}{0.000000,0.000000,0.000000}%
\pgfsetstrokecolor{currentstroke}%
\pgfsetstrokeopacity{0.000000}%
\pgfsetdash{}{0pt}%
\pgfpathmoveto{\pgfqpoint{4.059673in}{0.613486in}}%
\pgfpathlineto{\pgfqpoint{4.069678in}{0.613486in}}%
\pgfpathlineto{\pgfqpoint{4.069678in}{2.611074in}}%
\pgfpathlineto{\pgfqpoint{4.059673in}{2.611074in}}%
\pgfpathlineto{\pgfqpoint{4.059673in}{0.613486in}}%
\pgfpathclose%
\pgfusepath{fill}%
\end{pgfscope}%
\begin{pgfscope}%
\pgfpathrectangle{\pgfqpoint{0.693757in}{0.613486in}}{\pgfqpoint{5.541243in}{3.963477in}}%
\pgfusepath{clip}%
\pgfsetbuttcap%
\pgfsetmiterjoin%
\definecolor{currentfill}{rgb}{0.000000,0.000000,1.000000}%
\pgfsetfillcolor{currentfill}%
\pgfsetlinewidth{0.000000pt}%
\definecolor{currentstroke}{rgb}{0.000000,0.000000,0.000000}%
\pgfsetstrokecolor{currentstroke}%
\pgfsetstrokeopacity{0.000000}%
\pgfsetdash{}{0pt}%
\pgfpathmoveto{\pgfqpoint{4.072179in}{0.613486in}}%
\pgfpathlineto{\pgfqpoint{4.082184in}{0.613486in}}%
\pgfpathlineto{\pgfqpoint{4.082184in}{2.595224in}}%
\pgfpathlineto{\pgfqpoint{4.072179in}{2.595224in}}%
\pgfpathlineto{\pgfqpoint{4.072179in}{0.613486in}}%
\pgfpathclose%
\pgfusepath{fill}%
\end{pgfscope}%
\begin{pgfscope}%
\pgfpathrectangle{\pgfqpoint{0.693757in}{0.613486in}}{\pgfqpoint{5.541243in}{3.963477in}}%
\pgfusepath{clip}%
\pgfsetbuttcap%
\pgfsetmiterjoin%
\definecolor{currentfill}{rgb}{0.000000,0.000000,1.000000}%
\pgfsetfillcolor{currentfill}%
\pgfsetlinewidth{0.000000pt}%
\definecolor{currentstroke}{rgb}{0.000000,0.000000,0.000000}%
\pgfsetstrokecolor{currentstroke}%
\pgfsetstrokeopacity{0.000000}%
\pgfsetdash{}{0pt}%
\pgfpathmoveto{\pgfqpoint{4.084686in}{0.613486in}}%
\pgfpathlineto{\pgfqpoint{4.094691in}{0.613486in}}%
\pgfpathlineto{\pgfqpoint{4.094691in}{1.612282in}}%
\pgfpathlineto{\pgfqpoint{4.084686in}{1.612282in}}%
\pgfpathlineto{\pgfqpoint{4.084686in}{0.613486in}}%
\pgfpathclose%
\pgfusepath{fill}%
\end{pgfscope}%
\begin{pgfscope}%
\pgfpathrectangle{\pgfqpoint{0.693757in}{0.613486in}}{\pgfqpoint{5.541243in}{3.963477in}}%
\pgfusepath{clip}%
\pgfsetbuttcap%
\pgfsetmiterjoin%
\definecolor{currentfill}{rgb}{0.000000,0.000000,1.000000}%
\pgfsetfillcolor{currentfill}%
\pgfsetlinewidth{0.000000pt}%
\definecolor{currentstroke}{rgb}{0.000000,0.000000,0.000000}%
\pgfsetstrokecolor{currentstroke}%
\pgfsetstrokeopacity{0.000000}%
\pgfsetdash{}{0pt}%
\pgfpathmoveto{\pgfqpoint{4.097192in}{0.613486in}}%
\pgfpathlineto{\pgfqpoint{4.107197in}{0.613486in}}%
\pgfpathlineto{\pgfqpoint{4.107197in}{1.604355in}}%
\pgfpathlineto{\pgfqpoint{4.097192in}{1.604355in}}%
\pgfpathlineto{\pgfqpoint{4.097192in}{0.613486in}}%
\pgfpathclose%
\pgfusepath{fill}%
\end{pgfscope}%
\begin{pgfscope}%
\pgfpathrectangle{\pgfqpoint{0.693757in}{0.613486in}}{\pgfqpoint{5.541243in}{3.963477in}}%
\pgfusepath{clip}%
\pgfsetbuttcap%
\pgfsetmiterjoin%
\definecolor{currentfill}{rgb}{0.000000,0.000000,1.000000}%
\pgfsetfillcolor{currentfill}%
\pgfsetlinewidth{0.000000pt}%
\definecolor{currentstroke}{rgb}{0.000000,0.000000,0.000000}%
\pgfsetstrokecolor{currentstroke}%
\pgfsetstrokeopacity{0.000000}%
\pgfsetdash{}{0pt}%
\pgfpathmoveto{\pgfqpoint{4.109698in}{0.613486in}}%
\pgfpathlineto{\pgfqpoint{4.119703in}{0.613486in}}%
\pgfpathlineto{\pgfqpoint{4.119703in}{2.611074in}}%
\pgfpathlineto{\pgfqpoint{4.109698in}{2.611074in}}%
\pgfpathlineto{\pgfqpoint{4.109698in}{0.613486in}}%
\pgfpathclose%
\pgfusepath{fill}%
\end{pgfscope}%
\begin{pgfscope}%
\pgfpathrectangle{\pgfqpoint{0.693757in}{0.613486in}}{\pgfqpoint{5.541243in}{3.963477in}}%
\pgfusepath{clip}%
\pgfsetbuttcap%
\pgfsetmiterjoin%
\definecolor{currentfill}{rgb}{0.000000,0.000000,1.000000}%
\pgfsetfillcolor{currentfill}%
\pgfsetlinewidth{0.000000pt}%
\definecolor{currentstroke}{rgb}{0.000000,0.000000,0.000000}%
\pgfsetstrokecolor{currentstroke}%
\pgfsetstrokeopacity{0.000000}%
\pgfsetdash{}{0pt}%
\pgfpathmoveto{\pgfqpoint{4.122204in}{0.613486in}}%
\pgfpathlineto{\pgfqpoint{4.132209in}{0.613486in}}%
\pgfpathlineto{\pgfqpoint{4.132209in}{2.595224in}}%
\pgfpathlineto{\pgfqpoint{4.122204in}{2.595224in}}%
\pgfpathlineto{\pgfqpoint{4.122204in}{0.613486in}}%
\pgfpathclose%
\pgfusepath{fill}%
\end{pgfscope}%
\begin{pgfscope}%
\pgfpathrectangle{\pgfqpoint{0.693757in}{0.613486in}}{\pgfqpoint{5.541243in}{3.963477in}}%
\pgfusepath{clip}%
\pgfsetbuttcap%
\pgfsetmiterjoin%
\definecolor{currentfill}{rgb}{0.000000,0.000000,1.000000}%
\pgfsetfillcolor{currentfill}%
\pgfsetlinewidth{0.000000pt}%
\definecolor{currentstroke}{rgb}{0.000000,0.000000,0.000000}%
\pgfsetstrokecolor{currentstroke}%
\pgfsetstrokeopacity{0.000000}%
\pgfsetdash{}{0pt}%
\pgfpathmoveto{\pgfqpoint{4.134710in}{0.613486in}}%
\pgfpathlineto{\pgfqpoint{4.144715in}{0.613486in}}%
\pgfpathlineto{\pgfqpoint{4.144715in}{1.612282in}}%
\pgfpathlineto{\pgfqpoint{4.134710in}{1.612282in}}%
\pgfpathlineto{\pgfqpoint{4.134710in}{0.613486in}}%
\pgfpathclose%
\pgfusepath{fill}%
\end{pgfscope}%
\begin{pgfscope}%
\pgfpathrectangle{\pgfqpoint{0.693757in}{0.613486in}}{\pgfqpoint{5.541243in}{3.963477in}}%
\pgfusepath{clip}%
\pgfsetbuttcap%
\pgfsetmiterjoin%
\definecolor{currentfill}{rgb}{0.000000,0.000000,1.000000}%
\pgfsetfillcolor{currentfill}%
\pgfsetlinewidth{0.000000pt}%
\definecolor{currentstroke}{rgb}{0.000000,0.000000,0.000000}%
\pgfsetstrokecolor{currentstroke}%
\pgfsetstrokeopacity{0.000000}%
\pgfsetdash{}{0pt}%
\pgfpathmoveto{\pgfqpoint{4.147217in}{0.613486in}}%
\pgfpathlineto{\pgfqpoint{4.157221in}{0.613486in}}%
\pgfpathlineto{\pgfqpoint{4.157221in}{1.604355in}}%
\pgfpathlineto{\pgfqpoint{4.147217in}{1.604355in}}%
\pgfpathlineto{\pgfqpoint{4.147217in}{0.613486in}}%
\pgfpathclose%
\pgfusepath{fill}%
\end{pgfscope}%
\begin{pgfscope}%
\pgfpathrectangle{\pgfqpoint{0.693757in}{0.613486in}}{\pgfqpoint{5.541243in}{3.963477in}}%
\pgfusepath{clip}%
\pgfsetbuttcap%
\pgfsetmiterjoin%
\definecolor{currentfill}{rgb}{0.000000,0.000000,1.000000}%
\pgfsetfillcolor{currentfill}%
\pgfsetlinewidth{0.000000pt}%
\definecolor{currentstroke}{rgb}{0.000000,0.000000,0.000000}%
\pgfsetstrokecolor{currentstroke}%
\pgfsetstrokeopacity{0.000000}%
\pgfsetdash{}{0pt}%
\pgfpathmoveto{\pgfqpoint{4.159723in}{0.613486in}}%
\pgfpathlineto{\pgfqpoint{4.169728in}{0.613486in}}%
\pgfpathlineto{\pgfqpoint{4.169728in}{2.611074in}}%
\pgfpathlineto{\pgfqpoint{4.159723in}{2.611074in}}%
\pgfpathlineto{\pgfqpoint{4.159723in}{0.613486in}}%
\pgfpathclose%
\pgfusepath{fill}%
\end{pgfscope}%
\begin{pgfscope}%
\pgfpathrectangle{\pgfqpoint{0.693757in}{0.613486in}}{\pgfqpoint{5.541243in}{3.963477in}}%
\pgfusepath{clip}%
\pgfsetbuttcap%
\pgfsetmiterjoin%
\definecolor{currentfill}{rgb}{0.000000,0.000000,1.000000}%
\pgfsetfillcolor{currentfill}%
\pgfsetlinewidth{0.000000pt}%
\definecolor{currentstroke}{rgb}{0.000000,0.000000,0.000000}%
\pgfsetstrokecolor{currentstroke}%
\pgfsetstrokeopacity{0.000000}%
\pgfsetdash{}{0pt}%
\pgfpathmoveto{\pgfqpoint{4.172229in}{0.613486in}}%
\pgfpathlineto{\pgfqpoint{4.182234in}{0.613486in}}%
\pgfpathlineto{\pgfqpoint{4.182234in}{2.595224in}}%
\pgfpathlineto{\pgfqpoint{4.172229in}{2.595224in}}%
\pgfpathlineto{\pgfqpoint{4.172229in}{0.613486in}}%
\pgfpathclose%
\pgfusepath{fill}%
\end{pgfscope}%
\begin{pgfscope}%
\pgfpathrectangle{\pgfqpoint{0.693757in}{0.613486in}}{\pgfqpoint{5.541243in}{3.963477in}}%
\pgfusepath{clip}%
\pgfsetbuttcap%
\pgfsetmiterjoin%
\definecolor{currentfill}{rgb}{0.000000,0.000000,1.000000}%
\pgfsetfillcolor{currentfill}%
\pgfsetlinewidth{0.000000pt}%
\definecolor{currentstroke}{rgb}{0.000000,0.000000,0.000000}%
\pgfsetstrokecolor{currentstroke}%
\pgfsetstrokeopacity{0.000000}%
\pgfsetdash{}{0pt}%
\pgfpathmoveto{\pgfqpoint{4.184735in}{0.613486in}}%
\pgfpathlineto{\pgfqpoint{4.194740in}{0.613486in}}%
\pgfpathlineto{\pgfqpoint{4.194740in}{1.612282in}}%
\pgfpathlineto{\pgfqpoint{4.184735in}{1.612282in}}%
\pgfpathlineto{\pgfqpoint{4.184735in}{0.613486in}}%
\pgfpathclose%
\pgfusepath{fill}%
\end{pgfscope}%
\begin{pgfscope}%
\pgfpathrectangle{\pgfqpoint{0.693757in}{0.613486in}}{\pgfqpoint{5.541243in}{3.963477in}}%
\pgfusepath{clip}%
\pgfsetbuttcap%
\pgfsetmiterjoin%
\definecolor{currentfill}{rgb}{0.000000,0.000000,1.000000}%
\pgfsetfillcolor{currentfill}%
\pgfsetlinewidth{0.000000pt}%
\definecolor{currentstroke}{rgb}{0.000000,0.000000,0.000000}%
\pgfsetstrokecolor{currentstroke}%
\pgfsetstrokeopacity{0.000000}%
\pgfsetdash{}{0pt}%
\pgfpathmoveto{\pgfqpoint{4.197241in}{0.613486in}}%
\pgfpathlineto{\pgfqpoint{4.207246in}{0.613486in}}%
\pgfpathlineto{\pgfqpoint{4.207246in}{1.604355in}}%
\pgfpathlineto{\pgfqpoint{4.197241in}{1.604355in}}%
\pgfpathlineto{\pgfqpoint{4.197241in}{0.613486in}}%
\pgfpathclose%
\pgfusepath{fill}%
\end{pgfscope}%
\begin{pgfscope}%
\pgfpathrectangle{\pgfqpoint{0.693757in}{0.613486in}}{\pgfqpoint{5.541243in}{3.963477in}}%
\pgfusepath{clip}%
\pgfsetbuttcap%
\pgfsetmiterjoin%
\definecolor{currentfill}{rgb}{0.000000,0.000000,1.000000}%
\pgfsetfillcolor{currentfill}%
\pgfsetlinewidth{0.000000pt}%
\definecolor{currentstroke}{rgb}{0.000000,0.000000,0.000000}%
\pgfsetstrokecolor{currentstroke}%
\pgfsetstrokeopacity{0.000000}%
\pgfsetdash{}{0pt}%
\pgfpathmoveto{\pgfqpoint{4.209747in}{0.613486in}}%
\pgfpathlineto{\pgfqpoint{4.219752in}{0.613486in}}%
\pgfpathlineto{\pgfqpoint{4.219752in}{2.611074in}}%
\pgfpathlineto{\pgfqpoint{4.209747in}{2.611074in}}%
\pgfpathlineto{\pgfqpoint{4.209747in}{0.613486in}}%
\pgfpathclose%
\pgfusepath{fill}%
\end{pgfscope}%
\begin{pgfscope}%
\pgfpathrectangle{\pgfqpoint{0.693757in}{0.613486in}}{\pgfqpoint{5.541243in}{3.963477in}}%
\pgfusepath{clip}%
\pgfsetbuttcap%
\pgfsetmiterjoin%
\definecolor{currentfill}{rgb}{0.000000,0.000000,1.000000}%
\pgfsetfillcolor{currentfill}%
\pgfsetlinewidth{0.000000pt}%
\definecolor{currentstroke}{rgb}{0.000000,0.000000,0.000000}%
\pgfsetstrokecolor{currentstroke}%
\pgfsetstrokeopacity{0.000000}%
\pgfsetdash{}{0pt}%
\pgfpathmoveto{\pgfqpoint{4.222254in}{0.613486in}}%
\pgfpathlineto{\pgfqpoint{4.232259in}{0.613486in}}%
\pgfpathlineto{\pgfqpoint{4.232259in}{2.595224in}}%
\pgfpathlineto{\pgfqpoint{4.222254in}{2.595224in}}%
\pgfpathlineto{\pgfqpoint{4.222254in}{0.613486in}}%
\pgfpathclose%
\pgfusepath{fill}%
\end{pgfscope}%
\begin{pgfscope}%
\pgfpathrectangle{\pgfqpoint{0.693757in}{0.613486in}}{\pgfqpoint{5.541243in}{3.963477in}}%
\pgfusepath{clip}%
\pgfsetbuttcap%
\pgfsetmiterjoin%
\definecolor{currentfill}{rgb}{0.000000,0.000000,1.000000}%
\pgfsetfillcolor{currentfill}%
\pgfsetlinewidth{0.000000pt}%
\definecolor{currentstroke}{rgb}{0.000000,0.000000,0.000000}%
\pgfsetstrokecolor{currentstroke}%
\pgfsetstrokeopacity{0.000000}%
\pgfsetdash{}{0pt}%
\pgfpathmoveto{\pgfqpoint{4.234760in}{0.613486in}}%
\pgfpathlineto{\pgfqpoint{4.244765in}{0.613486in}}%
\pgfpathlineto{\pgfqpoint{4.244765in}{1.612282in}}%
\pgfpathlineto{\pgfqpoint{4.234760in}{1.612282in}}%
\pgfpathlineto{\pgfqpoint{4.234760in}{0.613486in}}%
\pgfpathclose%
\pgfusepath{fill}%
\end{pgfscope}%
\begin{pgfscope}%
\pgfpathrectangle{\pgfqpoint{0.693757in}{0.613486in}}{\pgfqpoint{5.541243in}{3.963477in}}%
\pgfusepath{clip}%
\pgfsetbuttcap%
\pgfsetmiterjoin%
\definecolor{currentfill}{rgb}{0.000000,0.000000,1.000000}%
\pgfsetfillcolor{currentfill}%
\pgfsetlinewidth{0.000000pt}%
\definecolor{currentstroke}{rgb}{0.000000,0.000000,0.000000}%
\pgfsetstrokecolor{currentstroke}%
\pgfsetstrokeopacity{0.000000}%
\pgfsetdash{}{0pt}%
\pgfpathmoveto{\pgfqpoint{4.247266in}{0.613486in}}%
\pgfpathlineto{\pgfqpoint{4.257271in}{0.613486in}}%
\pgfpathlineto{\pgfqpoint{4.257271in}{1.604355in}}%
\pgfpathlineto{\pgfqpoint{4.247266in}{1.604355in}}%
\pgfpathlineto{\pgfqpoint{4.247266in}{0.613486in}}%
\pgfpathclose%
\pgfusepath{fill}%
\end{pgfscope}%
\begin{pgfscope}%
\pgfpathrectangle{\pgfqpoint{0.693757in}{0.613486in}}{\pgfqpoint{5.541243in}{3.963477in}}%
\pgfusepath{clip}%
\pgfsetbuttcap%
\pgfsetmiterjoin%
\definecolor{currentfill}{rgb}{0.000000,0.000000,1.000000}%
\pgfsetfillcolor{currentfill}%
\pgfsetlinewidth{0.000000pt}%
\definecolor{currentstroke}{rgb}{0.000000,0.000000,0.000000}%
\pgfsetstrokecolor{currentstroke}%
\pgfsetstrokeopacity{0.000000}%
\pgfsetdash{}{0pt}%
\pgfpathmoveto{\pgfqpoint{4.259772in}{0.613486in}}%
\pgfpathlineto{\pgfqpoint{4.269777in}{0.613486in}}%
\pgfpathlineto{\pgfqpoint{4.269777in}{2.611074in}}%
\pgfpathlineto{\pgfqpoint{4.259772in}{2.611074in}}%
\pgfpathlineto{\pgfqpoint{4.259772in}{0.613486in}}%
\pgfpathclose%
\pgfusepath{fill}%
\end{pgfscope}%
\begin{pgfscope}%
\pgfpathrectangle{\pgfqpoint{0.693757in}{0.613486in}}{\pgfqpoint{5.541243in}{3.963477in}}%
\pgfusepath{clip}%
\pgfsetbuttcap%
\pgfsetmiterjoin%
\definecolor{currentfill}{rgb}{0.000000,0.000000,1.000000}%
\pgfsetfillcolor{currentfill}%
\pgfsetlinewidth{0.000000pt}%
\definecolor{currentstroke}{rgb}{0.000000,0.000000,0.000000}%
\pgfsetstrokecolor{currentstroke}%
\pgfsetstrokeopacity{0.000000}%
\pgfsetdash{}{0pt}%
\pgfpathmoveto{\pgfqpoint{4.272278in}{0.613486in}}%
\pgfpathlineto{\pgfqpoint{4.282283in}{0.613486in}}%
\pgfpathlineto{\pgfqpoint{4.282283in}{2.595224in}}%
\pgfpathlineto{\pgfqpoint{4.272278in}{2.595224in}}%
\pgfpathlineto{\pgfqpoint{4.272278in}{0.613486in}}%
\pgfpathclose%
\pgfusepath{fill}%
\end{pgfscope}%
\begin{pgfscope}%
\pgfpathrectangle{\pgfqpoint{0.693757in}{0.613486in}}{\pgfqpoint{5.541243in}{3.963477in}}%
\pgfusepath{clip}%
\pgfsetbuttcap%
\pgfsetmiterjoin%
\definecolor{currentfill}{rgb}{0.000000,0.000000,1.000000}%
\pgfsetfillcolor{currentfill}%
\pgfsetlinewidth{0.000000pt}%
\definecolor{currentstroke}{rgb}{0.000000,0.000000,0.000000}%
\pgfsetstrokecolor{currentstroke}%
\pgfsetstrokeopacity{0.000000}%
\pgfsetdash{}{0pt}%
\pgfpathmoveto{\pgfqpoint{4.284785in}{0.613486in}}%
\pgfpathlineto{\pgfqpoint{4.294790in}{0.613486in}}%
\pgfpathlineto{\pgfqpoint{4.294790in}{1.612282in}}%
\pgfpathlineto{\pgfqpoint{4.284785in}{1.612282in}}%
\pgfpathlineto{\pgfqpoint{4.284785in}{0.613486in}}%
\pgfpathclose%
\pgfusepath{fill}%
\end{pgfscope}%
\begin{pgfscope}%
\pgfpathrectangle{\pgfqpoint{0.693757in}{0.613486in}}{\pgfqpoint{5.541243in}{3.963477in}}%
\pgfusepath{clip}%
\pgfsetbuttcap%
\pgfsetmiterjoin%
\definecolor{currentfill}{rgb}{0.000000,0.000000,1.000000}%
\pgfsetfillcolor{currentfill}%
\pgfsetlinewidth{0.000000pt}%
\definecolor{currentstroke}{rgb}{0.000000,0.000000,0.000000}%
\pgfsetstrokecolor{currentstroke}%
\pgfsetstrokeopacity{0.000000}%
\pgfsetdash{}{0pt}%
\pgfpathmoveto{\pgfqpoint{4.297291in}{0.613486in}}%
\pgfpathlineto{\pgfqpoint{4.307296in}{0.613486in}}%
\pgfpathlineto{\pgfqpoint{4.307296in}{1.604355in}}%
\pgfpathlineto{\pgfqpoint{4.297291in}{1.604355in}}%
\pgfpathlineto{\pgfqpoint{4.297291in}{0.613486in}}%
\pgfpathclose%
\pgfusepath{fill}%
\end{pgfscope}%
\begin{pgfscope}%
\pgfpathrectangle{\pgfqpoint{0.693757in}{0.613486in}}{\pgfqpoint{5.541243in}{3.963477in}}%
\pgfusepath{clip}%
\pgfsetbuttcap%
\pgfsetmiterjoin%
\definecolor{currentfill}{rgb}{0.000000,0.000000,1.000000}%
\pgfsetfillcolor{currentfill}%
\pgfsetlinewidth{0.000000pt}%
\definecolor{currentstroke}{rgb}{0.000000,0.000000,0.000000}%
\pgfsetstrokecolor{currentstroke}%
\pgfsetstrokeopacity{0.000000}%
\pgfsetdash{}{0pt}%
\pgfpathmoveto{\pgfqpoint{4.309797in}{0.613486in}}%
\pgfpathlineto{\pgfqpoint{4.319802in}{0.613486in}}%
\pgfpathlineto{\pgfqpoint{4.319802in}{2.611074in}}%
\pgfpathlineto{\pgfqpoint{4.309797in}{2.611074in}}%
\pgfpathlineto{\pgfqpoint{4.309797in}{0.613486in}}%
\pgfpathclose%
\pgfusepath{fill}%
\end{pgfscope}%
\begin{pgfscope}%
\pgfpathrectangle{\pgfqpoint{0.693757in}{0.613486in}}{\pgfqpoint{5.541243in}{3.963477in}}%
\pgfusepath{clip}%
\pgfsetbuttcap%
\pgfsetmiterjoin%
\definecolor{currentfill}{rgb}{0.000000,0.000000,1.000000}%
\pgfsetfillcolor{currentfill}%
\pgfsetlinewidth{0.000000pt}%
\definecolor{currentstroke}{rgb}{0.000000,0.000000,0.000000}%
\pgfsetstrokecolor{currentstroke}%
\pgfsetstrokeopacity{0.000000}%
\pgfsetdash{}{0pt}%
\pgfpathmoveto{\pgfqpoint{4.322303in}{0.613486in}}%
\pgfpathlineto{\pgfqpoint{4.332308in}{0.613486in}}%
\pgfpathlineto{\pgfqpoint{4.332308in}{2.595224in}}%
\pgfpathlineto{\pgfqpoint{4.322303in}{2.595224in}}%
\pgfpathlineto{\pgfqpoint{4.322303in}{0.613486in}}%
\pgfpathclose%
\pgfusepath{fill}%
\end{pgfscope}%
\begin{pgfscope}%
\pgfpathrectangle{\pgfqpoint{0.693757in}{0.613486in}}{\pgfqpoint{5.541243in}{3.963477in}}%
\pgfusepath{clip}%
\pgfsetbuttcap%
\pgfsetmiterjoin%
\definecolor{currentfill}{rgb}{0.000000,0.000000,1.000000}%
\pgfsetfillcolor{currentfill}%
\pgfsetlinewidth{0.000000pt}%
\definecolor{currentstroke}{rgb}{0.000000,0.000000,0.000000}%
\pgfsetstrokecolor{currentstroke}%
\pgfsetstrokeopacity{0.000000}%
\pgfsetdash{}{0pt}%
\pgfpathmoveto{\pgfqpoint{4.334809in}{0.613486in}}%
\pgfpathlineto{\pgfqpoint{4.344814in}{0.613486in}}%
\pgfpathlineto{\pgfqpoint{4.344814in}{1.612282in}}%
\pgfpathlineto{\pgfqpoint{4.334809in}{1.612282in}}%
\pgfpathlineto{\pgfqpoint{4.334809in}{0.613486in}}%
\pgfpathclose%
\pgfusepath{fill}%
\end{pgfscope}%
\begin{pgfscope}%
\pgfpathrectangle{\pgfqpoint{0.693757in}{0.613486in}}{\pgfqpoint{5.541243in}{3.963477in}}%
\pgfusepath{clip}%
\pgfsetbuttcap%
\pgfsetmiterjoin%
\definecolor{currentfill}{rgb}{0.000000,0.000000,1.000000}%
\pgfsetfillcolor{currentfill}%
\pgfsetlinewidth{0.000000pt}%
\definecolor{currentstroke}{rgb}{0.000000,0.000000,0.000000}%
\pgfsetstrokecolor{currentstroke}%
\pgfsetstrokeopacity{0.000000}%
\pgfsetdash{}{0pt}%
\pgfpathmoveto{\pgfqpoint{4.347316in}{0.613486in}}%
\pgfpathlineto{\pgfqpoint{4.357321in}{0.613486in}}%
\pgfpathlineto{\pgfqpoint{4.357321in}{1.604355in}}%
\pgfpathlineto{\pgfqpoint{4.347316in}{1.604355in}}%
\pgfpathlineto{\pgfqpoint{4.347316in}{0.613486in}}%
\pgfpathclose%
\pgfusepath{fill}%
\end{pgfscope}%
\begin{pgfscope}%
\pgfpathrectangle{\pgfqpoint{0.693757in}{0.613486in}}{\pgfqpoint{5.541243in}{3.963477in}}%
\pgfusepath{clip}%
\pgfsetbuttcap%
\pgfsetmiterjoin%
\definecolor{currentfill}{rgb}{0.000000,0.000000,1.000000}%
\pgfsetfillcolor{currentfill}%
\pgfsetlinewidth{0.000000pt}%
\definecolor{currentstroke}{rgb}{0.000000,0.000000,0.000000}%
\pgfsetstrokecolor{currentstroke}%
\pgfsetstrokeopacity{0.000000}%
\pgfsetdash{}{0pt}%
\pgfpathmoveto{\pgfqpoint{4.359822in}{0.613486in}}%
\pgfpathlineto{\pgfqpoint{4.369827in}{0.613486in}}%
\pgfpathlineto{\pgfqpoint{4.369827in}{2.611074in}}%
\pgfpathlineto{\pgfqpoint{4.359822in}{2.611074in}}%
\pgfpathlineto{\pgfqpoint{4.359822in}{0.613486in}}%
\pgfpathclose%
\pgfusepath{fill}%
\end{pgfscope}%
\begin{pgfscope}%
\pgfpathrectangle{\pgfqpoint{0.693757in}{0.613486in}}{\pgfqpoint{5.541243in}{3.963477in}}%
\pgfusepath{clip}%
\pgfsetbuttcap%
\pgfsetmiterjoin%
\definecolor{currentfill}{rgb}{0.000000,0.000000,1.000000}%
\pgfsetfillcolor{currentfill}%
\pgfsetlinewidth{0.000000pt}%
\definecolor{currentstroke}{rgb}{0.000000,0.000000,0.000000}%
\pgfsetstrokecolor{currentstroke}%
\pgfsetstrokeopacity{0.000000}%
\pgfsetdash{}{0pt}%
\pgfpathmoveto{\pgfqpoint{4.372328in}{0.613486in}}%
\pgfpathlineto{\pgfqpoint{4.382333in}{0.613486in}}%
\pgfpathlineto{\pgfqpoint{4.382333in}{2.595224in}}%
\pgfpathlineto{\pgfqpoint{4.372328in}{2.595224in}}%
\pgfpathlineto{\pgfqpoint{4.372328in}{0.613486in}}%
\pgfpathclose%
\pgfusepath{fill}%
\end{pgfscope}%
\begin{pgfscope}%
\pgfpathrectangle{\pgfqpoint{0.693757in}{0.613486in}}{\pgfqpoint{5.541243in}{3.963477in}}%
\pgfusepath{clip}%
\pgfsetbuttcap%
\pgfsetmiterjoin%
\definecolor{currentfill}{rgb}{0.000000,0.000000,1.000000}%
\pgfsetfillcolor{currentfill}%
\pgfsetlinewidth{0.000000pt}%
\definecolor{currentstroke}{rgb}{0.000000,0.000000,0.000000}%
\pgfsetstrokecolor{currentstroke}%
\pgfsetstrokeopacity{0.000000}%
\pgfsetdash{}{0pt}%
\pgfpathmoveto{\pgfqpoint{4.384834in}{0.613486in}}%
\pgfpathlineto{\pgfqpoint{4.394839in}{0.613486in}}%
\pgfpathlineto{\pgfqpoint{4.394839in}{1.612282in}}%
\pgfpathlineto{\pgfqpoint{4.384834in}{1.612282in}}%
\pgfpathlineto{\pgfqpoint{4.384834in}{0.613486in}}%
\pgfpathclose%
\pgfusepath{fill}%
\end{pgfscope}%
\begin{pgfscope}%
\pgfpathrectangle{\pgfqpoint{0.693757in}{0.613486in}}{\pgfqpoint{5.541243in}{3.963477in}}%
\pgfusepath{clip}%
\pgfsetbuttcap%
\pgfsetmiterjoin%
\definecolor{currentfill}{rgb}{0.000000,0.000000,1.000000}%
\pgfsetfillcolor{currentfill}%
\pgfsetlinewidth{0.000000pt}%
\definecolor{currentstroke}{rgb}{0.000000,0.000000,0.000000}%
\pgfsetstrokecolor{currentstroke}%
\pgfsetstrokeopacity{0.000000}%
\pgfsetdash{}{0pt}%
\pgfpathmoveto{\pgfqpoint{4.397340in}{0.613486in}}%
\pgfpathlineto{\pgfqpoint{4.407345in}{0.613486in}}%
\pgfpathlineto{\pgfqpoint{4.407345in}{1.604355in}}%
\pgfpathlineto{\pgfqpoint{4.397340in}{1.604355in}}%
\pgfpathlineto{\pgfqpoint{4.397340in}{0.613486in}}%
\pgfpathclose%
\pgfusepath{fill}%
\end{pgfscope}%
\begin{pgfscope}%
\pgfpathrectangle{\pgfqpoint{0.693757in}{0.613486in}}{\pgfqpoint{5.541243in}{3.963477in}}%
\pgfusepath{clip}%
\pgfsetbuttcap%
\pgfsetmiterjoin%
\definecolor{currentfill}{rgb}{0.000000,0.000000,1.000000}%
\pgfsetfillcolor{currentfill}%
\pgfsetlinewidth{0.000000pt}%
\definecolor{currentstroke}{rgb}{0.000000,0.000000,0.000000}%
\pgfsetstrokecolor{currentstroke}%
\pgfsetstrokeopacity{0.000000}%
\pgfsetdash{}{0pt}%
\pgfpathmoveto{\pgfqpoint{4.409847in}{0.613486in}}%
\pgfpathlineto{\pgfqpoint{4.419852in}{0.613486in}}%
\pgfpathlineto{\pgfqpoint{4.419852in}{2.611074in}}%
\pgfpathlineto{\pgfqpoint{4.409847in}{2.611074in}}%
\pgfpathlineto{\pgfqpoint{4.409847in}{0.613486in}}%
\pgfpathclose%
\pgfusepath{fill}%
\end{pgfscope}%
\begin{pgfscope}%
\pgfpathrectangle{\pgfqpoint{0.693757in}{0.613486in}}{\pgfqpoint{5.541243in}{3.963477in}}%
\pgfusepath{clip}%
\pgfsetbuttcap%
\pgfsetmiterjoin%
\definecolor{currentfill}{rgb}{0.000000,0.000000,1.000000}%
\pgfsetfillcolor{currentfill}%
\pgfsetlinewidth{0.000000pt}%
\definecolor{currentstroke}{rgb}{0.000000,0.000000,0.000000}%
\pgfsetstrokecolor{currentstroke}%
\pgfsetstrokeopacity{0.000000}%
\pgfsetdash{}{0pt}%
\pgfpathmoveto{\pgfqpoint{4.422353in}{0.613486in}}%
\pgfpathlineto{\pgfqpoint{4.432358in}{0.613486in}}%
\pgfpathlineto{\pgfqpoint{4.432358in}{2.595224in}}%
\pgfpathlineto{\pgfqpoint{4.422353in}{2.595224in}}%
\pgfpathlineto{\pgfqpoint{4.422353in}{0.613486in}}%
\pgfpathclose%
\pgfusepath{fill}%
\end{pgfscope}%
\begin{pgfscope}%
\pgfpathrectangle{\pgfqpoint{0.693757in}{0.613486in}}{\pgfqpoint{5.541243in}{3.963477in}}%
\pgfusepath{clip}%
\pgfsetbuttcap%
\pgfsetmiterjoin%
\definecolor{currentfill}{rgb}{0.000000,0.000000,1.000000}%
\pgfsetfillcolor{currentfill}%
\pgfsetlinewidth{0.000000pt}%
\definecolor{currentstroke}{rgb}{0.000000,0.000000,0.000000}%
\pgfsetstrokecolor{currentstroke}%
\pgfsetstrokeopacity{0.000000}%
\pgfsetdash{}{0pt}%
\pgfpathmoveto{\pgfqpoint{4.434859in}{0.613486in}}%
\pgfpathlineto{\pgfqpoint{4.444864in}{0.613486in}}%
\pgfpathlineto{\pgfqpoint{4.444864in}{1.612282in}}%
\pgfpathlineto{\pgfqpoint{4.434859in}{1.612282in}}%
\pgfpathlineto{\pgfqpoint{4.434859in}{0.613486in}}%
\pgfpathclose%
\pgfusepath{fill}%
\end{pgfscope}%
\begin{pgfscope}%
\pgfpathrectangle{\pgfqpoint{0.693757in}{0.613486in}}{\pgfqpoint{5.541243in}{3.963477in}}%
\pgfusepath{clip}%
\pgfsetbuttcap%
\pgfsetmiterjoin%
\definecolor{currentfill}{rgb}{0.000000,0.000000,1.000000}%
\pgfsetfillcolor{currentfill}%
\pgfsetlinewidth{0.000000pt}%
\definecolor{currentstroke}{rgb}{0.000000,0.000000,0.000000}%
\pgfsetstrokecolor{currentstroke}%
\pgfsetstrokeopacity{0.000000}%
\pgfsetdash{}{0pt}%
\pgfpathmoveto{\pgfqpoint{4.447365in}{0.613486in}}%
\pgfpathlineto{\pgfqpoint{4.457370in}{0.613486in}}%
\pgfpathlineto{\pgfqpoint{4.457370in}{1.604355in}}%
\pgfpathlineto{\pgfqpoint{4.447365in}{1.604355in}}%
\pgfpathlineto{\pgfqpoint{4.447365in}{0.613486in}}%
\pgfpathclose%
\pgfusepath{fill}%
\end{pgfscope}%
\begin{pgfscope}%
\pgfpathrectangle{\pgfqpoint{0.693757in}{0.613486in}}{\pgfqpoint{5.541243in}{3.963477in}}%
\pgfusepath{clip}%
\pgfsetbuttcap%
\pgfsetmiterjoin%
\definecolor{currentfill}{rgb}{0.000000,0.000000,1.000000}%
\pgfsetfillcolor{currentfill}%
\pgfsetlinewidth{0.000000pt}%
\definecolor{currentstroke}{rgb}{0.000000,0.000000,0.000000}%
\pgfsetstrokecolor{currentstroke}%
\pgfsetstrokeopacity{0.000000}%
\pgfsetdash{}{0pt}%
\pgfpathmoveto{\pgfqpoint{4.459871in}{0.613486in}}%
\pgfpathlineto{\pgfqpoint{4.469876in}{0.613486in}}%
\pgfpathlineto{\pgfqpoint{4.469876in}{2.611074in}}%
\pgfpathlineto{\pgfqpoint{4.459871in}{2.611074in}}%
\pgfpathlineto{\pgfqpoint{4.459871in}{0.613486in}}%
\pgfpathclose%
\pgfusepath{fill}%
\end{pgfscope}%
\begin{pgfscope}%
\pgfpathrectangle{\pgfqpoint{0.693757in}{0.613486in}}{\pgfqpoint{5.541243in}{3.963477in}}%
\pgfusepath{clip}%
\pgfsetbuttcap%
\pgfsetmiterjoin%
\definecolor{currentfill}{rgb}{0.000000,0.000000,1.000000}%
\pgfsetfillcolor{currentfill}%
\pgfsetlinewidth{0.000000pt}%
\definecolor{currentstroke}{rgb}{0.000000,0.000000,0.000000}%
\pgfsetstrokecolor{currentstroke}%
\pgfsetstrokeopacity{0.000000}%
\pgfsetdash{}{0pt}%
\pgfpathmoveto{\pgfqpoint{4.472378in}{0.613486in}}%
\pgfpathlineto{\pgfqpoint{4.482382in}{0.613486in}}%
\pgfpathlineto{\pgfqpoint{4.482382in}{2.595224in}}%
\pgfpathlineto{\pgfqpoint{4.472378in}{2.595224in}}%
\pgfpathlineto{\pgfqpoint{4.472378in}{0.613486in}}%
\pgfpathclose%
\pgfusepath{fill}%
\end{pgfscope}%
\begin{pgfscope}%
\pgfpathrectangle{\pgfqpoint{0.693757in}{0.613486in}}{\pgfqpoint{5.541243in}{3.963477in}}%
\pgfusepath{clip}%
\pgfsetbuttcap%
\pgfsetmiterjoin%
\definecolor{currentfill}{rgb}{0.000000,0.000000,1.000000}%
\pgfsetfillcolor{currentfill}%
\pgfsetlinewidth{0.000000pt}%
\definecolor{currentstroke}{rgb}{0.000000,0.000000,0.000000}%
\pgfsetstrokecolor{currentstroke}%
\pgfsetstrokeopacity{0.000000}%
\pgfsetdash{}{0pt}%
\pgfpathmoveto{\pgfqpoint{4.484884in}{0.613486in}}%
\pgfpathlineto{\pgfqpoint{4.494889in}{0.613486in}}%
\pgfpathlineto{\pgfqpoint{4.494889in}{1.612282in}}%
\pgfpathlineto{\pgfqpoint{4.484884in}{1.612282in}}%
\pgfpathlineto{\pgfqpoint{4.484884in}{0.613486in}}%
\pgfpathclose%
\pgfusepath{fill}%
\end{pgfscope}%
\begin{pgfscope}%
\pgfpathrectangle{\pgfqpoint{0.693757in}{0.613486in}}{\pgfqpoint{5.541243in}{3.963477in}}%
\pgfusepath{clip}%
\pgfsetbuttcap%
\pgfsetmiterjoin%
\definecolor{currentfill}{rgb}{0.000000,0.000000,1.000000}%
\pgfsetfillcolor{currentfill}%
\pgfsetlinewidth{0.000000pt}%
\definecolor{currentstroke}{rgb}{0.000000,0.000000,0.000000}%
\pgfsetstrokecolor{currentstroke}%
\pgfsetstrokeopacity{0.000000}%
\pgfsetdash{}{0pt}%
\pgfpathmoveto{\pgfqpoint{4.497390in}{0.613486in}}%
\pgfpathlineto{\pgfqpoint{4.507395in}{0.613486in}}%
\pgfpathlineto{\pgfqpoint{4.507395in}{1.604355in}}%
\pgfpathlineto{\pgfqpoint{4.497390in}{1.604355in}}%
\pgfpathlineto{\pgfqpoint{4.497390in}{0.613486in}}%
\pgfpathclose%
\pgfusepath{fill}%
\end{pgfscope}%
\begin{pgfscope}%
\pgfpathrectangle{\pgfqpoint{0.693757in}{0.613486in}}{\pgfqpoint{5.541243in}{3.963477in}}%
\pgfusepath{clip}%
\pgfsetbuttcap%
\pgfsetmiterjoin%
\definecolor{currentfill}{rgb}{0.000000,0.000000,1.000000}%
\pgfsetfillcolor{currentfill}%
\pgfsetlinewidth{0.000000pt}%
\definecolor{currentstroke}{rgb}{0.000000,0.000000,0.000000}%
\pgfsetstrokecolor{currentstroke}%
\pgfsetstrokeopacity{0.000000}%
\pgfsetdash{}{0pt}%
\pgfpathmoveto{\pgfqpoint{4.509896in}{0.613486in}}%
\pgfpathlineto{\pgfqpoint{4.519901in}{0.613486in}}%
\pgfpathlineto{\pgfqpoint{4.519901in}{2.611074in}}%
\pgfpathlineto{\pgfqpoint{4.509896in}{2.611074in}}%
\pgfpathlineto{\pgfqpoint{4.509896in}{0.613486in}}%
\pgfpathclose%
\pgfusepath{fill}%
\end{pgfscope}%
\begin{pgfscope}%
\pgfpathrectangle{\pgfqpoint{0.693757in}{0.613486in}}{\pgfqpoint{5.541243in}{3.963477in}}%
\pgfusepath{clip}%
\pgfsetbuttcap%
\pgfsetmiterjoin%
\definecolor{currentfill}{rgb}{0.000000,0.000000,1.000000}%
\pgfsetfillcolor{currentfill}%
\pgfsetlinewidth{0.000000pt}%
\definecolor{currentstroke}{rgb}{0.000000,0.000000,0.000000}%
\pgfsetstrokecolor{currentstroke}%
\pgfsetstrokeopacity{0.000000}%
\pgfsetdash{}{0pt}%
\pgfpathmoveto{\pgfqpoint{4.522402in}{0.613486in}}%
\pgfpathlineto{\pgfqpoint{4.532407in}{0.613486in}}%
\pgfpathlineto{\pgfqpoint{4.532407in}{2.595224in}}%
\pgfpathlineto{\pgfqpoint{4.522402in}{2.595224in}}%
\pgfpathlineto{\pgfqpoint{4.522402in}{0.613486in}}%
\pgfpathclose%
\pgfusepath{fill}%
\end{pgfscope}%
\begin{pgfscope}%
\pgfpathrectangle{\pgfqpoint{0.693757in}{0.613486in}}{\pgfqpoint{5.541243in}{3.963477in}}%
\pgfusepath{clip}%
\pgfsetbuttcap%
\pgfsetmiterjoin%
\definecolor{currentfill}{rgb}{0.000000,0.000000,1.000000}%
\pgfsetfillcolor{currentfill}%
\pgfsetlinewidth{0.000000pt}%
\definecolor{currentstroke}{rgb}{0.000000,0.000000,0.000000}%
\pgfsetstrokecolor{currentstroke}%
\pgfsetstrokeopacity{0.000000}%
\pgfsetdash{}{0pt}%
\pgfpathmoveto{\pgfqpoint{4.534908in}{0.613486in}}%
\pgfpathlineto{\pgfqpoint{4.544913in}{0.613486in}}%
\pgfpathlineto{\pgfqpoint{4.544913in}{1.612282in}}%
\pgfpathlineto{\pgfqpoint{4.534908in}{1.612282in}}%
\pgfpathlineto{\pgfqpoint{4.534908in}{0.613486in}}%
\pgfpathclose%
\pgfusepath{fill}%
\end{pgfscope}%
\begin{pgfscope}%
\pgfpathrectangle{\pgfqpoint{0.693757in}{0.613486in}}{\pgfqpoint{5.541243in}{3.963477in}}%
\pgfusepath{clip}%
\pgfsetbuttcap%
\pgfsetmiterjoin%
\definecolor{currentfill}{rgb}{0.000000,0.000000,1.000000}%
\pgfsetfillcolor{currentfill}%
\pgfsetlinewidth{0.000000pt}%
\definecolor{currentstroke}{rgb}{0.000000,0.000000,0.000000}%
\pgfsetstrokecolor{currentstroke}%
\pgfsetstrokeopacity{0.000000}%
\pgfsetdash{}{0pt}%
\pgfpathmoveto{\pgfqpoint{4.547415in}{0.613486in}}%
\pgfpathlineto{\pgfqpoint{4.557420in}{0.613486in}}%
\pgfpathlineto{\pgfqpoint{4.557420in}{1.604355in}}%
\pgfpathlineto{\pgfqpoint{4.547415in}{1.604355in}}%
\pgfpathlineto{\pgfqpoint{4.547415in}{0.613486in}}%
\pgfpathclose%
\pgfusepath{fill}%
\end{pgfscope}%
\begin{pgfscope}%
\pgfpathrectangle{\pgfqpoint{0.693757in}{0.613486in}}{\pgfqpoint{5.541243in}{3.963477in}}%
\pgfusepath{clip}%
\pgfsetbuttcap%
\pgfsetmiterjoin%
\definecolor{currentfill}{rgb}{0.000000,0.000000,1.000000}%
\pgfsetfillcolor{currentfill}%
\pgfsetlinewidth{0.000000pt}%
\definecolor{currentstroke}{rgb}{0.000000,0.000000,0.000000}%
\pgfsetstrokecolor{currentstroke}%
\pgfsetstrokeopacity{0.000000}%
\pgfsetdash{}{0pt}%
\pgfpathmoveto{\pgfqpoint{4.559921in}{0.613486in}}%
\pgfpathlineto{\pgfqpoint{4.569926in}{0.613486in}}%
\pgfpathlineto{\pgfqpoint{4.569926in}{2.611074in}}%
\pgfpathlineto{\pgfqpoint{4.559921in}{2.611074in}}%
\pgfpathlineto{\pgfqpoint{4.559921in}{0.613486in}}%
\pgfpathclose%
\pgfusepath{fill}%
\end{pgfscope}%
\begin{pgfscope}%
\pgfpathrectangle{\pgfqpoint{0.693757in}{0.613486in}}{\pgfqpoint{5.541243in}{3.963477in}}%
\pgfusepath{clip}%
\pgfsetbuttcap%
\pgfsetmiterjoin%
\definecolor{currentfill}{rgb}{0.000000,0.000000,1.000000}%
\pgfsetfillcolor{currentfill}%
\pgfsetlinewidth{0.000000pt}%
\definecolor{currentstroke}{rgb}{0.000000,0.000000,0.000000}%
\pgfsetstrokecolor{currentstroke}%
\pgfsetstrokeopacity{0.000000}%
\pgfsetdash{}{0pt}%
\pgfpathmoveto{\pgfqpoint{4.572427in}{0.613486in}}%
\pgfpathlineto{\pgfqpoint{4.582432in}{0.613486in}}%
\pgfpathlineto{\pgfqpoint{4.582432in}{2.595224in}}%
\pgfpathlineto{\pgfqpoint{4.572427in}{2.595224in}}%
\pgfpathlineto{\pgfqpoint{4.572427in}{0.613486in}}%
\pgfpathclose%
\pgfusepath{fill}%
\end{pgfscope}%
\begin{pgfscope}%
\pgfpathrectangle{\pgfqpoint{0.693757in}{0.613486in}}{\pgfqpoint{5.541243in}{3.963477in}}%
\pgfusepath{clip}%
\pgfsetbuttcap%
\pgfsetmiterjoin%
\definecolor{currentfill}{rgb}{0.000000,0.000000,1.000000}%
\pgfsetfillcolor{currentfill}%
\pgfsetlinewidth{0.000000pt}%
\definecolor{currentstroke}{rgb}{0.000000,0.000000,0.000000}%
\pgfsetstrokecolor{currentstroke}%
\pgfsetstrokeopacity{0.000000}%
\pgfsetdash{}{0pt}%
\pgfpathmoveto{\pgfqpoint{4.584933in}{0.613486in}}%
\pgfpathlineto{\pgfqpoint{4.594938in}{0.613486in}}%
\pgfpathlineto{\pgfqpoint{4.594938in}{1.612282in}}%
\pgfpathlineto{\pgfqpoint{4.584933in}{1.612282in}}%
\pgfpathlineto{\pgfqpoint{4.584933in}{0.613486in}}%
\pgfpathclose%
\pgfusepath{fill}%
\end{pgfscope}%
\begin{pgfscope}%
\pgfpathrectangle{\pgfqpoint{0.693757in}{0.613486in}}{\pgfqpoint{5.541243in}{3.963477in}}%
\pgfusepath{clip}%
\pgfsetbuttcap%
\pgfsetmiterjoin%
\definecolor{currentfill}{rgb}{0.000000,0.000000,1.000000}%
\pgfsetfillcolor{currentfill}%
\pgfsetlinewidth{0.000000pt}%
\definecolor{currentstroke}{rgb}{0.000000,0.000000,0.000000}%
\pgfsetstrokecolor{currentstroke}%
\pgfsetstrokeopacity{0.000000}%
\pgfsetdash{}{0pt}%
\pgfpathmoveto{\pgfqpoint{4.597439in}{0.613486in}}%
\pgfpathlineto{\pgfqpoint{4.607444in}{0.613486in}}%
\pgfpathlineto{\pgfqpoint{4.607444in}{1.604355in}}%
\pgfpathlineto{\pgfqpoint{4.597439in}{1.604355in}}%
\pgfpathlineto{\pgfqpoint{4.597439in}{0.613486in}}%
\pgfpathclose%
\pgfusepath{fill}%
\end{pgfscope}%
\begin{pgfscope}%
\pgfpathrectangle{\pgfqpoint{0.693757in}{0.613486in}}{\pgfqpoint{5.541243in}{3.963477in}}%
\pgfusepath{clip}%
\pgfsetbuttcap%
\pgfsetmiterjoin%
\definecolor{currentfill}{rgb}{0.000000,0.000000,1.000000}%
\pgfsetfillcolor{currentfill}%
\pgfsetlinewidth{0.000000pt}%
\definecolor{currentstroke}{rgb}{0.000000,0.000000,0.000000}%
\pgfsetstrokecolor{currentstroke}%
\pgfsetstrokeopacity{0.000000}%
\pgfsetdash{}{0pt}%
\pgfpathmoveto{\pgfqpoint{4.609946in}{0.613486in}}%
\pgfpathlineto{\pgfqpoint{4.619951in}{0.613486in}}%
\pgfpathlineto{\pgfqpoint{4.619951in}{2.611074in}}%
\pgfpathlineto{\pgfqpoint{4.609946in}{2.611074in}}%
\pgfpathlineto{\pgfqpoint{4.609946in}{0.613486in}}%
\pgfpathclose%
\pgfusepath{fill}%
\end{pgfscope}%
\begin{pgfscope}%
\pgfpathrectangle{\pgfqpoint{0.693757in}{0.613486in}}{\pgfqpoint{5.541243in}{3.963477in}}%
\pgfusepath{clip}%
\pgfsetbuttcap%
\pgfsetmiterjoin%
\definecolor{currentfill}{rgb}{0.000000,0.000000,1.000000}%
\pgfsetfillcolor{currentfill}%
\pgfsetlinewidth{0.000000pt}%
\definecolor{currentstroke}{rgb}{0.000000,0.000000,0.000000}%
\pgfsetstrokecolor{currentstroke}%
\pgfsetstrokeopacity{0.000000}%
\pgfsetdash{}{0pt}%
\pgfpathmoveto{\pgfqpoint{4.622452in}{0.613486in}}%
\pgfpathlineto{\pgfqpoint{4.632457in}{0.613486in}}%
\pgfpathlineto{\pgfqpoint{4.632457in}{2.595224in}}%
\pgfpathlineto{\pgfqpoint{4.622452in}{2.595224in}}%
\pgfpathlineto{\pgfqpoint{4.622452in}{0.613486in}}%
\pgfpathclose%
\pgfusepath{fill}%
\end{pgfscope}%
\begin{pgfscope}%
\pgfpathrectangle{\pgfqpoint{0.693757in}{0.613486in}}{\pgfqpoint{5.541243in}{3.963477in}}%
\pgfusepath{clip}%
\pgfsetbuttcap%
\pgfsetmiterjoin%
\definecolor{currentfill}{rgb}{0.000000,0.000000,1.000000}%
\pgfsetfillcolor{currentfill}%
\pgfsetlinewidth{0.000000pt}%
\definecolor{currentstroke}{rgb}{0.000000,0.000000,0.000000}%
\pgfsetstrokecolor{currentstroke}%
\pgfsetstrokeopacity{0.000000}%
\pgfsetdash{}{0pt}%
\pgfpathmoveto{\pgfqpoint{4.634958in}{0.613486in}}%
\pgfpathlineto{\pgfqpoint{4.644963in}{0.613486in}}%
\pgfpathlineto{\pgfqpoint{4.644963in}{1.612282in}}%
\pgfpathlineto{\pgfqpoint{4.634958in}{1.612282in}}%
\pgfpathlineto{\pgfqpoint{4.634958in}{0.613486in}}%
\pgfpathclose%
\pgfusepath{fill}%
\end{pgfscope}%
\begin{pgfscope}%
\pgfpathrectangle{\pgfqpoint{0.693757in}{0.613486in}}{\pgfqpoint{5.541243in}{3.963477in}}%
\pgfusepath{clip}%
\pgfsetbuttcap%
\pgfsetmiterjoin%
\definecolor{currentfill}{rgb}{0.000000,0.000000,1.000000}%
\pgfsetfillcolor{currentfill}%
\pgfsetlinewidth{0.000000pt}%
\definecolor{currentstroke}{rgb}{0.000000,0.000000,0.000000}%
\pgfsetstrokecolor{currentstroke}%
\pgfsetstrokeopacity{0.000000}%
\pgfsetdash{}{0pt}%
\pgfpathmoveto{\pgfqpoint{4.647464in}{0.613486in}}%
\pgfpathlineto{\pgfqpoint{4.657469in}{0.613486in}}%
\pgfpathlineto{\pgfqpoint{4.657469in}{1.604355in}}%
\pgfpathlineto{\pgfqpoint{4.647464in}{1.604355in}}%
\pgfpathlineto{\pgfqpoint{4.647464in}{0.613486in}}%
\pgfpathclose%
\pgfusepath{fill}%
\end{pgfscope}%
\begin{pgfscope}%
\pgfpathrectangle{\pgfqpoint{0.693757in}{0.613486in}}{\pgfqpoint{5.541243in}{3.963477in}}%
\pgfusepath{clip}%
\pgfsetbuttcap%
\pgfsetmiterjoin%
\definecolor{currentfill}{rgb}{0.000000,0.000000,1.000000}%
\pgfsetfillcolor{currentfill}%
\pgfsetlinewidth{0.000000pt}%
\definecolor{currentstroke}{rgb}{0.000000,0.000000,0.000000}%
\pgfsetstrokecolor{currentstroke}%
\pgfsetstrokeopacity{0.000000}%
\pgfsetdash{}{0pt}%
\pgfpathmoveto{\pgfqpoint{4.659970in}{0.613486in}}%
\pgfpathlineto{\pgfqpoint{4.669975in}{0.613486in}}%
\pgfpathlineto{\pgfqpoint{4.669975in}{2.611074in}}%
\pgfpathlineto{\pgfqpoint{4.659970in}{2.611074in}}%
\pgfpathlineto{\pgfqpoint{4.659970in}{0.613486in}}%
\pgfpathclose%
\pgfusepath{fill}%
\end{pgfscope}%
\begin{pgfscope}%
\pgfpathrectangle{\pgfqpoint{0.693757in}{0.613486in}}{\pgfqpoint{5.541243in}{3.963477in}}%
\pgfusepath{clip}%
\pgfsetbuttcap%
\pgfsetmiterjoin%
\definecolor{currentfill}{rgb}{0.000000,0.000000,1.000000}%
\pgfsetfillcolor{currentfill}%
\pgfsetlinewidth{0.000000pt}%
\definecolor{currentstroke}{rgb}{0.000000,0.000000,0.000000}%
\pgfsetstrokecolor{currentstroke}%
\pgfsetstrokeopacity{0.000000}%
\pgfsetdash{}{0pt}%
\pgfpathmoveto{\pgfqpoint{4.672477in}{0.613486in}}%
\pgfpathlineto{\pgfqpoint{4.682482in}{0.613486in}}%
\pgfpathlineto{\pgfqpoint{4.682482in}{2.595224in}}%
\pgfpathlineto{\pgfqpoint{4.672477in}{2.595224in}}%
\pgfpathlineto{\pgfqpoint{4.672477in}{0.613486in}}%
\pgfpathclose%
\pgfusepath{fill}%
\end{pgfscope}%
\begin{pgfscope}%
\pgfpathrectangle{\pgfqpoint{0.693757in}{0.613486in}}{\pgfqpoint{5.541243in}{3.963477in}}%
\pgfusepath{clip}%
\pgfsetbuttcap%
\pgfsetmiterjoin%
\definecolor{currentfill}{rgb}{0.000000,0.000000,1.000000}%
\pgfsetfillcolor{currentfill}%
\pgfsetlinewidth{0.000000pt}%
\definecolor{currentstroke}{rgb}{0.000000,0.000000,0.000000}%
\pgfsetstrokecolor{currentstroke}%
\pgfsetstrokeopacity{0.000000}%
\pgfsetdash{}{0pt}%
\pgfpathmoveto{\pgfqpoint{4.684983in}{0.613486in}}%
\pgfpathlineto{\pgfqpoint{4.694988in}{0.613486in}}%
\pgfpathlineto{\pgfqpoint{4.694988in}{1.612282in}}%
\pgfpathlineto{\pgfqpoint{4.684983in}{1.612282in}}%
\pgfpathlineto{\pgfqpoint{4.684983in}{0.613486in}}%
\pgfpathclose%
\pgfusepath{fill}%
\end{pgfscope}%
\begin{pgfscope}%
\pgfpathrectangle{\pgfqpoint{0.693757in}{0.613486in}}{\pgfqpoint{5.541243in}{3.963477in}}%
\pgfusepath{clip}%
\pgfsetbuttcap%
\pgfsetmiterjoin%
\definecolor{currentfill}{rgb}{0.000000,0.000000,1.000000}%
\pgfsetfillcolor{currentfill}%
\pgfsetlinewidth{0.000000pt}%
\definecolor{currentstroke}{rgb}{0.000000,0.000000,0.000000}%
\pgfsetstrokecolor{currentstroke}%
\pgfsetstrokeopacity{0.000000}%
\pgfsetdash{}{0pt}%
\pgfpathmoveto{\pgfqpoint{4.697489in}{0.613486in}}%
\pgfpathlineto{\pgfqpoint{4.707494in}{0.613486in}}%
\pgfpathlineto{\pgfqpoint{4.707494in}{1.604355in}}%
\pgfpathlineto{\pgfqpoint{4.697489in}{1.604355in}}%
\pgfpathlineto{\pgfqpoint{4.697489in}{0.613486in}}%
\pgfpathclose%
\pgfusepath{fill}%
\end{pgfscope}%
\begin{pgfscope}%
\pgfpathrectangle{\pgfqpoint{0.693757in}{0.613486in}}{\pgfqpoint{5.541243in}{3.963477in}}%
\pgfusepath{clip}%
\pgfsetbuttcap%
\pgfsetmiterjoin%
\definecolor{currentfill}{rgb}{0.000000,0.000000,1.000000}%
\pgfsetfillcolor{currentfill}%
\pgfsetlinewidth{0.000000pt}%
\definecolor{currentstroke}{rgb}{0.000000,0.000000,0.000000}%
\pgfsetstrokecolor{currentstroke}%
\pgfsetstrokeopacity{0.000000}%
\pgfsetdash{}{0pt}%
\pgfpathmoveto{\pgfqpoint{4.709995in}{0.613486in}}%
\pgfpathlineto{\pgfqpoint{4.720000in}{0.613486in}}%
\pgfpathlineto{\pgfqpoint{4.720000in}{2.611074in}}%
\pgfpathlineto{\pgfqpoint{4.709995in}{2.611074in}}%
\pgfpathlineto{\pgfqpoint{4.709995in}{0.613486in}}%
\pgfpathclose%
\pgfusepath{fill}%
\end{pgfscope}%
\begin{pgfscope}%
\pgfpathrectangle{\pgfqpoint{0.693757in}{0.613486in}}{\pgfqpoint{5.541243in}{3.963477in}}%
\pgfusepath{clip}%
\pgfsetbuttcap%
\pgfsetmiterjoin%
\definecolor{currentfill}{rgb}{0.000000,0.000000,1.000000}%
\pgfsetfillcolor{currentfill}%
\pgfsetlinewidth{0.000000pt}%
\definecolor{currentstroke}{rgb}{0.000000,0.000000,0.000000}%
\pgfsetstrokecolor{currentstroke}%
\pgfsetstrokeopacity{0.000000}%
\pgfsetdash{}{0pt}%
\pgfpathmoveto{\pgfqpoint{4.722501in}{0.613486in}}%
\pgfpathlineto{\pgfqpoint{4.732506in}{0.613486in}}%
\pgfpathlineto{\pgfqpoint{4.732506in}{2.595224in}}%
\pgfpathlineto{\pgfqpoint{4.722501in}{2.595224in}}%
\pgfpathlineto{\pgfqpoint{4.722501in}{0.613486in}}%
\pgfpathclose%
\pgfusepath{fill}%
\end{pgfscope}%
\begin{pgfscope}%
\pgfpathrectangle{\pgfqpoint{0.693757in}{0.613486in}}{\pgfqpoint{5.541243in}{3.963477in}}%
\pgfusepath{clip}%
\pgfsetbuttcap%
\pgfsetmiterjoin%
\definecolor{currentfill}{rgb}{0.000000,0.000000,1.000000}%
\pgfsetfillcolor{currentfill}%
\pgfsetlinewidth{0.000000pt}%
\definecolor{currentstroke}{rgb}{0.000000,0.000000,0.000000}%
\pgfsetstrokecolor{currentstroke}%
\pgfsetstrokeopacity{0.000000}%
\pgfsetdash{}{0pt}%
\pgfpathmoveto{\pgfqpoint{4.735008in}{0.613486in}}%
\pgfpathlineto{\pgfqpoint{4.745012in}{0.613486in}}%
\pgfpathlineto{\pgfqpoint{4.745012in}{1.612282in}}%
\pgfpathlineto{\pgfqpoint{4.735008in}{1.612282in}}%
\pgfpathlineto{\pgfqpoint{4.735008in}{0.613486in}}%
\pgfpathclose%
\pgfusepath{fill}%
\end{pgfscope}%
\begin{pgfscope}%
\pgfpathrectangle{\pgfqpoint{0.693757in}{0.613486in}}{\pgfqpoint{5.541243in}{3.963477in}}%
\pgfusepath{clip}%
\pgfsetbuttcap%
\pgfsetmiterjoin%
\definecolor{currentfill}{rgb}{0.000000,0.000000,1.000000}%
\pgfsetfillcolor{currentfill}%
\pgfsetlinewidth{0.000000pt}%
\definecolor{currentstroke}{rgb}{0.000000,0.000000,0.000000}%
\pgfsetstrokecolor{currentstroke}%
\pgfsetstrokeopacity{0.000000}%
\pgfsetdash{}{0pt}%
\pgfpathmoveto{\pgfqpoint{4.747514in}{0.613486in}}%
\pgfpathlineto{\pgfqpoint{4.757519in}{0.613486in}}%
\pgfpathlineto{\pgfqpoint{4.757519in}{1.604355in}}%
\pgfpathlineto{\pgfqpoint{4.747514in}{1.604355in}}%
\pgfpathlineto{\pgfqpoint{4.747514in}{0.613486in}}%
\pgfpathclose%
\pgfusepath{fill}%
\end{pgfscope}%
\begin{pgfscope}%
\pgfpathrectangle{\pgfqpoint{0.693757in}{0.613486in}}{\pgfqpoint{5.541243in}{3.963477in}}%
\pgfusepath{clip}%
\pgfsetbuttcap%
\pgfsetmiterjoin%
\definecolor{currentfill}{rgb}{0.000000,0.000000,1.000000}%
\pgfsetfillcolor{currentfill}%
\pgfsetlinewidth{0.000000pt}%
\definecolor{currentstroke}{rgb}{0.000000,0.000000,0.000000}%
\pgfsetstrokecolor{currentstroke}%
\pgfsetstrokeopacity{0.000000}%
\pgfsetdash{}{0pt}%
\pgfpathmoveto{\pgfqpoint{4.760020in}{0.613486in}}%
\pgfpathlineto{\pgfqpoint{4.770025in}{0.613486in}}%
\pgfpathlineto{\pgfqpoint{4.770025in}{2.611074in}}%
\pgfpathlineto{\pgfqpoint{4.760020in}{2.611074in}}%
\pgfpathlineto{\pgfqpoint{4.760020in}{0.613486in}}%
\pgfpathclose%
\pgfusepath{fill}%
\end{pgfscope}%
\begin{pgfscope}%
\pgfpathrectangle{\pgfqpoint{0.693757in}{0.613486in}}{\pgfqpoint{5.541243in}{3.963477in}}%
\pgfusepath{clip}%
\pgfsetbuttcap%
\pgfsetmiterjoin%
\definecolor{currentfill}{rgb}{0.000000,0.000000,1.000000}%
\pgfsetfillcolor{currentfill}%
\pgfsetlinewidth{0.000000pt}%
\definecolor{currentstroke}{rgb}{0.000000,0.000000,0.000000}%
\pgfsetstrokecolor{currentstroke}%
\pgfsetstrokeopacity{0.000000}%
\pgfsetdash{}{0pt}%
\pgfpathmoveto{\pgfqpoint{4.772526in}{0.613486in}}%
\pgfpathlineto{\pgfqpoint{4.782531in}{0.613486in}}%
\pgfpathlineto{\pgfqpoint{4.782531in}{2.595224in}}%
\pgfpathlineto{\pgfqpoint{4.772526in}{2.595224in}}%
\pgfpathlineto{\pgfqpoint{4.772526in}{0.613486in}}%
\pgfpathclose%
\pgfusepath{fill}%
\end{pgfscope}%
\begin{pgfscope}%
\pgfpathrectangle{\pgfqpoint{0.693757in}{0.613486in}}{\pgfqpoint{5.541243in}{3.963477in}}%
\pgfusepath{clip}%
\pgfsetbuttcap%
\pgfsetmiterjoin%
\definecolor{currentfill}{rgb}{0.000000,0.000000,1.000000}%
\pgfsetfillcolor{currentfill}%
\pgfsetlinewidth{0.000000pt}%
\definecolor{currentstroke}{rgb}{0.000000,0.000000,0.000000}%
\pgfsetstrokecolor{currentstroke}%
\pgfsetstrokeopacity{0.000000}%
\pgfsetdash{}{0pt}%
\pgfpathmoveto{\pgfqpoint{4.785032in}{0.613486in}}%
\pgfpathlineto{\pgfqpoint{4.795037in}{0.613486in}}%
\pgfpathlineto{\pgfqpoint{4.795037in}{1.612282in}}%
\pgfpathlineto{\pgfqpoint{4.785032in}{1.612282in}}%
\pgfpathlineto{\pgfqpoint{4.785032in}{0.613486in}}%
\pgfpathclose%
\pgfusepath{fill}%
\end{pgfscope}%
\begin{pgfscope}%
\pgfpathrectangle{\pgfqpoint{0.693757in}{0.613486in}}{\pgfqpoint{5.541243in}{3.963477in}}%
\pgfusepath{clip}%
\pgfsetbuttcap%
\pgfsetmiterjoin%
\definecolor{currentfill}{rgb}{0.000000,0.000000,1.000000}%
\pgfsetfillcolor{currentfill}%
\pgfsetlinewidth{0.000000pt}%
\definecolor{currentstroke}{rgb}{0.000000,0.000000,0.000000}%
\pgfsetstrokecolor{currentstroke}%
\pgfsetstrokeopacity{0.000000}%
\pgfsetdash{}{0pt}%
\pgfpathmoveto{\pgfqpoint{4.797538in}{0.613486in}}%
\pgfpathlineto{\pgfqpoint{4.807543in}{0.613486in}}%
\pgfpathlineto{\pgfqpoint{4.807543in}{1.604355in}}%
\pgfpathlineto{\pgfqpoint{4.797538in}{1.604355in}}%
\pgfpathlineto{\pgfqpoint{4.797538in}{0.613486in}}%
\pgfpathclose%
\pgfusepath{fill}%
\end{pgfscope}%
\begin{pgfscope}%
\pgfpathrectangle{\pgfqpoint{0.693757in}{0.613486in}}{\pgfqpoint{5.541243in}{3.963477in}}%
\pgfusepath{clip}%
\pgfsetbuttcap%
\pgfsetmiterjoin%
\definecolor{currentfill}{rgb}{0.000000,0.000000,1.000000}%
\pgfsetfillcolor{currentfill}%
\pgfsetlinewidth{0.000000pt}%
\definecolor{currentstroke}{rgb}{0.000000,0.000000,0.000000}%
\pgfsetstrokecolor{currentstroke}%
\pgfsetstrokeopacity{0.000000}%
\pgfsetdash{}{0pt}%
\pgfpathmoveto{\pgfqpoint{4.810045in}{0.613486in}}%
\pgfpathlineto{\pgfqpoint{4.820050in}{0.613486in}}%
\pgfpathlineto{\pgfqpoint{4.820050in}{2.611074in}}%
\pgfpathlineto{\pgfqpoint{4.810045in}{2.611074in}}%
\pgfpathlineto{\pgfqpoint{4.810045in}{0.613486in}}%
\pgfpathclose%
\pgfusepath{fill}%
\end{pgfscope}%
\begin{pgfscope}%
\pgfpathrectangle{\pgfqpoint{0.693757in}{0.613486in}}{\pgfqpoint{5.541243in}{3.963477in}}%
\pgfusepath{clip}%
\pgfsetbuttcap%
\pgfsetmiterjoin%
\definecolor{currentfill}{rgb}{0.000000,0.000000,1.000000}%
\pgfsetfillcolor{currentfill}%
\pgfsetlinewidth{0.000000pt}%
\definecolor{currentstroke}{rgb}{0.000000,0.000000,0.000000}%
\pgfsetstrokecolor{currentstroke}%
\pgfsetstrokeopacity{0.000000}%
\pgfsetdash{}{0pt}%
\pgfpathmoveto{\pgfqpoint{4.822551in}{0.613486in}}%
\pgfpathlineto{\pgfqpoint{4.832556in}{0.613486in}}%
\pgfpathlineto{\pgfqpoint{4.832556in}{2.595224in}}%
\pgfpathlineto{\pgfqpoint{4.822551in}{2.595224in}}%
\pgfpathlineto{\pgfqpoint{4.822551in}{0.613486in}}%
\pgfpathclose%
\pgfusepath{fill}%
\end{pgfscope}%
\begin{pgfscope}%
\pgfpathrectangle{\pgfqpoint{0.693757in}{0.613486in}}{\pgfqpoint{5.541243in}{3.963477in}}%
\pgfusepath{clip}%
\pgfsetbuttcap%
\pgfsetmiterjoin%
\definecolor{currentfill}{rgb}{0.000000,0.000000,1.000000}%
\pgfsetfillcolor{currentfill}%
\pgfsetlinewidth{0.000000pt}%
\definecolor{currentstroke}{rgb}{0.000000,0.000000,0.000000}%
\pgfsetstrokecolor{currentstroke}%
\pgfsetstrokeopacity{0.000000}%
\pgfsetdash{}{0pt}%
\pgfpathmoveto{\pgfqpoint{4.835057in}{0.613486in}}%
\pgfpathlineto{\pgfqpoint{4.845062in}{0.613486in}}%
\pgfpathlineto{\pgfqpoint{4.845062in}{1.612282in}}%
\pgfpathlineto{\pgfqpoint{4.835057in}{1.612282in}}%
\pgfpathlineto{\pgfqpoint{4.835057in}{0.613486in}}%
\pgfpathclose%
\pgfusepath{fill}%
\end{pgfscope}%
\begin{pgfscope}%
\pgfpathrectangle{\pgfqpoint{0.693757in}{0.613486in}}{\pgfqpoint{5.541243in}{3.963477in}}%
\pgfusepath{clip}%
\pgfsetbuttcap%
\pgfsetmiterjoin%
\definecolor{currentfill}{rgb}{0.000000,0.000000,1.000000}%
\pgfsetfillcolor{currentfill}%
\pgfsetlinewidth{0.000000pt}%
\definecolor{currentstroke}{rgb}{0.000000,0.000000,0.000000}%
\pgfsetstrokecolor{currentstroke}%
\pgfsetstrokeopacity{0.000000}%
\pgfsetdash{}{0pt}%
\pgfpathmoveto{\pgfqpoint{4.847563in}{0.613486in}}%
\pgfpathlineto{\pgfqpoint{4.857568in}{0.613486in}}%
\pgfpathlineto{\pgfqpoint{4.857568in}{1.604355in}}%
\pgfpathlineto{\pgfqpoint{4.847563in}{1.604355in}}%
\pgfpathlineto{\pgfqpoint{4.847563in}{0.613486in}}%
\pgfpathclose%
\pgfusepath{fill}%
\end{pgfscope}%
\begin{pgfscope}%
\pgfpathrectangle{\pgfqpoint{0.693757in}{0.613486in}}{\pgfqpoint{5.541243in}{3.963477in}}%
\pgfusepath{clip}%
\pgfsetbuttcap%
\pgfsetmiterjoin%
\definecolor{currentfill}{rgb}{0.000000,0.000000,1.000000}%
\pgfsetfillcolor{currentfill}%
\pgfsetlinewidth{0.000000pt}%
\definecolor{currentstroke}{rgb}{0.000000,0.000000,0.000000}%
\pgfsetstrokecolor{currentstroke}%
\pgfsetstrokeopacity{0.000000}%
\pgfsetdash{}{0pt}%
\pgfpathmoveto{\pgfqpoint{4.860069in}{0.613486in}}%
\pgfpathlineto{\pgfqpoint{4.870074in}{0.613486in}}%
\pgfpathlineto{\pgfqpoint{4.870074in}{2.611074in}}%
\pgfpathlineto{\pgfqpoint{4.860069in}{2.611074in}}%
\pgfpathlineto{\pgfqpoint{4.860069in}{0.613486in}}%
\pgfpathclose%
\pgfusepath{fill}%
\end{pgfscope}%
\begin{pgfscope}%
\pgfpathrectangle{\pgfqpoint{0.693757in}{0.613486in}}{\pgfqpoint{5.541243in}{3.963477in}}%
\pgfusepath{clip}%
\pgfsetbuttcap%
\pgfsetmiterjoin%
\definecolor{currentfill}{rgb}{0.000000,0.000000,1.000000}%
\pgfsetfillcolor{currentfill}%
\pgfsetlinewidth{0.000000pt}%
\definecolor{currentstroke}{rgb}{0.000000,0.000000,0.000000}%
\pgfsetstrokecolor{currentstroke}%
\pgfsetstrokeopacity{0.000000}%
\pgfsetdash{}{0pt}%
\pgfpathmoveto{\pgfqpoint{4.872576in}{0.613486in}}%
\pgfpathlineto{\pgfqpoint{4.882581in}{0.613486in}}%
\pgfpathlineto{\pgfqpoint{4.882581in}{2.595224in}}%
\pgfpathlineto{\pgfqpoint{4.872576in}{2.595224in}}%
\pgfpathlineto{\pgfqpoint{4.872576in}{0.613486in}}%
\pgfpathclose%
\pgfusepath{fill}%
\end{pgfscope}%
\begin{pgfscope}%
\pgfpathrectangle{\pgfqpoint{0.693757in}{0.613486in}}{\pgfqpoint{5.541243in}{3.963477in}}%
\pgfusepath{clip}%
\pgfsetbuttcap%
\pgfsetmiterjoin%
\definecolor{currentfill}{rgb}{0.000000,0.000000,1.000000}%
\pgfsetfillcolor{currentfill}%
\pgfsetlinewidth{0.000000pt}%
\definecolor{currentstroke}{rgb}{0.000000,0.000000,0.000000}%
\pgfsetstrokecolor{currentstroke}%
\pgfsetstrokeopacity{0.000000}%
\pgfsetdash{}{0pt}%
\pgfpathmoveto{\pgfqpoint{4.885082in}{0.613486in}}%
\pgfpathlineto{\pgfqpoint{4.895087in}{0.613486in}}%
\pgfpathlineto{\pgfqpoint{4.895087in}{1.612282in}}%
\pgfpathlineto{\pgfqpoint{4.885082in}{1.612282in}}%
\pgfpathlineto{\pgfqpoint{4.885082in}{0.613486in}}%
\pgfpathclose%
\pgfusepath{fill}%
\end{pgfscope}%
\begin{pgfscope}%
\pgfpathrectangle{\pgfqpoint{0.693757in}{0.613486in}}{\pgfqpoint{5.541243in}{3.963477in}}%
\pgfusepath{clip}%
\pgfsetbuttcap%
\pgfsetmiterjoin%
\definecolor{currentfill}{rgb}{0.000000,0.000000,1.000000}%
\pgfsetfillcolor{currentfill}%
\pgfsetlinewidth{0.000000pt}%
\definecolor{currentstroke}{rgb}{0.000000,0.000000,0.000000}%
\pgfsetstrokecolor{currentstroke}%
\pgfsetstrokeopacity{0.000000}%
\pgfsetdash{}{0pt}%
\pgfpathmoveto{\pgfqpoint{4.897588in}{0.613486in}}%
\pgfpathlineto{\pgfqpoint{4.907593in}{0.613486in}}%
\pgfpathlineto{\pgfqpoint{4.907593in}{1.604355in}}%
\pgfpathlineto{\pgfqpoint{4.897588in}{1.604355in}}%
\pgfpathlineto{\pgfqpoint{4.897588in}{0.613486in}}%
\pgfpathclose%
\pgfusepath{fill}%
\end{pgfscope}%
\begin{pgfscope}%
\pgfpathrectangle{\pgfqpoint{0.693757in}{0.613486in}}{\pgfqpoint{5.541243in}{3.963477in}}%
\pgfusepath{clip}%
\pgfsetbuttcap%
\pgfsetmiterjoin%
\definecolor{currentfill}{rgb}{0.000000,0.000000,1.000000}%
\pgfsetfillcolor{currentfill}%
\pgfsetlinewidth{0.000000pt}%
\definecolor{currentstroke}{rgb}{0.000000,0.000000,0.000000}%
\pgfsetstrokecolor{currentstroke}%
\pgfsetstrokeopacity{0.000000}%
\pgfsetdash{}{0pt}%
\pgfpathmoveto{\pgfqpoint{4.910094in}{0.613486in}}%
\pgfpathlineto{\pgfqpoint{4.920099in}{0.613486in}}%
\pgfpathlineto{\pgfqpoint{4.920099in}{2.611074in}}%
\pgfpathlineto{\pgfqpoint{4.910094in}{2.611074in}}%
\pgfpathlineto{\pgfqpoint{4.910094in}{0.613486in}}%
\pgfpathclose%
\pgfusepath{fill}%
\end{pgfscope}%
\begin{pgfscope}%
\pgfpathrectangle{\pgfqpoint{0.693757in}{0.613486in}}{\pgfqpoint{5.541243in}{3.963477in}}%
\pgfusepath{clip}%
\pgfsetbuttcap%
\pgfsetmiterjoin%
\definecolor{currentfill}{rgb}{0.000000,0.000000,1.000000}%
\pgfsetfillcolor{currentfill}%
\pgfsetlinewidth{0.000000pt}%
\definecolor{currentstroke}{rgb}{0.000000,0.000000,0.000000}%
\pgfsetstrokecolor{currentstroke}%
\pgfsetstrokeopacity{0.000000}%
\pgfsetdash{}{0pt}%
\pgfpathmoveto{\pgfqpoint{4.922600in}{0.613486in}}%
\pgfpathlineto{\pgfqpoint{4.932605in}{0.613486in}}%
\pgfpathlineto{\pgfqpoint{4.932605in}{2.595224in}}%
\pgfpathlineto{\pgfqpoint{4.922600in}{2.595224in}}%
\pgfpathlineto{\pgfqpoint{4.922600in}{0.613486in}}%
\pgfpathclose%
\pgfusepath{fill}%
\end{pgfscope}%
\begin{pgfscope}%
\pgfpathrectangle{\pgfqpoint{0.693757in}{0.613486in}}{\pgfqpoint{5.541243in}{3.963477in}}%
\pgfusepath{clip}%
\pgfsetbuttcap%
\pgfsetmiterjoin%
\definecolor{currentfill}{rgb}{0.000000,0.000000,1.000000}%
\pgfsetfillcolor{currentfill}%
\pgfsetlinewidth{0.000000pt}%
\definecolor{currentstroke}{rgb}{0.000000,0.000000,0.000000}%
\pgfsetstrokecolor{currentstroke}%
\pgfsetstrokeopacity{0.000000}%
\pgfsetdash{}{0pt}%
\pgfpathmoveto{\pgfqpoint{4.935107in}{0.613486in}}%
\pgfpathlineto{\pgfqpoint{4.945112in}{0.613486in}}%
\pgfpathlineto{\pgfqpoint{4.945112in}{1.612282in}}%
\pgfpathlineto{\pgfqpoint{4.935107in}{1.612282in}}%
\pgfpathlineto{\pgfqpoint{4.935107in}{0.613486in}}%
\pgfpathclose%
\pgfusepath{fill}%
\end{pgfscope}%
\begin{pgfscope}%
\pgfpathrectangle{\pgfqpoint{0.693757in}{0.613486in}}{\pgfqpoint{5.541243in}{3.963477in}}%
\pgfusepath{clip}%
\pgfsetbuttcap%
\pgfsetmiterjoin%
\definecolor{currentfill}{rgb}{0.000000,0.000000,1.000000}%
\pgfsetfillcolor{currentfill}%
\pgfsetlinewidth{0.000000pt}%
\definecolor{currentstroke}{rgb}{0.000000,0.000000,0.000000}%
\pgfsetstrokecolor{currentstroke}%
\pgfsetstrokeopacity{0.000000}%
\pgfsetdash{}{0pt}%
\pgfpathmoveto{\pgfqpoint{4.947613in}{0.613486in}}%
\pgfpathlineto{\pgfqpoint{4.957618in}{0.613486in}}%
\pgfpathlineto{\pgfqpoint{4.957618in}{1.604355in}}%
\pgfpathlineto{\pgfqpoint{4.947613in}{1.604355in}}%
\pgfpathlineto{\pgfqpoint{4.947613in}{0.613486in}}%
\pgfpathclose%
\pgfusepath{fill}%
\end{pgfscope}%
\begin{pgfscope}%
\pgfpathrectangle{\pgfqpoint{0.693757in}{0.613486in}}{\pgfqpoint{5.541243in}{3.963477in}}%
\pgfusepath{clip}%
\pgfsetbuttcap%
\pgfsetmiterjoin%
\definecolor{currentfill}{rgb}{0.000000,0.000000,1.000000}%
\pgfsetfillcolor{currentfill}%
\pgfsetlinewidth{0.000000pt}%
\definecolor{currentstroke}{rgb}{0.000000,0.000000,0.000000}%
\pgfsetstrokecolor{currentstroke}%
\pgfsetstrokeopacity{0.000000}%
\pgfsetdash{}{0pt}%
\pgfpathmoveto{\pgfqpoint{4.960119in}{0.613486in}}%
\pgfpathlineto{\pgfqpoint{4.970124in}{0.613486in}}%
\pgfpathlineto{\pgfqpoint{4.970124in}{2.611074in}}%
\pgfpathlineto{\pgfqpoint{4.960119in}{2.611074in}}%
\pgfpathlineto{\pgfqpoint{4.960119in}{0.613486in}}%
\pgfpathclose%
\pgfusepath{fill}%
\end{pgfscope}%
\begin{pgfscope}%
\pgfpathrectangle{\pgfqpoint{0.693757in}{0.613486in}}{\pgfqpoint{5.541243in}{3.963477in}}%
\pgfusepath{clip}%
\pgfsetbuttcap%
\pgfsetmiterjoin%
\definecolor{currentfill}{rgb}{0.000000,0.000000,1.000000}%
\pgfsetfillcolor{currentfill}%
\pgfsetlinewidth{0.000000pt}%
\definecolor{currentstroke}{rgb}{0.000000,0.000000,0.000000}%
\pgfsetstrokecolor{currentstroke}%
\pgfsetstrokeopacity{0.000000}%
\pgfsetdash{}{0pt}%
\pgfpathmoveto{\pgfqpoint{4.972625in}{0.613486in}}%
\pgfpathlineto{\pgfqpoint{4.982630in}{0.613486in}}%
\pgfpathlineto{\pgfqpoint{4.982630in}{2.595224in}}%
\pgfpathlineto{\pgfqpoint{4.972625in}{2.595224in}}%
\pgfpathlineto{\pgfqpoint{4.972625in}{0.613486in}}%
\pgfpathclose%
\pgfusepath{fill}%
\end{pgfscope}%
\begin{pgfscope}%
\pgfpathrectangle{\pgfqpoint{0.693757in}{0.613486in}}{\pgfqpoint{5.541243in}{3.963477in}}%
\pgfusepath{clip}%
\pgfsetbuttcap%
\pgfsetmiterjoin%
\definecolor{currentfill}{rgb}{0.000000,0.000000,1.000000}%
\pgfsetfillcolor{currentfill}%
\pgfsetlinewidth{0.000000pt}%
\definecolor{currentstroke}{rgb}{0.000000,0.000000,0.000000}%
\pgfsetstrokecolor{currentstroke}%
\pgfsetstrokeopacity{0.000000}%
\pgfsetdash{}{0pt}%
\pgfpathmoveto{\pgfqpoint{4.985131in}{0.613486in}}%
\pgfpathlineto{\pgfqpoint{4.995136in}{0.613486in}}%
\pgfpathlineto{\pgfqpoint{4.995136in}{1.612282in}}%
\pgfpathlineto{\pgfqpoint{4.985131in}{1.612282in}}%
\pgfpathlineto{\pgfqpoint{4.985131in}{0.613486in}}%
\pgfpathclose%
\pgfusepath{fill}%
\end{pgfscope}%
\begin{pgfscope}%
\pgfpathrectangle{\pgfqpoint{0.693757in}{0.613486in}}{\pgfqpoint{5.541243in}{3.963477in}}%
\pgfusepath{clip}%
\pgfsetbuttcap%
\pgfsetmiterjoin%
\definecolor{currentfill}{rgb}{0.000000,0.000000,1.000000}%
\pgfsetfillcolor{currentfill}%
\pgfsetlinewidth{0.000000pt}%
\definecolor{currentstroke}{rgb}{0.000000,0.000000,0.000000}%
\pgfsetstrokecolor{currentstroke}%
\pgfsetstrokeopacity{0.000000}%
\pgfsetdash{}{0pt}%
\pgfpathmoveto{\pgfqpoint{4.997638in}{0.613486in}}%
\pgfpathlineto{\pgfqpoint{5.007642in}{0.613486in}}%
\pgfpathlineto{\pgfqpoint{5.007642in}{1.604355in}}%
\pgfpathlineto{\pgfqpoint{4.997638in}{1.604355in}}%
\pgfpathlineto{\pgfqpoint{4.997638in}{0.613486in}}%
\pgfpathclose%
\pgfusepath{fill}%
\end{pgfscope}%
\begin{pgfscope}%
\pgfpathrectangle{\pgfqpoint{0.693757in}{0.613486in}}{\pgfqpoint{5.541243in}{3.963477in}}%
\pgfusepath{clip}%
\pgfsetbuttcap%
\pgfsetmiterjoin%
\definecolor{currentfill}{rgb}{0.000000,0.000000,1.000000}%
\pgfsetfillcolor{currentfill}%
\pgfsetlinewidth{0.000000pt}%
\definecolor{currentstroke}{rgb}{0.000000,0.000000,0.000000}%
\pgfsetstrokecolor{currentstroke}%
\pgfsetstrokeopacity{0.000000}%
\pgfsetdash{}{0pt}%
\pgfpathmoveto{\pgfqpoint{5.010144in}{0.613486in}}%
\pgfpathlineto{\pgfqpoint{5.020149in}{0.613486in}}%
\pgfpathlineto{\pgfqpoint{5.020149in}{2.611074in}}%
\pgfpathlineto{\pgfqpoint{5.010144in}{2.611074in}}%
\pgfpathlineto{\pgfqpoint{5.010144in}{0.613486in}}%
\pgfpathclose%
\pgfusepath{fill}%
\end{pgfscope}%
\begin{pgfscope}%
\pgfpathrectangle{\pgfqpoint{0.693757in}{0.613486in}}{\pgfqpoint{5.541243in}{3.963477in}}%
\pgfusepath{clip}%
\pgfsetbuttcap%
\pgfsetmiterjoin%
\definecolor{currentfill}{rgb}{0.000000,0.000000,1.000000}%
\pgfsetfillcolor{currentfill}%
\pgfsetlinewidth{0.000000pt}%
\definecolor{currentstroke}{rgb}{0.000000,0.000000,0.000000}%
\pgfsetstrokecolor{currentstroke}%
\pgfsetstrokeopacity{0.000000}%
\pgfsetdash{}{0pt}%
\pgfpathmoveto{\pgfqpoint{5.022650in}{0.613486in}}%
\pgfpathlineto{\pgfqpoint{5.032655in}{0.613486in}}%
\pgfpathlineto{\pgfqpoint{5.032655in}{2.595224in}}%
\pgfpathlineto{\pgfqpoint{5.022650in}{2.595224in}}%
\pgfpathlineto{\pgfqpoint{5.022650in}{0.613486in}}%
\pgfpathclose%
\pgfusepath{fill}%
\end{pgfscope}%
\begin{pgfscope}%
\pgfpathrectangle{\pgfqpoint{0.693757in}{0.613486in}}{\pgfqpoint{5.541243in}{3.963477in}}%
\pgfusepath{clip}%
\pgfsetbuttcap%
\pgfsetmiterjoin%
\definecolor{currentfill}{rgb}{0.000000,0.000000,1.000000}%
\pgfsetfillcolor{currentfill}%
\pgfsetlinewidth{0.000000pt}%
\definecolor{currentstroke}{rgb}{0.000000,0.000000,0.000000}%
\pgfsetstrokecolor{currentstroke}%
\pgfsetstrokeopacity{0.000000}%
\pgfsetdash{}{0pt}%
\pgfpathmoveto{\pgfqpoint{5.035156in}{0.613486in}}%
\pgfpathlineto{\pgfqpoint{5.045161in}{0.613486in}}%
\pgfpathlineto{\pgfqpoint{5.045161in}{1.612282in}}%
\pgfpathlineto{\pgfqpoint{5.035156in}{1.612282in}}%
\pgfpathlineto{\pgfqpoint{5.035156in}{0.613486in}}%
\pgfpathclose%
\pgfusepath{fill}%
\end{pgfscope}%
\begin{pgfscope}%
\pgfpathrectangle{\pgfqpoint{0.693757in}{0.613486in}}{\pgfqpoint{5.541243in}{3.963477in}}%
\pgfusepath{clip}%
\pgfsetbuttcap%
\pgfsetmiterjoin%
\definecolor{currentfill}{rgb}{0.000000,0.000000,1.000000}%
\pgfsetfillcolor{currentfill}%
\pgfsetlinewidth{0.000000pt}%
\definecolor{currentstroke}{rgb}{0.000000,0.000000,0.000000}%
\pgfsetstrokecolor{currentstroke}%
\pgfsetstrokeopacity{0.000000}%
\pgfsetdash{}{0pt}%
\pgfpathmoveto{\pgfqpoint{5.047662in}{0.613486in}}%
\pgfpathlineto{\pgfqpoint{5.057667in}{0.613486in}}%
\pgfpathlineto{\pgfqpoint{5.057667in}{1.604355in}}%
\pgfpathlineto{\pgfqpoint{5.047662in}{1.604355in}}%
\pgfpathlineto{\pgfqpoint{5.047662in}{0.613486in}}%
\pgfpathclose%
\pgfusepath{fill}%
\end{pgfscope}%
\begin{pgfscope}%
\pgfpathrectangle{\pgfqpoint{0.693757in}{0.613486in}}{\pgfqpoint{5.541243in}{3.963477in}}%
\pgfusepath{clip}%
\pgfsetbuttcap%
\pgfsetmiterjoin%
\definecolor{currentfill}{rgb}{0.000000,0.000000,1.000000}%
\pgfsetfillcolor{currentfill}%
\pgfsetlinewidth{0.000000pt}%
\definecolor{currentstroke}{rgb}{0.000000,0.000000,0.000000}%
\pgfsetstrokecolor{currentstroke}%
\pgfsetstrokeopacity{0.000000}%
\pgfsetdash{}{0pt}%
\pgfpathmoveto{\pgfqpoint{5.060168in}{0.613486in}}%
\pgfpathlineto{\pgfqpoint{5.070173in}{0.613486in}}%
\pgfpathlineto{\pgfqpoint{5.070173in}{2.611074in}}%
\pgfpathlineto{\pgfqpoint{5.060168in}{2.611074in}}%
\pgfpathlineto{\pgfqpoint{5.060168in}{0.613486in}}%
\pgfpathclose%
\pgfusepath{fill}%
\end{pgfscope}%
\begin{pgfscope}%
\pgfpathrectangle{\pgfqpoint{0.693757in}{0.613486in}}{\pgfqpoint{5.541243in}{3.963477in}}%
\pgfusepath{clip}%
\pgfsetbuttcap%
\pgfsetmiterjoin%
\definecolor{currentfill}{rgb}{0.000000,0.000000,1.000000}%
\pgfsetfillcolor{currentfill}%
\pgfsetlinewidth{0.000000pt}%
\definecolor{currentstroke}{rgb}{0.000000,0.000000,0.000000}%
\pgfsetstrokecolor{currentstroke}%
\pgfsetstrokeopacity{0.000000}%
\pgfsetdash{}{0pt}%
\pgfpathmoveto{\pgfqpoint{5.072675in}{0.613486in}}%
\pgfpathlineto{\pgfqpoint{5.082680in}{0.613486in}}%
\pgfpathlineto{\pgfqpoint{5.082680in}{2.595224in}}%
\pgfpathlineto{\pgfqpoint{5.072675in}{2.595224in}}%
\pgfpathlineto{\pgfqpoint{5.072675in}{0.613486in}}%
\pgfpathclose%
\pgfusepath{fill}%
\end{pgfscope}%
\begin{pgfscope}%
\pgfpathrectangle{\pgfqpoint{0.693757in}{0.613486in}}{\pgfqpoint{5.541243in}{3.963477in}}%
\pgfusepath{clip}%
\pgfsetbuttcap%
\pgfsetmiterjoin%
\definecolor{currentfill}{rgb}{0.000000,0.000000,1.000000}%
\pgfsetfillcolor{currentfill}%
\pgfsetlinewidth{0.000000pt}%
\definecolor{currentstroke}{rgb}{0.000000,0.000000,0.000000}%
\pgfsetstrokecolor{currentstroke}%
\pgfsetstrokeopacity{0.000000}%
\pgfsetdash{}{0pt}%
\pgfpathmoveto{\pgfqpoint{5.085181in}{0.613486in}}%
\pgfpathlineto{\pgfqpoint{5.095186in}{0.613486in}}%
\pgfpathlineto{\pgfqpoint{5.095186in}{1.612282in}}%
\pgfpathlineto{\pgfqpoint{5.085181in}{1.612282in}}%
\pgfpathlineto{\pgfqpoint{5.085181in}{0.613486in}}%
\pgfpathclose%
\pgfusepath{fill}%
\end{pgfscope}%
\begin{pgfscope}%
\pgfpathrectangle{\pgfqpoint{0.693757in}{0.613486in}}{\pgfqpoint{5.541243in}{3.963477in}}%
\pgfusepath{clip}%
\pgfsetbuttcap%
\pgfsetmiterjoin%
\definecolor{currentfill}{rgb}{0.000000,0.000000,1.000000}%
\pgfsetfillcolor{currentfill}%
\pgfsetlinewidth{0.000000pt}%
\definecolor{currentstroke}{rgb}{0.000000,0.000000,0.000000}%
\pgfsetstrokecolor{currentstroke}%
\pgfsetstrokeopacity{0.000000}%
\pgfsetdash{}{0pt}%
\pgfpathmoveto{\pgfqpoint{5.097687in}{0.613486in}}%
\pgfpathlineto{\pgfqpoint{5.107692in}{0.613486in}}%
\pgfpathlineto{\pgfqpoint{5.107692in}{1.604355in}}%
\pgfpathlineto{\pgfqpoint{5.097687in}{1.604355in}}%
\pgfpathlineto{\pgfqpoint{5.097687in}{0.613486in}}%
\pgfpathclose%
\pgfusepath{fill}%
\end{pgfscope}%
\begin{pgfscope}%
\pgfpathrectangle{\pgfqpoint{0.693757in}{0.613486in}}{\pgfqpoint{5.541243in}{3.963477in}}%
\pgfusepath{clip}%
\pgfsetbuttcap%
\pgfsetmiterjoin%
\definecolor{currentfill}{rgb}{0.000000,0.000000,1.000000}%
\pgfsetfillcolor{currentfill}%
\pgfsetlinewidth{0.000000pt}%
\definecolor{currentstroke}{rgb}{0.000000,0.000000,0.000000}%
\pgfsetstrokecolor{currentstroke}%
\pgfsetstrokeopacity{0.000000}%
\pgfsetdash{}{0pt}%
\pgfpathmoveto{\pgfqpoint{5.110193in}{0.613486in}}%
\pgfpathlineto{\pgfqpoint{5.120198in}{0.613486in}}%
\pgfpathlineto{\pgfqpoint{5.120198in}{2.611074in}}%
\pgfpathlineto{\pgfqpoint{5.110193in}{2.611074in}}%
\pgfpathlineto{\pgfqpoint{5.110193in}{0.613486in}}%
\pgfpathclose%
\pgfusepath{fill}%
\end{pgfscope}%
\begin{pgfscope}%
\pgfpathrectangle{\pgfqpoint{0.693757in}{0.613486in}}{\pgfqpoint{5.541243in}{3.963477in}}%
\pgfusepath{clip}%
\pgfsetbuttcap%
\pgfsetmiterjoin%
\definecolor{currentfill}{rgb}{0.000000,0.000000,1.000000}%
\pgfsetfillcolor{currentfill}%
\pgfsetlinewidth{0.000000pt}%
\definecolor{currentstroke}{rgb}{0.000000,0.000000,0.000000}%
\pgfsetstrokecolor{currentstroke}%
\pgfsetstrokeopacity{0.000000}%
\pgfsetdash{}{0pt}%
\pgfpathmoveto{\pgfqpoint{5.122699in}{0.613486in}}%
\pgfpathlineto{\pgfqpoint{5.132704in}{0.613486in}}%
\pgfpathlineto{\pgfqpoint{5.132704in}{2.595224in}}%
\pgfpathlineto{\pgfqpoint{5.122699in}{2.595224in}}%
\pgfpathlineto{\pgfqpoint{5.122699in}{0.613486in}}%
\pgfpathclose%
\pgfusepath{fill}%
\end{pgfscope}%
\begin{pgfscope}%
\pgfpathrectangle{\pgfqpoint{0.693757in}{0.613486in}}{\pgfqpoint{5.541243in}{3.963477in}}%
\pgfusepath{clip}%
\pgfsetbuttcap%
\pgfsetmiterjoin%
\definecolor{currentfill}{rgb}{0.000000,0.000000,1.000000}%
\pgfsetfillcolor{currentfill}%
\pgfsetlinewidth{0.000000pt}%
\definecolor{currentstroke}{rgb}{0.000000,0.000000,0.000000}%
\pgfsetstrokecolor{currentstroke}%
\pgfsetstrokeopacity{0.000000}%
\pgfsetdash{}{0pt}%
\pgfpathmoveto{\pgfqpoint{5.135206in}{0.613486in}}%
\pgfpathlineto{\pgfqpoint{5.145211in}{0.613486in}}%
\pgfpathlineto{\pgfqpoint{5.145211in}{1.612282in}}%
\pgfpathlineto{\pgfqpoint{5.135206in}{1.612282in}}%
\pgfpathlineto{\pgfqpoint{5.135206in}{0.613486in}}%
\pgfpathclose%
\pgfusepath{fill}%
\end{pgfscope}%
\begin{pgfscope}%
\pgfpathrectangle{\pgfqpoint{0.693757in}{0.613486in}}{\pgfqpoint{5.541243in}{3.963477in}}%
\pgfusepath{clip}%
\pgfsetbuttcap%
\pgfsetmiterjoin%
\definecolor{currentfill}{rgb}{0.000000,0.000000,1.000000}%
\pgfsetfillcolor{currentfill}%
\pgfsetlinewidth{0.000000pt}%
\definecolor{currentstroke}{rgb}{0.000000,0.000000,0.000000}%
\pgfsetstrokecolor{currentstroke}%
\pgfsetstrokeopacity{0.000000}%
\pgfsetdash{}{0pt}%
\pgfpathmoveto{\pgfqpoint{5.147712in}{0.613486in}}%
\pgfpathlineto{\pgfqpoint{5.157717in}{0.613486in}}%
\pgfpathlineto{\pgfqpoint{5.157717in}{1.604355in}}%
\pgfpathlineto{\pgfqpoint{5.147712in}{1.604355in}}%
\pgfpathlineto{\pgfqpoint{5.147712in}{0.613486in}}%
\pgfpathclose%
\pgfusepath{fill}%
\end{pgfscope}%
\begin{pgfscope}%
\pgfpathrectangle{\pgfqpoint{0.693757in}{0.613486in}}{\pgfqpoint{5.541243in}{3.963477in}}%
\pgfusepath{clip}%
\pgfsetbuttcap%
\pgfsetmiterjoin%
\definecolor{currentfill}{rgb}{0.000000,0.000000,1.000000}%
\pgfsetfillcolor{currentfill}%
\pgfsetlinewidth{0.000000pt}%
\definecolor{currentstroke}{rgb}{0.000000,0.000000,0.000000}%
\pgfsetstrokecolor{currentstroke}%
\pgfsetstrokeopacity{0.000000}%
\pgfsetdash{}{0pt}%
\pgfpathmoveto{\pgfqpoint{5.160218in}{0.613486in}}%
\pgfpathlineto{\pgfqpoint{5.170223in}{0.613486in}}%
\pgfpathlineto{\pgfqpoint{5.170223in}{2.611074in}}%
\pgfpathlineto{\pgfqpoint{5.160218in}{2.611074in}}%
\pgfpathlineto{\pgfqpoint{5.160218in}{0.613486in}}%
\pgfpathclose%
\pgfusepath{fill}%
\end{pgfscope}%
\begin{pgfscope}%
\pgfpathrectangle{\pgfqpoint{0.693757in}{0.613486in}}{\pgfqpoint{5.541243in}{3.963477in}}%
\pgfusepath{clip}%
\pgfsetbuttcap%
\pgfsetmiterjoin%
\definecolor{currentfill}{rgb}{0.000000,0.000000,1.000000}%
\pgfsetfillcolor{currentfill}%
\pgfsetlinewidth{0.000000pt}%
\definecolor{currentstroke}{rgb}{0.000000,0.000000,0.000000}%
\pgfsetstrokecolor{currentstroke}%
\pgfsetstrokeopacity{0.000000}%
\pgfsetdash{}{0pt}%
\pgfpathmoveto{\pgfqpoint{5.172724in}{0.613486in}}%
\pgfpathlineto{\pgfqpoint{5.182729in}{0.613486in}}%
\pgfpathlineto{\pgfqpoint{5.182729in}{2.595224in}}%
\pgfpathlineto{\pgfqpoint{5.172724in}{2.595224in}}%
\pgfpathlineto{\pgfqpoint{5.172724in}{0.613486in}}%
\pgfpathclose%
\pgfusepath{fill}%
\end{pgfscope}%
\begin{pgfscope}%
\pgfpathrectangle{\pgfqpoint{0.693757in}{0.613486in}}{\pgfqpoint{5.541243in}{3.963477in}}%
\pgfusepath{clip}%
\pgfsetbuttcap%
\pgfsetmiterjoin%
\definecolor{currentfill}{rgb}{0.000000,0.000000,1.000000}%
\pgfsetfillcolor{currentfill}%
\pgfsetlinewidth{0.000000pt}%
\definecolor{currentstroke}{rgb}{0.000000,0.000000,0.000000}%
\pgfsetstrokecolor{currentstroke}%
\pgfsetstrokeopacity{0.000000}%
\pgfsetdash{}{0pt}%
\pgfpathmoveto{\pgfqpoint{5.185230in}{0.613486in}}%
\pgfpathlineto{\pgfqpoint{5.195235in}{0.613486in}}%
\pgfpathlineto{\pgfqpoint{5.195235in}{1.612282in}}%
\pgfpathlineto{\pgfqpoint{5.185230in}{1.612282in}}%
\pgfpathlineto{\pgfqpoint{5.185230in}{0.613486in}}%
\pgfpathclose%
\pgfusepath{fill}%
\end{pgfscope}%
\begin{pgfscope}%
\pgfpathrectangle{\pgfqpoint{0.693757in}{0.613486in}}{\pgfqpoint{5.541243in}{3.963477in}}%
\pgfusepath{clip}%
\pgfsetbuttcap%
\pgfsetmiterjoin%
\definecolor{currentfill}{rgb}{0.000000,0.000000,1.000000}%
\pgfsetfillcolor{currentfill}%
\pgfsetlinewidth{0.000000pt}%
\definecolor{currentstroke}{rgb}{0.000000,0.000000,0.000000}%
\pgfsetstrokecolor{currentstroke}%
\pgfsetstrokeopacity{0.000000}%
\pgfsetdash{}{0pt}%
\pgfpathmoveto{\pgfqpoint{5.197737in}{0.613486in}}%
\pgfpathlineto{\pgfqpoint{5.207742in}{0.613486in}}%
\pgfpathlineto{\pgfqpoint{5.207742in}{1.604355in}}%
\pgfpathlineto{\pgfqpoint{5.197737in}{1.604355in}}%
\pgfpathlineto{\pgfqpoint{5.197737in}{0.613486in}}%
\pgfpathclose%
\pgfusepath{fill}%
\end{pgfscope}%
\begin{pgfscope}%
\pgfpathrectangle{\pgfqpoint{0.693757in}{0.613486in}}{\pgfqpoint{5.541243in}{3.963477in}}%
\pgfusepath{clip}%
\pgfsetbuttcap%
\pgfsetmiterjoin%
\definecolor{currentfill}{rgb}{0.000000,0.000000,1.000000}%
\pgfsetfillcolor{currentfill}%
\pgfsetlinewidth{0.000000pt}%
\definecolor{currentstroke}{rgb}{0.000000,0.000000,0.000000}%
\pgfsetstrokecolor{currentstroke}%
\pgfsetstrokeopacity{0.000000}%
\pgfsetdash{}{0pt}%
\pgfpathmoveto{\pgfqpoint{5.210243in}{0.613486in}}%
\pgfpathlineto{\pgfqpoint{5.220248in}{0.613486in}}%
\pgfpathlineto{\pgfqpoint{5.220248in}{2.611074in}}%
\pgfpathlineto{\pgfqpoint{5.210243in}{2.611074in}}%
\pgfpathlineto{\pgfqpoint{5.210243in}{0.613486in}}%
\pgfpathclose%
\pgfusepath{fill}%
\end{pgfscope}%
\begin{pgfscope}%
\pgfpathrectangle{\pgfqpoint{0.693757in}{0.613486in}}{\pgfqpoint{5.541243in}{3.963477in}}%
\pgfusepath{clip}%
\pgfsetbuttcap%
\pgfsetmiterjoin%
\definecolor{currentfill}{rgb}{0.000000,0.000000,1.000000}%
\pgfsetfillcolor{currentfill}%
\pgfsetlinewidth{0.000000pt}%
\definecolor{currentstroke}{rgb}{0.000000,0.000000,0.000000}%
\pgfsetstrokecolor{currentstroke}%
\pgfsetstrokeopacity{0.000000}%
\pgfsetdash{}{0pt}%
\pgfpathmoveto{\pgfqpoint{5.222749in}{0.613486in}}%
\pgfpathlineto{\pgfqpoint{5.232754in}{0.613486in}}%
\pgfpathlineto{\pgfqpoint{5.232754in}{2.595224in}}%
\pgfpathlineto{\pgfqpoint{5.222749in}{2.595224in}}%
\pgfpathlineto{\pgfqpoint{5.222749in}{0.613486in}}%
\pgfpathclose%
\pgfusepath{fill}%
\end{pgfscope}%
\begin{pgfscope}%
\pgfpathrectangle{\pgfqpoint{0.693757in}{0.613486in}}{\pgfqpoint{5.541243in}{3.963477in}}%
\pgfusepath{clip}%
\pgfsetbuttcap%
\pgfsetmiterjoin%
\definecolor{currentfill}{rgb}{0.000000,0.000000,1.000000}%
\pgfsetfillcolor{currentfill}%
\pgfsetlinewidth{0.000000pt}%
\definecolor{currentstroke}{rgb}{0.000000,0.000000,0.000000}%
\pgfsetstrokecolor{currentstroke}%
\pgfsetstrokeopacity{0.000000}%
\pgfsetdash{}{0pt}%
\pgfpathmoveto{\pgfqpoint{5.235255in}{0.613486in}}%
\pgfpathlineto{\pgfqpoint{5.245260in}{0.613486in}}%
\pgfpathlineto{\pgfqpoint{5.245260in}{1.612282in}}%
\pgfpathlineto{\pgfqpoint{5.235255in}{1.612282in}}%
\pgfpathlineto{\pgfqpoint{5.235255in}{0.613486in}}%
\pgfpathclose%
\pgfusepath{fill}%
\end{pgfscope}%
\begin{pgfscope}%
\pgfpathrectangle{\pgfqpoint{0.693757in}{0.613486in}}{\pgfqpoint{5.541243in}{3.963477in}}%
\pgfusepath{clip}%
\pgfsetbuttcap%
\pgfsetmiterjoin%
\definecolor{currentfill}{rgb}{0.000000,0.000000,1.000000}%
\pgfsetfillcolor{currentfill}%
\pgfsetlinewidth{0.000000pt}%
\definecolor{currentstroke}{rgb}{0.000000,0.000000,0.000000}%
\pgfsetstrokecolor{currentstroke}%
\pgfsetstrokeopacity{0.000000}%
\pgfsetdash{}{0pt}%
\pgfpathmoveto{\pgfqpoint{5.247761in}{0.613486in}}%
\pgfpathlineto{\pgfqpoint{5.257766in}{0.613486in}}%
\pgfpathlineto{\pgfqpoint{5.257766in}{1.604355in}}%
\pgfpathlineto{\pgfqpoint{5.247761in}{1.604355in}}%
\pgfpathlineto{\pgfqpoint{5.247761in}{0.613486in}}%
\pgfpathclose%
\pgfusepath{fill}%
\end{pgfscope}%
\begin{pgfscope}%
\pgfpathrectangle{\pgfqpoint{0.693757in}{0.613486in}}{\pgfqpoint{5.541243in}{3.963477in}}%
\pgfusepath{clip}%
\pgfsetbuttcap%
\pgfsetmiterjoin%
\definecolor{currentfill}{rgb}{0.000000,0.000000,1.000000}%
\pgfsetfillcolor{currentfill}%
\pgfsetlinewidth{0.000000pt}%
\definecolor{currentstroke}{rgb}{0.000000,0.000000,0.000000}%
\pgfsetstrokecolor{currentstroke}%
\pgfsetstrokeopacity{0.000000}%
\pgfsetdash{}{0pt}%
\pgfpathmoveto{\pgfqpoint{5.260268in}{0.613486in}}%
\pgfpathlineto{\pgfqpoint{5.270272in}{0.613486in}}%
\pgfpathlineto{\pgfqpoint{5.270272in}{2.611074in}}%
\pgfpathlineto{\pgfqpoint{5.260268in}{2.611074in}}%
\pgfpathlineto{\pgfqpoint{5.260268in}{0.613486in}}%
\pgfpathclose%
\pgfusepath{fill}%
\end{pgfscope}%
\begin{pgfscope}%
\pgfpathrectangle{\pgfqpoint{0.693757in}{0.613486in}}{\pgfqpoint{5.541243in}{3.963477in}}%
\pgfusepath{clip}%
\pgfsetbuttcap%
\pgfsetmiterjoin%
\definecolor{currentfill}{rgb}{0.000000,0.000000,1.000000}%
\pgfsetfillcolor{currentfill}%
\pgfsetlinewidth{0.000000pt}%
\definecolor{currentstroke}{rgb}{0.000000,0.000000,0.000000}%
\pgfsetstrokecolor{currentstroke}%
\pgfsetstrokeopacity{0.000000}%
\pgfsetdash{}{0pt}%
\pgfpathmoveto{\pgfqpoint{5.272774in}{0.613486in}}%
\pgfpathlineto{\pgfqpoint{5.282779in}{0.613486in}}%
\pgfpathlineto{\pgfqpoint{5.282779in}{2.595224in}}%
\pgfpathlineto{\pgfqpoint{5.272774in}{2.595224in}}%
\pgfpathlineto{\pgfqpoint{5.272774in}{0.613486in}}%
\pgfpathclose%
\pgfusepath{fill}%
\end{pgfscope}%
\begin{pgfscope}%
\pgfpathrectangle{\pgfqpoint{0.693757in}{0.613486in}}{\pgfqpoint{5.541243in}{3.963477in}}%
\pgfusepath{clip}%
\pgfsetbuttcap%
\pgfsetmiterjoin%
\definecolor{currentfill}{rgb}{0.000000,0.000000,1.000000}%
\pgfsetfillcolor{currentfill}%
\pgfsetlinewidth{0.000000pt}%
\definecolor{currentstroke}{rgb}{0.000000,0.000000,0.000000}%
\pgfsetstrokecolor{currentstroke}%
\pgfsetstrokeopacity{0.000000}%
\pgfsetdash{}{0pt}%
\pgfpathmoveto{\pgfqpoint{5.285280in}{0.613486in}}%
\pgfpathlineto{\pgfqpoint{5.295285in}{0.613486in}}%
\pgfpathlineto{\pgfqpoint{5.295285in}{1.612282in}}%
\pgfpathlineto{\pgfqpoint{5.285280in}{1.612282in}}%
\pgfpathlineto{\pgfqpoint{5.285280in}{0.613486in}}%
\pgfpathclose%
\pgfusepath{fill}%
\end{pgfscope}%
\begin{pgfscope}%
\pgfpathrectangle{\pgfqpoint{0.693757in}{0.613486in}}{\pgfqpoint{5.541243in}{3.963477in}}%
\pgfusepath{clip}%
\pgfsetbuttcap%
\pgfsetmiterjoin%
\definecolor{currentfill}{rgb}{0.000000,0.000000,1.000000}%
\pgfsetfillcolor{currentfill}%
\pgfsetlinewidth{0.000000pt}%
\definecolor{currentstroke}{rgb}{0.000000,0.000000,0.000000}%
\pgfsetstrokecolor{currentstroke}%
\pgfsetstrokeopacity{0.000000}%
\pgfsetdash{}{0pt}%
\pgfpathmoveto{\pgfqpoint{5.297786in}{0.613486in}}%
\pgfpathlineto{\pgfqpoint{5.307791in}{0.613486in}}%
\pgfpathlineto{\pgfqpoint{5.307791in}{1.604355in}}%
\pgfpathlineto{\pgfqpoint{5.297786in}{1.604355in}}%
\pgfpathlineto{\pgfqpoint{5.297786in}{0.613486in}}%
\pgfpathclose%
\pgfusepath{fill}%
\end{pgfscope}%
\begin{pgfscope}%
\pgfpathrectangle{\pgfqpoint{0.693757in}{0.613486in}}{\pgfqpoint{5.541243in}{3.963477in}}%
\pgfusepath{clip}%
\pgfsetbuttcap%
\pgfsetmiterjoin%
\definecolor{currentfill}{rgb}{0.000000,0.000000,1.000000}%
\pgfsetfillcolor{currentfill}%
\pgfsetlinewidth{0.000000pt}%
\definecolor{currentstroke}{rgb}{0.000000,0.000000,0.000000}%
\pgfsetstrokecolor{currentstroke}%
\pgfsetstrokeopacity{0.000000}%
\pgfsetdash{}{0pt}%
\pgfpathmoveto{\pgfqpoint{5.310292in}{0.613486in}}%
\pgfpathlineto{\pgfqpoint{5.320297in}{0.613486in}}%
\pgfpathlineto{\pgfqpoint{5.320297in}{2.611074in}}%
\pgfpathlineto{\pgfqpoint{5.310292in}{2.611074in}}%
\pgfpathlineto{\pgfqpoint{5.310292in}{0.613486in}}%
\pgfpathclose%
\pgfusepath{fill}%
\end{pgfscope}%
\begin{pgfscope}%
\pgfpathrectangle{\pgfqpoint{0.693757in}{0.613486in}}{\pgfqpoint{5.541243in}{3.963477in}}%
\pgfusepath{clip}%
\pgfsetbuttcap%
\pgfsetmiterjoin%
\definecolor{currentfill}{rgb}{0.000000,0.000000,1.000000}%
\pgfsetfillcolor{currentfill}%
\pgfsetlinewidth{0.000000pt}%
\definecolor{currentstroke}{rgb}{0.000000,0.000000,0.000000}%
\pgfsetstrokecolor{currentstroke}%
\pgfsetstrokeopacity{0.000000}%
\pgfsetdash{}{0pt}%
\pgfpathmoveto{\pgfqpoint{5.322798in}{0.613486in}}%
\pgfpathlineto{\pgfqpoint{5.332803in}{0.613486in}}%
\pgfpathlineto{\pgfqpoint{5.332803in}{2.595224in}}%
\pgfpathlineto{\pgfqpoint{5.322798in}{2.595224in}}%
\pgfpathlineto{\pgfqpoint{5.322798in}{0.613486in}}%
\pgfpathclose%
\pgfusepath{fill}%
\end{pgfscope}%
\begin{pgfscope}%
\pgfpathrectangle{\pgfqpoint{0.693757in}{0.613486in}}{\pgfqpoint{5.541243in}{3.963477in}}%
\pgfusepath{clip}%
\pgfsetbuttcap%
\pgfsetmiterjoin%
\definecolor{currentfill}{rgb}{0.000000,0.000000,1.000000}%
\pgfsetfillcolor{currentfill}%
\pgfsetlinewidth{0.000000pt}%
\definecolor{currentstroke}{rgb}{0.000000,0.000000,0.000000}%
\pgfsetstrokecolor{currentstroke}%
\pgfsetstrokeopacity{0.000000}%
\pgfsetdash{}{0pt}%
\pgfpathmoveto{\pgfqpoint{5.335305in}{0.613486in}}%
\pgfpathlineto{\pgfqpoint{5.345310in}{0.613486in}}%
\pgfpathlineto{\pgfqpoint{5.345310in}{1.612282in}}%
\pgfpathlineto{\pgfqpoint{5.335305in}{1.612282in}}%
\pgfpathlineto{\pgfqpoint{5.335305in}{0.613486in}}%
\pgfpathclose%
\pgfusepath{fill}%
\end{pgfscope}%
\begin{pgfscope}%
\pgfpathrectangle{\pgfqpoint{0.693757in}{0.613486in}}{\pgfqpoint{5.541243in}{3.963477in}}%
\pgfusepath{clip}%
\pgfsetbuttcap%
\pgfsetmiterjoin%
\definecolor{currentfill}{rgb}{0.000000,0.000000,1.000000}%
\pgfsetfillcolor{currentfill}%
\pgfsetlinewidth{0.000000pt}%
\definecolor{currentstroke}{rgb}{0.000000,0.000000,0.000000}%
\pgfsetstrokecolor{currentstroke}%
\pgfsetstrokeopacity{0.000000}%
\pgfsetdash{}{0pt}%
\pgfpathmoveto{\pgfqpoint{5.347811in}{0.613486in}}%
\pgfpathlineto{\pgfqpoint{5.357816in}{0.613486in}}%
\pgfpathlineto{\pgfqpoint{5.357816in}{1.604355in}}%
\pgfpathlineto{\pgfqpoint{5.347811in}{1.604355in}}%
\pgfpathlineto{\pgfqpoint{5.347811in}{0.613486in}}%
\pgfpathclose%
\pgfusepath{fill}%
\end{pgfscope}%
\begin{pgfscope}%
\pgfpathrectangle{\pgfqpoint{0.693757in}{0.613486in}}{\pgfqpoint{5.541243in}{3.963477in}}%
\pgfusepath{clip}%
\pgfsetbuttcap%
\pgfsetmiterjoin%
\definecolor{currentfill}{rgb}{0.000000,0.000000,1.000000}%
\pgfsetfillcolor{currentfill}%
\pgfsetlinewidth{0.000000pt}%
\definecolor{currentstroke}{rgb}{0.000000,0.000000,0.000000}%
\pgfsetstrokecolor{currentstroke}%
\pgfsetstrokeopacity{0.000000}%
\pgfsetdash{}{0pt}%
\pgfpathmoveto{\pgfqpoint{5.360317in}{0.613486in}}%
\pgfpathlineto{\pgfqpoint{5.370322in}{0.613486in}}%
\pgfpathlineto{\pgfqpoint{5.370322in}{2.611074in}}%
\pgfpathlineto{\pgfqpoint{5.360317in}{2.611074in}}%
\pgfpathlineto{\pgfqpoint{5.360317in}{0.613486in}}%
\pgfpathclose%
\pgfusepath{fill}%
\end{pgfscope}%
\begin{pgfscope}%
\pgfpathrectangle{\pgfqpoint{0.693757in}{0.613486in}}{\pgfqpoint{5.541243in}{3.963477in}}%
\pgfusepath{clip}%
\pgfsetbuttcap%
\pgfsetmiterjoin%
\definecolor{currentfill}{rgb}{0.000000,0.000000,1.000000}%
\pgfsetfillcolor{currentfill}%
\pgfsetlinewidth{0.000000pt}%
\definecolor{currentstroke}{rgb}{0.000000,0.000000,0.000000}%
\pgfsetstrokecolor{currentstroke}%
\pgfsetstrokeopacity{0.000000}%
\pgfsetdash{}{0pt}%
\pgfpathmoveto{\pgfqpoint{5.372823in}{0.613486in}}%
\pgfpathlineto{\pgfqpoint{5.382828in}{0.613486in}}%
\pgfpathlineto{\pgfqpoint{5.382828in}{2.595224in}}%
\pgfpathlineto{\pgfqpoint{5.372823in}{2.595224in}}%
\pgfpathlineto{\pgfqpoint{5.372823in}{0.613486in}}%
\pgfpathclose%
\pgfusepath{fill}%
\end{pgfscope}%
\begin{pgfscope}%
\pgfpathrectangle{\pgfqpoint{0.693757in}{0.613486in}}{\pgfqpoint{5.541243in}{3.963477in}}%
\pgfusepath{clip}%
\pgfsetbuttcap%
\pgfsetmiterjoin%
\definecolor{currentfill}{rgb}{0.000000,0.000000,1.000000}%
\pgfsetfillcolor{currentfill}%
\pgfsetlinewidth{0.000000pt}%
\definecolor{currentstroke}{rgb}{0.000000,0.000000,0.000000}%
\pgfsetstrokecolor{currentstroke}%
\pgfsetstrokeopacity{0.000000}%
\pgfsetdash{}{0pt}%
\pgfpathmoveto{\pgfqpoint{5.385329in}{0.613486in}}%
\pgfpathlineto{\pgfqpoint{5.395334in}{0.613486in}}%
\pgfpathlineto{\pgfqpoint{5.395334in}{1.612282in}}%
\pgfpathlineto{\pgfqpoint{5.385329in}{1.612282in}}%
\pgfpathlineto{\pgfqpoint{5.385329in}{0.613486in}}%
\pgfpathclose%
\pgfusepath{fill}%
\end{pgfscope}%
\begin{pgfscope}%
\pgfpathrectangle{\pgfqpoint{0.693757in}{0.613486in}}{\pgfqpoint{5.541243in}{3.963477in}}%
\pgfusepath{clip}%
\pgfsetbuttcap%
\pgfsetmiterjoin%
\definecolor{currentfill}{rgb}{0.000000,0.000000,1.000000}%
\pgfsetfillcolor{currentfill}%
\pgfsetlinewidth{0.000000pt}%
\definecolor{currentstroke}{rgb}{0.000000,0.000000,0.000000}%
\pgfsetstrokecolor{currentstroke}%
\pgfsetstrokeopacity{0.000000}%
\pgfsetdash{}{0pt}%
\pgfpathmoveto{\pgfqpoint{5.397836in}{0.613486in}}%
\pgfpathlineto{\pgfqpoint{5.407841in}{0.613486in}}%
\pgfpathlineto{\pgfqpoint{5.407841in}{1.604355in}}%
\pgfpathlineto{\pgfqpoint{5.397836in}{1.604355in}}%
\pgfpathlineto{\pgfqpoint{5.397836in}{0.613486in}}%
\pgfpathclose%
\pgfusepath{fill}%
\end{pgfscope}%
\begin{pgfscope}%
\pgfpathrectangle{\pgfqpoint{0.693757in}{0.613486in}}{\pgfqpoint{5.541243in}{3.963477in}}%
\pgfusepath{clip}%
\pgfsetbuttcap%
\pgfsetmiterjoin%
\definecolor{currentfill}{rgb}{0.000000,0.000000,1.000000}%
\pgfsetfillcolor{currentfill}%
\pgfsetlinewidth{0.000000pt}%
\definecolor{currentstroke}{rgb}{0.000000,0.000000,0.000000}%
\pgfsetstrokecolor{currentstroke}%
\pgfsetstrokeopacity{0.000000}%
\pgfsetdash{}{0pt}%
\pgfpathmoveto{\pgfqpoint{5.410342in}{0.613486in}}%
\pgfpathlineto{\pgfqpoint{5.420347in}{0.613486in}}%
\pgfpathlineto{\pgfqpoint{5.420347in}{2.611074in}}%
\pgfpathlineto{\pgfqpoint{5.410342in}{2.611074in}}%
\pgfpathlineto{\pgfqpoint{5.410342in}{0.613486in}}%
\pgfpathclose%
\pgfusepath{fill}%
\end{pgfscope}%
\begin{pgfscope}%
\pgfpathrectangle{\pgfqpoint{0.693757in}{0.613486in}}{\pgfqpoint{5.541243in}{3.963477in}}%
\pgfusepath{clip}%
\pgfsetbuttcap%
\pgfsetmiterjoin%
\definecolor{currentfill}{rgb}{0.000000,0.000000,1.000000}%
\pgfsetfillcolor{currentfill}%
\pgfsetlinewidth{0.000000pt}%
\definecolor{currentstroke}{rgb}{0.000000,0.000000,0.000000}%
\pgfsetstrokecolor{currentstroke}%
\pgfsetstrokeopacity{0.000000}%
\pgfsetdash{}{0pt}%
\pgfpathmoveto{\pgfqpoint{5.422848in}{0.613486in}}%
\pgfpathlineto{\pgfqpoint{5.432853in}{0.613486in}}%
\pgfpathlineto{\pgfqpoint{5.432853in}{2.595224in}}%
\pgfpathlineto{\pgfqpoint{5.422848in}{2.595224in}}%
\pgfpathlineto{\pgfqpoint{5.422848in}{0.613486in}}%
\pgfpathclose%
\pgfusepath{fill}%
\end{pgfscope}%
\begin{pgfscope}%
\pgfpathrectangle{\pgfqpoint{0.693757in}{0.613486in}}{\pgfqpoint{5.541243in}{3.963477in}}%
\pgfusepath{clip}%
\pgfsetbuttcap%
\pgfsetmiterjoin%
\definecolor{currentfill}{rgb}{0.000000,0.000000,1.000000}%
\pgfsetfillcolor{currentfill}%
\pgfsetlinewidth{0.000000pt}%
\definecolor{currentstroke}{rgb}{0.000000,0.000000,0.000000}%
\pgfsetstrokecolor{currentstroke}%
\pgfsetstrokeopacity{0.000000}%
\pgfsetdash{}{0pt}%
\pgfpathmoveto{\pgfqpoint{5.435354in}{0.613486in}}%
\pgfpathlineto{\pgfqpoint{5.445359in}{0.613486in}}%
\pgfpathlineto{\pgfqpoint{5.445359in}{1.612282in}}%
\pgfpathlineto{\pgfqpoint{5.435354in}{1.612282in}}%
\pgfpathlineto{\pgfqpoint{5.435354in}{0.613486in}}%
\pgfpathclose%
\pgfusepath{fill}%
\end{pgfscope}%
\begin{pgfscope}%
\pgfpathrectangle{\pgfqpoint{0.693757in}{0.613486in}}{\pgfqpoint{5.541243in}{3.963477in}}%
\pgfusepath{clip}%
\pgfsetbuttcap%
\pgfsetmiterjoin%
\definecolor{currentfill}{rgb}{0.000000,0.000000,1.000000}%
\pgfsetfillcolor{currentfill}%
\pgfsetlinewidth{0.000000pt}%
\definecolor{currentstroke}{rgb}{0.000000,0.000000,0.000000}%
\pgfsetstrokecolor{currentstroke}%
\pgfsetstrokeopacity{0.000000}%
\pgfsetdash{}{0pt}%
\pgfpathmoveto{\pgfqpoint{5.447860in}{0.613486in}}%
\pgfpathlineto{\pgfqpoint{5.457865in}{0.613486in}}%
\pgfpathlineto{\pgfqpoint{5.457865in}{1.604355in}}%
\pgfpathlineto{\pgfqpoint{5.447860in}{1.604355in}}%
\pgfpathlineto{\pgfqpoint{5.447860in}{0.613486in}}%
\pgfpathclose%
\pgfusepath{fill}%
\end{pgfscope}%
\begin{pgfscope}%
\pgfpathrectangle{\pgfqpoint{0.693757in}{0.613486in}}{\pgfqpoint{5.541243in}{3.963477in}}%
\pgfusepath{clip}%
\pgfsetbuttcap%
\pgfsetmiterjoin%
\definecolor{currentfill}{rgb}{0.000000,0.000000,1.000000}%
\pgfsetfillcolor{currentfill}%
\pgfsetlinewidth{0.000000pt}%
\definecolor{currentstroke}{rgb}{0.000000,0.000000,0.000000}%
\pgfsetstrokecolor{currentstroke}%
\pgfsetstrokeopacity{0.000000}%
\pgfsetdash{}{0pt}%
\pgfpathmoveto{\pgfqpoint{5.460367in}{0.613486in}}%
\pgfpathlineto{\pgfqpoint{5.470372in}{0.613486in}}%
\pgfpathlineto{\pgfqpoint{5.470372in}{2.611074in}}%
\pgfpathlineto{\pgfqpoint{5.460367in}{2.611074in}}%
\pgfpathlineto{\pgfqpoint{5.460367in}{0.613486in}}%
\pgfpathclose%
\pgfusepath{fill}%
\end{pgfscope}%
\begin{pgfscope}%
\pgfpathrectangle{\pgfqpoint{0.693757in}{0.613486in}}{\pgfqpoint{5.541243in}{3.963477in}}%
\pgfusepath{clip}%
\pgfsetbuttcap%
\pgfsetmiterjoin%
\definecolor{currentfill}{rgb}{0.000000,0.000000,1.000000}%
\pgfsetfillcolor{currentfill}%
\pgfsetlinewidth{0.000000pt}%
\definecolor{currentstroke}{rgb}{0.000000,0.000000,0.000000}%
\pgfsetstrokecolor{currentstroke}%
\pgfsetstrokeopacity{0.000000}%
\pgfsetdash{}{0pt}%
\pgfpathmoveto{\pgfqpoint{5.472873in}{0.613486in}}%
\pgfpathlineto{\pgfqpoint{5.482878in}{0.613486in}}%
\pgfpathlineto{\pgfqpoint{5.482878in}{2.595224in}}%
\pgfpathlineto{\pgfqpoint{5.472873in}{2.595224in}}%
\pgfpathlineto{\pgfqpoint{5.472873in}{0.613486in}}%
\pgfpathclose%
\pgfusepath{fill}%
\end{pgfscope}%
\begin{pgfscope}%
\pgfpathrectangle{\pgfqpoint{0.693757in}{0.613486in}}{\pgfqpoint{5.541243in}{3.963477in}}%
\pgfusepath{clip}%
\pgfsetbuttcap%
\pgfsetmiterjoin%
\definecolor{currentfill}{rgb}{0.000000,0.000000,1.000000}%
\pgfsetfillcolor{currentfill}%
\pgfsetlinewidth{0.000000pt}%
\definecolor{currentstroke}{rgb}{0.000000,0.000000,0.000000}%
\pgfsetstrokecolor{currentstroke}%
\pgfsetstrokeopacity{0.000000}%
\pgfsetdash{}{0pt}%
\pgfpathmoveto{\pgfqpoint{5.485379in}{0.613486in}}%
\pgfpathlineto{\pgfqpoint{5.495384in}{0.613486in}}%
\pgfpathlineto{\pgfqpoint{5.495384in}{1.612282in}}%
\pgfpathlineto{\pgfqpoint{5.485379in}{1.612282in}}%
\pgfpathlineto{\pgfqpoint{5.485379in}{0.613486in}}%
\pgfpathclose%
\pgfusepath{fill}%
\end{pgfscope}%
\begin{pgfscope}%
\pgfpathrectangle{\pgfqpoint{0.693757in}{0.613486in}}{\pgfqpoint{5.541243in}{3.963477in}}%
\pgfusepath{clip}%
\pgfsetbuttcap%
\pgfsetmiterjoin%
\definecolor{currentfill}{rgb}{0.000000,0.000000,1.000000}%
\pgfsetfillcolor{currentfill}%
\pgfsetlinewidth{0.000000pt}%
\definecolor{currentstroke}{rgb}{0.000000,0.000000,0.000000}%
\pgfsetstrokecolor{currentstroke}%
\pgfsetstrokeopacity{0.000000}%
\pgfsetdash{}{0pt}%
\pgfpathmoveto{\pgfqpoint{5.497885in}{0.613486in}}%
\pgfpathlineto{\pgfqpoint{5.507890in}{0.613486in}}%
\pgfpathlineto{\pgfqpoint{5.507890in}{1.604355in}}%
\pgfpathlineto{\pgfqpoint{5.497885in}{1.604355in}}%
\pgfpathlineto{\pgfqpoint{5.497885in}{0.613486in}}%
\pgfpathclose%
\pgfusepath{fill}%
\end{pgfscope}%
\begin{pgfscope}%
\pgfpathrectangle{\pgfqpoint{0.693757in}{0.613486in}}{\pgfqpoint{5.541243in}{3.963477in}}%
\pgfusepath{clip}%
\pgfsetbuttcap%
\pgfsetmiterjoin%
\definecolor{currentfill}{rgb}{0.000000,0.000000,1.000000}%
\pgfsetfillcolor{currentfill}%
\pgfsetlinewidth{0.000000pt}%
\definecolor{currentstroke}{rgb}{0.000000,0.000000,0.000000}%
\pgfsetstrokecolor{currentstroke}%
\pgfsetstrokeopacity{0.000000}%
\pgfsetdash{}{0pt}%
\pgfpathmoveto{\pgfqpoint{5.510391in}{0.613486in}}%
\pgfpathlineto{\pgfqpoint{5.520396in}{0.613486in}}%
\pgfpathlineto{\pgfqpoint{5.520396in}{2.611074in}}%
\pgfpathlineto{\pgfqpoint{5.510391in}{2.611074in}}%
\pgfpathlineto{\pgfqpoint{5.510391in}{0.613486in}}%
\pgfpathclose%
\pgfusepath{fill}%
\end{pgfscope}%
\begin{pgfscope}%
\pgfpathrectangle{\pgfqpoint{0.693757in}{0.613486in}}{\pgfqpoint{5.541243in}{3.963477in}}%
\pgfusepath{clip}%
\pgfsetbuttcap%
\pgfsetmiterjoin%
\definecolor{currentfill}{rgb}{0.000000,0.000000,1.000000}%
\pgfsetfillcolor{currentfill}%
\pgfsetlinewidth{0.000000pt}%
\definecolor{currentstroke}{rgb}{0.000000,0.000000,0.000000}%
\pgfsetstrokecolor{currentstroke}%
\pgfsetstrokeopacity{0.000000}%
\pgfsetdash{}{0pt}%
\pgfpathmoveto{\pgfqpoint{5.522898in}{0.613486in}}%
\pgfpathlineto{\pgfqpoint{5.532902in}{0.613486in}}%
\pgfpathlineto{\pgfqpoint{5.532902in}{2.595224in}}%
\pgfpathlineto{\pgfqpoint{5.522898in}{2.595224in}}%
\pgfpathlineto{\pgfqpoint{5.522898in}{0.613486in}}%
\pgfpathclose%
\pgfusepath{fill}%
\end{pgfscope}%
\begin{pgfscope}%
\pgfpathrectangle{\pgfqpoint{0.693757in}{0.613486in}}{\pgfqpoint{5.541243in}{3.963477in}}%
\pgfusepath{clip}%
\pgfsetbuttcap%
\pgfsetmiterjoin%
\definecolor{currentfill}{rgb}{0.000000,0.000000,1.000000}%
\pgfsetfillcolor{currentfill}%
\pgfsetlinewidth{0.000000pt}%
\definecolor{currentstroke}{rgb}{0.000000,0.000000,0.000000}%
\pgfsetstrokecolor{currentstroke}%
\pgfsetstrokeopacity{0.000000}%
\pgfsetdash{}{0pt}%
\pgfpathmoveto{\pgfqpoint{5.535404in}{0.613486in}}%
\pgfpathlineto{\pgfqpoint{5.545409in}{0.613486in}}%
\pgfpathlineto{\pgfqpoint{5.545409in}{1.612282in}}%
\pgfpathlineto{\pgfqpoint{5.535404in}{1.612282in}}%
\pgfpathlineto{\pgfqpoint{5.535404in}{0.613486in}}%
\pgfpathclose%
\pgfusepath{fill}%
\end{pgfscope}%
\begin{pgfscope}%
\pgfpathrectangle{\pgfqpoint{0.693757in}{0.613486in}}{\pgfqpoint{5.541243in}{3.963477in}}%
\pgfusepath{clip}%
\pgfsetbuttcap%
\pgfsetmiterjoin%
\definecolor{currentfill}{rgb}{0.000000,0.000000,1.000000}%
\pgfsetfillcolor{currentfill}%
\pgfsetlinewidth{0.000000pt}%
\definecolor{currentstroke}{rgb}{0.000000,0.000000,0.000000}%
\pgfsetstrokecolor{currentstroke}%
\pgfsetstrokeopacity{0.000000}%
\pgfsetdash{}{0pt}%
\pgfpathmoveto{\pgfqpoint{5.547910in}{0.613486in}}%
\pgfpathlineto{\pgfqpoint{5.557915in}{0.613486in}}%
\pgfpathlineto{\pgfqpoint{5.557915in}{1.604355in}}%
\pgfpathlineto{\pgfqpoint{5.547910in}{1.604355in}}%
\pgfpathlineto{\pgfqpoint{5.547910in}{0.613486in}}%
\pgfpathclose%
\pgfusepath{fill}%
\end{pgfscope}%
\begin{pgfscope}%
\pgfpathrectangle{\pgfqpoint{0.693757in}{0.613486in}}{\pgfqpoint{5.541243in}{3.963477in}}%
\pgfusepath{clip}%
\pgfsetbuttcap%
\pgfsetmiterjoin%
\definecolor{currentfill}{rgb}{0.000000,0.000000,1.000000}%
\pgfsetfillcolor{currentfill}%
\pgfsetlinewidth{0.000000pt}%
\definecolor{currentstroke}{rgb}{0.000000,0.000000,0.000000}%
\pgfsetstrokecolor{currentstroke}%
\pgfsetstrokeopacity{0.000000}%
\pgfsetdash{}{0pt}%
\pgfpathmoveto{\pgfqpoint{5.560416in}{0.613486in}}%
\pgfpathlineto{\pgfqpoint{5.570421in}{0.613486in}}%
\pgfpathlineto{\pgfqpoint{5.570421in}{2.611074in}}%
\pgfpathlineto{\pgfqpoint{5.560416in}{2.611074in}}%
\pgfpathlineto{\pgfqpoint{5.560416in}{0.613486in}}%
\pgfpathclose%
\pgfusepath{fill}%
\end{pgfscope}%
\begin{pgfscope}%
\pgfpathrectangle{\pgfqpoint{0.693757in}{0.613486in}}{\pgfqpoint{5.541243in}{3.963477in}}%
\pgfusepath{clip}%
\pgfsetbuttcap%
\pgfsetmiterjoin%
\definecolor{currentfill}{rgb}{0.000000,0.000000,1.000000}%
\pgfsetfillcolor{currentfill}%
\pgfsetlinewidth{0.000000pt}%
\definecolor{currentstroke}{rgb}{0.000000,0.000000,0.000000}%
\pgfsetstrokecolor{currentstroke}%
\pgfsetstrokeopacity{0.000000}%
\pgfsetdash{}{0pt}%
\pgfpathmoveto{\pgfqpoint{5.572922in}{0.613486in}}%
\pgfpathlineto{\pgfqpoint{5.582927in}{0.613486in}}%
\pgfpathlineto{\pgfqpoint{5.582927in}{2.595224in}}%
\pgfpathlineto{\pgfqpoint{5.572922in}{2.595224in}}%
\pgfpathlineto{\pgfqpoint{5.572922in}{0.613486in}}%
\pgfpathclose%
\pgfusepath{fill}%
\end{pgfscope}%
\begin{pgfscope}%
\pgfpathrectangle{\pgfqpoint{0.693757in}{0.613486in}}{\pgfqpoint{5.541243in}{3.963477in}}%
\pgfusepath{clip}%
\pgfsetbuttcap%
\pgfsetmiterjoin%
\definecolor{currentfill}{rgb}{0.000000,0.000000,1.000000}%
\pgfsetfillcolor{currentfill}%
\pgfsetlinewidth{0.000000pt}%
\definecolor{currentstroke}{rgb}{0.000000,0.000000,0.000000}%
\pgfsetstrokecolor{currentstroke}%
\pgfsetstrokeopacity{0.000000}%
\pgfsetdash{}{0pt}%
\pgfpathmoveto{\pgfqpoint{5.585428in}{0.613486in}}%
\pgfpathlineto{\pgfqpoint{5.595433in}{0.613486in}}%
\pgfpathlineto{\pgfqpoint{5.595433in}{1.612282in}}%
\pgfpathlineto{\pgfqpoint{5.585428in}{1.612282in}}%
\pgfpathlineto{\pgfqpoint{5.585428in}{0.613486in}}%
\pgfpathclose%
\pgfusepath{fill}%
\end{pgfscope}%
\begin{pgfscope}%
\pgfpathrectangle{\pgfqpoint{0.693757in}{0.613486in}}{\pgfqpoint{5.541243in}{3.963477in}}%
\pgfusepath{clip}%
\pgfsetbuttcap%
\pgfsetmiterjoin%
\definecolor{currentfill}{rgb}{0.000000,0.000000,1.000000}%
\pgfsetfillcolor{currentfill}%
\pgfsetlinewidth{0.000000pt}%
\definecolor{currentstroke}{rgb}{0.000000,0.000000,0.000000}%
\pgfsetstrokecolor{currentstroke}%
\pgfsetstrokeopacity{0.000000}%
\pgfsetdash{}{0pt}%
\pgfpathmoveto{\pgfqpoint{5.597935in}{0.613486in}}%
\pgfpathlineto{\pgfqpoint{5.607940in}{0.613486in}}%
\pgfpathlineto{\pgfqpoint{5.607940in}{1.604355in}}%
\pgfpathlineto{\pgfqpoint{5.597935in}{1.604355in}}%
\pgfpathlineto{\pgfqpoint{5.597935in}{0.613486in}}%
\pgfpathclose%
\pgfusepath{fill}%
\end{pgfscope}%
\begin{pgfscope}%
\pgfpathrectangle{\pgfqpoint{0.693757in}{0.613486in}}{\pgfqpoint{5.541243in}{3.963477in}}%
\pgfusepath{clip}%
\pgfsetbuttcap%
\pgfsetmiterjoin%
\definecolor{currentfill}{rgb}{0.000000,0.000000,1.000000}%
\pgfsetfillcolor{currentfill}%
\pgfsetlinewidth{0.000000pt}%
\definecolor{currentstroke}{rgb}{0.000000,0.000000,0.000000}%
\pgfsetstrokecolor{currentstroke}%
\pgfsetstrokeopacity{0.000000}%
\pgfsetdash{}{0pt}%
\pgfpathmoveto{\pgfqpoint{5.610441in}{0.613486in}}%
\pgfpathlineto{\pgfqpoint{5.620446in}{0.613486in}}%
\pgfpathlineto{\pgfqpoint{5.620446in}{2.611074in}}%
\pgfpathlineto{\pgfqpoint{5.610441in}{2.611074in}}%
\pgfpathlineto{\pgfqpoint{5.610441in}{0.613486in}}%
\pgfpathclose%
\pgfusepath{fill}%
\end{pgfscope}%
\begin{pgfscope}%
\pgfpathrectangle{\pgfqpoint{0.693757in}{0.613486in}}{\pgfqpoint{5.541243in}{3.963477in}}%
\pgfusepath{clip}%
\pgfsetbuttcap%
\pgfsetmiterjoin%
\definecolor{currentfill}{rgb}{0.000000,0.000000,1.000000}%
\pgfsetfillcolor{currentfill}%
\pgfsetlinewidth{0.000000pt}%
\definecolor{currentstroke}{rgb}{0.000000,0.000000,0.000000}%
\pgfsetstrokecolor{currentstroke}%
\pgfsetstrokeopacity{0.000000}%
\pgfsetdash{}{0pt}%
\pgfpathmoveto{\pgfqpoint{5.622947in}{0.613486in}}%
\pgfpathlineto{\pgfqpoint{5.632952in}{0.613486in}}%
\pgfpathlineto{\pgfqpoint{5.632952in}{2.595224in}}%
\pgfpathlineto{\pgfqpoint{5.622947in}{2.595224in}}%
\pgfpathlineto{\pgfqpoint{5.622947in}{0.613486in}}%
\pgfpathclose%
\pgfusepath{fill}%
\end{pgfscope}%
\begin{pgfscope}%
\pgfpathrectangle{\pgfqpoint{0.693757in}{0.613486in}}{\pgfqpoint{5.541243in}{3.963477in}}%
\pgfusepath{clip}%
\pgfsetbuttcap%
\pgfsetmiterjoin%
\definecolor{currentfill}{rgb}{0.000000,0.000000,1.000000}%
\pgfsetfillcolor{currentfill}%
\pgfsetlinewidth{0.000000pt}%
\definecolor{currentstroke}{rgb}{0.000000,0.000000,0.000000}%
\pgfsetstrokecolor{currentstroke}%
\pgfsetstrokeopacity{0.000000}%
\pgfsetdash{}{0pt}%
\pgfpathmoveto{\pgfqpoint{5.635453in}{0.613486in}}%
\pgfpathlineto{\pgfqpoint{5.645458in}{0.613486in}}%
\pgfpathlineto{\pgfqpoint{5.645458in}{1.612282in}}%
\pgfpathlineto{\pgfqpoint{5.635453in}{1.612282in}}%
\pgfpathlineto{\pgfqpoint{5.635453in}{0.613486in}}%
\pgfpathclose%
\pgfusepath{fill}%
\end{pgfscope}%
\begin{pgfscope}%
\pgfpathrectangle{\pgfqpoint{0.693757in}{0.613486in}}{\pgfqpoint{5.541243in}{3.963477in}}%
\pgfusepath{clip}%
\pgfsetbuttcap%
\pgfsetmiterjoin%
\definecolor{currentfill}{rgb}{0.000000,0.000000,1.000000}%
\pgfsetfillcolor{currentfill}%
\pgfsetlinewidth{0.000000pt}%
\definecolor{currentstroke}{rgb}{0.000000,0.000000,0.000000}%
\pgfsetstrokecolor{currentstroke}%
\pgfsetstrokeopacity{0.000000}%
\pgfsetdash{}{0pt}%
\pgfpathmoveto{\pgfqpoint{5.647959in}{0.613486in}}%
\pgfpathlineto{\pgfqpoint{5.657964in}{0.613486in}}%
\pgfpathlineto{\pgfqpoint{5.657964in}{1.604355in}}%
\pgfpathlineto{\pgfqpoint{5.647959in}{1.604355in}}%
\pgfpathlineto{\pgfqpoint{5.647959in}{0.613486in}}%
\pgfpathclose%
\pgfusepath{fill}%
\end{pgfscope}%
\begin{pgfscope}%
\pgfpathrectangle{\pgfqpoint{0.693757in}{0.613486in}}{\pgfqpoint{5.541243in}{3.963477in}}%
\pgfusepath{clip}%
\pgfsetbuttcap%
\pgfsetmiterjoin%
\definecolor{currentfill}{rgb}{0.000000,0.000000,1.000000}%
\pgfsetfillcolor{currentfill}%
\pgfsetlinewidth{0.000000pt}%
\definecolor{currentstroke}{rgb}{0.000000,0.000000,0.000000}%
\pgfsetstrokecolor{currentstroke}%
\pgfsetstrokeopacity{0.000000}%
\pgfsetdash{}{0pt}%
\pgfpathmoveto{\pgfqpoint{5.660466in}{0.613486in}}%
\pgfpathlineto{\pgfqpoint{5.670471in}{0.613486in}}%
\pgfpathlineto{\pgfqpoint{5.670471in}{2.611074in}}%
\pgfpathlineto{\pgfqpoint{5.660466in}{2.611074in}}%
\pgfpathlineto{\pgfqpoint{5.660466in}{0.613486in}}%
\pgfpathclose%
\pgfusepath{fill}%
\end{pgfscope}%
\begin{pgfscope}%
\pgfpathrectangle{\pgfqpoint{0.693757in}{0.613486in}}{\pgfqpoint{5.541243in}{3.963477in}}%
\pgfusepath{clip}%
\pgfsetbuttcap%
\pgfsetmiterjoin%
\definecolor{currentfill}{rgb}{0.000000,0.000000,1.000000}%
\pgfsetfillcolor{currentfill}%
\pgfsetlinewidth{0.000000pt}%
\definecolor{currentstroke}{rgb}{0.000000,0.000000,0.000000}%
\pgfsetstrokecolor{currentstroke}%
\pgfsetstrokeopacity{0.000000}%
\pgfsetdash{}{0pt}%
\pgfpathmoveto{\pgfqpoint{5.672972in}{0.613486in}}%
\pgfpathlineto{\pgfqpoint{5.682977in}{0.613486in}}%
\pgfpathlineto{\pgfqpoint{5.682977in}{2.595224in}}%
\pgfpathlineto{\pgfqpoint{5.672972in}{2.595224in}}%
\pgfpathlineto{\pgfqpoint{5.672972in}{0.613486in}}%
\pgfpathclose%
\pgfusepath{fill}%
\end{pgfscope}%
\begin{pgfscope}%
\pgfpathrectangle{\pgfqpoint{0.693757in}{0.613486in}}{\pgfqpoint{5.541243in}{3.963477in}}%
\pgfusepath{clip}%
\pgfsetbuttcap%
\pgfsetmiterjoin%
\definecolor{currentfill}{rgb}{0.000000,0.000000,1.000000}%
\pgfsetfillcolor{currentfill}%
\pgfsetlinewidth{0.000000pt}%
\definecolor{currentstroke}{rgb}{0.000000,0.000000,0.000000}%
\pgfsetstrokecolor{currentstroke}%
\pgfsetstrokeopacity{0.000000}%
\pgfsetdash{}{0pt}%
\pgfpathmoveto{\pgfqpoint{5.685478in}{0.613486in}}%
\pgfpathlineto{\pgfqpoint{5.695483in}{0.613486in}}%
\pgfpathlineto{\pgfqpoint{5.695483in}{1.612282in}}%
\pgfpathlineto{\pgfqpoint{5.685478in}{1.612282in}}%
\pgfpathlineto{\pgfqpoint{5.685478in}{0.613486in}}%
\pgfpathclose%
\pgfusepath{fill}%
\end{pgfscope}%
\begin{pgfscope}%
\pgfpathrectangle{\pgfqpoint{0.693757in}{0.613486in}}{\pgfqpoint{5.541243in}{3.963477in}}%
\pgfusepath{clip}%
\pgfsetbuttcap%
\pgfsetmiterjoin%
\definecolor{currentfill}{rgb}{0.000000,0.000000,1.000000}%
\pgfsetfillcolor{currentfill}%
\pgfsetlinewidth{0.000000pt}%
\definecolor{currentstroke}{rgb}{0.000000,0.000000,0.000000}%
\pgfsetstrokecolor{currentstroke}%
\pgfsetstrokeopacity{0.000000}%
\pgfsetdash{}{0pt}%
\pgfpathmoveto{\pgfqpoint{5.697984in}{0.613486in}}%
\pgfpathlineto{\pgfqpoint{5.707989in}{0.613486in}}%
\pgfpathlineto{\pgfqpoint{5.707989in}{1.604355in}}%
\pgfpathlineto{\pgfqpoint{5.697984in}{1.604355in}}%
\pgfpathlineto{\pgfqpoint{5.697984in}{0.613486in}}%
\pgfpathclose%
\pgfusepath{fill}%
\end{pgfscope}%
\begin{pgfscope}%
\pgfpathrectangle{\pgfqpoint{0.693757in}{0.613486in}}{\pgfqpoint{5.541243in}{3.963477in}}%
\pgfusepath{clip}%
\pgfsetbuttcap%
\pgfsetmiterjoin%
\definecolor{currentfill}{rgb}{0.000000,0.000000,1.000000}%
\pgfsetfillcolor{currentfill}%
\pgfsetlinewidth{0.000000pt}%
\definecolor{currentstroke}{rgb}{0.000000,0.000000,0.000000}%
\pgfsetstrokecolor{currentstroke}%
\pgfsetstrokeopacity{0.000000}%
\pgfsetdash{}{0pt}%
\pgfpathmoveto{\pgfqpoint{5.710490in}{0.613486in}}%
\pgfpathlineto{\pgfqpoint{5.720495in}{0.613486in}}%
\pgfpathlineto{\pgfqpoint{5.720495in}{2.611074in}}%
\pgfpathlineto{\pgfqpoint{5.710490in}{2.611074in}}%
\pgfpathlineto{\pgfqpoint{5.710490in}{0.613486in}}%
\pgfpathclose%
\pgfusepath{fill}%
\end{pgfscope}%
\begin{pgfscope}%
\pgfpathrectangle{\pgfqpoint{0.693757in}{0.613486in}}{\pgfqpoint{5.541243in}{3.963477in}}%
\pgfusepath{clip}%
\pgfsetbuttcap%
\pgfsetmiterjoin%
\definecolor{currentfill}{rgb}{0.000000,0.000000,1.000000}%
\pgfsetfillcolor{currentfill}%
\pgfsetlinewidth{0.000000pt}%
\definecolor{currentstroke}{rgb}{0.000000,0.000000,0.000000}%
\pgfsetstrokecolor{currentstroke}%
\pgfsetstrokeopacity{0.000000}%
\pgfsetdash{}{0pt}%
\pgfpathmoveto{\pgfqpoint{5.722997in}{0.613486in}}%
\pgfpathlineto{\pgfqpoint{5.733002in}{0.613486in}}%
\pgfpathlineto{\pgfqpoint{5.733002in}{2.595224in}}%
\pgfpathlineto{\pgfqpoint{5.722997in}{2.595224in}}%
\pgfpathlineto{\pgfqpoint{5.722997in}{0.613486in}}%
\pgfpathclose%
\pgfusepath{fill}%
\end{pgfscope}%
\begin{pgfscope}%
\pgfpathrectangle{\pgfqpoint{0.693757in}{0.613486in}}{\pgfqpoint{5.541243in}{3.963477in}}%
\pgfusepath{clip}%
\pgfsetbuttcap%
\pgfsetmiterjoin%
\definecolor{currentfill}{rgb}{0.000000,0.000000,1.000000}%
\pgfsetfillcolor{currentfill}%
\pgfsetlinewidth{0.000000pt}%
\definecolor{currentstroke}{rgb}{0.000000,0.000000,0.000000}%
\pgfsetstrokecolor{currentstroke}%
\pgfsetstrokeopacity{0.000000}%
\pgfsetdash{}{0pt}%
\pgfpathmoveto{\pgfqpoint{5.735503in}{0.613486in}}%
\pgfpathlineto{\pgfqpoint{5.745508in}{0.613486in}}%
\pgfpathlineto{\pgfqpoint{5.745508in}{1.612282in}}%
\pgfpathlineto{\pgfqpoint{5.735503in}{1.612282in}}%
\pgfpathlineto{\pgfqpoint{5.735503in}{0.613486in}}%
\pgfpathclose%
\pgfusepath{fill}%
\end{pgfscope}%
\begin{pgfscope}%
\pgfpathrectangle{\pgfqpoint{0.693757in}{0.613486in}}{\pgfqpoint{5.541243in}{3.963477in}}%
\pgfusepath{clip}%
\pgfsetbuttcap%
\pgfsetmiterjoin%
\definecolor{currentfill}{rgb}{0.000000,0.000000,1.000000}%
\pgfsetfillcolor{currentfill}%
\pgfsetlinewidth{0.000000pt}%
\definecolor{currentstroke}{rgb}{0.000000,0.000000,0.000000}%
\pgfsetstrokecolor{currentstroke}%
\pgfsetstrokeopacity{0.000000}%
\pgfsetdash{}{0pt}%
\pgfpathmoveto{\pgfqpoint{5.748009in}{0.613486in}}%
\pgfpathlineto{\pgfqpoint{5.758014in}{0.613486in}}%
\pgfpathlineto{\pgfqpoint{5.758014in}{1.604355in}}%
\pgfpathlineto{\pgfqpoint{5.748009in}{1.604355in}}%
\pgfpathlineto{\pgfqpoint{5.748009in}{0.613486in}}%
\pgfpathclose%
\pgfusepath{fill}%
\end{pgfscope}%
\begin{pgfscope}%
\pgfpathrectangle{\pgfqpoint{0.693757in}{0.613486in}}{\pgfqpoint{5.541243in}{3.963477in}}%
\pgfusepath{clip}%
\pgfsetbuttcap%
\pgfsetmiterjoin%
\definecolor{currentfill}{rgb}{0.000000,0.000000,1.000000}%
\pgfsetfillcolor{currentfill}%
\pgfsetlinewidth{0.000000pt}%
\definecolor{currentstroke}{rgb}{0.000000,0.000000,0.000000}%
\pgfsetstrokecolor{currentstroke}%
\pgfsetstrokeopacity{0.000000}%
\pgfsetdash{}{0pt}%
\pgfpathmoveto{\pgfqpoint{5.760515in}{0.613486in}}%
\pgfpathlineto{\pgfqpoint{5.770520in}{0.613486in}}%
\pgfpathlineto{\pgfqpoint{5.770520in}{2.611074in}}%
\pgfpathlineto{\pgfqpoint{5.760515in}{2.611074in}}%
\pgfpathlineto{\pgfqpoint{5.760515in}{0.613486in}}%
\pgfpathclose%
\pgfusepath{fill}%
\end{pgfscope}%
\begin{pgfscope}%
\pgfpathrectangle{\pgfqpoint{0.693757in}{0.613486in}}{\pgfqpoint{5.541243in}{3.963477in}}%
\pgfusepath{clip}%
\pgfsetbuttcap%
\pgfsetmiterjoin%
\definecolor{currentfill}{rgb}{0.000000,0.000000,1.000000}%
\pgfsetfillcolor{currentfill}%
\pgfsetlinewidth{0.000000pt}%
\definecolor{currentstroke}{rgb}{0.000000,0.000000,0.000000}%
\pgfsetstrokecolor{currentstroke}%
\pgfsetstrokeopacity{0.000000}%
\pgfsetdash{}{0pt}%
\pgfpathmoveto{\pgfqpoint{5.773021in}{0.613486in}}%
\pgfpathlineto{\pgfqpoint{5.783026in}{0.613486in}}%
\pgfpathlineto{\pgfqpoint{5.783026in}{2.595224in}}%
\pgfpathlineto{\pgfqpoint{5.773021in}{2.595224in}}%
\pgfpathlineto{\pgfqpoint{5.773021in}{0.613486in}}%
\pgfpathclose%
\pgfusepath{fill}%
\end{pgfscope}%
\begin{pgfscope}%
\pgfpathrectangle{\pgfqpoint{0.693757in}{0.613486in}}{\pgfqpoint{5.541243in}{3.963477in}}%
\pgfusepath{clip}%
\pgfsetbuttcap%
\pgfsetmiterjoin%
\definecolor{currentfill}{rgb}{0.000000,0.000000,1.000000}%
\pgfsetfillcolor{currentfill}%
\pgfsetlinewidth{0.000000pt}%
\definecolor{currentstroke}{rgb}{0.000000,0.000000,0.000000}%
\pgfsetstrokecolor{currentstroke}%
\pgfsetstrokeopacity{0.000000}%
\pgfsetdash{}{0pt}%
\pgfpathmoveto{\pgfqpoint{5.785528in}{0.613486in}}%
\pgfpathlineto{\pgfqpoint{5.795532in}{0.613486in}}%
\pgfpathlineto{\pgfqpoint{5.795532in}{1.612282in}}%
\pgfpathlineto{\pgfqpoint{5.785528in}{1.612282in}}%
\pgfpathlineto{\pgfqpoint{5.785528in}{0.613486in}}%
\pgfpathclose%
\pgfusepath{fill}%
\end{pgfscope}%
\begin{pgfscope}%
\pgfpathrectangle{\pgfqpoint{0.693757in}{0.613486in}}{\pgfqpoint{5.541243in}{3.963477in}}%
\pgfusepath{clip}%
\pgfsetbuttcap%
\pgfsetmiterjoin%
\definecolor{currentfill}{rgb}{0.000000,0.000000,1.000000}%
\pgfsetfillcolor{currentfill}%
\pgfsetlinewidth{0.000000pt}%
\definecolor{currentstroke}{rgb}{0.000000,0.000000,0.000000}%
\pgfsetstrokecolor{currentstroke}%
\pgfsetstrokeopacity{0.000000}%
\pgfsetdash{}{0pt}%
\pgfpathmoveto{\pgfqpoint{5.798034in}{0.613486in}}%
\pgfpathlineto{\pgfqpoint{5.808039in}{0.613486in}}%
\pgfpathlineto{\pgfqpoint{5.808039in}{1.604355in}}%
\pgfpathlineto{\pgfqpoint{5.798034in}{1.604355in}}%
\pgfpathlineto{\pgfqpoint{5.798034in}{0.613486in}}%
\pgfpathclose%
\pgfusepath{fill}%
\end{pgfscope}%
\begin{pgfscope}%
\pgfpathrectangle{\pgfqpoint{0.693757in}{0.613486in}}{\pgfqpoint{5.541243in}{3.963477in}}%
\pgfusepath{clip}%
\pgfsetbuttcap%
\pgfsetmiterjoin%
\definecolor{currentfill}{rgb}{0.000000,0.000000,1.000000}%
\pgfsetfillcolor{currentfill}%
\pgfsetlinewidth{0.000000pt}%
\definecolor{currentstroke}{rgb}{0.000000,0.000000,0.000000}%
\pgfsetstrokecolor{currentstroke}%
\pgfsetstrokeopacity{0.000000}%
\pgfsetdash{}{0pt}%
\pgfpathmoveto{\pgfqpoint{5.810540in}{0.613486in}}%
\pgfpathlineto{\pgfqpoint{5.820545in}{0.613486in}}%
\pgfpathlineto{\pgfqpoint{5.820545in}{2.611074in}}%
\pgfpathlineto{\pgfqpoint{5.810540in}{2.611074in}}%
\pgfpathlineto{\pgfqpoint{5.810540in}{0.613486in}}%
\pgfpathclose%
\pgfusepath{fill}%
\end{pgfscope}%
\begin{pgfscope}%
\pgfpathrectangle{\pgfqpoint{0.693757in}{0.613486in}}{\pgfqpoint{5.541243in}{3.963477in}}%
\pgfusepath{clip}%
\pgfsetbuttcap%
\pgfsetmiterjoin%
\definecolor{currentfill}{rgb}{0.000000,0.000000,1.000000}%
\pgfsetfillcolor{currentfill}%
\pgfsetlinewidth{0.000000pt}%
\definecolor{currentstroke}{rgb}{0.000000,0.000000,0.000000}%
\pgfsetstrokecolor{currentstroke}%
\pgfsetstrokeopacity{0.000000}%
\pgfsetdash{}{0pt}%
\pgfpathmoveto{\pgfqpoint{5.823046in}{0.613486in}}%
\pgfpathlineto{\pgfqpoint{5.833051in}{0.613486in}}%
\pgfpathlineto{\pgfqpoint{5.833051in}{2.595224in}}%
\pgfpathlineto{\pgfqpoint{5.823046in}{2.595224in}}%
\pgfpathlineto{\pgfqpoint{5.823046in}{0.613486in}}%
\pgfpathclose%
\pgfusepath{fill}%
\end{pgfscope}%
\begin{pgfscope}%
\pgfpathrectangle{\pgfqpoint{0.693757in}{0.613486in}}{\pgfqpoint{5.541243in}{3.963477in}}%
\pgfusepath{clip}%
\pgfsetbuttcap%
\pgfsetmiterjoin%
\definecolor{currentfill}{rgb}{0.000000,0.000000,1.000000}%
\pgfsetfillcolor{currentfill}%
\pgfsetlinewidth{0.000000pt}%
\definecolor{currentstroke}{rgb}{0.000000,0.000000,0.000000}%
\pgfsetstrokecolor{currentstroke}%
\pgfsetstrokeopacity{0.000000}%
\pgfsetdash{}{0pt}%
\pgfpathmoveto{\pgfqpoint{5.835552in}{0.613486in}}%
\pgfpathlineto{\pgfqpoint{5.845557in}{0.613486in}}%
\pgfpathlineto{\pgfqpoint{5.845557in}{1.612282in}}%
\pgfpathlineto{\pgfqpoint{5.835552in}{1.612282in}}%
\pgfpathlineto{\pgfqpoint{5.835552in}{0.613486in}}%
\pgfpathclose%
\pgfusepath{fill}%
\end{pgfscope}%
\begin{pgfscope}%
\pgfpathrectangle{\pgfqpoint{0.693757in}{0.613486in}}{\pgfqpoint{5.541243in}{3.963477in}}%
\pgfusepath{clip}%
\pgfsetbuttcap%
\pgfsetmiterjoin%
\definecolor{currentfill}{rgb}{0.000000,0.000000,1.000000}%
\pgfsetfillcolor{currentfill}%
\pgfsetlinewidth{0.000000pt}%
\definecolor{currentstroke}{rgb}{0.000000,0.000000,0.000000}%
\pgfsetstrokecolor{currentstroke}%
\pgfsetstrokeopacity{0.000000}%
\pgfsetdash{}{0pt}%
\pgfpathmoveto{\pgfqpoint{5.848058in}{0.613486in}}%
\pgfpathlineto{\pgfqpoint{5.858063in}{0.613486in}}%
\pgfpathlineto{\pgfqpoint{5.858063in}{1.604355in}}%
\pgfpathlineto{\pgfqpoint{5.848058in}{1.604355in}}%
\pgfpathlineto{\pgfqpoint{5.848058in}{0.613486in}}%
\pgfpathclose%
\pgfusepath{fill}%
\end{pgfscope}%
\begin{pgfscope}%
\pgfpathrectangle{\pgfqpoint{0.693757in}{0.613486in}}{\pgfqpoint{5.541243in}{3.963477in}}%
\pgfusepath{clip}%
\pgfsetbuttcap%
\pgfsetmiterjoin%
\definecolor{currentfill}{rgb}{0.000000,0.000000,1.000000}%
\pgfsetfillcolor{currentfill}%
\pgfsetlinewidth{0.000000pt}%
\definecolor{currentstroke}{rgb}{0.000000,0.000000,0.000000}%
\pgfsetstrokecolor{currentstroke}%
\pgfsetstrokeopacity{0.000000}%
\pgfsetdash{}{0pt}%
\pgfpathmoveto{\pgfqpoint{5.860565in}{0.613486in}}%
\pgfpathlineto{\pgfqpoint{5.870570in}{0.613486in}}%
\pgfpathlineto{\pgfqpoint{5.870570in}{2.611074in}}%
\pgfpathlineto{\pgfqpoint{5.860565in}{2.611074in}}%
\pgfpathlineto{\pgfqpoint{5.860565in}{0.613486in}}%
\pgfpathclose%
\pgfusepath{fill}%
\end{pgfscope}%
\begin{pgfscope}%
\pgfpathrectangle{\pgfqpoint{0.693757in}{0.613486in}}{\pgfqpoint{5.541243in}{3.963477in}}%
\pgfusepath{clip}%
\pgfsetbuttcap%
\pgfsetmiterjoin%
\definecolor{currentfill}{rgb}{0.000000,0.000000,1.000000}%
\pgfsetfillcolor{currentfill}%
\pgfsetlinewidth{0.000000pt}%
\definecolor{currentstroke}{rgb}{0.000000,0.000000,0.000000}%
\pgfsetstrokecolor{currentstroke}%
\pgfsetstrokeopacity{0.000000}%
\pgfsetdash{}{0pt}%
\pgfpathmoveto{\pgfqpoint{5.873071in}{0.613486in}}%
\pgfpathlineto{\pgfqpoint{5.883076in}{0.613486in}}%
\pgfpathlineto{\pgfqpoint{5.883076in}{2.595224in}}%
\pgfpathlineto{\pgfqpoint{5.873071in}{2.595224in}}%
\pgfpathlineto{\pgfqpoint{5.873071in}{0.613486in}}%
\pgfpathclose%
\pgfusepath{fill}%
\end{pgfscope}%
\begin{pgfscope}%
\pgfpathrectangle{\pgfqpoint{0.693757in}{0.613486in}}{\pgfqpoint{5.541243in}{3.963477in}}%
\pgfusepath{clip}%
\pgfsetbuttcap%
\pgfsetmiterjoin%
\definecolor{currentfill}{rgb}{0.000000,0.000000,1.000000}%
\pgfsetfillcolor{currentfill}%
\pgfsetlinewidth{0.000000pt}%
\definecolor{currentstroke}{rgb}{0.000000,0.000000,0.000000}%
\pgfsetstrokecolor{currentstroke}%
\pgfsetstrokeopacity{0.000000}%
\pgfsetdash{}{0pt}%
\pgfpathmoveto{\pgfqpoint{5.885577in}{0.613486in}}%
\pgfpathlineto{\pgfqpoint{5.895582in}{0.613486in}}%
\pgfpathlineto{\pgfqpoint{5.895582in}{1.612282in}}%
\pgfpathlineto{\pgfqpoint{5.885577in}{1.612282in}}%
\pgfpathlineto{\pgfqpoint{5.885577in}{0.613486in}}%
\pgfpathclose%
\pgfusepath{fill}%
\end{pgfscope}%
\begin{pgfscope}%
\pgfpathrectangle{\pgfqpoint{0.693757in}{0.613486in}}{\pgfqpoint{5.541243in}{3.963477in}}%
\pgfusepath{clip}%
\pgfsetbuttcap%
\pgfsetmiterjoin%
\definecolor{currentfill}{rgb}{0.000000,0.000000,1.000000}%
\pgfsetfillcolor{currentfill}%
\pgfsetlinewidth{0.000000pt}%
\definecolor{currentstroke}{rgb}{0.000000,0.000000,0.000000}%
\pgfsetstrokecolor{currentstroke}%
\pgfsetstrokeopacity{0.000000}%
\pgfsetdash{}{0pt}%
\pgfpathmoveto{\pgfqpoint{5.898083in}{0.613486in}}%
\pgfpathlineto{\pgfqpoint{5.908088in}{0.613486in}}%
\pgfpathlineto{\pgfqpoint{5.908088in}{1.604355in}}%
\pgfpathlineto{\pgfqpoint{5.898083in}{1.604355in}}%
\pgfpathlineto{\pgfqpoint{5.898083in}{0.613486in}}%
\pgfpathclose%
\pgfusepath{fill}%
\end{pgfscope}%
\begin{pgfscope}%
\pgfpathrectangle{\pgfqpoint{0.693757in}{0.613486in}}{\pgfqpoint{5.541243in}{3.963477in}}%
\pgfusepath{clip}%
\pgfsetbuttcap%
\pgfsetmiterjoin%
\definecolor{currentfill}{rgb}{0.000000,0.000000,1.000000}%
\pgfsetfillcolor{currentfill}%
\pgfsetlinewidth{0.000000pt}%
\definecolor{currentstroke}{rgb}{0.000000,0.000000,0.000000}%
\pgfsetstrokecolor{currentstroke}%
\pgfsetstrokeopacity{0.000000}%
\pgfsetdash{}{0pt}%
\pgfpathmoveto{\pgfqpoint{5.910589in}{0.613486in}}%
\pgfpathlineto{\pgfqpoint{5.920594in}{0.613486in}}%
\pgfpathlineto{\pgfqpoint{5.920594in}{2.611074in}}%
\pgfpathlineto{\pgfqpoint{5.910589in}{2.611074in}}%
\pgfpathlineto{\pgfqpoint{5.910589in}{0.613486in}}%
\pgfpathclose%
\pgfusepath{fill}%
\end{pgfscope}%
\begin{pgfscope}%
\pgfpathrectangle{\pgfqpoint{0.693757in}{0.613486in}}{\pgfqpoint{5.541243in}{3.963477in}}%
\pgfusepath{clip}%
\pgfsetbuttcap%
\pgfsetmiterjoin%
\definecolor{currentfill}{rgb}{0.000000,0.000000,1.000000}%
\pgfsetfillcolor{currentfill}%
\pgfsetlinewidth{0.000000pt}%
\definecolor{currentstroke}{rgb}{0.000000,0.000000,0.000000}%
\pgfsetstrokecolor{currentstroke}%
\pgfsetstrokeopacity{0.000000}%
\pgfsetdash{}{0pt}%
\pgfpathmoveto{\pgfqpoint{5.923096in}{0.613486in}}%
\pgfpathlineto{\pgfqpoint{5.933101in}{0.613486in}}%
\pgfpathlineto{\pgfqpoint{5.933101in}{2.595224in}}%
\pgfpathlineto{\pgfqpoint{5.923096in}{2.595224in}}%
\pgfpathlineto{\pgfqpoint{5.923096in}{0.613486in}}%
\pgfpathclose%
\pgfusepath{fill}%
\end{pgfscope}%
\begin{pgfscope}%
\pgfpathrectangle{\pgfqpoint{0.693757in}{0.613486in}}{\pgfqpoint{5.541243in}{3.963477in}}%
\pgfusepath{clip}%
\pgfsetbuttcap%
\pgfsetmiterjoin%
\definecolor{currentfill}{rgb}{0.000000,0.000000,1.000000}%
\pgfsetfillcolor{currentfill}%
\pgfsetlinewidth{0.000000pt}%
\definecolor{currentstroke}{rgb}{0.000000,0.000000,0.000000}%
\pgfsetstrokecolor{currentstroke}%
\pgfsetstrokeopacity{0.000000}%
\pgfsetdash{}{0pt}%
\pgfpathmoveto{\pgfqpoint{5.935602in}{0.613486in}}%
\pgfpathlineto{\pgfqpoint{5.945607in}{0.613486in}}%
\pgfpathlineto{\pgfqpoint{5.945607in}{1.612282in}}%
\pgfpathlineto{\pgfqpoint{5.935602in}{1.612282in}}%
\pgfpathlineto{\pgfqpoint{5.935602in}{0.613486in}}%
\pgfpathclose%
\pgfusepath{fill}%
\end{pgfscope}%
\begin{pgfscope}%
\pgfpathrectangle{\pgfqpoint{0.693757in}{0.613486in}}{\pgfqpoint{5.541243in}{3.963477in}}%
\pgfusepath{clip}%
\pgfsetbuttcap%
\pgfsetmiterjoin%
\definecolor{currentfill}{rgb}{0.000000,0.000000,1.000000}%
\pgfsetfillcolor{currentfill}%
\pgfsetlinewidth{0.000000pt}%
\definecolor{currentstroke}{rgb}{0.000000,0.000000,0.000000}%
\pgfsetstrokecolor{currentstroke}%
\pgfsetstrokeopacity{0.000000}%
\pgfsetdash{}{0pt}%
\pgfpathmoveto{\pgfqpoint{5.948108in}{0.613486in}}%
\pgfpathlineto{\pgfqpoint{5.958113in}{0.613486in}}%
\pgfpathlineto{\pgfqpoint{5.958113in}{1.604355in}}%
\pgfpathlineto{\pgfqpoint{5.948108in}{1.604355in}}%
\pgfpathlineto{\pgfqpoint{5.948108in}{0.613486in}}%
\pgfpathclose%
\pgfusepath{fill}%
\end{pgfscope}%
\begin{pgfscope}%
\pgfpathrectangle{\pgfqpoint{0.693757in}{0.613486in}}{\pgfqpoint{5.541243in}{3.963477in}}%
\pgfusepath{clip}%
\pgfsetbuttcap%
\pgfsetmiterjoin%
\definecolor{currentfill}{rgb}{0.000000,0.000000,1.000000}%
\pgfsetfillcolor{currentfill}%
\pgfsetlinewidth{0.000000pt}%
\definecolor{currentstroke}{rgb}{0.000000,0.000000,0.000000}%
\pgfsetstrokecolor{currentstroke}%
\pgfsetstrokeopacity{0.000000}%
\pgfsetdash{}{0pt}%
\pgfpathmoveto{\pgfqpoint{5.960614in}{0.613486in}}%
\pgfpathlineto{\pgfqpoint{5.970619in}{0.613486in}}%
\pgfpathlineto{\pgfqpoint{5.970619in}{2.611074in}}%
\pgfpathlineto{\pgfqpoint{5.960614in}{2.611074in}}%
\pgfpathlineto{\pgfqpoint{5.960614in}{0.613486in}}%
\pgfpathclose%
\pgfusepath{fill}%
\end{pgfscope}%
\begin{pgfscope}%
\pgfpathrectangle{\pgfqpoint{0.693757in}{0.613486in}}{\pgfqpoint{5.541243in}{3.963477in}}%
\pgfusepath{clip}%
\pgfsetbuttcap%
\pgfsetmiterjoin%
\definecolor{currentfill}{rgb}{0.000000,0.000000,1.000000}%
\pgfsetfillcolor{currentfill}%
\pgfsetlinewidth{0.000000pt}%
\definecolor{currentstroke}{rgb}{0.000000,0.000000,0.000000}%
\pgfsetstrokecolor{currentstroke}%
\pgfsetstrokeopacity{0.000000}%
\pgfsetdash{}{0pt}%
\pgfpathmoveto{\pgfqpoint{5.973120in}{0.613486in}}%
\pgfpathlineto{\pgfqpoint{5.983125in}{0.613486in}}%
\pgfpathlineto{\pgfqpoint{5.983125in}{2.595224in}}%
\pgfpathlineto{\pgfqpoint{5.973120in}{2.595224in}}%
\pgfpathlineto{\pgfqpoint{5.973120in}{0.613486in}}%
\pgfpathclose%
\pgfusepath{fill}%
\end{pgfscope}%
\begin{pgfscope}%
\pgfsetbuttcap%
\pgfsetroundjoin%
\definecolor{currentfill}{rgb}{0.000000,0.000000,0.000000}%
\pgfsetfillcolor{currentfill}%
\pgfsetlinewidth{0.803000pt}%
\definecolor{currentstroke}{rgb}{0.000000,0.000000,0.000000}%
\pgfsetstrokecolor{currentstroke}%
\pgfsetdash{}{0pt}%
\pgfsys@defobject{currentmarker}{\pgfqpoint{0.000000in}{-0.048611in}}{\pgfqpoint{0.000000in}{0.000000in}}{%
\pgfpathmoveto{\pgfqpoint{0.000000in}{0.000000in}}%
\pgfpathlineto{\pgfqpoint{0.000000in}{-0.048611in}}%
\pgfusepath{stroke,fill}%
}%
\begin{pgfscope}%
\pgfsys@transformshift{0.938128in}{0.613486in}%
\pgfsys@useobject{currentmarker}{}%
\end{pgfscope}%
\end{pgfscope}%
\begin{pgfscope}%
\definecolor{textcolor}{rgb}{0.000000,0.000000,0.000000}%
\pgfsetstrokecolor{textcolor}%
\pgfsetfillcolor{textcolor}%
\pgftext[x=0.938128in,y=0.516264in,,top]{\color{textcolor}{\sffamily\fontsize{11.000000}{13.200000}\selectfont\catcode`\^=\active\def^{\ifmmode\sp\else\^{}\fi}\catcode`\%=\active\def%{\%}$\mathdefault{0}$}}%
\end{pgfscope}%
\begin{pgfscope}%
\pgfsetbuttcap%
\pgfsetroundjoin%
\definecolor{currentfill}{rgb}{0.000000,0.000000,0.000000}%
\pgfsetfillcolor{currentfill}%
\pgfsetlinewidth{0.803000pt}%
\definecolor{currentstroke}{rgb}{0.000000,0.000000,0.000000}%
\pgfsetstrokecolor{currentstroke}%
\pgfsetdash{}{0pt}%
\pgfsys@defobject{currentmarker}{\pgfqpoint{0.000000in}{-0.048611in}}{\pgfqpoint{0.000000in}{0.000000in}}{%
\pgfpathmoveto{\pgfqpoint{0.000000in}{0.000000in}}%
\pgfpathlineto{\pgfqpoint{0.000000in}{-0.048611in}}%
\pgfusepath{stroke,fill}%
}%
\begin{pgfscope}%
\pgfsys@transformshift{1.563438in}{0.613486in}%
\pgfsys@useobject{currentmarker}{}%
\end{pgfscope}%
\end{pgfscope}%
\begin{pgfscope}%
\definecolor{textcolor}{rgb}{0.000000,0.000000,0.000000}%
\pgfsetstrokecolor{textcolor}%
\pgfsetfillcolor{textcolor}%
\pgftext[x=1.563438in,y=0.516264in,,top]{\color{textcolor}{\sffamily\fontsize{11.000000}{13.200000}\selectfont\catcode`\^=\active\def^{\ifmmode\sp\else\^{}\fi}\catcode`\%=\active\def%{\%}$\mathdefault{50}$}}%
\end{pgfscope}%
\begin{pgfscope}%
\pgfsetbuttcap%
\pgfsetroundjoin%
\definecolor{currentfill}{rgb}{0.000000,0.000000,0.000000}%
\pgfsetfillcolor{currentfill}%
\pgfsetlinewidth{0.803000pt}%
\definecolor{currentstroke}{rgb}{0.000000,0.000000,0.000000}%
\pgfsetstrokecolor{currentstroke}%
\pgfsetdash{}{0pt}%
\pgfsys@defobject{currentmarker}{\pgfqpoint{0.000000in}{-0.048611in}}{\pgfqpoint{0.000000in}{0.000000in}}{%
\pgfpathmoveto{\pgfqpoint{0.000000in}{0.000000in}}%
\pgfpathlineto{\pgfqpoint{0.000000in}{-0.048611in}}%
\pgfusepath{stroke,fill}%
}%
\begin{pgfscope}%
\pgfsys@transformshift{2.188747in}{0.613486in}%
\pgfsys@useobject{currentmarker}{}%
\end{pgfscope}%
\end{pgfscope}%
\begin{pgfscope}%
\definecolor{textcolor}{rgb}{0.000000,0.000000,0.000000}%
\pgfsetstrokecolor{textcolor}%
\pgfsetfillcolor{textcolor}%
\pgftext[x=2.188747in,y=0.516264in,,top]{\color{textcolor}{\sffamily\fontsize{11.000000}{13.200000}\selectfont\catcode`\^=\active\def^{\ifmmode\sp\else\^{}\fi}\catcode`\%=\active\def%{\%}$\mathdefault{100}$}}%
\end{pgfscope}%
\begin{pgfscope}%
\pgfsetbuttcap%
\pgfsetroundjoin%
\definecolor{currentfill}{rgb}{0.000000,0.000000,0.000000}%
\pgfsetfillcolor{currentfill}%
\pgfsetlinewidth{0.803000pt}%
\definecolor{currentstroke}{rgb}{0.000000,0.000000,0.000000}%
\pgfsetstrokecolor{currentstroke}%
\pgfsetdash{}{0pt}%
\pgfsys@defobject{currentmarker}{\pgfqpoint{0.000000in}{-0.048611in}}{\pgfqpoint{0.000000in}{0.000000in}}{%
\pgfpathmoveto{\pgfqpoint{0.000000in}{0.000000in}}%
\pgfpathlineto{\pgfqpoint{0.000000in}{-0.048611in}}%
\pgfusepath{stroke,fill}%
}%
\begin{pgfscope}%
\pgfsys@transformshift{2.814057in}{0.613486in}%
\pgfsys@useobject{currentmarker}{}%
\end{pgfscope}%
\end{pgfscope}%
\begin{pgfscope}%
\definecolor{textcolor}{rgb}{0.000000,0.000000,0.000000}%
\pgfsetstrokecolor{textcolor}%
\pgfsetfillcolor{textcolor}%
\pgftext[x=2.814057in,y=0.516264in,,top]{\color{textcolor}{\sffamily\fontsize{11.000000}{13.200000}\selectfont\catcode`\^=\active\def^{\ifmmode\sp\else\^{}\fi}\catcode`\%=\active\def%{\%}$\mathdefault{150}$}}%
\end{pgfscope}%
\begin{pgfscope}%
\pgfsetbuttcap%
\pgfsetroundjoin%
\definecolor{currentfill}{rgb}{0.000000,0.000000,0.000000}%
\pgfsetfillcolor{currentfill}%
\pgfsetlinewidth{0.803000pt}%
\definecolor{currentstroke}{rgb}{0.000000,0.000000,0.000000}%
\pgfsetstrokecolor{currentstroke}%
\pgfsetdash{}{0pt}%
\pgfsys@defobject{currentmarker}{\pgfqpoint{0.000000in}{-0.048611in}}{\pgfqpoint{0.000000in}{0.000000in}}{%
\pgfpathmoveto{\pgfqpoint{0.000000in}{0.000000in}}%
\pgfpathlineto{\pgfqpoint{0.000000in}{-0.048611in}}%
\pgfusepath{stroke,fill}%
}%
\begin{pgfscope}%
\pgfsys@transformshift{3.439366in}{0.613486in}%
\pgfsys@useobject{currentmarker}{}%
\end{pgfscope}%
\end{pgfscope}%
\begin{pgfscope}%
\definecolor{textcolor}{rgb}{0.000000,0.000000,0.000000}%
\pgfsetstrokecolor{textcolor}%
\pgfsetfillcolor{textcolor}%
\pgftext[x=3.439366in,y=0.516264in,,top]{\color{textcolor}{\sffamily\fontsize{11.000000}{13.200000}\selectfont\catcode`\^=\active\def^{\ifmmode\sp\else\^{}\fi}\catcode`\%=\active\def%{\%}$\mathdefault{200}$}}%
\end{pgfscope}%
\begin{pgfscope}%
\pgfsetbuttcap%
\pgfsetroundjoin%
\definecolor{currentfill}{rgb}{0.000000,0.000000,0.000000}%
\pgfsetfillcolor{currentfill}%
\pgfsetlinewidth{0.803000pt}%
\definecolor{currentstroke}{rgb}{0.000000,0.000000,0.000000}%
\pgfsetstrokecolor{currentstroke}%
\pgfsetdash{}{0pt}%
\pgfsys@defobject{currentmarker}{\pgfqpoint{0.000000in}{-0.048611in}}{\pgfqpoint{0.000000in}{0.000000in}}{%
\pgfpathmoveto{\pgfqpoint{0.000000in}{0.000000in}}%
\pgfpathlineto{\pgfqpoint{0.000000in}{-0.048611in}}%
\pgfusepath{stroke,fill}%
}%
\begin{pgfscope}%
\pgfsys@transformshift{4.064676in}{0.613486in}%
\pgfsys@useobject{currentmarker}{}%
\end{pgfscope}%
\end{pgfscope}%
\begin{pgfscope}%
\definecolor{textcolor}{rgb}{0.000000,0.000000,0.000000}%
\pgfsetstrokecolor{textcolor}%
\pgfsetfillcolor{textcolor}%
\pgftext[x=4.064676in,y=0.516264in,,top]{\color{textcolor}{\sffamily\fontsize{11.000000}{13.200000}\selectfont\catcode`\^=\active\def^{\ifmmode\sp\else\^{}\fi}\catcode`\%=\active\def%{\%}$\mathdefault{250}$}}%
\end{pgfscope}%
\begin{pgfscope}%
\pgfsetbuttcap%
\pgfsetroundjoin%
\definecolor{currentfill}{rgb}{0.000000,0.000000,0.000000}%
\pgfsetfillcolor{currentfill}%
\pgfsetlinewidth{0.803000pt}%
\definecolor{currentstroke}{rgb}{0.000000,0.000000,0.000000}%
\pgfsetstrokecolor{currentstroke}%
\pgfsetdash{}{0pt}%
\pgfsys@defobject{currentmarker}{\pgfqpoint{0.000000in}{-0.048611in}}{\pgfqpoint{0.000000in}{0.000000in}}{%
\pgfpathmoveto{\pgfqpoint{0.000000in}{0.000000in}}%
\pgfpathlineto{\pgfqpoint{0.000000in}{-0.048611in}}%
\pgfusepath{stroke,fill}%
}%
\begin{pgfscope}%
\pgfsys@transformshift{4.689985in}{0.613486in}%
\pgfsys@useobject{currentmarker}{}%
\end{pgfscope}%
\end{pgfscope}%
\begin{pgfscope}%
\definecolor{textcolor}{rgb}{0.000000,0.000000,0.000000}%
\pgfsetstrokecolor{textcolor}%
\pgfsetfillcolor{textcolor}%
\pgftext[x=4.689985in,y=0.516264in,,top]{\color{textcolor}{\sffamily\fontsize{11.000000}{13.200000}\selectfont\catcode`\^=\active\def^{\ifmmode\sp\else\^{}\fi}\catcode`\%=\active\def%{\%}$\mathdefault{300}$}}%
\end{pgfscope}%
\begin{pgfscope}%
\pgfsetbuttcap%
\pgfsetroundjoin%
\definecolor{currentfill}{rgb}{0.000000,0.000000,0.000000}%
\pgfsetfillcolor{currentfill}%
\pgfsetlinewidth{0.803000pt}%
\definecolor{currentstroke}{rgb}{0.000000,0.000000,0.000000}%
\pgfsetstrokecolor{currentstroke}%
\pgfsetdash{}{0pt}%
\pgfsys@defobject{currentmarker}{\pgfqpoint{0.000000in}{-0.048611in}}{\pgfqpoint{0.000000in}{0.000000in}}{%
\pgfpathmoveto{\pgfqpoint{0.000000in}{0.000000in}}%
\pgfpathlineto{\pgfqpoint{0.000000in}{-0.048611in}}%
\pgfusepath{stroke,fill}%
}%
\begin{pgfscope}%
\pgfsys@transformshift{5.315295in}{0.613486in}%
\pgfsys@useobject{currentmarker}{}%
\end{pgfscope}%
\end{pgfscope}%
\begin{pgfscope}%
\definecolor{textcolor}{rgb}{0.000000,0.000000,0.000000}%
\pgfsetstrokecolor{textcolor}%
\pgfsetfillcolor{textcolor}%
\pgftext[x=5.315295in,y=0.516264in,,top]{\color{textcolor}{\sffamily\fontsize{11.000000}{13.200000}\selectfont\catcode`\^=\active\def^{\ifmmode\sp\else\^{}\fi}\catcode`\%=\active\def%{\%}$\mathdefault{350}$}}%
\end{pgfscope}%
\begin{pgfscope}%
\pgfsetbuttcap%
\pgfsetroundjoin%
\definecolor{currentfill}{rgb}{0.000000,0.000000,0.000000}%
\pgfsetfillcolor{currentfill}%
\pgfsetlinewidth{0.803000pt}%
\definecolor{currentstroke}{rgb}{0.000000,0.000000,0.000000}%
\pgfsetstrokecolor{currentstroke}%
\pgfsetdash{}{0pt}%
\pgfsys@defobject{currentmarker}{\pgfqpoint{0.000000in}{-0.048611in}}{\pgfqpoint{0.000000in}{0.000000in}}{%
\pgfpathmoveto{\pgfqpoint{0.000000in}{0.000000in}}%
\pgfpathlineto{\pgfqpoint{0.000000in}{-0.048611in}}%
\pgfusepath{stroke,fill}%
}%
\begin{pgfscope}%
\pgfsys@transformshift{5.940604in}{0.613486in}%
\pgfsys@useobject{currentmarker}{}%
\end{pgfscope}%
\end{pgfscope}%
\begin{pgfscope}%
\definecolor{textcolor}{rgb}{0.000000,0.000000,0.000000}%
\pgfsetstrokecolor{textcolor}%
\pgfsetfillcolor{textcolor}%
\pgftext[x=5.940604in,y=0.516264in,,top]{\color{textcolor}{\sffamily\fontsize{11.000000}{13.200000}\selectfont\catcode`\^=\active\def^{\ifmmode\sp\else\^{}\fi}\catcode`\%=\active\def%{\%}$\mathdefault{400}$}}%
\end{pgfscope}%
\begin{pgfscope}%
\definecolor{textcolor}{rgb}{0.000000,0.000000,0.000000}%
\pgfsetstrokecolor{textcolor}%
\pgfsetfillcolor{textcolor}%
\pgftext[x=3.464379in,y=0.312854in,,top]{\color{textcolor}{\sffamily\fontsize{11.000000}{13.200000}\selectfont\catcode`\^=\active\def^{\ifmmode\sp\else\^{}\fi}\catcode`\%=\active\def%{\%}Kernel index}}%
\end{pgfscope}%
\begin{pgfscope}%
\pgfsetbuttcap%
\pgfsetroundjoin%
\definecolor{currentfill}{rgb}{0.000000,0.000000,0.000000}%
\pgfsetfillcolor{currentfill}%
\pgfsetlinewidth{0.803000pt}%
\definecolor{currentstroke}{rgb}{0.000000,0.000000,0.000000}%
\pgfsetstrokecolor{currentstroke}%
\pgfsetdash{}{0pt}%
\pgfsys@defobject{currentmarker}{\pgfqpoint{-0.048611in}{0.000000in}}{\pgfqpoint{-0.000000in}{0.000000in}}{%
\pgfpathmoveto{\pgfqpoint{-0.000000in}{0.000000in}}%
\pgfpathlineto{\pgfqpoint{-0.048611in}{0.000000in}}%
\pgfusepath{stroke,fill}%
}%
\begin{pgfscope}%
\pgfsys@transformshift{0.693757in}{0.613486in}%
\pgfsys@useobject{currentmarker}{}%
\end{pgfscope}%
\end{pgfscope}%
\begin{pgfscope}%
\definecolor{textcolor}{rgb}{0.000000,0.000000,0.000000}%
\pgfsetstrokecolor{textcolor}%
\pgfsetfillcolor{textcolor}%
\pgftext[x=0.520493in, y=0.555448in, left, base]{\color{textcolor}{\sffamily\fontsize{11.000000}{13.200000}\selectfont\catcode`\^=\active\def^{\ifmmode\sp\else\^{}\fi}\catcode`\%=\active\def%{\%}$\mathdefault{0}$}}%
\end{pgfscope}%
\begin{pgfscope}%
\pgfsetbuttcap%
\pgfsetroundjoin%
\definecolor{currentfill}{rgb}{0.000000,0.000000,0.000000}%
\pgfsetfillcolor{currentfill}%
\pgfsetlinewidth{0.803000pt}%
\definecolor{currentstroke}{rgb}{0.000000,0.000000,0.000000}%
\pgfsetstrokecolor{currentstroke}%
\pgfsetdash{}{0pt}%
\pgfsys@defobject{currentmarker}{\pgfqpoint{-0.048611in}{0.000000in}}{\pgfqpoint{-0.000000in}{0.000000in}}{%
\pgfpathmoveto{\pgfqpoint{-0.000000in}{0.000000in}}%
\pgfpathlineto{\pgfqpoint{-0.048611in}{0.000000in}}%
\pgfusepath{stroke,fill}%
}%
\begin{pgfscope}%
\pgfsys@transformshift{0.693757in}{1.406181in}%
\pgfsys@useobject{currentmarker}{}%
\end{pgfscope}%
\end{pgfscope}%
\begin{pgfscope}%
\definecolor{textcolor}{rgb}{0.000000,0.000000,0.000000}%
\pgfsetstrokecolor{textcolor}%
\pgfsetfillcolor{textcolor}%
\pgftext[x=0.444451in, y=1.348144in, left, base]{\color{textcolor}{\sffamily\fontsize{11.000000}{13.200000}\selectfont\catcode`\^=\active\def^{\ifmmode\sp\else\^{}\fi}\catcode`\%=\active\def%{\%}$\mathdefault{20}$}}%
\end{pgfscope}%
\begin{pgfscope}%
\pgfsetbuttcap%
\pgfsetroundjoin%
\definecolor{currentfill}{rgb}{0.000000,0.000000,0.000000}%
\pgfsetfillcolor{currentfill}%
\pgfsetlinewidth{0.803000pt}%
\definecolor{currentstroke}{rgb}{0.000000,0.000000,0.000000}%
\pgfsetstrokecolor{currentstroke}%
\pgfsetdash{}{0pt}%
\pgfsys@defobject{currentmarker}{\pgfqpoint{-0.048611in}{0.000000in}}{\pgfqpoint{-0.000000in}{0.000000in}}{%
\pgfpathmoveto{\pgfqpoint{-0.000000in}{0.000000in}}%
\pgfpathlineto{\pgfqpoint{-0.048611in}{0.000000in}}%
\pgfusepath{stroke,fill}%
}%
\begin{pgfscope}%
\pgfsys@transformshift{0.693757in}{2.198876in}%
\pgfsys@useobject{currentmarker}{}%
\end{pgfscope}%
\end{pgfscope}%
\begin{pgfscope}%
\definecolor{textcolor}{rgb}{0.000000,0.000000,0.000000}%
\pgfsetstrokecolor{textcolor}%
\pgfsetfillcolor{textcolor}%
\pgftext[x=0.444451in, y=2.140839in, left, base]{\color{textcolor}{\sffamily\fontsize{11.000000}{13.200000}\selectfont\catcode`\^=\active\def^{\ifmmode\sp\else\^{}\fi}\catcode`\%=\active\def%{\%}$\mathdefault{40}$}}%
\end{pgfscope}%
\begin{pgfscope}%
\pgfsetbuttcap%
\pgfsetroundjoin%
\definecolor{currentfill}{rgb}{0.000000,0.000000,0.000000}%
\pgfsetfillcolor{currentfill}%
\pgfsetlinewidth{0.803000pt}%
\definecolor{currentstroke}{rgb}{0.000000,0.000000,0.000000}%
\pgfsetstrokecolor{currentstroke}%
\pgfsetdash{}{0pt}%
\pgfsys@defobject{currentmarker}{\pgfqpoint{-0.048611in}{0.000000in}}{\pgfqpoint{-0.000000in}{0.000000in}}{%
\pgfpathmoveto{\pgfqpoint{-0.000000in}{0.000000in}}%
\pgfpathlineto{\pgfqpoint{-0.048611in}{0.000000in}}%
\pgfusepath{stroke,fill}%
}%
\begin{pgfscope}%
\pgfsys@transformshift{0.693757in}{2.991572in}%
\pgfsys@useobject{currentmarker}{}%
\end{pgfscope}%
\end{pgfscope}%
\begin{pgfscope}%
\definecolor{textcolor}{rgb}{0.000000,0.000000,0.000000}%
\pgfsetstrokecolor{textcolor}%
\pgfsetfillcolor{textcolor}%
\pgftext[x=0.444451in, y=2.933534in, left, base]{\color{textcolor}{\sffamily\fontsize{11.000000}{13.200000}\selectfont\catcode`\^=\active\def^{\ifmmode\sp\else\^{}\fi}\catcode`\%=\active\def%{\%}$\mathdefault{60}$}}%
\end{pgfscope}%
\begin{pgfscope}%
\pgfsetbuttcap%
\pgfsetroundjoin%
\definecolor{currentfill}{rgb}{0.000000,0.000000,0.000000}%
\pgfsetfillcolor{currentfill}%
\pgfsetlinewidth{0.803000pt}%
\definecolor{currentstroke}{rgb}{0.000000,0.000000,0.000000}%
\pgfsetstrokecolor{currentstroke}%
\pgfsetdash{}{0pt}%
\pgfsys@defobject{currentmarker}{\pgfqpoint{-0.048611in}{0.000000in}}{\pgfqpoint{-0.000000in}{0.000000in}}{%
\pgfpathmoveto{\pgfqpoint{-0.000000in}{0.000000in}}%
\pgfpathlineto{\pgfqpoint{-0.048611in}{0.000000in}}%
\pgfusepath{stroke,fill}%
}%
\begin{pgfscope}%
\pgfsys@transformshift{0.693757in}{3.784267in}%
\pgfsys@useobject{currentmarker}{}%
\end{pgfscope}%
\end{pgfscope}%
\begin{pgfscope}%
\definecolor{textcolor}{rgb}{0.000000,0.000000,0.000000}%
\pgfsetstrokecolor{textcolor}%
\pgfsetfillcolor{textcolor}%
\pgftext[x=0.444451in, y=3.726229in, left, base]{\color{textcolor}{\sffamily\fontsize{11.000000}{13.200000}\selectfont\catcode`\^=\active\def^{\ifmmode\sp\else\^{}\fi}\catcode`\%=\active\def%{\%}$\mathdefault{80}$}}%
\end{pgfscope}%
\begin{pgfscope}%
\pgfsetbuttcap%
\pgfsetroundjoin%
\definecolor{currentfill}{rgb}{0.000000,0.000000,0.000000}%
\pgfsetfillcolor{currentfill}%
\pgfsetlinewidth{0.803000pt}%
\definecolor{currentstroke}{rgb}{0.000000,0.000000,0.000000}%
\pgfsetstrokecolor{currentstroke}%
\pgfsetdash{}{0pt}%
\pgfsys@defobject{currentmarker}{\pgfqpoint{-0.048611in}{0.000000in}}{\pgfqpoint{-0.000000in}{0.000000in}}{%
\pgfpathmoveto{\pgfqpoint{-0.000000in}{0.000000in}}%
\pgfpathlineto{\pgfqpoint{-0.048611in}{0.000000in}}%
\pgfusepath{stroke,fill}%
}%
\begin{pgfscope}%
\pgfsys@transformshift{0.693757in}{4.576962in}%
\pgfsys@useobject{currentmarker}{}%
\end{pgfscope}%
\end{pgfscope}%
\begin{pgfscope}%
\definecolor{textcolor}{rgb}{0.000000,0.000000,0.000000}%
\pgfsetstrokecolor{textcolor}%
\pgfsetfillcolor{textcolor}%
\pgftext[x=0.368410in, y=4.518925in, left, base]{\color{textcolor}{\sffamily\fontsize{11.000000}{13.200000}\selectfont\catcode`\^=\active\def^{\ifmmode\sp\else\^{}\fi}\catcode`\%=\active\def%{\%}$\mathdefault{100}$}}%
\end{pgfscope}%
\begin{pgfscope}%
\definecolor{textcolor}{rgb}{0.000000,0.000000,0.000000}%
\pgfsetstrokecolor{textcolor}%
\pgfsetfillcolor{textcolor}%
\pgftext[x=0.312854in,y=2.595224in,,bottom,rotate=90.000000]{\color{textcolor}{\sffamily\fontsize{11.000000}{13.200000}\selectfont\catcode`\^=\active\def^{\ifmmode\sp\else\^{}\fi}\catcode`\%=\active\def%{\%}Data reuse (in %)}}%
\end{pgfscope}%
\begin{pgfscope}%
\pgfsetrectcap%
\pgfsetmiterjoin%
\pgfsetlinewidth{0.803000pt}%
\definecolor{currentstroke}{rgb}{0.000000,0.000000,0.000000}%
\pgfsetstrokecolor{currentstroke}%
\pgfsetdash{}{0pt}%
\pgfpathmoveto{\pgfqpoint{0.693757in}{0.613486in}}%
\pgfpathlineto{\pgfqpoint{0.693757in}{4.576962in}}%
\pgfusepath{stroke}%
\end{pgfscope}%
\begin{pgfscope}%
\pgfsetrectcap%
\pgfsetmiterjoin%
\pgfsetlinewidth{0.803000pt}%
\definecolor{currentstroke}{rgb}{0.000000,0.000000,0.000000}%
\pgfsetstrokecolor{currentstroke}%
\pgfsetdash{}{0pt}%
\pgfpathmoveto{\pgfqpoint{6.235000in}{0.613486in}}%
\pgfpathlineto{\pgfqpoint{6.235000in}{4.576962in}}%
\pgfusepath{stroke}%
\end{pgfscope}%
\begin{pgfscope}%
\pgfsetrectcap%
\pgfsetmiterjoin%
\pgfsetlinewidth{0.803000pt}%
\definecolor{currentstroke}{rgb}{0.000000,0.000000,0.000000}%
\pgfsetstrokecolor{currentstroke}%
\pgfsetdash{}{0pt}%
\pgfpathmoveto{\pgfqpoint{0.693757in}{0.613486in}}%
\pgfpathlineto{\pgfqpoint{6.235000in}{0.613486in}}%
\pgfusepath{stroke}%
\end{pgfscope}%
\begin{pgfscope}%
\pgfsetrectcap%
\pgfsetmiterjoin%
\pgfsetlinewidth{0.803000pt}%
\definecolor{currentstroke}{rgb}{0.000000,0.000000,0.000000}%
\pgfsetstrokecolor{currentstroke}%
\pgfsetdash{}{0pt}%
\pgfpathmoveto{\pgfqpoint{0.693757in}{4.576962in}}%
\pgfpathlineto{\pgfqpoint{6.235000in}{4.576962in}}%
\pgfusepath{stroke}%
\end{pgfscope}%
\end{pgfpicture}%
\makeatother%
\endgroup%
}
        \caption{Forward data reuse}
        \label{fig:recg_forward_reuse}
    \end{subfigure}
    \begin{subfigure}{0.4\textwidth}
        \resizebox{\textwidth}{!}{%% Creator: Matplotlib, PGF backend
%%
%% To include the figure in your LaTeX document, write
%%   \input{<filename>.pgf}
%%
%% Make sure the required packages are loaded in your preamble
%%   \usepackage{pgf}
%%
%% Also ensure that all the required font packages are loaded; for instance,
%% the lmodern package is sometimes necessary when using math font.
%%   \usepackage{lmodern}
%%
%% Figures using additional raster images can only be included by \input if
%% they are in the same directory as the main LaTeX file. For loading figures
%% from other directories you can use the `import` package
%%   \usepackage{import}
%%
%% and then include the figures with
%%   \import{<path to file>}{<filename>.pgf}
%%
%% Matplotlib used the following preamble
%%   \def\mathdefault#1{#1}
%%   \everymath=\expandafter{\the\everymath\displaystyle}
%%   
%%   \usepackage{fontspec}
%%   \setmainfont{DejaVuSerif.ttf}[Path=\detokenize{/usr/lib/python3.11/site-packages/matplotlib/mpl-data/fonts/ttf/}]
%%   \setsansfont{DejaVuSans.ttf}[Path=\detokenize{/usr/lib/python3.11/site-packages/matplotlib/mpl-data/fonts/ttf/}]
%%   \setmonofont{DejaVuSansMono.ttf}[Path=\detokenize{/usr/lib/python3.11/site-packages/matplotlib/mpl-data/fonts/ttf/}]
%%   \makeatletter\@ifpackageloaded{underscore}{}{\usepackage[strings]{underscore}}\makeatother
%%
\begingroup%
\makeatletter%
\begin{pgfpicture}%
\pgfpathrectangle{\pgfpointorigin}{\pgfqpoint{6.400000in}{4.800000in}}%
\pgfusepath{use as bounding box, clip}%
\begin{pgfscope}%
\pgfsetbuttcap%
\pgfsetmiterjoin%
\definecolor{currentfill}{rgb}{1.000000,1.000000,1.000000}%
\pgfsetfillcolor{currentfill}%
\pgfsetlinewidth{0.000000pt}%
\definecolor{currentstroke}{rgb}{1.000000,1.000000,1.000000}%
\pgfsetstrokecolor{currentstroke}%
\pgfsetdash{}{0pt}%
\pgfpathmoveto{\pgfqpoint{0.000000in}{0.000000in}}%
\pgfpathlineto{\pgfqpoint{6.400000in}{0.000000in}}%
\pgfpathlineto{\pgfqpoint{6.400000in}{4.800000in}}%
\pgfpathlineto{\pgfqpoint{0.000000in}{4.800000in}}%
\pgfpathlineto{\pgfqpoint{0.000000in}{0.000000in}}%
\pgfpathclose%
\pgfusepath{fill}%
\end{pgfscope}%
\begin{pgfscope}%
\pgfsetbuttcap%
\pgfsetmiterjoin%
\definecolor{currentfill}{rgb}{1.000000,1.000000,1.000000}%
\pgfsetfillcolor{currentfill}%
\pgfsetlinewidth{0.000000pt}%
\definecolor{currentstroke}{rgb}{0.000000,0.000000,0.000000}%
\pgfsetstrokecolor{currentstroke}%
\pgfsetstrokeopacity{0.000000}%
\pgfsetdash{}{0pt}%
\pgfpathmoveto{\pgfqpoint{0.693757in}{0.613486in}}%
\pgfpathlineto{\pgfqpoint{6.235000in}{0.613486in}}%
\pgfpathlineto{\pgfqpoint{6.235000in}{4.576962in}}%
\pgfpathlineto{\pgfqpoint{0.693757in}{4.576962in}}%
\pgfpathlineto{\pgfqpoint{0.693757in}{0.613486in}}%
\pgfpathclose%
\pgfusepath{fill}%
\end{pgfscope}%
\begin{pgfscope}%
\pgfpathrectangle{\pgfqpoint{0.693757in}{0.613486in}}{\pgfqpoint{5.541243in}{3.963477in}}%
\pgfusepath{clip}%
\pgfsetbuttcap%
\pgfsetmiterjoin%
\definecolor{currentfill}{rgb}{0.000000,0.000000,1.000000}%
\pgfsetfillcolor{currentfill}%
\pgfsetlinewidth{0.000000pt}%
\definecolor{currentstroke}{rgb}{0.000000,0.000000,0.000000}%
\pgfsetstrokecolor{currentstroke}%
\pgfsetstrokeopacity{0.000000}%
\pgfsetdash{}{0pt}%
\pgfpathmoveto{\pgfqpoint{0.945632in}{0.613486in}}%
\pgfpathlineto{\pgfqpoint{0.955637in}{0.613486in}}%
\pgfpathlineto{\pgfqpoint{0.955637in}{1.612282in}}%
\pgfpathlineto{\pgfqpoint{0.945632in}{1.612282in}}%
\pgfpathlineto{\pgfqpoint{0.945632in}{0.613486in}}%
\pgfpathclose%
\pgfusepath{fill}%
\end{pgfscope}%
\begin{pgfscope}%
\pgfpathrectangle{\pgfqpoint{0.693757in}{0.613486in}}{\pgfqpoint{5.541243in}{3.963477in}}%
\pgfusepath{clip}%
\pgfsetbuttcap%
\pgfsetmiterjoin%
\definecolor{currentfill}{rgb}{0.000000,0.000000,1.000000}%
\pgfsetfillcolor{currentfill}%
\pgfsetlinewidth{0.000000pt}%
\definecolor{currentstroke}{rgb}{0.000000,0.000000,0.000000}%
\pgfsetstrokecolor{currentstroke}%
\pgfsetstrokeopacity{0.000000}%
\pgfsetdash{}{0pt}%
\pgfpathmoveto{\pgfqpoint{0.958138in}{0.613486in}}%
\pgfpathlineto{\pgfqpoint{0.968143in}{0.613486in}}%
\pgfpathlineto{\pgfqpoint{0.968143in}{2.595224in}}%
\pgfpathlineto{\pgfqpoint{0.958138in}{2.595224in}}%
\pgfpathlineto{\pgfqpoint{0.958138in}{0.613486in}}%
\pgfpathclose%
\pgfusepath{fill}%
\end{pgfscope}%
\begin{pgfscope}%
\pgfpathrectangle{\pgfqpoint{0.693757in}{0.613486in}}{\pgfqpoint{5.541243in}{3.963477in}}%
\pgfusepath{clip}%
\pgfsetbuttcap%
\pgfsetmiterjoin%
\definecolor{currentfill}{rgb}{0.000000,0.000000,1.000000}%
\pgfsetfillcolor{currentfill}%
\pgfsetlinewidth{0.000000pt}%
\definecolor{currentstroke}{rgb}{0.000000,0.000000,0.000000}%
\pgfsetstrokecolor{currentstroke}%
\pgfsetstrokeopacity{0.000000}%
\pgfsetdash{}{0pt}%
\pgfpathmoveto{\pgfqpoint{0.970644in}{0.613486in}}%
\pgfpathlineto{\pgfqpoint{0.980649in}{0.613486in}}%
\pgfpathlineto{\pgfqpoint{0.980649in}{2.611074in}}%
\pgfpathlineto{\pgfqpoint{0.970644in}{2.611074in}}%
\pgfpathlineto{\pgfqpoint{0.970644in}{0.613486in}}%
\pgfpathclose%
\pgfusepath{fill}%
\end{pgfscope}%
\begin{pgfscope}%
\pgfpathrectangle{\pgfqpoint{0.693757in}{0.613486in}}{\pgfqpoint{5.541243in}{3.963477in}}%
\pgfusepath{clip}%
\pgfsetbuttcap%
\pgfsetmiterjoin%
\definecolor{currentfill}{rgb}{0.000000,0.000000,1.000000}%
\pgfsetfillcolor{currentfill}%
\pgfsetlinewidth{0.000000pt}%
\definecolor{currentstroke}{rgb}{0.000000,0.000000,0.000000}%
\pgfsetstrokecolor{currentstroke}%
\pgfsetstrokeopacity{0.000000}%
\pgfsetdash{}{0pt}%
\pgfpathmoveto{\pgfqpoint{0.983150in}{0.613486in}}%
\pgfpathlineto{\pgfqpoint{0.993155in}{0.613486in}}%
\pgfpathlineto{\pgfqpoint{0.993155in}{1.604355in}}%
\pgfpathlineto{\pgfqpoint{0.983150in}{1.604355in}}%
\pgfpathlineto{\pgfqpoint{0.983150in}{0.613486in}}%
\pgfpathclose%
\pgfusepath{fill}%
\end{pgfscope}%
\begin{pgfscope}%
\pgfpathrectangle{\pgfqpoint{0.693757in}{0.613486in}}{\pgfqpoint{5.541243in}{3.963477in}}%
\pgfusepath{clip}%
\pgfsetbuttcap%
\pgfsetmiterjoin%
\definecolor{currentfill}{rgb}{0.000000,0.000000,1.000000}%
\pgfsetfillcolor{currentfill}%
\pgfsetlinewidth{0.000000pt}%
\definecolor{currentstroke}{rgb}{0.000000,0.000000,0.000000}%
\pgfsetstrokecolor{currentstroke}%
\pgfsetstrokeopacity{0.000000}%
\pgfsetdash{}{0pt}%
\pgfpathmoveto{\pgfqpoint{0.995657in}{0.613486in}}%
\pgfpathlineto{\pgfqpoint{1.005661in}{0.613486in}}%
\pgfpathlineto{\pgfqpoint{1.005661in}{1.612282in}}%
\pgfpathlineto{\pgfqpoint{0.995657in}{1.612282in}}%
\pgfpathlineto{\pgfqpoint{0.995657in}{0.613486in}}%
\pgfpathclose%
\pgfusepath{fill}%
\end{pgfscope}%
\begin{pgfscope}%
\pgfpathrectangle{\pgfqpoint{0.693757in}{0.613486in}}{\pgfqpoint{5.541243in}{3.963477in}}%
\pgfusepath{clip}%
\pgfsetbuttcap%
\pgfsetmiterjoin%
\definecolor{currentfill}{rgb}{0.000000,0.000000,1.000000}%
\pgfsetfillcolor{currentfill}%
\pgfsetlinewidth{0.000000pt}%
\definecolor{currentstroke}{rgb}{0.000000,0.000000,0.000000}%
\pgfsetstrokecolor{currentstroke}%
\pgfsetstrokeopacity{0.000000}%
\pgfsetdash{}{0pt}%
\pgfpathmoveto{\pgfqpoint{1.008163in}{0.613486in}}%
\pgfpathlineto{\pgfqpoint{1.018168in}{0.613486in}}%
\pgfpathlineto{\pgfqpoint{1.018168in}{2.595224in}}%
\pgfpathlineto{\pgfqpoint{1.008163in}{2.595224in}}%
\pgfpathlineto{\pgfqpoint{1.008163in}{0.613486in}}%
\pgfpathclose%
\pgfusepath{fill}%
\end{pgfscope}%
\begin{pgfscope}%
\pgfpathrectangle{\pgfqpoint{0.693757in}{0.613486in}}{\pgfqpoint{5.541243in}{3.963477in}}%
\pgfusepath{clip}%
\pgfsetbuttcap%
\pgfsetmiterjoin%
\definecolor{currentfill}{rgb}{0.000000,0.000000,1.000000}%
\pgfsetfillcolor{currentfill}%
\pgfsetlinewidth{0.000000pt}%
\definecolor{currentstroke}{rgb}{0.000000,0.000000,0.000000}%
\pgfsetstrokecolor{currentstroke}%
\pgfsetstrokeopacity{0.000000}%
\pgfsetdash{}{0pt}%
\pgfpathmoveto{\pgfqpoint{1.020669in}{0.613486in}}%
\pgfpathlineto{\pgfqpoint{1.030674in}{0.613486in}}%
\pgfpathlineto{\pgfqpoint{1.030674in}{2.611074in}}%
\pgfpathlineto{\pgfqpoint{1.020669in}{2.611074in}}%
\pgfpathlineto{\pgfqpoint{1.020669in}{0.613486in}}%
\pgfpathclose%
\pgfusepath{fill}%
\end{pgfscope}%
\begin{pgfscope}%
\pgfpathrectangle{\pgfqpoint{0.693757in}{0.613486in}}{\pgfqpoint{5.541243in}{3.963477in}}%
\pgfusepath{clip}%
\pgfsetbuttcap%
\pgfsetmiterjoin%
\definecolor{currentfill}{rgb}{0.000000,0.000000,1.000000}%
\pgfsetfillcolor{currentfill}%
\pgfsetlinewidth{0.000000pt}%
\definecolor{currentstroke}{rgb}{0.000000,0.000000,0.000000}%
\pgfsetstrokecolor{currentstroke}%
\pgfsetstrokeopacity{0.000000}%
\pgfsetdash{}{0pt}%
\pgfpathmoveto{\pgfqpoint{1.033175in}{0.613486in}}%
\pgfpathlineto{\pgfqpoint{1.043180in}{0.613486in}}%
\pgfpathlineto{\pgfqpoint{1.043180in}{1.604355in}}%
\pgfpathlineto{\pgfqpoint{1.033175in}{1.604355in}}%
\pgfpathlineto{\pgfqpoint{1.033175in}{0.613486in}}%
\pgfpathclose%
\pgfusepath{fill}%
\end{pgfscope}%
\begin{pgfscope}%
\pgfpathrectangle{\pgfqpoint{0.693757in}{0.613486in}}{\pgfqpoint{5.541243in}{3.963477in}}%
\pgfusepath{clip}%
\pgfsetbuttcap%
\pgfsetmiterjoin%
\definecolor{currentfill}{rgb}{0.000000,0.000000,1.000000}%
\pgfsetfillcolor{currentfill}%
\pgfsetlinewidth{0.000000pt}%
\definecolor{currentstroke}{rgb}{0.000000,0.000000,0.000000}%
\pgfsetstrokecolor{currentstroke}%
\pgfsetstrokeopacity{0.000000}%
\pgfsetdash{}{0pt}%
\pgfpathmoveto{\pgfqpoint{1.045681in}{0.613486in}}%
\pgfpathlineto{\pgfqpoint{1.055686in}{0.613486in}}%
\pgfpathlineto{\pgfqpoint{1.055686in}{1.612282in}}%
\pgfpathlineto{\pgfqpoint{1.045681in}{1.612282in}}%
\pgfpathlineto{\pgfqpoint{1.045681in}{0.613486in}}%
\pgfpathclose%
\pgfusepath{fill}%
\end{pgfscope}%
\begin{pgfscope}%
\pgfpathrectangle{\pgfqpoint{0.693757in}{0.613486in}}{\pgfqpoint{5.541243in}{3.963477in}}%
\pgfusepath{clip}%
\pgfsetbuttcap%
\pgfsetmiterjoin%
\definecolor{currentfill}{rgb}{0.000000,0.000000,1.000000}%
\pgfsetfillcolor{currentfill}%
\pgfsetlinewidth{0.000000pt}%
\definecolor{currentstroke}{rgb}{0.000000,0.000000,0.000000}%
\pgfsetstrokecolor{currentstroke}%
\pgfsetstrokeopacity{0.000000}%
\pgfsetdash{}{0pt}%
\pgfpathmoveto{\pgfqpoint{1.058187in}{0.613486in}}%
\pgfpathlineto{\pgfqpoint{1.068192in}{0.613486in}}%
\pgfpathlineto{\pgfqpoint{1.068192in}{2.595224in}}%
\pgfpathlineto{\pgfqpoint{1.058187in}{2.595224in}}%
\pgfpathlineto{\pgfqpoint{1.058187in}{0.613486in}}%
\pgfpathclose%
\pgfusepath{fill}%
\end{pgfscope}%
\begin{pgfscope}%
\pgfpathrectangle{\pgfqpoint{0.693757in}{0.613486in}}{\pgfqpoint{5.541243in}{3.963477in}}%
\pgfusepath{clip}%
\pgfsetbuttcap%
\pgfsetmiterjoin%
\definecolor{currentfill}{rgb}{0.000000,0.000000,1.000000}%
\pgfsetfillcolor{currentfill}%
\pgfsetlinewidth{0.000000pt}%
\definecolor{currentstroke}{rgb}{0.000000,0.000000,0.000000}%
\pgfsetstrokecolor{currentstroke}%
\pgfsetstrokeopacity{0.000000}%
\pgfsetdash{}{0pt}%
\pgfpathmoveto{\pgfqpoint{1.070694in}{0.613486in}}%
\pgfpathlineto{\pgfqpoint{1.080699in}{0.613486in}}%
\pgfpathlineto{\pgfqpoint{1.080699in}{2.611074in}}%
\pgfpathlineto{\pgfqpoint{1.070694in}{2.611074in}}%
\pgfpathlineto{\pgfqpoint{1.070694in}{0.613486in}}%
\pgfpathclose%
\pgfusepath{fill}%
\end{pgfscope}%
\begin{pgfscope}%
\pgfpathrectangle{\pgfqpoint{0.693757in}{0.613486in}}{\pgfqpoint{5.541243in}{3.963477in}}%
\pgfusepath{clip}%
\pgfsetbuttcap%
\pgfsetmiterjoin%
\definecolor{currentfill}{rgb}{0.000000,0.000000,1.000000}%
\pgfsetfillcolor{currentfill}%
\pgfsetlinewidth{0.000000pt}%
\definecolor{currentstroke}{rgb}{0.000000,0.000000,0.000000}%
\pgfsetstrokecolor{currentstroke}%
\pgfsetstrokeopacity{0.000000}%
\pgfsetdash{}{0pt}%
\pgfpathmoveto{\pgfqpoint{1.083200in}{0.613486in}}%
\pgfpathlineto{\pgfqpoint{1.093205in}{0.613486in}}%
\pgfpathlineto{\pgfqpoint{1.093205in}{1.604355in}}%
\pgfpathlineto{\pgfqpoint{1.083200in}{1.604355in}}%
\pgfpathlineto{\pgfqpoint{1.083200in}{0.613486in}}%
\pgfpathclose%
\pgfusepath{fill}%
\end{pgfscope}%
\begin{pgfscope}%
\pgfpathrectangle{\pgfqpoint{0.693757in}{0.613486in}}{\pgfqpoint{5.541243in}{3.963477in}}%
\pgfusepath{clip}%
\pgfsetbuttcap%
\pgfsetmiterjoin%
\definecolor{currentfill}{rgb}{0.000000,0.000000,1.000000}%
\pgfsetfillcolor{currentfill}%
\pgfsetlinewidth{0.000000pt}%
\definecolor{currentstroke}{rgb}{0.000000,0.000000,0.000000}%
\pgfsetstrokecolor{currentstroke}%
\pgfsetstrokeopacity{0.000000}%
\pgfsetdash{}{0pt}%
\pgfpathmoveto{\pgfqpoint{1.095706in}{0.613486in}}%
\pgfpathlineto{\pgfqpoint{1.105711in}{0.613486in}}%
\pgfpathlineto{\pgfqpoint{1.105711in}{1.612282in}}%
\pgfpathlineto{\pgfqpoint{1.095706in}{1.612282in}}%
\pgfpathlineto{\pgfqpoint{1.095706in}{0.613486in}}%
\pgfpathclose%
\pgfusepath{fill}%
\end{pgfscope}%
\begin{pgfscope}%
\pgfpathrectangle{\pgfqpoint{0.693757in}{0.613486in}}{\pgfqpoint{5.541243in}{3.963477in}}%
\pgfusepath{clip}%
\pgfsetbuttcap%
\pgfsetmiterjoin%
\definecolor{currentfill}{rgb}{0.000000,0.000000,1.000000}%
\pgfsetfillcolor{currentfill}%
\pgfsetlinewidth{0.000000pt}%
\definecolor{currentstroke}{rgb}{0.000000,0.000000,0.000000}%
\pgfsetstrokecolor{currentstroke}%
\pgfsetstrokeopacity{0.000000}%
\pgfsetdash{}{0pt}%
\pgfpathmoveto{\pgfqpoint{1.108212in}{0.613486in}}%
\pgfpathlineto{\pgfqpoint{1.118217in}{0.613486in}}%
\pgfpathlineto{\pgfqpoint{1.118217in}{2.595224in}}%
\pgfpathlineto{\pgfqpoint{1.108212in}{2.595224in}}%
\pgfpathlineto{\pgfqpoint{1.108212in}{0.613486in}}%
\pgfpathclose%
\pgfusepath{fill}%
\end{pgfscope}%
\begin{pgfscope}%
\pgfpathrectangle{\pgfqpoint{0.693757in}{0.613486in}}{\pgfqpoint{5.541243in}{3.963477in}}%
\pgfusepath{clip}%
\pgfsetbuttcap%
\pgfsetmiterjoin%
\definecolor{currentfill}{rgb}{0.000000,0.000000,1.000000}%
\pgfsetfillcolor{currentfill}%
\pgfsetlinewidth{0.000000pt}%
\definecolor{currentstroke}{rgb}{0.000000,0.000000,0.000000}%
\pgfsetstrokecolor{currentstroke}%
\pgfsetstrokeopacity{0.000000}%
\pgfsetdash{}{0pt}%
\pgfpathmoveto{\pgfqpoint{1.120718in}{0.613486in}}%
\pgfpathlineto{\pgfqpoint{1.130723in}{0.613486in}}%
\pgfpathlineto{\pgfqpoint{1.130723in}{2.611074in}}%
\pgfpathlineto{\pgfqpoint{1.120718in}{2.611074in}}%
\pgfpathlineto{\pgfqpoint{1.120718in}{0.613486in}}%
\pgfpathclose%
\pgfusepath{fill}%
\end{pgfscope}%
\begin{pgfscope}%
\pgfpathrectangle{\pgfqpoint{0.693757in}{0.613486in}}{\pgfqpoint{5.541243in}{3.963477in}}%
\pgfusepath{clip}%
\pgfsetbuttcap%
\pgfsetmiterjoin%
\definecolor{currentfill}{rgb}{0.000000,0.000000,1.000000}%
\pgfsetfillcolor{currentfill}%
\pgfsetlinewidth{0.000000pt}%
\definecolor{currentstroke}{rgb}{0.000000,0.000000,0.000000}%
\pgfsetstrokecolor{currentstroke}%
\pgfsetstrokeopacity{0.000000}%
\pgfsetdash{}{0pt}%
\pgfpathmoveto{\pgfqpoint{1.133225in}{0.613486in}}%
\pgfpathlineto{\pgfqpoint{1.143230in}{0.613486in}}%
\pgfpathlineto{\pgfqpoint{1.143230in}{1.604355in}}%
\pgfpathlineto{\pgfqpoint{1.133225in}{1.604355in}}%
\pgfpathlineto{\pgfqpoint{1.133225in}{0.613486in}}%
\pgfpathclose%
\pgfusepath{fill}%
\end{pgfscope}%
\begin{pgfscope}%
\pgfpathrectangle{\pgfqpoint{0.693757in}{0.613486in}}{\pgfqpoint{5.541243in}{3.963477in}}%
\pgfusepath{clip}%
\pgfsetbuttcap%
\pgfsetmiterjoin%
\definecolor{currentfill}{rgb}{0.000000,0.000000,1.000000}%
\pgfsetfillcolor{currentfill}%
\pgfsetlinewidth{0.000000pt}%
\definecolor{currentstroke}{rgb}{0.000000,0.000000,0.000000}%
\pgfsetstrokecolor{currentstroke}%
\pgfsetstrokeopacity{0.000000}%
\pgfsetdash{}{0pt}%
\pgfpathmoveto{\pgfqpoint{1.145731in}{0.613486in}}%
\pgfpathlineto{\pgfqpoint{1.155736in}{0.613486in}}%
\pgfpathlineto{\pgfqpoint{1.155736in}{1.612282in}}%
\pgfpathlineto{\pgfqpoint{1.145731in}{1.612282in}}%
\pgfpathlineto{\pgfqpoint{1.145731in}{0.613486in}}%
\pgfpathclose%
\pgfusepath{fill}%
\end{pgfscope}%
\begin{pgfscope}%
\pgfpathrectangle{\pgfqpoint{0.693757in}{0.613486in}}{\pgfqpoint{5.541243in}{3.963477in}}%
\pgfusepath{clip}%
\pgfsetbuttcap%
\pgfsetmiterjoin%
\definecolor{currentfill}{rgb}{0.000000,0.000000,1.000000}%
\pgfsetfillcolor{currentfill}%
\pgfsetlinewidth{0.000000pt}%
\definecolor{currentstroke}{rgb}{0.000000,0.000000,0.000000}%
\pgfsetstrokecolor{currentstroke}%
\pgfsetstrokeopacity{0.000000}%
\pgfsetdash{}{0pt}%
\pgfpathmoveto{\pgfqpoint{1.158237in}{0.613486in}}%
\pgfpathlineto{\pgfqpoint{1.168242in}{0.613486in}}%
\pgfpathlineto{\pgfqpoint{1.168242in}{2.595224in}}%
\pgfpathlineto{\pgfqpoint{1.158237in}{2.595224in}}%
\pgfpathlineto{\pgfqpoint{1.158237in}{0.613486in}}%
\pgfpathclose%
\pgfusepath{fill}%
\end{pgfscope}%
\begin{pgfscope}%
\pgfpathrectangle{\pgfqpoint{0.693757in}{0.613486in}}{\pgfqpoint{5.541243in}{3.963477in}}%
\pgfusepath{clip}%
\pgfsetbuttcap%
\pgfsetmiterjoin%
\definecolor{currentfill}{rgb}{0.000000,0.000000,1.000000}%
\pgfsetfillcolor{currentfill}%
\pgfsetlinewidth{0.000000pt}%
\definecolor{currentstroke}{rgb}{0.000000,0.000000,0.000000}%
\pgfsetstrokecolor{currentstroke}%
\pgfsetstrokeopacity{0.000000}%
\pgfsetdash{}{0pt}%
\pgfpathmoveto{\pgfqpoint{1.170743in}{0.613486in}}%
\pgfpathlineto{\pgfqpoint{1.180748in}{0.613486in}}%
\pgfpathlineto{\pgfqpoint{1.180748in}{2.611074in}}%
\pgfpathlineto{\pgfqpoint{1.170743in}{2.611074in}}%
\pgfpathlineto{\pgfqpoint{1.170743in}{0.613486in}}%
\pgfpathclose%
\pgfusepath{fill}%
\end{pgfscope}%
\begin{pgfscope}%
\pgfpathrectangle{\pgfqpoint{0.693757in}{0.613486in}}{\pgfqpoint{5.541243in}{3.963477in}}%
\pgfusepath{clip}%
\pgfsetbuttcap%
\pgfsetmiterjoin%
\definecolor{currentfill}{rgb}{0.000000,0.000000,1.000000}%
\pgfsetfillcolor{currentfill}%
\pgfsetlinewidth{0.000000pt}%
\definecolor{currentstroke}{rgb}{0.000000,0.000000,0.000000}%
\pgfsetstrokecolor{currentstroke}%
\pgfsetstrokeopacity{0.000000}%
\pgfsetdash{}{0pt}%
\pgfpathmoveto{\pgfqpoint{1.183249in}{0.613486in}}%
\pgfpathlineto{\pgfqpoint{1.193254in}{0.613486in}}%
\pgfpathlineto{\pgfqpoint{1.193254in}{1.604355in}}%
\pgfpathlineto{\pgfqpoint{1.183249in}{1.604355in}}%
\pgfpathlineto{\pgfqpoint{1.183249in}{0.613486in}}%
\pgfpathclose%
\pgfusepath{fill}%
\end{pgfscope}%
\begin{pgfscope}%
\pgfpathrectangle{\pgfqpoint{0.693757in}{0.613486in}}{\pgfqpoint{5.541243in}{3.963477in}}%
\pgfusepath{clip}%
\pgfsetbuttcap%
\pgfsetmiterjoin%
\definecolor{currentfill}{rgb}{0.000000,0.000000,1.000000}%
\pgfsetfillcolor{currentfill}%
\pgfsetlinewidth{0.000000pt}%
\definecolor{currentstroke}{rgb}{0.000000,0.000000,0.000000}%
\pgfsetstrokecolor{currentstroke}%
\pgfsetstrokeopacity{0.000000}%
\pgfsetdash{}{0pt}%
\pgfpathmoveto{\pgfqpoint{1.195756in}{0.613486in}}%
\pgfpathlineto{\pgfqpoint{1.205761in}{0.613486in}}%
\pgfpathlineto{\pgfqpoint{1.205761in}{1.612282in}}%
\pgfpathlineto{\pgfqpoint{1.195756in}{1.612282in}}%
\pgfpathlineto{\pgfqpoint{1.195756in}{0.613486in}}%
\pgfpathclose%
\pgfusepath{fill}%
\end{pgfscope}%
\begin{pgfscope}%
\pgfpathrectangle{\pgfqpoint{0.693757in}{0.613486in}}{\pgfqpoint{5.541243in}{3.963477in}}%
\pgfusepath{clip}%
\pgfsetbuttcap%
\pgfsetmiterjoin%
\definecolor{currentfill}{rgb}{0.000000,0.000000,1.000000}%
\pgfsetfillcolor{currentfill}%
\pgfsetlinewidth{0.000000pt}%
\definecolor{currentstroke}{rgb}{0.000000,0.000000,0.000000}%
\pgfsetstrokecolor{currentstroke}%
\pgfsetstrokeopacity{0.000000}%
\pgfsetdash{}{0pt}%
\pgfpathmoveto{\pgfqpoint{1.208262in}{0.613486in}}%
\pgfpathlineto{\pgfqpoint{1.218267in}{0.613486in}}%
\pgfpathlineto{\pgfqpoint{1.218267in}{2.595224in}}%
\pgfpathlineto{\pgfqpoint{1.208262in}{2.595224in}}%
\pgfpathlineto{\pgfqpoint{1.208262in}{0.613486in}}%
\pgfpathclose%
\pgfusepath{fill}%
\end{pgfscope}%
\begin{pgfscope}%
\pgfpathrectangle{\pgfqpoint{0.693757in}{0.613486in}}{\pgfqpoint{5.541243in}{3.963477in}}%
\pgfusepath{clip}%
\pgfsetbuttcap%
\pgfsetmiterjoin%
\definecolor{currentfill}{rgb}{0.000000,0.000000,1.000000}%
\pgfsetfillcolor{currentfill}%
\pgfsetlinewidth{0.000000pt}%
\definecolor{currentstroke}{rgb}{0.000000,0.000000,0.000000}%
\pgfsetstrokecolor{currentstroke}%
\pgfsetstrokeopacity{0.000000}%
\pgfsetdash{}{0pt}%
\pgfpathmoveto{\pgfqpoint{1.220768in}{0.613486in}}%
\pgfpathlineto{\pgfqpoint{1.230773in}{0.613486in}}%
\pgfpathlineto{\pgfqpoint{1.230773in}{2.611074in}}%
\pgfpathlineto{\pgfqpoint{1.220768in}{2.611074in}}%
\pgfpathlineto{\pgfqpoint{1.220768in}{0.613486in}}%
\pgfpathclose%
\pgfusepath{fill}%
\end{pgfscope}%
\begin{pgfscope}%
\pgfpathrectangle{\pgfqpoint{0.693757in}{0.613486in}}{\pgfqpoint{5.541243in}{3.963477in}}%
\pgfusepath{clip}%
\pgfsetbuttcap%
\pgfsetmiterjoin%
\definecolor{currentfill}{rgb}{0.000000,0.000000,1.000000}%
\pgfsetfillcolor{currentfill}%
\pgfsetlinewidth{0.000000pt}%
\definecolor{currentstroke}{rgb}{0.000000,0.000000,0.000000}%
\pgfsetstrokecolor{currentstroke}%
\pgfsetstrokeopacity{0.000000}%
\pgfsetdash{}{0pt}%
\pgfpathmoveto{\pgfqpoint{1.233274in}{0.613486in}}%
\pgfpathlineto{\pgfqpoint{1.243279in}{0.613486in}}%
\pgfpathlineto{\pgfqpoint{1.243279in}{1.604355in}}%
\pgfpathlineto{\pgfqpoint{1.233274in}{1.604355in}}%
\pgfpathlineto{\pgfqpoint{1.233274in}{0.613486in}}%
\pgfpathclose%
\pgfusepath{fill}%
\end{pgfscope}%
\begin{pgfscope}%
\pgfpathrectangle{\pgfqpoint{0.693757in}{0.613486in}}{\pgfqpoint{5.541243in}{3.963477in}}%
\pgfusepath{clip}%
\pgfsetbuttcap%
\pgfsetmiterjoin%
\definecolor{currentfill}{rgb}{0.000000,0.000000,1.000000}%
\pgfsetfillcolor{currentfill}%
\pgfsetlinewidth{0.000000pt}%
\definecolor{currentstroke}{rgb}{0.000000,0.000000,0.000000}%
\pgfsetstrokecolor{currentstroke}%
\pgfsetstrokeopacity{0.000000}%
\pgfsetdash{}{0pt}%
\pgfpathmoveto{\pgfqpoint{1.245780in}{0.613486in}}%
\pgfpathlineto{\pgfqpoint{1.255785in}{0.613486in}}%
\pgfpathlineto{\pgfqpoint{1.255785in}{1.612282in}}%
\pgfpathlineto{\pgfqpoint{1.245780in}{1.612282in}}%
\pgfpathlineto{\pgfqpoint{1.245780in}{0.613486in}}%
\pgfpathclose%
\pgfusepath{fill}%
\end{pgfscope}%
\begin{pgfscope}%
\pgfpathrectangle{\pgfqpoint{0.693757in}{0.613486in}}{\pgfqpoint{5.541243in}{3.963477in}}%
\pgfusepath{clip}%
\pgfsetbuttcap%
\pgfsetmiterjoin%
\definecolor{currentfill}{rgb}{0.000000,0.000000,1.000000}%
\pgfsetfillcolor{currentfill}%
\pgfsetlinewidth{0.000000pt}%
\definecolor{currentstroke}{rgb}{0.000000,0.000000,0.000000}%
\pgfsetstrokecolor{currentstroke}%
\pgfsetstrokeopacity{0.000000}%
\pgfsetdash{}{0pt}%
\pgfpathmoveto{\pgfqpoint{1.258287in}{0.613486in}}%
\pgfpathlineto{\pgfqpoint{1.268291in}{0.613486in}}%
\pgfpathlineto{\pgfqpoint{1.268291in}{2.595224in}}%
\pgfpathlineto{\pgfqpoint{1.258287in}{2.595224in}}%
\pgfpathlineto{\pgfqpoint{1.258287in}{0.613486in}}%
\pgfpathclose%
\pgfusepath{fill}%
\end{pgfscope}%
\begin{pgfscope}%
\pgfpathrectangle{\pgfqpoint{0.693757in}{0.613486in}}{\pgfqpoint{5.541243in}{3.963477in}}%
\pgfusepath{clip}%
\pgfsetbuttcap%
\pgfsetmiterjoin%
\definecolor{currentfill}{rgb}{0.000000,0.000000,1.000000}%
\pgfsetfillcolor{currentfill}%
\pgfsetlinewidth{0.000000pt}%
\definecolor{currentstroke}{rgb}{0.000000,0.000000,0.000000}%
\pgfsetstrokecolor{currentstroke}%
\pgfsetstrokeopacity{0.000000}%
\pgfsetdash{}{0pt}%
\pgfpathmoveto{\pgfqpoint{1.270793in}{0.613486in}}%
\pgfpathlineto{\pgfqpoint{1.280798in}{0.613486in}}%
\pgfpathlineto{\pgfqpoint{1.280798in}{2.611074in}}%
\pgfpathlineto{\pgfqpoint{1.270793in}{2.611074in}}%
\pgfpathlineto{\pgfqpoint{1.270793in}{0.613486in}}%
\pgfpathclose%
\pgfusepath{fill}%
\end{pgfscope}%
\begin{pgfscope}%
\pgfpathrectangle{\pgfqpoint{0.693757in}{0.613486in}}{\pgfqpoint{5.541243in}{3.963477in}}%
\pgfusepath{clip}%
\pgfsetbuttcap%
\pgfsetmiterjoin%
\definecolor{currentfill}{rgb}{0.000000,0.000000,1.000000}%
\pgfsetfillcolor{currentfill}%
\pgfsetlinewidth{0.000000pt}%
\definecolor{currentstroke}{rgb}{0.000000,0.000000,0.000000}%
\pgfsetstrokecolor{currentstroke}%
\pgfsetstrokeopacity{0.000000}%
\pgfsetdash{}{0pt}%
\pgfpathmoveto{\pgfqpoint{1.283299in}{0.613486in}}%
\pgfpathlineto{\pgfqpoint{1.293304in}{0.613486in}}%
\pgfpathlineto{\pgfqpoint{1.293304in}{1.604355in}}%
\pgfpathlineto{\pgfqpoint{1.283299in}{1.604355in}}%
\pgfpathlineto{\pgfqpoint{1.283299in}{0.613486in}}%
\pgfpathclose%
\pgfusepath{fill}%
\end{pgfscope}%
\begin{pgfscope}%
\pgfpathrectangle{\pgfqpoint{0.693757in}{0.613486in}}{\pgfqpoint{5.541243in}{3.963477in}}%
\pgfusepath{clip}%
\pgfsetbuttcap%
\pgfsetmiterjoin%
\definecolor{currentfill}{rgb}{0.000000,0.000000,1.000000}%
\pgfsetfillcolor{currentfill}%
\pgfsetlinewidth{0.000000pt}%
\definecolor{currentstroke}{rgb}{0.000000,0.000000,0.000000}%
\pgfsetstrokecolor{currentstroke}%
\pgfsetstrokeopacity{0.000000}%
\pgfsetdash{}{0pt}%
\pgfpathmoveto{\pgfqpoint{1.295805in}{0.613486in}}%
\pgfpathlineto{\pgfqpoint{1.305810in}{0.613486in}}%
\pgfpathlineto{\pgfqpoint{1.305810in}{1.612282in}}%
\pgfpathlineto{\pgfqpoint{1.295805in}{1.612282in}}%
\pgfpathlineto{\pgfqpoint{1.295805in}{0.613486in}}%
\pgfpathclose%
\pgfusepath{fill}%
\end{pgfscope}%
\begin{pgfscope}%
\pgfpathrectangle{\pgfqpoint{0.693757in}{0.613486in}}{\pgfqpoint{5.541243in}{3.963477in}}%
\pgfusepath{clip}%
\pgfsetbuttcap%
\pgfsetmiterjoin%
\definecolor{currentfill}{rgb}{0.000000,0.000000,1.000000}%
\pgfsetfillcolor{currentfill}%
\pgfsetlinewidth{0.000000pt}%
\definecolor{currentstroke}{rgb}{0.000000,0.000000,0.000000}%
\pgfsetstrokecolor{currentstroke}%
\pgfsetstrokeopacity{0.000000}%
\pgfsetdash{}{0pt}%
\pgfpathmoveto{\pgfqpoint{1.308311in}{0.613486in}}%
\pgfpathlineto{\pgfqpoint{1.318316in}{0.613486in}}%
\pgfpathlineto{\pgfqpoint{1.318316in}{2.595224in}}%
\pgfpathlineto{\pgfqpoint{1.308311in}{2.595224in}}%
\pgfpathlineto{\pgfqpoint{1.308311in}{0.613486in}}%
\pgfpathclose%
\pgfusepath{fill}%
\end{pgfscope}%
\begin{pgfscope}%
\pgfpathrectangle{\pgfqpoint{0.693757in}{0.613486in}}{\pgfqpoint{5.541243in}{3.963477in}}%
\pgfusepath{clip}%
\pgfsetbuttcap%
\pgfsetmiterjoin%
\definecolor{currentfill}{rgb}{0.000000,0.000000,1.000000}%
\pgfsetfillcolor{currentfill}%
\pgfsetlinewidth{0.000000pt}%
\definecolor{currentstroke}{rgb}{0.000000,0.000000,0.000000}%
\pgfsetstrokecolor{currentstroke}%
\pgfsetstrokeopacity{0.000000}%
\pgfsetdash{}{0pt}%
\pgfpathmoveto{\pgfqpoint{1.320817in}{0.613486in}}%
\pgfpathlineto{\pgfqpoint{1.330822in}{0.613486in}}%
\pgfpathlineto{\pgfqpoint{1.330822in}{2.611074in}}%
\pgfpathlineto{\pgfqpoint{1.320817in}{2.611074in}}%
\pgfpathlineto{\pgfqpoint{1.320817in}{0.613486in}}%
\pgfpathclose%
\pgfusepath{fill}%
\end{pgfscope}%
\begin{pgfscope}%
\pgfpathrectangle{\pgfqpoint{0.693757in}{0.613486in}}{\pgfqpoint{5.541243in}{3.963477in}}%
\pgfusepath{clip}%
\pgfsetbuttcap%
\pgfsetmiterjoin%
\definecolor{currentfill}{rgb}{0.000000,0.000000,1.000000}%
\pgfsetfillcolor{currentfill}%
\pgfsetlinewidth{0.000000pt}%
\definecolor{currentstroke}{rgb}{0.000000,0.000000,0.000000}%
\pgfsetstrokecolor{currentstroke}%
\pgfsetstrokeopacity{0.000000}%
\pgfsetdash{}{0pt}%
\pgfpathmoveto{\pgfqpoint{1.333324in}{0.613486in}}%
\pgfpathlineto{\pgfqpoint{1.343329in}{0.613486in}}%
\pgfpathlineto{\pgfqpoint{1.343329in}{1.604355in}}%
\pgfpathlineto{\pgfqpoint{1.333324in}{1.604355in}}%
\pgfpathlineto{\pgfqpoint{1.333324in}{0.613486in}}%
\pgfpathclose%
\pgfusepath{fill}%
\end{pgfscope}%
\begin{pgfscope}%
\pgfpathrectangle{\pgfqpoint{0.693757in}{0.613486in}}{\pgfqpoint{5.541243in}{3.963477in}}%
\pgfusepath{clip}%
\pgfsetbuttcap%
\pgfsetmiterjoin%
\definecolor{currentfill}{rgb}{0.000000,0.000000,1.000000}%
\pgfsetfillcolor{currentfill}%
\pgfsetlinewidth{0.000000pt}%
\definecolor{currentstroke}{rgb}{0.000000,0.000000,0.000000}%
\pgfsetstrokecolor{currentstroke}%
\pgfsetstrokeopacity{0.000000}%
\pgfsetdash{}{0pt}%
\pgfpathmoveto{\pgfqpoint{1.345830in}{0.613486in}}%
\pgfpathlineto{\pgfqpoint{1.355835in}{0.613486in}}%
\pgfpathlineto{\pgfqpoint{1.355835in}{1.612282in}}%
\pgfpathlineto{\pgfqpoint{1.345830in}{1.612282in}}%
\pgfpathlineto{\pgfqpoint{1.345830in}{0.613486in}}%
\pgfpathclose%
\pgfusepath{fill}%
\end{pgfscope}%
\begin{pgfscope}%
\pgfpathrectangle{\pgfqpoint{0.693757in}{0.613486in}}{\pgfqpoint{5.541243in}{3.963477in}}%
\pgfusepath{clip}%
\pgfsetbuttcap%
\pgfsetmiterjoin%
\definecolor{currentfill}{rgb}{0.000000,0.000000,1.000000}%
\pgfsetfillcolor{currentfill}%
\pgfsetlinewidth{0.000000pt}%
\definecolor{currentstroke}{rgb}{0.000000,0.000000,0.000000}%
\pgfsetstrokecolor{currentstroke}%
\pgfsetstrokeopacity{0.000000}%
\pgfsetdash{}{0pt}%
\pgfpathmoveto{\pgfqpoint{1.358336in}{0.613486in}}%
\pgfpathlineto{\pgfqpoint{1.368341in}{0.613486in}}%
\pgfpathlineto{\pgfqpoint{1.368341in}{2.595224in}}%
\pgfpathlineto{\pgfqpoint{1.358336in}{2.595224in}}%
\pgfpathlineto{\pgfqpoint{1.358336in}{0.613486in}}%
\pgfpathclose%
\pgfusepath{fill}%
\end{pgfscope}%
\begin{pgfscope}%
\pgfpathrectangle{\pgfqpoint{0.693757in}{0.613486in}}{\pgfqpoint{5.541243in}{3.963477in}}%
\pgfusepath{clip}%
\pgfsetbuttcap%
\pgfsetmiterjoin%
\definecolor{currentfill}{rgb}{0.000000,0.000000,1.000000}%
\pgfsetfillcolor{currentfill}%
\pgfsetlinewidth{0.000000pt}%
\definecolor{currentstroke}{rgb}{0.000000,0.000000,0.000000}%
\pgfsetstrokecolor{currentstroke}%
\pgfsetstrokeopacity{0.000000}%
\pgfsetdash{}{0pt}%
\pgfpathmoveto{\pgfqpoint{1.370842in}{0.613486in}}%
\pgfpathlineto{\pgfqpoint{1.380847in}{0.613486in}}%
\pgfpathlineto{\pgfqpoint{1.380847in}{2.611074in}}%
\pgfpathlineto{\pgfqpoint{1.370842in}{2.611074in}}%
\pgfpathlineto{\pgfqpoint{1.370842in}{0.613486in}}%
\pgfpathclose%
\pgfusepath{fill}%
\end{pgfscope}%
\begin{pgfscope}%
\pgfpathrectangle{\pgfqpoint{0.693757in}{0.613486in}}{\pgfqpoint{5.541243in}{3.963477in}}%
\pgfusepath{clip}%
\pgfsetbuttcap%
\pgfsetmiterjoin%
\definecolor{currentfill}{rgb}{0.000000,0.000000,1.000000}%
\pgfsetfillcolor{currentfill}%
\pgfsetlinewidth{0.000000pt}%
\definecolor{currentstroke}{rgb}{0.000000,0.000000,0.000000}%
\pgfsetstrokecolor{currentstroke}%
\pgfsetstrokeopacity{0.000000}%
\pgfsetdash{}{0pt}%
\pgfpathmoveto{\pgfqpoint{1.383348in}{0.613486in}}%
\pgfpathlineto{\pgfqpoint{1.393353in}{0.613486in}}%
\pgfpathlineto{\pgfqpoint{1.393353in}{1.604355in}}%
\pgfpathlineto{\pgfqpoint{1.383348in}{1.604355in}}%
\pgfpathlineto{\pgfqpoint{1.383348in}{0.613486in}}%
\pgfpathclose%
\pgfusepath{fill}%
\end{pgfscope}%
\begin{pgfscope}%
\pgfpathrectangle{\pgfqpoint{0.693757in}{0.613486in}}{\pgfqpoint{5.541243in}{3.963477in}}%
\pgfusepath{clip}%
\pgfsetbuttcap%
\pgfsetmiterjoin%
\definecolor{currentfill}{rgb}{0.000000,0.000000,1.000000}%
\pgfsetfillcolor{currentfill}%
\pgfsetlinewidth{0.000000pt}%
\definecolor{currentstroke}{rgb}{0.000000,0.000000,0.000000}%
\pgfsetstrokecolor{currentstroke}%
\pgfsetstrokeopacity{0.000000}%
\pgfsetdash{}{0pt}%
\pgfpathmoveto{\pgfqpoint{1.395855in}{0.613486in}}%
\pgfpathlineto{\pgfqpoint{1.405860in}{0.613486in}}%
\pgfpathlineto{\pgfqpoint{1.405860in}{1.612282in}}%
\pgfpathlineto{\pgfqpoint{1.395855in}{1.612282in}}%
\pgfpathlineto{\pgfqpoint{1.395855in}{0.613486in}}%
\pgfpathclose%
\pgfusepath{fill}%
\end{pgfscope}%
\begin{pgfscope}%
\pgfpathrectangle{\pgfqpoint{0.693757in}{0.613486in}}{\pgfqpoint{5.541243in}{3.963477in}}%
\pgfusepath{clip}%
\pgfsetbuttcap%
\pgfsetmiterjoin%
\definecolor{currentfill}{rgb}{0.000000,0.000000,1.000000}%
\pgfsetfillcolor{currentfill}%
\pgfsetlinewidth{0.000000pt}%
\definecolor{currentstroke}{rgb}{0.000000,0.000000,0.000000}%
\pgfsetstrokecolor{currentstroke}%
\pgfsetstrokeopacity{0.000000}%
\pgfsetdash{}{0pt}%
\pgfpathmoveto{\pgfqpoint{1.408361in}{0.613486in}}%
\pgfpathlineto{\pgfqpoint{1.418366in}{0.613486in}}%
\pgfpathlineto{\pgfqpoint{1.418366in}{2.595224in}}%
\pgfpathlineto{\pgfqpoint{1.408361in}{2.595224in}}%
\pgfpathlineto{\pgfqpoint{1.408361in}{0.613486in}}%
\pgfpathclose%
\pgfusepath{fill}%
\end{pgfscope}%
\begin{pgfscope}%
\pgfpathrectangle{\pgfqpoint{0.693757in}{0.613486in}}{\pgfqpoint{5.541243in}{3.963477in}}%
\pgfusepath{clip}%
\pgfsetbuttcap%
\pgfsetmiterjoin%
\definecolor{currentfill}{rgb}{0.000000,0.000000,1.000000}%
\pgfsetfillcolor{currentfill}%
\pgfsetlinewidth{0.000000pt}%
\definecolor{currentstroke}{rgb}{0.000000,0.000000,0.000000}%
\pgfsetstrokecolor{currentstroke}%
\pgfsetstrokeopacity{0.000000}%
\pgfsetdash{}{0pt}%
\pgfpathmoveto{\pgfqpoint{1.420867in}{0.613486in}}%
\pgfpathlineto{\pgfqpoint{1.430872in}{0.613486in}}%
\pgfpathlineto{\pgfqpoint{1.430872in}{2.611074in}}%
\pgfpathlineto{\pgfqpoint{1.420867in}{2.611074in}}%
\pgfpathlineto{\pgfqpoint{1.420867in}{0.613486in}}%
\pgfpathclose%
\pgfusepath{fill}%
\end{pgfscope}%
\begin{pgfscope}%
\pgfpathrectangle{\pgfqpoint{0.693757in}{0.613486in}}{\pgfqpoint{5.541243in}{3.963477in}}%
\pgfusepath{clip}%
\pgfsetbuttcap%
\pgfsetmiterjoin%
\definecolor{currentfill}{rgb}{0.000000,0.000000,1.000000}%
\pgfsetfillcolor{currentfill}%
\pgfsetlinewidth{0.000000pt}%
\definecolor{currentstroke}{rgb}{0.000000,0.000000,0.000000}%
\pgfsetstrokecolor{currentstroke}%
\pgfsetstrokeopacity{0.000000}%
\pgfsetdash{}{0pt}%
\pgfpathmoveto{\pgfqpoint{1.433373in}{0.613486in}}%
\pgfpathlineto{\pgfqpoint{1.443378in}{0.613486in}}%
\pgfpathlineto{\pgfqpoint{1.443378in}{1.604355in}}%
\pgfpathlineto{\pgfqpoint{1.433373in}{1.604355in}}%
\pgfpathlineto{\pgfqpoint{1.433373in}{0.613486in}}%
\pgfpathclose%
\pgfusepath{fill}%
\end{pgfscope}%
\begin{pgfscope}%
\pgfpathrectangle{\pgfqpoint{0.693757in}{0.613486in}}{\pgfqpoint{5.541243in}{3.963477in}}%
\pgfusepath{clip}%
\pgfsetbuttcap%
\pgfsetmiterjoin%
\definecolor{currentfill}{rgb}{0.000000,0.000000,1.000000}%
\pgfsetfillcolor{currentfill}%
\pgfsetlinewidth{0.000000pt}%
\definecolor{currentstroke}{rgb}{0.000000,0.000000,0.000000}%
\pgfsetstrokecolor{currentstroke}%
\pgfsetstrokeopacity{0.000000}%
\pgfsetdash{}{0pt}%
\pgfpathmoveto{\pgfqpoint{1.445879in}{0.613486in}}%
\pgfpathlineto{\pgfqpoint{1.455884in}{0.613486in}}%
\pgfpathlineto{\pgfqpoint{1.455884in}{1.612282in}}%
\pgfpathlineto{\pgfqpoint{1.445879in}{1.612282in}}%
\pgfpathlineto{\pgfqpoint{1.445879in}{0.613486in}}%
\pgfpathclose%
\pgfusepath{fill}%
\end{pgfscope}%
\begin{pgfscope}%
\pgfpathrectangle{\pgfqpoint{0.693757in}{0.613486in}}{\pgfqpoint{5.541243in}{3.963477in}}%
\pgfusepath{clip}%
\pgfsetbuttcap%
\pgfsetmiterjoin%
\definecolor{currentfill}{rgb}{0.000000,0.000000,1.000000}%
\pgfsetfillcolor{currentfill}%
\pgfsetlinewidth{0.000000pt}%
\definecolor{currentstroke}{rgb}{0.000000,0.000000,0.000000}%
\pgfsetstrokecolor{currentstroke}%
\pgfsetstrokeopacity{0.000000}%
\pgfsetdash{}{0pt}%
\pgfpathmoveto{\pgfqpoint{1.458386in}{0.613486in}}%
\pgfpathlineto{\pgfqpoint{1.468391in}{0.613486in}}%
\pgfpathlineto{\pgfqpoint{1.468391in}{2.595224in}}%
\pgfpathlineto{\pgfqpoint{1.458386in}{2.595224in}}%
\pgfpathlineto{\pgfqpoint{1.458386in}{0.613486in}}%
\pgfpathclose%
\pgfusepath{fill}%
\end{pgfscope}%
\begin{pgfscope}%
\pgfpathrectangle{\pgfqpoint{0.693757in}{0.613486in}}{\pgfqpoint{5.541243in}{3.963477in}}%
\pgfusepath{clip}%
\pgfsetbuttcap%
\pgfsetmiterjoin%
\definecolor{currentfill}{rgb}{0.000000,0.000000,1.000000}%
\pgfsetfillcolor{currentfill}%
\pgfsetlinewidth{0.000000pt}%
\definecolor{currentstroke}{rgb}{0.000000,0.000000,0.000000}%
\pgfsetstrokecolor{currentstroke}%
\pgfsetstrokeopacity{0.000000}%
\pgfsetdash{}{0pt}%
\pgfpathmoveto{\pgfqpoint{1.470892in}{0.613486in}}%
\pgfpathlineto{\pgfqpoint{1.480897in}{0.613486in}}%
\pgfpathlineto{\pgfqpoint{1.480897in}{2.611074in}}%
\pgfpathlineto{\pgfqpoint{1.470892in}{2.611074in}}%
\pgfpathlineto{\pgfqpoint{1.470892in}{0.613486in}}%
\pgfpathclose%
\pgfusepath{fill}%
\end{pgfscope}%
\begin{pgfscope}%
\pgfpathrectangle{\pgfqpoint{0.693757in}{0.613486in}}{\pgfqpoint{5.541243in}{3.963477in}}%
\pgfusepath{clip}%
\pgfsetbuttcap%
\pgfsetmiterjoin%
\definecolor{currentfill}{rgb}{0.000000,0.000000,1.000000}%
\pgfsetfillcolor{currentfill}%
\pgfsetlinewidth{0.000000pt}%
\definecolor{currentstroke}{rgb}{0.000000,0.000000,0.000000}%
\pgfsetstrokecolor{currentstroke}%
\pgfsetstrokeopacity{0.000000}%
\pgfsetdash{}{0pt}%
\pgfpathmoveto{\pgfqpoint{1.483398in}{0.613486in}}%
\pgfpathlineto{\pgfqpoint{1.493403in}{0.613486in}}%
\pgfpathlineto{\pgfqpoint{1.493403in}{1.604355in}}%
\pgfpathlineto{\pgfqpoint{1.483398in}{1.604355in}}%
\pgfpathlineto{\pgfqpoint{1.483398in}{0.613486in}}%
\pgfpathclose%
\pgfusepath{fill}%
\end{pgfscope}%
\begin{pgfscope}%
\pgfpathrectangle{\pgfqpoint{0.693757in}{0.613486in}}{\pgfqpoint{5.541243in}{3.963477in}}%
\pgfusepath{clip}%
\pgfsetbuttcap%
\pgfsetmiterjoin%
\definecolor{currentfill}{rgb}{0.000000,0.000000,1.000000}%
\pgfsetfillcolor{currentfill}%
\pgfsetlinewidth{0.000000pt}%
\definecolor{currentstroke}{rgb}{0.000000,0.000000,0.000000}%
\pgfsetstrokecolor{currentstroke}%
\pgfsetstrokeopacity{0.000000}%
\pgfsetdash{}{0pt}%
\pgfpathmoveto{\pgfqpoint{1.495904in}{0.613486in}}%
\pgfpathlineto{\pgfqpoint{1.505909in}{0.613486in}}%
\pgfpathlineto{\pgfqpoint{1.505909in}{1.612282in}}%
\pgfpathlineto{\pgfqpoint{1.495904in}{1.612282in}}%
\pgfpathlineto{\pgfqpoint{1.495904in}{0.613486in}}%
\pgfpathclose%
\pgfusepath{fill}%
\end{pgfscope}%
\begin{pgfscope}%
\pgfpathrectangle{\pgfqpoint{0.693757in}{0.613486in}}{\pgfqpoint{5.541243in}{3.963477in}}%
\pgfusepath{clip}%
\pgfsetbuttcap%
\pgfsetmiterjoin%
\definecolor{currentfill}{rgb}{0.000000,0.000000,1.000000}%
\pgfsetfillcolor{currentfill}%
\pgfsetlinewidth{0.000000pt}%
\definecolor{currentstroke}{rgb}{0.000000,0.000000,0.000000}%
\pgfsetstrokecolor{currentstroke}%
\pgfsetstrokeopacity{0.000000}%
\pgfsetdash{}{0pt}%
\pgfpathmoveto{\pgfqpoint{1.508410in}{0.613486in}}%
\pgfpathlineto{\pgfqpoint{1.518415in}{0.613486in}}%
\pgfpathlineto{\pgfqpoint{1.518415in}{2.595224in}}%
\pgfpathlineto{\pgfqpoint{1.508410in}{2.595224in}}%
\pgfpathlineto{\pgfqpoint{1.508410in}{0.613486in}}%
\pgfpathclose%
\pgfusepath{fill}%
\end{pgfscope}%
\begin{pgfscope}%
\pgfpathrectangle{\pgfqpoint{0.693757in}{0.613486in}}{\pgfqpoint{5.541243in}{3.963477in}}%
\pgfusepath{clip}%
\pgfsetbuttcap%
\pgfsetmiterjoin%
\definecolor{currentfill}{rgb}{0.000000,0.000000,1.000000}%
\pgfsetfillcolor{currentfill}%
\pgfsetlinewidth{0.000000pt}%
\definecolor{currentstroke}{rgb}{0.000000,0.000000,0.000000}%
\pgfsetstrokecolor{currentstroke}%
\pgfsetstrokeopacity{0.000000}%
\pgfsetdash{}{0pt}%
\pgfpathmoveto{\pgfqpoint{1.520917in}{0.613486in}}%
\pgfpathlineto{\pgfqpoint{1.530921in}{0.613486in}}%
\pgfpathlineto{\pgfqpoint{1.530921in}{2.611074in}}%
\pgfpathlineto{\pgfqpoint{1.520917in}{2.611074in}}%
\pgfpathlineto{\pgfqpoint{1.520917in}{0.613486in}}%
\pgfpathclose%
\pgfusepath{fill}%
\end{pgfscope}%
\begin{pgfscope}%
\pgfpathrectangle{\pgfqpoint{0.693757in}{0.613486in}}{\pgfqpoint{5.541243in}{3.963477in}}%
\pgfusepath{clip}%
\pgfsetbuttcap%
\pgfsetmiterjoin%
\definecolor{currentfill}{rgb}{0.000000,0.000000,1.000000}%
\pgfsetfillcolor{currentfill}%
\pgfsetlinewidth{0.000000pt}%
\definecolor{currentstroke}{rgb}{0.000000,0.000000,0.000000}%
\pgfsetstrokecolor{currentstroke}%
\pgfsetstrokeopacity{0.000000}%
\pgfsetdash{}{0pt}%
\pgfpathmoveto{\pgfqpoint{1.533423in}{0.613486in}}%
\pgfpathlineto{\pgfqpoint{1.543428in}{0.613486in}}%
\pgfpathlineto{\pgfqpoint{1.543428in}{1.604355in}}%
\pgfpathlineto{\pgfqpoint{1.533423in}{1.604355in}}%
\pgfpathlineto{\pgfqpoint{1.533423in}{0.613486in}}%
\pgfpathclose%
\pgfusepath{fill}%
\end{pgfscope}%
\begin{pgfscope}%
\pgfpathrectangle{\pgfqpoint{0.693757in}{0.613486in}}{\pgfqpoint{5.541243in}{3.963477in}}%
\pgfusepath{clip}%
\pgfsetbuttcap%
\pgfsetmiterjoin%
\definecolor{currentfill}{rgb}{0.000000,0.000000,1.000000}%
\pgfsetfillcolor{currentfill}%
\pgfsetlinewidth{0.000000pt}%
\definecolor{currentstroke}{rgb}{0.000000,0.000000,0.000000}%
\pgfsetstrokecolor{currentstroke}%
\pgfsetstrokeopacity{0.000000}%
\pgfsetdash{}{0pt}%
\pgfpathmoveto{\pgfqpoint{1.545929in}{0.613486in}}%
\pgfpathlineto{\pgfqpoint{1.555934in}{0.613486in}}%
\pgfpathlineto{\pgfqpoint{1.555934in}{1.612282in}}%
\pgfpathlineto{\pgfqpoint{1.545929in}{1.612282in}}%
\pgfpathlineto{\pgfqpoint{1.545929in}{0.613486in}}%
\pgfpathclose%
\pgfusepath{fill}%
\end{pgfscope}%
\begin{pgfscope}%
\pgfpathrectangle{\pgfqpoint{0.693757in}{0.613486in}}{\pgfqpoint{5.541243in}{3.963477in}}%
\pgfusepath{clip}%
\pgfsetbuttcap%
\pgfsetmiterjoin%
\definecolor{currentfill}{rgb}{0.000000,0.000000,1.000000}%
\pgfsetfillcolor{currentfill}%
\pgfsetlinewidth{0.000000pt}%
\definecolor{currentstroke}{rgb}{0.000000,0.000000,0.000000}%
\pgfsetstrokecolor{currentstroke}%
\pgfsetstrokeopacity{0.000000}%
\pgfsetdash{}{0pt}%
\pgfpathmoveto{\pgfqpoint{1.558435in}{0.613486in}}%
\pgfpathlineto{\pgfqpoint{1.568440in}{0.613486in}}%
\pgfpathlineto{\pgfqpoint{1.568440in}{2.595224in}}%
\pgfpathlineto{\pgfqpoint{1.558435in}{2.595224in}}%
\pgfpathlineto{\pgfqpoint{1.558435in}{0.613486in}}%
\pgfpathclose%
\pgfusepath{fill}%
\end{pgfscope}%
\begin{pgfscope}%
\pgfpathrectangle{\pgfqpoint{0.693757in}{0.613486in}}{\pgfqpoint{5.541243in}{3.963477in}}%
\pgfusepath{clip}%
\pgfsetbuttcap%
\pgfsetmiterjoin%
\definecolor{currentfill}{rgb}{0.000000,0.000000,1.000000}%
\pgfsetfillcolor{currentfill}%
\pgfsetlinewidth{0.000000pt}%
\definecolor{currentstroke}{rgb}{0.000000,0.000000,0.000000}%
\pgfsetstrokecolor{currentstroke}%
\pgfsetstrokeopacity{0.000000}%
\pgfsetdash{}{0pt}%
\pgfpathmoveto{\pgfqpoint{1.570941in}{0.613486in}}%
\pgfpathlineto{\pgfqpoint{1.580946in}{0.613486in}}%
\pgfpathlineto{\pgfqpoint{1.580946in}{2.611074in}}%
\pgfpathlineto{\pgfqpoint{1.570941in}{2.611074in}}%
\pgfpathlineto{\pgfqpoint{1.570941in}{0.613486in}}%
\pgfpathclose%
\pgfusepath{fill}%
\end{pgfscope}%
\begin{pgfscope}%
\pgfpathrectangle{\pgfqpoint{0.693757in}{0.613486in}}{\pgfqpoint{5.541243in}{3.963477in}}%
\pgfusepath{clip}%
\pgfsetbuttcap%
\pgfsetmiterjoin%
\definecolor{currentfill}{rgb}{0.000000,0.000000,1.000000}%
\pgfsetfillcolor{currentfill}%
\pgfsetlinewidth{0.000000pt}%
\definecolor{currentstroke}{rgb}{0.000000,0.000000,0.000000}%
\pgfsetstrokecolor{currentstroke}%
\pgfsetstrokeopacity{0.000000}%
\pgfsetdash{}{0pt}%
\pgfpathmoveto{\pgfqpoint{1.583447in}{0.613486in}}%
\pgfpathlineto{\pgfqpoint{1.593452in}{0.613486in}}%
\pgfpathlineto{\pgfqpoint{1.593452in}{1.604355in}}%
\pgfpathlineto{\pgfqpoint{1.583447in}{1.604355in}}%
\pgfpathlineto{\pgfqpoint{1.583447in}{0.613486in}}%
\pgfpathclose%
\pgfusepath{fill}%
\end{pgfscope}%
\begin{pgfscope}%
\pgfpathrectangle{\pgfqpoint{0.693757in}{0.613486in}}{\pgfqpoint{5.541243in}{3.963477in}}%
\pgfusepath{clip}%
\pgfsetbuttcap%
\pgfsetmiterjoin%
\definecolor{currentfill}{rgb}{0.000000,0.000000,1.000000}%
\pgfsetfillcolor{currentfill}%
\pgfsetlinewidth{0.000000pt}%
\definecolor{currentstroke}{rgb}{0.000000,0.000000,0.000000}%
\pgfsetstrokecolor{currentstroke}%
\pgfsetstrokeopacity{0.000000}%
\pgfsetdash{}{0pt}%
\pgfpathmoveto{\pgfqpoint{1.595954in}{0.613486in}}%
\pgfpathlineto{\pgfqpoint{1.605959in}{0.613486in}}%
\pgfpathlineto{\pgfqpoint{1.605959in}{1.612282in}}%
\pgfpathlineto{\pgfqpoint{1.595954in}{1.612282in}}%
\pgfpathlineto{\pgfqpoint{1.595954in}{0.613486in}}%
\pgfpathclose%
\pgfusepath{fill}%
\end{pgfscope}%
\begin{pgfscope}%
\pgfpathrectangle{\pgfqpoint{0.693757in}{0.613486in}}{\pgfqpoint{5.541243in}{3.963477in}}%
\pgfusepath{clip}%
\pgfsetbuttcap%
\pgfsetmiterjoin%
\definecolor{currentfill}{rgb}{0.000000,0.000000,1.000000}%
\pgfsetfillcolor{currentfill}%
\pgfsetlinewidth{0.000000pt}%
\definecolor{currentstroke}{rgb}{0.000000,0.000000,0.000000}%
\pgfsetstrokecolor{currentstroke}%
\pgfsetstrokeopacity{0.000000}%
\pgfsetdash{}{0pt}%
\pgfpathmoveto{\pgfqpoint{1.608460in}{0.613486in}}%
\pgfpathlineto{\pgfqpoint{1.618465in}{0.613486in}}%
\pgfpathlineto{\pgfqpoint{1.618465in}{2.595224in}}%
\pgfpathlineto{\pgfqpoint{1.608460in}{2.595224in}}%
\pgfpathlineto{\pgfqpoint{1.608460in}{0.613486in}}%
\pgfpathclose%
\pgfusepath{fill}%
\end{pgfscope}%
\begin{pgfscope}%
\pgfpathrectangle{\pgfqpoint{0.693757in}{0.613486in}}{\pgfqpoint{5.541243in}{3.963477in}}%
\pgfusepath{clip}%
\pgfsetbuttcap%
\pgfsetmiterjoin%
\definecolor{currentfill}{rgb}{0.000000,0.000000,1.000000}%
\pgfsetfillcolor{currentfill}%
\pgfsetlinewidth{0.000000pt}%
\definecolor{currentstroke}{rgb}{0.000000,0.000000,0.000000}%
\pgfsetstrokecolor{currentstroke}%
\pgfsetstrokeopacity{0.000000}%
\pgfsetdash{}{0pt}%
\pgfpathmoveto{\pgfqpoint{1.620966in}{0.613486in}}%
\pgfpathlineto{\pgfqpoint{1.630971in}{0.613486in}}%
\pgfpathlineto{\pgfqpoint{1.630971in}{2.611074in}}%
\pgfpathlineto{\pgfqpoint{1.620966in}{2.611074in}}%
\pgfpathlineto{\pgfqpoint{1.620966in}{0.613486in}}%
\pgfpathclose%
\pgfusepath{fill}%
\end{pgfscope}%
\begin{pgfscope}%
\pgfpathrectangle{\pgfqpoint{0.693757in}{0.613486in}}{\pgfqpoint{5.541243in}{3.963477in}}%
\pgfusepath{clip}%
\pgfsetbuttcap%
\pgfsetmiterjoin%
\definecolor{currentfill}{rgb}{0.000000,0.000000,1.000000}%
\pgfsetfillcolor{currentfill}%
\pgfsetlinewidth{0.000000pt}%
\definecolor{currentstroke}{rgb}{0.000000,0.000000,0.000000}%
\pgfsetstrokecolor{currentstroke}%
\pgfsetstrokeopacity{0.000000}%
\pgfsetdash{}{0pt}%
\pgfpathmoveto{\pgfqpoint{1.633472in}{0.613486in}}%
\pgfpathlineto{\pgfqpoint{1.643477in}{0.613486in}}%
\pgfpathlineto{\pgfqpoint{1.643477in}{1.604355in}}%
\pgfpathlineto{\pgfqpoint{1.633472in}{1.604355in}}%
\pgfpathlineto{\pgfqpoint{1.633472in}{0.613486in}}%
\pgfpathclose%
\pgfusepath{fill}%
\end{pgfscope}%
\begin{pgfscope}%
\pgfpathrectangle{\pgfqpoint{0.693757in}{0.613486in}}{\pgfqpoint{5.541243in}{3.963477in}}%
\pgfusepath{clip}%
\pgfsetbuttcap%
\pgfsetmiterjoin%
\definecolor{currentfill}{rgb}{0.000000,0.000000,1.000000}%
\pgfsetfillcolor{currentfill}%
\pgfsetlinewidth{0.000000pt}%
\definecolor{currentstroke}{rgb}{0.000000,0.000000,0.000000}%
\pgfsetstrokecolor{currentstroke}%
\pgfsetstrokeopacity{0.000000}%
\pgfsetdash{}{0pt}%
\pgfpathmoveto{\pgfqpoint{1.645978in}{0.613486in}}%
\pgfpathlineto{\pgfqpoint{1.655983in}{0.613486in}}%
\pgfpathlineto{\pgfqpoint{1.655983in}{1.612282in}}%
\pgfpathlineto{\pgfqpoint{1.645978in}{1.612282in}}%
\pgfpathlineto{\pgfqpoint{1.645978in}{0.613486in}}%
\pgfpathclose%
\pgfusepath{fill}%
\end{pgfscope}%
\begin{pgfscope}%
\pgfpathrectangle{\pgfqpoint{0.693757in}{0.613486in}}{\pgfqpoint{5.541243in}{3.963477in}}%
\pgfusepath{clip}%
\pgfsetbuttcap%
\pgfsetmiterjoin%
\definecolor{currentfill}{rgb}{0.000000,0.000000,1.000000}%
\pgfsetfillcolor{currentfill}%
\pgfsetlinewidth{0.000000pt}%
\definecolor{currentstroke}{rgb}{0.000000,0.000000,0.000000}%
\pgfsetstrokecolor{currentstroke}%
\pgfsetstrokeopacity{0.000000}%
\pgfsetdash{}{0pt}%
\pgfpathmoveto{\pgfqpoint{1.658485in}{0.613486in}}%
\pgfpathlineto{\pgfqpoint{1.668490in}{0.613486in}}%
\pgfpathlineto{\pgfqpoint{1.668490in}{2.595224in}}%
\pgfpathlineto{\pgfqpoint{1.658485in}{2.595224in}}%
\pgfpathlineto{\pgfqpoint{1.658485in}{0.613486in}}%
\pgfpathclose%
\pgfusepath{fill}%
\end{pgfscope}%
\begin{pgfscope}%
\pgfpathrectangle{\pgfqpoint{0.693757in}{0.613486in}}{\pgfqpoint{5.541243in}{3.963477in}}%
\pgfusepath{clip}%
\pgfsetbuttcap%
\pgfsetmiterjoin%
\definecolor{currentfill}{rgb}{0.000000,0.000000,1.000000}%
\pgfsetfillcolor{currentfill}%
\pgfsetlinewidth{0.000000pt}%
\definecolor{currentstroke}{rgb}{0.000000,0.000000,0.000000}%
\pgfsetstrokecolor{currentstroke}%
\pgfsetstrokeopacity{0.000000}%
\pgfsetdash{}{0pt}%
\pgfpathmoveto{\pgfqpoint{1.670991in}{0.613486in}}%
\pgfpathlineto{\pgfqpoint{1.680996in}{0.613486in}}%
\pgfpathlineto{\pgfqpoint{1.680996in}{2.611074in}}%
\pgfpathlineto{\pgfqpoint{1.670991in}{2.611074in}}%
\pgfpathlineto{\pgfqpoint{1.670991in}{0.613486in}}%
\pgfpathclose%
\pgfusepath{fill}%
\end{pgfscope}%
\begin{pgfscope}%
\pgfpathrectangle{\pgfqpoint{0.693757in}{0.613486in}}{\pgfqpoint{5.541243in}{3.963477in}}%
\pgfusepath{clip}%
\pgfsetbuttcap%
\pgfsetmiterjoin%
\definecolor{currentfill}{rgb}{0.000000,0.000000,1.000000}%
\pgfsetfillcolor{currentfill}%
\pgfsetlinewidth{0.000000pt}%
\definecolor{currentstroke}{rgb}{0.000000,0.000000,0.000000}%
\pgfsetstrokecolor{currentstroke}%
\pgfsetstrokeopacity{0.000000}%
\pgfsetdash{}{0pt}%
\pgfpathmoveto{\pgfqpoint{1.683497in}{0.613486in}}%
\pgfpathlineto{\pgfqpoint{1.693502in}{0.613486in}}%
\pgfpathlineto{\pgfqpoint{1.693502in}{1.604355in}}%
\pgfpathlineto{\pgfqpoint{1.683497in}{1.604355in}}%
\pgfpathlineto{\pgfqpoint{1.683497in}{0.613486in}}%
\pgfpathclose%
\pgfusepath{fill}%
\end{pgfscope}%
\begin{pgfscope}%
\pgfpathrectangle{\pgfqpoint{0.693757in}{0.613486in}}{\pgfqpoint{5.541243in}{3.963477in}}%
\pgfusepath{clip}%
\pgfsetbuttcap%
\pgfsetmiterjoin%
\definecolor{currentfill}{rgb}{0.000000,0.000000,1.000000}%
\pgfsetfillcolor{currentfill}%
\pgfsetlinewidth{0.000000pt}%
\definecolor{currentstroke}{rgb}{0.000000,0.000000,0.000000}%
\pgfsetstrokecolor{currentstroke}%
\pgfsetstrokeopacity{0.000000}%
\pgfsetdash{}{0pt}%
\pgfpathmoveto{\pgfqpoint{1.696003in}{0.613486in}}%
\pgfpathlineto{\pgfqpoint{1.706008in}{0.613486in}}%
\pgfpathlineto{\pgfqpoint{1.706008in}{1.612282in}}%
\pgfpathlineto{\pgfqpoint{1.696003in}{1.612282in}}%
\pgfpathlineto{\pgfqpoint{1.696003in}{0.613486in}}%
\pgfpathclose%
\pgfusepath{fill}%
\end{pgfscope}%
\begin{pgfscope}%
\pgfpathrectangle{\pgfqpoint{0.693757in}{0.613486in}}{\pgfqpoint{5.541243in}{3.963477in}}%
\pgfusepath{clip}%
\pgfsetbuttcap%
\pgfsetmiterjoin%
\definecolor{currentfill}{rgb}{0.000000,0.000000,1.000000}%
\pgfsetfillcolor{currentfill}%
\pgfsetlinewidth{0.000000pt}%
\definecolor{currentstroke}{rgb}{0.000000,0.000000,0.000000}%
\pgfsetstrokecolor{currentstroke}%
\pgfsetstrokeopacity{0.000000}%
\pgfsetdash{}{0pt}%
\pgfpathmoveto{\pgfqpoint{1.708509in}{0.613486in}}%
\pgfpathlineto{\pgfqpoint{1.718514in}{0.613486in}}%
\pgfpathlineto{\pgfqpoint{1.718514in}{2.595224in}}%
\pgfpathlineto{\pgfqpoint{1.708509in}{2.595224in}}%
\pgfpathlineto{\pgfqpoint{1.708509in}{0.613486in}}%
\pgfpathclose%
\pgfusepath{fill}%
\end{pgfscope}%
\begin{pgfscope}%
\pgfpathrectangle{\pgfqpoint{0.693757in}{0.613486in}}{\pgfqpoint{5.541243in}{3.963477in}}%
\pgfusepath{clip}%
\pgfsetbuttcap%
\pgfsetmiterjoin%
\definecolor{currentfill}{rgb}{0.000000,0.000000,1.000000}%
\pgfsetfillcolor{currentfill}%
\pgfsetlinewidth{0.000000pt}%
\definecolor{currentstroke}{rgb}{0.000000,0.000000,0.000000}%
\pgfsetstrokecolor{currentstroke}%
\pgfsetstrokeopacity{0.000000}%
\pgfsetdash{}{0pt}%
\pgfpathmoveto{\pgfqpoint{1.721016in}{0.613486in}}%
\pgfpathlineto{\pgfqpoint{1.731021in}{0.613486in}}%
\pgfpathlineto{\pgfqpoint{1.731021in}{2.611074in}}%
\pgfpathlineto{\pgfqpoint{1.721016in}{2.611074in}}%
\pgfpathlineto{\pgfqpoint{1.721016in}{0.613486in}}%
\pgfpathclose%
\pgfusepath{fill}%
\end{pgfscope}%
\begin{pgfscope}%
\pgfpathrectangle{\pgfqpoint{0.693757in}{0.613486in}}{\pgfqpoint{5.541243in}{3.963477in}}%
\pgfusepath{clip}%
\pgfsetbuttcap%
\pgfsetmiterjoin%
\definecolor{currentfill}{rgb}{0.000000,0.000000,1.000000}%
\pgfsetfillcolor{currentfill}%
\pgfsetlinewidth{0.000000pt}%
\definecolor{currentstroke}{rgb}{0.000000,0.000000,0.000000}%
\pgfsetstrokecolor{currentstroke}%
\pgfsetstrokeopacity{0.000000}%
\pgfsetdash{}{0pt}%
\pgfpathmoveto{\pgfqpoint{1.733522in}{0.613486in}}%
\pgfpathlineto{\pgfqpoint{1.743527in}{0.613486in}}%
\pgfpathlineto{\pgfqpoint{1.743527in}{1.604355in}}%
\pgfpathlineto{\pgfqpoint{1.733522in}{1.604355in}}%
\pgfpathlineto{\pgfqpoint{1.733522in}{0.613486in}}%
\pgfpathclose%
\pgfusepath{fill}%
\end{pgfscope}%
\begin{pgfscope}%
\pgfpathrectangle{\pgfqpoint{0.693757in}{0.613486in}}{\pgfqpoint{5.541243in}{3.963477in}}%
\pgfusepath{clip}%
\pgfsetbuttcap%
\pgfsetmiterjoin%
\definecolor{currentfill}{rgb}{0.000000,0.000000,1.000000}%
\pgfsetfillcolor{currentfill}%
\pgfsetlinewidth{0.000000pt}%
\definecolor{currentstroke}{rgb}{0.000000,0.000000,0.000000}%
\pgfsetstrokecolor{currentstroke}%
\pgfsetstrokeopacity{0.000000}%
\pgfsetdash{}{0pt}%
\pgfpathmoveto{\pgfqpoint{1.746028in}{0.613486in}}%
\pgfpathlineto{\pgfqpoint{1.756033in}{0.613486in}}%
\pgfpathlineto{\pgfqpoint{1.756033in}{1.612282in}}%
\pgfpathlineto{\pgfqpoint{1.746028in}{1.612282in}}%
\pgfpathlineto{\pgfqpoint{1.746028in}{0.613486in}}%
\pgfpathclose%
\pgfusepath{fill}%
\end{pgfscope}%
\begin{pgfscope}%
\pgfpathrectangle{\pgfqpoint{0.693757in}{0.613486in}}{\pgfqpoint{5.541243in}{3.963477in}}%
\pgfusepath{clip}%
\pgfsetbuttcap%
\pgfsetmiterjoin%
\definecolor{currentfill}{rgb}{0.000000,0.000000,1.000000}%
\pgfsetfillcolor{currentfill}%
\pgfsetlinewidth{0.000000pt}%
\definecolor{currentstroke}{rgb}{0.000000,0.000000,0.000000}%
\pgfsetstrokecolor{currentstroke}%
\pgfsetstrokeopacity{0.000000}%
\pgfsetdash{}{0pt}%
\pgfpathmoveto{\pgfqpoint{1.758534in}{0.613486in}}%
\pgfpathlineto{\pgfqpoint{1.768539in}{0.613486in}}%
\pgfpathlineto{\pgfqpoint{1.768539in}{2.595224in}}%
\pgfpathlineto{\pgfqpoint{1.758534in}{2.595224in}}%
\pgfpathlineto{\pgfqpoint{1.758534in}{0.613486in}}%
\pgfpathclose%
\pgfusepath{fill}%
\end{pgfscope}%
\begin{pgfscope}%
\pgfpathrectangle{\pgfqpoint{0.693757in}{0.613486in}}{\pgfqpoint{5.541243in}{3.963477in}}%
\pgfusepath{clip}%
\pgfsetbuttcap%
\pgfsetmiterjoin%
\definecolor{currentfill}{rgb}{0.000000,0.000000,1.000000}%
\pgfsetfillcolor{currentfill}%
\pgfsetlinewidth{0.000000pt}%
\definecolor{currentstroke}{rgb}{0.000000,0.000000,0.000000}%
\pgfsetstrokecolor{currentstroke}%
\pgfsetstrokeopacity{0.000000}%
\pgfsetdash{}{0pt}%
\pgfpathmoveto{\pgfqpoint{1.771040in}{0.613486in}}%
\pgfpathlineto{\pgfqpoint{1.781045in}{0.613486in}}%
\pgfpathlineto{\pgfqpoint{1.781045in}{2.611074in}}%
\pgfpathlineto{\pgfqpoint{1.771040in}{2.611074in}}%
\pgfpathlineto{\pgfqpoint{1.771040in}{0.613486in}}%
\pgfpathclose%
\pgfusepath{fill}%
\end{pgfscope}%
\begin{pgfscope}%
\pgfpathrectangle{\pgfqpoint{0.693757in}{0.613486in}}{\pgfqpoint{5.541243in}{3.963477in}}%
\pgfusepath{clip}%
\pgfsetbuttcap%
\pgfsetmiterjoin%
\definecolor{currentfill}{rgb}{0.000000,0.000000,1.000000}%
\pgfsetfillcolor{currentfill}%
\pgfsetlinewidth{0.000000pt}%
\definecolor{currentstroke}{rgb}{0.000000,0.000000,0.000000}%
\pgfsetstrokecolor{currentstroke}%
\pgfsetstrokeopacity{0.000000}%
\pgfsetdash{}{0pt}%
\pgfpathmoveto{\pgfqpoint{1.783547in}{0.613486in}}%
\pgfpathlineto{\pgfqpoint{1.793551in}{0.613486in}}%
\pgfpathlineto{\pgfqpoint{1.793551in}{1.604355in}}%
\pgfpathlineto{\pgfqpoint{1.783547in}{1.604355in}}%
\pgfpathlineto{\pgfqpoint{1.783547in}{0.613486in}}%
\pgfpathclose%
\pgfusepath{fill}%
\end{pgfscope}%
\begin{pgfscope}%
\pgfpathrectangle{\pgfqpoint{0.693757in}{0.613486in}}{\pgfqpoint{5.541243in}{3.963477in}}%
\pgfusepath{clip}%
\pgfsetbuttcap%
\pgfsetmiterjoin%
\definecolor{currentfill}{rgb}{0.000000,0.000000,1.000000}%
\pgfsetfillcolor{currentfill}%
\pgfsetlinewidth{0.000000pt}%
\definecolor{currentstroke}{rgb}{0.000000,0.000000,0.000000}%
\pgfsetstrokecolor{currentstroke}%
\pgfsetstrokeopacity{0.000000}%
\pgfsetdash{}{0pt}%
\pgfpathmoveto{\pgfqpoint{1.796053in}{0.613486in}}%
\pgfpathlineto{\pgfqpoint{1.806058in}{0.613486in}}%
\pgfpathlineto{\pgfqpoint{1.806058in}{1.612282in}}%
\pgfpathlineto{\pgfqpoint{1.796053in}{1.612282in}}%
\pgfpathlineto{\pgfqpoint{1.796053in}{0.613486in}}%
\pgfpathclose%
\pgfusepath{fill}%
\end{pgfscope}%
\begin{pgfscope}%
\pgfpathrectangle{\pgfqpoint{0.693757in}{0.613486in}}{\pgfqpoint{5.541243in}{3.963477in}}%
\pgfusepath{clip}%
\pgfsetbuttcap%
\pgfsetmiterjoin%
\definecolor{currentfill}{rgb}{0.000000,0.000000,1.000000}%
\pgfsetfillcolor{currentfill}%
\pgfsetlinewidth{0.000000pt}%
\definecolor{currentstroke}{rgb}{0.000000,0.000000,0.000000}%
\pgfsetstrokecolor{currentstroke}%
\pgfsetstrokeopacity{0.000000}%
\pgfsetdash{}{0pt}%
\pgfpathmoveto{\pgfqpoint{1.808559in}{0.613486in}}%
\pgfpathlineto{\pgfqpoint{1.818564in}{0.613486in}}%
\pgfpathlineto{\pgfqpoint{1.818564in}{2.595224in}}%
\pgfpathlineto{\pgfqpoint{1.808559in}{2.595224in}}%
\pgfpathlineto{\pgfqpoint{1.808559in}{0.613486in}}%
\pgfpathclose%
\pgfusepath{fill}%
\end{pgfscope}%
\begin{pgfscope}%
\pgfpathrectangle{\pgfqpoint{0.693757in}{0.613486in}}{\pgfqpoint{5.541243in}{3.963477in}}%
\pgfusepath{clip}%
\pgfsetbuttcap%
\pgfsetmiterjoin%
\definecolor{currentfill}{rgb}{0.000000,0.000000,1.000000}%
\pgfsetfillcolor{currentfill}%
\pgfsetlinewidth{0.000000pt}%
\definecolor{currentstroke}{rgb}{0.000000,0.000000,0.000000}%
\pgfsetstrokecolor{currentstroke}%
\pgfsetstrokeopacity{0.000000}%
\pgfsetdash{}{0pt}%
\pgfpathmoveto{\pgfqpoint{1.821065in}{0.613486in}}%
\pgfpathlineto{\pgfqpoint{1.831070in}{0.613486in}}%
\pgfpathlineto{\pgfqpoint{1.831070in}{2.611074in}}%
\pgfpathlineto{\pgfqpoint{1.821065in}{2.611074in}}%
\pgfpathlineto{\pgfqpoint{1.821065in}{0.613486in}}%
\pgfpathclose%
\pgfusepath{fill}%
\end{pgfscope}%
\begin{pgfscope}%
\pgfpathrectangle{\pgfqpoint{0.693757in}{0.613486in}}{\pgfqpoint{5.541243in}{3.963477in}}%
\pgfusepath{clip}%
\pgfsetbuttcap%
\pgfsetmiterjoin%
\definecolor{currentfill}{rgb}{0.000000,0.000000,1.000000}%
\pgfsetfillcolor{currentfill}%
\pgfsetlinewidth{0.000000pt}%
\definecolor{currentstroke}{rgb}{0.000000,0.000000,0.000000}%
\pgfsetstrokecolor{currentstroke}%
\pgfsetstrokeopacity{0.000000}%
\pgfsetdash{}{0pt}%
\pgfpathmoveto{\pgfqpoint{1.833571in}{0.613486in}}%
\pgfpathlineto{\pgfqpoint{1.843576in}{0.613486in}}%
\pgfpathlineto{\pgfqpoint{1.843576in}{1.604355in}}%
\pgfpathlineto{\pgfqpoint{1.833571in}{1.604355in}}%
\pgfpathlineto{\pgfqpoint{1.833571in}{0.613486in}}%
\pgfpathclose%
\pgfusepath{fill}%
\end{pgfscope}%
\begin{pgfscope}%
\pgfpathrectangle{\pgfqpoint{0.693757in}{0.613486in}}{\pgfqpoint{5.541243in}{3.963477in}}%
\pgfusepath{clip}%
\pgfsetbuttcap%
\pgfsetmiterjoin%
\definecolor{currentfill}{rgb}{0.000000,0.000000,1.000000}%
\pgfsetfillcolor{currentfill}%
\pgfsetlinewidth{0.000000pt}%
\definecolor{currentstroke}{rgb}{0.000000,0.000000,0.000000}%
\pgfsetstrokecolor{currentstroke}%
\pgfsetstrokeopacity{0.000000}%
\pgfsetdash{}{0pt}%
\pgfpathmoveto{\pgfqpoint{1.846077in}{0.613486in}}%
\pgfpathlineto{\pgfqpoint{1.856082in}{0.613486in}}%
\pgfpathlineto{\pgfqpoint{1.856082in}{1.612282in}}%
\pgfpathlineto{\pgfqpoint{1.846077in}{1.612282in}}%
\pgfpathlineto{\pgfqpoint{1.846077in}{0.613486in}}%
\pgfpathclose%
\pgfusepath{fill}%
\end{pgfscope}%
\begin{pgfscope}%
\pgfpathrectangle{\pgfqpoint{0.693757in}{0.613486in}}{\pgfqpoint{5.541243in}{3.963477in}}%
\pgfusepath{clip}%
\pgfsetbuttcap%
\pgfsetmiterjoin%
\definecolor{currentfill}{rgb}{0.000000,0.000000,1.000000}%
\pgfsetfillcolor{currentfill}%
\pgfsetlinewidth{0.000000pt}%
\definecolor{currentstroke}{rgb}{0.000000,0.000000,0.000000}%
\pgfsetstrokecolor{currentstroke}%
\pgfsetstrokeopacity{0.000000}%
\pgfsetdash{}{0pt}%
\pgfpathmoveto{\pgfqpoint{1.858584in}{0.613486in}}%
\pgfpathlineto{\pgfqpoint{1.868589in}{0.613486in}}%
\pgfpathlineto{\pgfqpoint{1.868589in}{2.595224in}}%
\pgfpathlineto{\pgfqpoint{1.858584in}{2.595224in}}%
\pgfpathlineto{\pgfqpoint{1.858584in}{0.613486in}}%
\pgfpathclose%
\pgfusepath{fill}%
\end{pgfscope}%
\begin{pgfscope}%
\pgfpathrectangle{\pgfqpoint{0.693757in}{0.613486in}}{\pgfqpoint{5.541243in}{3.963477in}}%
\pgfusepath{clip}%
\pgfsetbuttcap%
\pgfsetmiterjoin%
\definecolor{currentfill}{rgb}{0.000000,0.000000,1.000000}%
\pgfsetfillcolor{currentfill}%
\pgfsetlinewidth{0.000000pt}%
\definecolor{currentstroke}{rgb}{0.000000,0.000000,0.000000}%
\pgfsetstrokecolor{currentstroke}%
\pgfsetstrokeopacity{0.000000}%
\pgfsetdash{}{0pt}%
\pgfpathmoveto{\pgfqpoint{1.871090in}{0.613486in}}%
\pgfpathlineto{\pgfqpoint{1.881095in}{0.613486in}}%
\pgfpathlineto{\pgfqpoint{1.881095in}{2.611074in}}%
\pgfpathlineto{\pgfqpoint{1.871090in}{2.611074in}}%
\pgfpathlineto{\pgfqpoint{1.871090in}{0.613486in}}%
\pgfpathclose%
\pgfusepath{fill}%
\end{pgfscope}%
\begin{pgfscope}%
\pgfpathrectangle{\pgfqpoint{0.693757in}{0.613486in}}{\pgfqpoint{5.541243in}{3.963477in}}%
\pgfusepath{clip}%
\pgfsetbuttcap%
\pgfsetmiterjoin%
\definecolor{currentfill}{rgb}{0.000000,0.000000,1.000000}%
\pgfsetfillcolor{currentfill}%
\pgfsetlinewidth{0.000000pt}%
\definecolor{currentstroke}{rgb}{0.000000,0.000000,0.000000}%
\pgfsetstrokecolor{currentstroke}%
\pgfsetstrokeopacity{0.000000}%
\pgfsetdash{}{0pt}%
\pgfpathmoveto{\pgfqpoint{1.883596in}{0.613486in}}%
\pgfpathlineto{\pgfqpoint{1.893601in}{0.613486in}}%
\pgfpathlineto{\pgfqpoint{1.893601in}{1.604355in}}%
\pgfpathlineto{\pgfqpoint{1.883596in}{1.604355in}}%
\pgfpathlineto{\pgfqpoint{1.883596in}{0.613486in}}%
\pgfpathclose%
\pgfusepath{fill}%
\end{pgfscope}%
\begin{pgfscope}%
\pgfpathrectangle{\pgfqpoint{0.693757in}{0.613486in}}{\pgfqpoint{5.541243in}{3.963477in}}%
\pgfusepath{clip}%
\pgfsetbuttcap%
\pgfsetmiterjoin%
\definecolor{currentfill}{rgb}{0.000000,0.000000,1.000000}%
\pgfsetfillcolor{currentfill}%
\pgfsetlinewidth{0.000000pt}%
\definecolor{currentstroke}{rgb}{0.000000,0.000000,0.000000}%
\pgfsetstrokecolor{currentstroke}%
\pgfsetstrokeopacity{0.000000}%
\pgfsetdash{}{0pt}%
\pgfpathmoveto{\pgfqpoint{1.896102in}{0.613486in}}%
\pgfpathlineto{\pgfqpoint{1.906107in}{0.613486in}}%
\pgfpathlineto{\pgfqpoint{1.906107in}{1.612282in}}%
\pgfpathlineto{\pgfqpoint{1.896102in}{1.612282in}}%
\pgfpathlineto{\pgfqpoint{1.896102in}{0.613486in}}%
\pgfpathclose%
\pgfusepath{fill}%
\end{pgfscope}%
\begin{pgfscope}%
\pgfpathrectangle{\pgfqpoint{0.693757in}{0.613486in}}{\pgfqpoint{5.541243in}{3.963477in}}%
\pgfusepath{clip}%
\pgfsetbuttcap%
\pgfsetmiterjoin%
\definecolor{currentfill}{rgb}{0.000000,0.000000,1.000000}%
\pgfsetfillcolor{currentfill}%
\pgfsetlinewidth{0.000000pt}%
\definecolor{currentstroke}{rgb}{0.000000,0.000000,0.000000}%
\pgfsetstrokecolor{currentstroke}%
\pgfsetstrokeopacity{0.000000}%
\pgfsetdash{}{0pt}%
\pgfpathmoveto{\pgfqpoint{1.908608in}{0.613486in}}%
\pgfpathlineto{\pgfqpoint{1.918613in}{0.613486in}}%
\pgfpathlineto{\pgfqpoint{1.918613in}{2.595224in}}%
\pgfpathlineto{\pgfqpoint{1.908608in}{2.595224in}}%
\pgfpathlineto{\pgfqpoint{1.908608in}{0.613486in}}%
\pgfpathclose%
\pgfusepath{fill}%
\end{pgfscope}%
\begin{pgfscope}%
\pgfpathrectangle{\pgfqpoint{0.693757in}{0.613486in}}{\pgfqpoint{5.541243in}{3.963477in}}%
\pgfusepath{clip}%
\pgfsetbuttcap%
\pgfsetmiterjoin%
\definecolor{currentfill}{rgb}{0.000000,0.000000,1.000000}%
\pgfsetfillcolor{currentfill}%
\pgfsetlinewidth{0.000000pt}%
\definecolor{currentstroke}{rgb}{0.000000,0.000000,0.000000}%
\pgfsetstrokecolor{currentstroke}%
\pgfsetstrokeopacity{0.000000}%
\pgfsetdash{}{0pt}%
\pgfpathmoveto{\pgfqpoint{1.921115in}{0.613486in}}%
\pgfpathlineto{\pgfqpoint{1.931120in}{0.613486in}}%
\pgfpathlineto{\pgfqpoint{1.931120in}{2.611074in}}%
\pgfpathlineto{\pgfqpoint{1.921115in}{2.611074in}}%
\pgfpathlineto{\pgfqpoint{1.921115in}{0.613486in}}%
\pgfpathclose%
\pgfusepath{fill}%
\end{pgfscope}%
\begin{pgfscope}%
\pgfpathrectangle{\pgfqpoint{0.693757in}{0.613486in}}{\pgfqpoint{5.541243in}{3.963477in}}%
\pgfusepath{clip}%
\pgfsetbuttcap%
\pgfsetmiterjoin%
\definecolor{currentfill}{rgb}{0.000000,0.000000,1.000000}%
\pgfsetfillcolor{currentfill}%
\pgfsetlinewidth{0.000000pt}%
\definecolor{currentstroke}{rgb}{0.000000,0.000000,0.000000}%
\pgfsetstrokecolor{currentstroke}%
\pgfsetstrokeopacity{0.000000}%
\pgfsetdash{}{0pt}%
\pgfpathmoveto{\pgfqpoint{1.933621in}{0.613486in}}%
\pgfpathlineto{\pgfqpoint{1.943626in}{0.613486in}}%
\pgfpathlineto{\pgfqpoint{1.943626in}{1.604355in}}%
\pgfpathlineto{\pgfqpoint{1.933621in}{1.604355in}}%
\pgfpathlineto{\pgfqpoint{1.933621in}{0.613486in}}%
\pgfpathclose%
\pgfusepath{fill}%
\end{pgfscope}%
\begin{pgfscope}%
\pgfpathrectangle{\pgfqpoint{0.693757in}{0.613486in}}{\pgfqpoint{5.541243in}{3.963477in}}%
\pgfusepath{clip}%
\pgfsetbuttcap%
\pgfsetmiterjoin%
\definecolor{currentfill}{rgb}{0.000000,0.000000,1.000000}%
\pgfsetfillcolor{currentfill}%
\pgfsetlinewidth{0.000000pt}%
\definecolor{currentstroke}{rgb}{0.000000,0.000000,0.000000}%
\pgfsetstrokecolor{currentstroke}%
\pgfsetstrokeopacity{0.000000}%
\pgfsetdash{}{0pt}%
\pgfpathmoveto{\pgfqpoint{1.946127in}{0.613486in}}%
\pgfpathlineto{\pgfqpoint{1.956132in}{0.613486in}}%
\pgfpathlineto{\pgfqpoint{1.956132in}{1.612282in}}%
\pgfpathlineto{\pgfqpoint{1.946127in}{1.612282in}}%
\pgfpathlineto{\pgfqpoint{1.946127in}{0.613486in}}%
\pgfpathclose%
\pgfusepath{fill}%
\end{pgfscope}%
\begin{pgfscope}%
\pgfpathrectangle{\pgfqpoint{0.693757in}{0.613486in}}{\pgfqpoint{5.541243in}{3.963477in}}%
\pgfusepath{clip}%
\pgfsetbuttcap%
\pgfsetmiterjoin%
\definecolor{currentfill}{rgb}{0.000000,0.000000,1.000000}%
\pgfsetfillcolor{currentfill}%
\pgfsetlinewidth{0.000000pt}%
\definecolor{currentstroke}{rgb}{0.000000,0.000000,0.000000}%
\pgfsetstrokecolor{currentstroke}%
\pgfsetstrokeopacity{0.000000}%
\pgfsetdash{}{0pt}%
\pgfpathmoveto{\pgfqpoint{1.958633in}{0.613486in}}%
\pgfpathlineto{\pgfqpoint{1.968638in}{0.613486in}}%
\pgfpathlineto{\pgfqpoint{1.968638in}{2.595224in}}%
\pgfpathlineto{\pgfqpoint{1.958633in}{2.595224in}}%
\pgfpathlineto{\pgfqpoint{1.958633in}{0.613486in}}%
\pgfpathclose%
\pgfusepath{fill}%
\end{pgfscope}%
\begin{pgfscope}%
\pgfpathrectangle{\pgfqpoint{0.693757in}{0.613486in}}{\pgfqpoint{5.541243in}{3.963477in}}%
\pgfusepath{clip}%
\pgfsetbuttcap%
\pgfsetmiterjoin%
\definecolor{currentfill}{rgb}{0.000000,0.000000,1.000000}%
\pgfsetfillcolor{currentfill}%
\pgfsetlinewidth{0.000000pt}%
\definecolor{currentstroke}{rgb}{0.000000,0.000000,0.000000}%
\pgfsetstrokecolor{currentstroke}%
\pgfsetstrokeopacity{0.000000}%
\pgfsetdash{}{0pt}%
\pgfpathmoveto{\pgfqpoint{1.971139in}{0.613486in}}%
\pgfpathlineto{\pgfqpoint{1.981144in}{0.613486in}}%
\pgfpathlineto{\pgfqpoint{1.981144in}{2.611074in}}%
\pgfpathlineto{\pgfqpoint{1.971139in}{2.611074in}}%
\pgfpathlineto{\pgfqpoint{1.971139in}{0.613486in}}%
\pgfpathclose%
\pgfusepath{fill}%
\end{pgfscope}%
\begin{pgfscope}%
\pgfpathrectangle{\pgfqpoint{0.693757in}{0.613486in}}{\pgfqpoint{5.541243in}{3.963477in}}%
\pgfusepath{clip}%
\pgfsetbuttcap%
\pgfsetmiterjoin%
\definecolor{currentfill}{rgb}{0.000000,0.000000,1.000000}%
\pgfsetfillcolor{currentfill}%
\pgfsetlinewidth{0.000000pt}%
\definecolor{currentstroke}{rgb}{0.000000,0.000000,0.000000}%
\pgfsetstrokecolor{currentstroke}%
\pgfsetstrokeopacity{0.000000}%
\pgfsetdash{}{0pt}%
\pgfpathmoveto{\pgfqpoint{1.983646in}{0.613486in}}%
\pgfpathlineto{\pgfqpoint{1.993651in}{0.613486in}}%
\pgfpathlineto{\pgfqpoint{1.993651in}{1.604355in}}%
\pgfpathlineto{\pgfqpoint{1.983646in}{1.604355in}}%
\pgfpathlineto{\pgfqpoint{1.983646in}{0.613486in}}%
\pgfpathclose%
\pgfusepath{fill}%
\end{pgfscope}%
\begin{pgfscope}%
\pgfpathrectangle{\pgfqpoint{0.693757in}{0.613486in}}{\pgfqpoint{5.541243in}{3.963477in}}%
\pgfusepath{clip}%
\pgfsetbuttcap%
\pgfsetmiterjoin%
\definecolor{currentfill}{rgb}{0.000000,0.000000,1.000000}%
\pgfsetfillcolor{currentfill}%
\pgfsetlinewidth{0.000000pt}%
\definecolor{currentstroke}{rgb}{0.000000,0.000000,0.000000}%
\pgfsetstrokecolor{currentstroke}%
\pgfsetstrokeopacity{0.000000}%
\pgfsetdash{}{0pt}%
\pgfpathmoveto{\pgfqpoint{1.996152in}{0.613486in}}%
\pgfpathlineto{\pgfqpoint{2.006157in}{0.613486in}}%
\pgfpathlineto{\pgfqpoint{2.006157in}{1.612282in}}%
\pgfpathlineto{\pgfqpoint{1.996152in}{1.612282in}}%
\pgfpathlineto{\pgfqpoint{1.996152in}{0.613486in}}%
\pgfpathclose%
\pgfusepath{fill}%
\end{pgfscope}%
\begin{pgfscope}%
\pgfpathrectangle{\pgfqpoint{0.693757in}{0.613486in}}{\pgfqpoint{5.541243in}{3.963477in}}%
\pgfusepath{clip}%
\pgfsetbuttcap%
\pgfsetmiterjoin%
\definecolor{currentfill}{rgb}{0.000000,0.000000,1.000000}%
\pgfsetfillcolor{currentfill}%
\pgfsetlinewidth{0.000000pt}%
\definecolor{currentstroke}{rgb}{0.000000,0.000000,0.000000}%
\pgfsetstrokecolor{currentstroke}%
\pgfsetstrokeopacity{0.000000}%
\pgfsetdash{}{0pt}%
\pgfpathmoveto{\pgfqpoint{2.008658in}{0.613486in}}%
\pgfpathlineto{\pgfqpoint{2.018663in}{0.613486in}}%
\pgfpathlineto{\pgfqpoint{2.018663in}{2.595224in}}%
\pgfpathlineto{\pgfqpoint{2.008658in}{2.595224in}}%
\pgfpathlineto{\pgfqpoint{2.008658in}{0.613486in}}%
\pgfpathclose%
\pgfusepath{fill}%
\end{pgfscope}%
\begin{pgfscope}%
\pgfpathrectangle{\pgfqpoint{0.693757in}{0.613486in}}{\pgfqpoint{5.541243in}{3.963477in}}%
\pgfusepath{clip}%
\pgfsetbuttcap%
\pgfsetmiterjoin%
\definecolor{currentfill}{rgb}{0.000000,0.000000,1.000000}%
\pgfsetfillcolor{currentfill}%
\pgfsetlinewidth{0.000000pt}%
\definecolor{currentstroke}{rgb}{0.000000,0.000000,0.000000}%
\pgfsetstrokecolor{currentstroke}%
\pgfsetstrokeopacity{0.000000}%
\pgfsetdash{}{0pt}%
\pgfpathmoveto{\pgfqpoint{2.021164in}{0.613486in}}%
\pgfpathlineto{\pgfqpoint{2.031169in}{0.613486in}}%
\pgfpathlineto{\pgfqpoint{2.031169in}{2.611074in}}%
\pgfpathlineto{\pgfqpoint{2.021164in}{2.611074in}}%
\pgfpathlineto{\pgfqpoint{2.021164in}{0.613486in}}%
\pgfpathclose%
\pgfusepath{fill}%
\end{pgfscope}%
\begin{pgfscope}%
\pgfpathrectangle{\pgfqpoint{0.693757in}{0.613486in}}{\pgfqpoint{5.541243in}{3.963477in}}%
\pgfusepath{clip}%
\pgfsetbuttcap%
\pgfsetmiterjoin%
\definecolor{currentfill}{rgb}{0.000000,0.000000,1.000000}%
\pgfsetfillcolor{currentfill}%
\pgfsetlinewidth{0.000000pt}%
\definecolor{currentstroke}{rgb}{0.000000,0.000000,0.000000}%
\pgfsetstrokecolor{currentstroke}%
\pgfsetstrokeopacity{0.000000}%
\pgfsetdash{}{0pt}%
\pgfpathmoveto{\pgfqpoint{2.033670in}{0.613486in}}%
\pgfpathlineto{\pgfqpoint{2.043675in}{0.613486in}}%
\pgfpathlineto{\pgfqpoint{2.043675in}{1.604355in}}%
\pgfpathlineto{\pgfqpoint{2.033670in}{1.604355in}}%
\pgfpathlineto{\pgfqpoint{2.033670in}{0.613486in}}%
\pgfpathclose%
\pgfusepath{fill}%
\end{pgfscope}%
\begin{pgfscope}%
\pgfpathrectangle{\pgfqpoint{0.693757in}{0.613486in}}{\pgfqpoint{5.541243in}{3.963477in}}%
\pgfusepath{clip}%
\pgfsetbuttcap%
\pgfsetmiterjoin%
\definecolor{currentfill}{rgb}{0.000000,0.000000,1.000000}%
\pgfsetfillcolor{currentfill}%
\pgfsetlinewidth{0.000000pt}%
\definecolor{currentstroke}{rgb}{0.000000,0.000000,0.000000}%
\pgfsetstrokecolor{currentstroke}%
\pgfsetstrokeopacity{0.000000}%
\pgfsetdash{}{0pt}%
\pgfpathmoveto{\pgfqpoint{2.046177in}{0.613486in}}%
\pgfpathlineto{\pgfqpoint{2.056181in}{0.613486in}}%
\pgfpathlineto{\pgfqpoint{2.056181in}{1.612282in}}%
\pgfpathlineto{\pgfqpoint{2.046177in}{1.612282in}}%
\pgfpathlineto{\pgfqpoint{2.046177in}{0.613486in}}%
\pgfpathclose%
\pgfusepath{fill}%
\end{pgfscope}%
\begin{pgfscope}%
\pgfpathrectangle{\pgfqpoint{0.693757in}{0.613486in}}{\pgfqpoint{5.541243in}{3.963477in}}%
\pgfusepath{clip}%
\pgfsetbuttcap%
\pgfsetmiterjoin%
\definecolor{currentfill}{rgb}{0.000000,0.000000,1.000000}%
\pgfsetfillcolor{currentfill}%
\pgfsetlinewidth{0.000000pt}%
\definecolor{currentstroke}{rgb}{0.000000,0.000000,0.000000}%
\pgfsetstrokecolor{currentstroke}%
\pgfsetstrokeopacity{0.000000}%
\pgfsetdash{}{0pt}%
\pgfpathmoveto{\pgfqpoint{2.058683in}{0.613486in}}%
\pgfpathlineto{\pgfqpoint{2.068688in}{0.613486in}}%
\pgfpathlineto{\pgfqpoint{2.068688in}{2.595224in}}%
\pgfpathlineto{\pgfqpoint{2.058683in}{2.595224in}}%
\pgfpathlineto{\pgfqpoint{2.058683in}{0.613486in}}%
\pgfpathclose%
\pgfusepath{fill}%
\end{pgfscope}%
\begin{pgfscope}%
\pgfpathrectangle{\pgfqpoint{0.693757in}{0.613486in}}{\pgfqpoint{5.541243in}{3.963477in}}%
\pgfusepath{clip}%
\pgfsetbuttcap%
\pgfsetmiterjoin%
\definecolor{currentfill}{rgb}{0.000000,0.000000,1.000000}%
\pgfsetfillcolor{currentfill}%
\pgfsetlinewidth{0.000000pt}%
\definecolor{currentstroke}{rgb}{0.000000,0.000000,0.000000}%
\pgfsetstrokecolor{currentstroke}%
\pgfsetstrokeopacity{0.000000}%
\pgfsetdash{}{0pt}%
\pgfpathmoveto{\pgfqpoint{2.071189in}{0.613486in}}%
\pgfpathlineto{\pgfqpoint{2.081194in}{0.613486in}}%
\pgfpathlineto{\pgfqpoint{2.081194in}{2.611074in}}%
\pgfpathlineto{\pgfqpoint{2.071189in}{2.611074in}}%
\pgfpathlineto{\pgfqpoint{2.071189in}{0.613486in}}%
\pgfpathclose%
\pgfusepath{fill}%
\end{pgfscope}%
\begin{pgfscope}%
\pgfpathrectangle{\pgfqpoint{0.693757in}{0.613486in}}{\pgfqpoint{5.541243in}{3.963477in}}%
\pgfusepath{clip}%
\pgfsetbuttcap%
\pgfsetmiterjoin%
\definecolor{currentfill}{rgb}{0.000000,0.000000,1.000000}%
\pgfsetfillcolor{currentfill}%
\pgfsetlinewidth{0.000000pt}%
\definecolor{currentstroke}{rgb}{0.000000,0.000000,0.000000}%
\pgfsetstrokecolor{currentstroke}%
\pgfsetstrokeopacity{0.000000}%
\pgfsetdash{}{0pt}%
\pgfpathmoveto{\pgfqpoint{2.083695in}{0.613486in}}%
\pgfpathlineto{\pgfqpoint{2.093700in}{0.613486in}}%
\pgfpathlineto{\pgfqpoint{2.093700in}{1.604355in}}%
\pgfpathlineto{\pgfqpoint{2.083695in}{1.604355in}}%
\pgfpathlineto{\pgfqpoint{2.083695in}{0.613486in}}%
\pgfpathclose%
\pgfusepath{fill}%
\end{pgfscope}%
\begin{pgfscope}%
\pgfpathrectangle{\pgfqpoint{0.693757in}{0.613486in}}{\pgfqpoint{5.541243in}{3.963477in}}%
\pgfusepath{clip}%
\pgfsetbuttcap%
\pgfsetmiterjoin%
\definecolor{currentfill}{rgb}{0.000000,0.000000,1.000000}%
\pgfsetfillcolor{currentfill}%
\pgfsetlinewidth{0.000000pt}%
\definecolor{currentstroke}{rgb}{0.000000,0.000000,0.000000}%
\pgfsetstrokecolor{currentstroke}%
\pgfsetstrokeopacity{0.000000}%
\pgfsetdash{}{0pt}%
\pgfpathmoveto{\pgfqpoint{2.096201in}{0.613486in}}%
\pgfpathlineto{\pgfqpoint{2.106206in}{0.613486in}}%
\pgfpathlineto{\pgfqpoint{2.106206in}{1.612282in}}%
\pgfpathlineto{\pgfqpoint{2.096201in}{1.612282in}}%
\pgfpathlineto{\pgfqpoint{2.096201in}{0.613486in}}%
\pgfpathclose%
\pgfusepath{fill}%
\end{pgfscope}%
\begin{pgfscope}%
\pgfpathrectangle{\pgfqpoint{0.693757in}{0.613486in}}{\pgfqpoint{5.541243in}{3.963477in}}%
\pgfusepath{clip}%
\pgfsetbuttcap%
\pgfsetmiterjoin%
\definecolor{currentfill}{rgb}{0.000000,0.000000,1.000000}%
\pgfsetfillcolor{currentfill}%
\pgfsetlinewidth{0.000000pt}%
\definecolor{currentstroke}{rgb}{0.000000,0.000000,0.000000}%
\pgfsetstrokecolor{currentstroke}%
\pgfsetstrokeopacity{0.000000}%
\pgfsetdash{}{0pt}%
\pgfpathmoveto{\pgfqpoint{2.108707in}{0.613486in}}%
\pgfpathlineto{\pgfqpoint{2.118712in}{0.613486in}}%
\pgfpathlineto{\pgfqpoint{2.118712in}{2.595224in}}%
\pgfpathlineto{\pgfqpoint{2.108707in}{2.595224in}}%
\pgfpathlineto{\pgfqpoint{2.108707in}{0.613486in}}%
\pgfpathclose%
\pgfusepath{fill}%
\end{pgfscope}%
\begin{pgfscope}%
\pgfpathrectangle{\pgfqpoint{0.693757in}{0.613486in}}{\pgfqpoint{5.541243in}{3.963477in}}%
\pgfusepath{clip}%
\pgfsetbuttcap%
\pgfsetmiterjoin%
\definecolor{currentfill}{rgb}{0.000000,0.000000,1.000000}%
\pgfsetfillcolor{currentfill}%
\pgfsetlinewidth{0.000000pt}%
\definecolor{currentstroke}{rgb}{0.000000,0.000000,0.000000}%
\pgfsetstrokecolor{currentstroke}%
\pgfsetstrokeopacity{0.000000}%
\pgfsetdash{}{0pt}%
\pgfpathmoveto{\pgfqpoint{2.121214in}{0.613486in}}%
\pgfpathlineto{\pgfqpoint{2.131219in}{0.613486in}}%
\pgfpathlineto{\pgfqpoint{2.131219in}{2.611074in}}%
\pgfpathlineto{\pgfqpoint{2.121214in}{2.611074in}}%
\pgfpathlineto{\pgfqpoint{2.121214in}{0.613486in}}%
\pgfpathclose%
\pgfusepath{fill}%
\end{pgfscope}%
\begin{pgfscope}%
\pgfpathrectangle{\pgfqpoint{0.693757in}{0.613486in}}{\pgfqpoint{5.541243in}{3.963477in}}%
\pgfusepath{clip}%
\pgfsetbuttcap%
\pgfsetmiterjoin%
\definecolor{currentfill}{rgb}{0.000000,0.000000,1.000000}%
\pgfsetfillcolor{currentfill}%
\pgfsetlinewidth{0.000000pt}%
\definecolor{currentstroke}{rgb}{0.000000,0.000000,0.000000}%
\pgfsetstrokecolor{currentstroke}%
\pgfsetstrokeopacity{0.000000}%
\pgfsetdash{}{0pt}%
\pgfpathmoveto{\pgfqpoint{2.133720in}{0.613486in}}%
\pgfpathlineto{\pgfqpoint{2.143725in}{0.613486in}}%
\pgfpathlineto{\pgfqpoint{2.143725in}{1.604355in}}%
\pgfpathlineto{\pgfqpoint{2.133720in}{1.604355in}}%
\pgfpathlineto{\pgfqpoint{2.133720in}{0.613486in}}%
\pgfpathclose%
\pgfusepath{fill}%
\end{pgfscope}%
\begin{pgfscope}%
\pgfpathrectangle{\pgfqpoint{0.693757in}{0.613486in}}{\pgfqpoint{5.541243in}{3.963477in}}%
\pgfusepath{clip}%
\pgfsetbuttcap%
\pgfsetmiterjoin%
\definecolor{currentfill}{rgb}{0.000000,0.000000,1.000000}%
\pgfsetfillcolor{currentfill}%
\pgfsetlinewidth{0.000000pt}%
\definecolor{currentstroke}{rgb}{0.000000,0.000000,0.000000}%
\pgfsetstrokecolor{currentstroke}%
\pgfsetstrokeopacity{0.000000}%
\pgfsetdash{}{0pt}%
\pgfpathmoveto{\pgfqpoint{2.146226in}{0.613486in}}%
\pgfpathlineto{\pgfqpoint{2.156231in}{0.613486in}}%
\pgfpathlineto{\pgfqpoint{2.156231in}{1.612282in}}%
\pgfpathlineto{\pgfqpoint{2.146226in}{1.612282in}}%
\pgfpathlineto{\pgfqpoint{2.146226in}{0.613486in}}%
\pgfpathclose%
\pgfusepath{fill}%
\end{pgfscope}%
\begin{pgfscope}%
\pgfpathrectangle{\pgfqpoint{0.693757in}{0.613486in}}{\pgfqpoint{5.541243in}{3.963477in}}%
\pgfusepath{clip}%
\pgfsetbuttcap%
\pgfsetmiterjoin%
\definecolor{currentfill}{rgb}{0.000000,0.000000,1.000000}%
\pgfsetfillcolor{currentfill}%
\pgfsetlinewidth{0.000000pt}%
\definecolor{currentstroke}{rgb}{0.000000,0.000000,0.000000}%
\pgfsetstrokecolor{currentstroke}%
\pgfsetstrokeopacity{0.000000}%
\pgfsetdash{}{0pt}%
\pgfpathmoveto{\pgfqpoint{2.158732in}{0.613486in}}%
\pgfpathlineto{\pgfqpoint{2.168737in}{0.613486in}}%
\pgfpathlineto{\pgfqpoint{2.168737in}{2.595224in}}%
\pgfpathlineto{\pgfqpoint{2.158732in}{2.595224in}}%
\pgfpathlineto{\pgfqpoint{2.158732in}{0.613486in}}%
\pgfpathclose%
\pgfusepath{fill}%
\end{pgfscope}%
\begin{pgfscope}%
\pgfpathrectangle{\pgfqpoint{0.693757in}{0.613486in}}{\pgfqpoint{5.541243in}{3.963477in}}%
\pgfusepath{clip}%
\pgfsetbuttcap%
\pgfsetmiterjoin%
\definecolor{currentfill}{rgb}{0.000000,0.000000,1.000000}%
\pgfsetfillcolor{currentfill}%
\pgfsetlinewidth{0.000000pt}%
\definecolor{currentstroke}{rgb}{0.000000,0.000000,0.000000}%
\pgfsetstrokecolor{currentstroke}%
\pgfsetstrokeopacity{0.000000}%
\pgfsetdash{}{0pt}%
\pgfpathmoveto{\pgfqpoint{2.171238in}{0.613486in}}%
\pgfpathlineto{\pgfqpoint{2.181243in}{0.613486in}}%
\pgfpathlineto{\pgfqpoint{2.181243in}{2.611074in}}%
\pgfpathlineto{\pgfqpoint{2.171238in}{2.611074in}}%
\pgfpathlineto{\pgfqpoint{2.171238in}{0.613486in}}%
\pgfpathclose%
\pgfusepath{fill}%
\end{pgfscope}%
\begin{pgfscope}%
\pgfpathrectangle{\pgfqpoint{0.693757in}{0.613486in}}{\pgfqpoint{5.541243in}{3.963477in}}%
\pgfusepath{clip}%
\pgfsetbuttcap%
\pgfsetmiterjoin%
\definecolor{currentfill}{rgb}{0.000000,0.000000,1.000000}%
\pgfsetfillcolor{currentfill}%
\pgfsetlinewidth{0.000000pt}%
\definecolor{currentstroke}{rgb}{0.000000,0.000000,0.000000}%
\pgfsetstrokecolor{currentstroke}%
\pgfsetstrokeopacity{0.000000}%
\pgfsetdash{}{0pt}%
\pgfpathmoveto{\pgfqpoint{2.183745in}{0.613486in}}%
\pgfpathlineto{\pgfqpoint{2.193750in}{0.613486in}}%
\pgfpathlineto{\pgfqpoint{2.193750in}{1.604355in}}%
\pgfpathlineto{\pgfqpoint{2.183745in}{1.604355in}}%
\pgfpathlineto{\pgfqpoint{2.183745in}{0.613486in}}%
\pgfpathclose%
\pgfusepath{fill}%
\end{pgfscope}%
\begin{pgfscope}%
\pgfpathrectangle{\pgfqpoint{0.693757in}{0.613486in}}{\pgfqpoint{5.541243in}{3.963477in}}%
\pgfusepath{clip}%
\pgfsetbuttcap%
\pgfsetmiterjoin%
\definecolor{currentfill}{rgb}{0.000000,0.000000,1.000000}%
\pgfsetfillcolor{currentfill}%
\pgfsetlinewidth{0.000000pt}%
\definecolor{currentstroke}{rgb}{0.000000,0.000000,0.000000}%
\pgfsetstrokecolor{currentstroke}%
\pgfsetstrokeopacity{0.000000}%
\pgfsetdash{}{0pt}%
\pgfpathmoveto{\pgfqpoint{2.196251in}{0.613486in}}%
\pgfpathlineto{\pgfqpoint{2.206256in}{0.613486in}}%
\pgfpathlineto{\pgfqpoint{2.206256in}{1.612282in}}%
\pgfpathlineto{\pgfqpoint{2.196251in}{1.612282in}}%
\pgfpathlineto{\pgfqpoint{2.196251in}{0.613486in}}%
\pgfpathclose%
\pgfusepath{fill}%
\end{pgfscope}%
\begin{pgfscope}%
\pgfpathrectangle{\pgfqpoint{0.693757in}{0.613486in}}{\pgfqpoint{5.541243in}{3.963477in}}%
\pgfusepath{clip}%
\pgfsetbuttcap%
\pgfsetmiterjoin%
\definecolor{currentfill}{rgb}{0.000000,0.000000,1.000000}%
\pgfsetfillcolor{currentfill}%
\pgfsetlinewidth{0.000000pt}%
\definecolor{currentstroke}{rgb}{0.000000,0.000000,0.000000}%
\pgfsetstrokecolor{currentstroke}%
\pgfsetstrokeopacity{0.000000}%
\pgfsetdash{}{0pt}%
\pgfpathmoveto{\pgfqpoint{2.208757in}{0.613486in}}%
\pgfpathlineto{\pgfqpoint{2.218762in}{0.613486in}}%
\pgfpathlineto{\pgfqpoint{2.218762in}{2.595224in}}%
\pgfpathlineto{\pgfqpoint{2.208757in}{2.595224in}}%
\pgfpathlineto{\pgfqpoint{2.208757in}{0.613486in}}%
\pgfpathclose%
\pgfusepath{fill}%
\end{pgfscope}%
\begin{pgfscope}%
\pgfpathrectangle{\pgfqpoint{0.693757in}{0.613486in}}{\pgfqpoint{5.541243in}{3.963477in}}%
\pgfusepath{clip}%
\pgfsetbuttcap%
\pgfsetmiterjoin%
\definecolor{currentfill}{rgb}{0.000000,0.000000,1.000000}%
\pgfsetfillcolor{currentfill}%
\pgfsetlinewidth{0.000000pt}%
\definecolor{currentstroke}{rgb}{0.000000,0.000000,0.000000}%
\pgfsetstrokecolor{currentstroke}%
\pgfsetstrokeopacity{0.000000}%
\pgfsetdash{}{0pt}%
\pgfpathmoveto{\pgfqpoint{2.221263in}{0.613486in}}%
\pgfpathlineto{\pgfqpoint{2.231268in}{0.613486in}}%
\pgfpathlineto{\pgfqpoint{2.231268in}{2.611074in}}%
\pgfpathlineto{\pgfqpoint{2.221263in}{2.611074in}}%
\pgfpathlineto{\pgfqpoint{2.221263in}{0.613486in}}%
\pgfpathclose%
\pgfusepath{fill}%
\end{pgfscope}%
\begin{pgfscope}%
\pgfpathrectangle{\pgfqpoint{0.693757in}{0.613486in}}{\pgfqpoint{5.541243in}{3.963477in}}%
\pgfusepath{clip}%
\pgfsetbuttcap%
\pgfsetmiterjoin%
\definecolor{currentfill}{rgb}{0.000000,0.000000,1.000000}%
\pgfsetfillcolor{currentfill}%
\pgfsetlinewidth{0.000000pt}%
\definecolor{currentstroke}{rgb}{0.000000,0.000000,0.000000}%
\pgfsetstrokecolor{currentstroke}%
\pgfsetstrokeopacity{0.000000}%
\pgfsetdash{}{0pt}%
\pgfpathmoveto{\pgfqpoint{2.233769in}{0.613486in}}%
\pgfpathlineto{\pgfqpoint{2.243774in}{0.613486in}}%
\pgfpathlineto{\pgfqpoint{2.243774in}{1.604355in}}%
\pgfpathlineto{\pgfqpoint{2.233769in}{1.604355in}}%
\pgfpathlineto{\pgfqpoint{2.233769in}{0.613486in}}%
\pgfpathclose%
\pgfusepath{fill}%
\end{pgfscope}%
\begin{pgfscope}%
\pgfpathrectangle{\pgfqpoint{0.693757in}{0.613486in}}{\pgfqpoint{5.541243in}{3.963477in}}%
\pgfusepath{clip}%
\pgfsetbuttcap%
\pgfsetmiterjoin%
\definecolor{currentfill}{rgb}{0.000000,0.000000,1.000000}%
\pgfsetfillcolor{currentfill}%
\pgfsetlinewidth{0.000000pt}%
\definecolor{currentstroke}{rgb}{0.000000,0.000000,0.000000}%
\pgfsetstrokecolor{currentstroke}%
\pgfsetstrokeopacity{0.000000}%
\pgfsetdash{}{0pt}%
\pgfpathmoveto{\pgfqpoint{2.246276in}{0.613486in}}%
\pgfpathlineto{\pgfqpoint{2.256281in}{0.613486in}}%
\pgfpathlineto{\pgfqpoint{2.256281in}{1.612282in}}%
\pgfpathlineto{\pgfqpoint{2.246276in}{1.612282in}}%
\pgfpathlineto{\pgfqpoint{2.246276in}{0.613486in}}%
\pgfpathclose%
\pgfusepath{fill}%
\end{pgfscope}%
\begin{pgfscope}%
\pgfpathrectangle{\pgfqpoint{0.693757in}{0.613486in}}{\pgfqpoint{5.541243in}{3.963477in}}%
\pgfusepath{clip}%
\pgfsetbuttcap%
\pgfsetmiterjoin%
\definecolor{currentfill}{rgb}{0.000000,0.000000,1.000000}%
\pgfsetfillcolor{currentfill}%
\pgfsetlinewidth{0.000000pt}%
\definecolor{currentstroke}{rgb}{0.000000,0.000000,0.000000}%
\pgfsetstrokecolor{currentstroke}%
\pgfsetstrokeopacity{0.000000}%
\pgfsetdash{}{0pt}%
\pgfpathmoveto{\pgfqpoint{2.258782in}{0.613486in}}%
\pgfpathlineto{\pgfqpoint{2.268787in}{0.613486in}}%
\pgfpathlineto{\pgfqpoint{2.268787in}{2.595224in}}%
\pgfpathlineto{\pgfqpoint{2.258782in}{2.595224in}}%
\pgfpathlineto{\pgfqpoint{2.258782in}{0.613486in}}%
\pgfpathclose%
\pgfusepath{fill}%
\end{pgfscope}%
\begin{pgfscope}%
\pgfpathrectangle{\pgfqpoint{0.693757in}{0.613486in}}{\pgfqpoint{5.541243in}{3.963477in}}%
\pgfusepath{clip}%
\pgfsetbuttcap%
\pgfsetmiterjoin%
\definecolor{currentfill}{rgb}{0.000000,0.000000,1.000000}%
\pgfsetfillcolor{currentfill}%
\pgfsetlinewidth{0.000000pt}%
\definecolor{currentstroke}{rgb}{0.000000,0.000000,0.000000}%
\pgfsetstrokecolor{currentstroke}%
\pgfsetstrokeopacity{0.000000}%
\pgfsetdash{}{0pt}%
\pgfpathmoveto{\pgfqpoint{2.271288in}{0.613486in}}%
\pgfpathlineto{\pgfqpoint{2.281293in}{0.613486in}}%
\pgfpathlineto{\pgfqpoint{2.281293in}{2.611074in}}%
\pgfpathlineto{\pgfqpoint{2.271288in}{2.611074in}}%
\pgfpathlineto{\pgfqpoint{2.271288in}{0.613486in}}%
\pgfpathclose%
\pgfusepath{fill}%
\end{pgfscope}%
\begin{pgfscope}%
\pgfpathrectangle{\pgfqpoint{0.693757in}{0.613486in}}{\pgfqpoint{5.541243in}{3.963477in}}%
\pgfusepath{clip}%
\pgfsetbuttcap%
\pgfsetmiterjoin%
\definecolor{currentfill}{rgb}{0.000000,0.000000,1.000000}%
\pgfsetfillcolor{currentfill}%
\pgfsetlinewidth{0.000000pt}%
\definecolor{currentstroke}{rgb}{0.000000,0.000000,0.000000}%
\pgfsetstrokecolor{currentstroke}%
\pgfsetstrokeopacity{0.000000}%
\pgfsetdash{}{0pt}%
\pgfpathmoveto{\pgfqpoint{2.283794in}{0.613486in}}%
\pgfpathlineto{\pgfqpoint{2.293799in}{0.613486in}}%
\pgfpathlineto{\pgfqpoint{2.293799in}{1.604355in}}%
\pgfpathlineto{\pgfqpoint{2.283794in}{1.604355in}}%
\pgfpathlineto{\pgfqpoint{2.283794in}{0.613486in}}%
\pgfpathclose%
\pgfusepath{fill}%
\end{pgfscope}%
\begin{pgfscope}%
\pgfpathrectangle{\pgfqpoint{0.693757in}{0.613486in}}{\pgfqpoint{5.541243in}{3.963477in}}%
\pgfusepath{clip}%
\pgfsetbuttcap%
\pgfsetmiterjoin%
\definecolor{currentfill}{rgb}{0.000000,0.000000,1.000000}%
\pgfsetfillcolor{currentfill}%
\pgfsetlinewidth{0.000000pt}%
\definecolor{currentstroke}{rgb}{0.000000,0.000000,0.000000}%
\pgfsetstrokecolor{currentstroke}%
\pgfsetstrokeopacity{0.000000}%
\pgfsetdash{}{0pt}%
\pgfpathmoveto{\pgfqpoint{2.296300in}{0.613486in}}%
\pgfpathlineto{\pgfqpoint{2.306305in}{0.613486in}}%
\pgfpathlineto{\pgfqpoint{2.306305in}{1.612282in}}%
\pgfpathlineto{\pgfqpoint{2.296300in}{1.612282in}}%
\pgfpathlineto{\pgfqpoint{2.296300in}{0.613486in}}%
\pgfpathclose%
\pgfusepath{fill}%
\end{pgfscope}%
\begin{pgfscope}%
\pgfpathrectangle{\pgfqpoint{0.693757in}{0.613486in}}{\pgfqpoint{5.541243in}{3.963477in}}%
\pgfusepath{clip}%
\pgfsetbuttcap%
\pgfsetmiterjoin%
\definecolor{currentfill}{rgb}{0.000000,0.000000,1.000000}%
\pgfsetfillcolor{currentfill}%
\pgfsetlinewidth{0.000000pt}%
\definecolor{currentstroke}{rgb}{0.000000,0.000000,0.000000}%
\pgfsetstrokecolor{currentstroke}%
\pgfsetstrokeopacity{0.000000}%
\pgfsetdash{}{0pt}%
\pgfpathmoveto{\pgfqpoint{2.308807in}{0.613486in}}%
\pgfpathlineto{\pgfqpoint{2.318811in}{0.613486in}}%
\pgfpathlineto{\pgfqpoint{2.318811in}{2.595224in}}%
\pgfpathlineto{\pgfqpoint{2.308807in}{2.595224in}}%
\pgfpathlineto{\pgfqpoint{2.308807in}{0.613486in}}%
\pgfpathclose%
\pgfusepath{fill}%
\end{pgfscope}%
\begin{pgfscope}%
\pgfpathrectangle{\pgfqpoint{0.693757in}{0.613486in}}{\pgfqpoint{5.541243in}{3.963477in}}%
\pgfusepath{clip}%
\pgfsetbuttcap%
\pgfsetmiterjoin%
\definecolor{currentfill}{rgb}{0.000000,0.000000,1.000000}%
\pgfsetfillcolor{currentfill}%
\pgfsetlinewidth{0.000000pt}%
\definecolor{currentstroke}{rgb}{0.000000,0.000000,0.000000}%
\pgfsetstrokecolor{currentstroke}%
\pgfsetstrokeopacity{0.000000}%
\pgfsetdash{}{0pt}%
\pgfpathmoveto{\pgfqpoint{2.321313in}{0.613486in}}%
\pgfpathlineto{\pgfqpoint{2.331318in}{0.613486in}}%
\pgfpathlineto{\pgfqpoint{2.331318in}{2.611074in}}%
\pgfpathlineto{\pgfqpoint{2.321313in}{2.611074in}}%
\pgfpathlineto{\pgfqpoint{2.321313in}{0.613486in}}%
\pgfpathclose%
\pgfusepath{fill}%
\end{pgfscope}%
\begin{pgfscope}%
\pgfpathrectangle{\pgfqpoint{0.693757in}{0.613486in}}{\pgfqpoint{5.541243in}{3.963477in}}%
\pgfusepath{clip}%
\pgfsetbuttcap%
\pgfsetmiterjoin%
\definecolor{currentfill}{rgb}{0.000000,0.000000,1.000000}%
\pgfsetfillcolor{currentfill}%
\pgfsetlinewidth{0.000000pt}%
\definecolor{currentstroke}{rgb}{0.000000,0.000000,0.000000}%
\pgfsetstrokecolor{currentstroke}%
\pgfsetstrokeopacity{0.000000}%
\pgfsetdash{}{0pt}%
\pgfpathmoveto{\pgfqpoint{2.333819in}{0.613486in}}%
\pgfpathlineto{\pgfqpoint{2.343824in}{0.613486in}}%
\pgfpathlineto{\pgfqpoint{2.343824in}{1.604355in}}%
\pgfpathlineto{\pgfqpoint{2.333819in}{1.604355in}}%
\pgfpathlineto{\pgfqpoint{2.333819in}{0.613486in}}%
\pgfpathclose%
\pgfusepath{fill}%
\end{pgfscope}%
\begin{pgfscope}%
\pgfpathrectangle{\pgfqpoint{0.693757in}{0.613486in}}{\pgfqpoint{5.541243in}{3.963477in}}%
\pgfusepath{clip}%
\pgfsetbuttcap%
\pgfsetmiterjoin%
\definecolor{currentfill}{rgb}{0.000000,0.000000,1.000000}%
\pgfsetfillcolor{currentfill}%
\pgfsetlinewidth{0.000000pt}%
\definecolor{currentstroke}{rgb}{0.000000,0.000000,0.000000}%
\pgfsetstrokecolor{currentstroke}%
\pgfsetstrokeopacity{0.000000}%
\pgfsetdash{}{0pt}%
\pgfpathmoveto{\pgfqpoint{2.346325in}{0.613486in}}%
\pgfpathlineto{\pgfqpoint{2.356330in}{0.613486in}}%
\pgfpathlineto{\pgfqpoint{2.356330in}{1.612282in}}%
\pgfpathlineto{\pgfqpoint{2.346325in}{1.612282in}}%
\pgfpathlineto{\pgfqpoint{2.346325in}{0.613486in}}%
\pgfpathclose%
\pgfusepath{fill}%
\end{pgfscope}%
\begin{pgfscope}%
\pgfpathrectangle{\pgfqpoint{0.693757in}{0.613486in}}{\pgfqpoint{5.541243in}{3.963477in}}%
\pgfusepath{clip}%
\pgfsetbuttcap%
\pgfsetmiterjoin%
\definecolor{currentfill}{rgb}{0.000000,0.000000,1.000000}%
\pgfsetfillcolor{currentfill}%
\pgfsetlinewidth{0.000000pt}%
\definecolor{currentstroke}{rgb}{0.000000,0.000000,0.000000}%
\pgfsetstrokecolor{currentstroke}%
\pgfsetstrokeopacity{0.000000}%
\pgfsetdash{}{0pt}%
\pgfpathmoveto{\pgfqpoint{2.358831in}{0.613486in}}%
\pgfpathlineto{\pgfqpoint{2.368836in}{0.613486in}}%
\pgfpathlineto{\pgfqpoint{2.368836in}{2.595224in}}%
\pgfpathlineto{\pgfqpoint{2.358831in}{2.595224in}}%
\pgfpathlineto{\pgfqpoint{2.358831in}{0.613486in}}%
\pgfpathclose%
\pgfusepath{fill}%
\end{pgfscope}%
\begin{pgfscope}%
\pgfpathrectangle{\pgfqpoint{0.693757in}{0.613486in}}{\pgfqpoint{5.541243in}{3.963477in}}%
\pgfusepath{clip}%
\pgfsetbuttcap%
\pgfsetmiterjoin%
\definecolor{currentfill}{rgb}{0.000000,0.000000,1.000000}%
\pgfsetfillcolor{currentfill}%
\pgfsetlinewidth{0.000000pt}%
\definecolor{currentstroke}{rgb}{0.000000,0.000000,0.000000}%
\pgfsetstrokecolor{currentstroke}%
\pgfsetstrokeopacity{0.000000}%
\pgfsetdash{}{0pt}%
\pgfpathmoveto{\pgfqpoint{2.371337in}{0.613486in}}%
\pgfpathlineto{\pgfqpoint{2.381342in}{0.613486in}}%
\pgfpathlineto{\pgfqpoint{2.381342in}{2.611074in}}%
\pgfpathlineto{\pgfqpoint{2.371337in}{2.611074in}}%
\pgfpathlineto{\pgfqpoint{2.371337in}{0.613486in}}%
\pgfpathclose%
\pgfusepath{fill}%
\end{pgfscope}%
\begin{pgfscope}%
\pgfpathrectangle{\pgfqpoint{0.693757in}{0.613486in}}{\pgfqpoint{5.541243in}{3.963477in}}%
\pgfusepath{clip}%
\pgfsetbuttcap%
\pgfsetmiterjoin%
\definecolor{currentfill}{rgb}{0.000000,0.000000,1.000000}%
\pgfsetfillcolor{currentfill}%
\pgfsetlinewidth{0.000000pt}%
\definecolor{currentstroke}{rgb}{0.000000,0.000000,0.000000}%
\pgfsetstrokecolor{currentstroke}%
\pgfsetstrokeopacity{0.000000}%
\pgfsetdash{}{0pt}%
\pgfpathmoveto{\pgfqpoint{2.383844in}{0.613486in}}%
\pgfpathlineto{\pgfqpoint{2.393849in}{0.613486in}}%
\pgfpathlineto{\pgfqpoint{2.393849in}{1.604355in}}%
\pgfpathlineto{\pgfqpoint{2.383844in}{1.604355in}}%
\pgfpathlineto{\pgfqpoint{2.383844in}{0.613486in}}%
\pgfpathclose%
\pgfusepath{fill}%
\end{pgfscope}%
\begin{pgfscope}%
\pgfpathrectangle{\pgfqpoint{0.693757in}{0.613486in}}{\pgfqpoint{5.541243in}{3.963477in}}%
\pgfusepath{clip}%
\pgfsetbuttcap%
\pgfsetmiterjoin%
\definecolor{currentfill}{rgb}{0.000000,0.000000,1.000000}%
\pgfsetfillcolor{currentfill}%
\pgfsetlinewidth{0.000000pt}%
\definecolor{currentstroke}{rgb}{0.000000,0.000000,0.000000}%
\pgfsetstrokecolor{currentstroke}%
\pgfsetstrokeopacity{0.000000}%
\pgfsetdash{}{0pt}%
\pgfpathmoveto{\pgfqpoint{2.396350in}{0.613486in}}%
\pgfpathlineto{\pgfqpoint{2.406355in}{0.613486in}}%
\pgfpathlineto{\pgfqpoint{2.406355in}{1.612282in}}%
\pgfpathlineto{\pgfqpoint{2.396350in}{1.612282in}}%
\pgfpathlineto{\pgfqpoint{2.396350in}{0.613486in}}%
\pgfpathclose%
\pgfusepath{fill}%
\end{pgfscope}%
\begin{pgfscope}%
\pgfpathrectangle{\pgfqpoint{0.693757in}{0.613486in}}{\pgfqpoint{5.541243in}{3.963477in}}%
\pgfusepath{clip}%
\pgfsetbuttcap%
\pgfsetmiterjoin%
\definecolor{currentfill}{rgb}{0.000000,0.000000,1.000000}%
\pgfsetfillcolor{currentfill}%
\pgfsetlinewidth{0.000000pt}%
\definecolor{currentstroke}{rgb}{0.000000,0.000000,0.000000}%
\pgfsetstrokecolor{currentstroke}%
\pgfsetstrokeopacity{0.000000}%
\pgfsetdash{}{0pt}%
\pgfpathmoveto{\pgfqpoint{2.408856in}{0.613486in}}%
\pgfpathlineto{\pgfqpoint{2.418861in}{0.613486in}}%
\pgfpathlineto{\pgfqpoint{2.418861in}{2.595224in}}%
\pgfpathlineto{\pgfqpoint{2.408856in}{2.595224in}}%
\pgfpathlineto{\pgfqpoint{2.408856in}{0.613486in}}%
\pgfpathclose%
\pgfusepath{fill}%
\end{pgfscope}%
\begin{pgfscope}%
\pgfpathrectangle{\pgfqpoint{0.693757in}{0.613486in}}{\pgfqpoint{5.541243in}{3.963477in}}%
\pgfusepath{clip}%
\pgfsetbuttcap%
\pgfsetmiterjoin%
\definecolor{currentfill}{rgb}{0.000000,0.000000,1.000000}%
\pgfsetfillcolor{currentfill}%
\pgfsetlinewidth{0.000000pt}%
\definecolor{currentstroke}{rgb}{0.000000,0.000000,0.000000}%
\pgfsetstrokecolor{currentstroke}%
\pgfsetstrokeopacity{0.000000}%
\pgfsetdash{}{0pt}%
\pgfpathmoveto{\pgfqpoint{2.421362in}{0.613486in}}%
\pgfpathlineto{\pgfqpoint{2.431367in}{0.613486in}}%
\pgfpathlineto{\pgfqpoint{2.431367in}{2.611074in}}%
\pgfpathlineto{\pgfqpoint{2.421362in}{2.611074in}}%
\pgfpathlineto{\pgfqpoint{2.421362in}{0.613486in}}%
\pgfpathclose%
\pgfusepath{fill}%
\end{pgfscope}%
\begin{pgfscope}%
\pgfpathrectangle{\pgfqpoint{0.693757in}{0.613486in}}{\pgfqpoint{5.541243in}{3.963477in}}%
\pgfusepath{clip}%
\pgfsetbuttcap%
\pgfsetmiterjoin%
\definecolor{currentfill}{rgb}{0.000000,0.000000,1.000000}%
\pgfsetfillcolor{currentfill}%
\pgfsetlinewidth{0.000000pt}%
\definecolor{currentstroke}{rgb}{0.000000,0.000000,0.000000}%
\pgfsetstrokecolor{currentstroke}%
\pgfsetstrokeopacity{0.000000}%
\pgfsetdash{}{0pt}%
\pgfpathmoveto{\pgfqpoint{2.433868in}{0.613486in}}%
\pgfpathlineto{\pgfqpoint{2.443873in}{0.613486in}}%
\pgfpathlineto{\pgfqpoint{2.443873in}{1.604355in}}%
\pgfpathlineto{\pgfqpoint{2.433868in}{1.604355in}}%
\pgfpathlineto{\pgfqpoint{2.433868in}{0.613486in}}%
\pgfpathclose%
\pgfusepath{fill}%
\end{pgfscope}%
\begin{pgfscope}%
\pgfpathrectangle{\pgfqpoint{0.693757in}{0.613486in}}{\pgfqpoint{5.541243in}{3.963477in}}%
\pgfusepath{clip}%
\pgfsetbuttcap%
\pgfsetmiterjoin%
\definecolor{currentfill}{rgb}{0.000000,0.000000,1.000000}%
\pgfsetfillcolor{currentfill}%
\pgfsetlinewidth{0.000000pt}%
\definecolor{currentstroke}{rgb}{0.000000,0.000000,0.000000}%
\pgfsetstrokecolor{currentstroke}%
\pgfsetstrokeopacity{0.000000}%
\pgfsetdash{}{0pt}%
\pgfpathmoveto{\pgfqpoint{2.446375in}{0.613486in}}%
\pgfpathlineto{\pgfqpoint{2.456380in}{0.613486in}}%
\pgfpathlineto{\pgfqpoint{2.456380in}{1.612282in}}%
\pgfpathlineto{\pgfqpoint{2.446375in}{1.612282in}}%
\pgfpathlineto{\pgfqpoint{2.446375in}{0.613486in}}%
\pgfpathclose%
\pgfusepath{fill}%
\end{pgfscope}%
\begin{pgfscope}%
\pgfpathrectangle{\pgfqpoint{0.693757in}{0.613486in}}{\pgfqpoint{5.541243in}{3.963477in}}%
\pgfusepath{clip}%
\pgfsetbuttcap%
\pgfsetmiterjoin%
\definecolor{currentfill}{rgb}{0.000000,0.000000,1.000000}%
\pgfsetfillcolor{currentfill}%
\pgfsetlinewidth{0.000000pt}%
\definecolor{currentstroke}{rgb}{0.000000,0.000000,0.000000}%
\pgfsetstrokecolor{currentstroke}%
\pgfsetstrokeopacity{0.000000}%
\pgfsetdash{}{0pt}%
\pgfpathmoveto{\pgfqpoint{2.458881in}{0.613486in}}%
\pgfpathlineto{\pgfqpoint{2.468886in}{0.613486in}}%
\pgfpathlineto{\pgfqpoint{2.468886in}{2.595224in}}%
\pgfpathlineto{\pgfqpoint{2.458881in}{2.595224in}}%
\pgfpathlineto{\pgfqpoint{2.458881in}{0.613486in}}%
\pgfpathclose%
\pgfusepath{fill}%
\end{pgfscope}%
\begin{pgfscope}%
\pgfpathrectangle{\pgfqpoint{0.693757in}{0.613486in}}{\pgfqpoint{5.541243in}{3.963477in}}%
\pgfusepath{clip}%
\pgfsetbuttcap%
\pgfsetmiterjoin%
\definecolor{currentfill}{rgb}{0.000000,0.000000,1.000000}%
\pgfsetfillcolor{currentfill}%
\pgfsetlinewidth{0.000000pt}%
\definecolor{currentstroke}{rgb}{0.000000,0.000000,0.000000}%
\pgfsetstrokecolor{currentstroke}%
\pgfsetstrokeopacity{0.000000}%
\pgfsetdash{}{0pt}%
\pgfpathmoveto{\pgfqpoint{2.471387in}{0.613486in}}%
\pgfpathlineto{\pgfqpoint{2.481392in}{0.613486in}}%
\pgfpathlineto{\pgfqpoint{2.481392in}{2.611074in}}%
\pgfpathlineto{\pgfqpoint{2.471387in}{2.611074in}}%
\pgfpathlineto{\pgfqpoint{2.471387in}{0.613486in}}%
\pgfpathclose%
\pgfusepath{fill}%
\end{pgfscope}%
\begin{pgfscope}%
\pgfpathrectangle{\pgfqpoint{0.693757in}{0.613486in}}{\pgfqpoint{5.541243in}{3.963477in}}%
\pgfusepath{clip}%
\pgfsetbuttcap%
\pgfsetmiterjoin%
\definecolor{currentfill}{rgb}{0.000000,0.000000,1.000000}%
\pgfsetfillcolor{currentfill}%
\pgfsetlinewidth{0.000000pt}%
\definecolor{currentstroke}{rgb}{0.000000,0.000000,0.000000}%
\pgfsetstrokecolor{currentstroke}%
\pgfsetstrokeopacity{0.000000}%
\pgfsetdash{}{0pt}%
\pgfpathmoveto{\pgfqpoint{2.483893in}{0.613486in}}%
\pgfpathlineto{\pgfqpoint{2.493898in}{0.613486in}}%
\pgfpathlineto{\pgfqpoint{2.493898in}{1.604355in}}%
\pgfpathlineto{\pgfqpoint{2.483893in}{1.604355in}}%
\pgfpathlineto{\pgfqpoint{2.483893in}{0.613486in}}%
\pgfpathclose%
\pgfusepath{fill}%
\end{pgfscope}%
\begin{pgfscope}%
\pgfpathrectangle{\pgfqpoint{0.693757in}{0.613486in}}{\pgfqpoint{5.541243in}{3.963477in}}%
\pgfusepath{clip}%
\pgfsetbuttcap%
\pgfsetmiterjoin%
\definecolor{currentfill}{rgb}{0.000000,0.000000,1.000000}%
\pgfsetfillcolor{currentfill}%
\pgfsetlinewidth{0.000000pt}%
\definecolor{currentstroke}{rgb}{0.000000,0.000000,0.000000}%
\pgfsetstrokecolor{currentstroke}%
\pgfsetstrokeopacity{0.000000}%
\pgfsetdash{}{0pt}%
\pgfpathmoveto{\pgfqpoint{2.496399in}{0.613486in}}%
\pgfpathlineto{\pgfqpoint{2.506404in}{0.613486in}}%
\pgfpathlineto{\pgfqpoint{2.506404in}{1.612282in}}%
\pgfpathlineto{\pgfqpoint{2.496399in}{1.612282in}}%
\pgfpathlineto{\pgfqpoint{2.496399in}{0.613486in}}%
\pgfpathclose%
\pgfusepath{fill}%
\end{pgfscope}%
\begin{pgfscope}%
\pgfpathrectangle{\pgfqpoint{0.693757in}{0.613486in}}{\pgfqpoint{5.541243in}{3.963477in}}%
\pgfusepath{clip}%
\pgfsetbuttcap%
\pgfsetmiterjoin%
\definecolor{currentfill}{rgb}{0.000000,0.000000,1.000000}%
\pgfsetfillcolor{currentfill}%
\pgfsetlinewidth{0.000000pt}%
\definecolor{currentstroke}{rgb}{0.000000,0.000000,0.000000}%
\pgfsetstrokecolor{currentstroke}%
\pgfsetstrokeopacity{0.000000}%
\pgfsetdash{}{0pt}%
\pgfpathmoveto{\pgfqpoint{2.508906in}{0.613486in}}%
\pgfpathlineto{\pgfqpoint{2.518911in}{0.613486in}}%
\pgfpathlineto{\pgfqpoint{2.518911in}{2.595224in}}%
\pgfpathlineto{\pgfqpoint{2.508906in}{2.595224in}}%
\pgfpathlineto{\pgfqpoint{2.508906in}{0.613486in}}%
\pgfpathclose%
\pgfusepath{fill}%
\end{pgfscope}%
\begin{pgfscope}%
\pgfpathrectangle{\pgfqpoint{0.693757in}{0.613486in}}{\pgfqpoint{5.541243in}{3.963477in}}%
\pgfusepath{clip}%
\pgfsetbuttcap%
\pgfsetmiterjoin%
\definecolor{currentfill}{rgb}{0.000000,0.000000,1.000000}%
\pgfsetfillcolor{currentfill}%
\pgfsetlinewidth{0.000000pt}%
\definecolor{currentstroke}{rgb}{0.000000,0.000000,0.000000}%
\pgfsetstrokecolor{currentstroke}%
\pgfsetstrokeopacity{0.000000}%
\pgfsetdash{}{0pt}%
\pgfpathmoveto{\pgfqpoint{2.521412in}{0.613486in}}%
\pgfpathlineto{\pgfqpoint{2.531417in}{0.613486in}}%
\pgfpathlineto{\pgfqpoint{2.531417in}{2.611074in}}%
\pgfpathlineto{\pgfqpoint{2.521412in}{2.611074in}}%
\pgfpathlineto{\pgfqpoint{2.521412in}{0.613486in}}%
\pgfpathclose%
\pgfusepath{fill}%
\end{pgfscope}%
\begin{pgfscope}%
\pgfpathrectangle{\pgfqpoint{0.693757in}{0.613486in}}{\pgfqpoint{5.541243in}{3.963477in}}%
\pgfusepath{clip}%
\pgfsetbuttcap%
\pgfsetmiterjoin%
\definecolor{currentfill}{rgb}{0.000000,0.000000,1.000000}%
\pgfsetfillcolor{currentfill}%
\pgfsetlinewidth{0.000000pt}%
\definecolor{currentstroke}{rgb}{0.000000,0.000000,0.000000}%
\pgfsetstrokecolor{currentstroke}%
\pgfsetstrokeopacity{0.000000}%
\pgfsetdash{}{0pt}%
\pgfpathmoveto{\pgfqpoint{2.533918in}{0.613486in}}%
\pgfpathlineto{\pgfqpoint{2.543923in}{0.613486in}}%
\pgfpathlineto{\pgfqpoint{2.543923in}{1.604355in}}%
\pgfpathlineto{\pgfqpoint{2.533918in}{1.604355in}}%
\pgfpathlineto{\pgfqpoint{2.533918in}{0.613486in}}%
\pgfpathclose%
\pgfusepath{fill}%
\end{pgfscope}%
\begin{pgfscope}%
\pgfpathrectangle{\pgfqpoint{0.693757in}{0.613486in}}{\pgfqpoint{5.541243in}{3.963477in}}%
\pgfusepath{clip}%
\pgfsetbuttcap%
\pgfsetmiterjoin%
\definecolor{currentfill}{rgb}{0.000000,0.000000,1.000000}%
\pgfsetfillcolor{currentfill}%
\pgfsetlinewidth{0.000000pt}%
\definecolor{currentstroke}{rgb}{0.000000,0.000000,0.000000}%
\pgfsetstrokecolor{currentstroke}%
\pgfsetstrokeopacity{0.000000}%
\pgfsetdash{}{0pt}%
\pgfpathmoveto{\pgfqpoint{2.546424in}{0.613486in}}%
\pgfpathlineto{\pgfqpoint{2.556429in}{0.613486in}}%
\pgfpathlineto{\pgfqpoint{2.556429in}{1.612282in}}%
\pgfpathlineto{\pgfqpoint{2.546424in}{1.612282in}}%
\pgfpathlineto{\pgfqpoint{2.546424in}{0.613486in}}%
\pgfpathclose%
\pgfusepath{fill}%
\end{pgfscope}%
\begin{pgfscope}%
\pgfpathrectangle{\pgfqpoint{0.693757in}{0.613486in}}{\pgfqpoint{5.541243in}{3.963477in}}%
\pgfusepath{clip}%
\pgfsetbuttcap%
\pgfsetmiterjoin%
\definecolor{currentfill}{rgb}{0.000000,0.000000,1.000000}%
\pgfsetfillcolor{currentfill}%
\pgfsetlinewidth{0.000000pt}%
\definecolor{currentstroke}{rgb}{0.000000,0.000000,0.000000}%
\pgfsetstrokecolor{currentstroke}%
\pgfsetstrokeopacity{0.000000}%
\pgfsetdash{}{0pt}%
\pgfpathmoveto{\pgfqpoint{2.558930in}{0.613486in}}%
\pgfpathlineto{\pgfqpoint{2.568935in}{0.613486in}}%
\pgfpathlineto{\pgfqpoint{2.568935in}{2.595224in}}%
\pgfpathlineto{\pgfqpoint{2.558930in}{2.595224in}}%
\pgfpathlineto{\pgfqpoint{2.558930in}{0.613486in}}%
\pgfpathclose%
\pgfusepath{fill}%
\end{pgfscope}%
\begin{pgfscope}%
\pgfpathrectangle{\pgfqpoint{0.693757in}{0.613486in}}{\pgfqpoint{5.541243in}{3.963477in}}%
\pgfusepath{clip}%
\pgfsetbuttcap%
\pgfsetmiterjoin%
\definecolor{currentfill}{rgb}{0.000000,0.000000,1.000000}%
\pgfsetfillcolor{currentfill}%
\pgfsetlinewidth{0.000000pt}%
\definecolor{currentstroke}{rgb}{0.000000,0.000000,0.000000}%
\pgfsetstrokecolor{currentstroke}%
\pgfsetstrokeopacity{0.000000}%
\pgfsetdash{}{0pt}%
\pgfpathmoveto{\pgfqpoint{2.571437in}{0.613486in}}%
\pgfpathlineto{\pgfqpoint{2.581441in}{0.613486in}}%
\pgfpathlineto{\pgfqpoint{2.581441in}{2.611074in}}%
\pgfpathlineto{\pgfqpoint{2.571437in}{2.611074in}}%
\pgfpathlineto{\pgfqpoint{2.571437in}{0.613486in}}%
\pgfpathclose%
\pgfusepath{fill}%
\end{pgfscope}%
\begin{pgfscope}%
\pgfpathrectangle{\pgfqpoint{0.693757in}{0.613486in}}{\pgfqpoint{5.541243in}{3.963477in}}%
\pgfusepath{clip}%
\pgfsetbuttcap%
\pgfsetmiterjoin%
\definecolor{currentfill}{rgb}{0.000000,0.000000,1.000000}%
\pgfsetfillcolor{currentfill}%
\pgfsetlinewidth{0.000000pt}%
\definecolor{currentstroke}{rgb}{0.000000,0.000000,0.000000}%
\pgfsetstrokecolor{currentstroke}%
\pgfsetstrokeopacity{0.000000}%
\pgfsetdash{}{0pt}%
\pgfpathmoveto{\pgfqpoint{2.583943in}{0.613486in}}%
\pgfpathlineto{\pgfqpoint{2.593948in}{0.613486in}}%
\pgfpathlineto{\pgfqpoint{2.593948in}{1.604355in}}%
\pgfpathlineto{\pgfqpoint{2.583943in}{1.604355in}}%
\pgfpathlineto{\pgfqpoint{2.583943in}{0.613486in}}%
\pgfpathclose%
\pgfusepath{fill}%
\end{pgfscope}%
\begin{pgfscope}%
\pgfpathrectangle{\pgfqpoint{0.693757in}{0.613486in}}{\pgfqpoint{5.541243in}{3.963477in}}%
\pgfusepath{clip}%
\pgfsetbuttcap%
\pgfsetmiterjoin%
\definecolor{currentfill}{rgb}{0.000000,0.000000,1.000000}%
\pgfsetfillcolor{currentfill}%
\pgfsetlinewidth{0.000000pt}%
\definecolor{currentstroke}{rgb}{0.000000,0.000000,0.000000}%
\pgfsetstrokecolor{currentstroke}%
\pgfsetstrokeopacity{0.000000}%
\pgfsetdash{}{0pt}%
\pgfpathmoveto{\pgfqpoint{2.596449in}{0.613486in}}%
\pgfpathlineto{\pgfqpoint{2.606454in}{0.613486in}}%
\pgfpathlineto{\pgfqpoint{2.606454in}{1.612282in}}%
\pgfpathlineto{\pgfqpoint{2.596449in}{1.612282in}}%
\pgfpathlineto{\pgfqpoint{2.596449in}{0.613486in}}%
\pgfpathclose%
\pgfusepath{fill}%
\end{pgfscope}%
\begin{pgfscope}%
\pgfpathrectangle{\pgfqpoint{0.693757in}{0.613486in}}{\pgfqpoint{5.541243in}{3.963477in}}%
\pgfusepath{clip}%
\pgfsetbuttcap%
\pgfsetmiterjoin%
\definecolor{currentfill}{rgb}{0.000000,0.000000,1.000000}%
\pgfsetfillcolor{currentfill}%
\pgfsetlinewidth{0.000000pt}%
\definecolor{currentstroke}{rgb}{0.000000,0.000000,0.000000}%
\pgfsetstrokecolor{currentstroke}%
\pgfsetstrokeopacity{0.000000}%
\pgfsetdash{}{0pt}%
\pgfpathmoveto{\pgfqpoint{2.608955in}{0.613486in}}%
\pgfpathlineto{\pgfqpoint{2.618960in}{0.613486in}}%
\pgfpathlineto{\pgfqpoint{2.618960in}{2.595224in}}%
\pgfpathlineto{\pgfqpoint{2.608955in}{2.595224in}}%
\pgfpathlineto{\pgfqpoint{2.608955in}{0.613486in}}%
\pgfpathclose%
\pgfusepath{fill}%
\end{pgfscope}%
\begin{pgfscope}%
\pgfpathrectangle{\pgfqpoint{0.693757in}{0.613486in}}{\pgfqpoint{5.541243in}{3.963477in}}%
\pgfusepath{clip}%
\pgfsetbuttcap%
\pgfsetmiterjoin%
\definecolor{currentfill}{rgb}{0.000000,0.000000,1.000000}%
\pgfsetfillcolor{currentfill}%
\pgfsetlinewidth{0.000000pt}%
\definecolor{currentstroke}{rgb}{0.000000,0.000000,0.000000}%
\pgfsetstrokecolor{currentstroke}%
\pgfsetstrokeopacity{0.000000}%
\pgfsetdash{}{0pt}%
\pgfpathmoveto{\pgfqpoint{2.621461in}{0.613486in}}%
\pgfpathlineto{\pgfqpoint{2.631466in}{0.613486in}}%
\pgfpathlineto{\pgfqpoint{2.631466in}{2.611074in}}%
\pgfpathlineto{\pgfqpoint{2.621461in}{2.611074in}}%
\pgfpathlineto{\pgfqpoint{2.621461in}{0.613486in}}%
\pgfpathclose%
\pgfusepath{fill}%
\end{pgfscope}%
\begin{pgfscope}%
\pgfpathrectangle{\pgfqpoint{0.693757in}{0.613486in}}{\pgfqpoint{5.541243in}{3.963477in}}%
\pgfusepath{clip}%
\pgfsetbuttcap%
\pgfsetmiterjoin%
\definecolor{currentfill}{rgb}{0.000000,0.000000,1.000000}%
\pgfsetfillcolor{currentfill}%
\pgfsetlinewidth{0.000000pt}%
\definecolor{currentstroke}{rgb}{0.000000,0.000000,0.000000}%
\pgfsetstrokecolor{currentstroke}%
\pgfsetstrokeopacity{0.000000}%
\pgfsetdash{}{0pt}%
\pgfpathmoveto{\pgfqpoint{2.633967in}{0.613486in}}%
\pgfpathlineto{\pgfqpoint{2.643972in}{0.613486in}}%
\pgfpathlineto{\pgfqpoint{2.643972in}{1.604355in}}%
\pgfpathlineto{\pgfqpoint{2.633967in}{1.604355in}}%
\pgfpathlineto{\pgfqpoint{2.633967in}{0.613486in}}%
\pgfpathclose%
\pgfusepath{fill}%
\end{pgfscope}%
\begin{pgfscope}%
\pgfpathrectangle{\pgfqpoint{0.693757in}{0.613486in}}{\pgfqpoint{5.541243in}{3.963477in}}%
\pgfusepath{clip}%
\pgfsetbuttcap%
\pgfsetmiterjoin%
\definecolor{currentfill}{rgb}{0.000000,0.000000,1.000000}%
\pgfsetfillcolor{currentfill}%
\pgfsetlinewidth{0.000000pt}%
\definecolor{currentstroke}{rgb}{0.000000,0.000000,0.000000}%
\pgfsetstrokecolor{currentstroke}%
\pgfsetstrokeopacity{0.000000}%
\pgfsetdash{}{0pt}%
\pgfpathmoveto{\pgfqpoint{2.646474in}{0.613486in}}%
\pgfpathlineto{\pgfqpoint{2.656479in}{0.613486in}}%
\pgfpathlineto{\pgfqpoint{2.656479in}{1.612282in}}%
\pgfpathlineto{\pgfqpoint{2.646474in}{1.612282in}}%
\pgfpathlineto{\pgfqpoint{2.646474in}{0.613486in}}%
\pgfpathclose%
\pgfusepath{fill}%
\end{pgfscope}%
\begin{pgfscope}%
\pgfpathrectangle{\pgfqpoint{0.693757in}{0.613486in}}{\pgfqpoint{5.541243in}{3.963477in}}%
\pgfusepath{clip}%
\pgfsetbuttcap%
\pgfsetmiterjoin%
\definecolor{currentfill}{rgb}{0.000000,0.000000,1.000000}%
\pgfsetfillcolor{currentfill}%
\pgfsetlinewidth{0.000000pt}%
\definecolor{currentstroke}{rgb}{0.000000,0.000000,0.000000}%
\pgfsetstrokecolor{currentstroke}%
\pgfsetstrokeopacity{0.000000}%
\pgfsetdash{}{0pt}%
\pgfpathmoveto{\pgfqpoint{2.658980in}{0.613486in}}%
\pgfpathlineto{\pgfqpoint{2.668985in}{0.613486in}}%
\pgfpathlineto{\pgfqpoint{2.668985in}{2.595224in}}%
\pgfpathlineto{\pgfqpoint{2.658980in}{2.595224in}}%
\pgfpathlineto{\pgfqpoint{2.658980in}{0.613486in}}%
\pgfpathclose%
\pgfusepath{fill}%
\end{pgfscope}%
\begin{pgfscope}%
\pgfpathrectangle{\pgfqpoint{0.693757in}{0.613486in}}{\pgfqpoint{5.541243in}{3.963477in}}%
\pgfusepath{clip}%
\pgfsetbuttcap%
\pgfsetmiterjoin%
\definecolor{currentfill}{rgb}{0.000000,0.000000,1.000000}%
\pgfsetfillcolor{currentfill}%
\pgfsetlinewidth{0.000000pt}%
\definecolor{currentstroke}{rgb}{0.000000,0.000000,0.000000}%
\pgfsetstrokecolor{currentstroke}%
\pgfsetstrokeopacity{0.000000}%
\pgfsetdash{}{0pt}%
\pgfpathmoveto{\pgfqpoint{2.671486in}{0.613486in}}%
\pgfpathlineto{\pgfqpoint{2.681491in}{0.613486in}}%
\pgfpathlineto{\pgfqpoint{2.681491in}{2.611074in}}%
\pgfpathlineto{\pgfqpoint{2.671486in}{2.611074in}}%
\pgfpathlineto{\pgfqpoint{2.671486in}{0.613486in}}%
\pgfpathclose%
\pgfusepath{fill}%
\end{pgfscope}%
\begin{pgfscope}%
\pgfpathrectangle{\pgfqpoint{0.693757in}{0.613486in}}{\pgfqpoint{5.541243in}{3.963477in}}%
\pgfusepath{clip}%
\pgfsetbuttcap%
\pgfsetmiterjoin%
\definecolor{currentfill}{rgb}{0.000000,0.000000,1.000000}%
\pgfsetfillcolor{currentfill}%
\pgfsetlinewidth{0.000000pt}%
\definecolor{currentstroke}{rgb}{0.000000,0.000000,0.000000}%
\pgfsetstrokecolor{currentstroke}%
\pgfsetstrokeopacity{0.000000}%
\pgfsetdash{}{0pt}%
\pgfpathmoveto{\pgfqpoint{2.683992in}{0.613486in}}%
\pgfpathlineto{\pgfqpoint{2.693997in}{0.613486in}}%
\pgfpathlineto{\pgfqpoint{2.693997in}{1.604355in}}%
\pgfpathlineto{\pgfqpoint{2.683992in}{1.604355in}}%
\pgfpathlineto{\pgfqpoint{2.683992in}{0.613486in}}%
\pgfpathclose%
\pgfusepath{fill}%
\end{pgfscope}%
\begin{pgfscope}%
\pgfpathrectangle{\pgfqpoint{0.693757in}{0.613486in}}{\pgfqpoint{5.541243in}{3.963477in}}%
\pgfusepath{clip}%
\pgfsetbuttcap%
\pgfsetmiterjoin%
\definecolor{currentfill}{rgb}{0.000000,0.000000,1.000000}%
\pgfsetfillcolor{currentfill}%
\pgfsetlinewidth{0.000000pt}%
\definecolor{currentstroke}{rgb}{0.000000,0.000000,0.000000}%
\pgfsetstrokecolor{currentstroke}%
\pgfsetstrokeopacity{0.000000}%
\pgfsetdash{}{0pt}%
\pgfpathmoveto{\pgfqpoint{2.696498in}{0.613486in}}%
\pgfpathlineto{\pgfqpoint{2.706503in}{0.613486in}}%
\pgfpathlineto{\pgfqpoint{2.706503in}{1.612282in}}%
\pgfpathlineto{\pgfqpoint{2.696498in}{1.612282in}}%
\pgfpathlineto{\pgfqpoint{2.696498in}{0.613486in}}%
\pgfpathclose%
\pgfusepath{fill}%
\end{pgfscope}%
\begin{pgfscope}%
\pgfpathrectangle{\pgfqpoint{0.693757in}{0.613486in}}{\pgfqpoint{5.541243in}{3.963477in}}%
\pgfusepath{clip}%
\pgfsetbuttcap%
\pgfsetmiterjoin%
\definecolor{currentfill}{rgb}{0.000000,0.000000,1.000000}%
\pgfsetfillcolor{currentfill}%
\pgfsetlinewidth{0.000000pt}%
\definecolor{currentstroke}{rgb}{0.000000,0.000000,0.000000}%
\pgfsetstrokecolor{currentstroke}%
\pgfsetstrokeopacity{0.000000}%
\pgfsetdash{}{0pt}%
\pgfpathmoveto{\pgfqpoint{2.709005in}{0.613486in}}%
\pgfpathlineto{\pgfqpoint{2.719010in}{0.613486in}}%
\pgfpathlineto{\pgfqpoint{2.719010in}{2.595224in}}%
\pgfpathlineto{\pgfqpoint{2.709005in}{2.595224in}}%
\pgfpathlineto{\pgfqpoint{2.709005in}{0.613486in}}%
\pgfpathclose%
\pgfusepath{fill}%
\end{pgfscope}%
\begin{pgfscope}%
\pgfpathrectangle{\pgfqpoint{0.693757in}{0.613486in}}{\pgfqpoint{5.541243in}{3.963477in}}%
\pgfusepath{clip}%
\pgfsetbuttcap%
\pgfsetmiterjoin%
\definecolor{currentfill}{rgb}{0.000000,0.000000,1.000000}%
\pgfsetfillcolor{currentfill}%
\pgfsetlinewidth{0.000000pt}%
\definecolor{currentstroke}{rgb}{0.000000,0.000000,0.000000}%
\pgfsetstrokecolor{currentstroke}%
\pgfsetstrokeopacity{0.000000}%
\pgfsetdash{}{0pt}%
\pgfpathmoveto{\pgfqpoint{2.721511in}{0.613486in}}%
\pgfpathlineto{\pgfqpoint{2.731516in}{0.613486in}}%
\pgfpathlineto{\pgfqpoint{2.731516in}{2.611074in}}%
\pgfpathlineto{\pgfqpoint{2.721511in}{2.611074in}}%
\pgfpathlineto{\pgfqpoint{2.721511in}{0.613486in}}%
\pgfpathclose%
\pgfusepath{fill}%
\end{pgfscope}%
\begin{pgfscope}%
\pgfpathrectangle{\pgfqpoint{0.693757in}{0.613486in}}{\pgfqpoint{5.541243in}{3.963477in}}%
\pgfusepath{clip}%
\pgfsetbuttcap%
\pgfsetmiterjoin%
\definecolor{currentfill}{rgb}{0.000000,0.000000,1.000000}%
\pgfsetfillcolor{currentfill}%
\pgfsetlinewidth{0.000000pt}%
\definecolor{currentstroke}{rgb}{0.000000,0.000000,0.000000}%
\pgfsetstrokecolor{currentstroke}%
\pgfsetstrokeopacity{0.000000}%
\pgfsetdash{}{0pt}%
\pgfpathmoveto{\pgfqpoint{2.734017in}{0.613486in}}%
\pgfpathlineto{\pgfqpoint{2.744022in}{0.613486in}}%
\pgfpathlineto{\pgfqpoint{2.744022in}{1.604355in}}%
\pgfpathlineto{\pgfqpoint{2.734017in}{1.604355in}}%
\pgfpathlineto{\pgfqpoint{2.734017in}{0.613486in}}%
\pgfpathclose%
\pgfusepath{fill}%
\end{pgfscope}%
\begin{pgfscope}%
\pgfpathrectangle{\pgfqpoint{0.693757in}{0.613486in}}{\pgfqpoint{5.541243in}{3.963477in}}%
\pgfusepath{clip}%
\pgfsetbuttcap%
\pgfsetmiterjoin%
\definecolor{currentfill}{rgb}{0.000000,0.000000,1.000000}%
\pgfsetfillcolor{currentfill}%
\pgfsetlinewidth{0.000000pt}%
\definecolor{currentstroke}{rgb}{0.000000,0.000000,0.000000}%
\pgfsetstrokecolor{currentstroke}%
\pgfsetstrokeopacity{0.000000}%
\pgfsetdash{}{0pt}%
\pgfpathmoveto{\pgfqpoint{2.746523in}{0.613486in}}%
\pgfpathlineto{\pgfqpoint{2.756528in}{0.613486in}}%
\pgfpathlineto{\pgfqpoint{2.756528in}{1.612282in}}%
\pgfpathlineto{\pgfqpoint{2.746523in}{1.612282in}}%
\pgfpathlineto{\pgfqpoint{2.746523in}{0.613486in}}%
\pgfpathclose%
\pgfusepath{fill}%
\end{pgfscope}%
\begin{pgfscope}%
\pgfpathrectangle{\pgfqpoint{0.693757in}{0.613486in}}{\pgfqpoint{5.541243in}{3.963477in}}%
\pgfusepath{clip}%
\pgfsetbuttcap%
\pgfsetmiterjoin%
\definecolor{currentfill}{rgb}{0.000000,0.000000,1.000000}%
\pgfsetfillcolor{currentfill}%
\pgfsetlinewidth{0.000000pt}%
\definecolor{currentstroke}{rgb}{0.000000,0.000000,0.000000}%
\pgfsetstrokecolor{currentstroke}%
\pgfsetstrokeopacity{0.000000}%
\pgfsetdash{}{0pt}%
\pgfpathmoveto{\pgfqpoint{2.759029in}{0.613486in}}%
\pgfpathlineto{\pgfqpoint{2.769034in}{0.613486in}}%
\pgfpathlineto{\pgfqpoint{2.769034in}{2.595224in}}%
\pgfpathlineto{\pgfqpoint{2.759029in}{2.595224in}}%
\pgfpathlineto{\pgfqpoint{2.759029in}{0.613486in}}%
\pgfpathclose%
\pgfusepath{fill}%
\end{pgfscope}%
\begin{pgfscope}%
\pgfpathrectangle{\pgfqpoint{0.693757in}{0.613486in}}{\pgfqpoint{5.541243in}{3.963477in}}%
\pgfusepath{clip}%
\pgfsetbuttcap%
\pgfsetmiterjoin%
\definecolor{currentfill}{rgb}{0.000000,0.000000,1.000000}%
\pgfsetfillcolor{currentfill}%
\pgfsetlinewidth{0.000000pt}%
\definecolor{currentstroke}{rgb}{0.000000,0.000000,0.000000}%
\pgfsetstrokecolor{currentstroke}%
\pgfsetstrokeopacity{0.000000}%
\pgfsetdash{}{0pt}%
\pgfpathmoveto{\pgfqpoint{2.771536in}{0.613486in}}%
\pgfpathlineto{\pgfqpoint{2.781541in}{0.613486in}}%
\pgfpathlineto{\pgfqpoint{2.781541in}{2.611074in}}%
\pgfpathlineto{\pgfqpoint{2.771536in}{2.611074in}}%
\pgfpathlineto{\pgfqpoint{2.771536in}{0.613486in}}%
\pgfpathclose%
\pgfusepath{fill}%
\end{pgfscope}%
\begin{pgfscope}%
\pgfpathrectangle{\pgfqpoint{0.693757in}{0.613486in}}{\pgfqpoint{5.541243in}{3.963477in}}%
\pgfusepath{clip}%
\pgfsetbuttcap%
\pgfsetmiterjoin%
\definecolor{currentfill}{rgb}{0.000000,0.000000,1.000000}%
\pgfsetfillcolor{currentfill}%
\pgfsetlinewidth{0.000000pt}%
\definecolor{currentstroke}{rgb}{0.000000,0.000000,0.000000}%
\pgfsetstrokecolor{currentstroke}%
\pgfsetstrokeopacity{0.000000}%
\pgfsetdash{}{0pt}%
\pgfpathmoveto{\pgfqpoint{2.784042in}{0.613486in}}%
\pgfpathlineto{\pgfqpoint{2.794047in}{0.613486in}}%
\pgfpathlineto{\pgfqpoint{2.794047in}{1.604355in}}%
\pgfpathlineto{\pgfqpoint{2.784042in}{1.604355in}}%
\pgfpathlineto{\pgfqpoint{2.784042in}{0.613486in}}%
\pgfpathclose%
\pgfusepath{fill}%
\end{pgfscope}%
\begin{pgfscope}%
\pgfpathrectangle{\pgfqpoint{0.693757in}{0.613486in}}{\pgfqpoint{5.541243in}{3.963477in}}%
\pgfusepath{clip}%
\pgfsetbuttcap%
\pgfsetmiterjoin%
\definecolor{currentfill}{rgb}{0.000000,0.000000,1.000000}%
\pgfsetfillcolor{currentfill}%
\pgfsetlinewidth{0.000000pt}%
\definecolor{currentstroke}{rgb}{0.000000,0.000000,0.000000}%
\pgfsetstrokecolor{currentstroke}%
\pgfsetstrokeopacity{0.000000}%
\pgfsetdash{}{0pt}%
\pgfpathmoveto{\pgfqpoint{2.796548in}{0.613486in}}%
\pgfpathlineto{\pgfqpoint{2.806553in}{0.613486in}}%
\pgfpathlineto{\pgfqpoint{2.806553in}{1.612282in}}%
\pgfpathlineto{\pgfqpoint{2.796548in}{1.612282in}}%
\pgfpathlineto{\pgfqpoint{2.796548in}{0.613486in}}%
\pgfpathclose%
\pgfusepath{fill}%
\end{pgfscope}%
\begin{pgfscope}%
\pgfpathrectangle{\pgfqpoint{0.693757in}{0.613486in}}{\pgfqpoint{5.541243in}{3.963477in}}%
\pgfusepath{clip}%
\pgfsetbuttcap%
\pgfsetmiterjoin%
\definecolor{currentfill}{rgb}{0.000000,0.000000,1.000000}%
\pgfsetfillcolor{currentfill}%
\pgfsetlinewidth{0.000000pt}%
\definecolor{currentstroke}{rgb}{0.000000,0.000000,0.000000}%
\pgfsetstrokecolor{currentstroke}%
\pgfsetstrokeopacity{0.000000}%
\pgfsetdash{}{0pt}%
\pgfpathmoveto{\pgfqpoint{2.809054in}{0.613486in}}%
\pgfpathlineto{\pgfqpoint{2.819059in}{0.613486in}}%
\pgfpathlineto{\pgfqpoint{2.819059in}{2.595224in}}%
\pgfpathlineto{\pgfqpoint{2.809054in}{2.595224in}}%
\pgfpathlineto{\pgfqpoint{2.809054in}{0.613486in}}%
\pgfpathclose%
\pgfusepath{fill}%
\end{pgfscope}%
\begin{pgfscope}%
\pgfpathrectangle{\pgfqpoint{0.693757in}{0.613486in}}{\pgfqpoint{5.541243in}{3.963477in}}%
\pgfusepath{clip}%
\pgfsetbuttcap%
\pgfsetmiterjoin%
\definecolor{currentfill}{rgb}{0.000000,0.000000,1.000000}%
\pgfsetfillcolor{currentfill}%
\pgfsetlinewidth{0.000000pt}%
\definecolor{currentstroke}{rgb}{0.000000,0.000000,0.000000}%
\pgfsetstrokecolor{currentstroke}%
\pgfsetstrokeopacity{0.000000}%
\pgfsetdash{}{0pt}%
\pgfpathmoveto{\pgfqpoint{2.821560in}{0.613486in}}%
\pgfpathlineto{\pgfqpoint{2.831565in}{0.613486in}}%
\pgfpathlineto{\pgfqpoint{2.831565in}{2.611074in}}%
\pgfpathlineto{\pgfqpoint{2.821560in}{2.611074in}}%
\pgfpathlineto{\pgfqpoint{2.821560in}{0.613486in}}%
\pgfpathclose%
\pgfusepath{fill}%
\end{pgfscope}%
\begin{pgfscope}%
\pgfpathrectangle{\pgfqpoint{0.693757in}{0.613486in}}{\pgfqpoint{5.541243in}{3.963477in}}%
\pgfusepath{clip}%
\pgfsetbuttcap%
\pgfsetmiterjoin%
\definecolor{currentfill}{rgb}{0.000000,0.000000,1.000000}%
\pgfsetfillcolor{currentfill}%
\pgfsetlinewidth{0.000000pt}%
\definecolor{currentstroke}{rgb}{0.000000,0.000000,0.000000}%
\pgfsetstrokecolor{currentstroke}%
\pgfsetstrokeopacity{0.000000}%
\pgfsetdash{}{0pt}%
\pgfpathmoveto{\pgfqpoint{2.834067in}{0.613486in}}%
\pgfpathlineto{\pgfqpoint{2.844071in}{0.613486in}}%
\pgfpathlineto{\pgfqpoint{2.844071in}{1.604355in}}%
\pgfpathlineto{\pgfqpoint{2.834067in}{1.604355in}}%
\pgfpathlineto{\pgfqpoint{2.834067in}{0.613486in}}%
\pgfpathclose%
\pgfusepath{fill}%
\end{pgfscope}%
\begin{pgfscope}%
\pgfpathrectangle{\pgfqpoint{0.693757in}{0.613486in}}{\pgfqpoint{5.541243in}{3.963477in}}%
\pgfusepath{clip}%
\pgfsetbuttcap%
\pgfsetmiterjoin%
\definecolor{currentfill}{rgb}{0.000000,0.000000,1.000000}%
\pgfsetfillcolor{currentfill}%
\pgfsetlinewidth{0.000000pt}%
\definecolor{currentstroke}{rgb}{0.000000,0.000000,0.000000}%
\pgfsetstrokecolor{currentstroke}%
\pgfsetstrokeopacity{0.000000}%
\pgfsetdash{}{0pt}%
\pgfpathmoveto{\pgfqpoint{2.846573in}{0.613486in}}%
\pgfpathlineto{\pgfqpoint{2.856578in}{0.613486in}}%
\pgfpathlineto{\pgfqpoint{2.856578in}{1.612282in}}%
\pgfpathlineto{\pgfqpoint{2.846573in}{1.612282in}}%
\pgfpathlineto{\pgfqpoint{2.846573in}{0.613486in}}%
\pgfpathclose%
\pgfusepath{fill}%
\end{pgfscope}%
\begin{pgfscope}%
\pgfpathrectangle{\pgfqpoint{0.693757in}{0.613486in}}{\pgfqpoint{5.541243in}{3.963477in}}%
\pgfusepath{clip}%
\pgfsetbuttcap%
\pgfsetmiterjoin%
\definecolor{currentfill}{rgb}{0.000000,0.000000,1.000000}%
\pgfsetfillcolor{currentfill}%
\pgfsetlinewidth{0.000000pt}%
\definecolor{currentstroke}{rgb}{0.000000,0.000000,0.000000}%
\pgfsetstrokecolor{currentstroke}%
\pgfsetstrokeopacity{0.000000}%
\pgfsetdash{}{0pt}%
\pgfpathmoveto{\pgfqpoint{2.859079in}{0.613486in}}%
\pgfpathlineto{\pgfqpoint{2.869084in}{0.613486in}}%
\pgfpathlineto{\pgfqpoint{2.869084in}{2.595224in}}%
\pgfpathlineto{\pgfqpoint{2.859079in}{2.595224in}}%
\pgfpathlineto{\pgfqpoint{2.859079in}{0.613486in}}%
\pgfpathclose%
\pgfusepath{fill}%
\end{pgfscope}%
\begin{pgfscope}%
\pgfpathrectangle{\pgfqpoint{0.693757in}{0.613486in}}{\pgfqpoint{5.541243in}{3.963477in}}%
\pgfusepath{clip}%
\pgfsetbuttcap%
\pgfsetmiterjoin%
\definecolor{currentfill}{rgb}{0.000000,0.000000,1.000000}%
\pgfsetfillcolor{currentfill}%
\pgfsetlinewidth{0.000000pt}%
\definecolor{currentstroke}{rgb}{0.000000,0.000000,0.000000}%
\pgfsetstrokecolor{currentstroke}%
\pgfsetstrokeopacity{0.000000}%
\pgfsetdash{}{0pt}%
\pgfpathmoveto{\pgfqpoint{2.871585in}{0.613486in}}%
\pgfpathlineto{\pgfqpoint{2.881590in}{0.613486in}}%
\pgfpathlineto{\pgfqpoint{2.881590in}{2.611074in}}%
\pgfpathlineto{\pgfqpoint{2.871585in}{2.611074in}}%
\pgfpathlineto{\pgfqpoint{2.871585in}{0.613486in}}%
\pgfpathclose%
\pgfusepath{fill}%
\end{pgfscope}%
\begin{pgfscope}%
\pgfpathrectangle{\pgfqpoint{0.693757in}{0.613486in}}{\pgfqpoint{5.541243in}{3.963477in}}%
\pgfusepath{clip}%
\pgfsetbuttcap%
\pgfsetmiterjoin%
\definecolor{currentfill}{rgb}{0.000000,0.000000,1.000000}%
\pgfsetfillcolor{currentfill}%
\pgfsetlinewidth{0.000000pt}%
\definecolor{currentstroke}{rgb}{0.000000,0.000000,0.000000}%
\pgfsetstrokecolor{currentstroke}%
\pgfsetstrokeopacity{0.000000}%
\pgfsetdash{}{0pt}%
\pgfpathmoveto{\pgfqpoint{2.884091in}{0.613486in}}%
\pgfpathlineto{\pgfqpoint{2.894096in}{0.613486in}}%
\pgfpathlineto{\pgfqpoint{2.894096in}{1.604355in}}%
\pgfpathlineto{\pgfqpoint{2.884091in}{1.604355in}}%
\pgfpathlineto{\pgfqpoint{2.884091in}{0.613486in}}%
\pgfpathclose%
\pgfusepath{fill}%
\end{pgfscope}%
\begin{pgfscope}%
\pgfpathrectangle{\pgfqpoint{0.693757in}{0.613486in}}{\pgfqpoint{5.541243in}{3.963477in}}%
\pgfusepath{clip}%
\pgfsetbuttcap%
\pgfsetmiterjoin%
\definecolor{currentfill}{rgb}{0.000000,0.000000,1.000000}%
\pgfsetfillcolor{currentfill}%
\pgfsetlinewidth{0.000000pt}%
\definecolor{currentstroke}{rgb}{0.000000,0.000000,0.000000}%
\pgfsetstrokecolor{currentstroke}%
\pgfsetstrokeopacity{0.000000}%
\pgfsetdash{}{0pt}%
\pgfpathmoveto{\pgfqpoint{2.896597in}{0.613486in}}%
\pgfpathlineto{\pgfqpoint{2.906602in}{0.613486in}}%
\pgfpathlineto{\pgfqpoint{2.906602in}{1.612282in}}%
\pgfpathlineto{\pgfqpoint{2.896597in}{1.612282in}}%
\pgfpathlineto{\pgfqpoint{2.896597in}{0.613486in}}%
\pgfpathclose%
\pgfusepath{fill}%
\end{pgfscope}%
\begin{pgfscope}%
\pgfpathrectangle{\pgfqpoint{0.693757in}{0.613486in}}{\pgfqpoint{5.541243in}{3.963477in}}%
\pgfusepath{clip}%
\pgfsetbuttcap%
\pgfsetmiterjoin%
\definecolor{currentfill}{rgb}{0.000000,0.000000,1.000000}%
\pgfsetfillcolor{currentfill}%
\pgfsetlinewidth{0.000000pt}%
\definecolor{currentstroke}{rgb}{0.000000,0.000000,0.000000}%
\pgfsetstrokecolor{currentstroke}%
\pgfsetstrokeopacity{0.000000}%
\pgfsetdash{}{0pt}%
\pgfpathmoveto{\pgfqpoint{2.909104in}{0.613486in}}%
\pgfpathlineto{\pgfqpoint{2.919109in}{0.613486in}}%
\pgfpathlineto{\pgfqpoint{2.919109in}{2.595224in}}%
\pgfpathlineto{\pgfqpoint{2.909104in}{2.595224in}}%
\pgfpathlineto{\pgfqpoint{2.909104in}{0.613486in}}%
\pgfpathclose%
\pgfusepath{fill}%
\end{pgfscope}%
\begin{pgfscope}%
\pgfpathrectangle{\pgfqpoint{0.693757in}{0.613486in}}{\pgfqpoint{5.541243in}{3.963477in}}%
\pgfusepath{clip}%
\pgfsetbuttcap%
\pgfsetmiterjoin%
\definecolor{currentfill}{rgb}{0.000000,0.000000,1.000000}%
\pgfsetfillcolor{currentfill}%
\pgfsetlinewidth{0.000000pt}%
\definecolor{currentstroke}{rgb}{0.000000,0.000000,0.000000}%
\pgfsetstrokecolor{currentstroke}%
\pgfsetstrokeopacity{0.000000}%
\pgfsetdash{}{0pt}%
\pgfpathmoveto{\pgfqpoint{2.921610in}{0.613486in}}%
\pgfpathlineto{\pgfqpoint{2.931615in}{0.613486in}}%
\pgfpathlineto{\pgfqpoint{2.931615in}{2.611074in}}%
\pgfpathlineto{\pgfqpoint{2.921610in}{2.611074in}}%
\pgfpathlineto{\pgfqpoint{2.921610in}{0.613486in}}%
\pgfpathclose%
\pgfusepath{fill}%
\end{pgfscope}%
\begin{pgfscope}%
\pgfpathrectangle{\pgfqpoint{0.693757in}{0.613486in}}{\pgfqpoint{5.541243in}{3.963477in}}%
\pgfusepath{clip}%
\pgfsetbuttcap%
\pgfsetmiterjoin%
\definecolor{currentfill}{rgb}{0.000000,0.000000,1.000000}%
\pgfsetfillcolor{currentfill}%
\pgfsetlinewidth{0.000000pt}%
\definecolor{currentstroke}{rgb}{0.000000,0.000000,0.000000}%
\pgfsetstrokecolor{currentstroke}%
\pgfsetstrokeopacity{0.000000}%
\pgfsetdash{}{0pt}%
\pgfpathmoveto{\pgfqpoint{2.934116in}{0.613486in}}%
\pgfpathlineto{\pgfqpoint{2.944121in}{0.613486in}}%
\pgfpathlineto{\pgfqpoint{2.944121in}{1.604355in}}%
\pgfpathlineto{\pgfqpoint{2.934116in}{1.604355in}}%
\pgfpathlineto{\pgfqpoint{2.934116in}{0.613486in}}%
\pgfpathclose%
\pgfusepath{fill}%
\end{pgfscope}%
\begin{pgfscope}%
\pgfpathrectangle{\pgfqpoint{0.693757in}{0.613486in}}{\pgfqpoint{5.541243in}{3.963477in}}%
\pgfusepath{clip}%
\pgfsetbuttcap%
\pgfsetmiterjoin%
\definecolor{currentfill}{rgb}{0.000000,0.000000,1.000000}%
\pgfsetfillcolor{currentfill}%
\pgfsetlinewidth{0.000000pt}%
\definecolor{currentstroke}{rgb}{0.000000,0.000000,0.000000}%
\pgfsetstrokecolor{currentstroke}%
\pgfsetstrokeopacity{0.000000}%
\pgfsetdash{}{0pt}%
\pgfpathmoveto{\pgfqpoint{2.946622in}{0.613486in}}%
\pgfpathlineto{\pgfqpoint{2.956627in}{0.613486in}}%
\pgfpathlineto{\pgfqpoint{2.956627in}{1.612282in}}%
\pgfpathlineto{\pgfqpoint{2.946622in}{1.612282in}}%
\pgfpathlineto{\pgfqpoint{2.946622in}{0.613486in}}%
\pgfpathclose%
\pgfusepath{fill}%
\end{pgfscope}%
\begin{pgfscope}%
\pgfpathrectangle{\pgfqpoint{0.693757in}{0.613486in}}{\pgfqpoint{5.541243in}{3.963477in}}%
\pgfusepath{clip}%
\pgfsetbuttcap%
\pgfsetmiterjoin%
\definecolor{currentfill}{rgb}{0.000000,0.000000,1.000000}%
\pgfsetfillcolor{currentfill}%
\pgfsetlinewidth{0.000000pt}%
\definecolor{currentstroke}{rgb}{0.000000,0.000000,0.000000}%
\pgfsetstrokecolor{currentstroke}%
\pgfsetstrokeopacity{0.000000}%
\pgfsetdash{}{0pt}%
\pgfpathmoveto{\pgfqpoint{2.959128in}{0.613486in}}%
\pgfpathlineto{\pgfqpoint{2.969133in}{0.613486in}}%
\pgfpathlineto{\pgfqpoint{2.969133in}{2.595224in}}%
\pgfpathlineto{\pgfqpoint{2.959128in}{2.595224in}}%
\pgfpathlineto{\pgfqpoint{2.959128in}{0.613486in}}%
\pgfpathclose%
\pgfusepath{fill}%
\end{pgfscope}%
\begin{pgfscope}%
\pgfpathrectangle{\pgfqpoint{0.693757in}{0.613486in}}{\pgfqpoint{5.541243in}{3.963477in}}%
\pgfusepath{clip}%
\pgfsetbuttcap%
\pgfsetmiterjoin%
\definecolor{currentfill}{rgb}{0.000000,0.000000,1.000000}%
\pgfsetfillcolor{currentfill}%
\pgfsetlinewidth{0.000000pt}%
\definecolor{currentstroke}{rgb}{0.000000,0.000000,0.000000}%
\pgfsetstrokecolor{currentstroke}%
\pgfsetstrokeopacity{0.000000}%
\pgfsetdash{}{0pt}%
\pgfpathmoveto{\pgfqpoint{2.971635in}{0.613486in}}%
\pgfpathlineto{\pgfqpoint{2.981640in}{0.613486in}}%
\pgfpathlineto{\pgfqpoint{2.981640in}{2.611074in}}%
\pgfpathlineto{\pgfqpoint{2.971635in}{2.611074in}}%
\pgfpathlineto{\pgfqpoint{2.971635in}{0.613486in}}%
\pgfpathclose%
\pgfusepath{fill}%
\end{pgfscope}%
\begin{pgfscope}%
\pgfpathrectangle{\pgfqpoint{0.693757in}{0.613486in}}{\pgfqpoint{5.541243in}{3.963477in}}%
\pgfusepath{clip}%
\pgfsetbuttcap%
\pgfsetmiterjoin%
\definecolor{currentfill}{rgb}{0.000000,0.000000,1.000000}%
\pgfsetfillcolor{currentfill}%
\pgfsetlinewidth{0.000000pt}%
\definecolor{currentstroke}{rgb}{0.000000,0.000000,0.000000}%
\pgfsetstrokecolor{currentstroke}%
\pgfsetstrokeopacity{0.000000}%
\pgfsetdash{}{0pt}%
\pgfpathmoveto{\pgfqpoint{2.984141in}{0.613486in}}%
\pgfpathlineto{\pgfqpoint{2.994146in}{0.613486in}}%
\pgfpathlineto{\pgfqpoint{2.994146in}{1.604355in}}%
\pgfpathlineto{\pgfqpoint{2.984141in}{1.604355in}}%
\pgfpathlineto{\pgfqpoint{2.984141in}{0.613486in}}%
\pgfpathclose%
\pgfusepath{fill}%
\end{pgfscope}%
\begin{pgfscope}%
\pgfpathrectangle{\pgfqpoint{0.693757in}{0.613486in}}{\pgfqpoint{5.541243in}{3.963477in}}%
\pgfusepath{clip}%
\pgfsetbuttcap%
\pgfsetmiterjoin%
\definecolor{currentfill}{rgb}{0.000000,0.000000,1.000000}%
\pgfsetfillcolor{currentfill}%
\pgfsetlinewidth{0.000000pt}%
\definecolor{currentstroke}{rgb}{0.000000,0.000000,0.000000}%
\pgfsetstrokecolor{currentstroke}%
\pgfsetstrokeopacity{0.000000}%
\pgfsetdash{}{0pt}%
\pgfpathmoveto{\pgfqpoint{2.996647in}{0.613486in}}%
\pgfpathlineto{\pgfqpoint{3.006652in}{0.613486in}}%
\pgfpathlineto{\pgfqpoint{3.006652in}{1.612282in}}%
\pgfpathlineto{\pgfqpoint{2.996647in}{1.612282in}}%
\pgfpathlineto{\pgfqpoint{2.996647in}{0.613486in}}%
\pgfpathclose%
\pgfusepath{fill}%
\end{pgfscope}%
\begin{pgfscope}%
\pgfpathrectangle{\pgfqpoint{0.693757in}{0.613486in}}{\pgfqpoint{5.541243in}{3.963477in}}%
\pgfusepath{clip}%
\pgfsetbuttcap%
\pgfsetmiterjoin%
\definecolor{currentfill}{rgb}{0.000000,0.000000,1.000000}%
\pgfsetfillcolor{currentfill}%
\pgfsetlinewidth{0.000000pt}%
\definecolor{currentstroke}{rgb}{0.000000,0.000000,0.000000}%
\pgfsetstrokecolor{currentstroke}%
\pgfsetstrokeopacity{0.000000}%
\pgfsetdash{}{0pt}%
\pgfpathmoveto{\pgfqpoint{3.009153in}{0.613486in}}%
\pgfpathlineto{\pgfqpoint{3.019158in}{0.613486in}}%
\pgfpathlineto{\pgfqpoint{3.019158in}{2.595224in}}%
\pgfpathlineto{\pgfqpoint{3.009153in}{2.595224in}}%
\pgfpathlineto{\pgfqpoint{3.009153in}{0.613486in}}%
\pgfpathclose%
\pgfusepath{fill}%
\end{pgfscope}%
\begin{pgfscope}%
\pgfpathrectangle{\pgfqpoint{0.693757in}{0.613486in}}{\pgfqpoint{5.541243in}{3.963477in}}%
\pgfusepath{clip}%
\pgfsetbuttcap%
\pgfsetmiterjoin%
\definecolor{currentfill}{rgb}{0.000000,0.000000,1.000000}%
\pgfsetfillcolor{currentfill}%
\pgfsetlinewidth{0.000000pt}%
\definecolor{currentstroke}{rgb}{0.000000,0.000000,0.000000}%
\pgfsetstrokecolor{currentstroke}%
\pgfsetstrokeopacity{0.000000}%
\pgfsetdash{}{0pt}%
\pgfpathmoveto{\pgfqpoint{3.021659in}{0.613486in}}%
\pgfpathlineto{\pgfqpoint{3.031664in}{0.613486in}}%
\pgfpathlineto{\pgfqpoint{3.031664in}{2.611074in}}%
\pgfpathlineto{\pgfqpoint{3.021659in}{2.611074in}}%
\pgfpathlineto{\pgfqpoint{3.021659in}{0.613486in}}%
\pgfpathclose%
\pgfusepath{fill}%
\end{pgfscope}%
\begin{pgfscope}%
\pgfpathrectangle{\pgfqpoint{0.693757in}{0.613486in}}{\pgfqpoint{5.541243in}{3.963477in}}%
\pgfusepath{clip}%
\pgfsetbuttcap%
\pgfsetmiterjoin%
\definecolor{currentfill}{rgb}{0.000000,0.000000,1.000000}%
\pgfsetfillcolor{currentfill}%
\pgfsetlinewidth{0.000000pt}%
\definecolor{currentstroke}{rgb}{0.000000,0.000000,0.000000}%
\pgfsetstrokecolor{currentstroke}%
\pgfsetstrokeopacity{0.000000}%
\pgfsetdash{}{0pt}%
\pgfpathmoveto{\pgfqpoint{3.034166in}{0.613486in}}%
\pgfpathlineto{\pgfqpoint{3.044171in}{0.613486in}}%
\pgfpathlineto{\pgfqpoint{3.044171in}{1.604355in}}%
\pgfpathlineto{\pgfqpoint{3.034166in}{1.604355in}}%
\pgfpathlineto{\pgfqpoint{3.034166in}{0.613486in}}%
\pgfpathclose%
\pgfusepath{fill}%
\end{pgfscope}%
\begin{pgfscope}%
\pgfpathrectangle{\pgfqpoint{0.693757in}{0.613486in}}{\pgfqpoint{5.541243in}{3.963477in}}%
\pgfusepath{clip}%
\pgfsetbuttcap%
\pgfsetmiterjoin%
\definecolor{currentfill}{rgb}{0.000000,0.000000,1.000000}%
\pgfsetfillcolor{currentfill}%
\pgfsetlinewidth{0.000000pt}%
\definecolor{currentstroke}{rgb}{0.000000,0.000000,0.000000}%
\pgfsetstrokecolor{currentstroke}%
\pgfsetstrokeopacity{0.000000}%
\pgfsetdash{}{0pt}%
\pgfpathmoveto{\pgfqpoint{3.046672in}{0.613486in}}%
\pgfpathlineto{\pgfqpoint{3.056677in}{0.613486in}}%
\pgfpathlineto{\pgfqpoint{3.056677in}{1.612282in}}%
\pgfpathlineto{\pgfqpoint{3.046672in}{1.612282in}}%
\pgfpathlineto{\pgfqpoint{3.046672in}{0.613486in}}%
\pgfpathclose%
\pgfusepath{fill}%
\end{pgfscope}%
\begin{pgfscope}%
\pgfpathrectangle{\pgfqpoint{0.693757in}{0.613486in}}{\pgfqpoint{5.541243in}{3.963477in}}%
\pgfusepath{clip}%
\pgfsetbuttcap%
\pgfsetmiterjoin%
\definecolor{currentfill}{rgb}{0.000000,0.000000,1.000000}%
\pgfsetfillcolor{currentfill}%
\pgfsetlinewidth{0.000000pt}%
\definecolor{currentstroke}{rgb}{0.000000,0.000000,0.000000}%
\pgfsetstrokecolor{currentstroke}%
\pgfsetstrokeopacity{0.000000}%
\pgfsetdash{}{0pt}%
\pgfpathmoveto{\pgfqpoint{3.059178in}{0.613486in}}%
\pgfpathlineto{\pgfqpoint{3.069183in}{0.613486in}}%
\pgfpathlineto{\pgfqpoint{3.069183in}{2.595224in}}%
\pgfpathlineto{\pgfqpoint{3.059178in}{2.595224in}}%
\pgfpathlineto{\pgfqpoint{3.059178in}{0.613486in}}%
\pgfpathclose%
\pgfusepath{fill}%
\end{pgfscope}%
\begin{pgfscope}%
\pgfpathrectangle{\pgfqpoint{0.693757in}{0.613486in}}{\pgfqpoint{5.541243in}{3.963477in}}%
\pgfusepath{clip}%
\pgfsetbuttcap%
\pgfsetmiterjoin%
\definecolor{currentfill}{rgb}{0.000000,0.000000,1.000000}%
\pgfsetfillcolor{currentfill}%
\pgfsetlinewidth{0.000000pt}%
\definecolor{currentstroke}{rgb}{0.000000,0.000000,0.000000}%
\pgfsetstrokecolor{currentstroke}%
\pgfsetstrokeopacity{0.000000}%
\pgfsetdash{}{0pt}%
\pgfpathmoveto{\pgfqpoint{3.071684in}{0.613486in}}%
\pgfpathlineto{\pgfqpoint{3.081689in}{0.613486in}}%
\pgfpathlineto{\pgfqpoint{3.081689in}{2.611074in}}%
\pgfpathlineto{\pgfqpoint{3.071684in}{2.611074in}}%
\pgfpathlineto{\pgfqpoint{3.071684in}{0.613486in}}%
\pgfpathclose%
\pgfusepath{fill}%
\end{pgfscope}%
\begin{pgfscope}%
\pgfpathrectangle{\pgfqpoint{0.693757in}{0.613486in}}{\pgfqpoint{5.541243in}{3.963477in}}%
\pgfusepath{clip}%
\pgfsetbuttcap%
\pgfsetmiterjoin%
\definecolor{currentfill}{rgb}{0.000000,0.000000,1.000000}%
\pgfsetfillcolor{currentfill}%
\pgfsetlinewidth{0.000000pt}%
\definecolor{currentstroke}{rgb}{0.000000,0.000000,0.000000}%
\pgfsetstrokecolor{currentstroke}%
\pgfsetstrokeopacity{0.000000}%
\pgfsetdash{}{0pt}%
\pgfpathmoveto{\pgfqpoint{3.084190in}{0.613486in}}%
\pgfpathlineto{\pgfqpoint{3.094195in}{0.613486in}}%
\pgfpathlineto{\pgfqpoint{3.094195in}{1.604355in}}%
\pgfpathlineto{\pgfqpoint{3.084190in}{1.604355in}}%
\pgfpathlineto{\pgfqpoint{3.084190in}{0.613486in}}%
\pgfpathclose%
\pgfusepath{fill}%
\end{pgfscope}%
\begin{pgfscope}%
\pgfpathrectangle{\pgfqpoint{0.693757in}{0.613486in}}{\pgfqpoint{5.541243in}{3.963477in}}%
\pgfusepath{clip}%
\pgfsetbuttcap%
\pgfsetmiterjoin%
\definecolor{currentfill}{rgb}{0.000000,0.000000,1.000000}%
\pgfsetfillcolor{currentfill}%
\pgfsetlinewidth{0.000000pt}%
\definecolor{currentstroke}{rgb}{0.000000,0.000000,0.000000}%
\pgfsetstrokecolor{currentstroke}%
\pgfsetstrokeopacity{0.000000}%
\pgfsetdash{}{0pt}%
\pgfpathmoveto{\pgfqpoint{3.096697in}{0.613486in}}%
\pgfpathlineto{\pgfqpoint{3.106701in}{0.613486in}}%
\pgfpathlineto{\pgfqpoint{3.106701in}{1.612282in}}%
\pgfpathlineto{\pgfqpoint{3.096697in}{1.612282in}}%
\pgfpathlineto{\pgfqpoint{3.096697in}{0.613486in}}%
\pgfpathclose%
\pgfusepath{fill}%
\end{pgfscope}%
\begin{pgfscope}%
\pgfpathrectangle{\pgfqpoint{0.693757in}{0.613486in}}{\pgfqpoint{5.541243in}{3.963477in}}%
\pgfusepath{clip}%
\pgfsetbuttcap%
\pgfsetmiterjoin%
\definecolor{currentfill}{rgb}{0.000000,0.000000,1.000000}%
\pgfsetfillcolor{currentfill}%
\pgfsetlinewidth{0.000000pt}%
\definecolor{currentstroke}{rgb}{0.000000,0.000000,0.000000}%
\pgfsetstrokecolor{currentstroke}%
\pgfsetstrokeopacity{0.000000}%
\pgfsetdash{}{0pt}%
\pgfpathmoveto{\pgfqpoint{3.109203in}{0.613486in}}%
\pgfpathlineto{\pgfqpoint{3.119208in}{0.613486in}}%
\pgfpathlineto{\pgfqpoint{3.119208in}{2.595224in}}%
\pgfpathlineto{\pgfqpoint{3.109203in}{2.595224in}}%
\pgfpathlineto{\pgfqpoint{3.109203in}{0.613486in}}%
\pgfpathclose%
\pgfusepath{fill}%
\end{pgfscope}%
\begin{pgfscope}%
\pgfpathrectangle{\pgfqpoint{0.693757in}{0.613486in}}{\pgfqpoint{5.541243in}{3.963477in}}%
\pgfusepath{clip}%
\pgfsetbuttcap%
\pgfsetmiterjoin%
\definecolor{currentfill}{rgb}{0.000000,0.000000,1.000000}%
\pgfsetfillcolor{currentfill}%
\pgfsetlinewidth{0.000000pt}%
\definecolor{currentstroke}{rgb}{0.000000,0.000000,0.000000}%
\pgfsetstrokecolor{currentstroke}%
\pgfsetstrokeopacity{0.000000}%
\pgfsetdash{}{0pt}%
\pgfpathmoveto{\pgfqpoint{3.121709in}{0.613486in}}%
\pgfpathlineto{\pgfqpoint{3.131714in}{0.613486in}}%
\pgfpathlineto{\pgfqpoint{3.131714in}{2.611074in}}%
\pgfpathlineto{\pgfqpoint{3.121709in}{2.611074in}}%
\pgfpathlineto{\pgfqpoint{3.121709in}{0.613486in}}%
\pgfpathclose%
\pgfusepath{fill}%
\end{pgfscope}%
\begin{pgfscope}%
\pgfpathrectangle{\pgfqpoint{0.693757in}{0.613486in}}{\pgfqpoint{5.541243in}{3.963477in}}%
\pgfusepath{clip}%
\pgfsetbuttcap%
\pgfsetmiterjoin%
\definecolor{currentfill}{rgb}{0.000000,0.000000,1.000000}%
\pgfsetfillcolor{currentfill}%
\pgfsetlinewidth{0.000000pt}%
\definecolor{currentstroke}{rgb}{0.000000,0.000000,0.000000}%
\pgfsetstrokecolor{currentstroke}%
\pgfsetstrokeopacity{0.000000}%
\pgfsetdash{}{0pt}%
\pgfpathmoveto{\pgfqpoint{3.134215in}{0.613486in}}%
\pgfpathlineto{\pgfqpoint{3.144220in}{0.613486in}}%
\pgfpathlineto{\pgfqpoint{3.144220in}{1.604355in}}%
\pgfpathlineto{\pgfqpoint{3.134215in}{1.604355in}}%
\pgfpathlineto{\pgfqpoint{3.134215in}{0.613486in}}%
\pgfpathclose%
\pgfusepath{fill}%
\end{pgfscope}%
\begin{pgfscope}%
\pgfpathrectangle{\pgfqpoint{0.693757in}{0.613486in}}{\pgfqpoint{5.541243in}{3.963477in}}%
\pgfusepath{clip}%
\pgfsetbuttcap%
\pgfsetmiterjoin%
\definecolor{currentfill}{rgb}{0.000000,0.000000,1.000000}%
\pgfsetfillcolor{currentfill}%
\pgfsetlinewidth{0.000000pt}%
\definecolor{currentstroke}{rgb}{0.000000,0.000000,0.000000}%
\pgfsetstrokecolor{currentstroke}%
\pgfsetstrokeopacity{0.000000}%
\pgfsetdash{}{0pt}%
\pgfpathmoveto{\pgfqpoint{3.146721in}{0.613486in}}%
\pgfpathlineto{\pgfqpoint{3.156726in}{0.613486in}}%
\pgfpathlineto{\pgfqpoint{3.156726in}{1.612282in}}%
\pgfpathlineto{\pgfqpoint{3.146721in}{1.612282in}}%
\pgfpathlineto{\pgfqpoint{3.146721in}{0.613486in}}%
\pgfpathclose%
\pgfusepath{fill}%
\end{pgfscope}%
\begin{pgfscope}%
\pgfpathrectangle{\pgfqpoint{0.693757in}{0.613486in}}{\pgfqpoint{5.541243in}{3.963477in}}%
\pgfusepath{clip}%
\pgfsetbuttcap%
\pgfsetmiterjoin%
\definecolor{currentfill}{rgb}{0.000000,0.000000,1.000000}%
\pgfsetfillcolor{currentfill}%
\pgfsetlinewidth{0.000000pt}%
\definecolor{currentstroke}{rgb}{0.000000,0.000000,0.000000}%
\pgfsetstrokecolor{currentstroke}%
\pgfsetstrokeopacity{0.000000}%
\pgfsetdash{}{0pt}%
\pgfpathmoveto{\pgfqpoint{3.159227in}{0.613486in}}%
\pgfpathlineto{\pgfqpoint{3.169232in}{0.613486in}}%
\pgfpathlineto{\pgfqpoint{3.169232in}{2.595224in}}%
\pgfpathlineto{\pgfqpoint{3.159227in}{2.595224in}}%
\pgfpathlineto{\pgfqpoint{3.159227in}{0.613486in}}%
\pgfpathclose%
\pgfusepath{fill}%
\end{pgfscope}%
\begin{pgfscope}%
\pgfpathrectangle{\pgfqpoint{0.693757in}{0.613486in}}{\pgfqpoint{5.541243in}{3.963477in}}%
\pgfusepath{clip}%
\pgfsetbuttcap%
\pgfsetmiterjoin%
\definecolor{currentfill}{rgb}{0.000000,0.000000,1.000000}%
\pgfsetfillcolor{currentfill}%
\pgfsetlinewidth{0.000000pt}%
\definecolor{currentstroke}{rgb}{0.000000,0.000000,0.000000}%
\pgfsetstrokecolor{currentstroke}%
\pgfsetstrokeopacity{0.000000}%
\pgfsetdash{}{0pt}%
\pgfpathmoveto{\pgfqpoint{3.171734in}{0.613486in}}%
\pgfpathlineto{\pgfqpoint{3.181739in}{0.613486in}}%
\pgfpathlineto{\pgfqpoint{3.181739in}{2.611074in}}%
\pgfpathlineto{\pgfqpoint{3.171734in}{2.611074in}}%
\pgfpathlineto{\pgfqpoint{3.171734in}{0.613486in}}%
\pgfpathclose%
\pgfusepath{fill}%
\end{pgfscope}%
\begin{pgfscope}%
\pgfpathrectangle{\pgfqpoint{0.693757in}{0.613486in}}{\pgfqpoint{5.541243in}{3.963477in}}%
\pgfusepath{clip}%
\pgfsetbuttcap%
\pgfsetmiterjoin%
\definecolor{currentfill}{rgb}{0.000000,0.000000,1.000000}%
\pgfsetfillcolor{currentfill}%
\pgfsetlinewidth{0.000000pt}%
\definecolor{currentstroke}{rgb}{0.000000,0.000000,0.000000}%
\pgfsetstrokecolor{currentstroke}%
\pgfsetstrokeopacity{0.000000}%
\pgfsetdash{}{0pt}%
\pgfpathmoveto{\pgfqpoint{3.184240in}{0.613486in}}%
\pgfpathlineto{\pgfqpoint{3.194245in}{0.613486in}}%
\pgfpathlineto{\pgfqpoint{3.194245in}{1.604355in}}%
\pgfpathlineto{\pgfqpoint{3.184240in}{1.604355in}}%
\pgfpathlineto{\pgfqpoint{3.184240in}{0.613486in}}%
\pgfpathclose%
\pgfusepath{fill}%
\end{pgfscope}%
\begin{pgfscope}%
\pgfpathrectangle{\pgfqpoint{0.693757in}{0.613486in}}{\pgfqpoint{5.541243in}{3.963477in}}%
\pgfusepath{clip}%
\pgfsetbuttcap%
\pgfsetmiterjoin%
\definecolor{currentfill}{rgb}{0.000000,0.000000,1.000000}%
\pgfsetfillcolor{currentfill}%
\pgfsetlinewidth{0.000000pt}%
\definecolor{currentstroke}{rgb}{0.000000,0.000000,0.000000}%
\pgfsetstrokecolor{currentstroke}%
\pgfsetstrokeopacity{0.000000}%
\pgfsetdash{}{0pt}%
\pgfpathmoveto{\pgfqpoint{3.196746in}{0.613486in}}%
\pgfpathlineto{\pgfqpoint{3.206751in}{0.613486in}}%
\pgfpathlineto{\pgfqpoint{3.206751in}{1.612282in}}%
\pgfpathlineto{\pgfqpoint{3.196746in}{1.612282in}}%
\pgfpathlineto{\pgfqpoint{3.196746in}{0.613486in}}%
\pgfpathclose%
\pgfusepath{fill}%
\end{pgfscope}%
\begin{pgfscope}%
\pgfpathrectangle{\pgfqpoint{0.693757in}{0.613486in}}{\pgfqpoint{5.541243in}{3.963477in}}%
\pgfusepath{clip}%
\pgfsetbuttcap%
\pgfsetmiterjoin%
\definecolor{currentfill}{rgb}{0.000000,0.000000,1.000000}%
\pgfsetfillcolor{currentfill}%
\pgfsetlinewidth{0.000000pt}%
\definecolor{currentstroke}{rgb}{0.000000,0.000000,0.000000}%
\pgfsetstrokecolor{currentstroke}%
\pgfsetstrokeopacity{0.000000}%
\pgfsetdash{}{0pt}%
\pgfpathmoveto{\pgfqpoint{3.209252in}{0.613486in}}%
\pgfpathlineto{\pgfqpoint{3.219257in}{0.613486in}}%
\pgfpathlineto{\pgfqpoint{3.219257in}{2.595224in}}%
\pgfpathlineto{\pgfqpoint{3.209252in}{2.595224in}}%
\pgfpathlineto{\pgfqpoint{3.209252in}{0.613486in}}%
\pgfpathclose%
\pgfusepath{fill}%
\end{pgfscope}%
\begin{pgfscope}%
\pgfpathrectangle{\pgfqpoint{0.693757in}{0.613486in}}{\pgfqpoint{5.541243in}{3.963477in}}%
\pgfusepath{clip}%
\pgfsetbuttcap%
\pgfsetmiterjoin%
\definecolor{currentfill}{rgb}{0.000000,0.000000,1.000000}%
\pgfsetfillcolor{currentfill}%
\pgfsetlinewidth{0.000000pt}%
\definecolor{currentstroke}{rgb}{0.000000,0.000000,0.000000}%
\pgfsetstrokecolor{currentstroke}%
\pgfsetstrokeopacity{0.000000}%
\pgfsetdash{}{0pt}%
\pgfpathmoveto{\pgfqpoint{3.221758in}{0.613486in}}%
\pgfpathlineto{\pgfqpoint{3.231763in}{0.613486in}}%
\pgfpathlineto{\pgfqpoint{3.231763in}{2.611074in}}%
\pgfpathlineto{\pgfqpoint{3.221758in}{2.611074in}}%
\pgfpathlineto{\pgfqpoint{3.221758in}{0.613486in}}%
\pgfpathclose%
\pgfusepath{fill}%
\end{pgfscope}%
\begin{pgfscope}%
\pgfpathrectangle{\pgfqpoint{0.693757in}{0.613486in}}{\pgfqpoint{5.541243in}{3.963477in}}%
\pgfusepath{clip}%
\pgfsetbuttcap%
\pgfsetmiterjoin%
\definecolor{currentfill}{rgb}{0.000000,0.000000,1.000000}%
\pgfsetfillcolor{currentfill}%
\pgfsetlinewidth{0.000000pt}%
\definecolor{currentstroke}{rgb}{0.000000,0.000000,0.000000}%
\pgfsetstrokecolor{currentstroke}%
\pgfsetstrokeopacity{0.000000}%
\pgfsetdash{}{0pt}%
\pgfpathmoveto{\pgfqpoint{3.234265in}{0.613486in}}%
\pgfpathlineto{\pgfqpoint{3.244270in}{0.613486in}}%
\pgfpathlineto{\pgfqpoint{3.244270in}{1.604355in}}%
\pgfpathlineto{\pgfqpoint{3.234265in}{1.604355in}}%
\pgfpathlineto{\pgfqpoint{3.234265in}{0.613486in}}%
\pgfpathclose%
\pgfusepath{fill}%
\end{pgfscope}%
\begin{pgfscope}%
\pgfpathrectangle{\pgfqpoint{0.693757in}{0.613486in}}{\pgfqpoint{5.541243in}{3.963477in}}%
\pgfusepath{clip}%
\pgfsetbuttcap%
\pgfsetmiterjoin%
\definecolor{currentfill}{rgb}{0.000000,0.000000,1.000000}%
\pgfsetfillcolor{currentfill}%
\pgfsetlinewidth{0.000000pt}%
\definecolor{currentstroke}{rgb}{0.000000,0.000000,0.000000}%
\pgfsetstrokecolor{currentstroke}%
\pgfsetstrokeopacity{0.000000}%
\pgfsetdash{}{0pt}%
\pgfpathmoveto{\pgfqpoint{3.246771in}{0.613486in}}%
\pgfpathlineto{\pgfqpoint{3.256776in}{0.613486in}}%
\pgfpathlineto{\pgfqpoint{3.256776in}{1.612282in}}%
\pgfpathlineto{\pgfqpoint{3.246771in}{1.612282in}}%
\pgfpathlineto{\pgfqpoint{3.246771in}{0.613486in}}%
\pgfpathclose%
\pgfusepath{fill}%
\end{pgfscope}%
\begin{pgfscope}%
\pgfpathrectangle{\pgfqpoint{0.693757in}{0.613486in}}{\pgfqpoint{5.541243in}{3.963477in}}%
\pgfusepath{clip}%
\pgfsetbuttcap%
\pgfsetmiterjoin%
\definecolor{currentfill}{rgb}{0.000000,0.000000,1.000000}%
\pgfsetfillcolor{currentfill}%
\pgfsetlinewidth{0.000000pt}%
\definecolor{currentstroke}{rgb}{0.000000,0.000000,0.000000}%
\pgfsetstrokecolor{currentstroke}%
\pgfsetstrokeopacity{0.000000}%
\pgfsetdash{}{0pt}%
\pgfpathmoveto{\pgfqpoint{3.259277in}{0.613486in}}%
\pgfpathlineto{\pgfqpoint{3.269282in}{0.613486in}}%
\pgfpathlineto{\pgfqpoint{3.269282in}{2.595224in}}%
\pgfpathlineto{\pgfqpoint{3.259277in}{2.595224in}}%
\pgfpathlineto{\pgfqpoint{3.259277in}{0.613486in}}%
\pgfpathclose%
\pgfusepath{fill}%
\end{pgfscope}%
\begin{pgfscope}%
\pgfpathrectangle{\pgfqpoint{0.693757in}{0.613486in}}{\pgfqpoint{5.541243in}{3.963477in}}%
\pgfusepath{clip}%
\pgfsetbuttcap%
\pgfsetmiterjoin%
\definecolor{currentfill}{rgb}{0.000000,0.000000,1.000000}%
\pgfsetfillcolor{currentfill}%
\pgfsetlinewidth{0.000000pt}%
\definecolor{currentstroke}{rgb}{0.000000,0.000000,0.000000}%
\pgfsetstrokecolor{currentstroke}%
\pgfsetstrokeopacity{0.000000}%
\pgfsetdash{}{0pt}%
\pgfpathmoveto{\pgfqpoint{3.271783in}{0.613486in}}%
\pgfpathlineto{\pgfqpoint{3.281788in}{0.613486in}}%
\pgfpathlineto{\pgfqpoint{3.281788in}{2.611074in}}%
\pgfpathlineto{\pgfqpoint{3.271783in}{2.611074in}}%
\pgfpathlineto{\pgfqpoint{3.271783in}{0.613486in}}%
\pgfpathclose%
\pgfusepath{fill}%
\end{pgfscope}%
\begin{pgfscope}%
\pgfpathrectangle{\pgfqpoint{0.693757in}{0.613486in}}{\pgfqpoint{5.541243in}{3.963477in}}%
\pgfusepath{clip}%
\pgfsetbuttcap%
\pgfsetmiterjoin%
\definecolor{currentfill}{rgb}{0.000000,0.000000,1.000000}%
\pgfsetfillcolor{currentfill}%
\pgfsetlinewidth{0.000000pt}%
\definecolor{currentstroke}{rgb}{0.000000,0.000000,0.000000}%
\pgfsetstrokecolor{currentstroke}%
\pgfsetstrokeopacity{0.000000}%
\pgfsetdash{}{0pt}%
\pgfpathmoveto{\pgfqpoint{3.284289in}{0.613486in}}%
\pgfpathlineto{\pgfqpoint{3.294294in}{0.613486in}}%
\pgfpathlineto{\pgfqpoint{3.294294in}{1.604355in}}%
\pgfpathlineto{\pgfqpoint{3.284289in}{1.604355in}}%
\pgfpathlineto{\pgfqpoint{3.284289in}{0.613486in}}%
\pgfpathclose%
\pgfusepath{fill}%
\end{pgfscope}%
\begin{pgfscope}%
\pgfpathrectangle{\pgfqpoint{0.693757in}{0.613486in}}{\pgfqpoint{5.541243in}{3.963477in}}%
\pgfusepath{clip}%
\pgfsetbuttcap%
\pgfsetmiterjoin%
\definecolor{currentfill}{rgb}{0.000000,0.000000,1.000000}%
\pgfsetfillcolor{currentfill}%
\pgfsetlinewidth{0.000000pt}%
\definecolor{currentstroke}{rgb}{0.000000,0.000000,0.000000}%
\pgfsetstrokecolor{currentstroke}%
\pgfsetstrokeopacity{0.000000}%
\pgfsetdash{}{0pt}%
\pgfpathmoveto{\pgfqpoint{3.296796in}{0.613486in}}%
\pgfpathlineto{\pgfqpoint{3.306801in}{0.613486in}}%
\pgfpathlineto{\pgfqpoint{3.306801in}{1.612282in}}%
\pgfpathlineto{\pgfqpoint{3.296796in}{1.612282in}}%
\pgfpathlineto{\pgfqpoint{3.296796in}{0.613486in}}%
\pgfpathclose%
\pgfusepath{fill}%
\end{pgfscope}%
\begin{pgfscope}%
\pgfpathrectangle{\pgfqpoint{0.693757in}{0.613486in}}{\pgfqpoint{5.541243in}{3.963477in}}%
\pgfusepath{clip}%
\pgfsetbuttcap%
\pgfsetmiterjoin%
\definecolor{currentfill}{rgb}{0.000000,0.000000,1.000000}%
\pgfsetfillcolor{currentfill}%
\pgfsetlinewidth{0.000000pt}%
\definecolor{currentstroke}{rgb}{0.000000,0.000000,0.000000}%
\pgfsetstrokecolor{currentstroke}%
\pgfsetstrokeopacity{0.000000}%
\pgfsetdash{}{0pt}%
\pgfpathmoveto{\pgfqpoint{3.309302in}{0.613486in}}%
\pgfpathlineto{\pgfqpoint{3.319307in}{0.613486in}}%
\pgfpathlineto{\pgfqpoint{3.319307in}{2.595224in}}%
\pgfpathlineto{\pgfqpoint{3.309302in}{2.595224in}}%
\pgfpathlineto{\pgfqpoint{3.309302in}{0.613486in}}%
\pgfpathclose%
\pgfusepath{fill}%
\end{pgfscope}%
\begin{pgfscope}%
\pgfpathrectangle{\pgfqpoint{0.693757in}{0.613486in}}{\pgfqpoint{5.541243in}{3.963477in}}%
\pgfusepath{clip}%
\pgfsetbuttcap%
\pgfsetmiterjoin%
\definecolor{currentfill}{rgb}{0.000000,0.000000,1.000000}%
\pgfsetfillcolor{currentfill}%
\pgfsetlinewidth{0.000000pt}%
\definecolor{currentstroke}{rgb}{0.000000,0.000000,0.000000}%
\pgfsetstrokecolor{currentstroke}%
\pgfsetstrokeopacity{0.000000}%
\pgfsetdash{}{0pt}%
\pgfpathmoveto{\pgfqpoint{3.321808in}{0.613486in}}%
\pgfpathlineto{\pgfqpoint{3.331813in}{0.613486in}}%
\pgfpathlineto{\pgfqpoint{3.331813in}{2.611074in}}%
\pgfpathlineto{\pgfqpoint{3.321808in}{2.611074in}}%
\pgfpathlineto{\pgfqpoint{3.321808in}{0.613486in}}%
\pgfpathclose%
\pgfusepath{fill}%
\end{pgfscope}%
\begin{pgfscope}%
\pgfpathrectangle{\pgfqpoint{0.693757in}{0.613486in}}{\pgfqpoint{5.541243in}{3.963477in}}%
\pgfusepath{clip}%
\pgfsetbuttcap%
\pgfsetmiterjoin%
\definecolor{currentfill}{rgb}{0.000000,0.000000,1.000000}%
\pgfsetfillcolor{currentfill}%
\pgfsetlinewidth{0.000000pt}%
\definecolor{currentstroke}{rgb}{0.000000,0.000000,0.000000}%
\pgfsetstrokecolor{currentstroke}%
\pgfsetstrokeopacity{0.000000}%
\pgfsetdash{}{0pt}%
\pgfpathmoveto{\pgfqpoint{3.334314in}{0.613486in}}%
\pgfpathlineto{\pgfqpoint{3.344319in}{0.613486in}}%
\pgfpathlineto{\pgfqpoint{3.344319in}{1.604355in}}%
\pgfpathlineto{\pgfqpoint{3.334314in}{1.604355in}}%
\pgfpathlineto{\pgfqpoint{3.334314in}{0.613486in}}%
\pgfpathclose%
\pgfusepath{fill}%
\end{pgfscope}%
\begin{pgfscope}%
\pgfpathrectangle{\pgfqpoint{0.693757in}{0.613486in}}{\pgfqpoint{5.541243in}{3.963477in}}%
\pgfusepath{clip}%
\pgfsetbuttcap%
\pgfsetmiterjoin%
\definecolor{currentfill}{rgb}{0.000000,0.000000,1.000000}%
\pgfsetfillcolor{currentfill}%
\pgfsetlinewidth{0.000000pt}%
\definecolor{currentstroke}{rgb}{0.000000,0.000000,0.000000}%
\pgfsetstrokecolor{currentstroke}%
\pgfsetstrokeopacity{0.000000}%
\pgfsetdash{}{0pt}%
\pgfpathmoveto{\pgfqpoint{3.346820in}{0.613486in}}%
\pgfpathlineto{\pgfqpoint{3.356825in}{0.613486in}}%
\pgfpathlineto{\pgfqpoint{3.356825in}{1.612282in}}%
\pgfpathlineto{\pgfqpoint{3.346820in}{1.612282in}}%
\pgfpathlineto{\pgfqpoint{3.346820in}{0.613486in}}%
\pgfpathclose%
\pgfusepath{fill}%
\end{pgfscope}%
\begin{pgfscope}%
\pgfpathrectangle{\pgfqpoint{0.693757in}{0.613486in}}{\pgfqpoint{5.541243in}{3.963477in}}%
\pgfusepath{clip}%
\pgfsetbuttcap%
\pgfsetmiterjoin%
\definecolor{currentfill}{rgb}{0.000000,0.000000,1.000000}%
\pgfsetfillcolor{currentfill}%
\pgfsetlinewidth{0.000000pt}%
\definecolor{currentstroke}{rgb}{0.000000,0.000000,0.000000}%
\pgfsetstrokecolor{currentstroke}%
\pgfsetstrokeopacity{0.000000}%
\pgfsetdash{}{0pt}%
\pgfpathmoveto{\pgfqpoint{3.359327in}{0.613486in}}%
\pgfpathlineto{\pgfqpoint{3.369331in}{0.613486in}}%
\pgfpathlineto{\pgfqpoint{3.369331in}{2.595224in}}%
\pgfpathlineto{\pgfqpoint{3.359327in}{2.595224in}}%
\pgfpathlineto{\pgfqpoint{3.359327in}{0.613486in}}%
\pgfpathclose%
\pgfusepath{fill}%
\end{pgfscope}%
\begin{pgfscope}%
\pgfpathrectangle{\pgfqpoint{0.693757in}{0.613486in}}{\pgfqpoint{5.541243in}{3.963477in}}%
\pgfusepath{clip}%
\pgfsetbuttcap%
\pgfsetmiterjoin%
\definecolor{currentfill}{rgb}{0.000000,0.000000,1.000000}%
\pgfsetfillcolor{currentfill}%
\pgfsetlinewidth{0.000000pt}%
\definecolor{currentstroke}{rgb}{0.000000,0.000000,0.000000}%
\pgfsetstrokecolor{currentstroke}%
\pgfsetstrokeopacity{0.000000}%
\pgfsetdash{}{0pt}%
\pgfpathmoveto{\pgfqpoint{3.371833in}{0.613486in}}%
\pgfpathlineto{\pgfqpoint{3.381838in}{0.613486in}}%
\pgfpathlineto{\pgfqpoint{3.381838in}{2.611074in}}%
\pgfpathlineto{\pgfqpoint{3.371833in}{2.611074in}}%
\pgfpathlineto{\pgfqpoint{3.371833in}{0.613486in}}%
\pgfpathclose%
\pgfusepath{fill}%
\end{pgfscope}%
\begin{pgfscope}%
\pgfpathrectangle{\pgfqpoint{0.693757in}{0.613486in}}{\pgfqpoint{5.541243in}{3.963477in}}%
\pgfusepath{clip}%
\pgfsetbuttcap%
\pgfsetmiterjoin%
\definecolor{currentfill}{rgb}{0.000000,0.000000,1.000000}%
\pgfsetfillcolor{currentfill}%
\pgfsetlinewidth{0.000000pt}%
\definecolor{currentstroke}{rgb}{0.000000,0.000000,0.000000}%
\pgfsetstrokecolor{currentstroke}%
\pgfsetstrokeopacity{0.000000}%
\pgfsetdash{}{0pt}%
\pgfpathmoveto{\pgfqpoint{3.384339in}{0.613486in}}%
\pgfpathlineto{\pgfqpoint{3.394344in}{0.613486in}}%
\pgfpathlineto{\pgfqpoint{3.394344in}{1.604355in}}%
\pgfpathlineto{\pgfqpoint{3.384339in}{1.604355in}}%
\pgfpathlineto{\pgfqpoint{3.384339in}{0.613486in}}%
\pgfpathclose%
\pgfusepath{fill}%
\end{pgfscope}%
\begin{pgfscope}%
\pgfpathrectangle{\pgfqpoint{0.693757in}{0.613486in}}{\pgfqpoint{5.541243in}{3.963477in}}%
\pgfusepath{clip}%
\pgfsetbuttcap%
\pgfsetmiterjoin%
\definecolor{currentfill}{rgb}{0.000000,0.000000,1.000000}%
\pgfsetfillcolor{currentfill}%
\pgfsetlinewidth{0.000000pt}%
\definecolor{currentstroke}{rgb}{0.000000,0.000000,0.000000}%
\pgfsetstrokecolor{currentstroke}%
\pgfsetstrokeopacity{0.000000}%
\pgfsetdash{}{0pt}%
\pgfpathmoveto{\pgfqpoint{3.396845in}{0.613486in}}%
\pgfpathlineto{\pgfqpoint{3.406850in}{0.613486in}}%
\pgfpathlineto{\pgfqpoint{3.406850in}{1.612282in}}%
\pgfpathlineto{\pgfqpoint{3.396845in}{1.612282in}}%
\pgfpathlineto{\pgfqpoint{3.396845in}{0.613486in}}%
\pgfpathclose%
\pgfusepath{fill}%
\end{pgfscope}%
\begin{pgfscope}%
\pgfpathrectangle{\pgfqpoint{0.693757in}{0.613486in}}{\pgfqpoint{5.541243in}{3.963477in}}%
\pgfusepath{clip}%
\pgfsetbuttcap%
\pgfsetmiterjoin%
\definecolor{currentfill}{rgb}{0.000000,0.000000,1.000000}%
\pgfsetfillcolor{currentfill}%
\pgfsetlinewidth{0.000000pt}%
\definecolor{currentstroke}{rgb}{0.000000,0.000000,0.000000}%
\pgfsetstrokecolor{currentstroke}%
\pgfsetstrokeopacity{0.000000}%
\pgfsetdash{}{0pt}%
\pgfpathmoveto{\pgfqpoint{3.409351in}{0.613486in}}%
\pgfpathlineto{\pgfqpoint{3.419356in}{0.613486in}}%
\pgfpathlineto{\pgfqpoint{3.419356in}{2.595224in}}%
\pgfpathlineto{\pgfqpoint{3.409351in}{2.595224in}}%
\pgfpathlineto{\pgfqpoint{3.409351in}{0.613486in}}%
\pgfpathclose%
\pgfusepath{fill}%
\end{pgfscope}%
\begin{pgfscope}%
\pgfpathrectangle{\pgfqpoint{0.693757in}{0.613486in}}{\pgfqpoint{5.541243in}{3.963477in}}%
\pgfusepath{clip}%
\pgfsetbuttcap%
\pgfsetmiterjoin%
\definecolor{currentfill}{rgb}{0.000000,0.000000,1.000000}%
\pgfsetfillcolor{currentfill}%
\pgfsetlinewidth{0.000000pt}%
\definecolor{currentstroke}{rgb}{0.000000,0.000000,0.000000}%
\pgfsetstrokecolor{currentstroke}%
\pgfsetstrokeopacity{0.000000}%
\pgfsetdash{}{0pt}%
\pgfpathmoveto{\pgfqpoint{3.421857in}{0.613486in}}%
\pgfpathlineto{\pgfqpoint{3.431862in}{0.613486in}}%
\pgfpathlineto{\pgfqpoint{3.431862in}{2.611074in}}%
\pgfpathlineto{\pgfqpoint{3.421857in}{2.611074in}}%
\pgfpathlineto{\pgfqpoint{3.421857in}{0.613486in}}%
\pgfpathclose%
\pgfusepath{fill}%
\end{pgfscope}%
\begin{pgfscope}%
\pgfpathrectangle{\pgfqpoint{0.693757in}{0.613486in}}{\pgfqpoint{5.541243in}{3.963477in}}%
\pgfusepath{clip}%
\pgfsetbuttcap%
\pgfsetmiterjoin%
\definecolor{currentfill}{rgb}{0.000000,0.000000,1.000000}%
\pgfsetfillcolor{currentfill}%
\pgfsetlinewidth{0.000000pt}%
\definecolor{currentstroke}{rgb}{0.000000,0.000000,0.000000}%
\pgfsetstrokecolor{currentstroke}%
\pgfsetstrokeopacity{0.000000}%
\pgfsetdash{}{0pt}%
\pgfpathmoveto{\pgfqpoint{3.434364in}{0.613486in}}%
\pgfpathlineto{\pgfqpoint{3.444369in}{0.613486in}}%
\pgfpathlineto{\pgfqpoint{3.444369in}{1.604355in}}%
\pgfpathlineto{\pgfqpoint{3.434364in}{1.604355in}}%
\pgfpathlineto{\pgfqpoint{3.434364in}{0.613486in}}%
\pgfpathclose%
\pgfusepath{fill}%
\end{pgfscope}%
\begin{pgfscope}%
\pgfpathrectangle{\pgfqpoint{0.693757in}{0.613486in}}{\pgfqpoint{5.541243in}{3.963477in}}%
\pgfusepath{clip}%
\pgfsetbuttcap%
\pgfsetmiterjoin%
\definecolor{currentfill}{rgb}{0.000000,0.000000,1.000000}%
\pgfsetfillcolor{currentfill}%
\pgfsetlinewidth{0.000000pt}%
\definecolor{currentstroke}{rgb}{0.000000,0.000000,0.000000}%
\pgfsetstrokecolor{currentstroke}%
\pgfsetstrokeopacity{0.000000}%
\pgfsetdash{}{0pt}%
\pgfpathmoveto{\pgfqpoint{3.446870in}{0.613486in}}%
\pgfpathlineto{\pgfqpoint{3.456875in}{0.613486in}}%
\pgfpathlineto{\pgfqpoint{3.456875in}{1.612282in}}%
\pgfpathlineto{\pgfqpoint{3.446870in}{1.612282in}}%
\pgfpathlineto{\pgfqpoint{3.446870in}{0.613486in}}%
\pgfpathclose%
\pgfusepath{fill}%
\end{pgfscope}%
\begin{pgfscope}%
\pgfpathrectangle{\pgfqpoint{0.693757in}{0.613486in}}{\pgfqpoint{5.541243in}{3.963477in}}%
\pgfusepath{clip}%
\pgfsetbuttcap%
\pgfsetmiterjoin%
\definecolor{currentfill}{rgb}{0.000000,0.000000,1.000000}%
\pgfsetfillcolor{currentfill}%
\pgfsetlinewidth{0.000000pt}%
\definecolor{currentstroke}{rgb}{0.000000,0.000000,0.000000}%
\pgfsetstrokecolor{currentstroke}%
\pgfsetstrokeopacity{0.000000}%
\pgfsetdash{}{0pt}%
\pgfpathmoveto{\pgfqpoint{3.459376in}{0.613486in}}%
\pgfpathlineto{\pgfqpoint{3.469381in}{0.613486in}}%
\pgfpathlineto{\pgfqpoint{3.469381in}{2.595224in}}%
\pgfpathlineto{\pgfqpoint{3.459376in}{2.595224in}}%
\pgfpathlineto{\pgfqpoint{3.459376in}{0.613486in}}%
\pgfpathclose%
\pgfusepath{fill}%
\end{pgfscope}%
\begin{pgfscope}%
\pgfpathrectangle{\pgfqpoint{0.693757in}{0.613486in}}{\pgfqpoint{5.541243in}{3.963477in}}%
\pgfusepath{clip}%
\pgfsetbuttcap%
\pgfsetmiterjoin%
\definecolor{currentfill}{rgb}{0.000000,0.000000,1.000000}%
\pgfsetfillcolor{currentfill}%
\pgfsetlinewidth{0.000000pt}%
\definecolor{currentstroke}{rgb}{0.000000,0.000000,0.000000}%
\pgfsetstrokecolor{currentstroke}%
\pgfsetstrokeopacity{0.000000}%
\pgfsetdash{}{0pt}%
\pgfpathmoveto{\pgfqpoint{3.471882in}{0.613486in}}%
\pgfpathlineto{\pgfqpoint{3.481887in}{0.613486in}}%
\pgfpathlineto{\pgfqpoint{3.481887in}{2.611074in}}%
\pgfpathlineto{\pgfqpoint{3.471882in}{2.611074in}}%
\pgfpathlineto{\pgfqpoint{3.471882in}{0.613486in}}%
\pgfpathclose%
\pgfusepath{fill}%
\end{pgfscope}%
\begin{pgfscope}%
\pgfpathrectangle{\pgfqpoint{0.693757in}{0.613486in}}{\pgfqpoint{5.541243in}{3.963477in}}%
\pgfusepath{clip}%
\pgfsetbuttcap%
\pgfsetmiterjoin%
\definecolor{currentfill}{rgb}{0.000000,0.000000,1.000000}%
\pgfsetfillcolor{currentfill}%
\pgfsetlinewidth{0.000000pt}%
\definecolor{currentstroke}{rgb}{0.000000,0.000000,0.000000}%
\pgfsetstrokecolor{currentstroke}%
\pgfsetstrokeopacity{0.000000}%
\pgfsetdash{}{0pt}%
\pgfpathmoveto{\pgfqpoint{3.484388in}{0.613486in}}%
\pgfpathlineto{\pgfqpoint{3.494393in}{0.613486in}}%
\pgfpathlineto{\pgfqpoint{3.494393in}{1.604355in}}%
\pgfpathlineto{\pgfqpoint{3.484388in}{1.604355in}}%
\pgfpathlineto{\pgfqpoint{3.484388in}{0.613486in}}%
\pgfpathclose%
\pgfusepath{fill}%
\end{pgfscope}%
\begin{pgfscope}%
\pgfpathrectangle{\pgfqpoint{0.693757in}{0.613486in}}{\pgfqpoint{5.541243in}{3.963477in}}%
\pgfusepath{clip}%
\pgfsetbuttcap%
\pgfsetmiterjoin%
\definecolor{currentfill}{rgb}{0.000000,0.000000,1.000000}%
\pgfsetfillcolor{currentfill}%
\pgfsetlinewidth{0.000000pt}%
\definecolor{currentstroke}{rgb}{0.000000,0.000000,0.000000}%
\pgfsetstrokecolor{currentstroke}%
\pgfsetstrokeopacity{0.000000}%
\pgfsetdash{}{0pt}%
\pgfpathmoveto{\pgfqpoint{3.496895in}{0.613486in}}%
\pgfpathlineto{\pgfqpoint{3.506900in}{0.613486in}}%
\pgfpathlineto{\pgfqpoint{3.506900in}{1.612282in}}%
\pgfpathlineto{\pgfqpoint{3.496895in}{1.612282in}}%
\pgfpathlineto{\pgfqpoint{3.496895in}{0.613486in}}%
\pgfpathclose%
\pgfusepath{fill}%
\end{pgfscope}%
\begin{pgfscope}%
\pgfpathrectangle{\pgfqpoint{0.693757in}{0.613486in}}{\pgfqpoint{5.541243in}{3.963477in}}%
\pgfusepath{clip}%
\pgfsetbuttcap%
\pgfsetmiterjoin%
\definecolor{currentfill}{rgb}{0.000000,0.000000,1.000000}%
\pgfsetfillcolor{currentfill}%
\pgfsetlinewidth{0.000000pt}%
\definecolor{currentstroke}{rgb}{0.000000,0.000000,0.000000}%
\pgfsetstrokecolor{currentstroke}%
\pgfsetstrokeopacity{0.000000}%
\pgfsetdash{}{0pt}%
\pgfpathmoveto{\pgfqpoint{3.509401in}{0.613486in}}%
\pgfpathlineto{\pgfqpoint{3.519406in}{0.613486in}}%
\pgfpathlineto{\pgfqpoint{3.519406in}{2.595224in}}%
\pgfpathlineto{\pgfqpoint{3.509401in}{2.595224in}}%
\pgfpathlineto{\pgfqpoint{3.509401in}{0.613486in}}%
\pgfpathclose%
\pgfusepath{fill}%
\end{pgfscope}%
\begin{pgfscope}%
\pgfpathrectangle{\pgfqpoint{0.693757in}{0.613486in}}{\pgfqpoint{5.541243in}{3.963477in}}%
\pgfusepath{clip}%
\pgfsetbuttcap%
\pgfsetmiterjoin%
\definecolor{currentfill}{rgb}{0.000000,0.000000,1.000000}%
\pgfsetfillcolor{currentfill}%
\pgfsetlinewidth{0.000000pt}%
\definecolor{currentstroke}{rgb}{0.000000,0.000000,0.000000}%
\pgfsetstrokecolor{currentstroke}%
\pgfsetstrokeopacity{0.000000}%
\pgfsetdash{}{0pt}%
\pgfpathmoveto{\pgfqpoint{3.521907in}{0.613486in}}%
\pgfpathlineto{\pgfqpoint{3.531912in}{0.613486in}}%
\pgfpathlineto{\pgfqpoint{3.531912in}{2.611074in}}%
\pgfpathlineto{\pgfqpoint{3.521907in}{2.611074in}}%
\pgfpathlineto{\pgfqpoint{3.521907in}{0.613486in}}%
\pgfpathclose%
\pgfusepath{fill}%
\end{pgfscope}%
\begin{pgfscope}%
\pgfpathrectangle{\pgfqpoint{0.693757in}{0.613486in}}{\pgfqpoint{5.541243in}{3.963477in}}%
\pgfusepath{clip}%
\pgfsetbuttcap%
\pgfsetmiterjoin%
\definecolor{currentfill}{rgb}{0.000000,0.000000,1.000000}%
\pgfsetfillcolor{currentfill}%
\pgfsetlinewidth{0.000000pt}%
\definecolor{currentstroke}{rgb}{0.000000,0.000000,0.000000}%
\pgfsetstrokecolor{currentstroke}%
\pgfsetstrokeopacity{0.000000}%
\pgfsetdash{}{0pt}%
\pgfpathmoveto{\pgfqpoint{3.534413in}{0.613486in}}%
\pgfpathlineto{\pgfqpoint{3.544418in}{0.613486in}}%
\pgfpathlineto{\pgfqpoint{3.544418in}{1.604355in}}%
\pgfpathlineto{\pgfqpoint{3.534413in}{1.604355in}}%
\pgfpathlineto{\pgfqpoint{3.534413in}{0.613486in}}%
\pgfpathclose%
\pgfusepath{fill}%
\end{pgfscope}%
\begin{pgfscope}%
\pgfpathrectangle{\pgfqpoint{0.693757in}{0.613486in}}{\pgfqpoint{5.541243in}{3.963477in}}%
\pgfusepath{clip}%
\pgfsetbuttcap%
\pgfsetmiterjoin%
\definecolor{currentfill}{rgb}{0.000000,0.000000,1.000000}%
\pgfsetfillcolor{currentfill}%
\pgfsetlinewidth{0.000000pt}%
\definecolor{currentstroke}{rgb}{0.000000,0.000000,0.000000}%
\pgfsetstrokecolor{currentstroke}%
\pgfsetstrokeopacity{0.000000}%
\pgfsetdash{}{0pt}%
\pgfpathmoveto{\pgfqpoint{3.546919in}{0.613486in}}%
\pgfpathlineto{\pgfqpoint{3.556924in}{0.613486in}}%
\pgfpathlineto{\pgfqpoint{3.556924in}{1.612282in}}%
\pgfpathlineto{\pgfqpoint{3.546919in}{1.612282in}}%
\pgfpathlineto{\pgfqpoint{3.546919in}{0.613486in}}%
\pgfpathclose%
\pgfusepath{fill}%
\end{pgfscope}%
\begin{pgfscope}%
\pgfpathrectangle{\pgfqpoint{0.693757in}{0.613486in}}{\pgfqpoint{5.541243in}{3.963477in}}%
\pgfusepath{clip}%
\pgfsetbuttcap%
\pgfsetmiterjoin%
\definecolor{currentfill}{rgb}{0.000000,0.000000,1.000000}%
\pgfsetfillcolor{currentfill}%
\pgfsetlinewidth{0.000000pt}%
\definecolor{currentstroke}{rgb}{0.000000,0.000000,0.000000}%
\pgfsetstrokecolor{currentstroke}%
\pgfsetstrokeopacity{0.000000}%
\pgfsetdash{}{0pt}%
\pgfpathmoveto{\pgfqpoint{3.559426in}{0.613486in}}%
\pgfpathlineto{\pgfqpoint{3.569431in}{0.613486in}}%
\pgfpathlineto{\pgfqpoint{3.569431in}{2.595224in}}%
\pgfpathlineto{\pgfqpoint{3.559426in}{2.595224in}}%
\pgfpathlineto{\pgfqpoint{3.559426in}{0.613486in}}%
\pgfpathclose%
\pgfusepath{fill}%
\end{pgfscope}%
\begin{pgfscope}%
\pgfpathrectangle{\pgfqpoint{0.693757in}{0.613486in}}{\pgfqpoint{5.541243in}{3.963477in}}%
\pgfusepath{clip}%
\pgfsetbuttcap%
\pgfsetmiterjoin%
\definecolor{currentfill}{rgb}{0.000000,0.000000,1.000000}%
\pgfsetfillcolor{currentfill}%
\pgfsetlinewidth{0.000000pt}%
\definecolor{currentstroke}{rgb}{0.000000,0.000000,0.000000}%
\pgfsetstrokecolor{currentstroke}%
\pgfsetstrokeopacity{0.000000}%
\pgfsetdash{}{0pt}%
\pgfpathmoveto{\pgfqpoint{3.571932in}{0.613486in}}%
\pgfpathlineto{\pgfqpoint{3.581937in}{0.613486in}}%
\pgfpathlineto{\pgfqpoint{3.581937in}{2.611074in}}%
\pgfpathlineto{\pgfqpoint{3.571932in}{2.611074in}}%
\pgfpathlineto{\pgfqpoint{3.571932in}{0.613486in}}%
\pgfpathclose%
\pgfusepath{fill}%
\end{pgfscope}%
\begin{pgfscope}%
\pgfpathrectangle{\pgfqpoint{0.693757in}{0.613486in}}{\pgfqpoint{5.541243in}{3.963477in}}%
\pgfusepath{clip}%
\pgfsetbuttcap%
\pgfsetmiterjoin%
\definecolor{currentfill}{rgb}{0.000000,0.000000,1.000000}%
\pgfsetfillcolor{currentfill}%
\pgfsetlinewidth{0.000000pt}%
\definecolor{currentstroke}{rgb}{0.000000,0.000000,0.000000}%
\pgfsetstrokecolor{currentstroke}%
\pgfsetstrokeopacity{0.000000}%
\pgfsetdash{}{0pt}%
\pgfpathmoveto{\pgfqpoint{3.584438in}{0.613486in}}%
\pgfpathlineto{\pgfqpoint{3.594443in}{0.613486in}}%
\pgfpathlineto{\pgfqpoint{3.594443in}{1.604355in}}%
\pgfpathlineto{\pgfqpoint{3.584438in}{1.604355in}}%
\pgfpathlineto{\pgfqpoint{3.584438in}{0.613486in}}%
\pgfpathclose%
\pgfusepath{fill}%
\end{pgfscope}%
\begin{pgfscope}%
\pgfpathrectangle{\pgfqpoint{0.693757in}{0.613486in}}{\pgfqpoint{5.541243in}{3.963477in}}%
\pgfusepath{clip}%
\pgfsetbuttcap%
\pgfsetmiterjoin%
\definecolor{currentfill}{rgb}{0.000000,0.000000,1.000000}%
\pgfsetfillcolor{currentfill}%
\pgfsetlinewidth{0.000000pt}%
\definecolor{currentstroke}{rgb}{0.000000,0.000000,0.000000}%
\pgfsetstrokecolor{currentstroke}%
\pgfsetstrokeopacity{0.000000}%
\pgfsetdash{}{0pt}%
\pgfpathmoveto{\pgfqpoint{3.596944in}{0.613486in}}%
\pgfpathlineto{\pgfqpoint{3.606949in}{0.613486in}}%
\pgfpathlineto{\pgfqpoint{3.606949in}{1.612282in}}%
\pgfpathlineto{\pgfqpoint{3.596944in}{1.612282in}}%
\pgfpathlineto{\pgfqpoint{3.596944in}{0.613486in}}%
\pgfpathclose%
\pgfusepath{fill}%
\end{pgfscope}%
\begin{pgfscope}%
\pgfpathrectangle{\pgfqpoint{0.693757in}{0.613486in}}{\pgfqpoint{5.541243in}{3.963477in}}%
\pgfusepath{clip}%
\pgfsetbuttcap%
\pgfsetmiterjoin%
\definecolor{currentfill}{rgb}{0.000000,0.000000,1.000000}%
\pgfsetfillcolor{currentfill}%
\pgfsetlinewidth{0.000000pt}%
\definecolor{currentstroke}{rgb}{0.000000,0.000000,0.000000}%
\pgfsetstrokecolor{currentstroke}%
\pgfsetstrokeopacity{0.000000}%
\pgfsetdash{}{0pt}%
\pgfpathmoveto{\pgfqpoint{3.609450in}{0.613486in}}%
\pgfpathlineto{\pgfqpoint{3.619455in}{0.613486in}}%
\pgfpathlineto{\pgfqpoint{3.619455in}{2.595224in}}%
\pgfpathlineto{\pgfqpoint{3.609450in}{2.595224in}}%
\pgfpathlineto{\pgfqpoint{3.609450in}{0.613486in}}%
\pgfpathclose%
\pgfusepath{fill}%
\end{pgfscope}%
\begin{pgfscope}%
\pgfpathrectangle{\pgfqpoint{0.693757in}{0.613486in}}{\pgfqpoint{5.541243in}{3.963477in}}%
\pgfusepath{clip}%
\pgfsetbuttcap%
\pgfsetmiterjoin%
\definecolor{currentfill}{rgb}{0.000000,0.000000,1.000000}%
\pgfsetfillcolor{currentfill}%
\pgfsetlinewidth{0.000000pt}%
\definecolor{currentstroke}{rgb}{0.000000,0.000000,0.000000}%
\pgfsetstrokecolor{currentstroke}%
\pgfsetstrokeopacity{0.000000}%
\pgfsetdash{}{0pt}%
\pgfpathmoveto{\pgfqpoint{3.621957in}{0.613486in}}%
\pgfpathlineto{\pgfqpoint{3.631961in}{0.613486in}}%
\pgfpathlineto{\pgfqpoint{3.631961in}{2.611074in}}%
\pgfpathlineto{\pgfqpoint{3.621957in}{2.611074in}}%
\pgfpathlineto{\pgfqpoint{3.621957in}{0.613486in}}%
\pgfpathclose%
\pgfusepath{fill}%
\end{pgfscope}%
\begin{pgfscope}%
\pgfpathrectangle{\pgfqpoint{0.693757in}{0.613486in}}{\pgfqpoint{5.541243in}{3.963477in}}%
\pgfusepath{clip}%
\pgfsetbuttcap%
\pgfsetmiterjoin%
\definecolor{currentfill}{rgb}{0.000000,0.000000,1.000000}%
\pgfsetfillcolor{currentfill}%
\pgfsetlinewidth{0.000000pt}%
\definecolor{currentstroke}{rgb}{0.000000,0.000000,0.000000}%
\pgfsetstrokecolor{currentstroke}%
\pgfsetstrokeopacity{0.000000}%
\pgfsetdash{}{0pt}%
\pgfpathmoveto{\pgfqpoint{3.634463in}{0.613486in}}%
\pgfpathlineto{\pgfqpoint{3.644468in}{0.613486in}}%
\pgfpathlineto{\pgfqpoint{3.644468in}{1.604355in}}%
\pgfpathlineto{\pgfqpoint{3.634463in}{1.604355in}}%
\pgfpathlineto{\pgfqpoint{3.634463in}{0.613486in}}%
\pgfpathclose%
\pgfusepath{fill}%
\end{pgfscope}%
\begin{pgfscope}%
\pgfpathrectangle{\pgfqpoint{0.693757in}{0.613486in}}{\pgfqpoint{5.541243in}{3.963477in}}%
\pgfusepath{clip}%
\pgfsetbuttcap%
\pgfsetmiterjoin%
\definecolor{currentfill}{rgb}{0.000000,0.000000,1.000000}%
\pgfsetfillcolor{currentfill}%
\pgfsetlinewidth{0.000000pt}%
\definecolor{currentstroke}{rgb}{0.000000,0.000000,0.000000}%
\pgfsetstrokecolor{currentstroke}%
\pgfsetstrokeopacity{0.000000}%
\pgfsetdash{}{0pt}%
\pgfpathmoveto{\pgfqpoint{3.646969in}{0.613486in}}%
\pgfpathlineto{\pgfqpoint{3.656974in}{0.613486in}}%
\pgfpathlineto{\pgfqpoint{3.656974in}{1.612282in}}%
\pgfpathlineto{\pgfqpoint{3.646969in}{1.612282in}}%
\pgfpathlineto{\pgfqpoint{3.646969in}{0.613486in}}%
\pgfpathclose%
\pgfusepath{fill}%
\end{pgfscope}%
\begin{pgfscope}%
\pgfpathrectangle{\pgfqpoint{0.693757in}{0.613486in}}{\pgfqpoint{5.541243in}{3.963477in}}%
\pgfusepath{clip}%
\pgfsetbuttcap%
\pgfsetmiterjoin%
\definecolor{currentfill}{rgb}{0.000000,0.000000,1.000000}%
\pgfsetfillcolor{currentfill}%
\pgfsetlinewidth{0.000000pt}%
\definecolor{currentstroke}{rgb}{0.000000,0.000000,0.000000}%
\pgfsetstrokecolor{currentstroke}%
\pgfsetstrokeopacity{0.000000}%
\pgfsetdash{}{0pt}%
\pgfpathmoveto{\pgfqpoint{3.659475in}{0.613486in}}%
\pgfpathlineto{\pgfqpoint{3.669480in}{0.613486in}}%
\pgfpathlineto{\pgfqpoint{3.669480in}{2.595224in}}%
\pgfpathlineto{\pgfqpoint{3.659475in}{2.595224in}}%
\pgfpathlineto{\pgfqpoint{3.659475in}{0.613486in}}%
\pgfpathclose%
\pgfusepath{fill}%
\end{pgfscope}%
\begin{pgfscope}%
\pgfpathrectangle{\pgfqpoint{0.693757in}{0.613486in}}{\pgfqpoint{5.541243in}{3.963477in}}%
\pgfusepath{clip}%
\pgfsetbuttcap%
\pgfsetmiterjoin%
\definecolor{currentfill}{rgb}{0.000000,0.000000,1.000000}%
\pgfsetfillcolor{currentfill}%
\pgfsetlinewidth{0.000000pt}%
\definecolor{currentstroke}{rgb}{0.000000,0.000000,0.000000}%
\pgfsetstrokecolor{currentstroke}%
\pgfsetstrokeopacity{0.000000}%
\pgfsetdash{}{0pt}%
\pgfpathmoveto{\pgfqpoint{3.671981in}{0.613486in}}%
\pgfpathlineto{\pgfqpoint{3.681986in}{0.613486in}}%
\pgfpathlineto{\pgfqpoint{3.681986in}{2.611074in}}%
\pgfpathlineto{\pgfqpoint{3.671981in}{2.611074in}}%
\pgfpathlineto{\pgfqpoint{3.671981in}{0.613486in}}%
\pgfpathclose%
\pgfusepath{fill}%
\end{pgfscope}%
\begin{pgfscope}%
\pgfpathrectangle{\pgfqpoint{0.693757in}{0.613486in}}{\pgfqpoint{5.541243in}{3.963477in}}%
\pgfusepath{clip}%
\pgfsetbuttcap%
\pgfsetmiterjoin%
\definecolor{currentfill}{rgb}{0.000000,0.000000,1.000000}%
\pgfsetfillcolor{currentfill}%
\pgfsetlinewidth{0.000000pt}%
\definecolor{currentstroke}{rgb}{0.000000,0.000000,0.000000}%
\pgfsetstrokecolor{currentstroke}%
\pgfsetstrokeopacity{0.000000}%
\pgfsetdash{}{0pt}%
\pgfpathmoveto{\pgfqpoint{3.684487in}{0.613486in}}%
\pgfpathlineto{\pgfqpoint{3.694492in}{0.613486in}}%
\pgfpathlineto{\pgfqpoint{3.694492in}{1.604355in}}%
\pgfpathlineto{\pgfqpoint{3.684487in}{1.604355in}}%
\pgfpathlineto{\pgfqpoint{3.684487in}{0.613486in}}%
\pgfpathclose%
\pgfusepath{fill}%
\end{pgfscope}%
\begin{pgfscope}%
\pgfpathrectangle{\pgfqpoint{0.693757in}{0.613486in}}{\pgfqpoint{5.541243in}{3.963477in}}%
\pgfusepath{clip}%
\pgfsetbuttcap%
\pgfsetmiterjoin%
\definecolor{currentfill}{rgb}{0.000000,0.000000,1.000000}%
\pgfsetfillcolor{currentfill}%
\pgfsetlinewidth{0.000000pt}%
\definecolor{currentstroke}{rgb}{0.000000,0.000000,0.000000}%
\pgfsetstrokecolor{currentstroke}%
\pgfsetstrokeopacity{0.000000}%
\pgfsetdash{}{0pt}%
\pgfpathmoveto{\pgfqpoint{3.696994in}{0.613486in}}%
\pgfpathlineto{\pgfqpoint{3.706999in}{0.613486in}}%
\pgfpathlineto{\pgfqpoint{3.706999in}{1.612282in}}%
\pgfpathlineto{\pgfqpoint{3.696994in}{1.612282in}}%
\pgfpathlineto{\pgfqpoint{3.696994in}{0.613486in}}%
\pgfpathclose%
\pgfusepath{fill}%
\end{pgfscope}%
\begin{pgfscope}%
\pgfpathrectangle{\pgfqpoint{0.693757in}{0.613486in}}{\pgfqpoint{5.541243in}{3.963477in}}%
\pgfusepath{clip}%
\pgfsetbuttcap%
\pgfsetmiterjoin%
\definecolor{currentfill}{rgb}{0.000000,0.000000,1.000000}%
\pgfsetfillcolor{currentfill}%
\pgfsetlinewidth{0.000000pt}%
\definecolor{currentstroke}{rgb}{0.000000,0.000000,0.000000}%
\pgfsetstrokecolor{currentstroke}%
\pgfsetstrokeopacity{0.000000}%
\pgfsetdash{}{0pt}%
\pgfpathmoveto{\pgfqpoint{3.709500in}{0.613486in}}%
\pgfpathlineto{\pgfqpoint{3.719505in}{0.613486in}}%
\pgfpathlineto{\pgfqpoint{3.719505in}{2.595224in}}%
\pgfpathlineto{\pgfqpoint{3.709500in}{2.595224in}}%
\pgfpathlineto{\pgfqpoint{3.709500in}{0.613486in}}%
\pgfpathclose%
\pgfusepath{fill}%
\end{pgfscope}%
\begin{pgfscope}%
\pgfpathrectangle{\pgfqpoint{0.693757in}{0.613486in}}{\pgfqpoint{5.541243in}{3.963477in}}%
\pgfusepath{clip}%
\pgfsetbuttcap%
\pgfsetmiterjoin%
\definecolor{currentfill}{rgb}{0.000000,0.000000,1.000000}%
\pgfsetfillcolor{currentfill}%
\pgfsetlinewidth{0.000000pt}%
\definecolor{currentstroke}{rgb}{0.000000,0.000000,0.000000}%
\pgfsetstrokecolor{currentstroke}%
\pgfsetstrokeopacity{0.000000}%
\pgfsetdash{}{0pt}%
\pgfpathmoveto{\pgfqpoint{3.722006in}{0.613486in}}%
\pgfpathlineto{\pgfqpoint{3.732011in}{0.613486in}}%
\pgfpathlineto{\pgfqpoint{3.732011in}{2.611074in}}%
\pgfpathlineto{\pgfqpoint{3.722006in}{2.611074in}}%
\pgfpathlineto{\pgfqpoint{3.722006in}{0.613486in}}%
\pgfpathclose%
\pgfusepath{fill}%
\end{pgfscope}%
\begin{pgfscope}%
\pgfpathrectangle{\pgfqpoint{0.693757in}{0.613486in}}{\pgfqpoint{5.541243in}{3.963477in}}%
\pgfusepath{clip}%
\pgfsetbuttcap%
\pgfsetmiterjoin%
\definecolor{currentfill}{rgb}{0.000000,0.000000,1.000000}%
\pgfsetfillcolor{currentfill}%
\pgfsetlinewidth{0.000000pt}%
\definecolor{currentstroke}{rgb}{0.000000,0.000000,0.000000}%
\pgfsetstrokecolor{currentstroke}%
\pgfsetstrokeopacity{0.000000}%
\pgfsetdash{}{0pt}%
\pgfpathmoveto{\pgfqpoint{3.734512in}{0.613486in}}%
\pgfpathlineto{\pgfqpoint{3.744517in}{0.613486in}}%
\pgfpathlineto{\pgfqpoint{3.744517in}{1.604355in}}%
\pgfpathlineto{\pgfqpoint{3.734512in}{1.604355in}}%
\pgfpathlineto{\pgfqpoint{3.734512in}{0.613486in}}%
\pgfpathclose%
\pgfusepath{fill}%
\end{pgfscope}%
\begin{pgfscope}%
\pgfpathrectangle{\pgfqpoint{0.693757in}{0.613486in}}{\pgfqpoint{5.541243in}{3.963477in}}%
\pgfusepath{clip}%
\pgfsetbuttcap%
\pgfsetmiterjoin%
\definecolor{currentfill}{rgb}{0.000000,0.000000,1.000000}%
\pgfsetfillcolor{currentfill}%
\pgfsetlinewidth{0.000000pt}%
\definecolor{currentstroke}{rgb}{0.000000,0.000000,0.000000}%
\pgfsetstrokecolor{currentstroke}%
\pgfsetstrokeopacity{0.000000}%
\pgfsetdash{}{0pt}%
\pgfpathmoveto{\pgfqpoint{3.747018in}{0.613486in}}%
\pgfpathlineto{\pgfqpoint{3.757023in}{0.613486in}}%
\pgfpathlineto{\pgfqpoint{3.757023in}{1.612282in}}%
\pgfpathlineto{\pgfqpoint{3.747018in}{1.612282in}}%
\pgfpathlineto{\pgfqpoint{3.747018in}{0.613486in}}%
\pgfpathclose%
\pgfusepath{fill}%
\end{pgfscope}%
\begin{pgfscope}%
\pgfpathrectangle{\pgfqpoint{0.693757in}{0.613486in}}{\pgfqpoint{5.541243in}{3.963477in}}%
\pgfusepath{clip}%
\pgfsetbuttcap%
\pgfsetmiterjoin%
\definecolor{currentfill}{rgb}{0.000000,0.000000,1.000000}%
\pgfsetfillcolor{currentfill}%
\pgfsetlinewidth{0.000000pt}%
\definecolor{currentstroke}{rgb}{0.000000,0.000000,0.000000}%
\pgfsetstrokecolor{currentstroke}%
\pgfsetstrokeopacity{0.000000}%
\pgfsetdash{}{0pt}%
\pgfpathmoveto{\pgfqpoint{3.759525in}{0.613486in}}%
\pgfpathlineto{\pgfqpoint{3.769530in}{0.613486in}}%
\pgfpathlineto{\pgfqpoint{3.769530in}{2.595224in}}%
\pgfpathlineto{\pgfqpoint{3.759525in}{2.595224in}}%
\pgfpathlineto{\pgfqpoint{3.759525in}{0.613486in}}%
\pgfpathclose%
\pgfusepath{fill}%
\end{pgfscope}%
\begin{pgfscope}%
\pgfpathrectangle{\pgfqpoint{0.693757in}{0.613486in}}{\pgfqpoint{5.541243in}{3.963477in}}%
\pgfusepath{clip}%
\pgfsetbuttcap%
\pgfsetmiterjoin%
\definecolor{currentfill}{rgb}{0.000000,0.000000,1.000000}%
\pgfsetfillcolor{currentfill}%
\pgfsetlinewidth{0.000000pt}%
\definecolor{currentstroke}{rgb}{0.000000,0.000000,0.000000}%
\pgfsetstrokecolor{currentstroke}%
\pgfsetstrokeopacity{0.000000}%
\pgfsetdash{}{0pt}%
\pgfpathmoveto{\pgfqpoint{3.772031in}{0.613486in}}%
\pgfpathlineto{\pgfqpoint{3.782036in}{0.613486in}}%
\pgfpathlineto{\pgfqpoint{3.782036in}{2.611074in}}%
\pgfpathlineto{\pgfqpoint{3.772031in}{2.611074in}}%
\pgfpathlineto{\pgfqpoint{3.772031in}{0.613486in}}%
\pgfpathclose%
\pgfusepath{fill}%
\end{pgfscope}%
\begin{pgfscope}%
\pgfpathrectangle{\pgfqpoint{0.693757in}{0.613486in}}{\pgfqpoint{5.541243in}{3.963477in}}%
\pgfusepath{clip}%
\pgfsetbuttcap%
\pgfsetmiterjoin%
\definecolor{currentfill}{rgb}{0.000000,0.000000,1.000000}%
\pgfsetfillcolor{currentfill}%
\pgfsetlinewidth{0.000000pt}%
\definecolor{currentstroke}{rgb}{0.000000,0.000000,0.000000}%
\pgfsetstrokecolor{currentstroke}%
\pgfsetstrokeopacity{0.000000}%
\pgfsetdash{}{0pt}%
\pgfpathmoveto{\pgfqpoint{3.784537in}{0.613486in}}%
\pgfpathlineto{\pgfqpoint{3.794542in}{0.613486in}}%
\pgfpathlineto{\pgfqpoint{3.794542in}{1.604355in}}%
\pgfpathlineto{\pgfqpoint{3.784537in}{1.604355in}}%
\pgfpathlineto{\pgfqpoint{3.784537in}{0.613486in}}%
\pgfpathclose%
\pgfusepath{fill}%
\end{pgfscope}%
\begin{pgfscope}%
\pgfpathrectangle{\pgfqpoint{0.693757in}{0.613486in}}{\pgfqpoint{5.541243in}{3.963477in}}%
\pgfusepath{clip}%
\pgfsetbuttcap%
\pgfsetmiterjoin%
\definecolor{currentfill}{rgb}{0.000000,0.000000,1.000000}%
\pgfsetfillcolor{currentfill}%
\pgfsetlinewidth{0.000000pt}%
\definecolor{currentstroke}{rgb}{0.000000,0.000000,0.000000}%
\pgfsetstrokecolor{currentstroke}%
\pgfsetstrokeopacity{0.000000}%
\pgfsetdash{}{0pt}%
\pgfpathmoveto{\pgfqpoint{3.797043in}{0.613486in}}%
\pgfpathlineto{\pgfqpoint{3.807048in}{0.613486in}}%
\pgfpathlineto{\pgfqpoint{3.807048in}{1.612282in}}%
\pgfpathlineto{\pgfqpoint{3.797043in}{1.612282in}}%
\pgfpathlineto{\pgfqpoint{3.797043in}{0.613486in}}%
\pgfpathclose%
\pgfusepath{fill}%
\end{pgfscope}%
\begin{pgfscope}%
\pgfpathrectangle{\pgfqpoint{0.693757in}{0.613486in}}{\pgfqpoint{5.541243in}{3.963477in}}%
\pgfusepath{clip}%
\pgfsetbuttcap%
\pgfsetmiterjoin%
\definecolor{currentfill}{rgb}{0.000000,0.000000,1.000000}%
\pgfsetfillcolor{currentfill}%
\pgfsetlinewidth{0.000000pt}%
\definecolor{currentstroke}{rgb}{0.000000,0.000000,0.000000}%
\pgfsetstrokecolor{currentstroke}%
\pgfsetstrokeopacity{0.000000}%
\pgfsetdash{}{0pt}%
\pgfpathmoveto{\pgfqpoint{3.809549in}{0.613486in}}%
\pgfpathlineto{\pgfqpoint{3.819554in}{0.613486in}}%
\pgfpathlineto{\pgfqpoint{3.819554in}{2.595224in}}%
\pgfpathlineto{\pgfqpoint{3.809549in}{2.595224in}}%
\pgfpathlineto{\pgfqpoint{3.809549in}{0.613486in}}%
\pgfpathclose%
\pgfusepath{fill}%
\end{pgfscope}%
\begin{pgfscope}%
\pgfpathrectangle{\pgfqpoint{0.693757in}{0.613486in}}{\pgfqpoint{5.541243in}{3.963477in}}%
\pgfusepath{clip}%
\pgfsetbuttcap%
\pgfsetmiterjoin%
\definecolor{currentfill}{rgb}{0.000000,0.000000,1.000000}%
\pgfsetfillcolor{currentfill}%
\pgfsetlinewidth{0.000000pt}%
\definecolor{currentstroke}{rgb}{0.000000,0.000000,0.000000}%
\pgfsetstrokecolor{currentstroke}%
\pgfsetstrokeopacity{0.000000}%
\pgfsetdash{}{0pt}%
\pgfpathmoveto{\pgfqpoint{3.822056in}{0.613486in}}%
\pgfpathlineto{\pgfqpoint{3.832061in}{0.613486in}}%
\pgfpathlineto{\pgfqpoint{3.832061in}{2.611074in}}%
\pgfpathlineto{\pgfqpoint{3.822056in}{2.611074in}}%
\pgfpathlineto{\pgfqpoint{3.822056in}{0.613486in}}%
\pgfpathclose%
\pgfusepath{fill}%
\end{pgfscope}%
\begin{pgfscope}%
\pgfpathrectangle{\pgfqpoint{0.693757in}{0.613486in}}{\pgfqpoint{5.541243in}{3.963477in}}%
\pgfusepath{clip}%
\pgfsetbuttcap%
\pgfsetmiterjoin%
\definecolor{currentfill}{rgb}{0.000000,0.000000,1.000000}%
\pgfsetfillcolor{currentfill}%
\pgfsetlinewidth{0.000000pt}%
\definecolor{currentstroke}{rgb}{0.000000,0.000000,0.000000}%
\pgfsetstrokecolor{currentstroke}%
\pgfsetstrokeopacity{0.000000}%
\pgfsetdash{}{0pt}%
\pgfpathmoveto{\pgfqpoint{3.834562in}{0.613486in}}%
\pgfpathlineto{\pgfqpoint{3.844567in}{0.613486in}}%
\pgfpathlineto{\pgfqpoint{3.844567in}{1.604355in}}%
\pgfpathlineto{\pgfqpoint{3.834562in}{1.604355in}}%
\pgfpathlineto{\pgfqpoint{3.834562in}{0.613486in}}%
\pgfpathclose%
\pgfusepath{fill}%
\end{pgfscope}%
\begin{pgfscope}%
\pgfpathrectangle{\pgfqpoint{0.693757in}{0.613486in}}{\pgfqpoint{5.541243in}{3.963477in}}%
\pgfusepath{clip}%
\pgfsetbuttcap%
\pgfsetmiterjoin%
\definecolor{currentfill}{rgb}{0.000000,0.000000,1.000000}%
\pgfsetfillcolor{currentfill}%
\pgfsetlinewidth{0.000000pt}%
\definecolor{currentstroke}{rgb}{0.000000,0.000000,0.000000}%
\pgfsetstrokecolor{currentstroke}%
\pgfsetstrokeopacity{0.000000}%
\pgfsetdash{}{0pt}%
\pgfpathmoveto{\pgfqpoint{3.847068in}{0.613486in}}%
\pgfpathlineto{\pgfqpoint{3.857073in}{0.613486in}}%
\pgfpathlineto{\pgfqpoint{3.857073in}{1.612282in}}%
\pgfpathlineto{\pgfqpoint{3.847068in}{1.612282in}}%
\pgfpathlineto{\pgfqpoint{3.847068in}{0.613486in}}%
\pgfpathclose%
\pgfusepath{fill}%
\end{pgfscope}%
\begin{pgfscope}%
\pgfpathrectangle{\pgfqpoint{0.693757in}{0.613486in}}{\pgfqpoint{5.541243in}{3.963477in}}%
\pgfusepath{clip}%
\pgfsetbuttcap%
\pgfsetmiterjoin%
\definecolor{currentfill}{rgb}{0.000000,0.000000,1.000000}%
\pgfsetfillcolor{currentfill}%
\pgfsetlinewidth{0.000000pt}%
\definecolor{currentstroke}{rgb}{0.000000,0.000000,0.000000}%
\pgfsetstrokecolor{currentstroke}%
\pgfsetstrokeopacity{0.000000}%
\pgfsetdash{}{0pt}%
\pgfpathmoveto{\pgfqpoint{3.859574in}{0.613486in}}%
\pgfpathlineto{\pgfqpoint{3.869579in}{0.613486in}}%
\pgfpathlineto{\pgfqpoint{3.869579in}{2.595224in}}%
\pgfpathlineto{\pgfqpoint{3.859574in}{2.595224in}}%
\pgfpathlineto{\pgfqpoint{3.859574in}{0.613486in}}%
\pgfpathclose%
\pgfusepath{fill}%
\end{pgfscope}%
\begin{pgfscope}%
\pgfpathrectangle{\pgfqpoint{0.693757in}{0.613486in}}{\pgfqpoint{5.541243in}{3.963477in}}%
\pgfusepath{clip}%
\pgfsetbuttcap%
\pgfsetmiterjoin%
\definecolor{currentfill}{rgb}{0.000000,0.000000,1.000000}%
\pgfsetfillcolor{currentfill}%
\pgfsetlinewidth{0.000000pt}%
\definecolor{currentstroke}{rgb}{0.000000,0.000000,0.000000}%
\pgfsetstrokecolor{currentstroke}%
\pgfsetstrokeopacity{0.000000}%
\pgfsetdash{}{0pt}%
\pgfpathmoveto{\pgfqpoint{3.872080in}{0.613486in}}%
\pgfpathlineto{\pgfqpoint{3.882085in}{0.613486in}}%
\pgfpathlineto{\pgfqpoint{3.882085in}{2.611074in}}%
\pgfpathlineto{\pgfqpoint{3.872080in}{2.611074in}}%
\pgfpathlineto{\pgfqpoint{3.872080in}{0.613486in}}%
\pgfpathclose%
\pgfusepath{fill}%
\end{pgfscope}%
\begin{pgfscope}%
\pgfpathrectangle{\pgfqpoint{0.693757in}{0.613486in}}{\pgfqpoint{5.541243in}{3.963477in}}%
\pgfusepath{clip}%
\pgfsetbuttcap%
\pgfsetmiterjoin%
\definecolor{currentfill}{rgb}{0.000000,0.000000,1.000000}%
\pgfsetfillcolor{currentfill}%
\pgfsetlinewidth{0.000000pt}%
\definecolor{currentstroke}{rgb}{0.000000,0.000000,0.000000}%
\pgfsetstrokecolor{currentstroke}%
\pgfsetstrokeopacity{0.000000}%
\pgfsetdash{}{0pt}%
\pgfpathmoveto{\pgfqpoint{3.884587in}{0.613486in}}%
\pgfpathlineto{\pgfqpoint{3.894591in}{0.613486in}}%
\pgfpathlineto{\pgfqpoint{3.894591in}{1.604355in}}%
\pgfpathlineto{\pgfqpoint{3.884587in}{1.604355in}}%
\pgfpathlineto{\pgfqpoint{3.884587in}{0.613486in}}%
\pgfpathclose%
\pgfusepath{fill}%
\end{pgfscope}%
\begin{pgfscope}%
\pgfpathrectangle{\pgfqpoint{0.693757in}{0.613486in}}{\pgfqpoint{5.541243in}{3.963477in}}%
\pgfusepath{clip}%
\pgfsetbuttcap%
\pgfsetmiterjoin%
\definecolor{currentfill}{rgb}{0.000000,0.000000,1.000000}%
\pgfsetfillcolor{currentfill}%
\pgfsetlinewidth{0.000000pt}%
\definecolor{currentstroke}{rgb}{0.000000,0.000000,0.000000}%
\pgfsetstrokecolor{currentstroke}%
\pgfsetstrokeopacity{0.000000}%
\pgfsetdash{}{0pt}%
\pgfpathmoveto{\pgfqpoint{3.897093in}{0.613486in}}%
\pgfpathlineto{\pgfqpoint{3.907098in}{0.613486in}}%
\pgfpathlineto{\pgfqpoint{3.907098in}{1.612282in}}%
\pgfpathlineto{\pgfqpoint{3.897093in}{1.612282in}}%
\pgfpathlineto{\pgfqpoint{3.897093in}{0.613486in}}%
\pgfpathclose%
\pgfusepath{fill}%
\end{pgfscope}%
\begin{pgfscope}%
\pgfpathrectangle{\pgfqpoint{0.693757in}{0.613486in}}{\pgfqpoint{5.541243in}{3.963477in}}%
\pgfusepath{clip}%
\pgfsetbuttcap%
\pgfsetmiterjoin%
\definecolor{currentfill}{rgb}{0.000000,0.000000,1.000000}%
\pgfsetfillcolor{currentfill}%
\pgfsetlinewidth{0.000000pt}%
\definecolor{currentstroke}{rgb}{0.000000,0.000000,0.000000}%
\pgfsetstrokecolor{currentstroke}%
\pgfsetstrokeopacity{0.000000}%
\pgfsetdash{}{0pt}%
\pgfpathmoveto{\pgfqpoint{3.909599in}{0.613486in}}%
\pgfpathlineto{\pgfqpoint{3.919604in}{0.613486in}}%
\pgfpathlineto{\pgfqpoint{3.919604in}{2.595224in}}%
\pgfpathlineto{\pgfqpoint{3.909599in}{2.595224in}}%
\pgfpathlineto{\pgfqpoint{3.909599in}{0.613486in}}%
\pgfpathclose%
\pgfusepath{fill}%
\end{pgfscope}%
\begin{pgfscope}%
\pgfpathrectangle{\pgfqpoint{0.693757in}{0.613486in}}{\pgfqpoint{5.541243in}{3.963477in}}%
\pgfusepath{clip}%
\pgfsetbuttcap%
\pgfsetmiterjoin%
\definecolor{currentfill}{rgb}{0.000000,0.000000,1.000000}%
\pgfsetfillcolor{currentfill}%
\pgfsetlinewidth{0.000000pt}%
\definecolor{currentstroke}{rgb}{0.000000,0.000000,0.000000}%
\pgfsetstrokecolor{currentstroke}%
\pgfsetstrokeopacity{0.000000}%
\pgfsetdash{}{0pt}%
\pgfpathmoveto{\pgfqpoint{3.922105in}{0.613486in}}%
\pgfpathlineto{\pgfqpoint{3.932110in}{0.613486in}}%
\pgfpathlineto{\pgfqpoint{3.932110in}{2.611074in}}%
\pgfpathlineto{\pgfqpoint{3.922105in}{2.611074in}}%
\pgfpathlineto{\pgfqpoint{3.922105in}{0.613486in}}%
\pgfpathclose%
\pgfusepath{fill}%
\end{pgfscope}%
\begin{pgfscope}%
\pgfpathrectangle{\pgfqpoint{0.693757in}{0.613486in}}{\pgfqpoint{5.541243in}{3.963477in}}%
\pgfusepath{clip}%
\pgfsetbuttcap%
\pgfsetmiterjoin%
\definecolor{currentfill}{rgb}{0.000000,0.000000,1.000000}%
\pgfsetfillcolor{currentfill}%
\pgfsetlinewidth{0.000000pt}%
\definecolor{currentstroke}{rgb}{0.000000,0.000000,0.000000}%
\pgfsetstrokecolor{currentstroke}%
\pgfsetstrokeopacity{0.000000}%
\pgfsetdash{}{0pt}%
\pgfpathmoveto{\pgfqpoint{3.934611in}{0.613486in}}%
\pgfpathlineto{\pgfqpoint{3.944616in}{0.613486in}}%
\pgfpathlineto{\pgfqpoint{3.944616in}{1.604355in}}%
\pgfpathlineto{\pgfqpoint{3.934611in}{1.604355in}}%
\pgfpathlineto{\pgfqpoint{3.934611in}{0.613486in}}%
\pgfpathclose%
\pgfusepath{fill}%
\end{pgfscope}%
\begin{pgfscope}%
\pgfpathrectangle{\pgfqpoint{0.693757in}{0.613486in}}{\pgfqpoint{5.541243in}{3.963477in}}%
\pgfusepath{clip}%
\pgfsetbuttcap%
\pgfsetmiterjoin%
\definecolor{currentfill}{rgb}{0.000000,0.000000,1.000000}%
\pgfsetfillcolor{currentfill}%
\pgfsetlinewidth{0.000000pt}%
\definecolor{currentstroke}{rgb}{0.000000,0.000000,0.000000}%
\pgfsetstrokecolor{currentstroke}%
\pgfsetstrokeopacity{0.000000}%
\pgfsetdash{}{0pt}%
\pgfpathmoveto{\pgfqpoint{3.947117in}{0.613486in}}%
\pgfpathlineto{\pgfqpoint{3.957122in}{0.613486in}}%
\pgfpathlineto{\pgfqpoint{3.957122in}{1.612282in}}%
\pgfpathlineto{\pgfqpoint{3.947117in}{1.612282in}}%
\pgfpathlineto{\pgfqpoint{3.947117in}{0.613486in}}%
\pgfpathclose%
\pgfusepath{fill}%
\end{pgfscope}%
\begin{pgfscope}%
\pgfpathrectangle{\pgfqpoint{0.693757in}{0.613486in}}{\pgfqpoint{5.541243in}{3.963477in}}%
\pgfusepath{clip}%
\pgfsetbuttcap%
\pgfsetmiterjoin%
\definecolor{currentfill}{rgb}{0.000000,0.000000,1.000000}%
\pgfsetfillcolor{currentfill}%
\pgfsetlinewidth{0.000000pt}%
\definecolor{currentstroke}{rgb}{0.000000,0.000000,0.000000}%
\pgfsetstrokecolor{currentstroke}%
\pgfsetstrokeopacity{0.000000}%
\pgfsetdash{}{0pt}%
\pgfpathmoveto{\pgfqpoint{3.959624in}{0.613486in}}%
\pgfpathlineto{\pgfqpoint{3.969629in}{0.613486in}}%
\pgfpathlineto{\pgfqpoint{3.969629in}{2.595224in}}%
\pgfpathlineto{\pgfqpoint{3.959624in}{2.595224in}}%
\pgfpathlineto{\pgfqpoint{3.959624in}{0.613486in}}%
\pgfpathclose%
\pgfusepath{fill}%
\end{pgfscope}%
\begin{pgfscope}%
\pgfpathrectangle{\pgfqpoint{0.693757in}{0.613486in}}{\pgfqpoint{5.541243in}{3.963477in}}%
\pgfusepath{clip}%
\pgfsetbuttcap%
\pgfsetmiterjoin%
\definecolor{currentfill}{rgb}{0.000000,0.000000,1.000000}%
\pgfsetfillcolor{currentfill}%
\pgfsetlinewidth{0.000000pt}%
\definecolor{currentstroke}{rgb}{0.000000,0.000000,0.000000}%
\pgfsetstrokecolor{currentstroke}%
\pgfsetstrokeopacity{0.000000}%
\pgfsetdash{}{0pt}%
\pgfpathmoveto{\pgfqpoint{3.972130in}{0.613486in}}%
\pgfpathlineto{\pgfqpoint{3.982135in}{0.613486in}}%
\pgfpathlineto{\pgfqpoint{3.982135in}{2.611074in}}%
\pgfpathlineto{\pgfqpoint{3.972130in}{2.611074in}}%
\pgfpathlineto{\pgfqpoint{3.972130in}{0.613486in}}%
\pgfpathclose%
\pgfusepath{fill}%
\end{pgfscope}%
\begin{pgfscope}%
\pgfpathrectangle{\pgfqpoint{0.693757in}{0.613486in}}{\pgfqpoint{5.541243in}{3.963477in}}%
\pgfusepath{clip}%
\pgfsetbuttcap%
\pgfsetmiterjoin%
\definecolor{currentfill}{rgb}{0.000000,0.000000,1.000000}%
\pgfsetfillcolor{currentfill}%
\pgfsetlinewidth{0.000000pt}%
\definecolor{currentstroke}{rgb}{0.000000,0.000000,0.000000}%
\pgfsetstrokecolor{currentstroke}%
\pgfsetstrokeopacity{0.000000}%
\pgfsetdash{}{0pt}%
\pgfpathmoveto{\pgfqpoint{3.984636in}{0.613486in}}%
\pgfpathlineto{\pgfqpoint{3.994641in}{0.613486in}}%
\pgfpathlineto{\pgfqpoint{3.994641in}{1.604355in}}%
\pgfpathlineto{\pgfqpoint{3.984636in}{1.604355in}}%
\pgfpathlineto{\pgfqpoint{3.984636in}{0.613486in}}%
\pgfpathclose%
\pgfusepath{fill}%
\end{pgfscope}%
\begin{pgfscope}%
\pgfpathrectangle{\pgfqpoint{0.693757in}{0.613486in}}{\pgfqpoint{5.541243in}{3.963477in}}%
\pgfusepath{clip}%
\pgfsetbuttcap%
\pgfsetmiterjoin%
\definecolor{currentfill}{rgb}{0.000000,0.000000,1.000000}%
\pgfsetfillcolor{currentfill}%
\pgfsetlinewidth{0.000000pt}%
\definecolor{currentstroke}{rgb}{0.000000,0.000000,0.000000}%
\pgfsetstrokecolor{currentstroke}%
\pgfsetstrokeopacity{0.000000}%
\pgfsetdash{}{0pt}%
\pgfpathmoveto{\pgfqpoint{3.997142in}{0.613486in}}%
\pgfpathlineto{\pgfqpoint{4.007147in}{0.613486in}}%
\pgfpathlineto{\pgfqpoint{4.007147in}{1.612282in}}%
\pgfpathlineto{\pgfqpoint{3.997142in}{1.612282in}}%
\pgfpathlineto{\pgfqpoint{3.997142in}{0.613486in}}%
\pgfpathclose%
\pgfusepath{fill}%
\end{pgfscope}%
\begin{pgfscope}%
\pgfpathrectangle{\pgfqpoint{0.693757in}{0.613486in}}{\pgfqpoint{5.541243in}{3.963477in}}%
\pgfusepath{clip}%
\pgfsetbuttcap%
\pgfsetmiterjoin%
\definecolor{currentfill}{rgb}{0.000000,0.000000,1.000000}%
\pgfsetfillcolor{currentfill}%
\pgfsetlinewidth{0.000000pt}%
\definecolor{currentstroke}{rgb}{0.000000,0.000000,0.000000}%
\pgfsetstrokecolor{currentstroke}%
\pgfsetstrokeopacity{0.000000}%
\pgfsetdash{}{0pt}%
\pgfpathmoveto{\pgfqpoint{4.009648in}{0.613486in}}%
\pgfpathlineto{\pgfqpoint{4.019653in}{0.613486in}}%
\pgfpathlineto{\pgfqpoint{4.019653in}{2.595224in}}%
\pgfpathlineto{\pgfqpoint{4.009648in}{2.595224in}}%
\pgfpathlineto{\pgfqpoint{4.009648in}{0.613486in}}%
\pgfpathclose%
\pgfusepath{fill}%
\end{pgfscope}%
\begin{pgfscope}%
\pgfpathrectangle{\pgfqpoint{0.693757in}{0.613486in}}{\pgfqpoint{5.541243in}{3.963477in}}%
\pgfusepath{clip}%
\pgfsetbuttcap%
\pgfsetmiterjoin%
\definecolor{currentfill}{rgb}{0.000000,0.000000,1.000000}%
\pgfsetfillcolor{currentfill}%
\pgfsetlinewidth{0.000000pt}%
\definecolor{currentstroke}{rgb}{0.000000,0.000000,0.000000}%
\pgfsetstrokecolor{currentstroke}%
\pgfsetstrokeopacity{0.000000}%
\pgfsetdash{}{0pt}%
\pgfpathmoveto{\pgfqpoint{4.022155in}{0.613486in}}%
\pgfpathlineto{\pgfqpoint{4.032160in}{0.613486in}}%
\pgfpathlineto{\pgfqpoint{4.032160in}{2.611074in}}%
\pgfpathlineto{\pgfqpoint{4.022155in}{2.611074in}}%
\pgfpathlineto{\pgfqpoint{4.022155in}{0.613486in}}%
\pgfpathclose%
\pgfusepath{fill}%
\end{pgfscope}%
\begin{pgfscope}%
\pgfpathrectangle{\pgfqpoint{0.693757in}{0.613486in}}{\pgfqpoint{5.541243in}{3.963477in}}%
\pgfusepath{clip}%
\pgfsetbuttcap%
\pgfsetmiterjoin%
\definecolor{currentfill}{rgb}{0.000000,0.000000,1.000000}%
\pgfsetfillcolor{currentfill}%
\pgfsetlinewidth{0.000000pt}%
\definecolor{currentstroke}{rgb}{0.000000,0.000000,0.000000}%
\pgfsetstrokecolor{currentstroke}%
\pgfsetstrokeopacity{0.000000}%
\pgfsetdash{}{0pt}%
\pgfpathmoveto{\pgfqpoint{4.034661in}{0.613486in}}%
\pgfpathlineto{\pgfqpoint{4.044666in}{0.613486in}}%
\pgfpathlineto{\pgfqpoint{4.044666in}{1.604355in}}%
\pgfpathlineto{\pgfqpoint{4.034661in}{1.604355in}}%
\pgfpathlineto{\pgfqpoint{4.034661in}{0.613486in}}%
\pgfpathclose%
\pgfusepath{fill}%
\end{pgfscope}%
\begin{pgfscope}%
\pgfpathrectangle{\pgfqpoint{0.693757in}{0.613486in}}{\pgfqpoint{5.541243in}{3.963477in}}%
\pgfusepath{clip}%
\pgfsetbuttcap%
\pgfsetmiterjoin%
\definecolor{currentfill}{rgb}{0.000000,0.000000,1.000000}%
\pgfsetfillcolor{currentfill}%
\pgfsetlinewidth{0.000000pt}%
\definecolor{currentstroke}{rgb}{0.000000,0.000000,0.000000}%
\pgfsetstrokecolor{currentstroke}%
\pgfsetstrokeopacity{0.000000}%
\pgfsetdash{}{0pt}%
\pgfpathmoveto{\pgfqpoint{4.047167in}{0.613486in}}%
\pgfpathlineto{\pgfqpoint{4.057172in}{0.613486in}}%
\pgfpathlineto{\pgfqpoint{4.057172in}{1.612282in}}%
\pgfpathlineto{\pgfqpoint{4.047167in}{1.612282in}}%
\pgfpathlineto{\pgfqpoint{4.047167in}{0.613486in}}%
\pgfpathclose%
\pgfusepath{fill}%
\end{pgfscope}%
\begin{pgfscope}%
\pgfpathrectangle{\pgfqpoint{0.693757in}{0.613486in}}{\pgfqpoint{5.541243in}{3.963477in}}%
\pgfusepath{clip}%
\pgfsetbuttcap%
\pgfsetmiterjoin%
\definecolor{currentfill}{rgb}{0.000000,0.000000,1.000000}%
\pgfsetfillcolor{currentfill}%
\pgfsetlinewidth{0.000000pt}%
\definecolor{currentstroke}{rgb}{0.000000,0.000000,0.000000}%
\pgfsetstrokecolor{currentstroke}%
\pgfsetstrokeopacity{0.000000}%
\pgfsetdash{}{0pt}%
\pgfpathmoveto{\pgfqpoint{4.059673in}{0.613486in}}%
\pgfpathlineto{\pgfqpoint{4.069678in}{0.613486in}}%
\pgfpathlineto{\pgfqpoint{4.069678in}{2.595224in}}%
\pgfpathlineto{\pgfqpoint{4.059673in}{2.595224in}}%
\pgfpathlineto{\pgfqpoint{4.059673in}{0.613486in}}%
\pgfpathclose%
\pgfusepath{fill}%
\end{pgfscope}%
\begin{pgfscope}%
\pgfpathrectangle{\pgfqpoint{0.693757in}{0.613486in}}{\pgfqpoint{5.541243in}{3.963477in}}%
\pgfusepath{clip}%
\pgfsetbuttcap%
\pgfsetmiterjoin%
\definecolor{currentfill}{rgb}{0.000000,0.000000,1.000000}%
\pgfsetfillcolor{currentfill}%
\pgfsetlinewidth{0.000000pt}%
\definecolor{currentstroke}{rgb}{0.000000,0.000000,0.000000}%
\pgfsetstrokecolor{currentstroke}%
\pgfsetstrokeopacity{0.000000}%
\pgfsetdash{}{0pt}%
\pgfpathmoveto{\pgfqpoint{4.072179in}{0.613486in}}%
\pgfpathlineto{\pgfqpoint{4.082184in}{0.613486in}}%
\pgfpathlineto{\pgfqpoint{4.082184in}{2.611074in}}%
\pgfpathlineto{\pgfqpoint{4.072179in}{2.611074in}}%
\pgfpathlineto{\pgfqpoint{4.072179in}{0.613486in}}%
\pgfpathclose%
\pgfusepath{fill}%
\end{pgfscope}%
\begin{pgfscope}%
\pgfpathrectangle{\pgfqpoint{0.693757in}{0.613486in}}{\pgfqpoint{5.541243in}{3.963477in}}%
\pgfusepath{clip}%
\pgfsetbuttcap%
\pgfsetmiterjoin%
\definecolor{currentfill}{rgb}{0.000000,0.000000,1.000000}%
\pgfsetfillcolor{currentfill}%
\pgfsetlinewidth{0.000000pt}%
\definecolor{currentstroke}{rgb}{0.000000,0.000000,0.000000}%
\pgfsetstrokecolor{currentstroke}%
\pgfsetstrokeopacity{0.000000}%
\pgfsetdash{}{0pt}%
\pgfpathmoveto{\pgfqpoint{4.084686in}{0.613486in}}%
\pgfpathlineto{\pgfqpoint{4.094691in}{0.613486in}}%
\pgfpathlineto{\pgfqpoint{4.094691in}{1.604355in}}%
\pgfpathlineto{\pgfqpoint{4.084686in}{1.604355in}}%
\pgfpathlineto{\pgfqpoint{4.084686in}{0.613486in}}%
\pgfpathclose%
\pgfusepath{fill}%
\end{pgfscope}%
\begin{pgfscope}%
\pgfpathrectangle{\pgfqpoint{0.693757in}{0.613486in}}{\pgfqpoint{5.541243in}{3.963477in}}%
\pgfusepath{clip}%
\pgfsetbuttcap%
\pgfsetmiterjoin%
\definecolor{currentfill}{rgb}{0.000000,0.000000,1.000000}%
\pgfsetfillcolor{currentfill}%
\pgfsetlinewidth{0.000000pt}%
\definecolor{currentstroke}{rgb}{0.000000,0.000000,0.000000}%
\pgfsetstrokecolor{currentstroke}%
\pgfsetstrokeopacity{0.000000}%
\pgfsetdash{}{0pt}%
\pgfpathmoveto{\pgfqpoint{4.097192in}{0.613486in}}%
\pgfpathlineto{\pgfqpoint{4.107197in}{0.613486in}}%
\pgfpathlineto{\pgfqpoint{4.107197in}{1.612282in}}%
\pgfpathlineto{\pgfqpoint{4.097192in}{1.612282in}}%
\pgfpathlineto{\pgfqpoint{4.097192in}{0.613486in}}%
\pgfpathclose%
\pgfusepath{fill}%
\end{pgfscope}%
\begin{pgfscope}%
\pgfpathrectangle{\pgfqpoint{0.693757in}{0.613486in}}{\pgfqpoint{5.541243in}{3.963477in}}%
\pgfusepath{clip}%
\pgfsetbuttcap%
\pgfsetmiterjoin%
\definecolor{currentfill}{rgb}{0.000000,0.000000,1.000000}%
\pgfsetfillcolor{currentfill}%
\pgfsetlinewidth{0.000000pt}%
\definecolor{currentstroke}{rgb}{0.000000,0.000000,0.000000}%
\pgfsetstrokecolor{currentstroke}%
\pgfsetstrokeopacity{0.000000}%
\pgfsetdash{}{0pt}%
\pgfpathmoveto{\pgfqpoint{4.109698in}{0.613486in}}%
\pgfpathlineto{\pgfqpoint{4.119703in}{0.613486in}}%
\pgfpathlineto{\pgfqpoint{4.119703in}{2.595224in}}%
\pgfpathlineto{\pgfqpoint{4.109698in}{2.595224in}}%
\pgfpathlineto{\pgfqpoint{4.109698in}{0.613486in}}%
\pgfpathclose%
\pgfusepath{fill}%
\end{pgfscope}%
\begin{pgfscope}%
\pgfpathrectangle{\pgfqpoint{0.693757in}{0.613486in}}{\pgfqpoint{5.541243in}{3.963477in}}%
\pgfusepath{clip}%
\pgfsetbuttcap%
\pgfsetmiterjoin%
\definecolor{currentfill}{rgb}{0.000000,0.000000,1.000000}%
\pgfsetfillcolor{currentfill}%
\pgfsetlinewidth{0.000000pt}%
\definecolor{currentstroke}{rgb}{0.000000,0.000000,0.000000}%
\pgfsetstrokecolor{currentstroke}%
\pgfsetstrokeopacity{0.000000}%
\pgfsetdash{}{0pt}%
\pgfpathmoveto{\pgfqpoint{4.122204in}{0.613486in}}%
\pgfpathlineto{\pgfqpoint{4.132209in}{0.613486in}}%
\pgfpathlineto{\pgfqpoint{4.132209in}{2.611074in}}%
\pgfpathlineto{\pgfqpoint{4.122204in}{2.611074in}}%
\pgfpathlineto{\pgfqpoint{4.122204in}{0.613486in}}%
\pgfpathclose%
\pgfusepath{fill}%
\end{pgfscope}%
\begin{pgfscope}%
\pgfpathrectangle{\pgfqpoint{0.693757in}{0.613486in}}{\pgfqpoint{5.541243in}{3.963477in}}%
\pgfusepath{clip}%
\pgfsetbuttcap%
\pgfsetmiterjoin%
\definecolor{currentfill}{rgb}{0.000000,0.000000,1.000000}%
\pgfsetfillcolor{currentfill}%
\pgfsetlinewidth{0.000000pt}%
\definecolor{currentstroke}{rgb}{0.000000,0.000000,0.000000}%
\pgfsetstrokecolor{currentstroke}%
\pgfsetstrokeopacity{0.000000}%
\pgfsetdash{}{0pt}%
\pgfpathmoveto{\pgfqpoint{4.134710in}{0.613486in}}%
\pgfpathlineto{\pgfqpoint{4.144715in}{0.613486in}}%
\pgfpathlineto{\pgfqpoint{4.144715in}{1.604355in}}%
\pgfpathlineto{\pgfqpoint{4.134710in}{1.604355in}}%
\pgfpathlineto{\pgfqpoint{4.134710in}{0.613486in}}%
\pgfpathclose%
\pgfusepath{fill}%
\end{pgfscope}%
\begin{pgfscope}%
\pgfpathrectangle{\pgfqpoint{0.693757in}{0.613486in}}{\pgfqpoint{5.541243in}{3.963477in}}%
\pgfusepath{clip}%
\pgfsetbuttcap%
\pgfsetmiterjoin%
\definecolor{currentfill}{rgb}{0.000000,0.000000,1.000000}%
\pgfsetfillcolor{currentfill}%
\pgfsetlinewidth{0.000000pt}%
\definecolor{currentstroke}{rgb}{0.000000,0.000000,0.000000}%
\pgfsetstrokecolor{currentstroke}%
\pgfsetstrokeopacity{0.000000}%
\pgfsetdash{}{0pt}%
\pgfpathmoveto{\pgfqpoint{4.147217in}{0.613486in}}%
\pgfpathlineto{\pgfqpoint{4.157221in}{0.613486in}}%
\pgfpathlineto{\pgfqpoint{4.157221in}{1.612282in}}%
\pgfpathlineto{\pgfqpoint{4.147217in}{1.612282in}}%
\pgfpathlineto{\pgfqpoint{4.147217in}{0.613486in}}%
\pgfpathclose%
\pgfusepath{fill}%
\end{pgfscope}%
\begin{pgfscope}%
\pgfpathrectangle{\pgfqpoint{0.693757in}{0.613486in}}{\pgfqpoint{5.541243in}{3.963477in}}%
\pgfusepath{clip}%
\pgfsetbuttcap%
\pgfsetmiterjoin%
\definecolor{currentfill}{rgb}{0.000000,0.000000,1.000000}%
\pgfsetfillcolor{currentfill}%
\pgfsetlinewidth{0.000000pt}%
\definecolor{currentstroke}{rgb}{0.000000,0.000000,0.000000}%
\pgfsetstrokecolor{currentstroke}%
\pgfsetstrokeopacity{0.000000}%
\pgfsetdash{}{0pt}%
\pgfpathmoveto{\pgfqpoint{4.159723in}{0.613486in}}%
\pgfpathlineto{\pgfqpoint{4.169728in}{0.613486in}}%
\pgfpathlineto{\pgfqpoint{4.169728in}{2.595224in}}%
\pgfpathlineto{\pgfqpoint{4.159723in}{2.595224in}}%
\pgfpathlineto{\pgfqpoint{4.159723in}{0.613486in}}%
\pgfpathclose%
\pgfusepath{fill}%
\end{pgfscope}%
\begin{pgfscope}%
\pgfpathrectangle{\pgfqpoint{0.693757in}{0.613486in}}{\pgfqpoint{5.541243in}{3.963477in}}%
\pgfusepath{clip}%
\pgfsetbuttcap%
\pgfsetmiterjoin%
\definecolor{currentfill}{rgb}{0.000000,0.000000,1.000000}%
\pgfsetfillcolor{currentfill}%
\pgfsetlinewidth{0.000000pt}%
\definecolor{currentstroke}{rgb}{0.000000,0.000000,0.000000}%
\pgfsetstrokecolor{currentstroke}%
\pgfsetstrokeopacity{0.000000}%
\pgfsetdash{}{0pt}%
\pgfpathmoveto{\pgfqpoint{4.172229in}{0.613486in}}%
\pgfpathlineto{\pgfqpoint{4.182234in}{0.613486in}}%
\pgfpathlineto{\pgfqpoint{4.182234in}{2.611074in}}%
\pgfpathlineto{\pgfqpoint{4.172229in}{2.611074in}}%
\pgfpathlineto{\pgfqpoint{4.172229in}{0.613486in}}%
\pgfpathclose%
\pgfusepath{fill}%
\end{pgfscope}%
\begin{pgfscope}%
\pgfpathrectangle{\pgfqpoint{0.693757in}{0.613486in}}{\pgfqpoint{5.541243in}{3.963477in}}%
\pgfusepath{clip}%
\pgfsetbuttcap%
\pgfsetmiterjoin%
\definecolor{currentfill}{rgb}{0.000000,0.000000,1.000000}%
\pgfsetfillcolor{currentfill}%
\pgfsetlinewidth{0.000000pt}%
\definecolor{currentstroke}{rgb}{0.000000,0.000000,0.000000}%
\pgfsetstrokecolor{currentstroke}%
\pgfsetstrokeopacity{0.000000}%
\pgfsetdash{}{0pt}%
\pgfpathmoveto{\pgfqpoint{4.184735in}{0.613486in}}%
\pgfpathlineto{\pgfqpoint{4.194740in}{0.613486in}}%
\pgfpathlineto{\pgfqpoint{4.194740in}{1.604355in}}%
\pgfpathlineto{\pgfqpoint{4.184735in}{1.604355in}}%
\pgfpathlineto{\pgfqpoint{4.184735in}{0.613486in}}%
\pgfpathclose%
\pgfusepath{fill}%
\end{pgfscope}%
\begin{pgfscope}%
\pgfpathrectangle{\pgfqpoint{0.693757in}{0.613486in}}{\pgfqpoint{5.541243in}{3.963477in}}%
\pgfusepath{clip}%
\pgfsetbuttcap%
\pgfsetmiterjoin%
\definecolor{currentfill}{rgb}{0.000000,0.000000,1.000000}%
\pgfsetfillcolor{currentfill}%
\pgfsetlinewidth{0.000000pt}%
\definecolor{currentstroke}{rgb}{0.000000,0.000000,0.000000}%
\pgfsetstrokecolor{currentstroke}%
\pgfsetstrokeopacity{0.000000}%
\pgfsetdash{}{0pt}%
\pgfpathmoveto{\pgfqpoint{4.197241in}{0.613486in}}%
\pgfpathlineto{\pgfqpoint{4.207246in}{0.613486in}}%
\pgfpathlineto{\pgfqpoint{4.207246in}{1.612282in}}%
\pgfpathlineto{\pgfqpoint{4.197241in}{1.612282in}}%
\pgfpathlineto{\pgfqpoint{4.197241in}{0.613486in}}%
\pgfpathclose%
\pgfusepath{fill}%
\end{pgfscope}%
\begin{pgfscope}%
\pgfpathrectangle{\pgfqpoint{0.693757in}{0.613486in}}{\pgfqpoint{5.541243in}{3.963477in}}%
\pgfusepath{clip}%
\pgfsetbuttcap%
\pgfsetmiterjoin%
\definecolor{currentfill}{rgb}{0.000000,0.000000,1.000000}%
\pgfsetfillcolor{currentfill}%
\pgfsetlinewidth{0.000000pt}%
\definecolor{currentstroke}{rgb}{0.000000,0.000000,0.000000}%
\pgfsetstrokecolor{currentstroke}%
\pgfsetstrokeopacity{0.000000}%
\pgfsetdash{}{0pt}%
\pgfpathmoveto{\pgfqpoint{4.209747in}{0.613486in}}%
\pgfpathlineto{\pgfqpoint{4.219752in}{0.613486in}}%
\pgfpathlineto{\pgfqpoint{4.219752in}{2.595224in}}%
\pgfpathlineto{\pgfqpoint{4.209747in}{2.595224in}}%
\pgfpathlineto{\pgfqpoint{4.209747in}{0.613486in}}%
\pgfpathclose%
\pgfusepath{fill}%
\end{pgfscope}%
\begin{pgfscope}%
\pgfpathrectangle{\pgfqpoint{0.693757in}{0.613486in}}{\pgfqpoint{5.541243in}{3.963477in}}%
\pgfusepath{clip}%
\pgfsetbuttcap%
\pgfsetmiterjoin%
\definecolor{currentfill}{rgb}{0.000000,0.000000,1.000000}%
\pgfsetfillcolor{currentfill}%
\pgfsetlinewidth{0.000000pt}%
\definecolor{currentstroke}{rgb}{0.000000,0.000000,0.000000}%
\pgfsetstrokecolor{currentstroke}%
\pgfsetstrokeopacity{0.000000}%
\pgfsetdash{}{0pt}%
\pgfpathmoveto{\pgfqpoint{4.222254in}{0.613486in}}%
\pgfpathlineto{\pgfqpoint{4.232259in}{0.613486in}}%
\pgfpathlineto{\pgfqpoint{4.232259in}{2.611074in}}%
\pgfpathlineto{\pgfqpoint{4.222254in}{2.611074in}}%
\pgfpathlineto{\pgfqpoint{4.222254in}{0.613486in}}%
\pgfpathclose%
\pgfusepath{fill}%
\end{pgfscope}%
\begin{pgfscope}%
\pgfpathrectangle{\pgfqpoint{0.693757in}{0.613486in}}{\pgfqpoint{5.541243in}{3.963477in}}%
\pgfusepath{clip}%
\pgfsetbuttcap%
\pgfsetmiterjoin%
\definecolor{currentfill}{rgb}{0.000000,0.000000,1.000000}%
\pgfsetfillcolor{currentfill}%
\pgfsetlinewidth{0.000000pt}%
\definecolor{currentstroke}{rgb}{0.000000,0.000000,0.000000}%
\pgfsetstrokecolor{currentstroke}%
\pgfsetstrokeopacity{0.000000}%
\pgfsetdash{}{0pt}%
\pgfpathmoveto{\pgfqpoint{4.234760in}{0.613486in}}%
\pgfpathlineto{\pgfqpoint{4.244765in}{0.613486in}}%
\pgfpathlineto{\pgfqpoint{4.244765in}{1.604355in}}%
\pgfpathlineto{\pgfqpoint{4.234760in}{1.604355in}}%
\pgfpathlineto{\pgfqpoint{4.234760in}{0.613486in}}%
\pgfpathclose%
\pgfusepath{fill}%
\end{pgfscope}%
\begin{pgfscope}%
\pgfpathrectangle{\pgfqpoint{0.693757in}{0.613486in}}{\pgfqpoint{5.541243in}{3.963477in}}%
\pgfusepath{clip}%
\pgfsetbuttcap%
\pgfsetmiterjoin%
\definecolor{currentfill}{rgb}{0.000000,0.000000,1.000000}%
\pgfsetfillcolor{currentfill}%
\pgfsetlinewidth{0.000000pt}%
\definecolor{currentstroke}{rgb}{0.000000,0.000000,0.000000}%
\pgfsetstrokecolor{currentstroke}%
\pgfsetstrokeopacity{0.000000}%
\pgfsetdash{}{0pt}%
\pgfpathmoveto{\pgfqpoint{4.247266in}{0.613486in}}%
\pgfpathlineto{\pgfqpoint{4.257271in}{0.613486in}}%
\pgfpathlineto{\pgfqpoint{4.257271in}{1.612282in}}%
\pgfpathlineto{\pgfqpoint{4.247266in}{1.612282in}}%
\pgfpathlineto{\pgfqpoint{4.247266in}{0.613486in}}%
\pgfpathclose%
\pgfusepath{fill}%
\end{pgfscope}%
\begin{pgfscope}%
\pgfpathrectangle{\pgfqpoint{0.693757in}{0.613486in}}{\pgfqpoint{5.541243in}{3.963477in}}%
\pgfusepath{clip}%
\pgfsetbuttcap%
\pgfsetmiterjoin%
\definecolor{currentfill}{rgb}{0.000000,0.000000,1.000000}%
\pgfsetfillcolor{currentfill}%
\pgfsetlinewidth{0.000000pt}%
\definecolor{currentstroke}{rgb}{0.000000,0.000000,0.000000}%
\pgfsetstrokecolor{currentstroke}%
\pgfsetstrokeopacity{0.000000}%
\pgfsetdash{}{0pt}%
\pgfpathmoveto{\pgfqpoint{4.259772in}{0.613486in}}%
\pgfpathlineto{\pgfqpoint{4.269777in}{0.613486in}}%
\pgfpathlineto{\pgfqpoint{4.269777in}{2.595224in}}%
\pgfpathlineto{\pgfqpoint{4.259772in}{2.595224in}}%
\pgfpathlineto{\pgfqpoint{4.259772in}{0.613486in}}%
\pgfpathclose%
\pgfusepath{fill}%
\end{pgfscope}%
\begin{pgfscope}%
\pgfpathrectangle{\pgfqpoint{0.693757in}{0.613486in}}{\pgfqpoint{5.541243in}{3.963477in}}%
\pgfusepath{clip}%
\pgfsetbuttcap%
\pgfsetmiterjoin%
\definecolor{currentfill}{rgb}{0.000000,0.000000,1.000000}%
\pgfsetfillcolor{currentfill}%
\pgfsetlinewidth{0.000000pt}%
\definecolor{currentstroke}{rgb}{0.000000,0.000000,0.000000}%
\pgfsetstrokecolor{currentstroke}%
\pgfsetstrokeopacity{0.000000}%
\pgfsetdash{}{0pt}%
\pgfpathmoveto{\pgfqpoint{4.272278in}{0.613486in}}%
\pgfpathlineto{\pgfqpoint{4.282283in}{0.613486in}}%
\pgfpathlineto{\pgfqpoint{4.282283in}{2.611074in}}%
\pgfpathlineto{\pgfqpoint{4.272278in}{2.611074in}}%
\pgfpathlineto{\pgfqpoint{4.272278in}{0.613486in}}%
\pgfpathclose%
\pgfusepath{fill}%
\end{pgfscope}%
\begin{pgfscope}%
\pgfpathrectangle{\pgfqpoint{0.693757in}{0.613486in}}{\pgfqpoint{5.541243in}{3.963477in}}%
\pgfusepath{clip}%
\pgfsetbuttcap%
\pgfsetmiterjoin%
\definecolor{currentfill}{rgb}{0.000000,0.000000,1.000000}%
\pgfsetfillcolor{currentfill}%
\pgfsetlinewidth{0.000000pt}%
\definecolor{currentstroke}{rgb}{0.000000,0.000000,0.000000}%
\pgfsetstrokecolor{currentstroke}%
\pgfsetstrokeopacity{0.000000}%
\pgfsetdash{}{0pt}%
\pgfpathmoveto{\pgfqpoint{4.284785in}{0.613486in}}%
\pgfpathlineto{\pgfqpoint{4.294790in}{0.613486in}}%
\pgfpathlineto{\pgfqpoint{4.294790in}{1.604355in}}%
\pgfpathlineto{\pgfqpoint{4.284785in}{1.604355in}}%
\pgfpathlineto{\pgfqpoint{4.284785in}{0.613486in}}%
\pgfpathclose%
\pgfusepath{fill}%
\end{pgfscope}%
\begin{pgfscope}%
\pgfpathrectangle{\pgfqpoint{0.693757in}{0.613486in}}{\pgfqpoint{5.541243in}{3.963477in}}%
\pgfusepath{clip}%
\pgfsetbuttcap%
\pgfsetmiterjoin%
\definecolor{currentfill}{rgb}{0.000000,0.000000,1.000000}%
\pgfsetfillcolor{currentfill}%
\pgfsetlinewidth{0.000000pt}%
\definecolor{currentstroke}{rgb}{0.000000,0.000000,0.000000}%
\pgfsetstrokecolor{currentstroke}%
\pgfsetstrokeopacity{0.000000}%
\pgfsetdash{}{0pt}%
\pgfpathmoveto{\pgfqpoint{4.297291in}{0.613486in}}%
\pgfpathlineto{\pgfqpoint{4.307296in}{0.613486in}}%
\pgfpathlineto{\pgfqpoint{4.307296in}{1.612282in}}%
\pgfpathlineto{\pgfqpoint{4.297291in}{1.612282in}}%
\pgfpathlineto{\pgfqpoint{4.297291in}{0.613486in}}%
\pgfpathclose%
\pgfusepath{fill}%
\end{pgfscope}%
\begin{pgfscope}%
\pgfpathrectangle{\pgfqpoint{0.693757in}{0.613486in}}{\pgfqpoint{5.541243in}{3.963477in}}%
\pgfusepath{clip}%
\pgfsetbuttcap%
\pgfsetmiterjoin%
\definecolor{currentfill}{rgb}{0.000000,0.000000,1.000000}%
\pgfsetfillcolor{currentfill}%
\pgfsetlinewidth{0.000000pt}%
\definecolor{currentstroke}{rgb}{0.000000,0.000000,0.000000}%
\pgfsetstrokecolor{currentstroke}%
\pgfsetstrokeopacity{0.000000}%
\pgfsetdash{}{0pt}%
\pgfpathmoveto{\pgfqpoint{4.309797in}{0.613486in}}%
\pgfpathlineto{\pgfqpoint{4.319802in}{0.613486in}}%
\pgfpathlineto{\pgfqpoint{4.319802in}{2.595224in}}%
\pgfpathlineto{\pgfqpoint{4.309797in}{2.595224in}}%
\pgfpathlineto{\pgfqpoint{4.309797in}{0.613486in}}%
\pgfpathclose%
\pgfusepath{fill}%
\end{pgfscope}%
\begin{pgfscope}%
\pgfpathrectangle{\pgfqpoint{0.693757in}{0.613486in}}{\pgfqpoint{5.541243in}{3.963477in}}%
\pgfusepath{clip}%
\pgfsetbuttcap%
\pgfsetmiterjoin%
\definecolor{currentfill}{rgb}{0.000000,0.000000,1.000000}%
\pgfsetfillcolor{currentfill}%
\pgfsetlinewidth{0.000000pt}%
\definecolor{currentstroke}{rgb}{0.000000,0.000000,0.000000}%
\pgfsetstrokecolor{currentstroke}%
\pgfsetstrokeopacity{0.000000}%
\pgfsetdash{}{0pt}%
\pgfpathmoveto{\pgfqpoint{4.322303in}{0.613486in}}%
\pgfpathlineto{\pgfqpoint{4.332308in}{0.613486in}}%
\pgfpathlineto{\pgfqpoint{4.332308in}{2.611074in}}%
\pgfpathlineto{\pgfqpoint{4.322303in}{2.611074in}}%
\pgfpathlineto{\pgfqpoint{4.322303in}{0.613486in}}%
\pgfpathclose%
\pgfusepath{fill}%
\end{pgfscope}%
\begin{pgfscope}%
\pgfpathrectangle{\pgfqpoint{0.693757in}{0.613486in}}{\pgfqpoint{5.541243in}{3.963477in}}%
\pgfusepath{clip}%
\pgfsetbuttcap%
\pgfsetmiterjoin%
\definecolor{currentfill}{rgb}{0.000000,0.000000,1.000000}%
\pgfsetfillcolor{currentfill}%
\pgfsetlinewidth{0.000000pt}%
\definecolor{currentstroke}{rgb}{0.000000,0.000000,0.000000}%
\pgfsetstrokecolor{currentstroke}%
\pgfsetstrokeopacity{0.000000}%
\pgfsetdash{}{0pt}%
\pgfpathmoveto{\pgfqpoint{4.334809in}{0.613486in}}%
\pgfpathlineto{\pgfqpoint{4.344814in}{0.613486in}}%
\pgfpathlineto{\pgfqpoint{4.344814in}{1.604355in}}%
\pgfpathlineto{\pgfqpoint{4.334809in}{1.604355in}}%
\pgfpathlineto{\pgfqpoint{4.334809in}{0.613486in}}%
\pgfpathclose%
\pgfusepath{fill}%
\end{pgfscope}%
\begin{pgfscope}%
\pgfpathrectangle{\pgfqpoint{0.693757in}{0.613486in}}{\pgfqpoint{5.541243in}{3.963477in}}%
\pgfusepath{clip}%
\pgfsetbuttcap%
\pgfsetmiterjoin%
\definecolor{currentfill}{rgb}{0.000000,0.000000,1.000000}%
\pgfsetfillcolor{currentfill}%
\pgfsetlinewidth{0.000000pt}%
\definecolor{currentstroke}{rgb}{0.000000,0.000000,0.000000}%
\pgfsetstrokecolor{currentstroke}%
\pgfsetstrokeopacity{0.000000}%
\pgfsetdash{}{0pt}%
\pgfpathmoveto{\pgfqpoint{4.347316in}{0.613486in}}%
\pgfpathlineto{\pgfqpoint{4.357321in}{0.613486in}}%
\pgfpathlineto{\pgfqpoint{4.357321in}{1.612282in}}%
\pgfpathlineto{\pgfqpoint{4.347316in}{1.612282in}}%
\pgfpathlineto{\pgfqpoint{4.347316in}{0.613486in}}%
\pgfpathclose%
\pgfusepath{fill}%
\end{pgfscope}%
\begin{pgfscope}%
\pgfpathrectangle{\pgfqpoint{0.693757in}{0.613486in}}{\pgfqpoint{5.541243in}{3.963477in}}%
\pgfusepath{clip}%
\pgfsetbuttcap%
\pgfsetmiterjoin%
\definecolor{currentfill}{rgb}{0.000000,0.000000,1.000000}%
\pgfsetfillcolor{currentfill}%
\pgfsetlinewidth{0.000000pt}%
\definecolor{currentstroke}{rgb}{0.000000,0.000000,0.000000}%
\pgfsetstrokecolor{currentstroke}%
\pgfsetstrokeopacity{0.000000}%
\pgfsetdash{}{0pt}%
\pgfpathmoveto{\pgfqpoint{4.359822in}{0.613486in}}%
\pgfpathlineto{\pgfqpoint{4.369827in}{0.613486in}}%
\pgfpathlineto{\pgfqpoint{4.369827in}{2.595224in}}%
\pgfpathlineto{\pgfqpoint{4.359822in}{2.595224in}}%
\pgfpathlineto{\pgfqpoint{4.359822in}{0.613486in}}%
\pgfpathclose%
\pgfusepath{fill}%
\end{pgfscope}%
\begin{pgfscope}%
\pgfpathrectangle{\pgfqpoint{0.693757in}{0.613486in}}{\pgfqpoint{5.541243in}{3.963477in}}%
\pgfusepath{clip}%
\pgfsetbuttcap%
\pgfsetmiterjoin%
\definecolor{currentfill}{rgb}{0.000000,0.000000,1.000000}%
\pgfsetfillcolor{currentfill}%
\pgfsetlinewidth{0.000000pt}%
\definecolor{currentstroke}{rgb}{0.000000,0.000000,0.000000}%
\pgfsetstrokecolor{currentstroke}%
\pgfsetstrokeopacity{0.000000}%
\pgfsetdash{}{0pt}%
\pgfpathmoveto{\pgfqpoint{4.372328in}{0.613486in}}%
\pgfpathlineto{\pgfqpoint{4.382333in}{0.613486in}}%
\pgfpathlineto{\pgfqpoint{4.382333in}{2.611074in}}%
\pgfpathlineto{\pgfqpoint{4.372328in}{2.611074in}}%
\pgfpathlineto{\pgfqpoint{4.372328in}{0.613486in}}%
\pgfpathclose%
\pgfusepath{fill}%
\end{pgfscope}%
\begin{pgfscope}%
\pgfpathrectangle{\pgfqpoint{0.693757in}{0.613486in}}{\pgfqpoint{5.541243in}{3.963477in}}%
\pgfusepath{clip}%
\pgfsetbuttcap%
\pgfsetmiterjoin%
\definecolor{currentfill}{rgb}{0.000000,0.000000,1.000000}%
\pgfsetfillcolor{currentfill}%
\pgfsetlinewidth{0.000000pt}%
\definecolor{currentstroke}{rgb}{0.000000,0.000000,0.000000}%
\pgfsetstrokecolor{currentstroke}%
\pgfsetstrokeopacity{0.000000}%
\pgfsetdash{}{0pt}%
\pgfpathmoveto{\pgfqpoint{4.384834in}{0.613486in}}%
\pgfpathlineto{\pgfqpoint{4.394839in}{0.613486in}}%
\pgfpathlineto{\pgfqpoint{4.394839in}{1.604355in}}%
\pgfpathlineto{\pgfqpoint{4.384834in}{1.604355in}}%
\pgfpathlineto{\pgfqpoint{4.384834in}{0.613486in}}%
\pgfpathclose%
\pgfusepath{fill}%
\end{pgfscope}%
\begin{pgfscope}%
\pgfpathrectangle{\pgfqpoint{0.693757in}{0.613486in}}{\pgfqpoint{5.541243in}{3.963477in}}%
\pgfusepath{clip}%
\pgfsetbuttcap%
\pgfsetmiterjoin%
\definecolor{currentfill}{rgb}{0.000000,0.000000,1.000000}%
\pgfsetfillcolor{currentfill}%
\pgfsetlinewidth{0.000000pt}%
\definecolor{currentstroke}{rgb}{0.000000,0.000000,0.000000}%
\pgfsetstrokecolor{currentstroke}%
\pgfsetstrokeopacity{0.000000}%
\pgfsetdash{}{0pt}%
\pgfpathmoveto{\pgfqpoint{4.397340in}{0.613486in}}%
\pgfpathlineto{\pgfqpoint{4.407345in}{0.613486in}}%
\pgfpathlineto{\pgfqpoint{4.407345in}{1.612282in}}%
\pgfpathlineto{\pgfqpoint{4.397340in}{1.612282in}}%
\pgfpathlineto{\pgfqpoint{4.397340in}{0.613486in}}%
\pgfpathclose%
\pgfusepath{fill}%
\end{pgfscope}%
\begin{pgfscope}%
\pgfpathrectangle{\pgfqpoint{0.693757in}{0.613486in}}{\pgfqpoint{5.541243in}{3.963477in}}%
\pgfusepath{clip}%
\pgfsetbuttcap%
\pgfsetmiterjoin%
\definecolor{currentfill}{rgb}{0.000000,0.000000,1.000000}%
\pgfsetfillcolor{currentfill}%
\pgfsetlinewidth{0.000000pt}%
\definecolor{currentstroke}{rgb}{0.000000,0.000000,0.000000}%
\pgfsetstrokecolor{currentstroke}%
\pgfsetstrokeopacity{0.000000}%
\pgfsetdash{}{0pt}%
\pgfpathmoveto{\pgfqpoint{4.409847in}{0.613486in}}%
\pgfpathlineto{\pgfqpoint{4.419852in}{0.613486in}}%
\pgfpathlineto{\pgfqpoint{4.419852in}{2.595224in}}%
\pgfpathlineto{\pgfqpoint{4.409847in}{2.595224in}}%
\pgfpathlineto{\pgfqpoint{4.409847in}{0.613486in}}%
\pgfpathclose%
\pgfusepath{fill}%
\end{pgfscope}%
\begin{pgfscope}%
\pgfpathrectangle{\pgfqpoint{0.693757in}{0.613486in}}{\pgfqpoint{5.541243in}{3.963477in}}%
\pgfusepath{clip}%
\pgfsetbuttcap%
\pgfsetmiterjoin%
\definecolor{currentfill}{rgb}{0.000000,0.000000,1.000000}%
\pgfsetfillcolor{currentfill}%
\pgfsetlinewidth{0.000000pt}%
\definecolor{currentstroke}{rgb}{0.000000,0.000000,0.000000}%
\pgfsetstrokecolor{currentstroke}%
\pgfsetstrokeopacity{0.000000}%
\pgfsetdash{}{0pt}%
\pgfpathmoveto{\pgfqpoint{4.422353in}{0.613486in}}%
\pgfpathlineto{\pgfqpoint{4.432358in}{0.613486in}}%
\pgfpathlineto{\pgfqpoint{4.432358in}{2.611074in}}%
\pgfpathlineto{\pgfqpoint{4.422353in}{2.611074in}}%
\pgfpathlineto{\pgfqpoint{4.422353in}{0.613486in}}%
\pgfpathclose%
\pgfusepath{fill}%
\end{pgfscope}%
\begin{pgfscope}%
\pgfpathrectangle{\pgfqpoint{0.693757in}{0.613486in}}{\pgfqpoint{5.541243in}{3.963477in}}%
\pgfusepath{clip}%
\pgfsetbuttcap%
\pgfsetmiterjoin%
\definecolor{currentfill}{rgb}{0.000000,0.000000,1.000000}%
\pgfsetfillcolor{currentfill}%
\pgfsetlinewidth{0.000000pt}%
\definecolor{currentstroke}{rgb}{0.000000,0.000000,0.000000}%
\pgfsetstrokecolor{currentstroke}%
\pgfsetstrokeopacity{0.000000}%
\pgfsetdash{}{0pt}%
\pgfpathmoveto{\pgfqpoint{4.434859in}{0.613486in}}%
\pgfpathlineto{\pgfqpoint{4.444864in}{0.613486in}}%
\pgfpathlineto{\pgfqpoint{4.444864in}{1.604355in}}%
\pgfpathlineto{\pgfqpoint{4.434859in}{1.604355in}}%
\pgfpathlineto{\pgfqpoint{4.434859in}{0.613486in}}%
\pgfpathclose%
\pgfusepath{fill}%
\end{pgfscope}%
\begin{pgfscope}%
\pgfpathrectangle{\pgfqpoint{0.693757in}{0.613486in}}{\pgfqpoint{5.541243in}{3.963477in}}%
\pgfusepath{clip}%
\pgfsetbuttcap%
\pgfsetmiterjoin%
\definecolor{currentfill}{rgb}{0.000000,0.000000,1.000000}%
\pgfsetfillcolor{currentfill}%
\pgfsetlinewidth{0.000000pt}%
\definecolor{currentstroke}{rgb}{0.000000,0.000000,0.000000}%
\pgfsetstrokecolor{currentstroke}%
\pgfsetstrokeopacity{0.000000}%
\pgfsetdash{}{0pt}%
\pgfpathmoveto{\pgfqpoint{4.447365in}{0.613486in}}%
\pgfpathlineto{\pgfqpoint{4.457370in}{0.613486in}}%
\pgfpathlineto{\pgfqpoint{4.457370in}{1.612282in}}%
\pgfpathlineto{\pgfqpoint{4.447365in}{1.612282in}}%
\pgfpathlineto{\pgfqpoint{4.447365in}{0.613486in}}%
\pgfpathclose%
\pgfusepath{fill}%
\end{pgfscope}%
\begin{pgfscope}%
\pgfpathrectangle{\pgfqpoint{0.693757in}{0.613486in}}{\pgfqpoint{5.541243in}{3.963477in}}%
\pgfusepath{clip}%
\pgfsetbuttcap%
\pgfsetmiterjoin%
\definecolor{currentfill}{rgb}{0.000000,0.000000,1.000000}%
\pgfsetfillcolor{currentfill}%
\pgfsetlinewidth{0.000000pt}%
\definecolor{currentstroke}{rgb}{0.000000,0.000000,0.000000}%
\pgfsetstrokecolor{currentstroke}%
\pgfsetstrokeopacity{0.000000}%
\pgfsetdash{}{0pt}%
\pgfpathmoveto{\pgfqpoint{4.459871in}{0.613486in}}%
\pgfpathlineto{\pgfqpoint{4.469876in}{0.613486in}}%
\pgfpathlineto{\pgfqpoint{4.469876in}{2.595224in}}%
\pgfpathlineto{\pgfqpoint{4.459871in}{2.595224in}}%
\pgfpathlineto{\pgfqpoint{4.459871in}{0.613486in}}%
\pgfpathclose%
\pgfusepath{fill}%
\end{pgfscope}%
\begin{pgfscope}%
\pgfpathrectangle{\pgfqpoint{0.693757in}{0.613486in}}{\pgfqpoint{5.541243in}{3.963477in}}%
\pgfusepath{clip}%
\pgfsetbuttcap%
\pgfsetmiterjoin%
\definecolor{currentfill}{rgb}{0.000000,0.000000,1.000000}%
\pgfsetfillcolor{currentfill}%
\pgfsetlinewidth{0.000000pt}%
\definecolor{currentstroke}{rgb}{0.000000,0.000000,0.000000}%
\pgfsetstrokecolor{currentstroke}%
\pgfsetstrokeopacity{0.000000}%
\pgfsetdash{}{0pt}%
\pgfpathmoveto{\pgfqpoint{4.472378in}{0.613486in}}%
\pgfpathlineto{\pgfqpoint{4.482382in}{0.613486in}}%
\pgfpathlineto{\pgfqpoint{4.482382in}{2.611074in}}%
\pgfpathlineto{\pgfqpoint{4.472378in}{2.611074in}}%
\pgfpathlineto{\pgfqpoint{4.472378in}{0.613486in}}%
\pgfpathclose%
\pgfusepath{fill}%
\end{pgfscope}%
\begin{pgfscope}%
\pgfpathrectangle{\pgfqpoint{0.693757in}{0.613486in}}{\pgfqpoint{5.541243in}{3.963477in}}%
\pgfusepath{clip}%
\pgfsetbuttcap%
\pgfsetmiterjoin%
\definecolor{currentfill}{rgb}{0.000000,0.000000,1.000000}%
\pgfsetfillcolor{currentfill}%
\pgfsetlinewidth{0.000000pt}%
\definecolor{currentstroke}{rgb}{0.000000,0.000000,0.000000}%
\pgfsetstrokecolor{currentstroke}%
\pgfsetstrokeopacity{0.000000}%
\pgfsetdash{}{0pt}%
\pgfpathmoveto{\pgfqpoint{4.484884in}{0.613486in}}%
\pgfpathlineto{\pgfqpoint{4.494889in}{0.613486in}}%
\pgfpathlineto{\pgfqpoint{4.494889in}{1.604355in}}%
\pgfpathlineto{\pgfqpoint{4.484884in}{1.604355in}}%
\pgfpathlineto{\pgfqpoint{4.484884in}{0.613486in}}%
\pgfpathclose%
\pgfusepath{fill}%
\end{pgfscope}%
\begin{pgfscope}%
\pgfpathrectangle{\pgfqpoint{0.693757in}{0.613486in}}{\pgfqpoint{5.541243in}{3.963477in}}%
\pgfusepath{clip}%
\pgfsetbuttcap%
\pgfsetmiterjoin%
\definecolor{currentfill}{rgb}{0.000000,0.000000,1.000000}%
\pgfsetfillcolor{currentfill}%
\pgfsetlinewidth{0.000000pt}%
\definecolor{currentstroke}{rgb}{0.000000,0.000000,0.000000}%
\pgfsetstrokecolor{currentstroke}%
\pgfsetstrokeopacity{0.000000}%
\pgfsetdash{}{0pt}%
\pgfpathmoveto{\pgfqpoint{4.497390in}{0.613486in}}%
\pgfpathlineto{\pgfqpoint{4.507395in}{0.613486in}}%
\pgfpathlineto{\pgfqpoint{4.507395in}{1.612282in}}%
\pgfpathlineto{\pgfqpoint{4.497390in}{1.612282in}}%
\pgfpathlineto{\pgfqpoint{4.497390in}{0.613486in}}%
\pgfpathclose%
\pgfusepath{fill}%
\end{pgfscope}%
\begin{pgfscope}%
\pgfpathrectangle{\pgfqpoint{0.693757in}{0.613486in}}{\pgfqpoint{5.541243in}{3.963477in}}%
\pgfusepath{clip}%
\pgfsetbuttcap%
\pgfsetmiterjoin%
\definecolor{currentfill}{rgb}{0.000000,0.000000,1.000000}%
\pgfsetfillcolor{currentfill}%
\pgfsetlinewidth{0.000000pt}%
\definecolor{currentstroke}{rgb}{0.000000,0.000000,0.000000}%
\pgfsetstrokecolor{currentstroke}%
\pgfsetstrokeopacity{0.000000}%
\pgfsetdash{}{0pt}%
\pgfpathmoveto{\pgfqpoint{4.509896in}{0.613486in}}%
\pgfpathlineto{\pgfqpoint{4.519901in}{0.613486in}}%
\pgfpathlineto{\pgfqpoint{4.519901in}{2.595224in}}%
\pgfpathlineto{\pgfqpoint{4.509896in}{2.595224in}}%
\pgfpathlineto{\pgfqpoint{4.509896in}{0.613486in}}%
\pgfpathclose%
\pgfusepath{fill}%
\end{pgfscope}%
\begin{pgfscope}%
\pgfpathrectangle{\pgfqpoint{0.693757in}{0.613486in}}{\pgfqpoint{5.541243in}{3.963477in}}%
\pgfusepath{clip}%
\pgfsetbuttcap%
\pgfsetmiterjoin%
\definecolor{currentfill}{rgb}{0.000000,0.000000,1.000000}%
\pgfsetfillcolor{currentfill}%
\pgfsetlinewidth{0.000000pt}%
\definecolor{currentstroke}{rgb}{0.000000,0.000000,0.000000}%
\pgfsetstrokecolor{currentstroke}%
\pgfsetstrokeopacity{0.000000}%
\pgfsetdash{}{0pt}%
\pgfpathmoveto{\pgfqpoint{4.522402in}{0.613486in}}%
\pgfpathlineto{\pgfqpoint{4.532407in}{0.613486in}}%
\pgfpathlineto{\pgfqpoint{4.532407in}{2.611074in}}%
\pgfpathlineto{\pgfqpoint{4.522402in}{2.611074in}}%
\pgfpathlineto{\pgfqpoint{4.522402in}{0.613486in}}%
\pgfpathclose%
\pgfusepath{fill}%
\end{pgfscope}%
\begin{pgfscope}%
\pgfpathrectangle{\pgfqpoint{0.693757in}{0.613486in}}{\pgfqpoint{5.541243in}{3.963477in}}%
\pgfusepath{clip}%
\pgfsetbuttcap%
\pgfsetmiterjoin%
\definecolor{currentfill}{rgb}{0.000000,0.000000,1.000000}%
\pgfsetfillcolor{currentfill}%
\pgfsetlinewidth{0.000000pt}%
\definecolor{currentstroke}{rgb}{0.000000,0.000000,0.000000}%
\pgfsetstrokecolor{currentstroke}%
\pgfsetstrokeopacity{0.000000}%
\pgfsetdash{}{0pt}%
\pgfpathmoveto{\pgfqpoint{4.534908in}{0.613486in}}%
\pgfpathlineto{\pgfqpoint{4.544913in}{0.613486in}}%
\pgfpathlineto{\pgfqpoint{4.544913in}{1.604355in}}%
\pgfpathlineto{\pgfqpoint{4.534908in}{1.604355in}}%
\pgfpathlineto{\pgfqpoint{4.534908in}{0.613486in}}%
\pgfpathclose%
\pgfusepath{fill}%
\end{pgfscope}%
\begin{pgfscope}%
\pgfpathrectangle{\pgfqpoint{0.693757in}{0.613486in}}{\pgfqpoint{5.541243in}{3.963477in}}%
\pgfusepath{clip}%
\pgfsetbuttcap%
\pgfsetmiterjoin%
\definecolor{currentfill}{rgb}{0.000000,0.000000,1.000000}%
\pgfsetfillcolor{currentfill}%
\pgfsetlinewidth{0.000000pt}%
\definecolor{currentstroke}{rgb}{0.000000,0.000000,0.000000}%
\pgfsetstrokecolor{currentstroke}%
\pgfsetstrokeopacity{0.000000}%
\pgfsetdash{}{0pt}%
\pgfpathmoveto{\pgfqpoint{4.547415in}{0.613486in}}%
\pgfpathlineto{\pgfqpoint{4.557420in}{0.613486in}}%
\pgfpathlineto{\pgfqpoint{4.557420in}{1.612282in}}%
\pgfpathlineto{\pgfqpoint{4.547415in}{1.612282in}}%
\pgfpathlineto{\pgfqpoint{4.547415in}{0.613486in}}%
\pgfpathclose%
\pgfusepath{fill}%
\end{pgfscope}%
\begin{pgfscope}%
\pgfpathrectangle{\pgfqpoint{0.693757in}{0.613486in}}{\pgfqpoint{5.541243in}{3.963477in}}%
\pgfusepath{clip}%
\pgfsetbuttcap%
\pgfsetmiterjoin%
\definecolor{currentfill}{rgb}{0.000000,0.000000,1.000000}%
\pgfsetfillcolor{currentfill}%
\pgfsetlinewidth{0.000000pt}%
\definecolor{currentstroke}{rgb}{0.000000,0.000000,0.000000}%
\pgfsetstrokecolor{currentstroke}%
\pgfsetstrokeopacity{0.000000}%
\pgfsetdash{}{0pt}%
\pgfpathmoveto{\pgfqpoint{4.559921in}{0.613486in}}%
\pgfpathlineto{\pgfqpoint{4.569926in}{0.613486in}}%
\pgfpathlineto{\pgfqpoint{4.569926in}{2.595224in}}%
\pgfpathlineto{\pgfqpoint{4.559921in}{2.595224in}}%
\pgfpathlineto{\pgfqpoint{4.559921in}{0.613486in}}%
\pgfpathclose%
\pgfusepath{fill}%
\end{pgfscope}%
\begin{pgfscope}%
\pgfpathrectangle{\pgfqpoint{0.693757in}{0.613486in}}{\pgfqpoint{5.541243in}{3.963477in}}%
\pgfusepath{clip}%
\pgfsetbuttcap%
\pgfsetmiterjoin%
\definecolor{currentfill}{rgb}{0.000000,0.000000,1.000000}%
\pgfsetfillcolor{currentfill}%
\pgfsetlinewidth{0.000000pt}%
\definecolor{currentstroke}{rgb}{0.000000,0.000000,0.000000}%
\pgfsetstrokecolor{currentstroke}%
\pgfsetstrokeopacity{0.000000}%
\pgfsetdash{}{0pt}%
\pgfpathmoveto{\pgfqpoint{4.572427in}{0.613486in}}%
\pgfpathlineto{\pgfqpoint{4.582432in}{0.613486in}}%
\pgfpathlineto{\pgfqpoint{4.582432in}{2.611074in}}%
\pgfpathlineto{\pgfqpoint{4.572427in}{2.611074in}}%
\pgfpathlineto{\pgfqpoint{4.572427in}{0.613486in}}%
\pgfpathclose%
\pgfusepath{fill}%
\end{pgfscope}%
\begin{pgfscope}%
\pgfpathrectangle{\pgfqpoint{0.693757in}{0.613486in}}{\pgfqpoint{5.541243in}{3.963477in}}%
\pgfusepath{clip}%
\pgfsetbuttcap%
\pgfsetmiterjoin%
\definecolor{currentfill}{rgb}{0.000000,0.000000,1.000000}%
\pgfsetfillcolor{currentfill}%
\pgfsetlinewidth{0.000000pt}%
\definecolor{currentstroke}{rgb}{0.000000,0.000000,0.000000}%
\pgfsetstrokecolor{currentstroke}%
\pgfsetstrokeopacity{0.000000}%
\pgfsetdash{}{0pt}%
\pgfpathmoveto{\pgfqpoint{4.584933in}{0.613486in}}%
\pgfpathlineto{\pgfqpoint{4.594938in}{0.613486in}}%
\pgfpathlineto{\pgfqpoint{4.594938in}{1.604355in}}%
\pgfpathlineto{\pgfqpoint{4.584933in}{1.604355in}}%
\pgfpathlineto{\pgfqpoint{4.584933in}{0.613486in}}%
\pgfpathclose%
\pgfusepath{fill}%
\end{pgfscope}%
\begin{pgfscope}%
\pgfpathrectangle{\pgfqpoint{0.693757in}{0.613486in}}{\pgfqpoint{5.541243in}{3.963477in}}%
\pgfusepath{clip}%
\pgfsetbuttcap%
\pgfsetmiterjoin%
\definecolor{currentfill}{rgb}{0.000000,0.000000,1.000000}%
\pgfsetfillcolor{currentfill}%
\pgfsetlinewidth{0.000000pt}%
\definecolor{currentstroke}{rgb}{0.000000,0.000000,0.000000}%
\pgfsetstrokecolor{currentstroke}%
\pgfsetstrokeopacity{0.000000}%
\pgfsetdash{}{0pt}%
\pgfpathmoveto{\pgfqpoint{4.597439in}{0.613486in}}%
\pgfpathlineto{\pgfqpoint{4.607444in}{0.613486in}}%
\pgfpathlineto{\pgfqpoint{4.607444in}{1.612282in}}%
\pgfpathlineto{\pgfqpoint{4.597439in}{1.612282in}}%
\pgfpathlineto{\pgfqpoint{4.597439in}{0.613486in}}%
\pgfpathclose%
\pgfusepath{fill}%
\end{pgfscope}%
\begin{pgfscope}%
\pgfpathrectangle{\pgfqpoint{0.693757in}{0.613486in}}{\pgfqpoint{5.541243in}{3.963477in}}%
\pgfusepath{clip}%
\pgfsetbuttcap%
\pgfsetmiterjoin%
\definecolor{currentfill}{rgb}{0.000000,0.000000,1.000000}%
\pgfsetfillcolor{currentfill}%
\pgfsetlinewidth{0.000000pt}%
\definecolor{currentstroke}{rgb}{0.000000,0.000000,0.000000}%
\pgfsetstrokecolor{currentstroke}%
\pgfsetstrokeopacity{0.000000}%
\pgfsetdash{}{0pt}%
\pgfpathmoveto{\pgfqpoint{4.609946in}{0.613486in}}%
\pgfpathlineto{\pgfqpoint{4.619951in}{0.613486in}}%
\pgfpathlineto{\pgfqpoint{4.619951in}{2.595224in}}%
\pgfpathlineto{\pgfqpoint{4.609946in}{2.595224in}}%
\pgfpathlineto{\pgfqpoint{4.609946in}{0.613486in}}%
\pgfpathclose%
\pgfusepath{fill}%
\end{pgfscope}%
\begin{pgfscope}%
\pgfpathrectangle{\pgfqpoint{0.693757in}{0.613486in}}{\pgfqpoint{5.541243in}{3.963477in}}%
\pgfusepath{clip}%
\pgfsetbuttcap%
\pgfsetmiterjoin%
\definecolor{currentfill}{rgb}{0.000000,0.000000,1.000000}%
\pgfsetfillcolor{currentfill}%
\pgfsetlinewidth{0.000000pt}%
\definecolor{currentstroke}{rgb}{0.000000,0.000000,0.000000}%
\pgfsetstrokecolor{currentstroke}%
\pgfsetstrokeopacity{0.000000}%
\pgfsetdash{}{0pt}%
\pgfpathmoveto{\pgfqpoint{4.622452in}{0.613486in}}%
\pgfpathlineto{\pgfqpoint{4.632457in}{0.613486in}}%
\pgfpathlineto{\pgfqpoint{4.632457in}{2.611074in}}%
\pgfpathlineto{\pgfqpoint{4.622452in}{2.611074in}}%
\pgfpathlineto{\pgfqpoint{4.622452in}{0.613486in}}%
\pgfpathclose%
\pgfusepath{fill}%
\end{pgfscope}%
\begin{pgfscope}%
\pgfpathrectangle{\pgfqpoint{0.693757in}{0.613486in}}{\pgfqpoint{5.541243in}{3.963477in}}%
\pgfusepath{clip}%
\pgfsetbuttcap%
\pgfsetmiterjoin%
\definecolor{currentfill}{rgb}{0.000000,0.000000,1.000000}%
\pgfsetfillcolor{currentfill}%
\pgfsetlinewidth{0.000000pt}%
\definecolor{currentstroke}{rgb}{0.000000,0.000000,0.000000}%
\pgfsetstrokecolor{currentstroke}%
\pgfsetstrokeopacity{0.000000}%
\pgfsetdash{}{0pt}%
\pgfpathmoveto{\pgfqpoint{4.634958in}{0.613486in}}%
\pgfpathlineto{\pgfqpoint{4.644963in}{0.613486in}}%
\pgfpathlineto{\pgfqpoint{4.644963in}{1.604355in}}%
\pgfpathlineto{\pgfqpoint{4.634958in}{1.604355in}}%
\pgfpathlineto{\pgfqpoint{4.634958in}{0.613486in}}%
\pgfpathclose%
\pgfusepath{fill}%
\end{pgfscope}%
\begin{pgfscope}%
\pgfpathrectangle{\pgfqpoint{0.693757in}{0.613486in}}{\pgfqpoint{5.541243in}{3.963477in}}%
\pgfusepath{clip}%
\pgfsetbuttcap%
\pgfsetmiterjoin%
\definecolor{currentfill}{rgb}{0.000000,0.000000,1.000000}%
\pgfsetfillcolor{currentfill}%
\pgfsetlinewidth{0.000000pt}%
\definecolor{currentstroke}{rgb}{0.000000,0.000000,0.000000}%
\pgfsetstrokecolor{currentstroke}%
\pgfsetstrokeopacity{0.000000}%
\pgfsetdash{}{0pt}%
\pgfpathmoveto{\pgfqpoint{4.647464in}{0.613486in}}%
\pgfpathlineto{\pgfqpoint{4.657469in}{0.613486in}}%
\pgfpathlineto{\pgfqpoint{4.657469in}{1.612282in}}%
\pgfpathlineto{\pgfqpoint{4.647464in}{1.612282in}}%
\pgfpathlineto{\pgfqpoint{4.647464in}{0.613486in}}%
\pgfpathclose%
\pgfusepath{fill}%
\end{pgfscope}%
\begin{pgfscope}%
\pgfpathrectangle{\pgfqpoint{0.693757in}{0.613486in}}{\pgfqpoint{5.541243in}{3.963477in}}%
\pgfusepath{clip}%
\pgfsetbuttcap%
\pgfsetmiterjoin%
\definecolor{currentfill}{rgb}{0.000000,0.000000,1.000000}%
\pgfsetfillcolor{currentfill}%
\pgfsetlinewidth{0.000000pt}%
\definecolor{currentstroke}{rgb}{0.000000,0.000000,0.000000}%
\pgfsetstrokecolor{currentstroke}%
\pgfsetstrokeopacity{0.000000}%
\pgfsetdash{}{0pt}%
\pgfpathmoveto{\pgfqpoint{4.659970in}{0.613486in}}%
\pgfpathlineto{\pgfqpoint{4.669975in}{0.613486in}}%
\pgfpathlineto{\pgfqpoint{4.669975in}{2.595224in}}%
\pgfpathlineto{\pgfqpoint{4.659970in}{2.595224in}}%
\pgfpathlineto{\pgfqpoint{4.659970in}{0.613486in}}%
\pgfpathclose%
\pgfusepath{fill}%
\end{pgfscope}%
\begin{pgfscope}%
\pgfpathrectangle{\pgfqpoint{0.693757in}{0.613486in}}{\pgfqpoint{5.541243in}{3.963477in}}%
\pgfusepath{clip}%
\pgfsetbuttcap%
\pgfsetmiterjoin%
\definecolor{currentfill}{rgb}{0.000000,0.000000,1.000000}%
\pgfsetfillcolor{currentfill}%
\pgfsetlinewidth{0.000000pt}%
\definecolor{currentstroke}{rgb}{0.000000,0.000000,0.000000}%
\pgfsetstrokecolor{currentstroke}%
\pgfsetstrokeopacity{0.000000}%
\pgfsetdash{}{0pt}%
\pgfpathmoveto{\pgfqpoint{4.672477in}{0.613486in}}%
\pgfpathlineto{\pgfqpoint{4.682482in}{0.613486in}}%
\pgfpathlineto{\pgfqpoint{4.682482in}{2.611074in}}%
\pgfpathlineto{\pgfqpoint{4.672477in}{2.611074in}}%
\pgfpathlineto{\pgfqpoint{4.672477in}{0.613486in}}%
\pgfpathclose%
\pgfusepath{fill}%
\end{pgfscope}%
\begin{pgfscope}%
\pgfpathrectangle{\pgfqpoint{0.693757in}{0.613486in}}{\pgfqpoint{5.541243in}{3.963477in}}%
\pgfusepath{clip}%
\pgfsetbuttcap%
\pgfsetmiterjoin%
\definecolor{currentfill}{rgb}{0.000000,0.000000,1.000000}%
\pgfsetfillcolor{currentfill}%
\pgfsetlinewidth{0.000000pt}%
\definecolor{currentstroke}{rgb}{0.000000,0.000000,0.000000}%
\pgfsetstrokecolor{currentstroke}%
\pgfsetstrokeopacity{0.000000}%
\pgfsetdash{}{0pt}%
\pgfpathmoveto{\pgfqpoint{4.684983in}{0.613486in}}%
\pgfpathlineto{\pgfqpoint{4.694988in}{0.613486in}}%
\pgfpathlineto{\pgfqpoint{4.694988in}{1.604355in}}%
\pgfpathlineto{\pgfqpoint{4.684983in}{1.604355in}}%
\pgfpathlineto{\pgfqpoint{4.684983in}{0.613486in}}%
\pgfpathclose%
\pgfusepath{fill}%
\end{pgfscope}%
\begin{pgfscope}%
\pgfpathrectangle{\pgfqpoint{0.693757in}{0.613486in}}{\pgfqpoint{5.541243in}{3.963477in}}%
\pgfusepath{clip}%
\pgfsetbuttcap%
\pgfsetmiterjoin%
\definecolor{currentfill}{rgb}{0.000000,0.000000,1.000000}%
\pgfsetfillcolor{currentfill}%
\pgfsetlinewidth{0.000000pt}%
\definecolor{currentstroke}{rgb}{0.000000,0.000000,0.000000}%
\pgfsetstrokecolor{currentstroke}%
\pgfsetstrokeopacity{0.000000}%
\pgfsetdash{}{0pt}%
\pgfpathmoveto{\pgfqpoint{4.697489in}{0.613486in}}%
\pgfpathlineto{\pgfqpoint{4.707494in}{0.613486in}}%
\pgfpathlineto{\pgfqpoint{4.707494in}{1.612282in}}%
\pgfpathlineto{\pgfqpoint{4.697489in}{1.612282in}}%
\pgfpathlineto{\pgfqpoint{4.697489in}{0.613486in}}%
\pgfpathclose%
\pgfusepath{fill}%
\end{pgfscope}%
\begin{pgfscope}%
\pgfpathrectangle{\pgfqpoint{0.693757in}{0.613486in}}{\pgfqpoint{5.541243in}{3.963477in}}%
\pgfusepath{clip}%
\pgfsetbuttcap%
\pgfsetmiterjoin%
\definecolor{currentfill}{rgb}{0.000000,0.000000,1.000000}%
\pgfsetfillcolor{currentfill}%
\pgfsetlinewidth{0.000000pt}%
\definecolor{currentstroke}{rgb}{0.000000,0.000000,0.000000}%
\pgfsetstrokecolor{currentstroke}%
\pgfsetstrokeopacity{0.000000}%
\pgfsetdash{}{0pt}%
\pgfpathmoveto{\pgfqpoint{4.709995in}{0.613486in}}%
\pgfpathlineto{\pgfqpoint{4.720000in}{0.613486in}}%
\pgfpathlineto{\pgfqpoint{4.720000in}{2.595224in}}%
\pgfpathlineto{\pgfqpoint{4.709995in}{2.595224in}}%
\pgfpathlineto{\pgfqpoint{4.709995in}{0.613486in}}%
\pgfpathclose%
\pgfusepath{fill}%
\end{pgfscope}%
\begin{pgfscope}%
\pgfpathrectangle{\pgfqpoint{0.693757in}{0.613486in}}{\pgfqpoint{5.541243in}{3.963477in}}%
\pgfusepath{clip}%
\pgfsetbuttcap%
\pgfsetmiterjoin%
\definecolor{currentfill}{rgb}{0.000000,0.000000,1.000000}%
\pgfsetfillcolor{currentfill}%
\pgfsetlinewidth{0.000000pt}%
\definecolor{currentstroke}{rgb}{0.000000,0.000000,0.000000}%
\pgfsetstrokecolor{currentstroke}%
\pgfsetstrokeopacity{0.000000}%
\pgfsetdash{}{0pt}%
\pgfpathmoveto{\pgfqpoint{4.722501in}{0.613486in}}%
\pgfpathlineto{\pgfqpoint{4.732506in}{0.613486in}}%
\pgfpathlineto{\pgfqpoint{4.732506in}{2.611074in}}%
\pgfpathlineto{\pgfqpoint{4.722501in}{2.611074in}}%
\pgfpathlineto{\pgfqpoint{4.722501in}{0.613486in}}%
\pgfpathclose%
\pgfusepath{fill}%
\end{pgfscope}%
\begin{pgfscope}%
\pgfpathrectangle{\pgfqpoint{0.693757in}{0.613486in}}{\pgfqpoint{5.541243in}{3.963477in}}%
\pgfusepath{clip}%
\pgfsetbuttcap%
\pgfsetmiterjoin%
\definecolor{currentfill}{rgb}{0.000000,0.000000,1.000000}%
\pgfsetfillcolor{currentfill}%
\pgfsetlinewidth{0.000000pt}%
\definecolor{currentstroke}{rgb}{0.000000,0.000000,0.000000}%
\pgfsetstrokecolor{currentstroke}%
\pgfsetstrokeopacity{0.000000}%
\pgfsetdash{}{0pt}%
\pgfpathmoveto{\pgfqpoint{4.735008in}{0.613486in}}%
\pgfpathlineto{\pgfqpoint{4.745012in}{0.613486in}}%
\pgfpathlineto{\pgfqpoint{4.745012in}{1.604355in}}%
\pgfpathlineto{\pgfqpoint{4.735008in}{1.604355in}}%
\pgfpathlineto{\pgfqpoint{4.735008in}{0.613486in}}%
\pgfpathclose%
\pgfusepath{fill}%
\end{pgfscope}%
\begin{pgfscope}%
\pgfpathrectangle{\pgfqpoint{0.693757in}{0.613486in}}{\pgfqpoint{5.541243in}{3.963477in}}%
\pgfusepath{clip}%
\pgfsetbuttcap%
\pgfsetmiterjoin%
\definecolor{currentfill}{rgb}{0.000000,0.000000,1.000000}%
\pgfsetfillcolor{currentfill}%
\pgfsetlinewidth{0.000000pt}%
\definecolor{currentstroke}{rgb}{0.000000,0.000000,0.000000}%
\pgfsetstrokecolor{currentstroke}%
\pgfsetstrokeopacity{0.000000}%
\pgfsetdash{}{0pt}%
\pgfpathmoveto{\pgfqpoint{4.747514in}{0.613486in}}%
\pgfpathlineto{\pgfqpoint{4.757519in}{0.613486in}}%
\pgfpathlineto{\pgfqpoint{4.757519in}{1.612282in}}%
\pgfpathlineto{\pgfqpoint{4.747514in}{1.612282in}}%
\pgfpathlineto{\pgfqpoint{4.747514in}{0.613486in}}%
\pgfpathclose%
\pgfusepath{fill}%
\end{pgfscope}%
\begin{pgfscope}%
\pgfpathrectangle{\pgfqpoint{0.693757in}{0.613486in}}{\pgfqpoint{5.541243in}{3.963477in}}%
\pgfusepath{clip}%
\pgfsetbuttcap%
\pgfsetmiterjoin%
\definecolor{currentfill}{rgb}{0.000000,0.000000,1.000000}%
\pgfsetfillcolor{currentfill}%
\pgfsetlinewidth{0.000000pt}%
\definecolor{currentstroke}{rgb}{0.000000,0.000000,0.000000}%
\pgfsetstrokecolor{currentstroke}%
\pgfsetstrokeopacity{0.000000}%
\pgfsetdash{}{0pt}%
\pgfpathmoveto{\pgfqpoint{4.760020in}{0.613486in}}%
\pgfpathlineto{\pgfqpoint{4.770025in}{0.613486in}}%
\pgfpathlineto{\pgfqpoint{4.770025in}{2.595224in}}%
\pgfpathlineto{\pgfqpoint{4.760020in}{2.595224in}}%
\pgfpathlineto{\pgfqpoint{4.760020in}{0.613486in}}%
\pgfpathclose%
\pgfusepath{fill}%
\end{pgfscope}%
\begin{pgfscope}%
\pgfpathrectangle{\pgfqpoint{0.693757in}{0.613486in}}{\pgfqpoint{5.541243in}{3.963477in}}%
\pgfusepath{clip}%
\pgfsetbuttcap%
\pgfsetmiterjoin%
\definecolor{currentfill}{rgb}{0.000000,0.000000,1.000000}%
\pgfsetfillcolor{currentfill}%
\pgfsetlinewidth{0.000000pt}%
\definecolor{currentstroke}{rgb}{0.000000,0.000000,0.000000}%
\pgfsetstrokecolor{currentstroke}%
\pgfsetstrokeopacity{0.000000}%
\pgfsetdash{}{0pt}%
\pgfpathmoveto{\pgfqpoint{4.772526in}{0.613486in}}%
\pgfpathlineto{\pgfqpoint{4.782531in}{0.613486in}}%
\pgfpathlineto{\pgfqpoint{4.782531in}{2.611074in}}%
\pgfpathlineto{\pgfqpoint{4.772526in}{2.611074in}}%
\pgfpathlineto{\pgfqpoint{4.772526in}{0.613486in}}%
\pgfpathclose%
\pgfusepath{fill}%
\end{pgfscope}%
\begin{pgfscope}%
\pgfpathrectangle{\pgfqpoint{0.693757in}{0.613486in}}{\pgfqpoint{5.541243in}{3.963477in}}%
\pgfusepath{clip}%
\pgfsetbuttcap%
\pgfsetmiterjoin%
\definecolor{currentfill}{rgb}{0.000000,0.000000,1.000000}%
\pgfsetfillcolor{currentfill}%
\pgfsetlinewidth{0.000000pt}%
\definecolor{currentstroke}{rgb}{0.000000,0.000000,0.000000}%
\pgfsetstrokecolor{currentstroke}%
\pgfsetstrokeopacity{0.000000}%
\pgfsetdash{}{0pt}%
\pgfpathmoveto{\pgfqpoint{4.785032in}{0.613486in}}%
\pgfpathlineto{\pgfqpoint{4.795037in}{0.613486in}}%
\pgfpathlineto{\pgfqpoint{4.795037in}{1.604355in}}%
\pgfpathlineto{\pgfqpoint{4.785032in}{1.604355in}}%
\pgfpathlineto{\pgfqpoint{4.785032in}{0.613486in}}%
\pgfpathclose%
\pgfusepath{fill}%
\end{pgfscope}%
\begin{pgfscope}%
\pgfpathrectangle{\pgfqpoint{0.693757in}{0.613486in}}{\pgfqpoint{5.541243in}{3.963477in}}%
\pgfusepath{clip}%
\pgfsetbuttcap%
\pgfsetmiterjoin%
\definecolor{currentfill}{rgb}{0.000000,0.000000,1.000000}%
\pgfsetfillcolor{currentfill}%
\pgfsetlinewidth{0.000000pt}%
\definecolor{currentstroke}{rgb}{0.000000,0.000000,0.000000}%
\pgfsetstrokecolor{currentstroke}%
\pgfsetstrokeopacity{0.000000}%
\pgfsetdash{}{0pt}%
\pgfpathmoveto{\pgfqpoint{4.797538in}{0.613486in}}%
\pgfpathlineto{\pgfqpoint{4.807543in}{0.613486in}}%
\pgfpathlineto{\pgfqpoint{4.807543in}{1.612282in}}%
\pgfpathlineto{\pgfqpoint{4.797538in}{1.612282in}}%
\pgfpathlineto{\pgfqpoint{4.797538in}{0.613486in}}%
\pgfpathclose%
\pgfusepath{fill}%
\end{pgfscope}%
\begin{pgfscope}%
\pgfpathrectangle{\pgfqpoint{0.693757in}{0.613486in}}{\pgfqpoint{5.541243in}{3.963477in}}%
\pgfusepath{clip}%
\pgfsetbuttcap%
\pgfsetmiterjoin%
\definecolor{currentfill}{rgb}{0.000000,0.000000,1.000000}%
\pgfsetfillcolor{currentfill}%
\pgfsetlinewidth{0.000000pt}%
\definecolor{currentstroke}{rgb}{0.000000,0.000000,0.000000}%
\pgfsetstrokecolor{currentstroke}%
\pgfsetstrokeopacity{0.000000}%
\pgfsetdash{}{0pt}%
\pgfpathmoveto{\pgfqpoint{4.810045in}{0.613486in}}%
\pgfpathlineto{\pgfqpoint{4.820050in}{0.613486in}}%
\pgfpathlineto{\pgfqpoint{4.820050in}{2.595224in}}%
\pgfpathlineto{\pgfqpoint{4.810045in}{2.595224in}}%
\pgfpathlineto{\pgfqpoint{4.810045in}{0.613486in}}%
\pgfpathclose%
\pgfusepath{fill}%
\end{pgfscope}%
\begin{pgfscope}%
\pgfpathrectangle{\pgfqpoint{0.693757in}{0.613486in}}{\pgfqpoint{5.541243in}{3.963477in}}%
\pgfusepath{clip}%
\pgfsetbuttcap%
\pgfsetmiterjoin%
\definecolor{currentfill}{rgb}{0.000000,0.000000,1.000000}%
\pgfsetfillcolor{currentfill}%
\pgfsetlinewidth{0.000000pt}%
\definecolor{currentstroke}{rgb}{0.000000,0.000000,0.000000}%
\pgfsetstrokecolor{currentstroke}%
\pgfsetstrokeopacity{0.000000}%
\pgfsetdash{}{0pt}%
\pgfpathmoveto{\pgfqpoint{4.822551in}{0.613486in}}%
\pgfpathlineto{\pgfqpoint{4.832556in}{0.613486in}}%
\pgfpathlineto{\pgfqpoint{4.832556in}{2.611074in}}%
\pgfpathlineto{\pgfqpoint{4.822551in}{2.611074in}}%
\pgfpathlineto{\pgfqpoint{4.822551in}{0.613486in}}%
\pgfpathclose%
\pgfusepath{fill}%
\end{pgfscope}%
\begin{pgfscope}%
\pgfpathrectangle{\pgfqpoint{0.693757in}{0.613486in}}{\pgfqpoint{5.541243in}{3.963477in}}%
\pgfusepath{clip}%
\pgfsetbuttcap%
\pgfsetmiterjoin%
\definecolor{currentfill}{rgb}{0.000000,0.000000,1.000000}%
\pgfsetfillcolor{currentfill}%
\pgfsetlinewidth{0.000000pt}%
\definecolor{currentstroke}{rgb}{0.000000,0.000000,0.000000}%
\pgfsetstrokecolor{currentstroke}%
\pgfsetstrokeopacity{0.000000}%
\pgfsetdash{}{0pt}%
\pgfpathmoveto{\pgfqpoint{4.835057in}{0.613486in}}%
\pgfpathlineto{\pgfqpoint{4.845062in}{0.613486in}}%
\pgfpathlineto{\pgfqpoint{4.845062in}{1.604355in}}%
\pgfpathlineto{\pgfqpoint{4.835057in}{1.604355in}}%
\pgfpathlineto{\pgfqpoint{4.835057in}{0.613486in}}%
\pgfpathclose%
\pgfusepath{fill}%
\end{pgfscope}%
\begin{pgfscope}%
\pgfpathrectangle{\pgfqpoint{0.693757in}{0.613486in}}{\pgfqpoint{5.541243in}{3.963477in}}%
\pgfusepath{clip}%
\pgfsetbuttcap%
\pgfsetmiterjoin%
\definecolor{currentfill}{rgb}{0.000000,0.000000,1.000000}%
\pgfsetfillcolor{currentfill}%
\pgfsetlinewidth{0.000000pt}%
\definecolor{currentstroke}{rgb}{0.000000,0.000000,0.000000}%
\pgfsetstrokecolor{currentstroke}%
\pgfsetstrokeopacity{0.000000}%
\pgfsetdash{}{0pt}%
\pgfpathmoveto{\pgfqpoint{4.847563in}{0.613486in}}%
\pgfpathlineto{\pgfqpoint{4.857568in}{0.613486in}}%
\pgfpathlineto{\pgfqpoint{4.857568in}{1.612282in}}%
\pgfpathlineto{\pgfqpoint{4.847563in}{1.612282in}}%
\pgfpathlineto{\pgfqpoint{4.847563in}{0.613486in}}%
\pgfpathclose%
\pgfusepath{fill}%
\end{pgfscope}%
\begin{pgfscope}%
\pgfpathrectangle{\pgfqpoint{0.693757in}{0.613486in}}{\pgfqpoint{5.541243in}{3.963477in}}%
\pgfusepath{clip}%
\pgfsetbuttcap%
\pgfsetmiterjoin%
\definecolor{currentfill}{rgb}{0.000000,0.000000,1.000000}%
\pgfsetfillcolor{currentfill}%
\pgfsetlinewidth{0.000000pt}%
\definecolor{currentstroke}{rgb}{0.000000,0.000000,0.000000}%
\pgfsetstrokecolor{currentstroke}%
\pgfsetstrokeopacity{0.000000}%
\pgfsetdash{}{0pt}%
\pgfpathmoveto{\pgfqpoint{4.860069in}{0.613486in}}%
\pgfpathlineto{\pgfqpoint{4.870074in}{0.613486in}}%
\pgfpathlineto{\pgfqpoint{4.870074in}{2.595224in}}%
\pgfpathlineto{\pgfqpoint{4.860069in}{2.595224in}}%
\pgfpathlineto{\pgfqpoint{4.860069in}{0.613486in}}%
\pgfpathclose%
\pgfusepath{fill}%
\end{pgfscope}%
\begin{pgfscope}%
\pgfpathrectangle{\pgfqpoint{0.693757in}{0.613486in}}{\pgfqpoint{5.541243in}{3.963477in}}%
\pgfusepath{clip}%
\pgfsetbuttcap%
\pgfsetmiterjoin%
\definecolor{currentfill}{rgb}{0.000000,0.000000,1.000000}%
\pgfsetfillcolor{currentfill}%
\pgfsetlinewidth{0.000000pt}%
\definecolor{currentstroke}{rgb}{0.000000,0.000000,0.000000}%
\pgfsetstrokecolor{currentstroke}%
\pgfsetstrokeopacity{0.000000}%
\pgfsetdash{}{0pt}%
\pgfpathmoveto{\pgfqpoint{4.872576in}{0.613486in}}%
\pgfpathlineto{\pgfqpoint{4.882581in}{0.613486in}}%
\pgfpathlineto{\pgfqpoint{4.882581in}{2.611074in}}%
\pgfpathlineto{\pgfqpoint{4.872576in}{2.611074in}}%
\pgfpathlineto{\pgfqpoint{4.872576in}{0.613486in}}%
\pgfpathclose%
\pgfusepath{fill}%
\end{pgfscope}%
\begin{pgfscope}%
\pgfpathrectangle{\pgfqpoint{0.693757in}{0.613486in}}{\pgfqpoint{5.541243in}{3.963477in}}%
\pgfusepath{clip}%
\pgfsetbuttcap%
\pgfsetmiterjoin%
\definecolor{currentfill}{rgb}{0.000000,0.000000,1.000000}%
\pgfsetfillcolor{currentfill}%
\pgfsetlinewidth{0.000000pt}%
\definecolor{currentstroke}{rgb}{0.000000,0.000000,0.000000}%
\pgfsetstrokecolor{currentstroke}%
\pgfsetstrokeopacity{0.000000}%
\pgfsetdash{}{0pt}%
\pgfpathmoveto{\pgfqpoint{4.885082in}{0.613486in}}%
\pgfpathlineto{\pgfqpoint{4.895087in}{0.613486in}}%
\pgfpathlineto{\pgfqpoint{4.895087in}{1.604355in}}%
\pgfpathlineto{\pgfqpoint{4.885082in}{1.604355in}}%
\pgfpathlineto{\pgfqpoint{4.885082in}{0.613486in}}%
\pgfpathclose%
\pgfusepath{fill}%
\end{pgfscope}%
\begin{pgfscope}%
\pgfpathrectangle{\pgfqpoint{0.693757in}{0.613486in}}{\pgfqpoint{5.541243in}{3.963477in}}%
\pgfusepath{clip}%
\pgfsetbuttcap%
\pgfsetmiterjoin%
\definecolor{currentfill}{rgb}{0.000000,0.000000,1.000000}%
\pgfsetfillcolor{currentfill}%
\pgfsetlinewidth{0.000000pt}%
\definecolor{currentstroke}{rgb}{0.000000,0.000000,0.000000}%
\pgfsetstrokecolor{currentstroke}%
\pgfsetstrokeopacity{0.000000}%
\pgfsetdash{}{0pt}%
\pgfpathmoveto{\pgfqpoint{4.897588in}{0.613486in}}%
\pgfpathlineto{\pgfqpoint{4.907593in}{0.613486in}}%
\pgfpathlineto{\pgfqpoint{4.907593in}{1.612282in}}%
\pgfpathlineto{\pgfqpoint{4.897588in}{1.612282in}}%
\pgfpathlineto{\pgfqpoint{4.897588in}{0.613486in}}%
\pgfpathclose%
\pgfusepath{fill}%
\end{pgfscope}%
\begin{pgfscope}%
\pgfpathrectangle{\pgfqpoint{0.693757in}{0.613486in}}{\pgfqpoint{5.541243in}{3.963477in}}%
\pgfusepath{clip}%
\pgfsetbuttcap%
\pgfsetmiterjoin%
\definecolor{currentfill}{rgb}{0.000000,0.000000,1.000000}%
\pgfsetfillcolor{currentfill}%
\pgfsetlinewidth{0.000000pt}%
\definecolor{currentstroke}{rgb}{0.000000,0.000000,0.000000}%
\pgfsetstrokecolor{currentstroke}%
\pgfsetstrokeopacity{0.000000}%
\pgfsetdash{}{0pt}%
\pgfpathmoveto{\pgfqpoint{4.910094in}{0.613486in}}%
\pgfpathlineto{\pgfqpoint{4.920099in}{0.613486in}}%
\pgfpathlineto{\pgfqpoint{4.920099in}{2.595224in}}%
\pgfpathlineto{\pgfqpoint{4.910094in}{2.595224in}}%
\pgfpathlineto{\pgfqpoint{4.910094in}{0.613486in}}%
\pgfpathclose%
\pgfusepath{fill}%
\end{pgfscope}%
\begin{pgfscope}%
\pgfpathrectangle{\pgfqpoint{0.693757in}{0.613486in}}{\pgfqpoint{5.541243in}{3.963477in}}%
\pgfusepath{clip}%
\pgfsetbuttcap%
\pgfsetmiterjoin%
\definecolor{currentfill}{rgb}{0.000000,0.000000,1.000000}%
\pgfsetfillcolor{currentfill}%
\pgfsetlinewidth{0.000000pt}%
\definecolor{currentstroke}{rgb}{0.000000,0.000000,0.000000}%
\pgfsetstrokecolor{currentstroke}%
\pgfsetstrokeopacity{0.000000}%
\pgfsetdash{}{0pt}%
\pgfpathmoveto{\pgfqpoint{4.922600in}{0.613486in}}%
\pgfpathlineto{\pgfqpoint{4.932605in}{0.613486in}}%
\pgfpathlineto{\pgfqpoint{4.932605in}{2.611074in}}%
\pgfpathlineto{\pgfqpoint{4.922600in}{2.611074in}}%
\pgfpathlineto{\pgfqpoint{4.922600in}{0.613486in}}%
\pgfpathclose%
\pgfusepath{fill}%
\end{pgfscope}%
\begin{pgfscope}%
\pgfpathrectangle{\pgfqpoint{0.693757in}{0.613486in}}{\pgfqpoint{5.541243in}{3.963477in}}%
\pgfusepath{clip}%
\pgfsetbuttcap%
\pgfsetmiterjoin%
\definecolor{currentfill}{rgb}{0.000000,0.000000,1.000000}%
\pgfsetfillcolor{currentfill}%
\pgfsetlinewidth{0.000000pt}%
\definecolor{currentstroke}{rgb}{0.000000,0.000000,0.000000}%
\pgfsetstrokecolor{currentstroke}%
\pgfsetstrokeopacity{0.000000}%
\pgfsetdash{}{0pt}%
\pgfpathmoveto{\pgfqpoint{4.935107in}{0.613486in}}%
\pgfpathlineto{\pgfqpoint{4.945112in}{0.613486in}}%
\pgfpathlineto{\pgfqpoint{4.945112in}{1.604355in}}%
\pgfpathlineto{\pgfqpoint{4.935107in}{1.604355in}}%
\pgfpathlineto{\pgfqpoint{4.935107in}{0.613486in}}%
\pgfpathclose%
\pgfusepath{fill}%
\end{pgfscope}%
\begin{pgfscope}%
\pgfpathrectangle{\pgfqpoint{0.693757in}{0.613486in}}{\pgfqpoint{5.541243in}{3.963477in}}%
\pgfusepath{clip}%
\pgfsetbuttcap%
\pgfsetmiterjoin%
\definecolor{currentfill}{rgb}{0.000000,0.000000,1.000000}%
\pgfsetfillcolor{currentfill}%
\pgfsetlinewidth{0.000000pt}%
\definecolor{currentstroke}{rgb}{0.000000,0.000000,0.000000}%
\pgfsetstrokecolor{currentstroke}%
\pgfsetstrokeopacity{0.000000}%
\pgfsetdash{}{0pt}%
\pgfpathmoveto{\pgfqpoint{4.947613in}{0.613486in}}%
\pgfpathlineto{\pgfqpoint{4.957618in}{0.613486in}}%
\pgfpathlineto{\pgfqpoint{4.957618in}{1.612282in}}%
\pgfpathlineto{\pgfqpoint{4.947613in}{1.612282in}}%
\pgfpathlineto{\pgfqpoint{4.947613in}{0.613486in}}%
\pgfpathclose%
\pgfusepath{fill}%
\end{pgfscope}%
\begin{pgfscope}%
\pgfpathrectangle{\pgfqpoint{0.693757in}{0.613486in}}{\pgfqpoint{5.541243in}{3.963477in}}%
\pgfusepath{clip}%
\pgfsetbuttcap%
\pgfsetmiterjoin%
\definecolor{currentfill}{rgb}{0.000000,0.000000,1.000000}%
\pgfsetfillcolor{currentfill}%
\pgfsetlinewidth{0.000000pt}%
\definecolor{currentstroke}{rgb}{0.000000,0.000000,0.000000}%
\pgfsetstrokecolor{currentstroke}%
\pgfsetstrokeopacity{0.000000}%
\pgfsetdash{}{0pt}%
\pgfpathmoveto{\pgfqpoint{4.960119in}{0.613486in}}%
\pgfpathlineto{\pgfqpoint{4.970124in}{0.613486in}}%
\pgfpathlineto{\pgfqpoint{4.970124in}{2.595224in}}%
\pgfpathlineto{\pgfqpoint{4.960119in}{2.595224in}}%
\pgfpathlineto{\pgfqpoint{4.960119in}{0.613486in}}%
\pgfpathclose%
\pgfusepath{fill}%
\end{pgfscope}%
\begin{pgfscope}%
\pgfpathrectangle{\pgfqpoint{0.693757in}{0.613486in}}{\pgfqpoint{5.541243in}{3.963477in}}%
\pgfusepath{clip}%
\pgfsetbuttcap%
\pgfsetmiterjoin%
\definecolor{currentfill}{rgb}{0.000000,0.000000,1.000000}%
\pgfsetfillcolor{currentfill}%
\pgfsetlinewidth{0.000000pt}%
\definecolor{currentstroke}{rgb}{0.000000,0.000000,0.000000}%
\pgfsetstrokecolor{currentstroke}%
\pgfsetstrokeopacity{0.000000}%
\pgfsetdash{}{0pt}%
\pgfpathmoveto{\pgfqpoint{4.972625in}{0.613486in}}%
\pgfpathlineto{\pgfqpoint{4.982630in}{0.613486in}}%
\pgfpathlineto{\pgfqpoint{4.982630in}{2.611074in}}%
\pgfpathlineto{\pgfqpoint{4.972625in}{2.611074in}}%
\pgfpathlineto{\pgfqpoint{4.972625in}{0.613486in}}%
\pgfpathclose%
\pgfusepath{fill}%
\end{pgfscope}%
\begin{pgfscope}%
\pgfpathrectangle{\pgfqpoint{0.693757in}{0.613486in}}{\pgfqpoint{5.541243in}{3.963477in}}%
\pgfusepath{clip}%
\pgfsetbuttcap%
\pgfsetmiterjoin%
\definecolor{currentfill}{rgb}{0.000000,0.000000,1.000000}%
\pgfsetfillcolor{currentfill}%
\pgfsetlinewidth{0.000000pt}%
\definecolor{currentstroke}{rgb}{0.000000,0.000000,0.000000}%
\pgfsetstrokecolor{currentstroke}%
\pgfsetstrokeopacity{0.000000}%
\pgfsetdash{}{0pt}%
\pgfpathmoveto{\pgfqpoint{4.985131in}{0.613486in}}%
\pgfpathlineto{\pgfqpoint{4.995136in}{0.613486in}}%
\pgfpathlineto{\pgfqpoint{4.995136in}{1.604355in}}%
\pgfpathlineto{\pgfqpoint{4.985131in}{1.604355in}}%
\pgfpathlineto{\pgfqpoint{4.985131in}{0.613486in}}%
\pgfpathclose%
\pgfusepath{fill}%
\end{pgfscope}%
\begin{pgfscope}%
\pgfpathrectangle{\pgfqpoint{0.693757in}{0.613486in}}{\pgfqpoint{5.541243in}{3.963477in}}%
\pgfusepath{clip}%
\pgfsetbuttcap%
\pgfsetmiterjoin%
\definecolor{currentfill}{rgb}{0.000000,0.000000,1.000000}%
\pgfsetfillcolor{currentfill}%
\pgfsetlinewidth{0.000000pt}%
\definecolor{currentstroke}{rgb}{0.000000,0.000000,0.000000}%
\pgfsetstrokecolor{currentstroke}%
\pgfsetstrokeopacity{0.000000}%
\pgfsetdash{}{0pt}%
\pgfpathmoveto{\pgfqpoint{4.997638in}{0.613486in}}%
\pgfpathlineto{\pgfqpoint{5.007642in}{0.613486in}}%
\pgfpathlineto{\pgfqpoint{5.007642in}{1.612282in}}%
\pgfpathlineto{\pgfqpoint{4.997638in}{1.612282in}}%
\pgfpathlineto{\pgfqpoint{4.997638in}{0.613486in}}%
\pgfpathclose%
\pgfusepath{fill}%
\end{pgfscope}%
\begin{pgfscope}%
\pgfpathrectangle{\pgfqpoint{0.693757in}{0.613486in}}{\pgfqpoint{5.541243in}{3.963477in}}%
\pgfusepath{clip}%
\pgfsetbuttcap%
\pgfsetmiterjoin%
\definecolor{currentfill}{rgb}{0.000000,0.000000,1.000000}%
\pgfsetfillcolor{currentfill}%
\pgfsetlinewidth{0.000000pt}%
\definecolor{currentstroke}{rgb}{0.000000,0.000000,0.000000}%
\pgfsetstrokecolor{currentstroke}%
\pgfsetstrokeopacity{0.000000}%
\pgfsetdash{}{0pt}%
\pgfpathmoveto{\pgfqpoint{5.010144in}{0.613486in}}%
\pgfpathlineto{\pgfqpoint{5.020149in}{0.613486in}}%
\pgfpathlineto{\pgfqpoint{5.020149in}{2.595224in}}%
\pgfpathlineto{\pgfqpoint{5.010144in}{2.595224in}}%
\pgfpathlineto{\pgfqpoint{5.010144in}{0.613486in}}%
\pgfpathclose%
\pgfusepath{fill}%
\end{pgfscope}%
\begin{pgfscope}%
\pgfpathrectangle{\pgfqpoint{0.693757in}{0.613486in}}{\pgfqpoint{5.541243in}{3.963477in}}%
\pgfusepath{clip}%
\pgfsetbuttcap%
\pgfsetmiterjoin%
\definecolor{currentfill}{rgb}{0.000000,0.000000,1.000000}%
\pgfsetfillcolor{currentfill}%
\pgfsetlinewidth{0.000000pt}%
\definecolor{currentstroke}{rgb}{0.000000,0.000000,0.000000}%
\pgfsetstrokecolor{currentstroke}%
\pgfsetstrokeopacity{0.000000}%
\pgfsetdash{}{0pt}%
\pgfpathmoveto{\pgfqpoint{5.022650in}{0.613486in}}%
\pgfpathlineto{\pgfqpoint{5.032655in}{0.613486in}}%
\pgfpathlineto{\pgfqpoint{5.032655in}{2.611074in}}%
\pgfpathlineto{\pgfqpoint{5.022650in}{2.611074in}}%
\pgfpathlineto{\pgfqpoint{5.022650in}{0.613486in}}%
\pgfpathclose%
\pgfusepath{fill}%
\end{pgfscope}%
\begin{pgfscope}%
\pgfpathrectangle{\pgfqpoint{0.693757in}{0.613486in}}{\pgfqpoint{5.541243in}{3.963477in}}%
\pgfusepath{clip}%
\pgfsetbuttcap%
\pgfsetmiterjoin%
\definecolor{currentfill}{rgb}{0.000000,0.000000,1.000000}%
\pgfsetfillcolor{currentfill}%
\pgfsetlinewidth{0.000000pt}%
\definecolor{currentstroke}{rgb}{0.000000,0.000000,0.000000}%
\pgfsetstrokecolor{currentstroke}%
\pgfsetstrokeopacity{0.000000}%
\pgfsetdash{}{0pt}%
\pgfpathmoveto{\pgfqpoint{5.035156in}{0.613486in}}%
\pgfpathlineto{\pgfqpoint{5.045161in}{0.613486in}}%
\pgfpathlineto{\pgfqpoint{5.045161in}{1.604355in}}%
\pgfpathlineto{\pgfqpoint{5.035156in}{1.604355in}}%
\pgfpathlineto{\pgfqpoint{5.035156in}{0.613486in}}%
\pgfpathclose%
\pgfusepath{fill}%
\end{pgfscope}%
\begin{pgfscope}%
\pgfpathrectangle{\pgfqpoint{0.693757in}{0.613486in}}{\pgfqpoint{5.541243in}{3.963477in}}%
\pgfusepath{clip}%
\pgfsetbuttcap%
\pgfsetmiterjoin%
\definecolor{currentfill}{rgb}{0.000000,0.000000,1.000000}%
\pgfsetfillcolor{currentfill}%
\pgfsetlinewidth{0.000000pt}%
\definecolor{currentstroke}{rgb}{0.000000,0.000000,0.000000}%
\pgfsetstrokecolor{currentstroke}%
\pgfsetstrokeopacity{0.000000}%
\pgfsetdash{}{0pt}%
\pgfpathmoveto{\pgfqpoint{5.047662in}{0.613486in}}%
\pgfpathlineto{\pgfqpoint{5.057667in}{0.613486in}}%
\pgfpathlineto{\pgfqpoint{5.057667in}{1.612282in}}%
\pgfpathlineto{\pgfqpoint{5.047662in}{1.612282in}}%
\pgfpathlineto{\pgfqpoint{5.047662in}{0.613486in}}%
\pgfpathclose%
\pgfusepath{fill}%
\end{pgfscope}%
\begin{pgfscope}%
\pgfpathrectangle{\pgfqpoint{0.693757in}{0.613486in}}{\pgfqpoint{5.541243in}{3.963477in}}%
\pgfusepath{clip}%
\pgfsetbuttcap%
\pgfsetmiterjoin%
\definecolor{currentfill}{rgb}{0.000000,0.000000,1.000000}%
\pgfsetfillcolor{currentfill}%
\pgfsetlinewidth{0.000000pt}%
\definecolor{currentstroke}{rgb}{0.000000,0.000000,0.000000}%
\pgfsetstrokecolor{currentstroke}%
\pgfsetstrokeopacity{0.000000}%
\pgfsetdash{}{0pt}%
\pgfpathmoveto{\pgfqpoint{5.060168in}{0.613486in}}%
\pgfpathlineto{\pgfqpoint{5.070173in}{0.613486in}}%
\pgfpathlineto{\pgfqpoint{5.070173in}{2.595224in}}%
\pgfpathlineto{\pgfqpoint{5.060168in}{2.595224in}}%
\pgfpathlineto{\pgfqpoint{5.060168in}{0.613486in}}%
\pgfpathclose%
\pgfusepath{fill}%
\end{pgfscope}%
\begin{pgfscope}%
\pgfpathrectangle{\pgfqpoint{0.693757in}{0.613486in}}{\pgfqpoint{5.541243in}{3.963477in}}%
\pgfusepath{clip}%
\pgfsetbuttcap%
\pgfsetmiterjoin%
\definecolor{currentfill}{rgb}{0.000000,0.000000,1.000000}%
\pgfsetfillcolor{currentfill}%
\pgfsetlinewidth{0.000000pt}%
\definecolor{currentstroke}{rgb}{0.000000,0.000000,0.000000}%
\pgfsetstrokecolor{currentstroke}%
\pgfsetstrokeopacity{0.000000}%
\pgfsetdash{}{0pt}%
\pgfpathmoveto{\pgfqpoint{5.072675in}{0.613486in}}%
\pgfpathlineto{\pgfqpoint{5.082680in}{0.613486in}}%
\pgfpathlineto{\pgfqpoint{5.082680in}{2.611074in}}%
\pgfpathlineto{\pgfqpoint{5.072675in}{2.611074in}}%
\pgfpathlineto{\pgfqpoint{5.072675in}{0.613486in}}%
\pgfpathclose%
\pgfusepath{fill}%
\end{pgfscope}%
\begin{pgfscope}%
\pgfpathrectangle{\pgfqpoint{0.693757in}{0.613486in}}{\pgfqpoint{5.541243in}{3.963477in}}%
\pgfusepath{clip}%
\pgfsetbuttcap%
\pgfsetmiterjoin%
\definecolor{currentfill}{rgb}{0.000000,0.000000,1.000000}%
\pgfsetfillcolor{currentfill}%
\pgfsetlinewidth{0.000000pt}%
\definecolor{currentstroke}{rgb}{0.000000,0.000000,0.000000}%
\pgfsetstrokecolor{currentstroke}%
\pgfsetstrokeopacity{0.000000}%
\pgfsetdash{}{0pt}%
\pgfpathmoveto{\pgfqpoint{5.085181in}{0.613486in}}%
\pgfpathlineto{\pgfqpoint{5.095186in}{0.613486in}}%
\pgfpathlineto{\pgfqpoint{5.095186in}{1.604355in}}%
\pgfpathlineto{\pgfqpoint{5.085181in}{1.604355in}}%
\pgfpathlineto{\pgfqpoint{5.085181in}{0.613486in}}%
\pgfpathclose%
\pgfusepath{fill}%
\end{pgfscope}%
\begin{pgfscope}%
\pgfpathrectangle{\pgfqpoint{0.693757in}{0.613486in}}{\pgfqpoint{5.541243in}{3.963477in}}%
\pgfusepath{clip}%
\pgfsetbuttcap%
\pgfsetmiterjoin%
\definecolor{currentfill}{rgb}{0.000000,0.000000,1.000000}%
\pgfsetfillcolor{currentfill}%
\pgfsetlinewidth{0.000000pt}%
\definecolor{currentstroke}{rgb}{0.000000,0.000000,0.000000}%
\pgfsetstrokecolor{currentstroke}%
\pgfsetstrokeopacity{0.000000}%
\pgfsetdash{}{0pt}%
\pgfpathmoveto{\pgfqpoint{5.097687in}{0.613486in}}%
\pgfpathlineto{\pgfqpoint{5.107692in}{0.613486in}}%
\pgfpathlineto{\pgfqpoint{5.107692in}{1.612282in}}%
\pgfpathlineto{\pgfqpoint{5.097687in}{1.612282in}}%
\pgfpathlineto{\pgfqpoint{5.097687in}{0.613486in}}%
\pgfpathclose%
\pgfusepath{fill}%
\end{pgfscope}%
\begin{pgfscope}%
\pgfpathrectangle{\pgfqpoint{0.693757in}{0.613486in}}{\pgfqpoint{5.541243in}{3.963477in}}%
\pgfusepath{clip}%
\pgfsetbuttcap%
\pgfsetmiterjoin%
\definecolor{currentfill}{rgb}{0.000000,0.000000,1.000000}%
\pgfsetfillcolor{currentfill}%
\pgfsetlinewidth{0.000000pt}%
\definecolor{currentstroke}{rgb}{0.000000,0.000000,0.000000}%
\pgfsetstrokecolor{currentstroke}%
\pgfsetstrokeopacity{0.000000}%
\pgfsetdash{}{0pt}%
\pgfpathmoveto{\pgfqpoint{5.110193in}{0.613486in}}%
\pgfpathlineto{\pgfqpoint{5.120198in}{0.613486in}}%
\pgfpathlineto{\pgfqpoint{5.120198in}{2.595224in}}%
\pgfpathlineto{\pgfqpoint{5.110193in}{2.595224in}}%
\pgfpathlineto{\pgfqpoint{5.110193in}{0.613486in}}%
\pgfpathclose%
\pgfusepath{fill}%
\end{pgfscope}%
\begin{pgfscope}%
\pgfpathrectangle{\pgfqpoint{0.693757in}{0.613486in}}{\pgfqpoint{5.541243in}{3.963477in}}%
\pgfusepath{clip}%
\pgfsetbuttcap%
\pgfsetmiterjoin%
\definecolor{currentfill}{rgb}{0.000000,0.000000,1.000000}%
\pgfsetfillcolor{currentfill}%
\pgfsetlinewidth{0.000000pt}%
\definecolor{currentstroke}{rgb}{0.000000,0.000000,0.000000}%
\pgfsetstrokecolor{currentstroke}%
\pgfsetstrokeopacity{0.000000}%
\pgfsetdash{}{0pt}%
\pgfpathmoveto{\pgfqpoint{5.122699in}{0.613486in}}%
\pgfpathlineto{\pgfqpoint{5.132704in}{0.613486in}}%
\pgfpathlineto{\pgfqpoint{5.132704in}{2.611074in}}%
\pgfpathlineto{\pgfqpoint{5.122699in}{2.611074in}}%
\pgfpathlineto{\pgfqpoint{5.122699in}{0.613486in}}%
\pgfpathclose%
\pgfusepath{fill}%
\end{pgfscope}%
\begin{pgfscope}%
\pgfpathrectangle{\pgfqpoint{0.693757in}{0.613486in}}{\pgfqpoint{5.541243in}{3.963477in}}%
\pgfusepath{clip}%
\pgfsetbuttcap%
\pgfsetmiterjoin%
\definecolor{currentfill}{rgb}{0.000000,0.000000,1.000000}%
\pgfsetfillcolor{currentfill}%
\pgfsetlinewidth{0.000000pt}%
\definecolor{currentstroke}{rgb}{0.000000,0.000000,0.000000}%
\pgfsetstrokecolor{currentstroke}%
\pgfsetstrokeopacity{0.000000}%
\pgfsetdash{}{0pt}%
\pgfpathmoveto{\pgfqpoint{5.135206in}{0.613486in}}%
\pgfpathlineto{\pgfqpoint{5.145211in}{0.613486in}}%
\pgfpathlineto{\pgfqpoint{5.145211in}{1.604355in}}%
\pgfpathlineto{\pgfqpoint{5.135206in}{1.604355in}}%
\pgfpathlineto{\pgfqpoint{5.135206in}{0.613486in}}%
\pgfpathclose%
\pgfusepath{fill}%
\end{pgfscope}%
\begin{pgfscope}%
\pgfpathrectangle{\pgfqpoint{0.693757in}{0.613486in}}{\pgfqpoint{5.541243in}{3.963477in}}%
\pgfusepath{clip}%
\pgfsetbuttcap%
\pgfsetmiterjoin%
\definecolor{currentfill}{rgb}{0.000000,0.000000,1.000000}%
\pgfsetfillcolor{currentfill}%
\pgfsetlinewidth{0.000000pt}%
\definecolor{currentstroke}{rgb}{0.000000,0.000000,0.000000}%
\pgfsetstrokecolor{currentstroke}%
\pgfsetstrokeopacity{0.000000}%
\pgfsetdash{}{0pt}%
\pgfpathmoveto{\pgfqpoint{5.147712in}{0.613486in}}%
\pgfpathlineto{\pgfqpoint{5.157717in}{0.613486in}}%
\pgfpathlineto{\pgfqpoint{5.157717in}{1.612282in}}%
\pgfpathlineto{\pgfqpoint{5.147712in}{1.612282in}}%
\pgfpathlineto{\pgfqpoint{5.147712in}{0.613486in}}%
\pgfpathclose%
\pgfusepath{fill}%
\end{pgfscope}%
\begin{pgfscope}%
\pgfpathrectangle{\pgfqpoint{0.693757in}{0.613486in}}{\pgfqpoint{5.541243in}{3.963477in}}%
\pgfusepath{clip}%
\pgfsetbuttcap%
\pgfsetmiterjoin%
\definecolor{currentfill}{rgb}{0.000000,0.000000,1.000000}%
\pgfsetfillcolor{currentfill}%
\pgfsetlinewidth{0.000000pt}%
\definecolor{currentstroke}{rgb}{0.000000,0.000000,0.000000}%
\pgfsetstrokecolor{currentstroke}%
\pgfsetstrokeopacity{0.000000}%
\pgfsetdash{}{0pt}%
\pgfpathmoveto{\pgfqpoint{5.160218in}{0.613486in}}%
\pgfpathlineto{\pgfqpoint{5.170223in}{0.613486in}}%
\pgfpathlineto{\pgfqpoint{5.170223in}{2.595224in}}%
\pgfpathlineto{\pgfqpoint{5.160218in}{2.595224in}}%
\pgfpathlineto{\pgfqpoint{5.160218in}{0.613486in}}%
\pgfpathclose%
\pgfusepath{fill}%
\end{pgfscope}%
\begin{pgfscope}%
\pgfpathrectangle{\pgfqpoint{0.693757in}{0.613486in}}{\pgfqpoint{5.541243in}{3.963477in}}%
\pgfusepath{clip}%
\pgfsetbuttcap%
\pgfsetmiterjoin%
\definecolor{currentfill}{rgb}{0.000000,0.000000,1.000000}%
\pgfsetfillcolor{currentfill}%
\pgfsetlinewidth{0.000000pt}%
\definecolor{currentstroke}{rgb}{0.000000,0.000000,0.000000}%
\pgfsetstrokecolor{currentstroke}%
\pgfsetstrokeopacity{0.000000}%
\pgfsetdash{}{0pt}%
\pgfpathmoveto{\pgfqpoint{5.172724in}{0.613486in}}%
\pgfpathlineto{\pgfqpoint{5.182729in}{0.613486in}}%
\pgfpathlineto{\pgfqpoint{5.182729in}{2.611074in}}%
\pgfpathlineto{\pgfqpoint{5.172724in}{2.611074in}}%
\pgfpathlineto{\pgfqpoint{5.172724in}{0.613486in}}%
\pgfpathclose%
\pgfusepath{fill}%
\end{pgfscope}%
\begin{pgfscope}%
\pgfpathrectangle{\pgfqpoint{0.693757in}{0.613486in}}{\pgfqpoint{5.541243in}{3.963477in}}%
\pgfusepath{clip}%
\pgfsetbuttcap%
\pgfsetmiterjoin%
\definecolor{currentfill}{rgb}{0.000000,0.000000,1.000000}%
\pgfsetfillcolor{currentfill}%
\pgfsetlinewidth{0.000000pt}%
\definecolor{currentstroke}{rgb}{0.000000,0.000000,0.000000}%
\pgfsetstrokecolor{currentstroke}%
\pgfsetstrokeopacity{0.000000}%
\pgfsetdash{}{0pt}%
\pgfpathmoveto{\pgfqpoint{5.185230in}{0.613486in}}%
\pgfpathlineto{\pgfqpoint{5.195235in}{0.613486in}}%
\pgfpathlineto{\pgfqpoint{5.195235in}{1.604355in}}%
\pgfpathlineto{\pgfqpoint{5.185230in}{1.604355in}}%
\pgfpathlineto{\pgfqpoint{5.185230in}{0.613486in}}%
\pgfpathclose%
\pgfusepath{fill}%
\end{pgfscope}%
\begin{pgfscope}%
\pgfpathrectangle{\pgfqpoint{0.693757in}{0.613486in}}{\pgfqpoint{5.541243in}{3.963477in}}%
\pgfusepath{clip}%
\pgfsetbuttcap%
\pgfsetmiterjoin%
\definecolor{currentfill}{rgb}{0.000000,0.000000,1.000000}%
\pgfsetfillcolor{currentfill}%
\pgfsetlinewidth{0.000000pt}%
\definecolor{currentstroke}{rgb}{0.000000,0.000000,0.000000}%
\pgfsetstrokecolor{currentstroke}%
\pgfsetstrokeopacity{0.000000}%
\pgfsetdash{}{0pt}%
\pgfpathmoveto{\pgfqpoint{5.197737in}{0.613486in}}%
\pgfpathlineto{\pgfqpoint{5.207742in}{0.613486in}}%
\pgfpathlineto{\pgfqpoint{5.207742in}{1.612282in}}%
\pgfpathlineto{\pgfqpoint{5.197737in}{1.612282in}}%
\pgfpathlineto{\pgfqpoint{5.197737in}{0.613486in}}%
\pgfpathclose%
\pgfusepath{fill}%
\end{pgfscope}%
\begin{pgfscope}%
\pgfpathrectangle{\pgfqpoint{0.693757in}{0.613486in}}{\pgfqpoint{5.541243in}{3.963477in}}%
\pgfusepath{clip}%
\pgfsetbuttcap%
\pgfsetmiterjoin%
\definecolor{currentfill}{rgb}{0.000000,0.000000,1.000000}%
\pgfsetfillcolor{currentfill}%
\pgfsetlinewidth{0.000000pt}%
\definecolor{currentstroke}{rgb}{0.000000,0.000000,0.000000}%
\pgfsetstrokecolor{currentstroke}%
\pgfsetstrokeopacity{0.000000}%
\pgfsetdash{}{0pt}%
\pgfpathmoveto{\pgfqpoint{5.210243in}{0.613486in}}%
\pgfpathlineto{\pgfqpoint{5.220248in}{0.613486in}}%
\pgfpathlineto{\pgfqpoint{5.220248in}{2.595224in}}%
\pgfpathlineto{\pgfqpoint{5.210243in}{2.595224in}}%
\pgfpathlineto{\pgfqpoint{5.210243in}{0.613486in}}%
\pgfpathclose%
\pgfusepath{fill}%
\end{pgfscope}%
\begin{pgfscope}%
\pgfpathrectangle{\pgfqpoint{0.693757in}{0.613486in}}{\pgfqpoint{5.541243in}{3.963477in}}%
\pgfusepath{clip}%
\pgfsetbuttcap%
\pgfsetmiterjoin%
\definecolor{currentfill}{rgb}{0.000000,0.000000,1.000000}%
\pgfsetfillcolor{currentfill}%
\pgfsetlinewidth{0.000000pt}%
\definecolor{currentstroke}{rgb}{0.000000,0.000000,0.000000}%
\pgfsetstrokecolor{currentstroke}%
\pgfsetstrokeopacity{0.000000}%
\pgfsetdash{}{0pt}%
\pgfpathmoveto{\pgfqpoint{5.222749in}{0.613486in}}%
\pgfpathlineto{\pgfqpoint{5.232754in}{0.613486in}}%
\pgfpathlineto{\pgfqpoint{5.232754in}{2.611074in}}%
\pgfpathlineto{\pgfqpoint{5.222749in}{2.611074in}}%
\pgfpathlineto{\pgfqpoint{5.222749in}{0.613486in}}%
\pgfpathclose%
\pgfusepath{fill}%
\end{pgfscope}%
\begin{pgfscope}%
\pgfpathrectangle{\pgfqpoint{0.693757in}{0.613486in}}{\pgfqpoint{5.541243in}{3.963477in}}%
\pgfusepath{clip}%
\pgfsetbuttcap%
\pgfsetmiterjoin%
\definecolor{currentfill}{rgb}{0.000000,0.000000,1.000000}%
\pgfsetfillcolor{currentfill}%
\pgfsetlinewidth{0.000000pt}%
\definecolor{currentstroke}{rgb}{0.000000,0.000000,0.000000}%
\pgfsetstrokecolor{currentstroke}%
\pgfsetstrokeopacity{0.000000}%
\pgfsetdash{}{0pt}%
\pgfpathmoveto{\pgfqpoint{5.235255in}{0.613486in}}%
\pgfpathlineto{\pgfqpoint{5.245260in}{0.613486in}}%
\pgfpathlineto{\pgfqpoint{5.245260in}{1.604355in}}%
\pgfpathlineto{\pgfqpoint{5.235255in}{1.604355in}}%
\pgfpathlineto{\pgfqpoint{5.235255in}{0.613486in}}%
\pgfpathclose%
\pgfusepath{fill}%
\end{pgfscope}%
\begin{pgfscope}%
\pgfpathrectangle{\pgfqpoint{0.693757in}{0.613486in}}{\pgfqpoint{5.541243in}{3.963477in}}%
\pgfusepath{clip}%
\pgfsetbuttcap%
\pgfsetmiterjoin%
\definecolor{currentfill}{rgb}{0.000000,0.000000,1.000000}%
\pgfsetfillcolor{currentfill}%
\pgfsetlinewidth{0.000000pt}%
\definecolor{currentstroke}{rgb}{0.000000,0.000000,0.000000}%
\pgfsetstrokecolor{currentstroke}%
\pgfsetstrokeopacity{0.000000}%
\pgfsetdash{}{0pt}%
\pgfpathmoveto{\pgfqpoint{5.247761in}{0.613486in}}%
\pgfpathlineto{\pgfqpoint{5.257766in}{0.613486in}}%
\pgfpathlineto{\pgfqpoint{5.257766in}{1.612282in}}%
\pgfpathlineto{\pgfqpoint{5.247761in}{1.612282in}}%
\pgfpathlineto{\pgfqpoint{5.247761in}{0.613486in}}%
\pgfpathclose%
\pgfusepath{fill}%
\end{pgfscope}%
\begin{pgfscope}%
\pgfpathrectangle{\pgfqpoint{0.693757in}{0.613486in}}{\pgfqpoint{5.541243in}{3.963477in}}%
\pgfusepath{clip}%
\pgfsetbuttcap%
\pgfsetmiterjoin%
\definecolor{currentfill}{rgb}{0.000000,0.000000,1.000000}%
\pgfsetfillcolor{currentfill}%
\pgfsetlinewidth{0.000000pt}%
\definecolor{currentstroke}{rgb}{0.000000,0.000000,0.000000}%
\pgfsetstrokecolor{currentstroke}%
\pgfsetstrokeopacity{0.000000}%
\pgfsetdash{}{0pt}%
\pgfpathmoveto{\pgfqpoint{5.260268in}{0.613486in}}%
\pgfpathlineto{\pgfqpoint{5.270272in}{0.613486in}}%
\pgfpathlineto{\pgfqpoint{5.270272in}{2.595224in}}%
\pgfpathlineto{\pgfqpoint{5.260268in}{2.595224in}}%
\pgfpathlineto{\pgfqpoint{5.260268in}{0.613486in}}%
\pgfpathclose%
\pgfusepath{fill}%
\end{pgfscope}%
\begin{pgfscope}%
\pgfpathrectangle{\pgfqpoint{0.693757in}{0.613486in}}{\pgfqpoint{5.541243in}{3.963477in}}%
\pgfusepath{clip}%
\pgfsetbuttcap%
\pgfsetmiterjoin%
\definecolor{currentfill}{rgb}{0.000000,0.000000,1.000000}%
\pgfsetfillcolor{currentfill}%
\pgfsetlinewidth{0.000000pt}%
\definecolor{currentstroke}{rgb}{0.000000,0.000000,0.000000}%
\pgfsetstrokecolor{currentstroke}%
\pgfsetstrokeopacity{0.000000}%
\pgfsetdash{}{0pt}%
\pgfpathmoveto{\pgfqpoint{5.272774in}{0.613486in}}%
\pgfpathlineto{\pgfqpoint{5.282779in}{0.613486in}}%
\pgfpathlineto{\pgfqpoint{5.282779in}{2.611074in}}%
\pgfpathlineto{\pgfqpoint{5.272774in}{2.611074in}}%
\pgfpathlineto{\pgfqpoint{5.272774in}{0.613486in}}%
\pgfpathclose%
\pgfusepath{fill}%
\end{pgfscope}%
\begin{pgfscope}%
\pgfpathrectangle{\pgfqpoint{0.693757in}{0.613486in}}{\pgfqpoint{5.541243in}{3.963477in}}%
\pgfusepath{clip}%
\pgfsetbuttcap%
\pgfsetmiterjoin%
\definecolor{currentfill}{rgb}{0.000000,0.000000,1.000000}%
\pgfsetfillcolor{currentfill}%
\pgfsetlinewidth{0.000000pt}%
\definecolor{currentstroke}{rgb}{0.000000,0.000000,0.000000}%
\pgfsetstrokecolor{currentstroke}%
\pgfsetstrokeopacity{0.000000}%
\pgfsetdash{}{0pt}%
\pgfpathmoveto{\pgfqpoint{5.285280in}{0.613486in}}%
\pgfpathlineto{\pgfqpoint{5.295285in}{0.613486in}}%
\pgfpathlineto{\pgfqpoint{5.295285in}{1.604355in}}%
\pgfpathlineto{\pgfqpoint{5.285280in}{1.604355in}}%
\pgfpathlineto{\pgfqpoint{5.285280in}{0.613486in}}%
\pgfpathclose%
\pgfusepath{fill}%
\end{pgfscope}%
\begin{pgfscope}%
\pgfpathrectangle{\pgfqpoint{0.693757in}{0.613486in}}{\pgfqpoint{5.541243in}{3.963477in}}%
\pgfusepath{clip}%
\pgfsetbuttcap%
\pgfsetmiterjoin%
\definecolor{currentfill}{rgb}{0.000000,0.000000,1.000000}%
\pgfsetfillcolor{currentfill}%
\pgfsetlinewidth{0.000000pt}%
\definecolor{currentstroke}{rgb}{0.000000,0.000000,0.000000}%
\pgfsetstrokecolor{currentstroke}%
\pgfsetstrokeopacity{0.000000}%
\pgfsetdash{}{0pt}%
\pgfpathmoveto{\pgfqpoint{5.297786in}{0.613486in}}%
\pgfpathlineto{\pgfqpoint{5.307791in}{0.613486in}}%
\pgfpathlineto{\pgfqpoint{5.307791in}{1.612282in}}%
\pgfpathlineto{\pgfqpoint{5.297786in}{1.612282in}}%
\pgfpathlineto{\pgfqpoint{5.297786in}{0.613486in}}%
\pgfpathclose%
\pgfusepath{fill}%
\end{pgfscope}%
\begin{pgfscope}%
\pgfpathrectangle{\pgfqpoint{0.693757in}{0.613486in}}{\pgfqpoint{5.541243in}{3.963477in}}%
\pgfusepath{clip}%
\pgfsetbuttcap%
\pgfsetmiterjoin%
\definecolor{currentfill}{rgb}{0.000000,0.000000,1.000000}%
\pgfsetfillcolor{currentfill}%
\pgfsetlinewidth{0.000000pt}%
\definecolor{currentstroke}{rgb}{0.000000,0.000000,0.000000}%
\pgfsetstrokecolor{currentstroke}%
\pgfsetstrokeopacity{0.000000}%
\pgfsetdash{}{0pt}%
\pgfpathmoveto{\pgfqpoint{5.310292in}{0.613486in}}%
\pgfpathlineto{\pgfqpoint{5.320297in}{0.613486in}}%
\pgfpathlineto{\pgfqpoint{5.320297in}{2.595224in}}%
\pgfpathlineto{\pgfqpoint{5.310292in}{2.595224in}}%
\pgfpathlineto{\pgfqpoint{5.310292in}{0.613486in}}%
\pgfpathclose%
\pgfusepath{fill}%
\end{pgfscope}%
\begin{pgfscope}%
\pgfpathrectangle{\pgfqpoint{0.693757in}{0.613486in}}{\pgfqpoint{5.541243in}{3.963477in}}%
\pgfusepath{clip}%
\pgfsetbuttcap%
\pgfsetmiterjoin%
\definecolor{currentfill}{rgb}{0.000000,0.000000,1.000000}%
\pgfsetfillcolor{currentfill}%
\pgfsetlinewidth{0.000000pt}%
\definecolor{currentstroke}{rgb}{0.000000,0.000000,0.000000}%
\pgfsetstrokecolor{currentstroke}%
\pgfsetstrokeopacity{0.000000}%
\pgfsetdash{}{0pt}%
\pgfpathmoveto{\pgfqpoint{5.322798in}{0.613486in}}%
\pgfpathlineto{\pgfqpoint{5.332803in}{0.613486in}}%
\pgfpathlineto{\pgfqpoint{5.332803in}{2.611074in}}%
\pgfpathlineto{\pgfqpoint{5.322798in}{2.611074in}}%
\pgfpathlineto{\pgfqpoint{5.322798in}{0.613486in}}%
\pgfpathclose%
\pgfusepath{fill}%
\end{pgfscope}%
\begin{pgfscope}%
\pgfpathrectangle{\pgfqpoint{0.693757in}{0.613486in}}{\pgfqpoint{5.541243in}{3.963477in}}%
\pgfusepath{clip}%
\pgfsetbuttcap%
\pgfsetmiterjoin%
\definecolor{currentfill}{rgb}{0.000000,0.000000,1.000000}%
\pgfsetfillcolor{currentfill}%
\pgfsetlinewidth{0.000000pt}%
\definecolor{currentstroke}{rgb}{0.000000,0.000000,0.000000}%
\pgfsetstrokecolor{currentstroke}%
\pgfsetstrokeopacity{0.000000}%
\pgfsetdash{}{0pt}%
\pgfpathmoveto{\pgfqpoint{5.335305in}{0.613486in}}%
\pgfpathlineto{\pgfqpoint{5.345310in}{0.613486in}}%
\pgfpathlineto{\pgfqpoint{5.345310in}{1.604355in}}%
\pgfpathlineto{\pgfqpoint{5.335305in}{1.604355in}}%
\pgfpathlineto{\pgfqpoint{5.335305in}{0.613486in}}%
\pgfpathclose%
\pgfusepath{fill}%
\end{pgfscope}%
\begin{pgfscope}%
\pgfpathrectangle{\pgfqpoint{0.693757in}{0.613486in}}{\pgfqpoint{5.541243in}{3.963477in}}%
\pgfusepath{clip}%
\pgfsetbuttcap%
\pgfsetmiterjoin%
\definecolor{currentfill}{rgb}{0.000000,0.000000,1.000000}%
\pgfsetfillcolor{currentfill}%
\pgfsetlinewidth{0.000000pt}%
\definecolor{currentstroke}{rgb}{0.000000,0.000000,0.000000}%
\pgfsetstrokecolor{currentstroke}%
\pgfsetstrokeopacity{0.000000}%
\pgfsetdash{}{0pt}%
\pgfpathmoveto{\pgfqpoint{5.347811in}{0.613486in}}%
\pgfpathlineto{\pgfqpoint{5.357816in}{0.613486in}}%
\pgfpathlineto{\pgfqpoint{5.357816in}{1.612282in}}%
\pgfpathlineto{\pgfqpoint{5.347811in}{1.612282in}}%
\pgfpathlineto{\pgfqpoint{5.347811in}{0.613486in}}%
\pgfpathclose%
\pgfusepath{fill}%
\end{pgfscope}%
\begin{pgfscope}%
\pgfpathrectangle{\pgfqpoint{0.693757in}{0.613486in}}{\pgfqpoint{5.541243in}{3.963477in}}%
\pgfusepath{clip}%
\pgfsetbuttcap%
\pgfsetmiterjoin%
\definecolor{currentfill}{rgb}{0.000000,0.000000,1.000000}%
\pgfsetfillcolor{currentfill}%
\pgfsetlinewidth{0.000000pt}%
\definecolor{currentstroke}{rgb}{0.000000,0.000000,0.000000}%
\pgfsetstrokecolor{currentstroke}%
\pgfsetstrokeopacity{0.000000}%
\pgfsetdash{}{0pt}%
\pgfpathmoveto{\pgfqpoint{5.360317in}{0.613486in}}%
\pgfpathlineto{\pgfqpoint{5.370322in}{0.613486in}}%
\pgfpathlineto{\pgfqpoint{5.370322in}{2.595224in}}%
\pgfpathlineto{\pgfqpoint{5.360317in}{2.595224in}}%
\pgfpathlineto{\pgfqpoint{5.360317in}{0.613486in}}%
\pgfpathclose%
\pgfusepath{fill}%
\end{pgfscope}%
\begin{pgfscope}%
\pgfpathrectangle{\pgfqpoint{0.693757in}{0.613486in}}{\pgfqpoint{5.541243in}{3.963477in}}%
\pgfusepath{clip}%
\pgfsetbuttcap%
\pgfsetmiterjoin%
\definecolor{currentfill}{rgb}{0.000000,0.000000,1.000000}%
\pgfsetfillcolor{currentfill}%
\pgfsetlinewidth{0.000000pt}%
\definecolor{currentstroke}{rgb}{0.000000,0.000000,0.000000}%
\pgfsetstrokecolor{currentstroke}%
\pgfsetstrokeopacity{0.000000}%
\pgfsetdash{}{0pt}%
\pgfpathmoveto{\pgfqpoint{5.372823in}{0.613486in}}%
\pgfpathlineto{\pgfqpoint{5.382828in}{0.613486in}}%
\pgfpathlineto{\pgfqpoint{5.382828in}{2.611074in}}%
\pgfpathlineto{\pgfqpoint{5.372823in}{2.611074in}}%
\pgfpathlineto{\pgfqpoint{5.372823in}{0.613486in}}%
\pgfpathclose%
\pgfusepath{fill}%
\end{pgfscope}%
\begin{pgfscope}%
\pgfpathrectangle{\pgfqpoint{0.693757in}{0.613486in}}{\pgfqpoint{5.541243in}{3.963477in}}%
\pgfusepath{clip}%
\pgfsetbuttcap%
\pgfsetmiterjoin%
\definecolor{currentfill}{rgb}{0.000000,0.000000,1.000000}%
\pgfsetfillcolor{currentfill}%
\pgfsetlinewidth{0.000000pt}%
\definecolor{currentstroke}{rgb}{0.000000,0.000000,0.000000}%
\pgfsetstrokecolor{currentstroke}%
\pgfsetstrokeopacity{0.000000}%
\pgfsetdash{}{0pt}%
\pgfpathmoveto{\pgfqpoint{5.385329in}{0.613486in}}%
\pgfpathlineto{\pgfqpoint{5.395334in}{0.613486in}}%
\pgfpathlineto{\pgfqpoint{5.395334in}{1.604355in}}%
\pgfpathlineto{\pgfqpoint{5.385329in}{1.604355in}}%
\pgfpathlineto{\pgfqpoint{5.385329in}{0.613486in}}%
\pgfpathclose%
\pgfusepath{fill}%
\end{pgfscope}%
\begin{pgfscope}%
\pgfpathrectangle{\pgfqpoint{0.693757in}{0.613486in}}{\pgfqpoint{5.541243in}{3.963477in}}%
\pgfusepath{clip}%
\pgfsetbuttcap%
\pgfsetmiterjoin%
\definecolor{currentfill}{rgb}{0.000000,0.000000,1.000000}%
\pgfsetfillcolor{currentfill}%
\pgfsetlinewidth{0.000000pt}%
\definecolor{currentstroke}{rgb}{0.000000,0.000000,0.000000}%
\pgfsetstrokecolor{currentstroke}%
\pgfsetstrokeopacity{0.000000}%
\pgfsetdash{}{0pt}%
\pgfpathmoveto{\pgfqpoint{5.397836in}{0.613486in}}%
\pgfpathlineto{\pgfqpoint{5.407841in}{0.613486in}}%
\pgfpathlineto{\pgfqpoint{5.407841in}{1.612282in}}%
\pgfpathlineto{\pgfqpoint{5.397836in}{1.612282in}}%
\pgfpathlineto{\pgfqpoint{5.397836in}{0.613486in}}%
\pgfpathclose%
\pgfusepath{fill}%
\end{pgfscope}%
\begin{pgfscope}%
\pgfpathrectangle{\pgfqpoint{0.693757in}{0.613486in}}{\pgfqpoint{5.541243in}{3.963477in}}%
\pgfusepath{clip}%
\pgfsetbuttcap%
\pgfsetmiterjoin%
\definecolor{currentfill}{rgb}{0.000000,0.000000,1.000000}%
\pgfsetfillcolor{currentfill}%
\pgfsetlinewidth{0.000000pt}%
\definecolor{currentstroke}{rgb}{0.000000,0.000000,0.000000}%
\pgfsetstrokecolor{currentstroke}%
\pgfsetstrokeopacity{0.000000}%
\pgfsetdash{}{0pt}%
\pgfpathmoveto{\pgfqpoint{5.410342in}{0.613486in}}%
\pgfpathlineto{\pgfqpoint{5.420347in}{0.613486in}}%
\pgfpathlineto{\pgfqpoint{5.420347in}{2.595224in}}%
\pgfpathlineto{\pgfqpoint{5.410342in}{2.595224in}}%
\pgfpathlineto{\pgfqpoint{5.410342in}{0.613486in}}%
\pgfpathclose%
\pgfusepath{fill}%
\end{pgfscope}%
\begin{pgfscope}%
\pgfpathrectangle{\pgfqpoint{0.693757in}{0.613486in}}{\pgfqpoint{5.541243in}{3.963477in}}%
\pgfusepath{clip}%
\pgfsetbuttcap%
\pgfsetmiterjoin%
\definecolor{currentfill}{rgb}{0.000000,0.000000,1.000000}%
\pgfsetfillcolor{currentfill}%
\pgfsetlinewidth{0.000000pt}%
\definecolor{currentstroke}{rgb}{0.000000,0.000000,0.000000}%
\pgfsetstrokecolor{currentstroke}%
\pgfsetstrokeopacity{0.000000}%
\pgfsetdash{}{0pt}%
\pgfpathmoveto{\pgfqpoint{5.422848in}{0.613486in}}%
\pgfpathlineto{\pgfqpoint{5.432853in}{0.613486in}}%
\pgfpathlineto{\pgfqpoint{5.432853in}{2.611074in}}%
\pgfpathlineto{\pgfqpoint{5.422848in}{2.611074in}}%
\pgfpathlineto{\pgfqpoint{5.422848in}{0.613486in}}%
\pgfpathclose%
\pgfusepath{fill}%
\end{pgfscope}%
\begin{pgfscope}%
\pgfpathrectangle{\pgfqpoint{0.693757in}{0.613486in}}{\pgfqpoint{5.541243in}{3.963477in}}%
\pgfusepath{clip}%
\pgfsetbuttcap%
\pgfsetmiterjoin%
\definecolor{currentfill}{rgb}{0.000000,0.000000,1.000000}%
\pgfsetfillcolor{currentfill}%
\pgfsetlinewidth{0.000000pt}%
\definecolor{currentstroke}{rgb}{0.000000,0.000000,0.000000}%
\pgfsetstrokecolor{currentstroke}%
\pgfsetstrokeopacity{0.000000}%
\pgfsetdash{}{0pt}%
\pgfpathmoveto{\pgfqpoint{5.435354in}{0.613486in}}%
\pgfpathlineto{\pgfqpoint{5.445359in}{0.613486in}}%
\pgfpathlineto{\pgfqpoint{5.445359in}{1.604355in}}%
\pgfpathlineto{\pgfqpoint{5.435354in}{1.604355in}}%
\pgfpathlineto{\pgfqpoint{5.435354in}{0.613486in}}%
\pgfpathclose%
\pgfusepath{fill}%
\end{pgfscope}%
\begin{pgfscope}%
\pgfpathrectangle{\pgfqpoint{0.693757in}{0.613486in}}{\pgfqpoint{5.541243in}{3.963477in}}%
\pgfusepath{clip}%
\pgfsetbuttcap%
\pgfsetmiterjoin%
\definecolor{currentfill}{rgb}{0.000000,0.000000,1.000000}%
\pgfsetfillcolor{currentfill}%
\pgfsetlinewidth{0.000000pt}%
\definecolor{currentstroke}{rgb}{0.000000,0.000000,0.000000}%
\pgfsetstrokecolor{currentstroke}%
\pgfsetstrokeopacity{0.000000}%
\pgfsetdash{}{0pt}%
\pgfpathmoveto{\pgfqpoint{5.447860in}{0.613486in}}%
\pgfpathlineto{\pgfqpoint{5.457865in}{0.613486in}}%
\pgfpathlineto{\pgfqpoint{5.457865in}{1.612282in}}%
\pgfpathlineto{\pgfqpoint{5.447860in}{1.612282in}}%
\pgfpathlineto{\pgfqpoint{5.447860in}{0.613486in}}%
\pgfpathclose%
\pgfusepath{fill}%
\end{pgfscope}%
\begin{pgfscope}%
\pgfpathrectangle{\pgfqpoint{0.693757in}{0.613486in}}{\pgfqpoint{5.541243in}{3.963477in}}%
\pgfusepath{clip}%
\pgfsetbuttcap%
\pgfsetmiterjoin%
\definecolor{currentfill}{rgb}{0.000000,0.000000,1.000000}%
\pgfsetfillcolor{currentfill}%
\pgfsetlinewidth{0.000000pt}%
\definecolor{currentstroke}{rgb}{0.000000,0.000000,0.000000}%
\pgfsetstrokecolor{currentstroke}%
\pgfsetstrokeopacity{0.000000}%
\pgfsetdash{}{0pt}%
\pgfpathmoveto{\pgfqpoint{5.460367in}{0.613486in}}%
\pgfpathlineto{\pgfqpoint{5.470372in}{0.613486in}}%
\pgfpathlineto{\pgfqpoint{5.470372in}{2.595224in}}%
\pgfpathlineto{\pgfqpoint{5.460367in}{2.595224in}}%
\pgfpathlineto{\pgfqpoint{5.460367in}{0.613486in}}%
\pgfpathclose%
\pgfusepath{fill}%
\end{pgfscope}%
\begin{pgfscope}%
\pgfpathrectangle{\pgfqpoint{0.693757in}{0.613486in}}{\pgfqpoint{5.541243in}{3.963477in}}%
\pgfusepath{clip}%
\pgfsetbuttcap%
\pgfsetmiterjoin%
\definecolor{currentfill}{rgb}{0.000000,0.000000,1.000000}%
\pgfsetfillcolor{currentfill}%
\pgfsetlinewidth{0.000000pt}%
\definecolor{currentstroke}{rgb}{0.000000,0.000000,0.000000}%
\pgfsetstrokecolor{currentstroke}%
\pgfsetstrokeopacity{0.000000}%
\pgfsetdash{}{0pt}%
\pgfpathmoveto{\pgfqpoint{5.472873in}{0.613486in}}%
\pgfpathlineto{\pgfqpoint{5.482878in}{0.613486in}}%
\pgfpathlineto{\pgfqpoint{5.482878in}{2.611074in}}%
\pgfpathlineto{\pgfqpoint{5.472873in}{2.611074in}}%
\pgfpathlineto{\pgfqpoint{5.472873in}{0.613486in}}%
\pgfpathclose%
\pgfusepath{fill}%
\end{pgfscope}%
\begin{pgfscope}%
\pgfpathrectangle{\pgfqpoint{0.693757in}{0.613486in}}{\pgfqpoint{5.541243in}{3.963477in}}%
\pgfusepath{clip}%
\pgfsetbuttcap%
\pgfsetmiterjoin%
\definecolor{currentfill}{rgb}{0.000000,0.000000,1.000000}%
\pgfsetfillcolor{currentfill}%
\pgfsetlinewidth{0.000000pt}%
\definecolor{currentstroke}{rgb}{0.000000,0.000000,0.000000}%
\pgfsetstrokecolor{currentstroke}%
\pgfsetstrokeopacity{0.000000}%
\pgfsetdash{}{0pt}%
\pgfpathmoveto{\pgfqpoint{5.485379in}{0.613486in}}%
\pgfpathlineto{\pgfqpoint{5.495384in}{0.613486in}}%
\pgfpathlineto{\pgfqpoint{5.495384in}{1.604355in}}%
\pgfpathlineto{\pgfqpoint{5.485379in}{1.604355in}}%
\pgfpathlineto{\pgfqpoint{5.485379in}{0.613486in}}%
\pgfpathclose%
\pgfusepath{fill}%
\end{pgfscope}%
\begin{pgfscope}%
\pgfpathrectangle{\pgfqpoint{0.693757in}{0.613486in}}{\pgfqpoint{5.541243in}{3.963477in}}%
\pgfusepath{clip}%
\pgfsetbuttcap%
\pgfsetmiterjoin%
\definecolor{currentfill}{rgb}{0.000000,0.000000,1.000000}%
\pgfsetfillcolor{currentfill}%
\pgfsetlinewidth{0.000000pt}%
\definecolor{currentstroke}{rgb}{0.000000,0.000000,0.000000}%
\pgfsetstrokecolor{currentstroke}%
\pgfsetstrokeopacity{0.000000}%
\pgfsetdash{}{0pt}%
\pgfpathmoveto{\pgfqpoint{5.497885in}{0.613486in}}%
\pgfpathlineto{\pgfqpoint{5.507890in}{0.613486in}}%
\pgfpathlineto{\pgfqpoint{5.507890in}{1.612282in}}%
\pgfpathlineto{\pgfqpoint{5.497885in}{1.612282in}}%
\pgfpathlineto{\pgfqpoint{5.497885in}{0.613486in}}%
\pgfpathclose%
\pgfusepath{fill}%
\end{pgfscope}%
\begin{pgfscope}%
\pgfpathrectangle{\pgfqpoint{0.693757in}{0.613486in}}{\pgfqpoint{5.541243in}{3.963477in}}%
\pgfusepath{clip}%
\pgfsetbuttcap%
\pgfsetmiterjoin%
\definecolor{currentfill}{rgb}{0.000000,0.000000,1.000000}%
\pgfsetfillcolor{currentfill}%
\pgfsetlinewidth{0.000000pt}%
\definecolor{currentstroke}{rgb}{0.000000,0.000000,0.000000}%
\pgfsetstrokecolor{currentstroke}%
\pgfsetstrokeopacity{0.000000}%
\pgfsetdash{}{0pt}%
\pgfpathmoveto{\pgfqpoint{5.510391in}{0.613486in}}%
\pgfpathlineto{\pgfqpoint{5.520396in}{0.613486in}}%
\pgfpathlineto{\pgfqpoint{5.520396in}{2.595224in}}%
\pgfpathlineto{\pgfqpoint{5.510391in}{2.595224in}}%
\pgfpathlineto{\pgfqpoint{5.510391in}{0.613486in}}%
\pgfpathclose%
\pgfusepath{fill}%
\end{pgfscope}%
\begin{pgfscope}%
\pgfpathrectangle{\pgfqpoint{0.693757in}{0.613486in}}{\pgfqpoint{5.541243in}{3.963477in}}%
\pgfusepath{clip}%
\pgfsetbuttcap%
\pgfsetmiterjoin%
\definecolor{currentfill}{rgb}{0.000000,0.000000,1.000000}%
\pgfsetfillcolor{currentfill}%
\pgfsetlinewidth{0.000000pt}%
\definecolor{currentstroke}{rgb}{0.000000,0.000000,0.000000}%
\pgfsetstrokecolor{currentstroke}%
\pgfsetstrokeopacity{0.000000}%
\pgfsetdash{}{0pt}%
\pgfpathmoveto{\pgfqpoint{5.522898in}{0.613486in}}%
\pgfpathlineto{\pgfqpoint{5.532902in}{0.613486in}}%
\pgfpathlineto{\pgfqpoint{5.532902in}{2.611074in}}%
\pgfpathlineto{\pgfqpoint{5.522898in}{2.611074in}}%
\pgfpathlineto{\pgfqpoint{5.522898in}{0.613486in}}%
\pgfpathclose%
\pgfusepath{fill}%
\end{pgfscope}%
\begin{pgfscope}%
\pgfpathrectangle{\pgfqpoint{0.693757in}{0.613486in}}{\pgfqpoint{5.541243in}{3.963477in}}%
\pgfusepath{clip}%
\pgfsetbuttcap%
\pgfsetmiterjoin%
\definecolor{currentfill}{rgb}{0.000000,0.000000,1.000000}%
\pgfsetfillcolor{currentfill}%
\pgfsetlinewidth{0.000000pt}%
\definecolor{currentstroke}{rgb}{0.000000,0.000000,0.000000}%
\pgfsetstrokecolor{currentstroke}%
\pgfsetstrokeopacity{0.000000}%
\pgfsetdash{}{0pt}%
\pgfpathmoveto{\pgfqpoint{5.535404in}{0.613486in}}%
\pgfpathlineto{\pgfqpoint{5.545409in}{0.613486in}}%
\pgfpathlineto{\pgfqpoint{5.545409in}{1.604355in}}%
\pgfpathlineto{\pgfqpoint{5.535404in}{1.604355in}}%
\pgfpathlineto{\pgfqpoint{5.535404in}{0.613486in}}%
\pgfpathclose%
\pgfusepath{fill}%
\end{pgfscope}%
\begin{pgfscope}%
\pgfpathrectangle{\pgfqpoint{0.693757in}{0.613486in}}{\pgfqpoint{5.541243in}{3.963477in}}%
\pgfusepath{clip}%
\pgfsetbuttcap%
\pgfsetmiterjoin%
\definecolor{currentfill}{rgb}{0.000000,0.000000,1.000000}%
\pgfsetfillcolor{currentfill}%
\pgfsetlinewidth{0.000000pt}%
\definecolor{currentstroke}{rgb}{0.000000,0.000000,0.000000}%
\pgfsetstrokecolor{currentstroke}%
\pgfsetstrokeopacity{0.000000}%
\pgfsetdash{}{0pt}%
\pgfpathmoveto{\pgfqpoint{5.547910in}{0.613486in}}%
\pgfpathlineto{\pgfqpoint{5.557915in}{0.613486in}}%
\pgfpathlineto{\pgfqpoint{5.557915in}{1.612282in}}%
\pgfpathlineto{\pgfqpoint{5.547910in}{1.612282in}}%
\pgfpathlineto{\pgfqpoint{5.547910in}{0.613486in}}%
\pgfpathclose%
\pgfusepath{fill}%
\end{pgfscope}%
\begin{pgfscope}%
\pgfpathrectangle{\pgfqpoint{0.693757in}{0.613486in}}{\pgfqpoint{5.541243in}{3.963477in}}%
\pgfusepath{clip}%
\pgfsetbuttcap%
\pgfsetmiterjoin%
\definecolor{currentfill}{rgb}{0.000000,0.000000,1.000000}%
\pgfsetfillcolor{currentfill}%
\pgfsetlinewidth{0.000000pt}%
\definecolor{currentstroke}{rgb}{0.000000,0.000000,0.000000}%
\pgfsetstrokecolor{currentstroke}%
\pgfsetstrokeopacity{0.000000}%
\pgfsetdash{}{0pt}%
\pgfpathmoveto{\pgfqpoint{5.560416in}{0.613486in}}%
\pgfpathlineto{\pgfqpoint{5.570421in}{0.613486in}}%
\pgfpathlineto{\pgfqpoint{5.570421in}{2.595224in}}%
\pgfpathlineto{\pgfqpoint{5.560416in}{2.595224in}}%
\pgfpathlineto{\pgfqpoint{5.560416in}{0.613486in}}%
\pgfpathclose%
\pgfusepath{fill}%
\end{pgfscope}%
\begin{pgfscope}%
\pgfpathrectangle{\pgfqpoint{0.693757in}{0.613486in}}{\pgfqpoint{5.541243in}{3.963477in}}%
\pgfusepath{clip}%
\pgfsetbuttcap%
\pgfsetmiterjoin%
\definecolor{currentfill}{rgb}{0.000000,0.000000,1.000000}%
\pgfsetfillcolor{currentfill}%
\pgfsetlinewidth{0.000000pt}%
\definecolor{currentstroke}{rgb}{0.000000,0.000000,0.000000}%
\pgfsetstrokecolor{currentstroke}%
\pgfsetstrokeopacity{0.000000}%
\pgfsetdash{}{0pt}%
\pgfpathmoveto{\pgfqpoint{5.572922in}{0.613486in}}%
\pgfpathlineto{\pgfqpoint{5.582927in}{0.613486in}}%
\pgfpathlineto{\pgfqpoint{5.582927in}{2.611074in}}%
\pgfpathlineto{\pgfqpoint{5.572922in}{2.611074in}}%
\pgfpathlineto{\pgfqpoint{5.572922in}{0.613486in}}%
\pgfpathclose%
\pgfusepath{fill}%
\end{pgfscope}%
\begin{pgfscope}%
\pgfpathrectangle{\pgfqpoint{0.693757in}{0.613486in}}{\pgfqpoint{5.541243in}{3.963477in}}%
\pgfusepath{clip}%
\pgfsetbuttcap%
\pgfsetmiterjoin%
\definecolor{currentfill}{rgb}{0.000000,0.000000,1.000000}%
\pgfsetfillcolor{currentfill}%
\pgfsetlinewidth{0.000000pt}%
\definecolor{currentstroke}{rgb}{0.000000,0.000000,0.000000}%
\pgfsetstrokecolor{currentstroke}%
\pgfsetstrokeopacity{0.000000}%
\pgfsetdash{}{0pt}%
\pgfpathmoveto{\pgfqpoint{5.585428in}{0.613486in}}%
\pgfpathlineto{\pgfqpoint{5.595433in}{0.613486in}}%
\pgfpathlineto{\pgfqpoint{5.595433in}{1.604355in}}%
\pgfpathlineto{\pgfqpoint{5.585428in}{1.604355in}}%
\pgfpathlineto{\pgfqpoint{5.585428in}{0.613486in}}%
\pgfpathclose%
\pgfusepath{fill}%
\end{pgfscope}%
\begin{pgfscope}%
\pgfpathrectangle{\pgfqpoint{0.693757in}{0.613486in}}{\pgfqpoint{5.541243in}{3.963477in}}%
\pgfusepath{clip}%
\pgfsetbuttcap%
\pgfsetmiterjoin%
\definecolor{currentfill}{rgb}{0.000000,0.000000,1.000000}%
\pgfsetfillcolor{currentfill}%
\pgfsetlinewidth{0.000000pt}%
\definecolor{currentstroke}{rgb}{0.000000,0.000000,0.000000}%
\pgfsetstrokecolor{currentstroke}%
\pgfsetstrokeopacity{0.000000}%
\pgfsetdash{}{0pt}%
\pgfpathmoveto{\pgfqpoint{5.597935in}{0.613486in}}%
\pgfpathlineto{\pgfqpoint{5.607940in}{0.613486in}}%
\pgfpathlineto{\pgfqpoint{5.607940in}{1.612282in}}%
\pgfpathlineto{\pgfqpoint{5.597935in}{1.612282in}}%
\pgfpathlineto{\pgfqpoint{5.597935in}{0.613486in}}%
\pgfpathclose%
\pgfusepath{fill}%
\end{pgfscope}%
\begin{pgfscope}%
\pgfpathrectangle{\pgfqpoint{0.693757in}{0.613486in}}{\pgfqpoint{5.541243in}{3.963477in}}%
\pgfusepath{clip}%
\pgfsetbuttcap%
\pgfsetmiterjoin%
\definecolor{currentfill}{rgb}{0.000000,0.000000,1.000000}%
\pgfsetfillcolor{currentfill}%
\pgfsetlinewidth{0.000000pt}%
\definecolor{currentstroke}{rgb}{0.000000,0.000000,0.000000}%
\pgfsetstrokecolor{currentstroke}%
\pgfsetstrokeopacity{0.000000}%
\pgfsetdash{}{0pt}%
\pgfpathmoveto{\pgfqpoint{5.610441in}{0.613486in}}%
\pgfpathlineto{\pgfqpoint{5.620446in}{0.613486in}}%
\pgfpathlineto{\pgfqpoint{5.620446in}{2.595224in}}%
\pgfpathlineto{\pgfqpoint{5.610441in}{2.595224in}}%
\pgfpathlineto{\pgfqpoint{5.610441in}{0.613486in}}%
\pgfpathclose%
\pgfusepath{fill}%
\end{pgfscope}%
\begin{pgfscope}%
\pgfpathrectangle{\pgfqpoint{0.693757in}{0.613486in}}{\pgfqpoint{5.541243in}{3.963477in}}%
\pgfusepath{clip}%
\pgfsetbuttcap%
\pgfsetmiterjoin%
\definecolor{currentfill}{rgb}{0.000000,0.000000,1.000000}%
\pgfsetfillcolor{currentfill}%
\pgfsetlinewidth{0.000000pt}%
\definecolor{currentstroke}{rgb}{0.000000,0.000000,0.000000}%
\pgfsetstrokecolor{currentstroke}%
\pgfsetstrokeopacity{0.000000}%
\pgfsetdash{}{0pt}%
\pgfpathmoveto{\pgfqpoint{5.622947in}{0.613486in}}%
\pgfpathlineto{\pgfqpoint{5.632952in}{0.613486in}}%
\pgfpathlineto{\pgfqpoint{5.632952in}{2.611074in}}%
\pgfpathlineto{\pgfqpoint{5.622947in}{2.611074in}}%
\pgfpathlineto{\pgfqpoint{5.622947in}{0.613486in}}%
\pgfpathclose%
\pgfusepath{fill}%
\end{pgfscope}%
\begin{pgfscope}%
\pgfpathrectangle{\pgfqpoint{0.693757in}{0.613486in}}{\pgfqpoint{5.541243in}{3.963477in}}%
\pgfusepath{clip}%
\pgfsetbuttcap%
\pgfsetmiterjoin%
\definecolor{currentfill}{rgb}{0.000000,0.000000,1.000000}%
\pgfsetfillcolor{currentfill}%
\pgfsetlinewidth{0.000000pt}%
\definecolor{currentstroke}{rgb}{0.000000,0.000000,0.000000}%
\pgfsetstrokecolor{currentstroke}%
\pgfsetstrokeopacity{0.000000}%
\pgfsetdash{}{0pt}%
\pgfpathmoveto{\pgfqpoint{5.635453in}{0.613486in}}%
\pgfpathlineto{\pgfqpoint{5.645458in}{0.613486in}}%
\pgfpathlineto{\pgfqpoint{5.645458in}{1.604355in}}%
\pgfpathlineto{\pgfqpoint{5.635453in}{1.604355in}}%
\pgfpathlineto{\pgfqpoint{5.635453in}{0.613486in}}%
\pgfpathclose%
\pgfusepath{fill}%
\end{pgfscope}%
\begin{pgfscope}%
\pgfpathrectangle{\pgfqpoint{0.693757in}{0.613486in}}{\pgfqpoint{5.541243in}{3.963477in}}%
\pgfusepath{clip}%
\pgfsetbuttcap%
\pgfsetmiterjoin%
\definecolor{currentfill}{rgb}{0.000000,0.000000,1.000000}%
\pgfsetfillcolor{currentfill}%
\pgfsetlinewidth{0.000000pt}%
\definecolor{currentstroke}{rgb}{0.000000,0.000000,0.000000}%
\pgfsetstrokecolor{currentstroke}%
\pgfsetstrokeopacity{0.000000}%
\pgfsetdash{}{0pt}%
\pgfpathmoveto{\pgfqpoint{5.647959in}{0.613486in}}%
\pgfpathlineto{\pgfqpoint{5.657964in}{0.613486in}}%
\pgfpathlineto{\pgfqpoint{5.657964in}{1.612282in}}%
\pgfpathlineto{\pgfqpoint{5.647959in}{1.612282in}}%
\pgfpathlineto{\pgfqpoint{5.647959in}{0.613486in}}%
\pgfpathclose%
\pgfusepath{fill}%
\end{pgfscope}%
\begin{pgfscope}%
\pgfpathrectangle{\pgfqpoint{0.693757in}{0.613486in}}{\pgfqpoint{5.541243in}{3.963477in}}%
\pgfusepath{clip}%
\pgfsetbuttcap%
\pgfsetmiterjoin%
\definecolor{currentfill}{rgb}{0.000000,0.000000,1.000000}%
\pgfsetfillcolor{currentfill}%
\pgfsetlinewidth{0.000000pt}%
\definecolor{currentstroke}{rgb}{0.000000,0.000000,0.000000}%
\pgfsetstrokecolor{currentstroke}%
\pgfsetstrokeopacity{0.000000}%
\pgfsetdash{}{0pt}%
\pgfpathmoveto{\pgfqpoint{5.660466in}{0.613486in}}%
\pgfpathlineto{\pgfqpoint{5.670471in}{0.613486in}}%
\pgfpathlineto{\pgfqpoint{5.670471in}{2.595224in}}%
\pgfpathlineto{\pgfqpoint{5.660466in}{2.595224in}}%
\pgfpathlineto{\pgfqpoint{5.660466in}{0.613486in}}%
\pgfpathclose%
\pgfusepath{fill}%
\end{pgfscope}%
\begin{pgfscope}%
\pgfpathrectangle{\pgfqpoint{0.693757in}{0.613486in}}{\pgfqpoint{5.541243in}{3.963477in}}%
\pgfusepath{clip}%
\pgfsetbuttcap%
\pgfsetmiterjoin%
\definecolor{currentfill}{rgb}{0.000000,0.000000,1.000000}%
\pgfsetfillcolor{currentfill}%
\pgfsetlinewidth{0.000000pt}%
\definecolor{currentstroke}{rgb}{0.000000,0.000000,0.000000}%
\pgfsetstrokecolor{currentstroke}%
\pgfsetstrokeopacity{0.000000}%
\pgfsetdash{}{0pt}%
\pgfpathmoveto{\pgfqpoint{5.672972in}{0.613486in}}%
\pgfpathlineto{\pgfqpoint{5.682977in}{0.613486in}}%
\pgfpathlineto{\pgfqpoint{5.682977in}{2.611074in}}%
\pgfpathlineto{\pgfqpoint{5.672972in}{2.611074in}}%
\pgfpathlineto{\pgfqpoint{5.672972in}{0.613486in}}%
\pgfpathclose%
\pgfusepath{fill}%
\end{pgfscope}%
\begin{pgfscope}%
\pgfpathrectangle{\pgfqpoint{0.693757in}{0.613486in}}{\pgfqpoint{5.541243in}{3.963477in}}%
\pgfusepath{clip}%
\pgfsetbuttcap%
\pgfsetmiterjoin%
\definecolor{currentfill}{rgb}{0.000000,0.000000,1.000000}%
\pgfsetfillcolor{currentfill}%
\pgfsetlinewidth{0.000000pt}%
\definecolor{currentstroke}{rgb}{0.000000,0.000000,0.000000}%
\pgfsetstrokecolor{currentstroke}%
\pgfsetstrokeopacity{0.000000}%
\pgfsetdash{}{0pt}%
\pgfpathmoveto{\pgfqpoint{5.685478in}{0.613486in}}%
\pgfpathlineto{\pgfqpoint{5.695483in}{0.613486in}}%
\pgfpathlineto{\pgfqpoint{5.695483in}{1.604355in}}%
\pgfpathlineto{\pgfqpoint{5.685478in}{1.604355in}}%
\pgfpathlineto{\pgfqpoint{5.685478in}{0.613486in}}%
\pgfpathclose%
\pgfusepath{fill}%
\end{pgfscope}%
\begin{pgfscope}%
\pgfpathrectangle{\pgfqpoint{0.693757in}{0.613486in}}{\pgfqpoint{5.541243in}{3.963477in}}%
\pgfusepath{clip}%
\pgfsetbuttcap%
\pgfsetmiterjoin%
\definecolor{currentfill}{rgb}{0.000000,0.000000,1.000000}%
\pgfsetfillcolor{currentfill}%
\pgfsetlinewidth{0.000000pt}%
\definecolor{currentstroke}{rgb}{0.000000,0.000000,0.000000}%
\pgfsetstrokecolor{currentstroke}%
\pgfsetstrokeopacity{0.000000}%
\pgfsetdash{}{0pt}%
\pgfpathmoveto{\pgfqpoint{5.697984in}{0.613486in}}%
\pgfpathlineto{\pgfqpoint{5.707989in}{0.613486in}}%
\pgfpathlineto{\pgfqpoint{5.707989in}{1.612282in}}%
\pgfpathlineto{\pgfqpoint{5.697984in}{1.612282in}}%
\pgfpathlineto{\pgfqpoint{5.697984in}{0.613486in}}%
\pgfpathclose%
\pgfusepath{fill}%
\end{pgfscope}%
\begin{pgfscope}%
\pgfpathrectangle{\pgfqpoint{0.693757in}{0.613486in}}{\pgfqpoint{5.541243in}{3.963477in}}%
\pgfusepath{clip}%
\pgfsetbuttcap%
\pgfsetmiterjoin%
\definecolor{currentfill}{rgb}{0.000000,0.000000,1.000000}%
\pgfsetfillcolor{currentfill}%
\pgfsetlinewidth{0.000000pt}%
\definecolor{currentstroke}{rgb}{0.000000,0.000000,0.000000}%
\pgfsetstrokecolor{currentstroke}%
\pgfsetstrokeopacity{0.000000}%
\pgfsetdash{}{0pt}%
\pgfpathmoveto{\pgfqpoint{5.710490in}{0.613486in}}%
\pgfpathlineto{\pgfqpoint{5.720495in}{0.613486in}}%
\pgfpathlineto{\pgfqpoint{5.720495in}{2.595224in}}%
\pgfpathlineto{\pgfqpoint{5.710490in}{2.595224in}}%
\pgfpathlineto{\pgfqpoint{5.710490in}{0.613486in}}%
\pgfpathclose%
\pgfusepath{fill}%
\end{pgfscope}%
\begin{pgfscope}%
\pgfpathrectangle{\pgfqpoint{0.693757in}{0.613486in}}{\pgfqpoint{5.541243in}{3.963477in}}%
\pgfusepath{clip}%
\pgfsetbuttcap%
\pgfsetmiterjoin%
\definecolor{currentfill}{rgb}{0.000000,0.000000,1.000000}%
\pgfsetfillcolor{currentfill}%
\pgfsetlinewidth{0.000000pt}%
\definecolor{currentstroke}{rgb}{0.000000,0.000000,0.000000}%
\pgfsetstrokecolor{currentstroke}%
\pgfsetstrokeopacity{0.000000}%
\pgfsetdash{}{0pt}%
\pgfpathmoveto{\pgfqpoint{5.722997in}{0.613486in}}%
\pgfpathlineto{\pgfqpoint{5.733002in}{0.613486in}}%
\pgfpathlineto{\pgfqpoint{5.733002in}{2.611074in}}%
\pgfpathlineto{\pgfqpoint{5.722997in}{2.611074in}}%
\pgfpathlineto{\pgfqpoint{5.722997in}{0.613486in}}%
\pgfpathclose%
\pgfusepath{fill}%
\end{pgfscope}%
\begin{pgfscope}%
\pgfpathrectangle{\pgfqpoint{0.693757in}{0.613486in}}{\pgfqpoint{5.541243in}{3.963477in}}%
\pgfusepath{clip}%
\pgfsetbuttcap%
\pgfsetmiterjoin%
\definecolor{currentfill}{rgb}{0.000000,0.000000,1.000000}%
\pgfsetfillcolor{currentfill}%
\pgfsetlinewidth{0.000000pt}%
\definecolor{currentstroke}{rgb}{0.000000,0.000000,0.000000}%
\pgfsetstrokecolor{currentstroke}%
\pgfsetstrokeopacity{0.000000}%
\pgfsetdash{}{0pt}%
\pgfpathmoveto{\pgfqpoint{5.735503in}{0.613486in}}%
\pgfpathlineto{\pgfqpoint{5.745508in}{0.613486in}}%
\pgfpathlineto{\pgfqpoint{5.745508in}{1.604355in}}%
\pgfpathlineto{\pgfqpoint{5.735503in}{1.604355in}}%
\pgfpathlineto{\pgfqpoint{5.735503in}{0.613486in}}%
\pgfpathclose%
\pgfusepath{fill}%
\end{pgfscope}%
\begin{pgfscope}%
\pgfpathrectangle{\pgfqpoint{0.693757in}{0.613486in}}{\pgfqpoint{5.541243in}{3.963477in}}%
\pgfusepath{clip}%
\pgfsetbuttcap%
\pgfsetmiterjoin%
\definecolor{currentfill}{rgb}{0.000000,0.000000,1.000000}%
\pgfsetfillcolor{currentfill}%
\pgfsetlinewidth{0.000000pt}%
\definecolor{currentstroke}{rgb}{0.000000,0.000000,0.000000}%
\pgfsetstrokecolor{currentstroke}%
\pgfsetstrokeopacity{0.000000}%
\pgfsetdash{}{0pt}%
\pgfpathmoveto{\pgfqpoint{5.748009in}{0.613486in}}%
\pgfpathlineto{\pgfqpoint{5.758014in}{0.613486in}}%
\pgfpathlineto{\pgfqpoint{5.758014in}{1.612282in}}%
\pgfpathlineto{\pgfqpoint{5.748009in}{1.612282in}}%
\pgfpathlineto{\pgfqpoint{5.748009in}{0.613486in}}%
\pgfpathclose%
\pgfusepath{fill}%
\end{pgfscope}%
\begin{pgfscope}%
\pgfpathrectangle{\pgfqpoint{0.693757in}{0.613486in}}{\pgfqpoint{5.541243in}{3.963477in}}%
\pgfusepath{clip}%
\pgfsetbuttcap%
\pgfsetmiterjoin%
\definecolor{currentfill}{rgb}{0.000000,0.000000,1.000000}%
\pgfsetfillcolor{currentfill}%
\pgfsetlinewidth{0.000000pt}%
\definecolor{currentstroke}{rgb}{0.000000,0.000000,0.000000}%
\pgfsetstrokecolor{currentstroke}%
\pgfsetstrokeopacity{0.000000}%
\pgfsetdash{}{0pt}%
\pgfpathmoveto{\pgfqpoint{5.760515in}{0.613486in}}%
\pgfpathlineto{\pgfqpoint{5.770520in}{0.613486in}}%
\pgfpathlineto{\pgfqpoint{5.770520in}{2.595224in}}%
\pgfpathlineto{\pgfqpoint{5.760515in}{2.595224in}}%
\pgfpathlineto{\pgfqpoint{5.760515in}{0.613486in}}%
\pgfpathclose%
\pgfusepath{fill}%
\end{pgfscope}%
\begin{pgfscope}%
\pgfpathrectangle{\pgfqpoint{0.693757in}{0.613486in}}{\pgfqpoint{5.541243in}{3.963477in}}%
\pgfusepath{clip}%
\pgfsetbuttcap%
\pgfsetmiterjoin%
\definecolor{currentfill}{rgb}{0.000000,0.000000,1.000000}%
\pgfsetfillcolor{currentfill}%
\pgfsetlinewidth{0.000000pt}%
\definecolor{currentstroke}{rgb}{0.000000,0.000000,0.000000}%
\pgfsetstrokecolor{currentstroke}%
\pgfsetstrokeopacity{0.000000}%
\pgfsetdash{}{0pt}%
\pgfpathmoveto{\pgfqpoint{5.773021in}{0.613486in}}%
\pgfpathlineto{\pgfqpoint{5.783026in}{0.613486in}}%
\pgfpathlineto{\pgfqpoint{5.783026in}{2.611074in}}%
\pgfpathlineto{\pgfqpoint{5.773021in}{2.611074in}}%
\pgfpathlineto{\pgfqpoint{5.773021in}{0.613486in}}%
\pgfpathclose%
\pgfusepath{fill}%
\end{pgfscope}%
\begin{pgfscope}%
\pgfpathrectangle{\pgfqpoint{0.693757in}{0.613486in}}{\pgfqpoint{5.541243in}{3.963477in}}%
\pgfusepath{clip}%
\pgfsetbuttcap%
\pgfsetmiterjoin%
\definecolor{currentfill}{rgb}{0.000000,0.000000,1.000000}%
\pgfsetfillcolor{currentfill}%
\pgfsetlinewidth{0.000000pt}%
\definecolor{currentstroke}{rgb}{0.000000,0.000000,0.000000}%
\pgfsetstrokecolor{currentstroke}%
\pgfsetstrokeopacity{0.000000}%
\pgfsetdash{}{0pt}%
\pgfpathmoveto{\pgfqpoint{5.785528in}{0.613486in}}%
\pgfpathlineto{\pgfqpoint{5.795532in}{0.613486in}}%
\pgfpathlineto{\pgfqpoint{5.795532in}{1.604355in}}%
\pgfpathlineto{\pgfqpoint{5.785528in}{1.604355in}}%
\pgfpathlineto{\pgfqpoint{5.785528in}{0.613486in}}%
\pgfpathclose%
\pgfusepath{fill}%
\end{pgfscope}%
\begin{pgfscope}%
\pgfpathrectangle{\pgfqpoint{0.693757in}{0.613486in}}{\pgfqpoint{5.541243in}{3.963477in}}%
\pgfusepath{clip}%
\pgfsetbuttcap%
\pgfsetmiterjoin%
\definecolor{currentfill}{rgb}{0.000000,0.000000,1.000000}%
\pgfsetfillcolor{currentfill}%
\pgfsetlinewidth{0.000000pt}%
\definecolor{currentstroke}{rgb}{0.000000,0.000000,0.000000}%
\pgfsetstrokecolor{currentstroke}%
\pgfsetstrokeopacity{0.000000}%
\pgfsetdash{}{0pt}%
\pgfpathmoveto{\pgfqpoint{5.798034in}{0.613486in}}%
\pgfpathlineto{\pgfqpoint{5.808039in}{0.613486in}}%
\pgfpathlineto{\pgfqpoint{5.808039in}{1.612282in}}%
\pgfpathlineto{\pgfqpoint{5.798034in}{1.612282in}}%
\pgfpathlineto{\pgfqpoint{5.798034in}{0.613486in}}%
\pgfpathclose%
\pgfusepath{fill}%
\end{pgfscope}%
\begin{pgfscope}%
\pgfpathrectangle{\pgfqpoint{0.693757in}{0.613486in}}{\pgfqpoint{5.541243in}{3.963477in}}%
\pgfusepath{clip}%
\pgfsetbuttcap%
\pgfsetmiterjoin%
\definecolor{currentfill}{rgb}{0.000000,0.000000,1.000000}%
\pgfsetfillcolor{currentfill}%
\pgfsetlinewidth{0.000000pt}%
\definecolor{currentstroke}{rgb}{0.000000,0.000000,0.000000}%
\pgfsetstrokecolor{currentstroke}%
\pgfsetstrokeopacity{0.000000}%
\pgfsetdash{}{0pt}%
\pgfpathmoveto{\pgfqpoint{5.810540in}{0.613486in}}%
\pgfpathlineto{\pgfqpoint{5.820545in}{0.613486in}}%
\pgfpathlineto{\pgfqpoint{5.820545in}{2.595224in}}%
\pgfpathlineto{\pgfqpoint{5.810540in}{2.595224in}}%
\pgfpathlineto{\pgfqpoint{5.810540in}{0.613486in}}%
\pgfpathclose%
\pgfusepath{fill}%
\end{pgfscope}%
\begin{pgfscope}%
\pgfpathrectangle{\pgfqpoint{0.693757in}{0.613486in}}{\pgfqpoint{5.541243in}{3.963477in}}%
\pgfusepath{clip}%
\pgfsetbuttcap%
\pgfsetmiterjoin%
\definecolor{currentfill}{rgb}{0.000000,0.000000,1.000000}%
\pgfsetfillcolor{currentfill}%
\pgfsetlinewidth{0.000000pt}%
\definecolor{currentstroke}{rgb}{0.000000,0.000000,0.000000}%
\pgfsetstrokecolor{currentstroke}%
\pgfsetstrokeopacity{0.000000}%
\pgfsetdash{}{0pt}%
\pgfpathmoveto{\pgfqpoint{5.823046in}{0.613486in}}%
\pgfpathlineto{\pgfqpoint{5.833051in}{0.613486in}}%
\pgfpathlineto{\pgfqpoint{5.833051in}{2.611074in}}%
\pgfpathlineto{\pgfqpoint{5.823046in}{2.611074in}}%
\pgfpathlineto{\pgfqpoint{5.823046in}{0.613486in}}%
\pgfpathclose%
\pgfusepath{fill}%
\end{pgfscope}%
\begin{pgfscope}%
\pgfpathrectangle{\pgfqpoint{0.693757in}{0.613486in}}{\pgfqpoint{5.541243in}{3.963477in}}%
\pgfusepath{clip}%
\pgfsetbuttcap%
\pgfsetmiterjoin%
\definecolor{currentfill}{rgb}{0.000000,0.000000,1.000000}%
\pgfsetfillcolor{currentfill}%
\pgfsetlinewidth{0.000000pt}%
\definecolor{currentstroke}{rgb}{0.000000,0.000000,0.000000}%
\pgfsetstrokecolor{currentstroke}%
\pgfsetstrokeopacity{0.000000}%
\pgfsetdash{}{0pt}%
\pgfpathmoveto{\pgfqpoint{5.835552in}{0.613486in}}%
\pgfpathlineto{\pgfqpoint{5.845557in}{0.613486in}}%
\pgfpathlineto{\pgfqpoint{5.845557in}{1.604355in}}%
\pgfpathlineto{\pgfqpoint{5.835552in}{1.604355in}}%
\pgfpathlineto{\pgfqpoint{5.835552in}{0.613486in}}%
\pgfpathclose%
\pgfusepath{fill}%
\end{pgfscope}%
\begin{pgfscope}%
\pgfpathrectangle{\pgfqpoint{0.693757in}{0.613486in}}{\pgfqpoint{5.541243in}{3.963477in}}%
\pgfusepath{clip}%
\pgfsetbuttcap%
\pgfsetmiterjoin%
\definecolor{currentfill}{rgb}{0.000000,0.000000,1.000000}%
\pgfsetfillcolor{currentfill}%
\pgfsetlinewidth{0.000000pt}%
\definecolor{currentstroke}{rgb}{0.000000,0.000000,0.000000}%
\pgfsetstrokecolor{currentstroke}%
\pgfsetstrokeopacity{0.000000}%
\pgfsetdash{}{0pt}%
\pgfpathmoveto{\pgfqpoint{5.848058in}{0.613486in}}%
\pgfpathlineto{\pgfqpoint{5.858063in}{0.613486in}}%
\pgfpathlineto{\pgfqpoint{5.858063in}{1.612282in}}%
\pgfpathlineto{\pgfqpoint{5.848058in}{1.612282in}}%
\pgfpathlineto{\pgfqpoint{5.848058in}{0.613486in}}%
\pgfpathclose%
\pgfusepath{fill}%
\end{pgfscope}%
\begin{pgfscope}%
\pgfpathrectangle{\pgfqpoint{0.693757in}{0.613486in}}{\pgfqpoint{5.541243in}{3.963477in}}%
\pgfusepath{clip}%
\pgfsetbuttcap%
\pgfsetmiterjoin%
\definecolor{currentfill}{rgb}{0.000000,0.000000,1.000000}%
\pgfsetfillcolor{currentfill}%
\pgfsetlinewidth{0.000000pt}%
\definecolor{currentstroke}{rgb}{0.000000,0.000000,0.000000}%
\pgfsetstrokecolor{currentstroke}%
\pgfsetstrokeopacity{0.000000}%
\pgfsetdash{}{0pt}%
\pgfpathmoveto{\pgfqpoint{5.860565in}{0.613486in}}%
\pgfpathlineto{\pgfqpoint{5.870570in}{0.613486in}}%
\pgfpathlineto{\pgfqpoint{5.870570in}{2.595224in}}%
\pgfpathlineto{\pgfqpoint{5.860565in}{2.595224in}}%
\pgfpathlineto{\pgfqpoint{5.860565in}{0.613486in}}%
\pgfpathclose%
\pgfusepath{fill}%
\end{pgfscope}%
\begin{pgfscope}%
\pgfpathrectangle{\pgfqpoint{0.693757in}{0.613486in}}{\pgfqpoint{5.541243in}{3.963477in}}%
\pgfusepath{clip}%
\pgfsetbuttcap%
\pgfsetmiterjoin%
\definecolor{currentfill}{rgb}{0.000000,0.000000,1.000000}%
\pgfsetfillcolor{currentfill}%
\pgfsetlinewidth{0.000000pt}%
\definecolor{currentstroke}{rgb}{0.000000,0.000000,0.000000}%
\pgfsetstrokecolor{currentstroke}%
\pgfsetstrokeopacity{0.000000}%
\pgfsetdash{}{0pt}%
\pgfpathmoveto{\pgfqpoint{5.873071in}{0.613486in}}%
\pgfpathlineto{\pgfqpoint{5.883076in}{0.613486in}}%
\pgfpathlineto{\pgfqpoint{5.883076in}{2.611074in}}%
\pgfpathlineto{\pgfqpoint{5.873071in}{2.611074in}}%
\pgfpathlineto{\pgfqpoint{5.873071in}{0.613486in}}%
\pgfpathclose%
\pgfusepath{fill}%
\end{pgfscope}%
\begin{pgfscope}%
\pgfpathrectangle{\pgfqpoint{0.693757in}{0.613486in}}{\pgfqpoint{5.541243in}{3.963477in}}%
\pgfusepath{clip}%
\pgfsetbuttcap%
\pgfsetmiterjoin%
\definecolor{currentfill}{rgb}{0.000000,0.000000,1.000000}%
\pgfsetfillcolor{currentfill}%
\pgfsetlinewidth{0.000000pt}%
\definecolor{currentstroke}{rgb}{0.000000,0.000000,0.000000}%
\pgfsetstrokecolor{currentstroke}%
\pgfsetstrokeopacity{0.000000}%
\pgfsetdash{}{0pt}%
\pgfpathmoveto{\pgfqpoint{5.885577in}{0.613486in}}%
\pgfpathlineto{\pgfqpoint{5.895582in}{0.613486in}}%
\pgfpathlineto{\pgfqpoint{5.895582in}{1.604355in}}%
\pgfpathlineto{\pgfqpoint{5.885577in}{1.604355in}}%
\pgfpathlineto{\pgfqpoint{5.885577in}{0.613486in}}%
\pgfpathclose%
\pgfusepath{fill}%
\end{pgfscope}%
\begin{pgfscope}%
\pgfpathrectangle{\pgfqpoint{0.693757in}{0.613486in}}{\pgfqpoint{5.541243in}{3.963477in}}%
\pgfusepath{clip}%
\pgfsetbuttcap%
\pgfsetmiterjoin%
\definecolor{currentfill}{rgb}{0.000000,0.000000,1.000000}%
\pgfsetfillcolor{currentfill}%
\pgfsetlinewidth{0.000000pt}%
\definecolor{currentstroke}{rgb}{0.000000,0.000000,0.000000}%
\pgfsetstrokecolor{currentstroke}%
\pgfsetstrokeopacity{0.000000}%
\pgfsetdash{}{0pt}%
\pgfpathmoveto{\pgfqpoint{5.898083in}{0.613486in}}%
\pgfpathlineto{\pgfqpoint{5.908088in}{0.613486in}}%
\pgfpathlineto{\pgfqpoint{5.908088in}{1.612282in}}%
\pgfpathlineto{\pgfqpoint{5.898083in}{1.612282in}}%
\pgfpathlineto{\pgfqpoint{5.898083in}{0.613486in}}%
\pgfpathclose%
\pgfusepath{fill}%
\end{pgfscope}%
\begin{pgfscope}%
\pgfpathrectangle{\pgfqpoint{0.693757in}{0.613486in}}{\pgfqpoint{5.541243in}{3.963477in}}%
\pgfusepath{clip}%
\pgfsetbuttcap%
\pgfsetmiterjoin%
\definecolor{currentfill}{rgb}{0.000000,0.000000,1.000000}%
\pgfsetfillcolor{currentfill}%
\pgfsetlinewidth{0.000000pt}%
\definecolor{currentstroke}{rgb}{0.000000,0.000000,0.000000}%
\pgfsetstrokecolor{currentstroke}%
\pgfsetstrokeopacity{0.000000}%
\pgfsetdash{}{0pt}%
\pgfpathmoveto{\pgfqpoint{5.910589in}{0.613486in}}%
\pgfpathlineto{\pgfqpoint{5.920594in}{0.613486in}}%
\pgfpathlineto{\pgfqpoint{5.920594in}{2.595224in}}%
\pgfpathlineto{\pgfqpoint{5.910589in}{2.595224in}}%
\pgfpathlineto{\pgfqpoint{5.910589in}{0.613486in}}%
\pgfpathclose%
\pgfusepath{fill}%
\end{pgfscope}%
\begin{pgfscope}%
\pgfpathrectangle{\pgfqpoint{0.693757in}{0.613486in}}{\pgfqpoint{5.541243in}{3.963477in}}%
\pgfusepath{clip}%
\pgfsetbuttcap%
\pgfsetmiterjoin%
\definecolor{currentfill}{rgb}{0.000000,0.000000,1.000000}%
\pgfsetfillcolor{currentfill}%
\pgfsetlinewidth{0.000000pt}%
\definecolor{currentstroke}{rgb}{0.000000,0.000000,0.000000}%
\pgfsetstrokecolor{currentstroke}%
\pgfsetstrokeopacity{0.000000}%
\pgfsetdash{}{0pt}%
\pgfpathmoveto{\pgfqpoint{5.923096in}{0.613486in}}%
\pgfpathlineto{\pgfqpoint{5.933101in}{0.613486in}}%
\pgfpathlineto{\pgfqpoint{5.933101in}{2.611074in}}%
\pgfpathlineto{\pgfqpoint{5.923096in}{2.611074in}}%
\pgfpathlineto{\pgfqpoint{5.923096in}{0.613486in}}%
\pgfpathclose%
\pgfusepath{fill}%
\end{pgfscope}%
\begin{pgfscope}%
\pgfpathrectangle{\pgfqpoint{0.693757in}{0.613486in}}{\pgfqpoint{5.541243in}{3.963477in}}%
\pgfusepath{clip}%
\pgfsetbuttcap%
\pgfsetmiterjoin%
\definecolor{currentfill}{rgb}{0.000000,0.000000,1.000000}%
\pgfsetfillcolor{currentfill}%
\pgfsetlinewidth{0.000000pt}%
\definecolor{currentstroke}{rgb}{0.000000,0.000000,0.000000}%
\pgfsetstrokecolor{currentstroke}%
\pgfsetstrokeopacity{0.000000}%
\pgfsetdash{}{0pt}%
\pgfpathmoveto{\pgfqpoint{5.935602in}{0.613486in}}%
\pgfpathlineto{\pgfqpoint{5.945607in}{0.613486in}}%
\pgfpathlineto{\pgfqpoint{5.945607in}{1.604355in}}%
\pgfpathlineto{\pgfqpoint{5.935602in}{1.604355in}}%
\pgfpathlineto{\pgfqpoint{5.935602in}{0.613486in}}%
\pgfpathclose%
\pgfusepath{fill}%
\end{pgfscope}%
\begin{pgfscope}%
\pgfpathrectangle{\pgfqpoint{0.693757in}{0.613486in}}{\pgfqpoint{5.541243in}{3.963477in}}%
\pgfusepath{clip}%
\pgfsetbuttcap%
\pgfsetmiterjoin%
\definecolor{currentfill}{rgb}{0.000000,0.000000,1.000000}%
\pgfsetfillcolor{currentfill}%
\pgfsetlinewidth{0.000000pt}%
\definecolor{currentstroke}{rgb}{0.000000,0.000000,0.000000}%
\pgfsetstrokecolor{currentstroke}%
\pgfsetstrokeopacity{0.000000}%
\pgfsetdash{}{0pt}%
\pgfpathmoveto{\pgfqpoint{5.948108in}{0.613486in}}%
\pgfpathlineto{\pgfqpoint{5.958113in}{0.613486in}}%
\pgfpathlineto{\pgfqpoint{5.958113in}{1.612282in}}%
\pgfpathlineto{\pgfqpoint{5.948108in}{1.612282in}}%
\pgfpathlineto{\pgfqpoint{5.948108in}{0.613486in}}%
\pgfpathclose%
\pgfusepath{fill}%
\end{pgfscope}%
\begin{pgfscope}%
\pgfpathrectangle{\pgfqpoint{0.693757in}{0.613486in}}{\pgfqpoint{5.541243in}{3.963477in}}%
\pgfusepath{clip}%
\pgfsetbuttcap%
\pgfsetmiterjoin%
\definecolor{currentfill}{rgb}{0.000000,0.000000,1.000000}%
\pgfsetfillcolor{currentfill}%
\pgfsetlinewidth{0.000000pt}%
\definecolor{currentstroke}{rgb}{0.000000,0.000000,0.000000}%
\pgfsetstrokecolor{currentstroke}%
\pgfsetstrokeopacity{0.000000}%
\pgfsetdash{}{0pt}%
\pgfpathmoveto{\pgfqpoint{5.960614in}{0.613486in}}%
\pgfpathlineto{\pgfqpoint{5.970619in}{0.613486in}}%
\pgfpathlineto{\pgfqpoint{5.970619in}{2.595224in}}%
\pgfpathlineto{\pgfqpoint{5.960614in}{2.595224in}}%
\pgfpathlineto{\pgfqpoint{5.960614in}{0.613486in}}%
\pgfpathclose%
\pgfusepath{fill}%
\end{pgfscope}%
\begin{pgfscope}%
\pgfpathrectangle{\pgfqpoint{0.693757in}{0.613486in}}{\pgfqpoint{5.541243in}{3.963477in}}%
\pgfusepath{clip}%
\pgfsetbuttcap%
\pgfsetmiterjoin%
\definecolor{currentfill}{rgb}{0.000000,0.000000,1.000000}%
\pgfsetfillcolor{currentfill}%
\pgfsetlinewidth{0.000000pt}%
\definecolor{currentstroke}{rgb}{0.000000,0.000000,0.000000}%
\pgfsetstrokecolor{currentstroke}%
\pgfsetstrokeopacity{0.000000}%
\pgfsetdash{}{0pt}%
\pgfpathmoveto{\pgfqpoint{5.973120in}{0.613486in}}%
\pgfpathlineto{\pgfqpoint{5.983125in}{0.613486in}}%
\pgfpathlineto{\pgfqpoint{5.983125in}{2.611074in}}%
\pgfpathlineto{\pgfqpoint{5.973120in}{2.611074in}}%
\pgfpathlineto{\pgfqpoint{5.973120in}{0.613486in}}%
\pgfpathclose%
\pgfusepath{fill}%
\end{pgfscope}%
\begin{pgfscope}%
\pgfsetbuttcap%
\pgfsetroundjoin%
\definecolor{currentfill}{rgb}{0.000000,0.000000,0.000000}%
\pgfsetfillcolor{currentfill}%
\pgfsetlinewidth{0.803000pt}%
\definecolor{currentstroke}{rgb}{0.000000,0.000000,0.000000}%
\pgfsetstrokecolor{currentstroke}%
\pgfsetdash{}{0pt}%
\pgfsys@defobject{currentmarker}{\pgfqpoint{0.000000in}{-0.048611in}}{\pgfqpoint{0.000000in}{0.000000in}}{%
\pgfpathmoveto{\pgfqpoint{0.000000in}{0.000000in}}%
\pgfpathlineto{\pgfqpoint{0.000000in}{-0.048611in}}%
\pgfusepath{stroke,fill}%
}%
\begin{pgfscope}%
\pgfsys@transformshift{0.925622in}{0.613486in}%
\pgfsys@useobject{currentmarker}{}%
\end{pgfscope}%
\end{pgfscope}%
\begin{pgfscope}%
\definecolor{textcolor}{rgb}{0.000000,0.000000,0.000000}%
\pgfsetstrokecolor{textcolor}%
\pgfsetfillcolor{textcolor}%
\pgftext[x=0.925622in,y=0.516264in,,top]{\color{textcolor}{\sffamily\fontsize{11.000000}{13.200000}\selectfont\catcode`\^=\active\def^{\ifmmode\sp\else\^{}\fi}\catcode`\%=\active\def%{\%}$\mathdefault{0}$}}%
\end{pgfscope}%
\begin{pgfscope}%
\pgfsetbuttcap%
\pgfsetroundjoin%
\definecolor{currentfill}{rgb}{0.000000,0.000000,0.000000}%
\pgfsetfillcolor{currentfill}%
\pgfsetlinewidth{0.803000pt}%
\definecolor{currentstroke}{rgb}{0.000000,0.000000,0.000000}%
\pgfsetstrokecolor{currentstroke}%
\pgfsetdash{}{0pt}%
\pgfsys@defobject{currentmarker}{\pgfqpoint{0.000000in}{-0.048611in}}{\pgfqpoint{0.000000in}{0.000000in}}{%
\pgfpathmoveto{\pgfqpoint{0.000000in}{0.000000in}}%
\pgfpathlineto{\pgfqpoint{0.000000in}{-0.048611in}}%
\pgfusepath{stroke,fill}%
}%
\begin{pgfscope}%
\pgfsys@transformshift{1.550931in}{0.613486in}%
\pgfsys@useobject{currentmarker}{}%
\end{pgfscope}%
\end{pgfscope}%
\begin{pgfscope}%
\definecolor{textcolor}{rgb}{0.000000,0.000000,0.000000}%
\pgfsetstrokecolor{textcolor}%
\pgfsetfillcolor{textcolor}%
\pgftext[x=1.550931in,y=0.516264in,,top]{\color{textcolor}{\sffamily\fontsize{11.000000}{13.200000}\selectfont\catcode`\^=\active\def^{\ifmmode\sp\else\^{}\fi}\catcode`\%=\active\def%{\%}$\mathdefault{50}$}}%
\end{pgfscope}%
\begin{pgfscope}%
\pgfsetbuttcap%
\pgfsetroundjoin%
\definecolor{currentfill}{rgb}{0.000000,0.000000,0.000000}%
\pgfsetfillcolor{currentfill}%
\pgfsetlinewidth{0.803000pt}%
\definecolor{currentstroke}{rgb}{0.000000,0.000000,0.000000}%
\pgfsetstrokecolor{currentstroke}%
\pgfsetdash{}{0pt}%
\pgfsys@defobject{currentmarker}{\pgfqpoint{0.000000in}{-0.048611in}}{\pgfqpoint{0.000000in}{0.000000in}}{%
\pgfpathmoveto{\pgfqpoint{0.000000in}{0.000000in}}%
\pgfpathlineto{\pgfqpoint{0.000000in}{-0.048611in}}%
\pgfusepath{stroke,fill}%
}%
\begin{pgfscope}%
\pgfsys@transformshift{2.176241in}{0.613486in}%
\pgfsys@useobject{currentmarker}{}%
\end{pgfscope}%
\end{pgfscope}%
\begin{pgfscope}%
\definecolor{textcolor}{rgb}{0.000000,0.000000,0.000000}%
\pgfsetstrokecolor{textcolor}%
\pgfsetfillcolor{textcolor}%
\pgftext[x=2.176241in,y=0.516264in,,top]{\color{textcolor}{\sffamily\fontsize{11.000000}{13.200000}\selectfont\catcode`\^=\active\def^{\ifmmode\sp\else\^{}\fi}\catcode`\%=\active\def%{\%}$\mathdefault{100}$}}%
\end{pgfscope}%
\begin{pgfscope}%
\pgfsetbuttcap%
\pgfsetroundjoin%
\definecolor{currentfill}{rgb}{0.000000,0.000000,0.000000}%
\pgfsetfillcolor{currentfill}%
\pgfsetlinewidth{0.803000pt}%
\definecolor{currentstroke}{rgb}{0.000000,0.000000,0.000000}%
\pgfsetstrokecolor{currentstroke}%
\pgfsetdash{}{0pt}%
\pgfsys@defobject{currentmarker}{\pgfqpoint{0.000000in}{-0.048611in}}{\pgfqpoint{0.000000in}{0.000000in}}{%
\pgfpathmoveto{\pgfqpoint{0.000000in}{0.000000in}}%
\pgfpathlineto{\pgfqpoint{0.000000in}{-0.048611in}}%
\pgfusepath{stroke,fill}%
}%
\begin{pgfscope}%
\pgfsys@transformshift{2.801550in}{0.613486in}%
\pgfsys@useobject{currentmarker}{}%
\end{pgfscope}%
\end{pgfscope}%
\begin{pgfscope}%
\definecolor{textcolor}{rgb}{0.000000,0.000000,0.000000}%
\pgfsetstrokecolor{textcolor}%
\pgfsetfillcolor{textcolor}%
\pgftext[x=2.801550in,y=0.516264in,,top]{\color{textcolor}{\sffamily\fontsize{11.000000}{13.200000}\selectfont\catcode`\^=\active\def^{\ifmmode\sp\else\^{}\fi}\catcode`\%=\active\def%{\%}$\mathdefault{150}$}}%
\end{pgfscope}%
\begin{pgfscope}%
\pgfsetbuttcap%
\pgfsetroundjoin%
\definecolor{currentfill}{rgb}{0.000000,0.000000,0.000000}%
\pgfsetfillcolor{currentfill}%
\pgfsetlinewidth{0.803000pt}%
\definecolor{currentstroke}{rgb}{0.000000,0.000000,0.000000}%
\pgfsetstrokecolor{currentstroke}%
\pgfsetdash{}{0pt}%
\pgfsys@defobject{currentmarker}{\pgfqpoint{0.000000in}{-0.048611in}}{\pgfqpoint{0.000000in}{0.000000in}}{%
\pgfpathmoveto{\pgfqpoint{0.000000in}{0.000000in}}%
\pgfpathlineto{\pgfqpoint{0.000000in}{-0.048611in}}%
\pgfusepath{stroke,fill}%
}%
\begin{pgfscope}%
\pgfsys@transformshift{3.426860in}{0.613486in}%
\pgfsys@useobject{currentmarker}{}%
\end{pgfscope}%
\end{pgfscope}%
\begin{pgfscope}%
\definecolor{textcolor}{rgb}{0.000000,0.000000,0.000000}%
\pgfsetstrokecolor{textcolor}%
\pgfsetfillcolor{textcolor}%
\pgftext[x=3.426860in,y=0.516264in,,top]{\color{textcolor}{\sffamily\fontsize{11.000000}{13.200000}\selectfont\catcode`\^=\active\def^{\ifmmode\sp\else\^{}\fi}\catcode`\%=\active\def%{\%}$\mathdefault{200}$}}%
\end{pgfscope}%
\begin{pgfscope}%
\pgfsetbuttcap%
\pgfsetroundjoin%
\definecolor{currentfill}{rgb}{0.000000,0.000000,0.000000}%
\pgfsetfillcolor{currentfill}%
\pgfsetlinewidth{0.803000pt}%
\definecolor{currentstroke}{rgb}{0.000000,0.000000,0.000000}%
\pgfsetstrokecolor{currentstroke}%
\pgfsetdash{}{0pt}%
\pgfsys@defobject{currentmarker}{\pgfqpoint{0.000000in}{-0.048611in}}{\pgfqpoint{0.000000in}{0.000000in}}{%
\pgfpathmoveto{\pgfqpoint{0.000000in}{0.000000in}}%
\pgfpathlineto{\pgfqpoint{0.000000in}{-0.048611in}}%
\pgfusepath{stroke,fill}%
}%
\begin{pgfscope}%
\pgfsys@transformshift{4.052169in}{0.613486in}%
\pgfsys@useobject{currentmarker}{}%
\end{pgfscope}%
\end{pgfscope}%
\begin{pgfscope}%
\definecolor{textcolor}{rgb}{0.000000,0.000000,0.000000}%
\pgfsetstrokecolor{textcolor}%
\pgfsetfillcolor{textcolor}%
\pgftext[x=4.052169in,y=0.516264in,,top]{\color{textcolor}{\sffamily\fontsize{11.000000}{13.200000}\selectfont\catcode`\^=\active\def^{\ifmmode\sp\else\^{}\fi}\catcode`\%=\active\def%{\%}$\mathdefault{250}$}}%
\end{pgfscope}%
\begin{pgfscope}%
\pgfsetbuttcap%
\pgfsetroundjoin%
\definecolor{currentfill}{rgb}{0.000000,0.000000,0.000000}%
\pgfsetfillcolor{currentfill}%
\pgfsetlinewidth{0.803000pt}%
\definecolor{currentstroke}{rgb}{0.000000,0.000000,0.000000}%
\pgfsetstrokecolor{currentstroke}%
\pgfsetdash{}{0pt}%
\pgfsys@defobject{currentmarker}{\pgfqpoint{0.000000in}{-0.048611in}}{\pgfqpoint{0.000000in}{0.000000in}}{%
\pgfpathmoveto{\pgfqpoint{0.000000in}{0.000000in}}%
\pgfpathlineto{\pgfqpoint{0.000000in}{-0.048611in}}%
\pgfusepath{stroke,fill}%
}%
\begin{pgfscope}%
\pgfsys@transformshift{4.677479in}{0.613486in}%
\pgfsys@useobject{currentmarker}{}%
\end{pgfscope}%
\end{pgfscope}%
\begin{pgfscope}%
\definecolor{textcolor}{rgb}{0.000000,0.000000,0.000000}%
\pgfsetstrokecolor{textcolor}%
\pgfsetfillcolor{textcolor}%
\pgftext[x=4.677479in,y=0.516264in,,top]{\color{textcolor}{\sffamily\fontsize{11.000000}{13.200000}\selectfont\catcode`\^=\active\def^{\ifmmode\sp\else\^{}\fi}\catcode`\%=\active\def%{\%}$\mathdefault{300}$}}%
\end{pgfscope}%
\begin{pgfscope}%
\pgfsetbuttcap%
\pgfsetroundjoin%
\definecolor{currentfill}{rgb}{0.000000,0.000000,0.000000}%
\pgfsetfillcolor{currentfill}%
\pgfsetlinewidth{0.803000pt}%
\definecolor{currentstroke}{rgb}{0.000000,0.000000,0.000000}%
\pgfsetstrokecolor{currentstroke}%
\pgfsetdash{}{0pt}%
\pgfsys@defobject{currentmarker}{\pgfqpoint{0.000000in}{-0.048611in}}{\pgfqpoint{0.000000in}{0.000000in}}{%
\pgfpathmoveto{\pgfqpoint{0.000000in}{0.000000in}}%
\pgfpathlineto{\pgfqpoint{0.000000in}{-0.048611in}}%
\pgfusepath{stroke,fill}%
}%
\begin{pgfscope}%
\pgfsys@transformshift{5.302789in}{0.613486in}%
\pgfsys@useobject{currentmarker}{}%
\end{pgfscope}%
\end{pgfscope}%
\begin{pgfscope}%
\definecolor{textcolor}{rgb}{0.000000,0.000000,0.000000}%
\pgfsetstrokecolor{textcolor}%
\pgfsetfillcolor{textcolor}%
\pgftext[x=5.302789in,y=0.516264in,,top]{\color{textcolor}{\sffamily\fontsize{11.000000}{13.200000}\selectfont\catcode`\^=\active\def^{\ifmmode\sp\else\^{}\fi}\catcode`\%=\active\def%{\%}$\mathdefault{350}$}}%
\end{pgfscope}%
\begin{pgfscope}%
\pgfsetbuttcap%
\pgfsetroundjoin%
\definecolor{currentfill}{rgb}{0.000000,0.000000,0.000000}%
\pgfsetfillcolor{currentfill}%
\pgfsetlinewidth{0.803000pt}%
\definecolor{currentstroke}{rgb}{0.000000,0.000000,0.000000}%
\pgfsetstrokecolor{currentstroke}%
\pgfsetdash{}{0pt}%
\pgfsys@defobject{currentmarker}{\pgfqpoint{0.000000in}{-0.048611in}}{\pgfqpoint{0.000000in}{0.000000in}}{%
\pgfpathmoveto{\pgfqpoint{0.000000in}{0.000000in}}%
\pgfpathlineto{\pgfqpoint{0.000000in}{-0.048611in}}%
\pgfusepath{stroke,fill}%
}%
\begin{pgfscope}%
\pgfsys@transformshift{5.928098in}{0.613486in}%
\pgfsys@useobject{currentmarker}{}%
\end{pgfscope}%
\end{pgfscope}%
\begin{pgfscope}%
\definecolor{textcolor}{rgb}{0.000000,0.000000,0.000000}%
\pgfsetstrokecolor{textcolor}%
\pgfsetfillcolor{textcolor}%
\pgftext[x=5.928098in,y=0.516264in,,top]{\color{textcolor}{\sffamily\fontsize{11.000000}{13.200000}\selectfont\catcode`\^=\active\def^{\ifmmode\sp\else\^{}\fi}\catcode`\%=\active\def%{\%}$\mathdefault{400}$}}%
\end{pgfscope}%
\begin{pgfscope}%
\definecolor{textcolor}{rgb}{0.000000,0.000000,0.000000}%
\pgfsetstrokecolor{textcolor}%
\pgfsetfillcolor{textcolor}%
\pgftext[x=3.464379in,y=0.312854in,,top]{\color{textcolor}{\sffamily\fontsize{11.000000}{13.200000}\selectfont\catcode`\^=\active\def^{\ifmmode\sp\else\^{}\fi}\catcode`\%=\active\def%{\%}Kernel index}}%
\end{pgfscope}%
\begin{pgfscope}%
\pgfsetbuttcap%
\pgfsetroundjoin%
\definecolor{currentfill}{rgb}{0.000000,0.000000,0.000000}%
\pgfsetfillcolor{currentfill}%
\pgfsetlinewidth{0.803000pt}%
\definecolor{currentstroke}{rgb}{0.000000,0.000000,0.000000}%
\pgfsetstrokecolor{currentstroke}%
\pgfsetdash{}{0pt}%
\pgfsys@defobject{currentmarker}{\pgfqpoint{-0.048611in}{0.000000in}}{\pgfqpoint{-0.000000in}{0.000000in}}{%
\pgfpathmoveto{\pgfqpoint{-0.000000in}{0.000000in}}%
\pgfpathlineto{\pgfqpoint{-0.048611in}{0.000000in}}%
\pgfusepath{stroke,fill}%
}%
\begin{pgfscope}%
\pgfsys@transformshift{0.693757in}{0.613486in}%
\pgfsys@useobject{currentmarker}{}%
\end{pgfscope}%
\end{pgfscope}%
\begin{pgfscope}%
\definecolor{textcolor}{rgb}{0.000000,0.000000,0.000000}%
\pgfsetstrokecolor{textcolor}%
\pgfsetfillcolor{textcolor}%
\pgftext[x=0.520493in, y=0.555448in, left, base]{\color{textcolor}{\sffamily\fontsize{11.000000}{13.200000}\selectfont\catcode`\^=\active\def^{\ifmmode\sp\else\^{}\fi}\catcode`\%=\active\def%{\%}$\mathdefault{0}$}}%
\end{pgfscope}%
\begin{pgfscope}%
\pgfsetbuttcap%
\pgfsetroundjoin%
\definecolor{currentfill}{rgb}{0.000000,0.000000,0.000000}%
\pgfsetfillcolor{currentfill}%
\pgfsetlinewidth{0.803000pt}%
\definecolor{currentstroke}{rgb}{0.000000,0.000000,0.000000}%
\pgfsetstrokecolor{currentstroke}%
\pgfsetdash{}{0pt}%
\pgfsys@defobject{currentmarker}{\pgfqpoint{-0.048611in}{0.000000in}}{\pgfqpoint{-0.000000in}{0.000000in}}{%
\pgfpathmoveto{\pgfqpoint{-0.000000in}{0.000000in}}%
\pgfpathlineto{\pgfqpoint{-0.048611in}{0.000000in}}%
\pgfusepath{stroke,fill}%
}%
\begin{pgfscope}%
\pgfsys@transformshift{0.693757in}{1.406181in}%
\pgfsys@useobject{currentmarker}{}%
\end{pgfscope}%
\end{pgfscope}%
\begin{pgfscope}%
\definecolor{textcolor}{rgb}{0.000000,0.000000,0.000000}%
\pgfsetstrokecolor{textcolor}%
\pgfsetfillcolor{textcolor}%
\pgftext[x=0.444451in, y=1.348144in, left, base]{\color{textcolor}{\sffamily\fontsize{11.000000}{13.200000}\selectfont\catcode`\^=\active\def^{\ifmmode\sp\else\^{}\fi}\catcode`\%=\active\def%{\%}$\mathdefault{20}$}}%
\end{pgfscope}%
\begin{pgfscope}%
\pgfsetbuttcap%
\pgfsetroundjoin%
\definecolor{currentfill}{rgb}{0.000000,0.000000,0.000000}%
\pgfsetfillcolor{currentfill}%
\pgfsetlinewidth{0.803000pt}%
\definecolor{currentstroke}{rgb}{0.000000,0.000000,0.000000}%
\pgfsetstrokecolor{currentstroke}%
\pgfsetdash{}{0pt}%
\pgfsys@defobject{currentmarker}{\pgfqpoint{-0.048611in}{0.000000in}}{\pgfqpoint{-0.000000in}{0.000000in}}{%
\pgfpathmoveto{\pgfqpoint{-0.000000in}{0.000000in}}%
\pgfpathlineto{\pgfqpoint{-0.048611in}{0.000000in}}%
\pgfusepath{stroke,fill}%
}%
\begin{pgfscope}%
\pgfsys@transformshift{0.693757in}{2.198876in}%
\pgfsys@useobject{currentmarker}{}%
\end{pgfscope}%
\end{pgfscope}%
\begin{pgfscope}%
\definecolor{textcolor}{rgb}{0.000000,0.000000,0.000000}%
\pgfsetstrokecolor{textcolor}%
\pgfsetfillcolor{textcolor}%
\pgftext[x=0.444451in, y=2.140839in, left, base]{\color{textcolor}{\sffamily\fontsize{11.000000}{13.200000}\selectfont\catcode`\^=\active\def^{\ifmmode\sp\else\^{}\fi}\catcode`\%=\active\def%{\%}$\mathdefault{40}$}}%
\end{pgfscope}%
\begin{pgfscope}%
\pgfsetbuttcap%
\pgfsetroundjoin%
\definecolor{currentfill}{rgb}{0.000000,0.000000,0.000000}%
\pgfsetfillcolor{currentfill}%
\pgfsetlinewidth{0.803000pt}%
\definecolor{currentstroke}{rgb}{0.000000,0.000000,0.000000}%
\pgfsetstrokecolor{currentstroke}%
\pgfsetdash{}{0pt}%
\pgfsys@defobject{currentmarker}{\pgfqpoint{-0.048611in}{0.000000in}}{\pgfqpoint{-0.000000in}{0.000000in}}{%
\pgfpathmoveto{\pgfqpoint{-0.000000in}{0.000000in}}%
\pgfpathlineto{\pgfqpoint{-0.048611in}{0.000000in}}%
\pgfusepath{stroke,fill}%
}%
\begin{pgfscope}%
\pgfsys@transformshift{0.693757in}{2.991572in}%
\pgfsys@useobject{currentmarker}{}%
\end{pgfscope}%
\end{pgfscope}%
\begin{pgfscope}%
\definecolor{textcolor}{rgb}{0.000000,0.000000,0.000000}%
\pgfsetstrokecolor{textcolor}%
\pgfsetfillcolor{textcolor}%
\pgftext[x=0.444451in, y=2.933534in, left, base]{\color{textcolor}{\sffamily\fontsize{11.000000}{13.200000}\selectfont\catcode`\^=\active\def^{\ifmmode\sp\else\^{}\fi}\catcode`\%=\active\def%{\%}$\mathdefault{60}$}}%
\end{pgfscope}%
\begin{pgfscope}%
\pgfsetbuttcap%
\pgfsetroundjoin%
\definecolor{currentfill}{rgb}{0.000000,0.000000,0.000000}%
\pgfsetfillcolor{currentfill}%
\pgfsetlinewidth{0.803000pt}%
\definecolor{currentstroke}{rgb}{0.000000,0.000000,0.000000}%
\pgfsetstrokecolor{currentstroke}%
\pgfsetdash{}{0pt}%
\pgfsys@defobject{currentmarker}{\pgfqpoint{-0.048611in}{0.000000in}}{\pgfqpoint{-0.000000in}{0.000000in}}{%
\pgfpathmoveto{\pgfqpoint{-0.000000in}{0.000000in}}%
\pgfpathlineto{\pgfqpoint{-0.048611in}{0.000000in}}%
\pgfusepath{stroke,fill}%
}%
\begin{pgfscope}%
\pgfsys@transformshift{0.693757in}{3.784267in}%
\pgfsys@useobject{currentmarker}{}%
\end{pgfscope}%
\end{pgfscope}%
\begin{pgfscope}%
\definecolor{textcolor}{rgb}{0.000000,0.000000,0.000000}%
\pgfsetstrokecolor{textcolor}%
\pgfsetfillcolor{textcolor}%
\pgftext[x=0.444451in, y=3.726229in, left, base]{\color{textcolor}{\sffamily\fontsize{11.000000}{13.200000}\selectfont\catcode`\^=\active\def^{\ifmmode\sp\else\^{}\fi}\catcode`\%=\active\def%{\%}$\mathdefault{80}$}}%
\end{pgfscope}%
\begin{pgfscope}%
\pgfsetbuttcap%
\pgfsetroundjoin%
\definecolor{currentfill}{rgb}{0.000000,0.000000,0.000000}%
\pgfsetfillcolor{currentfill}%
\pgfsetlinewidth{0.803000pt}%
\definecolor{currentstroke}{rgb}{0.000000,0.000000,0.000000}%
\pgfsetstrokecolor{currentstroke}%
\pgfsetdash{}{0pt}%
\pgfsys@defobject{currentmarker}{\pgfqpoint{-0.048611in}{0.000000in}}{\pgfqpoint{-0.000000in}{0.000000in}}{%
\pgfpathmoveto{\pgfqpoint{-0.000000in}{0.000000in}}%
\pgfpathlineto{\pgfqpoint{-0.048611in}{0.000000in}}%
\pgfusepath{stroke,fill}%
}%
\begin{pgfscope}%
\pgfsys@transformshift{0.693757in}{4.576962in}%
\pgfsys@useobject{currentmarker}{}%
\end{pgfscope}%
\end{pgfscope}%
\begin{pgfscope}%
\definecolor{textcolor}{rgb}{0.000000,0.000000,0.000000}%
\pgfsetstrokecolor{textcolor}%
\pgfsetfillcolor{textcolor}%
\pgftext[x=0.368410in, y=4.518925in, left, base]{\color{textcolor}{\sffamily\fontsize{11.000000}{13.200000}\selectfont\catcode`\^=\active\def^{\ifmmode\sp\else\^{}\fi}\catcode`\%=\active\def%{\%}$\mathdefault{100}$}}%
\end{pgfscope}%
\begin{pgfscope}%
\definecolor{textcolor}{rgb}{0.000000,0.000000,0.000000}%
\pgfsetstrokecolor{textcolor}%
\pgfsetfillcolor{textcolor}%
\pgftext[x=0.312854in,y=2.595224in,,bottom,rotate=90.000000]{\color{textcolor}{\sffamily\fontsize{11.000000}{13.200000}\selectfont\catcode`\^=\active\def^{\ifmmode\sp\else\^{}\fi}\catcode`\%=\active\def%{\%}Data reuse (in %)}}%
\end{pgfscope}%
\begin{pgfscope}%
\pgfsetrectcap%
\pgfsetmiterjoin%
\pgfsetlinewidth{0.803000pt}%
\definecolor{currentstroke}{rgb}{0.000000,0.000000,0.000000}%
\pgfsetstrokecolor{currentstroke}%
\pgfsetdash{}{0pt}%
\pgfpathmoveto{\pgfqpoint{0.693757in}{0.613486in}}%
\pgfpathlineto{\pgfqpoint{0.693757in}{4.576962in}}%
\pgfusepath{stroke}%
\end{pgfscope}%
\begin{pgfscope}%
\pgfsetrectcap%
\pgfsetmiterjoin%
\pgfsetlinewidth{0.803000pt}%
\definecolor{currentstroke}{rgb}{0.000000,0.000000,0.000000}%
\pgfsetstrokecolor{currentstroke}%
\pgfsetdash{}{0pt}%
\pgfpathmoveto{\pgfqpoint{6.235000in}{0.613486in}}%
\pgfpathlineto{\pgfqpoint{6.235000in}{4.576962in}}%
\pgfusepath{stroke}%
\end{pgfscope}%
\begin{pgfscope}%
\pgfsetrectcap%
\pgfsetmiterjoin%
\pgfsetlinewidth{0.803000pt}%
\definecolor{currentstroke}{rgb}{0.000000,0.000000,0.000000}%
\pgfsetstrokecolor{currentstroke}%
\pgfsetdash{}{0pt}%
\pgfpathmoveto{\pgfqpoint{0.693757in}{0.613486in}}%
\pgfpathlineto{\pgfqpoint{6.235000in}{0.613486in}}%
\pgfusepath{stroke}%
\end{pgfscope}%
\begin{pgfscope}%
\pgfsetrectcap%
\pgfsetmiterjoin%
\pgfsetlinewidth{0.803000pt}%
\definecolor{currentstroke}{rgb}{0.000000,0.000000,0.000000}%
\pgfsetstrokecolor{currentstroke}%
\pgfsetdash{}{0pt}%
\pgfpathmoveto{\pgfqpoint{0.693757in}{4.576962in}}%
\pgfpathlineto{\pgfqpoint{6.235000in}{4.576962in}}%
\pgfusepath{stroke}%
\end{pgfscope}%
\end{pgfpicture}%
\makeatother%
\endgroup%
}
        \caption{Backward data reuse}
        \label{fig:recg_backward_reuse}
    \end{subfigure}
    \caption{Inter-kernel data reuse in recursiveGaussian}
    \label{fig:recg_reuse}
\end{figure}

In \cref{fig:dct_reuse}, you can see the data reuse ratio for the DCT workload.
We've analyzed both the forward and backward data reuse for each kernel:
\begin{itemize}
    \item \textbf{Forward data reuse:} the amount of unique memory addresses in kernel $k_{i-1}$ that are also present in kernel $k_{i}$ (as seen in \cref{fig:dct_forward_reuse}); i.e.
    \begin{align}
        \frac{|\text{footprint } k_{i-1} \cap \text{footprint } k_i|}{|\text{footprint } k_{i-1}|}
    \end{align}
    \item \textbf{Backward data reuse:} the amount of unique memory addresses in kernel $k_{i}$ that are also present in kernel $k_{i-1}$ (as seen in \cref{fig:dct_backward_reuse}); i.e.
    \begin{align}
        \frac{|\text{footprint } k_{i-1} \cap \text{footprint } k_i|}{|\text{footprint } k_i|}
    \end{align}
\end{itemize}

When looking at the DCT workload, most kernels have a very high degree of data reuse, both backward and forward.
As we analyzed before, this workload suffers heavily from the cold-start problem (in hardware).

For contrast, we've also analyzed \verb|recursiveGaussian| (from the CUDA SDK); as you can see in \cref{fig:recg_reuse}.
Roughly half of the kernels hit 50\% of data reuse (in either direction), while the other half has a much lower degree of data reuse.
This is also reflected in the IPC difference due to the cold-start problem: only about 1.3\% of the workload suffers from at least 5\% IPC difference.

The notion of data reuse, more specifically forward data reuse, will be used in the final chapter, when we discuss possible mitigations to the cold-start problem.

\FloatBarrier
\section{Hardware conclusion}\label{sec:hw-conclusion}
From this, we can conclude that the cold-start problem does exist in hardware.
Additionally, we noticed that the problem changes severity when we take each relative kernel's instruction count into account, giving us other workloads to focus on.
Finally, we suspected that workloads with high inter-kernel data reuse would suffer more from the cold-start problem.
This was confirmed by the DCT workload, as we've shown in \cref{fig:dct_reuse}.

In the next chapter, we'll be analyzing the impact of the cold-start problem in the AccelSim simulator.
We'll mostly use the DCT and 3D U-Net workloads for this, as well as the OceanFFT one.
