\chapter{Conclusion}\label{ch:conclusion}

In this thesis, we have analyzed the cold-start problem in GPUs.
We have shown that the cold-start problem is a real issue, affecting the performance of GPUs in a significant way.
We have also shown that the cold-start problem is not limited to a single workload, but that it can affect a wide range of workloads.
In this chapter, we will present our conclusions, discussing both quantification and mitigation.

\section{Quantification}\label{sec:quantification}
We started by analyzing whether the cold-start problem is an actual issue.
To this end, we analyzed multiple workloads on real hardware, and a limited number of them in the Accel-Sim simulator.
From this, we have concluded that the cold-start problem is a real issue, affecting the performance of GPUs in a significant way.

\subsection{The cold-start problem in hardware}\label{subsec:cold-start-in-hardware}
We have analyzed the cold-start problem in real hardware, and have shown that it is a real issue.
Multiple workloads have shown a significant impact from the cold-start problem.

Additionally, we have weighted each kernel based on its instruction count, and have seen that its severity changes based on the kernel's weight.
Taking into account is important because sampling methods like \textit{Sieve} uses these weights to select the kernels to simulate.

Finally, we also introduced the notion of \textit{forward} and \textit{backward data reuse}.
These were defined as the number of unique memory instructions that are reused.
For any given kernel $k_i$, we defined $M_i$ as the set of memory instructions used by $k_i$.
From this, we can compute both \textit{forward} ($fwd_i$) and \textit{backward data reuse} ($bwd_i$) as:
\begin{align}
    fwd_i &= \frac{|M_i \cap M_{i+1}|}{|M_i|} \\
    bwd_i &= \frac{|M_{i-1} \cap M_i|}{|M_i|}
\end{align}

\subsection{The cold-start problem in simulation}\label{subsec:cold-start-in-simulation}
From all workloads we have analyzed in the hardware, we have selected a subset to analyze in the Accel-Sim simulator.
The main ones we have analyzed were DCT, 3D U-Net, and OceanFFT\@.

From this analysis, we have concluded that the cold-start problem is less severe in the simulator.
The maximum impact of the cold-start problem on 3D-UNet drops from over 20\% to around 15\%.

This is probably due to the fact that the simulator is an idealized version of the hardware.
Additionally, some hardware details might have gotten lost in the conversion (especially if those details aren't publicly available).

\subsubsection{OceanFFT}\label{subsubsec:oceanfft}
The OceanFFT workload really stands out, compared to other workloads.
Strangely enough, the simulator shows that it runs faster when the caches are flushed; especially the second kernel.

Our main assumption is that the first kernel fills up the caches with (dirty) unused data.
When the second kernel starts, it needs to perform a lot of write-backs to DRAM, causing flooding in the memory hierarchy.
In turn, this stalls the processors, causing the second kernel to run slower.

This will prove problematic for one of our mitigations.

\section{Mitigation}\label{sec:mitigation}
The last chapter of this thesis, \Cref{ch:mitigation}, was dedicated to finding a mitigation for the cold-start problem.
We have identified three different avenues, two of which proved to be practical.
Finally, we have also compared their effectiveness.

The three avenues discussed were:
\begin{itemize}
    \item \textbf{Simulating preceding kernels in full:} This method involves ensuring a consistent, warm hardware state by simulating preceding kernels in full.
    However, this is quite infeasible, and goes against the idea of sampling (which is used to speed up simulation).
    \item \textbf{Memory-only simulation of preceding kernels:} We only simulate the memory instructions of preceding kernels.
    This is a very feasible and accurate method, partly thanks to memory locality.
    \item \textbf{Correction factor:} We have also proposed a correction factor, which is based on the data reuse of the kernels, among others.
    This removes the overhead of any additional simulation, at the cost of a slightly lower accuracy.
\end{itemize}

\subsection{Memory-only simulation}\label{subsec:memory-only-simulation}
From the DCT workload, we have selected four kernels to analyze.
For each of those kernels, we simulated the memory instructions of up to 10 previous kernels.
To be able to do this, we have modified Accel-Sim's builtin NVBit tool.

Our main conclusion from this analysis is that the memory-only simulation is very accurate.
Even simulating only a single additional kernel results in incredible accuracy.
This makes sense, because of the principle of memory locality.

\subsection{Correction factor}\label{subsec:correction-factor}
Our final avenue was the correction factor.
Using a selection of factors (like data reuse, forward data reuse, and backward data reuse), we have proposed a formula to correct the IPC values of kernels.

Our main goal was to compute an upper bound for the number of cycles lost due to the cold-start problem.
We approximated this number to be proportional to:
\begin{itemize}
    \item The number of (unique) DRAM requests made; this gives an idea of the number of cold misses.
    We scale this value by dividing it by the number of memory controllers used (and the cache line size).
    \item The forward data reuse factor of the previous kernel; this gives an idea of the fraction of cold misses due to the cold-start problem.
    \item The difference in latencies between LLC (L2) and DRAM\@.
\end{itemize}

With these observations, we came up with the following formula:
\begin{align}
    IPC &= \frac{instructions}{cycles_f - \Delta} \\
    \Delta &= \frac{accesses}{controllers \cdot line} \cdot fwd_{i-1} \cdot (DRAM - L2)
\end{align}

Where:
\begin{itemize}
    \item $IPC$ is the final, corrected IPC;
    \item $instructions$ is the number of instructions executed by the kernel;
    \item $cycles_f$ is the number of cycles as reported by the simulator (in cold-start mode);
    \item $\Delta$ is the correction factor;
    \item $accesses$ is the number of unique DRAM accesses made by the kernel;
    \item $controllers$ is the number of memory controllers used\footnote{We count memory controllers that are used less than others, as partial memory controllers; see \Cref{subsubsec:mem-ctrl}.};
    \item $line$ is the cache line size;
    \item $fwd_{i-1}$ is the forward data reuse factor of the \textit{previous} kernel; and
    \item $DRAM$ and $L2$ are the DRAM and L2 cache latencies, respectively.
\end{itemize}

We found that this formula does manage to eliminate the (lower) outliers in IPC values, at the cost of a slightly lower median accuracy.

\subsection{Comparison}\label{subsec:comparison}
In the final section, \Cref{sec:sim-vs-corr}, we have compared the two methods.
We have found that the memory-only simulation is the most accurate.
A strange result here is that full memory simulation is less accurate than simulating just a single kernel.
Since the numbers are quite close together, we assume that this is largely due to non-determinism in the simulator.