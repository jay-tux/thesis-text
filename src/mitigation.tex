\chapter{Mitigation}\label{ch:mitigation}

So far, we have talked about the cold-start problem in both hardware and simulation.
The reason why we have looked into it, is because of how we approach simulating large workloads in GPUs.
Instead of simulating each and every kernel, we only simulate a select subset.
Usually, these are selected using both profiling and some form of clustering based on characteristics.
However, since we only simulate intermittent kernels, cache state might be lost and/or incorrect; which leads to inaccuracies.

Taking this all into account means that we see a few options for mitigation:
\begin{itemize}
    \item \textbf{Simulate preceding kernels:} this is the most naive, but also the most computationally intensive approach.
    Simulating all instructions from one or more kernels that come right before the one we need ensures that the cache state is as close to the real state as possible, but also increases the simulation time.
    \item \textbf{Simulate certain instructions from preceding kernels:} this option is more refined than the previous, finding a balance between computation time and accuracy.
    Since we have the full trace of each kernel, we can select only the memory instructions from the preceding kernels, simulating those to warm up the caches artificially.
    Especially when kernels contain a lot of computations, and fewer memory loads/stores, this could allow for a higher accuracy at a very low cost.
    \item \textbf{Compute a correction factor:} logic dictates that there should be a way to compute a correction factor for the cold-start problem.
    There are different factors that we could analyze ahead-of-time to compute this factor.
\end{itemize}

In \cref{sec:trace-info,sec:kernel-selection}, we will focus on the second option, opting to simulate memory instructions in order to artificially warm up the caches.
\Cref{sec:correction-factor} will focus on the third option, offering some possible factors that could be used to compute a correction factor.

\FloatBarrier
\section{Gathering trace info}\label{sec:trace-info}
The AccelSim~\cite{accelsim} tool comes with an NVBit~\cite{nvbit} tool and post-processor which are already used to generate the instruction-level traces that are eventually fed to the simulator itself.
By default, this tool generates a number of files:
\begin{itemize}
    \item \verb|traces/kernelslist|: a list of all kernels that were run, with their respective kernel IDs.
    \item \verb|traces/stats.csv|: a CSV file containing some global statistics about each kernel.
    \item \verb|traces/kernel-<number>.trace|: the actual instruction-level trace for each kernel, identified by their numbers.
\end{itemize}

We have expanded this tool to output an additional file for each kernel: \verb|traces/kernel-mem-<number>.trace|.
This file contains all memory instructions issued by the selected kernel, as well as the final \verb|EXIT| instruction (omitting this one would lead to a segmentation fault in the simulator).
The original NVBIT tool reports for each instruction it instruments whether it is a memory instruction or not.
We use that information to filter out the instructions we need.

As with the original traces, we need to post-process each memory trace before feeding it to the simulator.
By using the same format as the original tool, we ensure that the existing post-processor can also handle these new files\footnote{
    \textit{Note:} AccelSim by default uses a compressed format for memory instructions, taking up less disk space in their traces.
    However, due to time constraints, we have elected to work with the uncompressed format, which takes up more disk space.
}.

\FloatBarrier
\section{Kernel selection}\label{sec:kernel-selection}
In order to warm up the simulator's caches, we will be simulating the memory instructions from preceding kernels.
However, this is once again a trade-off: the more kernels we use to warm up the caches, the more accurate the simulation will be, but the longer it will take.
We have selected four kernels from the DCT8x8 workload with high IPC differences, as shown in \cref{fig:dct_kernels}.

\begin{figure}
    \centering
    \appendW{figures/simulated/dct_kernels}{}
    \caption{High IPC difference kernels in DCT}
    \label{fig:dct_kernels}
\end{figure}

For each of these kernels, we used multiple simulations:
\begin{itemize}
    \item \textbf{Perfect warmup:} the kernel is simulated in order, with all preceding kernels simulated in full.
    \item \textbf{Full memory warmup:} the kernel is simulated in order, with all memory instructions preceding it simulated (for kernel $i$, this means all memory instructions from kernels $1$ until $i - 1$ are simulated).
    \item \textbf{Partial memory warmup:} the memory instructions of up to 10 preceding kernels are simulated.
\end{itemize}

\begin{figure}
    \centering
    \appendW{./figures/simulated/kernel-11-abs.pgf}{0.47}
    \appendW{./figures/simulated/kernel-13-abs.pgf}{0.47}
    \appendW{./figures/simulated/kernel-114-abs.pgf}{0.47}
    \appendW{./figures/simulated/kernel-115-abs.pgf}{0.47}
    \append{./figures/simulated/mitigation_absolute}
    \caption{DCT Mitigation results (absolute IPC values)}
    \label{fig:mitig_abs}
\end{figure}

\begin{figure}
    \centering
    \appendW{./figures/simulated/kernel-11-acc.pgf}{0.47}
    \appendW{./figures/simulated/kernel-13-acc.pgf}{0.47}
    \appendW{./figures/simulated/kernel-114-acc.pgf}{0.47}
    \appendW{./figures/simulated/kernel-115-acc.pgf}{0.47}
    \append{./figures/simulated/mitigation_accuracy}
    \caption{DCT Mitigation results (accuracy)}
    \label{fig:mitig_acc}
\end{figure}

In \cref{fig:mitig_abs,fig:mitig_acc}, we have plotted the results of these simulations.
For each kernel, we have plotted the result of \textit{perfect warmup} as a black line, this was the golden reference we tried to reach.
Additionally, the light-brown line represents \textit{full memory warmup}, while the blue line represents \textit{partial memory warmup}.
Both raw IPC values and accuracy (in percents) are shown.

From these figures, we can quickly deduce that even a single preceding kernel can lead to drastic accuracy increases: most kernels reach over 99\% accuracy with just one kernel warmed up.

An additional striking detail is that more kernels does not always imply a higher accuracy.
This might be due to some non-determinism in the simulator.

\FloatBarrier

\section{Correction factor}\label{sec:correction-factor}

Since the cold-start problem is inherently tied to the cache state, we should look at factors that can impact this state in order to find a correction factor.
Tools like NVIDIA's Nsight Compute~\cite{nsight} and NVBit~\cite{nvbit} can provide detailed cache statistics, which might be used to compute this factor.
Some possible factors are:
\begin{itemize}
    \item \textit{Profiler differences:} by profiling a workload twice, once with and once without flushing, one can see the impact of the cold-start problem in silicon.
    This data might be used to improve simulation results.
    \item \textit{Profiler cache statistics:} a detailed profiler can provide statistics about hits and misses, which in turn could be used to compute the correction factor.
    \item \textit{Degree of data reuse:} by using an instruction-level trace, it is possible to extract all memory references.
    The ratio $\frac{\text{total memory references}}{\text{unique memory references}}$ could inform us about the cache state.
    We have outlined the main ideas of \textit{forward data reuse} in \cref{sec:data-reuse} (that section also includes the notion of \textit{backward data reuse}, but that will prove not as useful here).
    \item \textit{The DRAM delay:} the difference in cycles between a cache hit and a DRAM access could be useful to get an idea of the amount of cycles lost due to cold-start.
    \item \textit{The number of memory controllers:} since each memory controller can issue requests in parallel, the number of controllers might impact the cache state, as well as the cost of a cache miss.
    \item \textit{L2 miss rate:} this value could be used together with the \textit{number of memory instructions}.
\end{itemize}

\begin{figure}[hb]
    \centering
    \begin{tabular}{|c|c|c|c|c|c|c|}
  \hline
    index & memory fraction & memory footprint & accuracy & L2 miss rate & forward reuse & backward reuse \\
  \hline
  \hline
    1 & 20.25\% & 32814 & 99.19\% & 100.00\% & 0.00\% & 100.00\% \\
    2 & 20.25\% & 32814 & 99.62\% & 75.00\% & 100.00\% & 100.00\% \\
    3 & 20.25\% & 32814 & 99.78\% & 66.67\% & 100.00\% & 100.00\% \\
  \hline
  \hline
    10 & 20.25\% & 32814 & 99.63\% & 55.00\% & 100.00\% & 100.00\% \\
    \textit{11} & \textit{8.04\%} & \textit{32768} & \textit{88.17\%} & \textit{50.05\%} & \textit{100.00\%} & \textit{100.00\%} \\
    12 & 21.92\% & 32814 & 99.80\% & 50.04\% & 100.00\% & 12.50\% \\
    \textit{13} & \textit{33.16\%} & \textit{65632} & \textit{89.15\%} & \textit{53.38\%} & \textit{25.00\%} & \textit{100.00\%} \\
    14 & 33.16\% & 65632 & 98.54\% & 50.07\% & 100.00\% & 100.00\% \\
    15 & 33.16\% & 65632 & 94.06\% & 47.43\% & 100.00\% & 100.00\% \\
  \hline
  \hline
    113 & 33.16\% & 65632 & 99.45\% & 26.98\% & 100.00\% & 25.00\% \\
    \textit{114} & \textit{6.45\%} & \textit{32768} & \textit{90.62\%} & \textit{26.96\%} & \textit{12.50\%} & \textit{12.50\%} \\
    \textit{115} & \textit{33.16\%} & \textit{65632} & \textit{86.92\%} & \textit{26.94\%} & \textit{25.00\%} & \textit{50.00\%} \\
    116 & 21.08\% & 32944 & 95.28\% & 26.94\% & 25.00\% & 12.50\% \\
    117 & 5.93\% & 16384 & 97.66\% & 26.81\% & 6.25\% & 6.25\% \\
    118 & 20.35\% & 32944 & 92.77\% & 26.81\% & 12.50\% & 0.00\% \\
  \hline
\end{tabular}

    \caption{DCT mitigation statistics}
    \label{fig:dct_mitig}
    \begin{tabular}{|c|c|c|c|c|c|c|}
  \hline
    index & memory fraction & memory footprint & accuracy & L2 miss rate & forward reuse & backward reuse \\
  \hline
  \hline
    1 & 2.45\% & 33564671 & 99.72\% & 85.87\% & 0.00\% & 25.00\% \\
    \textit{2} & \textit{16.03\%} & \textit{16779666} & \textit{54.23\%} & \textit{34.78\%} & \textit{12.50\%} & \textit{25.00\%} \\
    3 & 15.45\% & 16780062 & 99.55\% & 44.33\% & 25.00\% & 0.00\% \\
    4 & 13.64\% & 16777216 & 99.99\% & 49.88\% & 0.00\% & 18.18\% \\
    5 & 16.21\% & 46137344 & 100.00\% & 51.06\% & 50.00\% & 0.00\% \\
  \hline
\end{tabular}

    \caption{OceanFFT mitigation statistics}
    \label{fig:ocean_mitig}
\end{figure}

We have gathered some of these factors in \cref{fig:dct_mitig} for a subset of the DCT workload.
The kernel invocations for which a mitigation is required (kernels 11, 13, 114, and 115) are emphasized.
In \cref{fig:ocean_mitig}, we have done the same for the OceanFFT workload (with its single high-impact kernel marked).

However, we have to be careful when finding a correction factor.
Since some factors that could impact cache state (e.g.\ the number of cache evictions) are hardware- and platform-dependent, we could end up with a correction factor that only works for a specific platform.

\FloatBarrier
\section{Conclusion}\label{sec:mitigation-conclusion}

We have looked into three possible mitigations, and analyzed their strengths and weaknesses.
First, there is the naive approach of simulating all preceding kernels in full, guaranteeing a correct warmup, and thus a ``perfectly accurate'' simulation.
However, this has the drawback of being incredibly computationally intensive, negating the work done by sampling kernels in the first place.

Secondly, we have looked into simulating only parts of the previous kernels.
In this case, we have selected the memory instructions of preceding kernels and simulated those to artificially warm up the caches.
Even for kernels that suffered a lot from the cold-start problem, we have shown that warming up using a single kernel has a drastic impact on the accuracy already.
In many cases, a single kernel was enough to reach over 99\% accuracy.

Finally, we have looked into possible correction factors.
Some possible factors that can be used to create a comprehensive function to compute a correction factor have been outlined.
These factors range from L2 miss rate and DRAM delay to higher-level statistics about inter-kernel data reusability.
However, no conclusive results were found.
We also noted that it might be possible to over-tune a solution like this.
Computing a highly accurate correction factor like this might work perfectly on one platform but reduce accuracy on another, missing the point of GPU simulation entirely.