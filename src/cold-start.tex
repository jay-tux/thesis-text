\chapter{The Cold-Start Problem}\label{ch:the-cold-start-problem}
\section{Caches}\label{sec:caches}
The cold-start problem is inherently tied to caches.
Caches are a hardware component that is used to improve the performance of a computer system.
They attempt to mitigate the latency of accessing data from main memory (DRAM) by storing a subset of the data that is recently accessed.
Older research has shown that many applications exhibit temporal locality, meaning that if a piece of data is accessed once, it is likely to be accessed again in the near future.
Caches exploit this property by storing recently accessed data in a small, fast memory that is closer to the processor (in this case, the GPU) than the main memory.

When the GPU issues a memory request (often due to a load or store instruction), the cache controller checks if the data is present in the cache.
If it is, and the data is not stale, the cache controller returns the data to the GPU, avoiding the expensive DRAM access.
If the data is not present in the cache, the cache controller must fetch the data from the main memory.
Modern GPUs have multiple levels of cache, each with their own size, latency, and other parameters.
Often, the caches are organized in a hierarchy, where closer caches (which have a lower latency), are smaller.
We will mainly focus on the last-level cache (LLC), which is the cache closest to the DRAM\@.

\subsection{Cold Caches}\label{subsec:cold-caches}
Initially, the caches of a GPU are cold, meaning that they do not contain any data.
This means that the first time a new piece of data is accessed, it will not be present in the cache, and the GPU must fetch it from DRAM\@.
This causes a significant latency penalty, as the latency of accessing DRAM is much higher than the latency of accessing the LLC\@.
From then on, repeat accesses to the same data will be much faster, as the data will be present in the cache.

The benefit of caches is that they can persist data between kernel invocations.
This implies that a kernel can ``warm up'' the caches by accessing all data that the next kernel might need.

\begin{figure}[hbt]
    \centering
    \append{figures/coldstart/normal_exec}
    \caption{Caches example}
    \label{fig:normal-exec}
\end{figure}

In~\Cref{fig:normal-exec}, we see an example of two kernels, and how caching can improve performance.
This small example abstracts away of how the caches decide which cache line is used for which memory address.
The first kernel executes some memory instructions, which the second one builds upon.
The full ``code'' executed (we used a simplified representation of the memory instructions) is as follows:
\begin{minted}[linenos]{nasm}
    ; Kernel 1:
    ld 0x00AA ; performs a load
    ld 0x11BB
    st 0x00AA ; performs a store
    st 0x22CC
    ld 0x33DD

    ; Kernel 2:
    ld 0x33DD
    ld 0x11BB
    st 0x00AA
\end{minted}

The figure shows for each instruction the cache state before and after the execution (if the cache state changed).
We can clearly see that, for kernel 1, the caches start out cold, but get warmed up by the unique memory accesses (\verb|0x00AA|, \verb|0x11BB|, \verb|0x22CC|, and \verb|0x33DD|).
Because the cache couldn't supply us with the data, we call this a cache miss.
More specifically, since we access a memory location that had not been accessed before, it is called a cold miss.
Each cache miss causes a DRAM access, costing us precious cycles.
However, the repeat access of \verb|0x00AA| on line 4 (\mintinline{nasm}{st 0x00AA}) goes only through the cache, speeding up the access.

Because we do not flush the caches between two kernel invocations in this example, kernel 2 can benefit from the cache state that kernel 1 left behind.
At this point, we have no cold misses anymore, as all memory locations that are accessed have been accessed before.
However, line 10 causes another miss, as the cache line associated with \verb|0x33DD| was overwritten by the access of \verb|0x11BB| on line 9, causing a conflict miss.
These conflict misses occur when multiple memory locations are mapped to the same cache line, and one of them is evicted from the cache.

These evictions are a normal part of cache operation.
The controller will choose a line to remove from the cache based on its eviction policy.
Some common cache eviction policies are\footnote{Cache eviction policies only have an effect when a memory location can be placed in one of multiple cache lines (the cache has an associativity of at least 2). If a memory location can be placed in only one cache line, it has to replace that specific cache line.}:
\begin{description}
    \item[LRU (Least Recently Used)] The line that has not been accessed for the longest time is evicted.
    This choice is based on the assumption that, if a line has not been accessed for a long time, it is less likely to be accessed in the near future.
    \item[LFU (Least Frequently Used)] The line that has been accessed the least amount of times is evicted.
    \item[FIFO (First In, First Out)] The line that has been in the cache the longest is evicted.
    \item[RR (Random Replacement)] A random line is evicted.
\end{description}

For simplicity, we will assume that all instructions are run serially, i.e.\ no two instructions are issued at the same time.
Additionally, we will assume that a cache hit takes 180 cycles, and a DRAM access takes 250 cycles (which is close to what the simulator will use).
From these numbers we can compute how many cycles each kernel will take to execute.
For kernel 1, we have 4 misses and 1 hit, resulting in a total of 1180 cycles; while kernel 2 can be executed in 610 cycles.

\section{The Problem}\label{sec:the-problem}
As we discussed in \Cref{sec:kernel-sampling}, most kernels in a workload won't be simulated.
This means that a lot of intermittent kernels will never be run, which might impact the cache states.
In effect, we can assume that each kernel will be run with cold caches, as the kernels right before a kernel might not have been run.

\begin{figure}[hbt]
    \centering
    \append{figures/coldstart/cold_start_exec}
    \caption{Sampled example}
    \label{fig:flush-exec}
\end{figure}

In~\Cref{fig:flush-exec}, we revisit the same example, assuming that the sampling method only selected kernel 2 to be simulated.
Below is the code excerpt for kernel 2:
\begin{minted}[linenos,firstnumber=8]{nasm}
    ; Kernel 2:
    ld 0x33DD
    ld 0x11BB
    st 0x00AA
\end{minted}

This means that kernel 1 was not run, and kernel 2 can not benefit from the cache state that kernel 1 left behind.
This causes all memory accesses to be cold misses, as the caches are cold.
Again, we can compute the number of cycles kernel 2 will take to execute.
This time, we have 3 misses, resulting in a total of 750 cycles (taking into account our assumptions from before).
This is a significant increase in execution time compared to the 610 cycles we had before (about 23\%).