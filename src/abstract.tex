\thispagestyle{plain}
\begin{center}
    \textbf{Abstract}
\end{center}

Many modern applications and workloads depend on GPUs, ranging from gaming and graphics to machine learning and graph analysis.
To support this ever-growing need for high-performance parallel processing, constant innovation in GPUs and their architectures is a must.
These innovations require methods to verify that changes made are beneficial for real workloads.
However, it would be costly and impractical to verify in silicon, so simulators like AccelSim\cite{accelsim} are used.
The computational overhead of the simulation implies that only a subset of all kernels can be simulated within reasonable time.
To remedy this, techniques like PKS\cite{pks} and Sieve\cite{sieve} are used.
After profiling a workload, they use sampling to select certain kernels to be simulated, speeding up the time until results.
In my thesis, I aim to show you that these approaches are imperfect, and provide a mitigation for the issue.
