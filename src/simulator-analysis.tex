\chapter{The Cold-Start Problem in AccelSim}\label{ch:sim-analysis}

\section{Simulation results}\label{sec:simulation-results}
The initial in-silicon analysis proved that the cold-start problem is, in fact, present in GPGPU hardware.
In this chapter, we'll show that the impact of this problem persists in the AccelSim simulator.
However, one of the first things we've noticed is that the simulator is affected much differently from the real hardware.

Due to the inherent overhead in simulating GPUs, we had to limit the number of kernels we could simulate, as well as which workloads we could focus on.
The DCT and 3D U-Net workloads were selected for this analysis, as they showed promising results in the hardware analysis.
Following our discussion about inter-kernel data reuse, we also included the OceanFFT workload (from the CUDA SDK) in this analysis.

We set up AccelSim to use the existing, tested configurations for the NVIDIA GeForce RTX 3070 hardware (SM 86).
The basic configuration parameters are shown in \cref{fig:sim_config}.
Again, each workload was run twice: once with the cache flushed between kernel invocations, and once without.
AccelSim natively supports flushing caches using the \verb|-flush-l1 -flush-l2| arguments.

\begin{figure}[ht]
    \centering
    \begin{tabular}{|l|c|}
    \hline
    \textbf{Configuration parameter} & \textbf{Value} \\
    \hline
    \hline
    L2 cache size & 4 MB \\
    Number of sets in L2 cache & 64 \\
    L2 cache block size & 128 B \\
    L2 cache associativity & 16 \\
    Number of memory controllers & 16 \\
    Number of SMs & 46 \\
    L2 Latency & 187 cycles \\
    DRAM Latency & 254 cycles \\
    \hline
\end{tabular}
    \caption{Simulator configuration parameters}
    \label{fig:sim_config}
\end{figure}

\begin{figure}[ht]
    \centering
    \begin{minipage}[c]{0.45\textwidth}
        \resizebox{\textwidth}{!}{\begin{tabular}{|c|c|c|c|c|}
\hline
  \textbf{Workload} & \textbf{5\% difference} & \textbf{10\% difference} & \textbf{15\% difference} & \textbf{20\% difference}\\
\hline
\hline
  oceanFFT & 21.69 \% & 21.69 \% & 21.69 \% & 21.69 \%\\
  dct8x8 & 5.84 \% & 3.75 \% & 0.69 \% & 0.00 \%\\
  3d-unet & 3.75 \% & 0.00 \% & 0.00 \% & 0.00 \%\\
\hline
\end{tabular}
}
    \end{minipage}
    \begin{minipage}[c]{0.45\textwidth}
        \resizebox{\textwidth}{!}{%% Creator: Matplotlib, PGF backend
%%
%% To include the figure in your LaTeX document, write
%%   \input{<filename>.pgf}
%%
%% Make sure the required packages are loaded in your preamble
%%   \usepackage{pgf}
%%
%% Also ensure that all the required font packages are loaded; for instance,
%% the lmodern package is sometimes necessary when using math font.
%%   \usepackage{lmodern}
%%
%% Figures using additional raster images can only be included by \input if
%% they are in the same directory as the main LaTeX file. For loading figures
%% from other directories you can use the `import` package
%%   \usepackage{import}
%%
%% and then include the figures with
%%   \import{<path to file>}{<filename>.pgf}
%%
%% Matplotlib used the following preamble
%%   \def\mathdefault#1{#1}
%%   \everymath=\expandafter{\the\everymath\displaystyle}
%%   
%%   \usepackage{fontspec}
%%   \setmainfont{DejaVuSerif.ttf}[Path=\detokenize{/home/data/ugent/thesis/4Jonas/lib/python3.11/site-packages/matplotlib/mpl-data/fonts/ttf/}]
%%   \setsansfont{DejaVuSans.ttf}[Path=\detokenize{/home/data/ugent/thesis/4Jonas/lib/python3.11/site-packages/matplotlib/mpl-data/fonts/ttf/}]
%%   \setmonofont{DejaVuSansMono.ttf}[Path=\detokenize{/home/data/ugent/thesis/4Jonas/lib/python3.11/site-packages/matplotlib/mpl-data/fonts/ttf/}]
%%   \makeatletter\@ifpackageloaded{underscore}{}{\usepackage[strings]{underscore}}\makeatother
%%
\begingroup%
\makeatletter%
\begin{pgfpicture}%
\pgfpathrectangle{\pgfpointorigin}{\pgfqpoint{6.400000in}{4.800000in}}%
\pgfusepath{use as bounding box, clip}%
\begin{pgfscope}%
\pgfsetbuttcap%
\pgfsetmiterjoin%
\definecolor{currentfill}{rgb}{1.000000,1.000000,1.000000}%
\pgfsetfillcolor{currentfill}%
\pgfsetlinewidth{0.000000pt}%
\definecolor{currentstroke}{rgb}{1.000000,1.000000,1.000000}%
\pgfsetstrokecolor{currentstroke}%
\pgfsetdash{}{0pt}%
\pgfpathmoveto{\pgfqpoint{0.000000in}{0.000000in}}%
\pgfpathlineto{\pgfqpoint{6.400000in}{0.000000in}}%
\pgfpathlineto{\pgfqpoint{6.400000in}{4.800000in}}%
\pgfpathlineto{\pgfqpoint{0.000000in}{4.800000in}}%
\pgfpathlineto{\pgfqpoint{0.000000in}{0.000000in}}%
\pgfpathclose%
\pgfusepath{fill}%
\end{pgfscope}%
\begin{pgfscope}%
\pgfsetbuttcap%
\pgfsetmiterjoin%
\definecolor{currentfill}{rgb}{1.000000,1.000000,1.000000}%
\pgfsetfillcolor{currentfill}%
\pgfsetlinewidth{0.000000pt}%
\definecolor{currentstroke}{rgb}{0.000000,0.000000,0.000000}%
\pgfsetstrokecolor{currentstroke}%
\pgfsetstrokeopacity{0.000000}%
\pgfsetdash{}{0pt}%
\pgfpathmoveto{\pgfqpoint{0.617715in}{0.881761in}}%
\pgfpathlineto{\pgfqpoint{6.235000in}{0.881761in}}%
\pgfpathlineto{\pgfqpoint{6.235000in}{4.412376in}}%
\pgfpathlineto{\pgfqpoint{0.617715in}{4.412376in}}%
\pgfpathlineto{\pgfqpoint{0.617715in}{0.881761in}}%
\pgfpathclose%
\pgfusepath{fill}%
\end{pgfscope}%
\begin{pgfscope}%
\pgfpathrectangle{\pgfqpoint{0.617715in}{0.881761in}}{\pgfqpoint{5.617285in}{3.530616in}}%
\pgfusepath{clip}%
\pgfsetbuttcap%
\pgfsetmiterjoin%
\definecolor{currentfill}{rgb}{0.121569,0.466667,0.705882}%
\pgfsetfillcolor{currentfill}%
\pgfsetlinewidth{0.000000pt}%
\definecolor{currentstroke}{rgb}{0.000000,0.000000,0.000000}%
\pgfsetstrokecolor{currentstroke}%
\pgfsetstrokeopacity{0.000000}%
\pgfsetdash{}{0pt}%
\pgfpathmoveto{\pgfqpoint{0.873046in}{0.881761in}}%
\pgfpathlineto{\pgfqpoint{1.237805in}{0.881761in}}%
\pgfpathlineto{\pgfqpoint{1.237805in}{4.244252in}}%
\pgfpathlineto{\pgfqpoint{0.873046in}{4.244252in}}%
\pgfpathlineto{\pgfqpoint{0.873046in}{0.881761in}}%
\pgfpathclose%
\pgfusepath{fill}%
\end{pgfscope}%
\begin{pgfscope}%
\pgfpathrectangle{\pgfqpoint{0.617715in}{0.881761in}}{\pgfqpoint{5.617285in}{3.530616in}}%
\pgfusepath{clip}%
\pgfsetbuttcap%
\pgfsetmiterjoin%
\definecolor{currentfill}{rgb}{0.121569,0.466667,0.705882}%
\pgfsetfillcolor{currentfill}%
\pgfsetlinewidth{0.000000pt}%
\definecolor{currentstroke}{rgb}{0.000000,0.000000,0.000000}%
\pgfsetstrokecolor{currentstroke}%
\pgfsetstrokeopacity{0.000000}%
\pgfsetdash{}{0pt}%
\pgfpathmoveto{\pgfqpoint{2.696840in}{0.881761in}}%
\pgfpathlineto{\pgfqpoint{3.061599in}{0.881761in}}%
\pgfpathlineto{\pgfqpoint{3.061599in}{1.786808in}}%
\pgfpathlineto{\pgfqpoint{2.696840in}{1.786808in}}%
\pgfpathlineto{\pgfqpoint{2.696840in}{0.881761in}}%
\pgfpathclose%
\pgfusepath{fill}%
\end{pgfscope}%
\begin{pgfscope}%
\pgfpathrectangle{\pgfqpoint{0.617715in}{0.881761in}}{\pgfqpoint{5.617285in}{3.530616in}}%
\pgfusepath{clip}%
\pgfsetbuttcap%
\pgfsetmiterjoin%
\definecolor{currentfill}{rgb}{0.121569,0.466667,0.705882}%
\pgfsetfillcolor{currentfill}%
\pgfsetlinewidth{0.000000pt}%
\definecolor{currentstroke}{rgb}{0.000000,0.000000,0.000000}%
\pgfsetstrokecolor{currentstroke}%
\pgfsetstrokeopacity{0.000000}%
\pgfsetdash{}{0pt}%
\pgfpathmoveto{\pgfqpoint{4.520634in}{0.881761in}}%
\pgfpathlineto{\pgfqpoint{4.885393in}{0.881761in}}%
\pgfpathlineto{\pgfqpoint{4.885393in}{1.463770in}}%
\pgfpathlineto{\pgfqpoint{4.520634in}{1.463770in}}%
\pgfpathlineto{\pgfqpoint{4.520634in}{0.881761in}}%
\pgfpathclose%
\pgfusepath{fill}%
\end{pgfscope}%
\begin{pgfscope}%
\pgfpathrectangle{\pgfqpoint{0.617715in}{0.881761in}}{\pgfqpoint{5.617285in}{3.530616in}}%
\pgfusepath{clip}%
\pgfsetbuttcap%
\pgfsetmiterjoin%
\definecolor{currentfill}{rgb}{1.000000,0.498039,0.054902}%
\pgfsetfillcolor{currentfill}%
\pgfsetlinewidth{0.000000pt}%
\definecolor{currentstroke}{rgb}{0.000000,0.000000,0.000000}%
\pgfsetstrokecolor{currentstroke}%
\pgfsetstrokeopacity{0.000000}%
\pgfsetdash{}{0pt}%
\pgfpathmoveto{\pgfqpoint{1.237805in}{0.881761in}}%
\pgfpathlineto{\pgfqpoint{1.602564in}{0.881761in}}%
\pgfpathlineto{\pgfqpoint{1.602564in}{4.244252in}}%
\pgfpathlineto{\pgfqpoint{1.237805in}{4.244252in}}%
\pgfpathlineto{\pgfqpoint{1.237805in}{0.881761in}}%
\pgfpathclose%
\pgfusepath{fill}%
\end{pgfscope}%
\begin{pgfscope}%
\pgfpathrectangle{\pgfqpoint{0.617715in}{0.881761in}}{\pgfqpoint{5.617285in}{3.530616in}}%
\pgfusepath{clip}%
\pgfsetbuttcap%
\pgfsetmiterjoin%
\definecolor{currentfill}{rgb}{1.000000,0.498039,0.054902}%
\pgfsetfillcolor{currentfill}%
\pgfsetlinewidth{0.000000pt}%
\definecolor{currentstroke}{rgb}{0.000000,0.000000,0.000000}%
\pgfsetstrokecolor{currentstroke}%
\pgfsetstrokeopacity{0.000000}%
\pgfsetdash{}{0pt}%
\pgfpathmoveto{\pgfqpoint{3.061599in}{0.881761in}}%
\pgfpathlineto{\pgfqpoint{3.426358in}{0.881761in}}%
\pgfpathlineto{\pgfqpoint{3.426358in}{1.462392in}}%
\pgfpathlineto{\pgfqpoint{3.061599in}{1.462392in}}%
\pgfpathlineto{\pgfqpoint{3.061599in}{0.881761in}}%
\pgfpathclose%
\pgfusepath{fill}%
\end{pgfscope}%
\begin{pgfscope}%
\pgfpathrectangle{\pgfqpoint{0.617715in}{0.881761in}}{\pgfqpoint{5.617285in}{3.530616in}}%
\pgfusepath{clip}%
\pgfsetbuttcap%
\pgfsetmiterjoin%
\definecolor{currentfill}{rgb}{1.000000,0.498039,0.054902}%
\pgfsetfillcolor{currentfill}%
\pgfsetlinewidth{0.000000pt}%
\definecolor{currentstroke}{rgb}{0.000000,0.000000,0.000000}%
\pgfsetstrokecolor{currentstroke}%
\pgfsetstrokeopacity{0.000000}%
\pgfsetdash{}{0pt}%
\pgfpathmoveto{\pgfqpoint{4.885393in}{0.881761in}}%
\pgfpathlineto{\pgfqpoint{5.250151in}{0.881761in}}%
\pgfpathlineto{\pgfqpoint{5.250151in}{0.881761in}}%
\pgfpathlineto{\pgfqpoint{4.885393in}{0.881761in}}%
\pgfpathlineto{\pgfqpoint{4.885393in}{0.881761in}}%
\pgfpathclose%
\pgfusepath{fill}%
\end{pgfscope}%
\begin{pgfscope}%
\pgfpathrectangle{\pgfqpoint{0.617715in}{0.881761in}}{\pgfqpoint{5.617285in}{3.530616in}}%
\pgfusepath{clip}%
\pgfsetbuttcap%
\pgfsetmiterjoin%
\definecolor{currentfill}{rgb}{0.172549,0.627451,0.172549}%
\pgfsetfillcolor{currentfill}%
\pgfsetlinewidth{0.000000pt}%
\definecolor{currentstroke}{rgb}{0.000000,0.000000,0.000000}%
\pgfsetstrokecolor{currentstroke}%
\pgfsetstrokeopacity{0.000000}%
\pgfsetdash{}{0pt}%
\pgfpathmoveto{\pgfqpoint{1.602564in}{0.881761in}}%
\pgfpathlineto{\pgfqpoint{1.967323in}{0.881761in}}%
\pgfpathlineto{\pgfqpoint{1.967323in}{4.244252in}}%
\pgfpathlineto{\pgfqpoint{1.602564in}{4.244252in}}%
\pgfpathlineto{\pgfqpoint{1.602564in}{0.881761in}}%
\pgfpathclose%
\pgfusepath{fill}%
\end{pgfscope}%
\begin{pgfscope}%
\pgfpathrectangle{\pgfqpoint{0.617715in}{0.881761in}}{\pgfqpoint{5.617285in}{3.530616in}}%
\pgfusepath{clip}%
\pgfsetbuttcap%
\pgfsetmiterjoin%
\definecolor{currentfill}{rgb}{0.172549,0.627451,0.172549}%
\pgfsetfillcolor{currentfill}%
\pgfsetlinewidth{0.000000pt}%
\definecolor{currentstroke}{rgb}{0.000000,0.000000,0.000000}%
\pgfsetstrokecolor{currentstroke}%
\pgfsetstrokeopacity{0.000000}%
\pgfsetdash{}{0pt}%
\pgfpathmoveto{\pgfqpoint{3.426358in}{0.881761in}}%
\pgfpathlineto{\pgfqpoint{3.791116in}{0.881761in}}%
\pgfpathlineto{\pgfqpoint{3.791116in}{0.988897in}}%
\pgfpathlineto{\pgfqpoint{3.426358in}{0.988897in}}%
\pgfpathlineto{\pgfqpoint{3.426358in}{0.881761in}}%
\pgfpathclose%
\pgfusepath{fill}%
\end{pgfscope}%
\begin{pgfscope}%
\pgfpathrectangle{\pgfqpoint{0.617715in}{0.881761in}}{\pgfqpoint{5.617285in}{3.530616in}}%
\pgfusepath{clip}%
\pgfsetbuttcap%
\pgfsetmiterjoin%
\definecolor{currentfill}{rgb}{0.172549,0.627451,0.172549}%
\pgfsetfillcolor{currentfill}%
\pgfsetlinewidth{0.000000pt}%
\definecolor{currentstroke}{rgb}{0.000000,0.000000,0.000000}%
\pgfsetstrokecolor{currentstroke}%
\pgfsetstrokeopacity{0.000000}%
\pgfsetdash{}{0pt}%
\pgfpathmoveto{\pgfqpoint{5.250151in}{0.881761in}}%
\pgfpathlineto{\pgfqpoint{5.614910in}{0.881761in}}%
\pgfpathlineto{\pgfqpoint{5.614910in}{0.881761in}}%
\pgfpathlineto{\pgfqpoint{5.250151in}{0.881761in}}%
\pgfpathlineto{\pgfqpoint{5.250151in}{0.881761in}}%
\pgfpathclose%
\pgfusepath{fill}%
\end{pgfscope}%
\begin{pgfscope}%
\pgfpathrectangle{\pgfqpoint{0.617715in}{0.881761in}}{\pgfqpoint{5.617285in}{3.530616in}}%
\pgfusepath{clip}%
\pgfsetbuttcap%
\pgfsetmiterjoin%
\definecolor{currentfill}{rgb}{0.839216,0.152941,0.156863}%
\pgfsetfillcolor{currentfill}%
\pgfsetlinewidth{0.000000pt}%
\definecolor{currentstroke}{rgb}{0.000000,0.000000,0.000000}%
\pgfsetstrokecolor{currentstroke}%
\pgfsetstrokeopacity{0.000000}%
\pgfsetdash{}{0pt}%
\pgfpathmoveto{\pgfqpoint{1.967323in}{0.881761in}}%
\pgfpathlineto{\pgfqpoint{2.332081in}{0.881761in}}%
\pgfpathlineto{\pgfqpoint{2.332081in}{4.244252in}}%
\pgfpathlineto{\pgfqpoint{1.967323in}{4.244252in}}%
\pgfpathlineto{\pgfqpoint{1.967323in}{0.881761in}}%
\pgfpathclose%
\pgfusepath{fill}%
\end{pgfscope}%
\begin{pgfscope}%
\pgfpathrectangle{\pgfqpoint{0.617715in}{0.881761in}}{\pgfqpoint{5.617285in}{3.530616in}}%
\pgfusepath{clip}%
\pgfsetbuttcap%
\pgfsetmiterjoin%
\definecolor{currentfill}{rgb}{0.839216,0.152941,0.156863}%
\pgfsetfillcolor{currentfill}%
\pgfsetlinewidth{0.000000pt}%
\definecolor{currentstroke}{rgb}{0.000000,0.000000,0.000000}%
\pgfsetstrokecolor{currentstroke}%
\pgfsetstrokeopacity{0.000000}%
\pgfsetdash{}{0pt}%
\pgfpathmoveto{\pgfqpoint{3.791116in}{0.881761in}}%
\pgfpathlineto{\pgfqpoint{4.155875in}{0.881761in}}%
\pgfpathlineto{\pgfqpoint{4.155875in}{0.881761in}}%
\pgfpathlineto{\pgfqpoint{3.791116in}{0.881761in}}%
\pgfpathlineto{\pgfqpoint{3.791116in}{0.881761in}}%
\pgfpathclose%
\pgfusepath{fill}%
\end{pgfscope}%
\begin{pgfscope}%
\pgfpathrectangle{\pgfqpoint{0.617715in}{0.881761in}}{\pgfqpoint{5.617285in}{3.530616in}}%
\pgfusepath{clip}%
\pgfsetbuttcap%
\pgfsetmiterjoin%
\definecolor{currentfill}{rgb}{0.839216,0.152941,0.156863}%
\pgfsetfillcolor{currentfill}%
\pgfsetlinewidth{0.000000pt}%
\definecolor{currentstroke}{rgb}{0.000000,0.000000,0.000000}%
\pgfsetstrokecolor{currentstroke}%
\pgfsetstrokeopacity{0.000000}%
\pgfsetdash{}{0pt}%
\pgfpathmoveto{\pgfqpoint{5.614910in}{0.881761in}}%
\pgfpathlineto{\pgfqpoint{5.979669in}{0.881761in}}%
\pgfpathlineto{\pgfqpoint{5.979669in}{0.881761in}}%
\pgfpathlineto{\pgfqpoint{5.614910in}{0.881761in}}%
\pgfpathlineto{\pgfqpoint{5.614910in}{0.881761in}}%
\pgfpathclose%
\pgfusepath{fill}%
\end{pgfscope}%
\begin{pgfscope}%
\pgfsetbuttcap%
\pgfsetroundjoin%
\definecolor{currentfill}{rgb}{0.000000,0.000000,0.000000}%
\pgfsetfillcolor{currentfill}%
\pgfsetlinewidth{0.803000pt}%
\definecolor{currentstroke}{rgb}{0.000000,0.000000,0.000000}%
\pgfsetstrokecolor{currentstroke}%
\pgfsetdash{}{0pt}%
\pgfsys@defobject{currentmarker}{\pgfqpoint{0.000000in}{-0.048611in}}{\pgfqpoint{0.000000in}{0.000000in}}{%
\pgfpathmoveto{\pgfqpoint{0.000000in}{0.000000in}}%
\pgfpathlineto{\pgfqpoint{0.000000in}{-0.048611in}}%
\pgfusepath{stroke,fill}%
}%
\begin{pgfscope}%
\pgfsys@transformshift{1.420185in}{0.881761in}%
\pgfsys@useobject{currentmarker}{}%
\end{pgfscope}%
\end{pgfscope}%
\begin{pgfscope}%
\definecolor{textcolor}{rgb}{0.000000,0.000000,0.000000}%
\pgfsetstrokecolor{textcolor}%
\pgfsetfillcolor{textcolor}%
\pgftext[x=1.192493in, y=0.187471in, left, base,rotate=45.000000]{\color{textcolor}{\sffamily\fontsize{11.000000}{13.200000}\selectfont\catcode`\^=\active\def^{\ifmmode\sp\else\^{}\fi}\catcode`\%=\active\def%{\%}oceanFFT}}%
\end{pgfscope}%
\begin{pgfscope}%
\pgfsetbuttcap%
\pgfsetroundjoin%
\definecolor{currentfill}{rgb}{0.000000,0.000000,0.000000}%
\pgfsetfillcolor{currentfill}%
\pgfsetlinewidth{0.803000pt}%
\definecolor{currentstroke}{rgb}{0.000000,0.000000,0.000000}%
\pgfsetstrokecolor{currentstroke}%
\pgfsetdash{}{0pt}%
\pgfsys@defobject{currentmarker}{\pgfqpoint{0.000000in}{-0.048611in}}{\pgfqpoint{0.000000in}{0.000000in}}{%
\pgfpathmoveto{\pgfqpoint{0.000000in}{0.000000in}}%
\pgfpathlineto{\pgfqpoint{0.000000in}{-0.048611in}}%
\pgfusepath{stroke,fill}%
}%
\begin{pgfscope}%
\pgfsys@transformshift{3.243978in}{0.881761in}%
\pgfsys@useobject{currentmarker}{}%
\end{pgfscope}%
\end{pgfscope}%
\begin{pgfscope}%
\definecolor{textcolor}{rgb}{0.000000,0.000000,0.000000}%
\pgfsetstrokecolor{textcolor}%
\pgfsetfillcolor{textcolor}%
\pgftext[x=3.087920in, y=0.330738in, left, base,rotate=45.000000]{\color{textcolor}{\sffamily\fontsize{11.000000}{13.200000}\selectfont\catcode`\^=\active\def^{\ifmmode\sp\else\^{}\fi}\catcode`\%=\active\def%{\%}dct8x8}}%
\end{pgfscope}%
\begin{pgfscope}%
\pgfsetbuttcap%
\pgfsetroundjoin%
\definecolor{currentfill}{rgb}{0.000000,0.000000,0.000000}%
\pgfsetfillcolor{currentfill}%
\pgfsetlinewidth{0.803000pt}%
\definecolor{currentstroke}{rgb}{0.000000,0.000000,0.000000}%
\pgfsetstrokecolor{currentstroke}%
\pgfsetdash{}{0pt}%
\pgfsys@defobject{currentmarker}{\pgfqpoint{0.000000in}{-0.048611in}}{\pgfqpoint{0.000000in}{0.000000in}}{%
\pgfpathmoveto{\pgfqpoint{0.000000in}{0.000000in}}%
\pgfpathlineto{\pgfqpoint{0.000000in}{-0.048611in}}%
\pgfusepath{stroke,fill}%
}%
\begin{pgfscope}%
\pgfsys@transformshift{5.067772in}{0.881761in}%
\pgfsys@useobject{currentmarker}{}%
\end{pgfscope}%
\end{pgfscope}%
\begin{pgfscope}%
\definecolor{textcolor}{rgb}{0.000000,0.000000,0.000000}%
\pgfsetstrokecolor{textcolor}%
\pgfsetfillcolor{textcolor}%
\pgftext[x=4.886552in, y=0.280415in, left, base,rotate=45.000000]{\color{textcolor}{\sffamily\fontsize{11.000000}{13.200000}\selectfont\catcode`\^=\active\def^{\ifmmode\sp\else\^{}\fi}\catcode`\%=\active\def%{\%}3d-unet}}%
\end{pgfscope}%
\begin{pgfscope}%
\pgfsetbuttcap%
\pgfsetroundjoin%
\definecolor{currentfill}{rgb}{0.000000,0.000000,0.000000}%
\pgfsetfillcolor{currentfill}%
\pgfsetlinewidth{0.803000pt}%
\definecolor{currentstroke}{rgb}{0.000000,0.000000,0.000000}%
\pgfsetstrokecolor{currentstroke}%
\pgfsetdash{}{0pt}%
\pgfsys@defobject{currentmarker}{\pgfqpoint{-0.048611in}{0.000000in}}{\pgfqpoint{-0.000000in}{0.000000in}}{%
\pgfpathmoveto{\pgfqpoint{-0.000000in}{0.000000in}}%
\pgfpathlineto{\pgfqpoint{-0.048611in}{0.000000in}}%
\pgfusepath{stroke,fill}%
}%
\begin{pgfscope}%
\pgfsys@transformshift{0.617715in}{0.881761in}%
\pgfsys@useobject{currentmarker}{}%
\end{pgfscope}%
\end{pgfscope}%
\begin{pgfscope}%
\definecolor{textcolor}{rgb}{0.000000,0.000000,0.000000}%
\pgfsetstrokecolor{textcolor}%
\pgfsetfillcolor{textcolor}%
\pgftext[x=0.444451in, y=0.823723in, left, base]{\color{textcolor}{\sffamily\fontsize{11.000000}{13.200000}\selectfont\catcode`\^=\active\def^{\ifmmode\sp\else\^{}\fi}\catcode`\%=\active\def%{\%}$\mathdefault{0}$}}%
\end{pgfscope}%
\begin{pgfscope}%
\pgfsetbuttcap%
\pgfsetroundjoin%
\definecolor{currentfill}{rgb}{0.000000,0.000000,0.000000}%
\pgfsetfillcolor{currentfill}%
\pgfsetlinewidth{0.803000pt}%
\definecolor{currentstroke}{rgb}{0.000000,0.000000,0.000000}%
\pgfsetstrokecolor{currentstroke}%
\pgfsetdash{}{0pt}%
\pgfsys@defobject{currentmarker}{\pgfqpoint{-0.048611in}{0.000000in}}{\pgfqpoint{-0.000000in}{0.000000in}}{%
\pgfpathmoveto{\pgfqpoint{-0.000000in}{0.000000in}}%
\pgfpathlineto{\pgfqpoint{-0.048611in}{0.000000in}}%
\pgfusepath{stroke,fill}%
}%
\begin{pgfscope}%
\pgfsys@transformshift{0.617715in}{1.657059in}%
\pgfsys@useobject{currentmarker}{}%
\end{pgfscope}%
\end{pgfscope}%
\begin{pgfscope}%
\definecolor{textcolor}{rgb}{0.000000,0.000000,0.000000}%
\pgfsetstrokecolor{textcolor}%
\pgfsetfillcolor{textcolor}%
\pgftext[x=0.444451in, y=1.599022in, left, base]{\color{textcolor}{\sffamily\fontsize{11.000000}{13.200000}\selectfont\catcode`\^=\active\def^{\ifmmode\sp\else\^{}\fi}\catcode`\%=\active\def%{\%}$\mathdefault{5}$}}%
\end{pgfscope}%
\begin{pgfscope}%
\pgfsetbuttcap%
\pgfsetroundjoin%
\definecolor{currentfill}{rgb}{0.000000,0.000000,0.000000}%
\pgfsetfillcolor{currentfill}%
\pgfsetlinewidth{0.803000pt}%
\definecolor{currentstroke}{rgb}{0.000000,0.000000,0.000000}%
\pgfsetstrokecolor{currentstroke}%
\pgfsetdash{}{0pt}%
\pgfsys@defobject{currentmarker}{\pgfqpoint{-0.048611in}{0.000000in}}{\pgfqpoint{-0.000000in}{0.000000in}}{%
\pgfpathmoveto{\pgfqpoint{-0.000000in}{0.000000in}}%
\pgfpathlineto{\pgfqpoint{-0.048611in}{0.000000in}}%
\pgfusepath{stroke,fill}%
}%
\begin{pgfscope}%
\pgfsys@transformshift{0.617715in}{2.432358in}%
\pgfsys@useobject{currentmarker}{}%
\end{pgfscope}%
\end{pgfscope}%
\begin{pgfscope}%
\definecolor{textcolor}{rgb}{0.000000,0.000000,0.000000}%
\pgfsetstrokecolor{textcolor}%
\pgfsetfillcolor{textcolor}%
\pgftext[x=0.368410in, y=2.374321in, left, base]{\color{textcolor}{\sffamily\fontsize{11.000000}{13.200000}\selectfont\catcode`\^=\active\def^{\ifmmode\sp\else\^{}\fi}\catcode`\%=\active\def%{\%}$\mathdefault{10}$}}%
\end{pgfscope}%
\begin{pgfscope}%
\pgfsetbuttcap%
\pgfsetroundjoin%
\definecolor{currentfill}{rgb}{0.000000,0.000000,0.000000}%
\pgfsetfillcolor{currentfill}%
\pgfsetlinewidth{0.803000pt}%
\definecolor{currentstroke}{rgb}{0.000000,0.000000,0.000000}%
\pgfsetstrokecolor{currentstroke}%
\pgfsetdash{}{0pt}%
\pgfsys@defobject{currentmarker}{\pgfqpoint{-0.048611in}{0.000000in}}{\pgfqpoint{-0.000000in}{0.000000in}}{%
\pgfpathmoveto{\pgfqpoint{-0.000000in}{0.000000in}}%
\pgfpathlineto{\pgfqpoint{-0.048611in}{0.000000in}}%
\pgfusepath{stroke,fill}%
}%
\begin{pgfscope}%
\pgfsys@transformshift{0.617715in}{3.207657in}%
\pgfsys@useobject{currentmarker}{}%
\end{pgfscope}%
\end{pgfscope}%
\begin{pgfscope}%
\definecolor{textcolor}{rgb}{0.000000,0.000000,0.000000}%
\pgfsetstrokecolor{textcolor}%
\pgfsetfillcolor{textcolor}%
\pgftext[x=0.368410in, y=3.149620in, left, base]{\color{textcolor}{\sffamily\fontsize{11.000000}{13.200000}\selectfont\catcode`\^=\active\def^{\ifmmode\sp\else\^{}\fi}\catcode`\%=\active\def%{\%}$\mathdefault{15}$}}%
\end{pgfscope}%
\begin{pgfscope}%
\pgfsetbuttcap%
\pgfsetroundjoin%
\definecolor{currentfill}{rgb}{0.000000,0.000000,0.000000}%
\pgfsetfillcolor{currentfill}%
\pgfsetlinewidth{0.803000pt}%
\definecolor{currentstroke}{rgb}{0.000000,0.000000,0.000000}%
\pgfsetstrokecolor{currentstroke}%
\pgfsetdash{}{0pt}%
\pgfsys@defobject{currentmarker}{\pgfqpoint{-0.048611in}{0.000000in}}{\pgfqpoint{-0.000000in}{0.000000in}}{%
\pgfpathmoveto{\pgfqpoint{-0.000000in}{0.000000in}}%
\pgfpathlineto{\pgfqpoint{-0.048611in}{0.000000in}}%
\pgfusepath{stroke,fill}%
}%
\begin{pgfscope}%
\pgfsys@transformshift{0.617715in}{3.982956in}%
\pgfsys@useobject{currentmarker}{}%
\end{pgfscope}%
\end{pgfscope}%
\begin{pgfscope}%
\definecolor{textcolor}{rgb}{0.000000,0.000000,0.000000}%
\pgfsetstrokecolor{textcolor}%
\pgfsetfillcolor{textcolor}%
\pgftext[x=0.368410in, y=3.924919in, left, base]{\color{textcolor}{\sffamily\fontsize{11.000000}{13.200000}\selectfont\catcode`\^=\active\def^{\ifmmode\sp\else\^{}\fi}\catcode`\%=\active\def%{\%}$\mathdefault{20}$}}%
\end{pgfscope}%
\begin{pgfscope}%
\definecolor{textcolor}{rgb}{0.000000,0.000000,0.000000}%
\pgfsetstrokecolor{textcolor}%
\pgfsetfillcolor{textcolor}%
\pgftext[x=0.312854in,y=2.647068in,,bottom,rotate=90.000000]{\color{textcolor}{\sffamily\fontsize{11.000000}{13.200000}\selectfont\catcode`\^=\active\def^{\ifmmode\sp\else\^{}\fi}\catcode`\%=\active\def%{\%}Affected part of the workload (% insn)}}%
\end{pgfscope}%
\begin{pgfscope}%
\pgfsetrectcap%
\pgfsetmiterjoin%
\pgfsetlinewidth{0.803000pt}%
\definecolor{currentstroke}{rgb}{0.000000,0.000000,0.000000}%
\pgfsetstrokecolor{currentstroke}%
\pgfsetdash{}{0pt}%
\pgfpathmoveto{\pgfqpoint{0.617715in}{0.881761in}}%
\pgfpathlineto{\pgfqpoint{0.617715in}{4.412376in}}%
\pgfusepath{stroke}%
\end{pgfscope}%
\begin{pgfscope}%
\pgfsetrectcap%
\pgfsetmiterjoin%
\pgfsetlinewidth{0.803000pt}%
\definecolor{currentstroke}{rgb}{0.000000,0.000000,0.000000}%
\pgfsetstrokecolor{currentstroke}%
\pgfsetdash{}{0pt}%
\pgfpathmoveto{\pgfqpoint{6.235000in}{0.881761in}}%
\pgfpathlineto{\pgfqpoint{6.235000in}{4.412376in}}%
\pgfusepath{stroke}%
\end{pgfscope}%
\begin{pgfscope}%
\pgfsetrectcap%
\pgfsetmiterjoin%
\pgfsetlinewidth{0.803000pt}%
\definecolor{currentstroke}{rgb}{0.000000,0.000000,0.000000}%
\pgfsetstrokecolor{currentstroke}%
\pgfsetdash{}{0pt}%
\pgfpathmoveto{\pgfqpoint{0.617715in}{0.881761in}}%
\pgfpathlineto{\pgfqpoint{6.235000in}{0.881761in}}%
\pgfusepath{stroke}%
\end{pgfscope}%
\begin{pgfscope}%
\pgfsetrectcap%
\pgfsetmiterjoin%
\pgfsetlinewidth{0.803000pt}%
\definecolor{currentstroke}{rgb}{0.000000,0.000000,0.000000}%
\pgfsetstrokecolor{currentstroke}%
\pgfsetdash{}{0pt}%
\pgfpathmoveto{\pgfqpoint{0.617715in}{4.412376in}}%
\pgfpathlineto{\pgfqpoint{6.235000in}{4.412376in}}%
\pgfusepath{stroke}%
\end{pgfscope}%
\begin{pgfscope}%
\definecolor{textcolor}{rgb}{0.000000,0.000000,0.000000}%
\pgfsetstrokecolor{textcolor}%
\pgfsetfillcolor{textcolor}%
\pgftext[x=3.426358in,y=4.495710in,,base]{\color{textcolor}{\sffamily\fontsize{13.200000}{15.840000}\selectfont\catcode`\^=\active\def^{\ifmmode\sp\else\^{}\fi}\catcode`\%=\active\def%{\%}Effect of flushing on workloads}}%
\end{pgfscope}%
\begin{pgfscope}%
\pgfsetbuttcap%
\pgfsetmiterjoin%
\definecolor{currentfill}{rgb}{1.000000,1.000000,1.000000}%
\pgfsetfillcolor{currentfill}%
\pgfsetfillopacity{0.800000}%
\pgfsetlinewidth{1.003750pt}%
\definecolor{currentstroke}{rgb}{0.800000,0.800000,0.800000}%
\pgfsetstrokecolor{currentstroke}%
\pgfsetstrokeopacity{0.800000}%
\pgfsetdash{}{0pt}%
\pgfpathmoveto{\pgfqpoint{3.885606in}{3.393182in}}%
\pgfpathlineto{\pgfqpoint{6.128056in}{3.393182in}}%
\pgfpathquadraticcurveto{\pgfqpoint{6.158611in}{3.393182in}}{\pgfqpoint{6.158611in}{3.423738in}}%
\pgfpathlineto{\pgfqpoint{6.158611in}{4.305432in}}%
\pgfpathquadraticcurveto{\pgfqpoint{6.158611in}{4.335987in}}{\pgfqpoint{6.128056in}{4.335987in}}%
\pgfpathlineto{\pgfqpoint{3.885606in}{4.335987in}}%
\pgfpathquadraticcurveto{\pgfqpoint{3.855051in}{4.335987in}}{\pgfqpoint{3.855051in}{4.305432in}}%
\pgfpathlineto{\pgfqpoint{3.855051in}{3.423738in}}%
\pgfpathquadraticcurveto{\pgfqpoint{3.855051in}{3.393182in}}{\pgfqpoint{3.885606in}{3.393182in}}%
\pgfpathlineto{\pgfqpoint{3.885606in}{3.393182in}}%
\pgfpathclose%
\pgfusepath{stroke,fill}%
\end{pgfscope}%
\begin{pgfscope}%
\pgfsetbuttcap%
\pgfsetmiterjoin%
\definecolor{currentfill}{rgb}{0.121569,0.466667,0.705882}%
\pgfsetfillcolor{currentfill}%
\pgfsetlinewidth{0.000000pt}%
\definecolor{currentstroke}{rgb}{0.000000,0.000000,0.000000}%
\pgfsetstrokecolor{currentstroke}%
\pgfsetstrokeopacity{0.000000}%
\pgfsetdash{}{0pt}%
\pgfpathmoveto{\pgfqpoint{3.916162in}{4.158801in}}%
\pgfpathlineto{\pgfqpoint{4.221717in}{4.158801in}}%
\pgfpathlineto{\pgfqpoint{4.221717in}{4.265746in}}%
\pgfpathlineto{\pgfqpoint{3.916162in}{4.265746in}}%
\pgfpathlineto{\pgfqpoint{3.916162in}{4.158801in}}%
\pgfpathclose%
\pgfusepath{fill}%
\end{pgfscope}%
\begin{pgfscope}%
\definecolor{textcolor}{rgb}{0.000000,0.000000,0.000000}%
\pgfsetstrokecolor{textcolor}%
\pgfsetfillcolor{textcolor}%
\pgftext[x=4.343939in,y=4.158801in,left,base]{\color{textcolor}{\sffamily\fontsize{11.000000}{13.200000}\selectfont\catcode`\^=\active\def^{\ifmmode\sp\else\^{}\fi}\catcode`\%=\active\def%{\%}$\geq5%$ difference in IPC}}%
\end{pgfscope}%
\begin{pgfscope}%
\pgfsetbuttcap%
\pgfsetmiterjoin%
\definecolor{currentfill}{rgb}{1.000000,0.498039,0.054902}%
\pgfsetfillcolor{currentfill}%
\pgfsetlinewidth{0.000000pt}%
\definecolor{currentstroke}{rgb}{0.000000,0.000000,0.000000}%
\pgfsetstrokecolor{currentstroke}%
\pgfsetstrokeopacity{0.000000}%
\pgfsetdash{}{0pt}%
\pgfpathmoveto{\pgfqpoint{3.916162in}{3.934558in}}%
\pgfpathlineto{\pgfqpoint{4.221717in}{3.934558in}}%
\pgfpathlineto{\pgfqpoint{4.221717in}{4.041503in}}%
\pgfpathlineto{\pgfqpoint{3.916162in}{4.041503in}}%
\pgfpathlineto{\pgfqpoint{3.916162in}{3.934558in}}%
\pgfpathclose%
\pgfusepath{fill}%
\end{pgfscope}%
\begin{pgfscope}%
\definecolor{textcolor}{rgb}{0.000000,0.000000,0.000000}%
\pgfsetstrokecolor{textcolor}%
\pgfsetfillcolor{textcolor}%
\pgftext[x=4.343939in,y=3.934558in,left,base]{\color{textcolor}{\sffamily\fontsize{11.000000}{13.200000}\selectfont\catcode`\^=\active\def^{\ifmmode\sp\else\^{}\fi}\catcode`\%=\active\def%{\%}$\geq10%$ difference in IPC}}%
\end{pgfscope}%
\begin{pgfscope}%
\pgfsetbuttcap%
\pgfsetmiterjoin%
\definecolor{currentfill}{rgb}{0.172549,0.627451,0.172549}%
\pgfsetfillcolor{currentfill}%
\pgfsetlinewidth{0.000000pt}%
\definecolor{currentstroke}{rgb}{0.000000,0.000000,0.000000}%
\pgfsetstrokecolor{currentstroke}%
\pgfsetstrokeopacity{0.000000}%
\pgfsetdash{}{0pt}%
\pgfpathmoveto{\pgfqpoint{3.916162in}{3.710315in}}%
\pgfpathlineto{\pgfqpoint{4.221717in}{3.710315in}}%
\pgfpathlineto{\pgfqpoint{4.221717in}{3.817260in}}%
\pgfpathlineto{\pgfqpoint{3.916162in}{3.817260in}}%
\pgfpathlineto{\pgfqpoint{3.916162in}{3.710315in}}%
\pgfpathclose%
\pgfusepath{fill}%
\end{pgfscope}%
\begin{pgfscope}%
\definecolor{textcolor}{rgb}{0.000000,0.000000,0.000000}%
\pgfsetstrokecolor{textcolor}%
\pgfsetfillcolor{textcolor}%
\pgftext[x=4.343939in,y=3.710315in,left,base]{\color{textcolor}{\sffamily\fontsize{11.000000}{13.200000}\selectfont\catcode`\^=\active\def^{\ifmmode\sp\else\^{}\fi}\catcode`\%=\active\def%{\%}$\geq15%$ difference in IPC}}%
\end{pgfscope}%
\begin{pgfscope}%
\pgfsetbuttcap%
\pgfsetmiterjoin%
\definecolor{currentfill}{rgb}{0.839216,0.152941,0.156863}%
\pgfsetfillcolor{currentfill}%
\pgfsetlinewidth{0.000000pt}%
\definecolor{currentstroke}{rgb}{0.000000,0.000000,0.000000}%
\pgfsetstrokecolor{currentstroke}%
\pgfsetstrokeopacity{0.000000}%
\pgfsetdash{}{0pt}%
\pgfpathmoveto{\pgfqpoint{3.916162in}{3.486072in}}%
\pgfpathlineto{\pgfqpoint{4.221717in}{3.486072in}}%
\pgfpathlineto{\pgfqpoint{4.221717in}{3.593017in}}%
\pgfpathlineto{\pgfqpoint{3.916162in}{3.593017in}}%
\pgfpathlineto{\pgfqpoint{3.916162in}{3.486072in}}%
\pgfpathclose%
\pgfusepath{fill}%
\end{pgfscope}%
\begin{pgfscope}%
\definecolor{textcolor}{rgb}{0.000000,0.000000,0.000000}%
\pgfsetstrokecolor{textcolor}%
\pgfsetfillcolor{textcolor}%
\pgftext[x=4.343939in,y=3.486072in,left,base]{\color{textcolor}{\sffamily\fontsize{11.000000}{13.200000}\selectfont\catcode`\^=\active\def^{\ifmmode\sp\else\^{}\fi}\catcode`\%=\active\def%{\%}$\geq20%$ difference in IPC}}%
\end{pgfscope}%
\end{pgfpicture}%
\makeatother%
\endgroup%
}
    \end{minipage}
    \caption{Weighted IPC differences}
    \label{fig:sim_ipc_diff}
\end{figure}

The graphs in \cref{fig:sim_ipc_diff} show the results of this analysis.
They follow the same structure as \cref{fig:weight_ipc_diff} from the hardware analysis: showing weighted IPC differences for each workload.
The first thing we noticed is that the simulator results are very different from the hardware results.
Where 3D U-Net had kernels with a relative IPC difference of over 15\% in hardware, the differences in the simulator cap out around 5\%.
For DCT we notice a similar trend: the simulator results are much lower than the hardware results.
Where the hardware results showed that more than 50\% of the workload suffered from at least 20\% in IPC difference, the simulator results give a maximum of 15\% (worth less than 1\% of the workload).

\subsection{OceanFFT}\label{subsec:oceanfft}
In this figure, the OceanFFT workload really stands out, compared to the other workloads.
When looking into the structure of the OceanFFT workload, we noticed that it consists of only five kernels, of which one (kernel \#2) is significantly affected by the cold-start problem.
The other four kernels (kernel \#1, \#3, \#4, and \#5) are not affected by the cold-start problem at all (all of them suffering less than 0.5\% relative IPC difference), as you can see in \cref{fig:ocean_kernels}.

\begin{figure}[ht]
    \centering
    \resizebox{0.66\textwidth}{!}{  \begin{tabular}{|c|c|c|}
    \hline
    \textbf{Kernel Name} & \textbf{IPC Difference} & \textbf{weight} \\
    \hline
    \hline 
    \verb|generateSpectrumKernel| & 0.28\% & 44.43\% \\
    \verb|regular_fft| & \textit{31.40\%} & \textit{21.69\%} \\
    \verb|vector_fft| & 0.45\% & 17.81\% \\
    \verb|updateHeightmapKernel| & 0.01\% & 6.00\% \\ 
    \verb|calculateSlopeKernel| & 0.00\% & 10.08\% \\
    \hline 
  \end{tabular}
}
    \caption{OceanFFT kernels}
    \label{fig:ocean_kernels}
\end{figure}

\FloatBarrier
\section{Simulator conclusion}\label{sec:simulator-conclusion}
Once again, we see that the cold-start problem persists.
However, it is much less severe in the simulator.

In the next chapter, we'll go looking for a mitigation, limiting the impact of the cold-start problem.
We'll mostly analyze the DCT workload, due to its short runtime and thus reasonable runtime.
The OceanFFT will prove to not be very promising, having the ``problematic'' kernel as second kernel chronologically.