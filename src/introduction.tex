\chapter{Introduction}\label{ch:introduction}

\section{GPUs}\label{sec:gpus}
Graphics Processing Units (or GPUs for short), are specialized electronic circuits (or chips), originally designed to accelerate computer graphics.
In the modern day, these chips are most known in their standalone form: as a separate card that can be plugged into a computer's motherboard.
However, they are also often integrated into CPUs, as is the case with Intel's integrated GPUs, or AMD's APUs.

The history of GPUs goes back to 1968, when the \textit{Evans and Sutherland Computer Corporation} was founded~\cite{gpu-evolution}.
This company built special-purpose 3D graphics hardware, which was mostly used in flight simulators.
The hardware they made took several racks, and was extremely expensive.

The advent of 3D video gaming consoles, around 1994, brought about the era of mass-market 3D gaming.
This led to the founding of a lot of companies, all focused on bringing 2.5D and 3D graphics hardware to PCs.
One of these companies was NVIDIA, which was founded in 1993.

They released the GeForce 256 in 1999, which is commonly regarded as the first ``real'' GPU\@.
Up to this point, the vertex processing (transforming 3D data into 2D coordinates for the screen), was done by the CPU, which was not built for this task.
The actual rendering (computing lighting and shading) was already done by the GPU\@.
The GeForce 256 allowed offloading the vertex processing to the GPU as well, allowing higher geometric complexity.
At this point, computations were still done in fixed-function pipelines, which were hardwired to perform specific tasks.

This would change with the introduction of the GeForce 3 (also known as the NV20) GPU in 2001.
It introduced programmable vertex shaders, where the vertex processing stage could be programmed by the user.
In the following year, GeForce FX introduced programmable fragment shaders, allowing the user to take full control of the rendering pipeline.

At this point, GPUs were already very powerful, allowing for a lot of parallel computations to be done.
The GeForce 6, for example, had a peak performance of 108 GFLOPS (billion floating-point operations per second).
This dwarfed the performance of CPUs, which were only capable of a few GFLOPS at the time.
The flexibility of the pipeline allowed for a lot of different applications to be run on the GPU, making it a very attractive platform for scientific computing.
Despite this, programming for GPUs was still extremely challenging, as both in- and output were very restrictive.
Problems had to be translated to a set of vertices, which were then rendered, and the results had to be read back from the framebuffer.

2006 saw the introduction of the \textit{streaming multiprocessor} (SM), a piece of hardware that was used to run both vertex and fragment shaders.
Additionally, these SMs could also run \textit{compute shaders}, which allowed users to run code on the GPU, independent of the graphics pipeline.

This grew into the CUDA (Compute Unified Device Architecture) platform~\cite{cuda, cuda-prog}, which was introduced in 2007.
CUDA allows users to write code in C, which is then compiled to run on the GPU\@.
The full CUDA software stack consists of three main elements:
\begin{itemize}
    \item The CUDA hardware driver, which is responsible for managing the GPU and scheduling work.
    \item The CUDA API and its associated runtime.
    The API is an extension to the C programming language.
    \item Mathematical libraries, optimized to be run on GPUs.
\end{itemize}

We will focus mostly on the compute capabilities of GPUs (specifically newer generations of NVIDIA GPUs), as they are the most relevant to our research.

Every year, more and more AI and machine learning papers are published~\cite{aiindex}.
Many of these machine learning algorithms rely on GPU computations to train their models, creating a growing need for more and faster GPUs.
Innovations to GPU hardware often require changing the micro-architecture.
These changes require verification to ensure that they are, in fact, improvements.
However, the creation of a new, physical GPU is very costly and time-consuming.
To avoid this overhead and allow for a faster iteration cycle, simulation is used.

\section{GPU Simulation}\label{sec:simulation}
The main idea behind simulation is to model the behavior of a system, and then run this model on a computer.
This allows for the system to be tested in a controlled environment, without the need for the actual hardware.
This is especially useful for GPUs, as the hardware is very expensive, and the development cycle is very long.

A good (architectural) simulator should be flexible, to allow the end-user to make drastic changes to the hardware architecture, without excessive effort.
That being said, an accurate model is very hard to build.
Two factors hinder us in this regard:
\begin{itemize}
    \item The desire for increased abstraction; many models are kept at a high level of abstraction, allowing for easier understanding.
    However, the hidden details can make the model less accurate.
    \item Most hardware implementations are proprietary, i.e.\ the actual details are not publicly known.
\end{itemize}
This means that many simulators used in academia suffer a hit on their accuracy.

Most GPU simulators are cycle-level, meaning that they simulate the hardware at the clock cycle level.
Some contemporary examples are Barra~\cite{barra}, GPUOcelot~\cite{ocelot}, Multi2Sim~\cite{multi2sim}, gem5~\cite{gem5}, GPGPU-Sim~\cite{gpgpu-sim}, and its derivative AccelSim~\cite{accelsim}.

\subsection{GPU Code}\label{subsec:gpu-code}
The typical GPU uses a two-step compilation process:
\begin{enumerate}
    \item High-level code is compiled to an intermediate representation, which is defined by the vendor.
    Often, this intermediate representation is referred to as the \textit{virtual Instruction Set Architecture} (vISA).
    For NVIDIA GPUs, this vISA is called PTX (Parallel Thread Execution).
    The PTX format is thoroughly documented, and remains relatively stable across GPU generations, which makes it a good target for simulators.
    \item The vISA code is compiled to the actual hardware instructions, referred to as the \textit{machine Instruction Set Architecture} (mISA).
    This just-in-time compilation step is done at runtime.
    For NVIDIA GPUs, this mISA is called SASS (Source and Assembly).
    Compared to the PTX format, SASS is very poorly documented, and undergoes changes with every new GPU generation.
\end{enumerate}

As Jain et al.~\cite{sim-method} point out, the choice of which ISA to simulate can have an impact on the accuracy of the simulation results.
They compared the accuracy of simulating at the PTX level to simulating at the SASS level when using the GPGPU-Sim simulator.
From the workloads they analyzed, most tend to have a better accuracy on the mISA level.

When simulating SASS code, GPGPU-Sim translates the instructions to a new format, called PTXPlus.
This PTXPlus format is a 1:1 representation of the GT200 SASS code, but translated to resemble PTX in syntax.

\section{Improving simulation speed}\label{sec:improving-simulation-speed}
However, simulation is not without its drawbacks.
The main issue is that simulation comes with a very large overhead compared to running algorithms on the actual hardware.
This overhead is so large that applications that would take only about an hour on real hardware could easily take upwards of a century~\cite{pks}.
Additionally, the accuracy of the simulation is often lower than that of the real hardware.

Some research has gone into improving the speed of GPU simulation.
As most GPU workloads consist of a large number of kernels, sampling a subset of these kernels can be used to speed up the simulation.
Techniques like Principal Kernel Analysis~\cite{pks} and Sieve~\cite{sieve} have been developed to identify which kernels to simulate, and how to generalize the results.

This gives us the general outline of the simulation process:
\begin{enumerate}
    \item Profile the workload on real hardware, to both gather performance numbers and create instruction-level traces.
    For this, we will use NVIDIA's Nsight Compute~\cite{nsight} and NVBit~\cite{nvbit}.
    \item Use these performance numbers, together with a sampling method, to identify which kernels to simulate.
    \item Simulate the traces of these selected kernels on the simulator, gathering performance numbers.
    Our simulator of choice is Accel-Sim~\cite{accelsim}.
    \item Use these performance numbers to predict the performance of the entire workload.
\end{enumerate}

However, this methodology of simulation can cause another drop in the accuracy of the simulation.
The performance of a kernel can be influenced by the kernels that run before and after it.
In this case, we will focus on the cache state of the GPU\@.

\section{The cold-start problem}\label{sec:the-cold-start-problem}
The cold-start problem is a problem that arises due to sampling, both in CPU and GPU\@.\footnote{As we focus on GPU simulation, we will refer to units of execution as kernels. In CPU simulation, these might be functions or basic blocks.}
Each kernel modifies the state of the caches by executing memory instructions (e.g.\ loads and stores).
When we sample the workload to select only a subset of the kernels, there is a large chance that we will choose only intermittent kernels.
This means that the state of the caches between kernel invocations will not reflect the actual hardware state.

Suppose a workload consists of four consecutive kernels; A, B, C, and D\@.
If kernel A were to be an initializing kernel, readying data for the other kernels, then it might also warm up the caches, filling them with data.
This means that kernel B, which might reuse a lot of that data, might not suffer the latency of its memory instructions, as it would only have to fetch the data from the cache and not DRAM\@.
However, if we were to only sample kernel B, each cache hit might turn into a miss, drastically reducing the IPC of the kernel.

This problem is known as the cold-start problem, and it can have a significant impact on the accuracy of the simulation.
It also gives us our research question: \textit{how big is the impact of the cold-start problem (both in hardware and simulation), and can we mitigate it?}

\subsection{Quantifying}\label{subsec:quantifying}
In \Cref{ch:hw-analysis}, we will quantify the cold-start problem in hardware.
We will do this by running a set of benchmarks on real hardware, and then simulating a subset of these (the most promising ones).
The benchmarks range from machine learning inference to graph traversal and molecular simulation.
Each of these benchmarks is run twice; once normally, and once with a forced cold start.
We force the cold starts by flushing the caches between every pair of kernels.

By doing this, we show that the cold-start problem exists in GPU hardware, and that it can have a significant impact on the performance of the workload.
We also look at the impact of the cold-start problem when taking into account each kernel's weight (i.e.\ what portion of the workload it represents).
Additionally, found that the cold-start problem is tied to the degree of inter-kernel data reuse, which we formalize in \Cref{sec:data-reuse}.
From this data, we select the following workloads:
\begin{itemize}
    \item \textbf{DCT}, from the CUDA samples;
    \item \textbf{OceanFFT}, from the CUDA SDK; and
    \item \textbf{3D-UNet inference}, a machine learning benchmark from MLPerf~\cite{mlperf}.
\end{itemize}
These are selected based both on how much they were impacted by the cold-start problem, and by their feasible runtime in the simulator (for 3D-UNet we limited the execution to the first 130 kernels).

After profiling using the NVBit tool that comes with AccelSim, we will simulate these workloads, analyzing the results in \Cref{ch:sim-analysis}.
Each workload will be run twice again, so that we can compare the performance impact of the cold-start problem on these workloads.
The main conclusion will be that the cold-start problem does exist in the simulator as well, albeit with a very different impact.
Overall, the simulator will suffer less from the cold-start problem than the hardware, but the impact will still be noticeable.

\subsection{Mitigating}\label{subsec:mitigating}
After quantifying the cold-start problem, we will look at ways to mitigate it.
\Cref{ch:mitigation} will focus on three avenues we've looked into:
\begin{description}
    \item[Cache warming:] in order to force the cache state to be accurate, we can simulate each preceding kernel.
    However, this is not a feasible solution, as it will increase the simulation time by a lot.
    This is what we tried to avoid by using sampling.
    \item[Selected simulation:] only the kernels directly before a selected kernel will have a significant impact on the cache state.
    Additionally, only the memory instructions do actually modify the cache state.
    From these two observations, we will simulate only the memory instructions of a low number of preceding kernels.
    The main conclusion here will be that simulating the memory instructions of a single preceding kernel will give us a very high accuracy (often up to 99\%) at a rather low computational cost.
    \item[Correction factors:] using certain factors we can compute a factor to correct the performance numbers of a cold-started kernel.
    The main idea will be to find a subtractive factor for the number of cycles.
    % TODO: conclusion for correction factor???
\end{description}

\section{Conclusions}\label{sec:conclusions}
In this thesis, we will quantify the cold-start problem in GPU hardware and simulation.
We will show that the cold-start problem exists in both hardware and simulation, and that it can have a significant impact on the performance of a workload.
We will also show that this problem is tied to the degree of inter-kernel data reuse.
Additionally, we will show that the cold-start problem is less severe in simulation than in hardware.
Finally, we will look into ways to mitigate the cold-start problem, exploring three different avenues.
The first avenue, simulating all preceding kernels, will be discarded as it is not feasible, especially for larger workloads.
The second avenue, simulating only the memory instructions of a single preceding kernel, will be shown to give a very high accuracy at a low computational cost.
The third avenue, using correction factors, will be shown to be a viable alternative to the second avenue. % TODO conclusion???