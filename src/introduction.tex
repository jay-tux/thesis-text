\chapter{Introduction}\label{ch:introduction}

Every year, more and more AI and machine learning papers are published\cite{aiindex}.
Many machine learning algorithms rely on GPU computations to train their models, creating a growing need for more and faster GPUs.
Improvements to GPUs can be done, among other means, by improving upon their internal micro-architecture.
These changes, however, always need to be verified to ensure that they are, in fact, improvements.

When verifying, simulation usually provides results at a lower cost, albeit with lower accuracy.
To this end, simulators like AccelSim\cite{accelsim} are used.
However, it has also been shown that simulation has a huge overhead; applications that would take only about an hour on real hardware could easily take upwards of a century\cite{pks}.
Some research has gone to speeding up this simulation.
For example Principle Kernel Analysis\cite{pks}, Sieve\cite{sieve}, and Photon\cite{photon} use sampling to select a few kernels, using those as representatives for the entire workload.

The speed-up gained from these techniques comes with a cost to accuracy, as intermediate kernels might have modified the cache state.
This might lead to superfluous cache misses or incorrect cache hits, as loads and/or evictions might not have been performed.
This problem is known as the cold-start problem, and has been researched with respect to the simulation of CPUs\cite{coolsim}.

I aim to show that this problem persists when simulating GPUs, as well as show that it can be mitigated, albeit at the cost of a slightly higher simulation time.